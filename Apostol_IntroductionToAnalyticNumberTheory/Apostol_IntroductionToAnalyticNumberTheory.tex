\documentclass{article}
\usepackage{amsfonts}
\usepackage{amsmath}
\usepackage{amssymb}
\usepackage{centernot}
\usepackage{hyperref}
\usepackage[none]{hyphenat}
\usepackage{mathrsfs}
\usepackage{mathtools}
\usepackage{physics}
\usepackage{tikz-cd}
\parindent=0pt



\title{\textbf{Notes on the book: \\
\emph{Apostol, Introduction to Analytic Number Theory}}}
\author{Meng-Gen Tsai \\ plover@gmail.com}



\begin{document}
\maketitle
\tableofcontents



%%%%%%%%%%%%%%%%%%%%%%%%%%%%%%%%%%%%%%%%%%%%%%%%%%%%%%%%%%%%%%%%%%%%%%%%%%%%%%%%
%%%%%%%%%%%%%%%%%%%%%%%%%%%%%%%%%%%%%%%%%%%%%%%%%%%%%%%%%%%%%%%%%%%%%%%%%%%%%%%%



% Reference:



%%%%%%%%%%%%%%%%%%%%%%%%%%%%%%%%%%%%%%%%%%%%%%%%%%%%%%%%%%%%%%%%%%%%%%%%%%%%%%%%
%%%%%%%%%%%%%%%%%%%%%%%%%%%%%%%%%%%%%%%%%%%%%%%%%%%%%%%%%%%%%%%%%%%%%%%%%%%%%%%%



\newpage
\section*{Chapter 1: The Fundamental Theorem of Arithmetic \\}
\addcontentsline{toc}{section}{Chapter 1: The Fundamental Theorem of Arithmetic}



\subsubsection*{Exercise 1.15.}
\addcontentsline{toc}{subsubsection}{Exercise 1.15.}
\emph{Prove that every $n \geq 12$ is the sum of two composite numbers.} \\



\emph{Proof.}
  Write $n = 2m$ (resp. $n = 2m + 1$) where $m \in \mathbb{Z}$, $m \geq 6$.
  Then $n = 8 + 2(m-4)$ (resp. $n = 9 + 2(m-4)$) is the sum of two composite numbers.
$\Box$ \\\\



%%%%%%%%%%%%%%%%%%%%%%%%%%%%%%%%%%%%%%%%%%%%%%%%%%%%%%%%%%%%%%%%%%%%%%%%%%%%%%%%



\subsubsection*{Exercise 1.30.}
\addcontentsline{toc}{subsubsection}{Exercise 1.30.}
\emph{If $n > 1$ prove that the sum
\[
  \sum_{k=1}^{n} \frac{1}{k}
\]
is not an integer.} \\



\emph{Proof.}
\begin{enumerate}
\item[(1)]
  (Reductio ad absurdum)
  Suppose
  \[
    H := \sum_{k=1}^{n} \frac{1}{k}
  \]
  were an integer.

\item[(2)]
  Let $s$ be the largest integer such that $2^s \leq n$.
  So the integer number
  \begin{align*}
    2^{s-1} H
    &= \sum_{k=1}^{n} \frac{2^{s-1}}{k} \\
    &= 2^{s-1} + 2^{s-2} + \frac{2^{s-1}}{3} + 2^{s-3} + \frac{2^{s-1}}{5} + \frac{2^{s-2}}{3}
      + \cdots + \frac{1}{2} + \cdots.
  \end{align*}
  has only one term of even denominators (as $n > 1$)
  if we write all terms in irreducible fractions.
  That is,
  \[
  	2^{s-1} H = \frac{1}{2} + \frac{c}{d} \in \mathbb{Z}
  \]
  where $\frac{c}{d}$ is an irreducible fraction with odd $d$.
  Hence it suffices to show that $2 \mid d$ to get a contradiction.

\item[(3)]
  By
  \[
  	\frac{1}{2} + \frac{c}{d} = \frac{d + 2c}{2d} \in \mathbb{Z}
  \]
  we have $d + 2c = 2dd'$ for some $d' \in \mathbb{Z}$.
  Note that $2$ is a prime.
  So $2 \mid (d + 2c)$ or $2 \mid d$, which is absurd.
\end{enumerate}
$\Box$ \\\\



%%%%%%%%%%%%%%%%%%%%%%%%%%%%%%%%%%%%%%%%%%%%%%%%%%%%%%%%%%%%%%%%%%%%%%%%%%%%%%%%
%%%%%%%%%%%%%%%%%%%%%%%%%%%%%%%%%%%%%%%%%%%%%%%%%%%%%%%%%%%%%%%%%%%%%%%%%%%%%%%%



\newpage
\section*{Chapter 2: Arithmetical functions and Dirichlet multiplication \\}
\addcontentsline{toc}{section}{Chapter 2: Arithmetical functions and Dirichlet multiplication}



%%%%%%%%%%%%%%%%%%%%%%%%%%%%%%%%%%%%%%%%%%%%%%%%%%%%%%%%%%%%%%%%%%%%%%%%%%%%%%%%



\subsubsection*{Exercise 2.3.}
\addcontentsline{toc}{subsubsection}{Exercise 2.3.}
\emph{Prove that}
\[
  \frac{n}{\varphi(n)} = \sum_{d|n} \frac{\mu^2(d)}{\varphi(d)}.
\]

\emph{Proof.}
\begin{enumerate}
\item[(1)]
  Note that $fg$, $f/g$ and $f * g$ are multiplicative if $f$ and $g$ are multiplicative
  (Example 5 on page 34 and Theorem 2.14).
  Hence $\frac{n}{\varphi(n)}$ and $\sum_{d|n} \frac{\mu^2(d)}{\varphi(d)}$
  are multiplicative.
  Hence it might assume that $n = p^a$ for some prime $p$ and integer $a \geq 1$.
  (The case $n=1$ is trivial.)

\item[(2)]
  \[
    \frac{p^a}{\varphi(p^a)}
    = \frac{p^a}{p^a - p^{a-1}}
    = \frac{p}{p-1}.
  \]

\item[(3)]
  \begin{align*}
    \sum_{d|p^a} \frac{\mu^2(d)}{\varphi(d)}
    &= \frac{\mu^2(1)}{\varphi(1)} + \frac{\mu^2(p)}{\varphi(p)}
      + \frac{\overbrace{\mu^2(p^2)}^{= 0}}{\varphi(p^2)} + \cdots
      + \frac{\overbrace{\mu^2(p^a)}^{=0}}{\varphi(p^a)} \\
    &= 1 + \frac{1}{p-1} + 0 + \cdots + 0 \\
    &= \frac{p}{p-1}.
  \end{align*}
\end{enumerate}
$\Box$ \\



\subsubsection*{Supplement 2.3.1. (Chinese remainder theorem)}
\addcontentsline{toc}{subsubsection}{Supplement 2.3.1. (Chinese remainder theorem)}
(Exercise I.3.5 in the textbook: \emph{J\"orgen Neukirch, Algebraic Number Theory}.)
\emph{The quotient ring $\mathcal{O}/\mathfrak{a}$ of a Dedekind domain by
an ideal $\mathfrak{a} \neq 0$ is a principal ideal domain.
(Hint: For $\mathfrak{a} = \mathfrak{p}^n$ the only proper ideals of $\mathcal{O}/\mathfrak{a}$
are given by $\mathfrak{p}/\mathfrak{p}^n, \ldots, \mathfrak{p}^{n-1}/\mathfrak{p}^n$.
Choose $\pi \in \mathfrak{p} \smallsetminus \mathfrak{p}^2$
and show that $\mathfrak{p}^{\nu} = \mathcal{O}\pi^{\nu} + \mathfrak{p}^n$.)} \\

\emph{Proof.}
\begin{enumerate}
\item[(1)]
  By the Chinese remainder theorem,
  it suffices to show the case $\mathfrak{a} = \mathfrak{p}^n$ where $\mathfrak{p}$ is prime.

\item[(2)]
  There is a natural correspondence between
  \[
    \{ \text{ideals of $\mathcal{O}/\mathfrak{p}^n$} \}
    \longleftrightarrow
    \{ \text{ideals of $\mathcal{O}$ containing $\mathfrak{p}^n$} \}.
  \]
  Hence the proper ideals of $\mathcal{O}/\mathfrak{p}^n$
  are given by $\mathfrak{p}/\mathfrak{p}^n, \ldots, \mathfrak{p}^{n-1}/\mathfrak{p}^n$.

\item[(3)]
  Similar to Exercise I.3.4,
  choose $\pi \in \mathfrak{p} \smallsetminus \mathfrak{p}^2$
  and thus $\mathfrak{p}^{\nu} = \mathcal{O}\pi^{\nu} + \mathfrak{p}^n$
  ($\nu = 1, \ldots, n-1$)
  since they have the same prime factorization.
  Hence
  $\mathfrak{p}^{\nu}/\mathfrak{p}^n = (\pi^{\nu} + \mathfrak{p}^n)$ is principal.
\end{enumerate}
$\Box$ \\\\



%%%%%%%%%%%%%%%%%%%%%%%%%%%%%%%%%%%%%%%%%%%%%%%%%%%%%%%%%%%%%%%%%%%%%%%%%%%%%%%%



\subsubsection*{Exercise 2.4.}
\addcontentsline{toc}{subsubsection}{Exercise 2.4.}
\emph{Prove that $\varphi(n) > \frac{n}{6}$ for all $n$ with at most $8$ distinct prime factors.} \\



\emph{Proof.}
\begin{enumerate}
\item[(1)]
  \begin{align*}
    \varphi(n)
    = &\:
    n \prod_{p|n}\left( 1 - \frac{1}{p} \right)
      &(\text{Theorem 2.4}) \\
    \geq &\:
    n \left( 1 - \frac{1}{2} \right) \left( 1 - \frac{1}{3} \right)
      \left( 1 - \frac{1}{5} \right) \left( 1 - \frac{1}{7} \right) \\
    & \: \qquad
      \left( 1 - \frac{1}{11} \right) \left( 1 - \frac{1}{13} \right)
      \left( 1 - \frac{1}{17} \right) \left( 1 - \frac{1}{19} \right) \\
    = &\: \frac{55296}{323323} n \\
    > &\: \frac{n}{6}.
  \end{align*}

\item[(2)]
  The conclusion does not hold if $n$ has more than $9$ distinct prime factors.
\end{enumerate}
$\Box$ \\\\



%%%%%%%%%%%%%%%%%%%%%%%%%%%%%%%%%%%%%%%%%%%%%%%%%%%%%%%%%%%%%%%%%%%%%%%%%%%%%%%%
%%%%%%%%%%%%%%%%%%%%%%%%%%%%%%%%%%%%%%%%%%%%%%%%%%%%%%%%%%%%%%%%%%%%%%%%%%%%%%%%



\end{document}