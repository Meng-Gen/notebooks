\documentclass{article}
\usepackage{amsfonts}
\usepackage{amsmath}
\usepackage{amssymb}
\usepackage{centernot}
\usepackage{hyperref}
\usepackage[none]{hyphenat}
\usepackage{mathrsfs}
\usepackage{mathtools}
\usepackage{physics}
\usepackage{tikz-cd}
\parindent=0pt



\title{\textbf{Notes on the book: \\
\emph{Apostol, Introduction to Analytic Number Theory}}}
\author{Meng-Gen Tsai \\ plover@gmail.com}



\begin{document}
\maketitle
\tableofcontents



%%%%%%%%%%%%%%%%%%%%%%%%%%%%%%%%%%%%%%%%%%%%%%%%%%%%%%%%%%%%%%%%%%%%%%%%%%%%%%%%
%%%%%%%%%%%%%%%%%%%%%%%%%%%%%%%%%%%%%%%%%%%%%%%%%%%%%%%%%%%%%%%%%%%%%%%%%%%%%%%%



% Reference:



%%%%%%%%%%%%%%%%%%%%%%%%%%%%%%%%%%%%%%%%%%%%%%%%%%%%%%%%%%%%%%%%%%%%%%%%%%%%%%%%
%%%%%%%%%%%%%%%%%%%%%%%%%%%%%%%%%%%%%%%%%%%%%%%%%%%%%%%%%%%%%%%%%%%%%%%%%%%%%%%%



\newpage
\section*{Chapter 1: The Fundamental Theorem of Arithmetic \\}
\addcontentsline{toc}{section}{Chapter 1: The Fundamental Theorem of Arithmetic}



%%%%%%%%%%%%%%%%%%%%%%%%%%%%%%%%%%%%%%%%%%%%%%%%%%%%%%%%%%%%%%%%%%%%%%%%%%%%%%%%



\subsubsection*{Exercise 1.14.}
\addcontentsline{toc}{subsubsection}{Exercise 1.14.}
\emph{Prove that $n^4 + 4$ is composite if $n > 1$.} \\



\emph{Proof.}
  \[
    n^4 + 4
    = (\underbrace{(n-1)^2 + 1}_{> 1})(\underbrace{(n+1)^2 + 1}_{> 1})
  \]
  since $n > 1$.
$\Box$ \\\\



%%%%%%%%%%%%%%%%%%%%%%%%%%%%%%%%%%%%%%%%%%%%%%%%%%%%%%%%%%%%%%%%%%%%%%%%%%%%%%%%



\subsubsection*{Exercise 1.15.}
\addcontentsline{toc}{subsubsection}{Exercise 1.15.}
\emph{Prove that every $n \geq 12$ is the sum of two composite numbers.} \\



\emph{Proof.}
  Write $n = 2m$ (resp. $n = 2m + 1$) where $m \in \mathbb{Z}$, $m \geq 6$.
  Then $n = 8 + 2(m-4)$ (resp. $n = 9 + 2(m-4)$) is the sum of two composite numbers.
$\Box$ \\\\



%%%%%%%%%%%%%%%%%%%%%%%%%%%%%%%%%%%%%%%%%%%%%%%%%%%%%%%%%%%%%%%%%%%%%%%%%%%%%%%%



\subsubsection*{Exercise 1.30.}
\addcontentsline{toc}{subsubsection}{Exercise 1.30.}
\emph{If $n > 1$ prove that the sum
\[
  \sum_{k=1}^{n} \frac{1}{k}
\]
is not an integer.} \\



\emph{Proof.}
\begin{enumerate}
\item[(1)]
  (Reductio ad absurdum)
  Suppose
  \[
    H := \sum_{k=1}^{n} \frac{1}{k}
  \]
  were an integer.

\item[(2)]
  Let $s$ be the largest integer such that $2^s \leq n$.
  So the integer number
  \begin{align*}
    2^{s-1} H
    &= \sum_{k=1}^{n} \frac{2^{s-1}}{k} \\
    &= 2^{s-1} + 2^{s-2} + \frac{2^{s-1}}{3} + 2^{s-3} + \frac{2^{s-1}}{5} + \frac{2^{s-2}}{3}
      + \cdots + \frac{1}{2} + \cdots.
  \end{align*}
  has only one term of even denominators (as $n > 1$)
  if we write all terms in irreducible fractions.
  That is,
  \[
  	2^{s-1} H = \frac{1}{2} + \frac{c}{d} \in \mathbb{Z}
  \]
  where $\frac{c}{d}$ is an irreducible fraction with odd $d$.
  Hence it suffices to show that $2 \mid d$ to get a contradiction.

\item[(3)]
  By
  \[
  	\frac{1}{2} + \frac{c}{d} = \frac{d + 2c}{2d} \in \mathbb{Z}
  \]
  we have $d + 2c = 2dd'$ for some $d' \in \mathbb{Z}$.
  Note that $2$ is a prime.
  So $2 \mid (d + 2c)$ or $2 \mid d$, which is absurd.
\end{enumerate}
$\Box$ \\\\



%%%%%%%%%%%%%%%%%%%%%%%%%%%%%%%%%%%%%%%%%%%%%%%%%%%%%%%%%%%%%%%%%%%%%%%%%%%%%%%%
%%%%%%%%%%%%%%%%%%%%%%%%%%%%%%%%%%%%%%%%%%%%%%%%%%%%%%%%%%%%%%%%%%%%%%%%%%%%%%%%



\newpage
\section*{Chapter 2: Arithmetical functions and Dirichlet multiplication \\}
\addcontentsline{toc}{section}{Chapter 2: Arithmetical functions and Dirichlet multiplication}



%%%%%%%%%%%%%%%%%%%%%%%%%%%%%%%%%%%%%%%%%%%%%%%%%%%%%%%%%%%%%%%%%%%%%%%%%%%%%%%%



\subsubsection*{Exercise 2.1.}
\addcontentsline{toc}{subsubsection}{Exercise 2.1.}
\emph{Find all integers $n$ such that}
\begin{enumerate}
\item[(a)]
  \emph{$\varphi(n) = \frac{n}{2}$,}

\item[(b)]
  \emph{$\varphi(n) = \varphi(2n)$,}

\item[(c)]
  \emph{$\varphi(n) = 12$.} \\
\end{enumerate}



\emph{Proof of (a).}
  \[
    \varphi(n) = n \prod_{p|n}\left( 1 - \frac{1}{p} \right) = \frac{n}{2}
  \]
  (Theorem 2.4) implies that $n = 2$.
$\Box$ \\



\emph{Proof of (b).}
\begin{enumerate}
\item[(1)]
  $\varphi(n) = \varphi(2n)$ implies that
  \[
    n \prod_{p|n}\left( 1 - \frac{1}{p} \right)
    = 2n \prod_{p|(2n)}\left( 1 - \frac{1}{p} \right).
  \]

\item[(2)]
  If $2 | n$, then $n = 2n$ or $n = 0$, which is absurd.

\item[(3)]
  If $2 \nmid n$, then
  \[
    n \prod_{p|n}\left( 1 - \frac{1}{p} \right)
    = 2n \prod_{p|(2n)}\left( 1 - \frac{1}{p} \right)
    = \underbrace{2n \left( 1 - \frac{1}{2} \right)}_{= n}
      \prod_{p|n}\left( 1 - \frac{1}{p} \right)
  \]
  is always true.
  Hence $n$ is odd if $\varphi(n) = \varphi(2n)$.
\end{enumerate}
$\Box$ \\



\emph{Proof of (c).}
\begin{enumerate}
\item[(1)]
  \emph{Show that the solutions of $\varphi(n) = 12$
  are $n = 13, 26, 21, 28, 42, 36$.}
  Write $n = p_1^{\alpha_1} \cdots p_r^{\alpha_r}$
  where $p_1 < p_2 < \ldots$.
  Then
  \[
    12 = \varphi(n) = \prod_{i=1}^{r} p_i^{\alpha_i - 1}(p_i - 1).
  \]
  (Theorem 2.5).
  It implies that $p_i \in \{ 2, 3, 5, 7, 13 \}$ if $\alpha_i > 0$.
  Consider all possible cases of the greatest prime divisor $p_r$ of $n$ as follows.

\item[(2)]
  If $p_r = 13$, then $\alpha_r = 1$ since $13 \nmid 12$.
  So
  \[
    12
    = \varphi(n)
    = \underbrace{\varphi(13)}_{= 12} \varphi\left(\frac{n}{13}\right)
  \]
  or $1 = \varphi\left(\frac{n}{13}\right)$.
  Hence $\frac{n}{13} = 1, 2$.
  In this case $n = 13, 26$.

\item[(3)]
  If $p_r = 7$, then $\alpha_r = 1$ since $7 \nmid 12$.
  So
  \[
    12
    = \varphi(n)
    = \underbrace{\varphi(7)}_{= 6} \varphi\left(\frac{n}{7}\right)
  \]
  or $2 = \varphi\left(\frac{n}{7}\right)$.
  Hence $\frac{n}{7} = 3, 4, 6$.
  In this case $n = 21, 28, 42$.

\item[(5)]
  If $p_r = 5$, then $\alpha_r = 1$ since $5 \nmid 12$.
  So $12 = \varphi(5)\varphi\left(\frac{n}{5}\right)$
  or $3 = \varphi\left(\frac{n}{5}\right)$, which is impossible.

\item[(6)]
  If $p_r = 3$, then $\alpha_r = 1, 2$.
  $\alpha_r = 1$ is impossible since $3 | 12$.
  So
  \[
    12
    = \varphi(n)
    = \underbrace{\varphi(3^2)}_{= 6}\varphi\left(\frac{n}{3^2}\right)
  \]
  or $2 = \varphi\left(\frac{n}{3^2}\right)$.
  Hence $\frac{n}{3^2} = 4$.
  (By assumption $\frac{n}{3^2}$ cannot have any prime factor $> 3$.)
  In this case $n = 36$.
\end{enumerate}
$\Box$ \\\\



%%%%%%%%%%%%%%%%%%%%%%%%%%%%%%%%%%%%%%%%%%%%%%%%%%%%%%%%%%%%%%%%%%%%%%%%%%%%%%%%



\subsubsection*{Exercise 2.2.}
\addcontentsline{toc}{subsubsection}{Exercise 2.2.}
\emph{For each of the following statements either give a proof or
exhibit a counter example.}
\begin{enumerate}
\item[(a)]
  \emph{If $(m,n)=1$ then $(\varphi(m),\varphi(n)) = 1$.}

\item[(b)]
  \emph{If $n$ is composite, then $(n,\varphi(n)) > 1$.}

\item[(c)]
  \emph{If the same primes divide $m$ and $n$, then $n\varphi(m) = m \varphi(n)$.} \\
\end{enumerate}



\emph{Proof of (a).}
  It is false since $(5, 13) = 1$ and $(\varphi(5),\varphi(13)) = (4, 12) = 4$.
$\Box$ \\


\emph{Proof of (b).}
  It is false since $(15, \varphi(15)) = (15, 8) = 1$.
$\Box$ \\


\emph{Proof of (c).}
\begin{enumerate}
\item[(1)]
  It is true.

\item[(2)]
  If the same primes divide $m$ and $n$, then
  \[
    \frac{\varphi(n)}{n}
    = \prod_{p|n}\left( 1 - \frac{1}{p} \right)
    = \prod_{p|m}\left( 1 - \frac{1}{p} \right)
    = \frac{\varphi(m)}{m}
  \]
  (Theorem 2.4).
  Hence $n\varphi(m) = m \varphi(n)$.
\end{enumerate}
$\Box$ \\\\



%%%%%%%%%%%%%%%%%%%%%%%%%%%%%%%%%%%%%%%%%%%%%%%%%%%%%%%%%%%%%%%%%%%%%%%%%%%%%%%%



\subsubsection*{Exercise 2.3.}
\addcontentsline{toc}{subsubsection}{Exercise 2.3.}
\emph{Prove that}
\[
  \frac{n}{\varphi(n)} = \sum_{d|n} \frac{\mu(d)^2}{\varphi(d)}.
\]

\emph{Proof.}
\begin{enumerate}
\item[(1)]
  Note that $fg$, $f/g$ and $f * g$ are multiplicative if $f$ and $g$ are multiplicative
  (Example 5 on page 34 and Theorem 2.14).
  Hence $\frac{n}{\varphi(n)}$ and $\sum_{d|n} \frac{\mu^2(d)}{\varphi(d)}$
  are multiplicative.
  Hence it might assume that $n = p^a$ for some prime $p$ and integer $a \geq 1$.
  (The case $n=1$ is trivial.)

\item[(2)]
  \[
    \frac{p^a}{\varphi(p^a)}
    = \frac{p^a}{p^a - p^{a-1}}
    = \frac{p}{p-1}.
  \]

\item[(3)]
  \begin{align*}
    \sum_{d|p^a} \frac{\mu(d)^2}{\varphi(d)}
    &= \frac{\mu(1)^2}{\varphi(1)} + \frac{\mu(p)^2}{\varphi(p)}
      + \frac{\overbrace{\mu(p^2)^2}^{= 0}}{\varphi(p^2)} + \cdots
      + \frac{\overbrace{\mu(p^a)^2}^{=0}}{\varphi(p^a)} \\
    &= 1 + \frac{1}{p-1} + 0 + \cdots + 0 \\
    &= \frac{p}{p-1}.
  \end{align*}

\item[(4)]
  Or apply Theorems 2.4 and 2.18 to get
  \begin{align*}
    \sum_{d|n} \frac{\mu(d)^2}{\varphi(d)}
    &= \prod_{p|n} \left(1 - \frac{\mu(p)}{\varphi(p)} \right) \\
    &= \prod_{p|n} \left(1 - \frac{-1}{p-1} \right) \\
    &= \prod_{p|n} \frac{p}{p-1} \\
    &= \frac{n}{\varphi(n)}.
  \end{align*}
\end{enumerate}
$\Box$ \\



\subsubsection*{Supplement 2.3.1. (Chinese remainder theorem)}
\addcontentsline{toc}{subsubsection}{Supplement 2.3.1. (Chinese remainder theorem)}
(Exercise I.3.5 in the textbook: \emph{J\"orgen Neukirch, Algebraic Number Theory}.)
\emph{The quotient ring $\mathcal{O}/\mathfrak{a}$ of a Dedekind domain by
an ideal $\mathfrak{a} \neq 0$ is a principal ideal domain.
(Hint: For $\mathfrak{a} = \mathfrak{p}^n$ the only proper ideals of $\mathcal{O}/\mathfrak{a}$
are given by $\mathfrak{p}/\mathfrak{p}^n, \ldots, \mathfrak{p}^{n-1}/\mathfrak{p}^n$.
Choose $\pi \in \mathfrak{p} \smallsetminus \mathfrak{p}^2$
and show that $\mathfrak{p}^{\nu} = \mathcal{O}\pi^{\nu} + \mathfrak{p}^n$.)} \\

\emph{Proof.}
\begin{enumerate}
\item[(1)]
  By the Chinese remainder theorem,
  it suffices to show the case $\mathfrak{a} = \mathfrak{p}^n$ where $\mathfrak{p}$ is prime.

\item[(2)]
  There is a natural correspondence between
  \[
    \{ \text{ideals of $\mathcal{O}/\mathfrak{p}^n$} \}
    \longleftrightarrow
    \{ \text{ideals of $\mathcal{O}$ containing $\mathfrak{p}^n$} \}.
  \]
  Hence the proper ideals of $\mathcal{O}/\mathfrak{p}^n$
  are given by $\mathfrak{p}/\mathfrak{p}^n, \ldots, \mathfrak{p}^{n-1}/\mathfrak{p}^n$.

\item[(3)]
  Similar to Exercise I.3.4,
  choose $\pi \in \mathfrak{p} \smallsetminus \mathfrak{p}^2$
  and thus $\mathfrak{p}^{\nu} = \mathcal{O}\pi^{\nu} + \mathfrak{p}^n$
  ($\nu = 1, \ldots, n-1$)
  since they have the same prime factorization.
  Hence
  $\mathfrak{p}^{\nu}/\mathfrak{p}^n = (\pi^{\nu} + \mathfrak{p}^n)$ is principal.
\end{enumerate}
$\Box$ \\\\



%%%%%%%%%%%%%%%%%%%%%%%%%%%%%%%%%%%%%%%%%%%%%%%%%%%%%%%%%%%%%%%%%%%%%%%%%%%%%%%%



\subsubsection*{Exercise 2.4.}
\addcontentsline{toc}{subsubsection}{Exercise 2.4.}
\emph{Prove that $\varphi(n) > \frac{n}{6}$ for all $n$ with at most $8$ distinct prime factors.} \\



\emph{Proof.}
\begin{enumerate}
\item[(1)]
  \begin{align*}
    \varphi(n)
    = &\:
    n \prod_{p|n}\left( 1 - \frac{1}{p} \right)
      &(\text{Theorem 2.4}) \\
    \geq &\:
    n \left( 1 - \frac{1}{2} \right) \left( 1 - \frac{1}{3} \right)
      \left( 1 - \frac{1}{5} \right) \left( 1 - \frac{1}{7} \right) \\
    & \: \qquad
      \left( 1 - \frac{1}{11} \right) \left( 1 - \frac{1}{13} \right)
      \left( 1 - \frac{1}{17} \right) \left( 1 - \frac{1}{19} \right) \\
    = &\: \frac{55296}{323323} n \\
    > &\: \frac{n}{6}.
  \end{align*}

\item[(2)]
  The conclusion does not hold if $n$ has more than $9$ distinct prime factors.
\end{enumerate}
$\Box$ \\\\



%%%%%%%%%%%%%%%%%%%%%%%%%%%%%%%%%%%%%%%%%%%%%%%%%%%%%%%%%%%%%%%%%%%%%%%%%%%%%%%%



%%%%%%%%%%%%%%%%%%%%%%%%%%%%%%%%%%%%%%%%%%%%%%%%%%%%%%%%%%%%%%%%%%%%%%%%%%%%%%%%



\subsubsection*{Exercise 2.6.}
\addcontentsline{toc}{subsubsection}{Exercise 2.6.}
\emph{Prove that
\[
  \sum_{d^2 | n} \mu(d) = \mu(n)^2
\]
and, more generally
\begin{equation*}
  \sum_{d^k | n} \mu(d) =
  \begin{cases}
    0 & \text{if $m^k | n$ for some $m > 1$}, \\
    1 & \text{otherwise}.
  \end{cases}
\end{equation*}
The last sum is extended over all positive divisors $d$ of $n$
whose $k$th power also divide $n$.} \\



\emph{Proof.}
\begin{enumerate}
\item[(1)]
  Write $n = p_1^{\alpha_1} \cdots p_r^{\alpha_r} q_1^{\beta_1} \cdots q_s^{\beta_s}$
  where $\alpha_i \geq 2$ and $\beta_j = 1$.
  The proof is similar to Theorem 2.1.

\item[(2)]
  If $p_1^{\alpha_1} \cdots p_r^{\alpha_r} = 1$,
  then $\sum_{d^2 | n} \mu(n) = \mu(1) = 1$.

\item[(3)]
  If $p_1^{\alpha_1} \cdots p_r^{\alpha_r} > 1$,
  then
  \begin{align*}
    \sum_{d^2 | n} \mu(d)
    = & \: \mu(1) + \mu(p_1) + \cdots \mu(p_r) \\
      & \: + \mu(p_1 p_2) + \cdots \mu(p_{r-1} p_r) + \cdots + \mu(p_1 \cdots p_r) \\
    = & \: 1 + {r \choose 1}(-1) + {r \choose 2}(-1)^2 + \cdots + {r \choose r}(-1)^r \\
    = & \: (1-1)^k \\
    = & \: 0.
  \end{align*}

\item[(4)]
  By (2)(3), $\sum_{d^2 | n} \mu(d) = \mu(n)^2$.
  Besides, we have
  \begin{equation*}
    \sum_{d^k | n} \mu(d) =
    \begin{cases}
      0 & \text{if $m^k | n$ for some $m > 1$}, \\
      1 & \text{otherwise}
    \end{cases}
  \end{equation*}
  by the same argument as (1)(2)(3).
\end{enumerate}
$\Box$ \\\\



%%%%%%%%%%%%%%%%%%%%%%%%%%%%%%%%%%%%%%%%%%%%%%%%%%%%%%%%%%%%%%%%%%%%%%%%%%%%%%%%



\subsubsection*{Exercise 2.7.}
\addcontentsline{toc}{subsubsection}{Exercise 2.7.}
\emph{Let $\mu(p,d)$ denote the value of the M\"obius function at the gcd of $p$ and $d$.
Prove that for every prime $p$ we have
\begin{equation*}
  \sum_{d|n} \mu(d) \mu(p,d) =
  \begin{cases}
    1 & \text{if $n = 1$}, \\
    2 & \text{if $n = p^a$, $a \geq 1$}, \\
    0 & \text{otherwise}.
  \end{cases}
\end{equation*}
} \\



\emph{Proof.}
\begin{enumerate}
\item[(1)]
  It suffices to show that $\mu(p,n)$ is multiplicative.
  If so, then
  \[
    h(n) := \sum_{d|n} \mu(d)\mu(p,d)
  \]
  is also multiplicative
  by taking $f(n) := \mu(n)\mu(p,n)$ and $g(n) := 1$ in Theorem 2.14.

\item[(2)]
  A direct calculation shows that $h(1) = 1$ (or by Theorem 2.12) and
  \begin{align*}
    h(p^a) &= \mu(1)\mu(p,1) + \mu(p)\mu(p,p) = 1 \cdot 1 + (-1) \cdot (-1) = 2, \\
    h(q^b) &= \mu(1)\mu(p,1) + \mu(q)\mu(p,q) = 1 \cdot 1 + (-1) \cdot 1 = 0
  \end{align*}
  where $q \neq p$ and $a, b \geq 1$.
  Hence (1) and Theorem 2.13 show that
  \begin{equation*}
    h(n) =
    \begin{cases}
      1 & \text{if $n = 1$}, \\
      2 & \text{if $n = p^a$, $a \geq 1$}, \\
      0 & \text{otherwise}.
    \end{cases}
  \end{equation*}

\item[(3)]
  \emph{Show that $\mu(p,n)$ is multiplicative.}
  Suppose $(m,n) = 1$.
  There are two possible cases: $p \nmid mn$ and $p|mn$.
  \begin{enumerate}
  \item[(a)]
    If $p \neq mn$, then all $\mu(p,mn), \mu(p,m), \mu(p,n)$ are equal to $\mu(1) = 1$.

  \item[(b)]
    If $p | mn$, then $p|m$ or $p|n$.
    Note that $(m,n) = 1$ and thus $p$ cannot be a common divisor of $m, n$.
    Hence $\mu(p,mn) = \mu(p) = -1$ and $\mu(p,m)\mu(p,n) = \mu(p)\mu(1) = -1$.
  \end{enumerate}
  In any case $\mu(p,mn) = \mu(p,m)\mu(p,n)$ if $(m,n) = 1$.
\end{enumerate}
$\Box$ \\\\



%%%%%%%%%%%%%%%%%%%%%%%%%%%%%%%%%%%%%%%%%%%%%%%%%%%%%%%%%%%%%%%%%%%%%%%%%%%%%%%%



\subsubsection*{Exercise 2.9.}
\addcontentsline{toc}{subsubsection}{Exercise 2.9.}
\emph{If $x$ is real, $x \geq 1$, let $\varphi(x,n)$ denote
the number of positive integers $\leq x$ that are relatively prime to $n$.
[Note that $\varphi(n,n) = \varphi(n)$.]
Prove that
\[
  \varphi(x,n) = \sum_{d|n} \mu(d) \left[ \frac{x}{d} \right],
  \qquad
  \sum_{d|n} \varphi\left( \frac{x}{d},\frac{n}{d} \right) = [x].
\]}



\emph{Proof.}
\begin{enumerate}
\item[(1)]
  \emph{Show that $\varphi(x,n) = \sum_{d|n} \mu(d) \left[ \frac{x}{d} \right]$.}
  Similar to the proof of Theorem 2.3.
  $\varphi(x,n)$ can be written in the form
  \[
    \varphi(x,n) = \sum_{1 \leq k \leq x}\left[ \frac{1}{(n,k)} \right],
  \]
  where now $k$ runs through all integers $\leq x$.
  Now we use Theorem 2.1 with $n$ replaced by $(n,k)$ to obtain
  \[
    \varphi(x,n)
    = \sum_{1 \leq k \leq x} \sum_{d|(n,k)} \mu(d)
    = \sum_{1 \leq k \leq x} \sum_{\substack{d|n \\ d|k}} \mu(d).
  \]
  For a fixed divisor $d$ of $n$ we must sum over all those $k$ in the range $1 \leq k \leq x$
  which are multiples of $d$.
  If we write $k = qd$ then $1 \leq k \leq x$ if and only if
  $1 \leq q \leq \left[ \frac{x}{d} \right]$.
  Hence the last sum for $\varphi(x,n)$ can be written as
  \[
    \varphi(x,n)
    = \sum_{d|n} \sum_{1 \leq q \leq \left[ \frac{x}{d} \right]} \mu(d)
    = \sum_{d|n} \mu(d) \sum_{1 \leq q \leq \left[ \frac{x}{d} \right]} 1
    = \sum_{d|n} \mu(d) \left[ \frac{x}{d} \right].
  \]

\item[(2)]
  \emph{Show that $\sum_{d|n} \varphi\left( \frac{x}{d},\frac{n}{d} \right) = [x]$.}
  Similar to the proof of Theorem 2.2.
  Let $S$ denote the set $\{ 1, 2, \ldots, [x] \}$.
  We distribute the integers of $S$ into disjoint sets as follows.
  For each divisor $d$ of $n$, let
  \[
    A(d) = \{ k : (k,n) = d, 1 \leq k \leq x \}.
  \]
  That is, $A(d)$ contains those elements of $S$ which have the gcd $d$ with $n$.
  The sets $A(d)$ form a disjoint collection whose union is $S$.
  Therefore if $f(d)$ denotes the number of integers in $A(d)$ we have
  \[
    \sum_{d|n} f(d) = [x].
  \]
  But $(k,n) = d$ if and only if $\left( \frac{k}{d}, \frac{n}{d} \right) = 1$,
  and $0 < k \leq x$ if and only if $0 < \frac{k}{d} \leq \frac{x}{d}$.
  Therefore, if we let $q = \frac{k}{d}$,
  there is a one-to-one correspondence between the elements in $A(d)$
  and those integers $q$ satisfying
  $0 < q \leq \frac{x}{d}$, $\left( q, \frac{n}{d} \right) = 1$.
  The number of such $q$ is $\varphi\left( \frac{x}{d},\frac{n}{d} \right)$.
  Hence $f(d) = \varphi\left( \frac{x}{d},\frac{n}{d} \right)$ and thus
  \[
    \sum_{d|n} \varphi\left( \frac{x}{d},\frac{n}{d} \right) = [x].
  \]
\end{enumerate}
$\Box$ \\\\



%%%%%%%%%%%%%%%%%%%%%%%%%%%%%%%%%%%%%%%%%%%%%%%%%%%%%%%%%%%%%%%%%%%%%%%%%%%%%%%%



\emph{In Exercise 2.10, 2.11 and 2.12, $d(n)$ denotes the number of positive divisors of $n$.} \\



\subsubsection*{Exercise 2.10.}
\addcontentsline{toc}{subsubsection}{Exercise 2.10.}
\emph{Prove that $\prod_{t|n} t = n^{\frac{d(n)}{2}}$.} \\



\emph{Proof.}
\begin{enumerate}
\item[(1)]
  Note that
  $d(1) = 1$ and
  \[
    d(p_1^{\alpha_1} \cdots p_r^{\alpha_r})
    = (\alpha_1 + 1) \cdots (\alpha_r + 1)
    = d(p_1^{\alpha_1}) \cdots d(p_r^{\alpha_r}).
  \]
  Hence $d(n)$ is multiplicative (Theorem 2.13).

\item[(2)]
  \emph{Show that $\prod_{t|n} t = n^{\frac{d(n)}{2}}$.}
  $n = 1$ is trivial.
  Assume $n = p_1^{\alpha_1} \cdots p_r^{\alpha_r} > 1$.
  Then $t|n$ if and only if $t = p_1^{x_1} \cdots p_r^{x_r}$
  with $0 \leq x_i \leq \alpha_i$ ($i = 1, \ldots, r$).
  So
  \begin{align*}
    \prod_{t|n} t
    &= \prod_{\substack{0 \leq x_1 \leq \alpha_1 \\ \cdots \\ 0 \leq x_r \leq \alpha_r}}
      p_1^{x_1} \cdots p_r^{x_r} \\
    &= p_1^{(0 + 1 + \cdots + \alpha_1)(\alpha_2 + 1) \cdots (\alpha_r + 1)}
      \cdots
      p_r^{(\alpha_1 + 1) \cdots (\alpha_{r-1} + 1)(0 + 1 + \cdots + \alpha_r)} \\
    &= p_1^{\frac{\alpha_1(\alpha_1 + 1)}{2} \cdot (\alpha_2 + 1) \cdots (\alpha_r + 1)}
      \cdots
      p_r^{(\alpha_1 + 1) \cdots (\alpha_{r-1} + 1) \cdot \frac{\alpha_r(\alpha_r + 1)}{2}} \\
    &= p_1^{\alpha_1 \frac{d(n)}{2}} \cdots p_r^{\alpha_r \frac{d(n)}{2}} \\
    &= (p_1^{\alpha_1} \cdots p_r^{\alpha_r})^{\frac{d(n)}{2}} \\
    &= n^{\frac{d(n)}{2}}.
  \end{align*}
\end{enumerate}
$\Box$ \\\\



%%%%%%%%%%%%%%%%%%%%%%%%%%%%%%%%%%%%%%%%%%%%%%%%%%%%%%%%%%%%%%%%%%%%%%%%%%%%%%%%



\subsubsection*{Exercise 2.11.}
\addcontentsline{toc}{subsubsection}{Exercise 2.11.}
\emph{Prove that $d(n)$ is odd if, and only if, $n$ is a square.} \\



\emph{Proof.}
  $n = 1$ is trivial.
  Assume $n = p_1^{\alpha_1} \cdots p_r^{\alpha_r} > 1$.
  Then
  \begin{align*}
    & \:
    \text{$d(n) = (\alpha_1 + 1) \cdots (\alpha_r + 1)$ is odd}
      &(\text{Exercise 2.10}) \\
    \Longleftrightarrow & \:
    \text{$\alpha_1 + 1, \ldots, \alpha_r + 1$ are odd} \\
    \Longleftrightarrow & \:
    \text{$\alpha_1, \ldots, \alpha_r$ are even} \\
    \Longleftrightarrow & \:
    \text{$n$ is a square}.
  \end{align*}
$\Box$ \\\\



%%%%%%%%%%%%%%%%%%%%%%%%%%%%%%%%%%%%%%%%%%%%%%%%%%%%%%%%%%%%%%%%%%%%%%%%%%%%%%%%



\subsubsection*{Exercise 2.12.}
\addcontentsline{toc}{subsubsection}{Exercise 2.12.}
\emph{Prove that $\sum_{t|n} d(t)^3 = \left(\sum_{t|n} d(t)\right)^2$.} \\



\emph{Proof.}
\begin{enumerate}
\item[(1)]
  Exercise 2.10 shows that $d(n)$ is multiplicative.
  Similar to the proof of Exercise 2.7,
  both $f(n) := \sum_{t|n} d(t)^3$ and $g(n) := \left(\sum_{t|n} d(t)\right)^2$
  are multiplicative.
  So it suffices to show that $f(p^a) = g(p^a)$ (Theorem 2.13).

\item[(2)]
  A direct calculation shows that
  \begin{align*}
    f(p^a)
    &= \sum_{t|p^a} d(t)^3 \\
    &= d(1)^3 + d(p)^3 + \cdots + d(p^a)^3 \\
    &= 1^3 + 2^3 + \cdots + (a+1)^3 \\
    &= \left( \frac{(a+1)(a+2)}{2} \right)^2
  \end{align*}
  and
  \begin{align*}
    g(p^a)
    &= \left( \sum_{t|p^a} d(t) \right)^2 \\
    &= (d(1) + d(p) + \cdots + d(p^a))^2 \\
    &= (1 + 2 + \cdots + (a+1))^2 \\
    &= \left( \frac{(a+1)(a+2)}{2} \right)^2
  \end{align*}
  are equal.
\end{enumerate}
$\Box$ \\\\



%%%%%%%%%%%%%%%%%%%%%%%%%%%%%%%%%%%%%%%%%%%%%%%%%%%%%%%%%%%%%%%%%%%%%%%%%%%%%%%%
%%%%%%%%%%%%%%%%%%%%%%%%%%%%%%%%%%%%%%%%%%%%%%%%%%%%%%%%%%%%%%%%%%%%%%%%%%%%%%%%



\newpage
\section*{Chapter 3: Average of arithmetical functions \\}
\addcontentsline{toc}{section}{Chapter 3: Average of arithmetical functions}



\subsubsection*{Exercise 3.1.}
\addcontentsline{toc}{subsubsection}{Exercise 3.1.}
\emph{Use Euler's summation formula to deduce the following for $x \geq 2$:}
\begin{enumerate}
\item[(a)]
  \emph{$\sum_{n \leq x} \frac{\log n}{n}
  = \frac{1}{2} (\log x)^2 + A + O\left( \frac{\log x}{x} \right)$,
  where $A$ is a constant.}

\item[(b)]
  \emph{$\sum_{2 \leq n \leq x} \frac{1}{n\log n}
  = \log\log x + B + O\left( \frac{1}{x\log x} \right)$,
  where $B$ is a constant.} \\
\end{enumerate}



\emph{Proof of (a).}
\begin{enumerate}
\item[(1)]
  Similar to the proof of Theorem 3.2.
  We take $f(t) = \frac{\log t}{t}$ in Euler's summation formula to obtain
  \begin{align*}
    \sum_{n \leq x} \frac{\log n}{n}
    = & \: \int_{1}^{x} \frac{\log t}{t} dt
        + \int_{1}^{x}(t-[t])\frac{1-\log t}{t^2} dt \\
      & \: + \frac{\log x}{x}([x]-x) - \underbrace{\frac{\log(1)}{1}([1]-1)}_{= 0} \\
    = & \: \frac{1}{2} (\log x)^2
        + \int_{1}^{x}(t-[t])\frac{1-\log t}{t^2} dt + O\left( \frac{\log x}{x} \right) \\
    = & \: \frac{1}{2} (\log x)^2 + \int_{1}^{\infty}(t-[t])\frac{1-\log t}{t^2} dt \\
      & \: - \int_{x}^{\infty}(t-[t])\frac{1-\log t}{t^2} dt + O\left( \frac{\log x}{x} \right).
  \end{align*}

\item[(2)]
  The improper integral $\int_{1}^{\infty}(t-[t])\frac{1-\log t}{t^2} dt$
  exists since it is dominated by
  $\int_{1}^{e} \frac{1-\log t}{t^2} dt + \int_{e}^{\infty} \frac{\log t - 1}{t^2} dt = 2e^{-1}$.

\item[(3)]
  Might assume that $x \geq e$.
  So
  \[
    0
    \leq -\int_{x}^{\infty}(t-[t])\frac{1-\log t}{t^2} dt
    \leq \int_{x}^{\infty}\frac{\log t - 1}{t^2} dt
    = \frac{\log x}{x}.
  \]

\item[(4)]
  Therefore
  \[
    \sum_{n \leq x} \frac{\log n}{n}
    = \frac{1}{2} (\log x)^2 + A + O\left( \frac{\log x}{x} \right)
  \]
  where $A = \int_{1}^{\infty}(t-[t])\frac{1-\log t}{t^2} dt$ is a constant.
\end{enumerate}
$\Box$ \\



\emph{Proof of (b).}
\begin{enumerate}
\item[(1)]
  We take $f(t) = \frac{1}{t\log t}$ in Euler's summation formula to obtain
  \begin{align*}
    \sum_{2 \leq n \leq x} \frac{1}{n\log n}
    = & \: \int_{2}^{x} \frac{1}{t\log t} dt
        + \int_{2}^{x} -(t-[t])\frac{\log t + 1}{t^2 (\log t)^2} dt \\
      & \: + \frac{1}{x \log x}([x]-x) - \underbrace{\frac{1}{2 \cdot \log(2)}([2]-2)}_{= 0} \\
    = & \: \log\log x - \log\log2
        - \int_{2}^{x} (t-[t])\frac{\log t + 1}{t^2 (\log t)^2} dt \\
      & \: + O\left( \frac{1}{x \log x} \right) \\
    = & \: \log\log x - \log\log2
        - \int_{2}^{\infty} (t-[t])\frac{\log t + 1}{t^2 (\log t)^2} dt \\
      & \: + \int_{x}^{\infty} (t-[t])\frac{\log t + 1}{t^2 (\log t)^2} dt
        + O\left( \frac{1}{x \log x} \right).
  \end{align*}

\item[(2)]
  The improper integral $\int_{2}^{\infty} (t-[t])\frac{\log t + 1}{t^2 (\log t)^2} dt$
  exists since it is dominated by
  $\int_{2}^{\infty} \frac{\log t + 1}{t^2 (\log t)^2} dt = \frac{1}{2\log 2} < \infty$.

\item[(3)]
  \[
    0
    \leq \int_{x}^{\infty} (t-[t])\frac{\log t + 1}{t^2 (\log t)^2} dt
    \leq \int_{x}^{\infty} \frac{\log t + 1}{t^2 (\log t)^2} dt
    = \frac{1}{x\log x}.
  \]

\item[(4)]
  Therefore
  \[
    \sum_{2 \leq n \leq x} \frac{1}{n\log n}
    = \log\log x + B + O\left( \frac{1}{x\log x} \right)
  \]
  where
  $B = - \log\log2 - \int_{2}^{\infty} (t-[t])\frac{\log t + 1}{t^2 (\log t)^2} dt$
  is a constant.
\end{enumerate}
$\Box$ \\\\



%%%%%%%%%%%%%%%%%%%%%%%%%%%%%%%%%%%%%%%%%%%%%%%%%%%%%%%%%%%%%%%%%%%%%%%%%%%%%%%%



\subsubsection*{Exercise 3.2.}
\addcontentsline{toc}{subsubsection}{Exercise 3.2.}
\emph{If $x \geq 2$ prove that
\[
  \sum_{n \leq x} \frac{d(n)}{n}
  = \frac{1}{2}(\log x)^2 + 2 C \log x + O(1),
\]
where $C$ is Euler's constant.} \\



\emph{Proof.}
  Similar to the proof of Theorem 3.3, we have
  \[
    \sum_{n \leq x} \frac{d(n)}{n}
    = \sum_{n \leq x} \frac{1}{n} \sum_{d|n} 1
    = \sum_{\substack{q,d \\ qd \leq x}} \frac{1}{qd}
    = \sum_{d \leq x} \frac{1}{d} \sum_{q \leq \frac{x}{d}} \frac{1}{q}.
  \]
  Now we use Theorem 3.2(a) to obtain
  \[
    \sum_{q \leq \frac{x}{d}} \frac{1}{q}
    = \log\frac{x}{d} + C + O\left(\frac{d}{x}\right)
    = \log x - \log d + C + O\left(\frac{d}{x}\right).
  \]
  Using this along with Theorem 3.2(a) and Exercise 3.1 we find
  \begin{align*}
    \sum_{n \leq x} \frac{d(n)}{n}
    =& \: \sum_{d \leq x}
        \frac{1}{d} \left\{ \log x - \log d + C + O\left(\frac{d}{x}\right) \right\} \\
    =& \: (\log x + C) \sum_{d \leq x} \frac{1}{d}
        - \sum_{d \leq x} \frac{\log d}{d} + \sum_{d \leq x} O\left(\frac{1}{x}\right) \\
    =& \: (\log x + C)\left\{ \log x + C + O\left(\frac{1}{x}\right) \right\} \\
      & - \left\{ \frac{1}{2} (\log x)^2 + A + O\left( \frac{\log x}{x} \right) \right\} + O(1) \\
    =& \: (\log x)^2 + 2 C \log x - \frac{1}{2} (\log x)^2 + O(1) \\
    =& \: \frac{1}{2} (\log x)^2 + 2C \log x + O(1).
  \end{align*}
$\Box$ \\\\



%%%%%%%%%%%%%%%%%%%%%%%%%%%%%%%%%%%%%%%%%%%%%%%%%%%%%%%%%%%%%%%%%%%%%%%%%%%%%%%%



\subsubsection*{Exercise 3.3.}
\addcontentsline{toc}{subsubsection}{Exercise 3.3.}
\emph{If $x \geq 2$ and $\alpha > 0$, $\alpha \neq 1$, prove that
\[
  \sum_{n \leq x} \frac{d(n)}{n^{\alpha}}
  = \frac{x^{1-\alpha} \log x}{1 - \alpha} + \zeta(\alpha)^2 + O(x^{1-\alpha}).
\]}



\emph{Proof.}
\begin{enumerate}
\item[(1)]
  Similar to Exercise 3.2.
  \[
    \sum_{n \leq x} \frac{d(n)}{n^{\alpha}}
    = \sum_{n \leq x} \frac{1}{n^{\alpha}} \sum_{d|n} 1
    = \sum_{\substack{q,d \\ qd \leq x}} \frac{1}{q^{\alpha}d^{\alpha}}
    = \sum_{d \leq x} \frac{1}{d^{\alpha}} \sum_{q \leq \frac{x}{d}} \frac{1}{q^{\alpha}}.
  \]
  Now we use Theorem 3.2(b) to obtain
  \[
    \sum_{q \leq \frac{x}{d}} \frac{1}{q^{\alpha}}
    = \frac{1}{d^{1-\alpha}} \cdot \frac{x^{1-\alpha}}{1-\alpha} + \zeta(\alpha)
        + O\left(\frac{d^{\alpha}}{x^{\alpha}}\right).
  \]
  Using this along with Theorem 3.2 we find
  \begin{align*}
    \sum_{n \leq x} \frac{d(n)}{n^{\alpha}}
    =& \: \sum_{d \leq x}
        \frac{1}{d^{\alpha}} \left\{ \frac{1}{d^{1-\alpha}} \cdot \frac{x^{1-\alpha}}{1-\alpha}
            + \zeta(\alpha)
            + O\left(\frac{d^{\alpha}}{x^{\alpha}}\right) \right\} \\
    =& \: \frac{x^{1-\alpha}}{1-\alpha} \sum_{d \leq x} \frac{1}{d}
        + \zeta(\alpha) \sum_{d \leq x} \frac{1}{d^{\alpha}}
        + \sum_{d \leq x} O(x^{-\alpha}) \\
    =& \: \frac{x^{1-\alpha}}{1-\alpha} \left\{ \log x + C + O(x^{-1}) \right\} \\
      & + \zeta(\alpha)
          \left\{ \frac{x^{1-\alpha}}{1-\alpha} + \zeta(\alpha) + O(x^{-\alpha}) \right\}
          + O(x^{1-\alpha}) \\
    =& \: \frac{x^{1-\alpha} \log x}{1 - \alpha} + \zeta(\alpha)^2 + O(x^{1-\alpha}).
  \end{align*}
\end{enumerate}
$\Box$ \\\\



%%%%%%%%%%%%%%%%%%%%%%%%%%%%%%%%%%%%%%%%%%%%%%%%%%%%%%%%%%%%%%%%%%%%%%%%%%%%%%%%



%%%%%%%%%%%%%%%%%%%%%%%%%%%%%%%%%%%%%%%%%%%%%%%%%%%%%%%%%%%%%%%%%%%%%%%%%%%%%%%%



\subsubsection*{Exercise 3.5.}
\addcontentsline{toc}{subsubsection}{Exercise 3.5.}
\emph{If $x \geq 1$ prove that:}
\begin{enumerate}
\item[(a)]
  \emph{$\sum_{n \leq x} \varphi(n)
  = \frac{1}{2} \sum_{n \leq x} \mu(n)\left[ \frac{x}{n} \right]^2 + \frac{1}{2}$.}

\item[(b)]
  \emph{$\sum_{n \leq x} \frac{\varphi(n)}{n}
  = \sum_{n \leq x} \frac{\mu(n)}{n}\left[ \frac{x}{n} \right]$.}
\end{enumerate}
\emph{These formulas, together with those in Exercise 3.4, show that, for $x \geq 2$,
\[
  \sum_{n \leq x} \varphi(n)
  = \frac{1}{2} \frac{x^2}{\zeta(2)} + O(x \log x),
  \qquad
  \sum_{n \leq x} \frac{\varphi(n)}{n}
  = \frac{x}{\zeta(2)} + O(\log x).
\]} \\

The last two formulas are trivial and we omit the proof. \\

\emph{Proof of (a).}
  Same as the proof of Theorem 3.7.
  \begin{align*}
    \sum_{n \leq x} \varphi(n)
    &= \sum_{n \leq x} \sum_{d|n} \mu(d) \frac{n}{d} \\
    &= \sum_{\substack{q, d \\ qd \leq x}} \mu(d) q \\
    &= \sum_{d \leq x} \mu(d) \sum_{q \leq \frac{x}{d}} q \\
    &= \sum_{d \leq x}
        \mu(d) \frac{1}{2} \left[\frac{x}{d}\right]\left(1 + \left[\frac{x}{d}\right]\right) \\
    &= \frac{1}{2} \sum_{d \leq x} \mu(d)\left[ \frac{x}{d} \right]^2
        + \frac{1}{2} \sum_{d \leq x} \mu(d)\left[ \frac{x}{d} \right] \\
    &= \frac{1}{2} \sum_{d \leq x} \mu(d)\left[ \frac{x}{d} \right]^2 + \frac{1}{2}
      &(\text{Theorem 3.12})
  \end{align*}
$\Box$ \\



\emph{Proof of (b).}
\begin{enumerate}
\item[(1)]
  \begin{align*}
    \sum_{n \leq x} \frac{\varphi(n)}{n}
    &= \sum_{n \leq x} \sum_{d|n} \frac{\mu(d)}{d}
      &(\text{Theorem 2.3}) \\
    &= \sum_{n \leq x} \frac{\mu(n)}{n} \left[ \frac{x}{n} \right].
      &(\text{Theorem 3.11})
  \end{align*}
\end{enumerate}
$\Box$ \\\\



%%%%%%%%%%%%%%%%%%%%%%%%%%%%%%%%%%%%%%%%%%%%%%%%%%%%%%%%%%%%%%%%%%%%%%%%%%%%%%%%



%%%%%%%%%%%%%%%%%%%%%%%%%%%%%%%%%%%%%%%%%%%%%%%%%%%%%%%%%%%%%%%%%%%%%%%%%%%%%%%%



%%%%%%%%%%%%%%%%%%%%%%%%%%%%%%%%%%%%%%%%%%%%%%%%%%%%%%%%%%%%%%%%%%%%%%%%%%%%%%%%
%%%%%%%%%%%%%%%%%%%%%%%%%%%%%%%%%%%%%%%%%%%%%%%%%%%%%%%%%%%%%%%%%%%%%%%%%%%%%%%%



\end{document}