\documentclass{article}
\usepackage{amsfonts}
\usepackage{amsmath}
\usepackage{amssymb}
\usepackage{centernot}
\usepackage{hyperref}
\usepackage[none]{hyphenat}
\usepackage{mathrsfs}
\usepackage{mathtools}
\usepackage{physics}
\usepackage{tikz-cd}
\parindent=0pt

\def\upint{\mathchoice%
    {\mkern13mu\overline{\vphantom{\intop}\mkern7mu}\mkern-20mu}%
    {\mkern7mu\overline{\vphantom{\intop}\mkern7mu}\mkern-14mu}%
    {\mkern7mu\overline{\vphantom{\intop}\mkern7mu}\mkern-14mu}%
    {\mkern7mu\overline{\vphantom{\intop}\mkern7mu}\mkern-14mu}%
  \int}
\def\lowint{\mkern3mu\underline{\vphantom{\intop}\mkern7mu}\mkern-10mu\int}

\makeatletter
\newcommand*{\mint}[1]{%
  % #1: overlay symbol
  \mint@l{#1}{}%
}
\newcommand*{\mint@l}[2]{%
  % #1: overlay symbol
  % #2: limits
  \@ifnextchar\limits{%
    \mint@l{#1}%
  }{%
    \@ifnextchar\nolimits{%
      \mint@l{#1}%
    }{%
      \@ifnextchar\displaylimits{%
        \mint@l{#1}%
      }{%
        \mint@s{#2}{#1}%
      }%
    }%
  }%
}
\newcommand*{\mint@s}[2]{%
  % #1: limits
  % #2: overlay symbol
  \@ifnextchar_{%
    \mint@sub{#1}{#2}%
  }{%
    \@ifnextchar^{%
      \mint@sup{#1}{#2}%
    }{%
      \mint@{#1}{#2}{}{}%
    }%
  }%
}
\def\mint@sub#1#2_#3{%
  \@ifnextchar^{%
    \mint@sub@sup{#1}{#2}{#3}%
  }{%
    \mint@{#1}{#2}{#3}{}%
  }%
}
\def\mint@sup#1#2^#3{%
  \@ifnextchar_{%
    \mint@sup@sub{#1}{#2}{#3}%
  }{%
    \mint@{#1}{#2}{}{#3}%
  }%
}
\def\mint@sub@sup#1#2#3^#4{%
  \mint@{#1}{#2}{#3}{#4}%
}
\def\mint@sup@sub#1#2#3_#4{%
  \mint@{#1}{#2}{#4}{#3}%
}
\newcommand*{\mint@}[4]{%
  % #1: \limits, \nolimits, \displaylimits
  % #2: overlay symbol: -, =, ...
  % #3: subscript
  % #4: superscript
  \mathop{}%
  \mkern-\thinmuskip
  \mathchoice{%
    \mint@@{#1}{#2}{#3}{#4}%
        \displaystyle\textstyle\scriptstyle
  }{%
    \mint@@{#1}{#2}{#3}{#4}%
        \textstyle\scriptstyle\scriptstyle
  }{%
    \mint@@{#1}{#2}{#3}{#4}%
        \scriptstyle\scriptscriptstyle\scriptscriptstyle
  }{%
    \mint@@{#1}{#2}{#3}{#4}%
        \scriptscriptstyle\scriptscriptstyle\scriptscriptstyle
  }%
  \mkern-\thinmuskip
  \int#1%
  \ifx\\#3\\\else_{#3}\fi
  \ifx\\#4\\\else^{#4}\fi
}
\newcommand*{\mint@@}[7]{%
  % #1: limits
  % #2: overlay symbol
  % #3: subscript
  % #4: superscript
  % #5: math style
  % #6: math style for overlay symbol
  % #7: math style for subscript/superscript
  \begingroup
    \sbox0{$#5\int\m@th$}%
    \sbox2{$#5\int_{}\m@th$}%
    \dimen2=\wd0 %
    % => \dimen2 = width of \int
    \let\mint@limits=#1\relax
    \ifx\mint@limits\relax
      \sbox4{$#5\int_{\kern1sp}^{\kern1sp}\m@th$}%
      \ifdim\wd4>\wd2 %
        \let\mint@limits=\nolimits
      \else
        \let\mint@limits=\limits
      \fi
    \fi
    \ifx\mint@limits\displaylimits
      \ifx#5\displaystyle
        \let\mint@limits=\limits
      \fi
    \fi
    \ifx\mint@limits\limits
      \sbox0{$#7#3\m@th$}%
      \sbox2{$#7#4\m@th$}%
      \ifdim\wd0>\dimen2 %
        \dimen2=\wd0 %
      \fi
      \ifdim\wd2>\dimen2 %
        \dimen2=\wd2 %
      \fi
    \fi
    \rlap{%
      $#5%
        \vcenter{%
          \hbox to\dimen2{%
            \hss
            $#6{#2}\m@th$%
            \hss
          }%
        }%
      $%
    }%
  \endgroup
}



\title{\textbf{Solutions to the book: \\
\emph{Lawrence C. Evans, Partial Differential Equations}}}
\author{Meng-Gen Tsai \\ plover@gmail.com}



\begin{document}
\maketitle
\tableofcontents



%%%%%%%%%%%%%%%%%%%%%%%%%%%%%%%%%%%%%%%%%%%%%%%%%%%%%%%%%%%%%%%%%%%%%%%%%%%%%%%%
%%%%%%%%%%%%%%%%%%%%%%%%%%%%%%%%%%%%%%%%%%%%%%%%%%%%%%%%%%%%%%%%%%%%%%%%%%%%%%%%



% Reference:
% https://math24.files.wordpress.com/2013/02/evans_solutions-ch2.pdf
% http://math.gmu.edu/~memelian/teaching/Fall11/math678/hw/hw1sol.pdf
% http://www-users.math.umn.edu/~sands265/writeups/PDE_homework1.pdf
% http://www-users.math.umn.edu/~sands265/writeups/PDE_homework2.pdf



%%%%%%%%%%%%%%%%%%%%%%%%%%%%%%%%%%%%%%%%%%%%%%%%%%%%%%%%%%%%%%%%%%%%%%%%%%%%%%%%
%%%%%%%%%%%%%%%%%%%%%%%%%%%%%%%%%%%%%%%%%%%%%%%%%%%%%%%%%%%%%%%%%%%%%%%%%%%%%%%%



\newpage
\section*{Chapter 2: Four Important Linear PDE \\}
\addcontentsline{toc}{section}{Chapter 2: Four Important Linear PDE}



\subsubsection*{Notes.}
\addcontentsline{toc}{subsubsection}{Notes.}
\begin{enumerate}
\item[(1)]
  (Equation (7) in \S 2.2.2)
  \[
    |D\Phi(x)| \leq \frac{C}{|x|^{n-1}}, \qquad
    |D^2\Phi(x)| \leq \frac{C}{|x|^{n}} \qquad (x \neq 0)
  \]
  for some constant $C > 0$.
  In fact,
  \begin{align*}
  \frac{\partial}{\partial x_i}\Phi(x)
  &= -\frac{1}{n\alpha(n)} x_i |x|^{-n}, \\
  \frac{\partial^2}{\partial x_i \partial x_j}\Phi(x)
  &= \frac{1}{n\alpha(n)} (n x_i x_j - |x|^2 \delta_{ij}) |x|^{-n-2}.
  \end{align*}

\item[(2)]
  (Equation (12) in \S 2.2.2)
  The constant $C$ is rescaled.
  It is just a constant.

\item[(3)]
  (Equation (13) in \S 2.2.2)
  Take $U \mapsto B(0,\varepsilon)$, $u(y) \mapsto \Phi(y)$ and $v(y) \mapsto f(x-y)$
  in the integration by parts (Green's first identity):
  \[
    \int_{U} Dv \cdot Du \: dx
    = -\int_{U} u \Delta v \: dx
      + \int_{\partial U} \frac{\partial v}{\partial \nu} u \: dS.
  \] \\
\end{enumerate}



%%%%%%%%%%%%%%%%%%%%%%%%%%%%%%%%%%%%%%%%%%%%%%%%%%%%%%%%%%%%%%%%%%%%%%%%%%%%%%%%



\subsubsection*{Problem 2.1.}
\addcontentsline{toc}{subsubsection}{Problem 2.1.}
\emph{Write down an explicit formula for a function $u$ solving
the initial-value problem
\begin{equation*}
  \begin{cases}
    u_t + b \cdot Du + cu = 0 & \text{ in } \mathbb{R}^n \times (0,\infty) \\
    u                     = g & \text{ on } \mathbb{R}^n \times \{ t=0 \}.
  \end{cases}
\end{equation*}
Here $c \in \mathbb{R}$ and $b \in \mathbb{R}^n$ are constants.} \\

\emph{Proof (Transport equation).}
Define
\[
  z(s) = u(x+sb,t+s) \qquad (s \in \mathbb{R}).
\]
So
\begin{align*}
  \dot{z}(s)
  &= Du(x+sb,t+s) \cdot b + u_t(x+sb,t+s) \\
  &= -cu(x+sb,t+s) \\
  &= -cz(s).
\end{align*}
Solve this ODE to get
\begin{align*}
  z(s) = z(0) e^{-cs}
  &\Longrightarrow
  u(x+sb,t+s) = u(x,t) e^{-cs} \\
  &\Longrightarrow
  u(x-tb,0) = u(x,t) e^{ct}
    &\text{(Let $s = -t$)} \\
  &\Longrightarrow
  g(x-tb) = u(x,t) e^{ct} \\
  &\Longrightarrow
  u(x,t) = g(x-tb) e^{-ct}.
\end{align*}
$\Box$ \\\\



%%%%%%%%%%%%%%%%%%%%%%%%%%%%%%%%%%%%%%%%%%%%%%%%%%%%%%%%%%%%%%%%%%%%%%%%%%%%%%%%



\subsubsection*{Problem 2.2.}
\addcontentsline{toc}{subsubsection}{Problem 2.2.}
\emph{Prove that Laplace's equation $\Delta u = 0$ is rotation invariant;
that is, if $O$ is an orthogonal $n \times n$ matrix and we define
\[
  v(x) := u(Ox) \qquad (x \in \mathbb{R}^n),
\]
then $\Delta v = 0$.} \\

\emph{Proof.}
\begin{enumerate}
\item[(1)]
  Let $O = [O_{ij}]$.
  $O$ is orthogonal if $OO^{T} = O^{T}O = I$,
  or
  \[
    \sum_{i=1}^{n} O_{pi} O_{qi} = \delta_{pq}
  \]
  where $\delta_{pq}$ is the Kronecker delta.

\item[(2)]
  Let $y = Ox$.
  So that
  \begin{align*}
    D_i v(x)
    &= \sum_{p=1}^{n} D_p u(y) O_{pi}, \\
    D_{ij} v(x)
    &= \sum_{q=1}^{n} \sum_{p=1}^{n} D_{pq}u(y) O_{pi} O_{qj}, \\
    \Delta v(x)
    &= \sum_{i=1}^{n} D_{ii} v(x) \\
    &= \sum_{i=1}^{n} \sum_{q=1}^{n}\sum_{p=1}^{n} D_{pq}u(y) O_{pi} O_{qi} \\
    &= \sum_{q=1}^{n} \sum_{p=1}^{n} D_{pq} u(y) \left( \sum_{i=1}^{n} O_{pi} O_{qi} \right) \\
    &= \sum_{q=1}^{n} \sum_{p=1}^{n} D_{pq} \delta_{pq} \\
    &= \sum_{q=1}^{n} D_{qq}u(y) \\
    &= \Delta u(y).
  \end{align*}

\item[(3)]
  As $\Delta u(y) = 0$, $\Delta v(x) = 0$.
\end{enumerate}
$\Box$ \\\\



%%%%%%%%%%%%%%%%%%%%%%%%%%%%%%%%%%%%%%%%%%%%%%%%%%%%%%%%%%%%%%%%%%%%%%%%%%%%%%%%



\textbf{Problem 2.4.}
\emph{We say $v \in C^2(\overline{U})$ is \textbf{subharmonic} if
\[
  - \Delta v \leq 0
  \qquad
  \text{in $U$}.
\]}
\begin{enumerate}
\item[(a)]
  \emph{Prove for subharmonic $v$ that}
  \[
    v(x) \leq \mint{-}_{B(x,r)} v dy
    \qquad
    \text{for all $B(x,r) \subseteq U$}.
  \]

\item[(b)]
  \emph{Prove that therefore $\max_{\overline{U}} v = \max_{\partial U} v$.}

\item[(c)]
  \emph{Let $\phi: \mathbb{R} \to \mathbb{R}$ be smooth and convex.
  Assume $u$ is harmonic and $v := \phi(u)$.
  Prove that $v$ is subharmonic.}

\item[(d)]
  \emph{Prove $v := |Du|^2$ is subharmonic, whenever $u$ is harmonic.} \\
\end{enumerate}



\emph{Proof of (a).}
It is exactly the same as the proof of Theorem 2 (Mean-value theorem for Laplace's equation)
in \S 2.2.2.
\begin{enumerate}
\item[(1)]
  Set
  \[
    \phi(r)
    := \mint{-}_{\partial B(x,r)} v(y) dS(y)
    = \mint{-}_{\partial B(0,1)} v(x+rz) dS(z)
  \]
  ($r > 0$).
  Then
  \begin{align*}
    \phi'(r)
    &= \mint{-}_{\partial B(0,1)} Dv(\underbrace{x+rz}_{= y}) \cdot z dS(z) \\
    &= \mint{-}_{\partial B(x,y)} Dv(y) \cdot
      \underbrace{\frac{y - x}{r}}_{= \nu} dS(y) \\
    &= \mint{-}_{\partial B(x,y)} \frac{\partial v}{\partial \nu} dS(y) \\
    &= \frac{r}{n} \mint{-}_{B(x,y)} \Delta u(y) dy
      &(\text{Green's first identity}) \\
    &\geq 0
      &(\text{By assumption})
  \end{align*}
  or $\phi(r)$ is increasing.

\item[(2)]
  Note that
  \[
    \lim_{t \to 0} \phi(t)
    = \lim_{t \to 0} \mint{-}_{\partial B(x,t)} v(y) dS(y)
    = v(x).
  \]
  So that
  \[
    v(x) = \lim_{t \to 0} \phi(t) \leq \phi(r) = \mint{-}_{\partial B(x,r)} v(y) dS(y).
  \]

\item[(3)]
  Hence, for all $B(x,r) \subseteq U$
  we have
  \begin{align*}
    \mint{-}_{B(x,r)} v dy
    &= \frac{1}{\alpha(n) r^n} \int_{B(x,r)} v dy \\
    &= \frac{1}{\alpha(n) r^n} \int_{0}^{r} \int_{\partial B(x,\rho)} v(y) dS(y) d\rho
      &(\text{Polar coordinates}) \\
    &\geq \frac{1}{\alpha(n) r^n} \int_{0}^{r} n\alpha(n) \rho^{n-1} v(x) d\rho
      &((2)) \\
    &= v(x) \frac{1}{r^n} \underbrace{\int_{0}^{r} n \rho^{n-1} d\rho}_{= r^n} \\
    &= v(x).
  \end{align*}
\end{enumerate}
$\Box$ \\



\emph{Proof of (b).}
Similar to the proof of Theorem 4 (Strong maximum principle) in \S 2.2.2.

\begin{enumerate}
\item[(1)]
  Suppose there exists a point $x_0 \in U$ with
  $v(x_0) = M := \max_{\overline{U}} v$.
  Then for $0 < r < \mathrm{dist}(x_0,\partial U)$, the mean-value property (in (a)) asserts 
  \[
    M = v(x_0) \leq \mint{-}_{B(x_0,r)} v dy \leq M.
  \]
  As equality holds only if $v \equiv  M$ within $B(x_0,r)$,
  we see $v = M$ for all $y \in B(x,r)$.
  Hence the set $\{ x \in U : v(x) = M \}$ is both open and closed in $U$ (since $v \in C(\overline{U})$),
  and thus eqauls to one connected component $U_\alpha$ of $U$.
  By the definition of
  $\partial U_{\alpha} \subseteq \overline{U_{\alpha}}$
  and continuity of $v$,
  $v|_{\partial U_{\alpha}} \equiv M$.
  As $\partial U_{\alpha} \subseteq \partial U$,
  the result is established.

\item[(2)]
  If no such point $x_0 \in U$ with $v(x_0) = \max_{\overline{U}} v$,
  then $\max_{\overline{U}} v = \max_{\partial U} v$ is trivial.
\end{enumerate}
$\Box$ \\



\emph{Proof of (c).}
\begin{enumerate}
\item[(1)]
  \begin{align*}
    \Delta v
    &= \sum_{i=1}^{n} v_{x_i x_i} \\
    &= \sum_{i=1}^{n} (\phi'(u) u_{x_i})_{x_i} \\
    &= \sum_{i=1}^{n} \phi''(u) u_{x_i}^2 + \phi'(u) u_{x_i x_i} \\
    &= \phi''(u) |Du|^2 + \phi'(u) \Delta u.
  \end{align*}

\item[(2)]
  As $u$ is harmonic ($\Delta u = 0$) and $\phi$ is convex ($\phi''(u) \geq 0$ by
  Exercise 5.14 in the textbook:
  \emph{Rudin, Principles of Mathematical Analysis, 3rd edition}),
  $\Delta v \geq 0$ (by (1)).

\end{enumerate}
$\Box$ \\



\emph{Proof of (d).}
\begin{enumerate}
\item[(1)]
  Since $u$ is smooth,
  $u$ is harmonic implies that $u_{x_j}$ is harmonic for all $x_j$.
  In fact,
  \begin{align*}
    \Delta (u_{x_j})
    &= \sum_{i=1}^{n} (u_{x_j})_{x_i x_i} \\
    &= \sum_{i=1}^{n} u_{x_i x_i x_j}
      &(\text{Smoothness of $u$}) \\
    &= \left(\sum_{i=1}^{n} u_{x_i x_i}\right)_{x_j} \\
    &= \left(\Delta u\right)_{x_j} \\
    &= 0.
  \end{align*}

\item[(2)]
  Since $x \mapsto x^2$ is convex and $u_{x_i}$ is harmonic (by (1)),
  \begin{align*}
    v := |Du|^2 = \sum_{i=1}^n (u_{x_i})^2
  \end{align*}
  is a finite sum of subharmonic functions by (3),
  which is also subharmonic.
\end{enumerate}
$\Box$ \\\\



%%%%%%%%%%%%%%%%%%%%%%%%%%%%%%%%%%%%%%%%%%%%%%%%%%%%%%%%%%%%%%%%%%%%%%%%%%%%%%%%
%%%%%%%%%%%%%%%%%%%%%%%%%%%%%%%%%%%%%%%%%%%%%%%%%%%%%%%%%%%%%%%%%%%%%%%%%%%%%%%%



\end{document}