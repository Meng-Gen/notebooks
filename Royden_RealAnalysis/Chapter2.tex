\documentclass{article}
\usepackage{amsfonts}
\usepackage{amsmath}
\usepackage{amssymb}
\usepackage{hyperref}
\usepackage{mathrsfs}
\parindent=0pt

\def\upint{\mathchoice%
    {\mkern13mu\overline{\vphantom{\intop}\mkern7mu}\mkern-20mu}%
    {\mkern7mu\overline{\vphantom{\intop}\mkern7mu}\mkern-14mu}%
    {\mkern7mu\overline{\vphantom{\intop}\mkern7mu}\mkern-14mu}%
    {\mkern7mu\overline{\vphantom{\intop}\mkern7mu}\mkern-14mu}%
  \int}
\def\lowint{\mkern3mu\underline{\vphantom{\intop}\mkern7mu}\mkern-10mu\int}

\begin{document}

\textbf{\Large Chapter 2: The Real Number System} \\\\



\emph{Author: Meng-Gen Tsai} \\
\emph{Email: plover@gmail.com} \\\\



\textbf{Problem 2.1.}
\emph{Show that $1 \in P$.} \\

\emph{Proof.}
By the field axioms,
\begin{enumerate}
\item[(a)]
$1 \in \mathbb{R}$ such that $1 \neq 0$.
\item[(b)]
$-1 \in \mathbb{R}$.
\item[(c)]
$(-1) \cdot (-1) = 1$.
\end{enumerate}

By the axioms of order,
$1 = 0$ or $1 \in P$ or $-1 \in P$.
Consider three possible cases,
\begin{enumerate}
\item[(1)]
$1 = 0$, contrary to the field axioms $1 \neq 0$.
\item[(2)]
$1 \in P$.
\item[(3)]
$-1 \in P$.
By the axioms of order, $(-1)(-1) \in P$.
Since $(-1)(-1) = 1$ by the field axioms, $1 \in P$.
By the axioms of order, $-1 \not\in P$, contrary to $-1 \in P$.
\end{enumerate}
By (1)(2)(3), $1 \in P$.
$\Box$ \\

Applying the similar argument to $\sqrt{-1}$,
we get $\sqrt{-1} \not\in \mathbb{R}$
as our expectation. \\\\



\end{document}