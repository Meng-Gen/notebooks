\documentclass{article}
\usepackage{amsfonts}
\usepackage{amsmath}
\usepackage{amssymb}
\usepackage{hyperref}
\usepackage{mathrsfs}
\parindent=0pt

\def\upint{\mathchoice%
    {\mkern13mu\overline{\vphantom{\intop}\mkern7mu}\mkern-20mu}%
    {\mkern7mu\overline{\vphantom{\intop}\mkern7mu}\mkern-14mu}%
    {\mkern7mu\overline{\vphantom{\intop}\mkern7mu}\mkern-14mu}%
    {\mkern7mu\overline{\vphantom{\intop}\mkern7mu}\mkern-14mu}%
  \int}
\def\lowint{\mkern3mu\underline{\vphantom{\intop}\mkern7mu}\mkern-10mu\int}

\begin{document}

\textbf{\Large Chapter 1: Set Theory} \\\\



\emph{Author: Meng-Gen Tsai} \\
\emph{Email: plover@gmail.com} \\\\



\textbf{Problem 1.1.}
\emph{Show that $\{ x : x \neq x \} = \varnothing$.} \\

\emph{Proof.}
Every element $x$ of $\{ x : x \neq x \}$ satisfying $x \neq x$,
contrary to $x = x$. That is, there are no elements in $\{ x : x \neq x \}$,
or $\{ x : x \neq x \} = \varnothing$.
$\Box$ \\\\



%%%%%%%%%%%%%%%%%%%%%%%%%%%%%%%%%%%%%%%%%%%%%%%%%%%%%%%%%%%%%%%%%%%%%%%%%%%%%%%%



\textbf{Problem 1.2.}
\emph{Show that if $x \in \varnothing$, then $x$ is a green-eyed lion.} \\

\emph{Proof.}
$\varnothing \subseteq \{ \text{a green-eyed lion} \}$.
$\Box$ \\\\



%%%%%%%%%%%%%%%%%%%%%%%%%%%%%%%%%%%%%%%%%%%%%%%%%%%%%%%%%%%%%%%%%%%%%%%%%%%%%%%%



\textbf{Problem 1.4.}
\emph{Show that the well-ordering principle implies
the principle of mathematical induction.
(Hint: Consider the set $\{ n \in \mathbb{N} : P(n) \text{ is false} \}$.)} \\

\emph{Proof (Hint).}
Suppose that
\begin{enumerate}
\item[(1)]
$P(n)$ be a proposition defined for each $n \in \mathbb{N}$,
\item[(2)]
$P(1)$ is true,
\item[(3)]
$[P(n) \Rightarrow P(n+1)]$ is true.
\end{enumerate}

Consider the set
$$S = \{ n \in \mathbb{N} : P(n) \text{ is false} \} \subseteq \mathbb{N}.$$
Want to show
\emph{$S$ is empty, or the principle of mathematical induction holds.}
If $S$ were nonempty,
by the well-ordering principle $S$ has a smallest element $m$.
$m$ cannot be $1$ by (2).
Say $m > 1$.
Therefore, $m - 1 \in \mathbb{N}$
and $P(m-1)$ is true by the minimality of $m$.
By (3), $P((m-1)+1) = P(m)$ is true, which is absurd.
$\Box$ \\\\



%%%%%%%%%%%%%%%%%%%%%%%%%%%%%%%%%%%%%%%%%%%%%%%%%%%%%%%%%%%%%%%%%%%%%%%%%%%%%%%%



\textbf{Problem 1.5.}
\emph{Use mathematical induction to establish
that the well-ordering principle.
(Hint: Given a set $S$ of positive integers,
let $P(n)$ be the proposition
`If $n \in S$, then $S$ has a least element'.)} \\

\emph{Proof (Modified hint).}
\begin{enumerate}
\item[(1)]
Given a set $S$ of positive integers,
let $P(n)$ be the proposition
`If $m \in S$ for some $m \leq n$, then $S$ has a least element'.
Want to show $P(n)$ is true for all $n \in \mathbb{N}$.
\begin{enumerate}
\item[(a)]
$P(1)$ is true.
For $m \in S$ with $m \leq n = 1$,
or $m = 1$ by the minimality of $1 \in \mathbb{N}$,
$S$ has a least element $1$ ($m$ itself) in $\mathbb{N}$.
\item[(b)]
Suppose $P(n)$ is true.
If $n+1 \in S$, then there are only two possible cases.
  \begin{enumerate}
  \item[(i)]
  There is a positive integer $m \in S$ less than $n+1$.
  So $n \geq m \in S$.
  Since $P(n)$ is true, $S$ has a least element.
  \item[(ii)]
  There is no positive integer $m \in S$ less than $n+1$.
  In this case $n+1$ is the least element in $S$.
  \end{enumerate}
  In any cases (i)(ii), $S$ has a least element, or $P(n+1)$ is true.
\end{enumerate}
By mathematical induction, $P(n)$ is true for all $n \in \mathbb{N}$.
\item[(2)]
\emph{Show that the well-ordering principle holds.}
Let $T$ be a nonempty subset of $\mathbb{N}$,
so there exists a positive integer $k \in T$.
Notice that $P(k)$ is true by (1),
thus $T$ has a least element since $k \leq k$.
\end{enumerate}
$\Box$ \\\\



%%%%%%%%%%%%%%%%%%%%%%%%%%%%%%%%%%%%%%%%%%%%%%%%%%%%%%%%%%%%%%%%%%%%%%%%%%%%%%%%



\textbf{Problem 1.9.}
\emph{Show that
$A \subseteq B
\Leftrightarrow A \cap B = A
\Leftrightarrow A \cup B = B$.} \\

\emph{Proof.}
\begin{enumerate}
\item[(1)]
\emph{$A \subseteq B \Longleftrightarrow A \cap B = A$.}
  \begin{enumerate}
  \item[(a)]
  \emph{$(\Longrightarrow)$}
  It suffices to show $A \cap B \supseteq A$.
  For any $x \in A$, $x \in B$ by $A \subseteq B$, so $x \in A \cap B$,
  so $A \cap B \supseteq A$.
  \item[(b)]
  \emph{$(\Longleftarrow)$}
  $A = A \cap B \subseteq B$.
  \end{enumerate}
\item[(2)]
\emph{$A \subseteq B \Leftrightarrow A \cup B = B$.}
  \begin{enumerate}
  \item[(a)]
  \emph{$(\Longrightarrow)$}
  It suffices to show $A \cup B \subseteq B$.
  For any $x \in A \cup B$, $x \in A$ or $x \in B$.
  By $A \subseteq B$, $x \in B$ or $x \in B$. $x \in B$,
  so $A \cup B \subseteq B$.
  \item[(b)]
  \emph{$(\Longleftarrow)$}
  $A \subseteq A \cup B = B$.
  \end{enumerate}
\end{enumerate}
$\Box$ \\\\



%%%%%%%%%%%%%%%%%%%%%%%%%%%%%%%%%%%%%%%%%%%%%%%%%%%%%%%%%%%%%%%%%%%%%%%%%%%%%%%%



\textbf{Problem 1.11.}
\emph{Show that
$A \subseteq B \Leftrightarrow \widetilde{B} \subseteq \widetilde{A}$.} \\

\emph{Proof.}
\begin{align*}
A \subseteq B
& \Longleftrightarrow
x \in A \Rightarrow x \in B \\
& \Longleftrightarrow
x \not\in B \Rightarrow x \not\in A \\
& \Longleftrightarrow
\widetilde{B} \subseteq \widetilde{A}.
\end{align*}
$\Box$ \\\\



%%%%%%%%%%%%%%%%%%%%%%%%%%%%%%%%%%%%%%%%%%%%%%%%%%%%%%%%%%%%%%%%%%%%%%%%%%%%%%%%



\textbf{Problem 1.14.}
\emph{Show that
$$B \cap \left[ \bigcup_{A \in \mathscr{C}} A \right]
= \bigcup_{A \in \mathscr{C}}(B \cap A).$$} \\

\emph{Proof.}
\begin{align*}
x \in B \cap \left[ \bigcup_{A \in \mathscr{C}} A \right]
& \Longleftrightarrow
x \in B \text{ and } x \in \bigcup_{A \in \mathscr{C}} A \\
& \Longleftrightarrow
x \in B \text{ and } x \in A \text{ for some } A \in \mathscr{C} \\
& \Longleftrightarrow
x \in B \cap A \text{ for some } A \in \mathscr{C} \\
& \Longleftrightarrow
x \in \bigcup_{A \in \mathscr{C}}(B \cap A).
\end{align*}
$\Box$ \\\\



%%%%%%%%%%%%%%%%%%%%%%%%%%%%%%%%%%%%%%%%%%%%%%%%%%%%%%%%%%%%%%%%%%%%%%%%%%%%%%%%
%%%%%%%%%%%%%%%%%%%%%%%%%%%%%%%%%%%%%%%%%%%%%%%%%%%%%%%%%%%%%%%%%%%%%%%%%%%%%%%%



\end{document}