\documentclass{article}
\usepackage{amsfonts}
\usepackage{amsmath}
\usepackage{amssymb}
\usepackage{hyperref}
\usepackage{mathrsfs}
\parindent=0pt

\def\upint{\mathchoice%
    {\mkern13mu\overline{\vphantom{\intop}\mkern7mu}\mkern-20mu}%
    {\mkern7mu\overline{\vphantom{\intop}\mkern7mu}\mkern-14mu}%
    {\mkern7mu\overline{\vphantom{\intop}\mkern7mu}\mkern-14mu}%
    {\mkern7mu\overline{\vphantom{\intop}\mkern7mu}\mkern-14mu}%
  \int}
\def\lowint{\mkern3mu\underline{\vphantom{\intop}\mkern7mu}\mkern-10mu\int}

\begin{document}



\textbf{\Large Chapter 3: Lebesgue Measure} \\\\



\emph{Author: Meng-Gen Tsai} \\
\emph{Email: plover@gmail.com} \\\\



%%%%%%%%%%%%%%%%%%%%%%%%%%%%%%%%%%%%%%%%%%%%%%%%%%%%%%%%%%%%%%%%%%%%%%%%%%%%%%%%



\textbf{\large Section 3.1: Introduction} \\\\



\textbf{Problem 3.1.}
\emph{If $A$ and $B$ are two sets in $\mathfrak{M}$ with
$A \subseteq B$, then $mA \leq mB$.
This property is called monotonicity.} \\

\emph{Proof.}
Write
$$B
= B \cap X
= B \cap (A \cup \widetilde{A})
= (B \cap A) \cup (B \cap \widetilde{A})
= A \cup (B - A).$$
Here $B \cap A = A$ comes from $A \subseteq B$ (Problem 1.9).
Notice that $A$ and $B - A$ are disjoint.
Since $m$ is a countably additive measure ($m$ is nonnegative)
on a $\sigma$-algebra $\mathfrak{M}$,
$$mB = mA + m(B-A) \geq mA.$$
$\Box$ \\\\



\textbf{Problem 3.2.}
\emph{Let $\langle E_n \rangle$ be any sequence of sets in $\mathfrak{M}$.
Then $m(\bigcup E_n) \leq \sum mE_n$. (Hint: Use Proposition 1.2)
This property of a measure is called countable subadditivity.} \\

As the argument in Problem 3.1. \\

\emph{Proof.}
Since $\langle E_n \rangle$ is a sequence of sets in $\sigma$-algebra $\mathfrak{M}$,
by Proposition 1.2 and its proof,
there is a sequence $\langle F_n \rangle$ of sets in $\sigma$-algebra $\mathfrak{M}$
such that all $F_n$ are pairwise disjoint, $F_n \subseteq E_n$, and
$$\bigcup E_n = \bigcup F_n.$$
Since $m$ is a countably additive measure
on a $\sigma$-algebra $\mathfrak{M}$,
$$m\left( \bigcup E_n \right)
= m\left( \bigcup F_n \right)
= \sum mF_n
\geq \sum mE_n.$$
The last inequality holds by applying Problem 3.1 on $F_n \subseteq E_n$ for any $n$.
$\Box$ \\\\



\textbf{Problem 3.3.}
\emph{If there is a set $A$ in $\mathfrak{M}$ such that $mA < \infty$,
then $m\varnothing = 0$.} \\

\emph{Proof.}
For such $A$, write $A = A \cup \varnothing$.
$A$ and $\varnothing$ are disjoint.
Since $m$ is a countably additive measure
on a $\sigma$-algebra $\mathfrak{M}$,
$$mA = mA + m\varnothing.$$
Since $mA < \infty$,
we can cancel out $mA$ on the both sides to get
$m\varnothing = 0$.
$\Box$ \\\\



%%%%%%%%%%%%%%%%%%%%%%%%%%%%%%%%%%%%%%%%%%%%%%%%%%%%%%%%%%%%%%%%%%%%%%%%%%%%%%%%
%%%%%%%%%%%%%%%%%%%%%%%%%%%%%%%%%%%%%%%%%%%%%%%%%%%%%%%%%%%%%%%%%%%%%%%%%%%%%%%%



\textbf{\large Section 3.2: Outer Measure} \\\\



\textbf{Problem 3.5.}
\emph{Let $A$ be the set of rational numbers between $0$ and $1$, and
let $\{ I_n\}$ be a finite collection of open intervals covering $A$.
Then $\sum l(I_n) \geq 1$.} \\

\emph{Idea.}
If $\{ I_n\}$ is a covering of $[0, 1]$ then we are done
since the length of $[0, 1]$ is $1$.
However, $\{ I_n\}$ only covers $A$ and not necessarily covers $[0, 1]$.
(For example,
$\{ I_n\}
= \left\{
\left( -89, \frac{1}{\sqrt{2}} \right),
\left( \frac{1}{\sqrt{2}}, 64 \right)
\right\}$ covers $A$ but not $\frac{1}{\sqrt{2}}$.)
Hence, it is natural to consider the closure of $A$ and
the closure of $I_n$.
Now $\{ \overline{I_n} \}$ is a (closed) covering of
$\overline{A} = [0, 1]$. \\

\emph{Proof.}
\begin{align*}
1
&= m^{*}[0, 1]
  &\text{(Proposition 3.1)} \\
&= m^{*}\overline{A}
  &\text{($A$ is dense in $[0, 1]$)} \\
&\leq m^{*}\left( \overline{\bigcup I_n} \right)
  &\text{(Proposition 2.10)} \\
&= m^{*}\left( \bigcup \overline{I_n} \right)
  &\text{(Proposition 2.10)} \\
&\leq \sum m^{*}(\overline{I_n})
  &\text{(Proposition 3.2)} \\
&= \sum l(\overline{I_n})
  &\text{(Proposition 3.1)} \\
&= \sum l(I_n).
  &\text{(definition of length)}
\end{align*}
$\Box$ \\

\textbf{Supplement.}
Exercise about considering the closure.
(Exercise 4.52 in T. M. Apostol, Mathematical Analysis, 2nd Edition.)
\emph{Assume that $f$ is uniformly continuous on a bounded set $S$ in $\mathbb{R}^n$.
Prove that $f$ must be bounded on $S$.} \\

\emph{Proof.}
\begin{enumerate}
\item[(1)]
Since $f: S \rightarrow T$ is uniformly continuous,
given any $\epsilon > 0$, there is $\delta > 0$ such that
$d_T(f(x), f(y)) < \epsilon$ whenever $d_S(x, y) < \delta$.
Choose $\epsilon = 1 > 0$.
\item[(2)]
For such $\delta > 0$, construct an open covering of $\overline{S} \subseteq \mathbb{R}^n$.
Pick a collection $\mathscr{F}$ of open balls
$B(a;\delta) \subseteq \mathbb{R}^n$
where $a$ runs over all elements of $S$.
$\mathscr{F}$ covers $\overline{S}$ (by the definition of accumulation points).
Since $\overline{S} $ is closed and bounded (since $S$ is bounded),
$\overline{S}$ is compact
So there is a finite subcollection $\mathscr{F}'$ of $\mathscr{F}$
also covers $\overline{S}$, say
$$\mathscr{F}'
= \left\{B(a_1;\delta)), B(a_2;\delta), ..., B(a_m;\delta) \right\}.$$
\item[(3)]
Given any $x \in S \subseteq \overline{S}$,
there is some $a_i \in S$ $(1 \leq i \leq m)$ such that $x \in B(a_i;\delta)$.
In such ball, $d_S(x, a_i) < \delta$.
By (1), $\Vert f(x) - f(a_i) \Vert < 1$,
or $\Vert f(x) \Vert < 1 + \Vert f(a_i) \Vert$.
Therefore, for any $x \in S$,
$$\Vert f(x) \Vert < 1 + \max_{1 \leq i \leq m}{\Vert f(a_i) \Vert}.$$
\end{enumerate}
$\Box$ \\



\textbf{Problem 3.6.}
\emph{Prove Proposition 5:
Given any set $A$ and any $\epsilon > 0$,
there is an open set $O$ such that $A \subseteq O$
and $m^{*}O \leq m^{*}A + \epsilon$.
There is a $G \in G_{\delta}$ such that $m^{*}G = m^{*}A$.} \\

\emph{Proof.}
\begin{enumerate}
\item[(1)]
\emph{Use the definition of the outer measure.}
By the definition of $m^{*}$,
for such $\epsilon > 0$ there exists a countable collection
$\{ I_n \}$ of open intervals that covers $A$ and
$$m^{*} A + \epsilon \geq \sum l(I_n).$$
\item[(2)]
\emph{Construct an open set $O$.}
Let $O = \bigcup I_n \supseteq A$
which is the union of any collection of open sets $I_n$.
By Proposition 2.7, $O$ is open.
\item[(3)]
\emph{Show that $m^{*}O \leq m^{*}A + \epsilon$.}
By Proposition 3.2 and 3.1,
$$m^{*}O
= m^{*} \left( \bigcup I_n \right)
\leq \sum m^{*} I_n
= \sum l(I_n)
\leq m^{*} A + \epsilon.$$
\end{enumerate}
Therefore, given any set $A$ and any $\epsilon > 0$,
there is an open set $O$ such that $A \subseteq O$
and $m^{*}O \leq m^{*}A + \epsilon$.

\begin{enumerate}
\item[(4)]
\emph{Construct $G \in G_{\delta}$ in a natural way.}
Given any $n \in \mathbb{N}$, there exists an open set $O_n$
such that $O_n \supseteq A$ and $m^{*}O_n \leq m^{*}A + \frac{1}{n}$.
Let
$$G = \bigcap_{n=1}^{\infty} O_n \in G_{\delta}.$$
\item[(5)]
\emph{Show that $m^{*}G = m^{*}A$.}
  \begin{enumerate}
  \item[(a)]
  Since $A \subseteq O_n$ for any $n \in \mathbb{N}$,
  $A \subseteq \bigcap_{n=1}^{\infty} O_n = G$.
  Thus $m^{*}A \leq m^{*}G$.
  \item[(b)]
  Since $O_n \supseteq \bigcap_{n=1}^{\infty} O_n = G$ for any $n \in \mathbb{N}$,
  $$m^{*}A + \frac{1}{n} \geq m^{*}O_n \geq m^{*}G$$
  for any $n \in \mathbb{N}$.
  Since $n \in \mathbb{N}$ is arbitrary, $m^{*}A \geq m^{*}G$.
  \end{enumerate}
By (a)(b), $m^{*}A = m^{*}G$.
\end{enumerate}
$\Box$ \\\\



\textbf{Problem 3.7.}
\emph{Prove that $m^{*}$ is translation invariant.} \\

\emph{Proof.}
Given $E \in \mathfrak{M}$ and $y \in \mathbb{R}$.
\begin{enumerate}
\item[(1)]
$m^{*}(E + y) \leq m^{*}E$.
Let $\{ I_n \}$ of open intervals that cover $E$.
Then $\{ I_n+y \}$ of open intervals that cover $E+y$.
Notice that the definition of $m^{*}$ and $l(I_n+y) = l(I_n)$, then
$$m^{*}(E + y) \leq \sum l(I_n+y) = \sum l(I_n).$$
Take the infimum of all such sum $\sum l(I_n)$,
$m^{*}(E + y) \leq m^{*}E$.
\item[(2)]
$m^{*}(E) \leq m^{*}(E + y)$.
Similar to (1).
\end{enumerate}
By (1)(2), $m^{*}(E + y) = m^{*}E$, that is, $m^{*}$ is translation invariant.
$\Box$ \\\\



\textbf{Problem 3.8.}
\emph{Prove that if $m^{*}A = 0$, then $m^{*}(A \cup B) = m^{*}B$.} \\

\emph{Proof.}
\begin{enumerate}
\item[(1)]
$m^{*}(A \cup B) \geq m^{*}B$ since $A \cup B \supseteq B$
and the definition of $m^{*}$.
(Any covering of $A \cup B$ by open intervals is also a covering of $B$
so that the latter infimum is taken over a larger collection than the former.)
\item[(2)]
$m^{*}(A \cup B) \leq m^{*}B$.
By Proposition 3.2,
$$m^{*}(A \cup B) \leq m^{*}A + m^{*}B = 0 + m^{*}B = m^{*}B.$$
\end{enumerate}
By (1)(2), $m^{*}(A \cup B) = m^{*}B$.
$\Box$ \\\\



% No exercises left.

%%%%%%%%%%%%%%%%%%%%%%%%%%%%%%%%%%%%%%%%%%%%%%%%%%%%%%%%%%%%%%%%%%%%%%%%%%%%%%%%
%%%%%%%%%%%%%%%%%%%%%%%%%%%%%%%%%%%%%%%%%%%%%%%%%%%%%%%%%%%%%%%%%%%%%%%%%%%%%%%%



\textbf{\large Section 3.3: Measurable Sets and Lebesgue Measure} \\\\



\textbf{Problem 3.9.}
\emph{Show that if $E$ is a measurable set, then each translate $E+y$ of $E$
is also measurable.} \\

\emph{Proof.}
\begin{enumerate}
\item[(1)]
\emph{$E$ is measurable if and only if
for each set $A$, each $y \in \mathbb{R}$,
$$m^{*}(A+y)
= m^{*}((A+y) \cap E) + m^{*}((A+y) \cap \widetilde{E}).$$}
  \begin{enumerate}
  \item[(a)]
  $(\Longrightarrow)$
  $E$ is measurable and
  $A+y$ is a set (for any set $A$ and $y \in \mathbb{R}$).
  \item[(b)]
  $(\Longleftarrow)$
  $A = (A-y) + y$ for any set $A$ and $y \in \mathbb{R}$.
  \end{enumerate}
\item[(2)]
For any set $E$ and $y \in \mathbb{R}$,
$\widetilde{E+y} = \widetilde{E}+y$ by the definition of translation.
\item[(3)]
For any sets $E_1$, $E_2$ and $y \in \mathbb{R}$,
$(E_1 \cap E_2)+y = (E_1+y) + (E_2+y)$ by the definition of translation.
\item[(4)]
For each set $A$ and $y \in \mathbb{R}$,
\begin{align*}
 & m^{*}((A+y) \cap (E+y)) + m^{*}((A+y) \cap (\widetilde{E+y}))
  & \\
=& m^{*}((A+y) \cap (E+y)) + m^{*}((A+y) \cap (\widetilde{E}+y))
  &\text{((2))} \\
=& m^{*}((A \cap E)+y) + m^{*}((A \cap \widetilde{E})+y)
  &\text{((3))} \\
=& m^{*}(A \cap E) + m^{*}(A \cap \widetilde{E})
  &\text{(Problem 3.7)} \\
=& m^{*}A
  &\text{(Measurability of $E$)} \\
=& m^{*}(A+y).
  &\text{(Problem 3.7)}
\end{align*}
By (1), $E+y$ is measurable.
\end{enumerate}
$\Box$ \\\\



\textbf{Problem 3.10.}
\emph{Show that if $E_1$ and $E_2$ are measurable, then
$$m(E_1 \cup E_2) + m(E_1 \cap E_2) = mE_1 + mE_2.$$}

\emph{Proof.}
Since the collection $\mathfrak{M}$ of measurable sets is a $\sigma$-algebra
(Theorem 3.10) and $m$ is countable additive (Proposition 3.13),
\begin{align*}
m(E_1 \cup E_2) + m(E_1 \cap E_2)
& = \left( m(E_1) + m(E_2 \cap \widetilde{E_1}) \right) + m(E_2 \cap E_1) \\
& = m(E_1) + \left( m(E_2 \cap \widetilde{E_1}) + m(E_2 \cap E_1) \right) \\
& = m(E_1) + m(E_2).
\end{align*}
($E_1$ and $E_2 \cap \widetilde{E_1}$ are disjoint.
$E_2 \cap \widetilde{E_1}$ and $E_2 \cap E_1$ are disjoint too.)
$\Box$ \\\\



\textbf{Problem 3.11.}
\emph{Show that the condition $mE_1 < \infty$ is necessary in Proposition 3.14 by
giving a decreasing sequence $\langle E_n \rangle$ of measurable sets with
$\varnothing = \bigcap E_n$ and $mE_n = \infty$ for each $n$.} \\

\emph{Proof.}
Set $$E_n = (n, \infty)$$ for each $n \in \mathbb{N}$.
\begin{enumerate}
\item[(1)]
\emph{$\langle E_n \rangle$ is a decreasing sequence of measurable sets.}
$E_n \supseteq E_{n+1}$ by definition.
Besides, each $E_n$ is measurable by Lemma 3.11.
\item[(2)]
\emph{$\bigcap E_n = \varnothing$.}
For each $x \in \mathbb{R}$, $x \notin E_1$ if $x \leq 1$;
$x \notin E_{[x]}$ if $x \geq 1$ where $x \mapsto [x]$ is the floor function.
\item[(3)]
\emph{$mE_n = \infty$ for each $n$.}
The length of each $E_n$ is $\infty$ (Proposition 3.1).
\end{enumerate}
$\Box$ \\\\



\textbf{Problem 3.12.}
\emph{Let $\langle E_n \rangle$ be a sequence of disjoint measurable sets and $A$ any set.
Then
$m^{*}\left( A \cap \bigcup_{i=1}^{\infty}E_i \right)
= \sum_{i=1}^{\infty} m^{*}(A \cap E_i)$.} \\

\emph{Proof.}
\begin{enumerate}
\item[(1)]
$A \cap \bigcup_{i=1}^{\infty}E_i
= \bigcup_{i=1}^{\infty}(A \cap E_i)$ (Problem 1.14).
\item[(2)]
$m^{*}\left( \bigcup_{i=1}^{\infty}(A \cap E_i) \right)
\leq \sum_{i=1}^{\infty} m^{*}(A \cap E_i)$
by the subadditivity of $m^{*}$ (Proposition 3.2).
\item[(3)]
By Lemma 3.9,
$$m^{*}\left( \bigcup_{i=1}^{n}(A \cap E_i) \right)
= \sum_{i=1}^{n} m^{*}(A \cap E_i)$$
for any $n \in \mathbb{N}$.
Since
$\bigcup_{i=1}^{\infty}(A \cap E_i) \supseteq \bigcup_{i=1}^{n}(A \cap E_i)$,
$m^{*}\left( \bigcup_{i=1}^{\infty}(A \cap E_i) \right)
\geq m^{*}\left( \bigcup_{i=1}^{n}(A \cap E_i) \right)$ by the monotonicity of $m^{*}$.
Thus,
$$
m^{*}\left( \bigcup_{i=1}^{\infty}(A \cap E_i) \right)
\geq
\sum_{i=1}^{n} m^{*}(A \cap E_i)$$
for any $n \in \mathbb{N}$.
Since $\sum_{i=1}^{n} m^{*}(A \cap E_i)$ is bounded and increasing
(by the non-negativity of $m^{*}$),
$$
m^{*}\left( \bigcup_{i=1}^{\infty}(A \cap E_i) \right)
\geq
\sum_{i=1}^{\infty} m^{*}(A \cap E_i).$$
\end{enumerate}
By (2)(3),
$m^{*}\left( A \cap \bigcup_{i=1}^{\infty}E_i \right)
= \sum_{i=1}^{\infty} m^{*}(A \cap E_i)$.
$\Box$ \\\\



%%%%%%%%%%%%%%%%%%%%%%%%%%%%%%%%%%%%%%%%%%%%%%%%%%%%%%%%%%%%%%%%%%%%%%%%%%%%%%%%
%%%%%%%%%%%%%%%%%%%%%%%%%%%%%%%%%%%%%%%%%%%%%%%%%%%%%%%%%%%%%%%%%%%%%%%%%%%%%%%%



\textbf{\large Section 3.4: A Nonmeasurable Set} \\\\



%%%%%%%%%%%%%%%%%%%%%%%%%%%%%%%%%%%%%%%%%%%%%%%%%%%%%%%%%%%%%%%%%%%%%%%%%%%%%%%%
%%%%%%%%%%%%%%%%%%%%%%%%%%%%%%%%%%%%%%%%%%%%%%%%%%%%%%%%%%%%%%%%%%%%%%%%%%%%%%%%



\textbf{\large Section 3.5: Measurable Functions} \\\\



%%%%%%%%%%%%%%%%%%%%%%%%%%%%%%%%%%%%%%%%%%%%%%%%%%%%%%%%%%%%%%%%%%%%%%%%%%%%%%%%
%%%%%%%%%%%%%%%%%%%%%%%%%%%%%%%%%%%%%%%%%%%%%%%%%%%%%%%%%%%%%%%%%%%%%%%%%%%%%%%%



\textbf{\large Section 3.6: Littlewood's Three Principles} \\\\



%%%%%%%%%%%%%%%%%%%%%%%%%%%%%%%%%%%%%%%%%%%%%%%%%%%%%%%%%%%%%%%%%%%%%%%%%%%%%%%%



\end{document}