\documentclass{article}
\usepackage{amsfonts}
\usepackage{amsmath}
\usepackage{amssymb}
\usepackage{centernot}
\usepackage{hyperref}
\usepackage[none]{hyphenat}
\usepackage{mathrsfs}
\usepackage{mathtools}
\usepackage{physics}
\usepackage{tikz-cd}
\parindent=0pt



\title{\textbf{Solutions to the book: \\ \emph{H. L. Royden, Real Analysis, 3rd edition}}}
\author{Meng-Gen Tsai \\ plover@gmail.com}



\begin{document}
\maketitle
\tableofcontents



%%%%%%%%%%%%%%%%%%%%%%%%%%%%%%%%%%%%%%%%%%%%%%%%%%%%%%%%%%%%%%%%%%%%%%%%%%%%%%%%
%%%%%%%%%%%%%%%%%%%%%%%%%%%%%%%%%%%%%%%%%%%%%%%%%%%%%%%%%%%%%%%%%%%%%%%%%%%%%%%%



% Reference:
% https://issuu.com/inttegrattor/docs/real_analysis_-_h.l._royden_-_solutions



%%%%%%%%%%%%%%%%%%%%%%%%%%%%%%%%%%%%%%%%%%%%%%%%%%%%%%%%%%%%%%%%%%%%%%%%%%%%%%%%
%%%%%%%%%%%%%%%%%%%%%%%%%%%%%%%%%%%%%%%%%%%%%%%%%%%%%%%%%%%%%%%%%%%%%%%%%%%%%%%%



\newpage
\section*{Chapter 1: Set Theory \\}
\addcontentsline{toc}{section}{Chapter 1: Set Theory}



\subsection*{1.1: Introduction \\}
\addcontentsline{toc}{subsection}{1.1: Introduction}



\subsubsection*{Problem 1.1.}
\addcontentsline{toc}{subsubsection}{Problem 1.1.}
\emph{Show that $\{ x : x \neq x \} = \varnothing$.} \\

\emph{Proof.}
Every element $x$ of $\{ x : x \neq x \}$ satisfying $x \neq x$,
contrary to $x = x$. That is, there are no elements in $\{ x : x \neq x \}$,
or $\{ x : x \neq x \} = \varnothing$.
$\Box$ \\\\



%%%%%%%%%%%%%%%%%%%%%%%%%%%%%%%%%%%%%%%%%%%%%%%%%%%%%%%%%%%%%%%%%%%%%%%%%%%%%%%%



\subsubsection*{Problem 1.2.}
\addcontentsline{toc}{subsubsection}{Problem 1.2.}
\emph{Show that if $x \in \varnothing$, then $x$ is a green-eyed lion.} \\

\emph{Proof.}
$\varnothing \subseteq \{ \text{a green-eyed lion} \}$.
$\Box$ \\\\



%%%%%%%%%%%%%%%%%%%%%%%%%%%%%%%%%%%%%%%%%%%%%%%%%%%%%%%%%%%%%%%%%%%%%%%%%%%%%%%%



\subsubsection*{Problem 1.3.}
\addcontentsline{toc}{subsubsection}{Problem 1.3.}
\emph{Show that in general the sets $X \times (Y \times Z)$ and
$(X \times Y) \times Z$ are different but
that there is a natural correspondence between each of them and $X \times Y \times Z$.} \\



\emph{Proof.}
\begin{enumerate}
\item[(1)]
  \begin{align*}
    X \times (Y \times Z)
    &= \{ \langle x, \langle y, z \rangle \rangle : x \in X, y \in Y, z \in Z \} \\
    (X \times Y) \times Z
    &= \{ \langle \langle x, y \rangle, z \rangle : x \in X, y \in Y, z \in Z \} \\
    X \times Y \times Z
    &= \{ \langle x, y, z \rangle : x \in X, y \in Y, z \in Z \}
  \end{align*}
  are all not the same.

\item[(2)]
  There is a natural correspondence
  \[
    X \times (Y \times Z)
    \longleftrightarrow (X \times Y) \times Z
    \longleftrightarrow X \times Y \times Z
  \]
  defined by
  \[
    \langle x, \langle y, z \rangle \rangle
    \longleftrightarrow \langle \langle x, y \rangle, z \rangle
    \longleftrightarrow \langle x, y, z \rangle
  \]
  for every $x \in X, y \in Y, z \in Z$.
\end{enumerate}
$\Box$ \\\\



%%%%%%%%%%%%%%%%%%%%%%%%%%%%%%%%%%%%%%%%%%%%%%%%%%%%%%%%%%%%%%%%%%%%%%%%%%%%%%%%



\subsubsection*{Problem 1.4.}
\addcontentsline{toc}{subsubsection}{Problem 1.4.}
\emph{Show that the well-ordering principle implies
the principle of mathematical induction.
(Hint: Consider the set $\{ n \in \mathbb{N} : P(n) \text{ is false} \}$.)} \\

\emph{Proof (Hint).}
Suppose that
\begin{enumerate}
\item[(1)]
$P(n)$ be a proposition defined for each $n \in \mathbb{N}$,
\item[(2)]
$P(1)$ is true,
\item[(3)]
$[P(n) \Rightarrow P(n+1)]$ is true.
\end{enumerate}

Consider the set
$$S = \{ n \in \mathbb{N} : P(n) \text{ is false} \} \subseteq \mathbb{N}.$$
Want to show
\emph{$S$ is empty, or the principle of mathematical induction holds.}
If $S$ were nonempty,
by the well-ordering principle $S$ has a smallest element $m$.
$m$ cannot be $1$ by (2).
Say $m > 1$.
Therefore, $m - 1 \in \mathbb{N}$
and $P(m-1)$ is true by the minimality of $m$.
By (3), $P((m-1)+1) = P(m)$ is true, which is absurd.
$\Box$ \\\\



%%%%%%%%%%%%%%%%%%%%%%%%%%%%%%%%%%%%%%%%%%%%%%%%%%%%%%%%%%%%%%%%%%%%%%%%%%%%%%%%



\subsubsection*{Problem 1.5.}
\addcontentsline{toc}{subsubsection}{Problem 1.5.}
\emph{Use mathematical induction to establish
that the well-ordering principle.
(Hint: Given a set $S$ of positive integers,
let $P(n)$ be the proposition
`If $n \in S$, then $S$ has a least element'.)} \\

\emph{Proof (Modified hint).}
\begin{enumerate}
\item[(1)]
Given a set $S$ of positive integers,
let $P(n)$ be the proposition
`If $m \in S$ for some $m \leq n$, then $S$ has a least element'.
Want to show $P(n)$ is true for all $n \in \mathbb{N}$.
\begin{enumerate}
\item[(a)]
$P(1)$ is true.
For $m \in S$ with $m \leq n = 1$,
or $m = 1$ by the minimality of $1 \in \mathbb{N}$,
$S$ has a least element $1$ ($m$ itself) in $\mathbb{N}$.
\item[(b)]
Suppose $P(n)$ is true.
If $n+1 \in S$, then there are only two possible cases.
  \begin{enumerate}
  \item[(i)]
  There is a positive integer $m \in S$ less than $n+1$.
  So $n \geq m \in S$.
  Since $P(n)$ is true, $S$ has a least element.
  \item[(ii)]
  There is no positive integer $m \in S$ less than $n+1$.
  In this case $n+1$ is the least element in $S$.
  \end{enumerate}
  In any cases (i)(ii), $S$ has a least element, or $P(n+1)$ is true.
\end{enumerate}
By mathematical induction, $P(n)$ is true for all $n \in \mathbb{N}$.
\item[(2)]
\emph{Show that the well-ordering principle holds.}
Let $T$ be a nonempty subset of $\mathbb{N}$,
so there exists a positive integer $k \in T$.
Notice that $P(k)$ is true by (1),
thus $T$ has a least element since $k \leq k$.
\end{enumerate}
$\Box$ \\\\



%%%%%%%%%%%%%%%%%%%%%%%%%%%%%%%%%%%%%%%%%%%%%%%%%%%%%%%%%%%%%%%%%%%%%%%%%%%%%%%%



\subsection*{1.3: Unions, Intersections, and Complements \\}
\addcontentsline{toc}{subsection}{1.3: Unions, Intersections, and Complements}



\subsubsection*{Problem 1.9.}
\addcontentsline{toc}{subsubsection}{Problem 1.9.}
\emph{Show that
$A \subseteq B
\Leftrightarrow A \cap B = A
\Leftrightarrow A \cup B = B$.} \\

\emph{Proof.}
\begin{enumerate}
\item[(1)]
\emph{$A \subseteq B \Longleftrightarrow A \cap B = A$.}
  \begin{enumerate}
  \item[(a)]
  \emph{$(\Longrightarrow)$}
  It suffices to show $A \cap B \supseteq A$.
  For any $x \in A$, $x \in B$ by $A \subseteq B$, so $x \in A \cap B$,
  so $A \cap B \supseteq A$.
  \item[(b)]
  \emph{$(\Longleftarrow)$}
  $A = A \cap B \subseteq B$.
  \end{enumerate}
\item[(2)]
\emph{$A \subseteq B \Leftrightarrow A \cup B = B$.}
  \begin{enumerate}
  \item[(a)]
  \emph{$(\Longrightarrow)$}
  It suffices to show $A \cup B \subseteq B$.
  For any $x \in A \cup B$, $x \in A$ or $x \in B$.
  By $A \subseteq B$, $x \in B$ or $x \in B$. $x \in B$,
  so $A \cup B \subseteq B$.
  \item[(b)]
  \emph{$(\Longleftarrow)$}
  $A \subseteq A \cup B = B$.
  \end{enumerate}
\end{enumerate}
$\Box$ \\\\



%%%%%%%%%%%%%%%%%%%%%%%%%%%%%%%%%%%%%%%%%%%%%%%%%%%%%%%%%%%%%%%%%%%%%%%%%%%%%%%%



\subsubsection*{Problem 1.11.}
\addcontentsline{toc}{subsubsection}{Problem 1.11.}
\emph{Show that
$A \subseteq B \Leftrightarrow \widetilde{B} \subseteq \widetilde{A}$.} \\

\emph{Proof.}
\begin{align*}
A \subseteq B
& \Longleftrightarrow
x \in A \Rightarrow x \in B \\
& \Longleftrightarrow
x \not\in B \Rightarrow x \not\in A \\
& \Longleftrightarrow
\widetilde{B} \subseteq \widetilde{A}.
\end{align*}
$\Box$ \\\\



%%%%%%%%%%%%%%%%%%%%%%%%%%%%%%%%%%%%%%%%%%%%%%%%%%%%%%%%%%%%%%%%%%%%%%%%%%%%%%%%



\subsubsection*{Problem 1.14.}
\addcontentsline{toc}{subsubsection}{Problem 1.14.}
\emph{Show that
$$B \cap \left[ \bigcup_{A \in \mathscr{C}} A \right]
= \bigcup_{A \in \mathscr{C}}(B \cap A).$$} \\

\emph{Proof.}
\begin{align*}
x \in B \cap \left[ \bigcup_{A \in \mathscr{C}} A \right]
& \Longleftrightarrow
x \in B \text{ and } x \in \bigcup_{A \in \mathscr{C}} A \\
& \Longleftrightarrow
x \in B \text{ and } x \in A \text{ for some } A \in \mathscr{C} \\
& \Longleftrightarrow
x \in B \cap A \text{ for some } A \in \mathscr{C} \\
& \Longleftrightarrow
x \in \bigcup_{A \in \mathscr{C}}(B \cap A).
\end{align*}
$\Box$ \\\\



%%%%%%%%%%%%%%%%%%%%%%%%%%%%%%%%%%%%%%%%%%%%%%%%%%%%%%%%%%%%%%%%%%%%%%%%%%%%%%%%
%%%%%%%%%%%%%%%%%%%%%%%%%%%%%%%%%%%%%%%%%%%%%%%%%%%%%%%%%%%%%%%%%%%%%%%%%%%%%%%%
%%%%%%%%%%%%%%%%%%%%%%%%%%%%%%%%%%%%%%%%%%%%%%%%%%%%%%%%%%%%%%%%%%%%%%%%%%%%%%%%
%%%%%%%%%%%%%%%%%%%%%%%%%%%%%%%%%%%%%%%%%%%%%%%%%%%%%%%%%%%%%%%%%%%%%%%%%%%%%%%%



\newpage
\section*{Chapter 2: The Real Number System \\}
\addcontentsline{toc}{section}{Chapter 2: The Real Number System}



\subsection*{2.1: Axioms for the Real Numbers \\}
\addcontentsline{toc}{subsection}{2.1: Axioms for the Real Numbers}



\subsubsection*{Problem 2.1.}
\addcontentsline{toc}{subsubsection}{Problem 2.1.}
\emph{Show that $1 \in P$.} \\

\emph{Proof.}
By the field axioms,
\begin{enumerate}
\item[(a)]
$1 \in \mathbb{R}$ such that $1 \neq 0$.
\item[(b)]
$-1 \in \mathbb{R}$.
\item[(c)]
$(-1) \cdot (-1) = 1$.
\end{enumerate}

By the axioms of order,
$1 = 0$ or $1 \in P$ or $-1 \in P$.
Consider three possible cases,
\begin{enumerate}
\item[(1)]
$1 = 0$, contrary to the field axioms $1 \neq 0$.
\item[(2)]
$1 \in P$.
\item[(3)]
$-1 \in P$.
By the axioms of order, $(-1)(-1) \in P$.
Since $(-1)(-1) = 1$ by the field axioms, $1 \in P$.
By the axioms of order, $-1 \not\in P$, contrary to $-1 \in P$.
\end{enumerate}
By (1)(2)(3), $1 \in P$.
$\Box$ \\

Applying the similar argument to $\sqrt{-1}$,
we get $\sqrt{-1} \not\in \mathbb{R}$
as our expectation. \\\\



%%%%%%%%%%%%%%%%%%%%%%%%%%%%%%%%%%%%%%%%%%%%%%%%%%%%%%%%%%%%%%%%%%%%%%%%%%%%%%%%
%%%%%%%%%%%%%%%%%%%%%%%%%%%%%%%%%%%%%%%%%%%%%%%%%%%%%%%%%%%%%%%%%%%%%%%%%%%%%%%%
%%%%%%%%%%%%%%%%%%%%%%%%%%%%%%%%%%%%%%%%%%%%%%%%%%%%%%%%%%%%%%%%%%%%%%%%%%%%%%%%
%%%%%%%%%%%%%%%%%%%%%%%%%%%%%%%%%%%%%%%%%%%%%%%%%%%%%%%%%%%%%%%%%%%%%%%%%%%%%%%%



\newpage
\section*{Chapter 3: Lebesgue Measure \\}
\addcontentsline{toc}{section}{Chapter 3: Lebesgue Measure}



\subsection*{3.1: Introduction \\}
\addcontentsline{toc}{subsection}{3.1: Introduction}



\subsubsection*{Problem 3.1. (Monotonicity)}
\addcontentsline{toc}{subsubsection}{Problem 3.1. (Monotonicity)}
\emph{If $A$ and $B$ are two sets in $\mathfrak{M}$ with
$A \subseteq B$, then $mA \leq mB$.
This property is called monotonicity.} \\



\emph{Proof.}
Write
\[
  B
  = B \cap X
  = B \cap (A \cup \widetilde{A})
  = (B \cap A) \cup (B \cap \widetilde{A})
  = A \cup (B - A).
\]
Here $B \cap A = A$ comes from $A \subseteq B$ (Problem 1.9).
Notice that $A$ and $B - A$ are disjoint.
Since $m$ is a countably additive measure ($m$ is nonnegative)
on a $\sigma$-algebra $\mathfrak{M}$,
\[
  mB = mA + m(B-A) \geq mA.
\]
$\Box$ \\\\



%%%%%%%%%%%%%%%%%%%%%%%%%%%%%%%%%%%%%%%%%%%%%%%%%%%%%%%%%%%%%%%%%%%%%%%%%%%%%%%%



\subsubsection*{Problem 3.2. (Countable subadditivity)}
\addcontentsline{toc}{subsubsection}{Problem 3.2. (Countable subadditivity)}
\emph{Let $\langle E_n \rangle$ be any sequence of sets in $\mathfrak{M}$.
Then $m(\bigcup E_n) \leq \sum mE_n$. (Hint: Use Proposition 1.2)
This property of a measure is called countable subadditivity.} \\

As the argument in Problem 3.1. \\



\emph{Proof.}
Since $\langle E_n \rangle$ is a sequence of sets in $\sigma$-algebra $\mathfrak{M}$,
by Proposition 1.2 and its proof,
there is a sequence $\langle F_n \rangle$ of sets in $\sigma$-algebra $\mathfrak{M}$
such that all $F_n$ are pairwise disjoint, $F_n \subseteq E_n$, and
$$\bigcup E_n = \bigcup F_n.$$
Since $m$ is a countably additive measure
on a $\sigma$-algebra $\mathfrak{M}$,
\[
  m\left( \bigcup E_n \right)
  = m\left( \bigcup F_n \right)
  = \sum mF_n
  \geq \sum mE_n.
\]
The last inequality holds by applying Problem 3.1 on $F_n \subseteq E_n$ for any $n$.
$\Box$ \\\\



%%%%%%%%%%%%%%%%%%%%%%%%%%%%%%%%%%%%%%%%%%%%%%%%%%%%%%%%%%%%%%%%%%%%%%%%%%%%%%%%



\subsubsection*{Problem 3.3.}
\addcontentsline{toc}{subsubsection}{Problem 3.3.}
\emph{If there is a set $A$ in $\mathfrak{M}$ such that $mA < \infty$,
then $m\varnothing = 0$.} \\



\emph{Proof.}
For such $A$, write $A = A \cup \varnothing$.
$A$ and $\varnothing$ are disjoint.
Since $m$ is a countably additive measure
on a $\sigma$-algebra $\mathfrak{M}$,
\[
  mA = mA + m\varnothing.
\]
Since $mA < \infty$,
we can cancel out $mA$ on the both sides to get
$m\varnothing = 0$.
$\Box$ \\\\



%%%%%%%%%%%%%%%%%%%%%%%%%%%%%%%%%%%%%%%%%%%%%%%%%%%%%%%%%%%%%%%%%%%%%%%%%%%%%%%%



\subsubsection*{Problem 3.4. (Counting measure)}
\addcontentsline{toc}{subsubsection}{Problem 3.4. (Counting measure)}
\emph{Let $nE$ be $\infty$ for an infinite set $E$ and
be equal to the number of elements of $E$ for a finite set.
Show that $n$ is a countably additive set function that is translation invariant and
defined for all sets of real numbers.
This measure is called the \textbf{counting measure}.} \\



\emph{Proof.}
\begin{enumerate}
\item[(1)]
  \emph{Show that $n$ is a countably additive set function.}
  Note that $n$ is defined on any subset of real numbers
  since the finiteness is defined on any subset of real numbers.
  Suppose $\langle E_m \rangle$ is a sequence of disjoint sets of real numbers.
  We need to show that $n\left( \bigcup E_m \right) = \sum n E_m$.

\item[(2)]
  If $E_m$ is infinite for some $m = k$, then $\bigcup E_m$ is also infinite.
  Hence, $n \left( \bigcup E_m \right) = \infty$,
  and $\sum n E_m \geq n E_k = \infty \Longrightarrow \sum n E_m = \infty$.

\item[(3)]
  Suppose all $E_n$ are finite.
  Note that
  $\bigcup E_m$ is infinite if and only if all but finitely many $E_m \neq \varnothing$
  if and only if $\sum n E_m = \infty$.
  Besides, if $\bigcup E_m$ is finite, then all but finitely many $E_m = \varnothing$
  and thus
  \[
    n\left( \bigcup_m E_m \right)
    = \sum_{E_m \neq \varnothing} n E_m
    = \sum_{m} n E_m
    < \infty.
  \]

\item[(4)]
  Since
  \begin{align*}
    n(E+y)
    &= n(\{ x + y : x \in E\}) \\
    &= \text{the number of elements $x \in E$} \\
    &= n(E),
  \end{align*}
  $n$ is translation invariant.
\end{enumerate}
$\Box$ \\\\



%%%%%%%%%%%%%%%%%%%%%%%%%%%%%%%%%%%%%%%%%%%%%%%%%%%%%%%%%%%%%%%%%%%%%%%%%%%%%%%%
%%%%%%%%%%%%%%%%%%%%%%%%%%%%%%%%%%%%%%%%%%%%%%%%%%%%%%%%%%%%%%%%%%%%%%%%%%%%%%%%



\subsection*{3.2: Outer Measure \\}
\addcontentsline{toc}{subsection}{3.2: Outer Measure}



\subsubsection*{Problem 3.5.}
\addcontentsline{toc}{subsubsection}{Problem 3.5.}
\emph{Let $A$ be the set of rational numbers between $0$ and $1$, and
let $\{ I_n \}$ be a finite collection of open intervals covering $A$.
Then $\sum \ell(I_n) \geq 1$.} \\



\emph{Idea.}
If $\{ I_n\}$ is a covering of $[0, 1]$ then we are done
since the length of $[0, 1]$ is $1$.
However, $\{ I_n\}$ only covers $A$ and not necessarily covers $[0, 1]$.
(For example,
$\{ I_n \}
= \left\{
\left( -89, \frac{1}{\sqrt{2}} \right),
\left( \frac{1}{\sqrt{2}}, 64 \right)
\right\}$ covers $A$ but not $\frac{1}{\sqrt{2}}$.)
Hence, it is natural to consider the closure of $A$ and
the closure of $I_n$.
Now $\{ \overline{I_n} \}$ is a (closed) covering of
$\overline{A} = [0, 1]$. \\



\emph{Proof.}
\begin{align*}
  1
  &= m^{*}[0, 1]
    &\text{(Proposition 3.1)} \\
  &= m^{*}\overline{A}
    &\text{($A$ is dense in $[0, 1]$)} \\
  &\leq m^{*}\left( \overline{\bigcup I_n} \right)
    &\text{(Proposition 2.10)} \\
  &= m^{*}\left( \bigcup \overline{I_n} \right)
    &\text{(Proposition 2.10)} \\
  &\leq \sum m^{*}(\overline{I_n})
    &\text{(Proposition 3.2)} \\
  &= \sum \ell(\overline{I_n})
    &\text{(Proposition 3.1)} \\
  &= \sum \ell(I_n).
    &\text{(Definition of length)}
  \end{align*}
$\Box$ \\



\subsubsection*{Supplement 3.5.}
\addcontentsline{toc}{subsubsection}{Supplement 3.5.}
Exercise about considering the closure.
(Exercise 4.52  in the textbook: \emph{T. M. Apostol, Mathematical Analysis, 2nd edition.})
\emph{Assume that $f$ is uniformly continuous on a bounded set $S$ in $\mathbb{R}^n$.
Prove that $f$ must be bounded on $S$.} \\

\emph{Proof.}
\begin{enumerate}
\item[(1)]
  Since $f: S \rightarrow T$ is uniformly continuous,
  given any $\varepsilon > 0$, there is $\delta > 0$ such that
  $d_T(f(x), f(y)) < \varepsilon$ whenever $d_S(x, y) < \delta$.
  Choose $\varepsilon = 1 > 0$.

\item[(2)]
  For such $\delta > 0$, construct an open covering of $\overline{S} \subseteq \mathbb{R}^n$.
  Pick a collection $\mathscr{F}$ of open balls
  $B(a;\delta) \subseteq \mathbb{R}^n$
  where $a$ runs over all elements of $S$.
  $\mathscr{F}$ covers $\overline{S}$ (by the definition of accumulation points).
  Since $\overline{S} $ is closed and bounded (since $S$ is bounded),
  $\overline{S}$ is compact
  So there is a finite subcollection $\mathscr{F}'$ of $\mathscr{F}$
  also covers $\overline{S}$, say
  \[
    \mathscr{F}'
    = \left\{B(a_1;\delta)), B(a_2;\delta), \ldots, B(a_m;\delta) \right\}.
  \]

\item[(3)]
  Given any $x \in S \subseteq \overline{S}$,
  there is some $a_i \in S$ $(1 \leq i \leq m)$ such that $x \in B(a_i;\delta)$.
  In such ball, $d_S(x, a_i) < \delta$.
  By (1), $\Vert f(x) - f(a_i) \Vert < 1$,
  or $\Vert f(x) \Vert < 1 + \Vert f(a_i) \Vert$.
  Therefore, for any $x \in S$,
  \[
    \Vert f(x) \Vert < 1 + \max_{1 \leq i \leq m}{\Vert f(a_i) \Vert}.
  \]
\end{enumerate}
$\Box$ \\



%%%%%%%%%%%%%%%%%%%%%%%%%%%%%%%%%%%%%%%%%%%%%%%%%%%%%%%%%%%%%%%%%%%%%%%%%%%%%%%%



\subsubsection*{Problem 3.6.}
\addcontentsline{toc}{subsubsection}{Problem 3.6.}
\emph{Prove Proposition 5:
Given any set $A$ and any $\varepsilon > 0$,
there is an open set $O$ such that $A \subseteq O$
and $m^{*}O \leq m^{*}A + \varepsilon$.
There is a $G \in G_{\delta}$ such that $m^{*}G = m^{*}A$.} \\



\emph{Proof.}
\begin{enumerate}
\item[(1)]
  \emph{Use the definition of the outer measure.}
  By the definition of $m^{*}$,
  for such $\varepsilon > 0$ there exists a countable collection
  $\{ I_n \}$ of open intervals that covers $A$ and
  $$m^{*} A + \varepsilon \geq \sum \ell(I_n).$$

\item[(2)]
  \emph{Construct an open set $O$.}
  Let $O = \bigcup I_n \supseteq A$
  which is the union of any collection of open sets $I_n$.
  By Proposition 2.7, $O$ is open.

\item[(3)]
  \emph{Show that $m^{*}O \leq m^{*}A + \varepsilon$.}
  By Proposition 3.2 and 3.1,
  \[
    m^{*}O
    = m^{*} \left( \bigcup I_n \right)
    \leq \sum m^{*} I_n
    = \sum \ell(I_n)
    \leq m^{*} A + \varepsilon.
  \]
\end{enumerate}
Therefore, given any set $A$ and any $\varepsilon > 0$,
there is an open set $O$ such that $A \subseteq O$
and $m^{*}O \leq m^{*}A + \varepsilon$.

\begin{enumerate}
\item[(4)]
  \emph{Construct $G \in G_{\delta}$ in a natural way.}
  Given any $n \in \mathbb{N}$, there exists an open set $O_n$
  such that $O_n \supseteq A$ and $m^{*}O_n \leq m^{*}A + \frac{1}{n}$.
  Let
  \[
    G = \bigcap_{n=1}^{\infty} O_n \in G_{\delta}.
  \]

\item[(5)]
\emph{Show that $m^{*}G = m^{*}A$.}
  \begin{enumerate}
  \item[(a)]
    Since $A \subseteq O_n$ for any $n \in \mathbb{N}$,
    $A \subseteq \bigcap_{n=1}^{\infty} O_n = G$.
    Thus $m^{*}A \leq m^{*}G$.

  \item[(b)]
    Since $O_n \supseteq \bigcap_{n=1}^{\infty} O_n = G$ for any $n \in \mathbb{N}$,
    \[
      m^{*}A + \frac{1}{n} \geq m^{*}O_n \geq m^{*}G
    \]
    for any $n \in \mathbb{N}$.
    Since $n \in \mathbb{N}$ is arbitrary, $m^{*}A \geq m^{*}G$.
  \end{enumerate}
By (a)(b), $m^{*}A = m^{*}G$.
\end{enumerate}
$\Box$ \\\\



%%%%%%%%%%%%%%%%%%%%%%%%%%%%%%%%%%%%%%%%%%%%%%%%%%%%%%%%%%%%%%%%%%%%%%%%%%%%%%%%



\subsubsection*{Problem 3.7. (Translation invariant)}
\addcontentsline{toc}{subsubsection}{Problem 3.7. (Translation invariant)}
\emph{Prove that $m^{*}$ is translation invariant.} \\



\emph{Proof.}
Given $E \in \mathfrak{M}$ and $y \in \mathbb{R}$.
\begin{enumerate}
\item[(1)]
  $m^{*}(E + y) \leq m^{*}E$.
  Let $\{ I_n \}$ of open intervals that cover $E$.
  Then $\{ I_n+y \}$ of open intervals that cover $E+y$.
  Notice that the definition of $m^{*}$ and $\ell(I_n+y) = \ell(I_n)$, then
  \[
    m^{*}(E + y) \leq \sum \ell(I_n+y) = \sum \ell(I_n).
  \]
  Take the infimum of all such sum $\sum \ell(I_n)$,
  $m^{*}(E + y) \leq m^{*}E$.

\item[(2)]
  $m^{*}(E) \leq m^{*}(E + y)$.
  Similar to (1).
\end{enumerate}
By (1)(2), $m^{*}(E + y) = m^{*}E$, that is, $m^{*}$ is translation invariant.
$\Box$ \\\\



%%%%%%%%%%%%%%%%%%%%%%%%%%%%%%%%%%%%%%%%%%%%%%%%%%%%%%%%%%%%%%%%%%%%%%%%%%%%%%%%



\subsubsection*{Problem 3.8.}
\addcontentsline{toc}{subsubsection}{Problem 3.8.}
\emph{Prove that if $m^{*}A = 0$, then $m^{*}(A \cup B) = m^{*}B$.} \\



\emph{Proof.}
\begin{enumerate}
\item[(1)]
  $m^{*}(A \cup B) \geq m^{*}B$ since $A \cup B \supseteq B$
  and the definition of $m^{*}$.
  (Any covering of $A \cup B$ by open intervals is also a covering of $B$
  so that the latter infimum is taken over a larger collection than the former.)

\item[(2)]
  $m^{*}(A \cup B) \leq m^{*}B$.
  By Proposition 3.2,
  \[
    m^{*}(A \cup B) \leq m^{*}A + m^{*}B = 0 + m^{*}B = m^{*}B.
  \]
\end{enumerate}
By (1)(2), $m^{*}(A \cup B) = m^{*}B$.
$\Box$ \\\\



%%%%%%%%%%%%%%%%%%%%%%%%%%%%%%%%%%%%%%%%%%%%%%%%%%%%%%%%%%%%%%%%%%%%%%%%%%%%%%%%
%%%%%%%%%%%%%%%%%%%%%%%%%%%%%%%%%%%%%%%%%%%%%%%%%%%%%%%%%%%%%%%%%%%%%%%%%%%%%%%%



\subsection*{3.3: Measurable Sets and Lebesgue Measure \\}
\addcontentsline{toc}{subsection}{3.3: Measurable Sets and Lebesgue Measure}



\subsubsection*{Problem 3.9.}
\addcontentsline{toc}{subsubsection}{Problem 3.9.}
\emph{Show that if $E$ is a measurable set, then each translate $E+y$ of $E$
is also measurable.} \\



\emph{Proof.}
\begin{enumerate}
\item[(1)]
  \emph{$E$ is measurable if and only if
  for each set $A$, each $y \in \mathbb{R}$,}
  \[
    m^{*}(A+y)
    = m^{*}((A+y) \cap E) + m^{*}((A+y) \cap \widetilde{E}).
  \]
  \begin{enumerate}
  \item[(a)]
    $(\Longrightarrow)$
    $E$ is measurable and
    $A+y$ is a set (for any set $A$ and $y \in \mathbb{R}$).

  \item[(b)]
    $(\Longleftarrow)$
    $A = (A-y) + y$ for any set $A$ and $y \in \mathbb{R}$.
  \end{enumerate}

\item[(2)]
  For any set $E$ and $y \in \mathbb{R}$,
  $\widetilde{E+y} = \widetilde{E}+y$ by the definition of translation.

\item[(3)]
  For any sets $E_1$, $E_2$ and $y \in \mathbb{R}$,
  $(E_1 \cap E_2)+y = (E_1+y) + (E_2+y)$ by the definition of translation.

\item[(4)]
  For each set $A$ and $y \in \mathbb{R}$,
  \begin{align*}
    &\: m^{*}((A+y) \cap (E+y)) + m^{*}((A+y) \cap (\widetilde{E+y})) \\
    =&\: m^{*}((A+y) \cap (E+y)) + m^{*}((A+y) \cap (\widetilde{E}+y))
      &\text{((2))} \\
    =&\: m^{*}((A \cap E)+y) + m^{*}((A \cap \widetilde{E})+y)
      &\text{((3))} \\
    =&\: m^{*}(A \cap E) + m^{*}(A \cap \widetilde{E})
      &\text{(Problem 3.7)} \\
    =&\: m^{*}A
      &\text{(Measurability of $E$)} \\
    =&\: m^{*}(A+y).
      &\text{(Problem 3.7)}
  \end{align*}
  By (1), $E+y$ is measurable.
\end{enumerate}
$\Box$ \\\\



%%%%%%%%%%%%%%%%%%%%%%%%%%%%%%%%%%%%%%%%%%%%%%%%%%%%%%%%%%%%%%%%%%%%%%%%%%%%%%%%



\subsubsection*{Problem 3.10.}
\addcontentsline{toc}{subsubsection}{Problem 3.10.}
\emph{Show that if $E_1$ and $E_2$ are measurable, then
$$m(E_1 \cup E_2) + m(E_1 \cap E_2) = mE_1 + mE_2.$$}



\emph{Proof.}
Since the collection $\mathfrak{M}$ of measurable sets is a $\sigma$-algebra
(Theorem 3.10) and $m$ is countable additive (Proposition 3.13),
\begin{align*}
  m(E_1 \cup E_2) + m(E_1 \cap E_2)
  & = \left( m(E_1) + m(E_2 \cap \widetilde{E_1}) \right) + m(E_2 \cap E_1) \\
  & = m(E_1) + \left( m(E_2 \cap \widetilde{E_1}) + m(E_2 \cap E_1) \right) \\
  & = m(E_1) + m(E_2).
\end{align*}
($E_1$ and $E_2 \cap \widetilde{E_1}$ are disjoint.
$E_2 \cap \widetilde{E_1}$ and $E_2 \cap E_1$ are disjoint too.)
$\Box$ \\\\



%%%%%%%%%%%%%%%%%%%%%%%%%%%%%%%%%%%%%%%%%%%%%%%%%%%%%%%%%%%%%%%%%%%%%%%%%%%%%%%%



\subsubsection*{Problem 3.11.}
\addcontentsline{toc}{subsubsection}{Problem 3.11.}
\emph{Show that the condition $mE_1 < \infty$ is necessary in Proposition 3.14 by
giving a decreasing sequence $\langle E_n \rangle$ of measurable sets with
$\varnothing = \bigcap E_n$ and $mE_n = \infty$ for each $n$.} \\



\emph{Proof.}
Set $$E_n = (n, \infty)$$ for each $n \in \mathbb{N}$.
\begin{enumerate}
\item[(1)]
  \emph{$\langle E_n \rangle$ is a decreasing sequence of measurable sets.}
  $E_n \supseteq E_{n+1}$ by definition.
  Besides, each $E_n$ is measurable by Lemma 3.11.

\item[(2)]
  \emph{$\bigcap E_n = \varnothing$.}
  For each $x \in \mathbb{R}$, $x \notin E_1$ if $x \leq 1$;
  $x \notin E_{[x]}$ if $x \geq 1$ where $x \mapsto [x]$ is the floor function.

\item[(3)]
  \emph{$mE_n = \infty$ for each $n$.}
  The length of each $E_n$ is $\infty$ (Proposition 3.1).
\end{enumerate}
$\Box$ \\\\



%%%%%%%%%%%%%%%%%%%%%%%%%%%%%%%%%%%%%%%%%%%%%%%%%%%%%%%%%%%%%%%%%%%%%%%%%%%%%%%%



\subsubsection*{Problem 3.12.}
\addcontentsline{toc}{subsubsection}{Problem 3.12.}
\emph{Let $\langle E_n \rangle$ be a sequence of disjoint measurable sets and $A$ any set.
Then
$m^{*}\left( A \bigcap \bigcup_{i=1}^{\infty}E_i \right)
= \sum_{i=1}^{\infty} m^{*}(A \cap E_i)$.} \\



\emph{Proof.}
\begin{enumerate}
\item[(1)]
  $A \bigcap \bigcup_{i=1}^{\infty}E_i
  = \bigcup_{i=1}^{\infty}(A \cap E_i)$ (Problem 1.14).

\item[(2)]
  $m^{*}\left( \bigcup_{i=1}^{\infty}(A \cap E_i) \right)
  \leq \sum_{i=1}^{\infty} m^{*}(A \cap E_i)$
  by the subadditivity of $m^{*}$ (Proposition 3.2).

\item[(3)]
  By Lemma 3.9,
  $$m^{*}\left( \bigcup_{i=1}^{n}(A \cap E_i) \right)
  = \sum_{i=1}^{n} m^{*}(A \cap E_i)$$
  for any $n \in \mathbb{N}$.
  Since
  $\bigcup_{i=1}^{\infty}(A \cap E_i) \supseteq \bigcup_{i=1}^{n}(A \cap E_i)$,
  $m^{*}\left( \bigcup_{i=1}^{\infty}(A \cap E_i) \right)
  \geq m^{*}\left( \bigcup_{i=1}^{n}(A \cap E_i) \right)$ by the monotonicity of $m^{*}$.
  Thus,
  \[
    m^{*}\left( \bigcup_{i=1}^{\infty}(A \cap E_i) \right)
    \geq
    \sum_{i=1}^{n} m^{*}(A \cap E_i)
  \]
  for any $n \in \mathbb{N}$.
  Since $\sum_{i=1}^{n} m^{*}(A \cap E_i)$ is bounded and increasing
  (by the non-negativity of $m^{*}$),
  \[
    m^{*}\left( \bigcup_{i=1}^{\infty}(A \cap E_i) \right)
    \geq
    \sum_{i=1}^{\infty} m^{*}(A \cap E_i).
  \]
\end{enumerate}
By (2)(3),
$m^{*}\left( A \cap \bigcup_{i=1}^{\infty}E_i \right)
= \sum_{i=1}^{\infty} m^{*}(A \cap E_i)$.
$\Box$ \\\\



%%%%%%%%%%%%%%%%%%%%%%%%%%%%%%%%%%%%%%%%%%%%%%%%%%%%%%%%%%%%%%%%%%%%%%%%%%%%%%%%



\subsubsection*{Problem 3.13.}
\addcontentsline{toc}{subsubsection}{Problem 3.13.}
\emph{Prove Proposition 15: Let $E$ be a given set.
The the following five statements are equivalent:}
\begin{enumerate}
\item[(i)]
  \emph{$E$ is measurable.}

\item[(ii)]
  \emph{Given $\varepsilon > 0$, there is an open set $O \supseteq E$ with
  $m^{*}(O - E) < \varepsilon$.}

\item[(iii)]
  \emph{Given $\varepsilon > 0$, there is an open set $F \subseteq E$ with
  $m^{*}(E - F) < \varepsilon$.}

\item[(iv)]
  \emph{There is an $G$ in $G_{\delta}$ with $G \supseteq E$, $m^{*}(G - E) = 0$.}

\item[(v)]
  \emph{There is an $F$ in $F_{\sigma}$ with $F \subseteq E$, $m^{*}(E - F) = 0$.}
\end{enumerate}
\emph{If $m^{*}E$ is finite, the above statements are equivalent to:}
\begin{enumerate}
\item[(vi)]
  \emph{Given $\varepsilon > 0$, there is a finite union $U$ of open intervals
  such that}
  \[
    m^{*}(U \Delta E) < \varepsilon.
  \]
\end{enumerate}
\emph{(Hints:}
\begin{enumerate}
\item[(a)]
  \emph{Show that for $m^{*}E < \infty$,
  (i) $\Rightarrow$ (ii) $\Leftrightarrow$ (vi) (cf. Proposition 5).}

\item[(b)]
  \emph{Use (a) to show that for arbitrary sets $E$,
  (i) $\Rightarrow$ (ii) $\Rightarrow$ (vi) $\Rightarrow$ (i).}

\item[(c)]
  \emph{Use (b) to show that (i) $\Rightarrow$ (iii).)} \\
\end{enumerate}



\emph{Proof.}
\begin{enumerate}
\item[(1)]
  \emph{Show that for $m^{*}E < \infty$, (i) $\Rightarrow$ (ii).}
  Given $\varepsilon > 0$,
  there is a countable collections $\{ I_n \}$ of open intervals that cover $E$
  such that
  \[
    \sum \ell(I_n) < m^{*}E + \varepsilon = mE + \varepsilon
  \]
  by the definition of the outer measure and measurable sets.
  Take $O = \bigcup I_n$ be an open set which contains $E$.
  Hence,
  \begin{align*}
    m^{*}(O - E)
    &= m(O - E)
      &(\text{$O$, $E$: measurable}) \\
    &= mO - mE \\
    &= m\left( \bigcup I_n \right) - mE \\
    &\leq \sum \ell(I_n) - mE \\
    &< \varepsilon.
  \end{align*}

\item[(2)]

\end{enumerate}
$\Box$\\




%%%%%%%%%%%%%%%%%%%%%%%%%%%%%%%%%%%%%%%%%%%%%%%%%%%%%%%%%%%%%%%%%%%%%%%%%%%%%%%%
%%%%%%%%%%%%%%%%%%%%%%%%%%%%%%%%%%%%%%%%%%%%%%%%%%%%%%%%%%%%%%%%%%%%%%%%%%%%%%%%



\subsection*{3.4: A Nonmeasurable Set \\}
\addcontentsline{toc}{subsection}{3.4: A Nonmeasurable Set}



%%%%%%%%%%%%%%%%%%%%%%%%%%%%%%%%%%%%%%%%%%%%%%%%%%%%%%%%%%%%%%%%%%%%%%%%%%%%%%%%
%%%%%%%%%%%%%%%%%%%%%%%%%%%%%%%%%%%%%%%%%%%%%%%%%%%%%%%%%%%%%%%%%%%%%%%%%%%%%%%%



\subsection*{3.5: Measurable Functions \\}
\addcontentsline{toc}{subsection}{3.5: Measurable Functions}



%%%%%%%%%%%%%%%%%%%%%%%%%%%%%%%%%%%%%%%%%%%%%%%%%%%%%%%%%%%%%%%%%%%%%%%%%%%%%%%%
%%%%%%%%%%%%%%%%%%%%%%%%%%%%%%%%%%%%%%%%%%%%%%%%%%%%%%%%%%%%%%%%%%%%%%%%%%%%%%%%



\subsection*{3.6: Littlewood's Three Principles \\}
\addcontentsline{toc}{subsection}{3.6: Littlewood's Three Principles}



%%%%%%%%%%%%%%%%%%%%%%%%%%%%%%%%%%%%%%%%%%%%%%%%%%%%%%%%%%%%%%%%%%%%%%%%%%%%%%%%
%%%%%%%%%%%%%%%%%%%%%%%%%%%%%%%%%%%%%%%%%%%%%%%%%%%%%%%%%%%%%%%%%%%%%%%%%%%%%%%%
%%%%%%%%%%%%%%%%%%%%%%%%%%%%%%%%%%%%%%%%%%%%%%%%%%%%%%%%%%%%%%%%%%%%%%%%%%%%%%%%
%%%%%%%%%%%%%%%%%%%%%%%%%%%%%%%%%%%%%%%%%%%%%%%%%%%%%%%%%%%%%%%%%%%%%%%%%%%%%%%%


\end{document}