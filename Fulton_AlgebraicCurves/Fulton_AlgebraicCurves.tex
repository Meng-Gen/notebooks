\documentclass{article}
\usepackage{amsfonts}
\usepackage{amsmath}
\usepackage{amssymb}
\usepackage{hyperref}
\usepackage[none]{hyphenat}
\usepackage{mathrsfs}
\parindent=0pt



\title{\textbf{Solutions to Algebraic Curves}}
\author{Meng-Gen Tsai \\ plover@gmail.com}



\begin{document}
\maketitle
\tableofcontents



%%%%%%%%%%%%%%%%%%%%%%%%%%%%%%%%%%%%%%%%%%%%%%%%%%%%%%%%%%%%%%%%%%%%%%%%%%%%%%%%
%%%%%%%%%%%%%%%%%%%%%%%%%%%%%%%%%%%%%%%%%%%%%%%%%%%%%%%%%%%%%%%%%%%%%%%%%%%%%%%%



\newpage
\section*{Chapter 1: Affine Algebraic Sets \\}
\addcontentsline{toc}{section}{Chapter 1: Affine Algebraic Sets}



%%%%%%%%%%%%%%%%%%%%%%%%%%%%%%%%%%%%%%%%%%%%%%%%%%%%%%%%%%%%%%%%%%%%%%%%%%%%%%%%



\subsection*{1.1. Algebraic Preliminaries \\}
\addcontentsline{toc}{subsection}{1.1. Algebraic Preliminaries}



\subsubsection*{Problem 1.1.*}
\addcontentsline{toc}{subsubsection}{Problem 1.1.*}
\emph{Let $R$ be a domain.}
\begin{enumerate}
\item[(a)]
  \emph{If $F$, $G$ are forms of degree $r$, $s$ respectively in $R[X_1,\ldots,X_n]$,
  show that $FG$ is a form of degree $r+s$.}

\item[(b)]
  \emph{Show that any factor of a form in $R[X_1,\ldots,X_n]$ is also a form. } \\
\end{enumerate}

\emph{Proof of (a).}
\begin{enumerate}
\item[(1)]
  Write
  \begin{align*}
    F &= \sum_{(i)} a_{(i)} X^{(i)}, \\
    G &= \sum_{(j)} b_{(j)} X^{(j)},
  \end{align*}
  where $\sum_{(i)}$ is the summation over $(i) = (i_1,\ldots,i_n)$ with $i_1+\cdots+i_n = r$
  and $\sum_{(j)}$ is the summation over $(j) = (j_1,\ldots,j_n)$ with $j_1+\cdots+j_n = s$.

\item[(2)]
  Hence,
  \begin{align*}
    FG
    &= \sum_{(i)} \sum_{(j)} a_{(i)}b_{(j)} X^{(i)}X^{(j)} \\
    &= \sum_{(i),(j)} a_{(i)}b_{(j)} X^{(k)}
  \end{align*}
  where $(k) = (i_1+j_1,\ldots,i_n+j_n)$ with $(i_1+j_1)+\cdots+(i_n+j_n) = r+s$.
  Each $X^{(k)}$ is the form of degree $r+s$ and $a_{(i)}b_{(j)} \in R$.
  Hence $FG$ is a form of degree $r+s$.
\end{enumerate}
$\Box$\\



\emph{Proof of (b).}
\begin{enumerate}
\item[(1)]
  Given any form $F \in R[X_1,\ldots,X_n]$, and write $F = GH$.
  \emph{It suffices to show that $G$ (or $H$) is a form as well.}

\item[(2)]
  Write
  \begin{align*}
    G &= G_0+\cdots+G_r, \\
    H &= H_0+\cdots+H_s
  \end{align*}
  where $G_r \neq 0$ and $H_s \neq 0$.
  So
  \[
    F = GH = G_0H_0 + \cdots + G_r H_s.
  \]
  Since $R$ is a domain, $R[X_1,\ldots,X_n]$ is a domain and thus $G_r H_s \neq 0$.
  The maximality of $r$ and $s$ implies that $\deg(F) = r+s$.
  Therefore, by the maximality of $r+s$,
  $F = G_r H_s$, or $G = G_r$, or $G$ is a form.
\end{enumerate}
$\Box$\\\\



%%%%%%%%%%%%%%%%%%%%%%%%%%%%%%%%%%%%%%%%%%%%%%%%%%%%%%%%%%%%%%%%%%%%%%%%%%%%%%%%



\subsubsection*{Problem 1.5.*}
\addcontentsline{toc}{subsubsection}{Problem 1.5.*}
\emph{Let $k$ be any field.
Show that there are an infinitely number of irreducible monic polynomials in $k[X]$.
(Hint: Suppose $F_1,\ldots,F_n$ were all of them, and factor $F_1\cdots F_n+1$ into irreducible factors.)} \\

\emph{Proof (Due to Euclid).}
\begin{enumerate}
\item[(1)]
  If
  $F_1, F_2, \ldots, F_n$ were all irreducible monic polynomials, then
  we consider
  \[
    G = F_1 F_2 \cdots F_n + 1 \in k[X].
  \]
  So there is an irreducible monic polynomial $F$ dividing $G$
  since
  \[
    \deg G = \deg F_1 + \deg F_2 + \cdots + \deg F_n \geq 1.
  \]

\item[(2)]
  $F$ can not be any of $F_i$ for all $i$;
  otherwise $F$ would divide the difference $G - F_1 F_2 \cdots F_n = 1$.
  That is, $F \neq F_i$ for all $i$, contrary to the assumption.
\end{enumerate}
$\Box$\\\\



%%%%%%%%%%%%%%%%%%%%%%%%%%%%%%%%%%%%%%%%%%%%%%%%%%%%%%%%%%%%%%%%%%%%%%%%%%%%%%%%



\subsubsection*{Problem 1.6.*}
\addcontentsline{toc}{subsubsection}{Problem 1.6.*}
\emph{Show that any algebraically closed field is infinite.
(Hint: The irreducible monic polynomials are $X - a$, $a \in k$.)} \\

\emph{Proof (Due to Euclid).}
\begin{enumerate}
\item[(1)]
  Let $k$ be an algebraically closed field.
  If $a_1, \ldots, a_n$ were all elements in $k$, then
  we consider a monic polynomials
  \[
    F(X) = (X - a_1) \cdots (X - a_n) + 1 \in k[X].
  \]

\item[(2)]
  Since $k$ is algebraically closed,
  there is an element $a \in k$ such that $F(a) = 0$.
  By assumption, $a = a_i$ for some $1 \leq i \leq n$,
  and thus $F(a) = F(a_i) = 1$, contrary to the fact that
  a field is a commutative ring where $0 \neq 1$ and all nonzero elements are invertible.
\end{enumerate}
$\Box$\\\\



%%%%%%%%%%%%%%%%%%%%%%%%%%%%%%%%%%%%%%%%%%%%%%%%%%%%%%%%%%%%%%%%%%%%%%%%%%%%%%%%
%%%%%%%%%%%%%%%%%%%%%%%%%%%%%%%%%%%%%%%%%%%%%%%%%%%%%%%%%%%%%%%%%%%%%%%%%%%%%%%%



\subsection*{1.2. Affine Space and Algebraic Sets \\}
\addcontentsline{toc}{subsection}{1.2. Affine Space and Algebraic Sets}



%%%%%%%%%%%%%%%%%%%%%%%%%%%%%%%%%%%%%%%%%%%%%%%%%%%%%%%%%%%%%%%%%%%%%%%%%%%%%%%%



\subsubsection*{Problem 1.8.*}
\addcontentsline{toc}{subsubsection}{Problem 1.8.*}
\emph{Show that the algebraic subsets of $\mathbb{A}^1(k)$ are just the finite subsets, together
with $\mathbb{A}^1(k)$ itself.} \\

\emph{Proof.}
\begin{enumerate}
\item[(1)]
  \emph{Show that $k[X]$ is a PID if $k$ is a field.}
  \begin{enumerate}
  \item[(a)]
    Let $I$ be an ideal of $k[X]$.

  \item[(b)]
    If $I = \{0\}$ then $I = (0)$ and $I$ is principal.

  \item[(c)]
    If $I \neq \{0\}$, then take $F$ to be a polynomial of minimal degree in $I$.
    It suffices to show that $I = (F)$.
    Clearly, $(F) \subseteq I$ since $I$ is an ideal.
    Conversely, for any $G \in I$,
    \[
      G = FQ + R
    \]
    for some $Q, R \in k[X]$ with $R = 0$ or $\deg R < \deg F$.
    Now as
    \[
      R = G - FQ \in I,
    \]
    $R = 0$ (otherwise contrary to the minimality of $F$),
    we have $G \in (F)$ for all $G \in I$.
  \end{enumerate}

\item[(2)]
  Let $X$ be an algebraic subset of $\mathbb{A}^1(k)$,
  say $X = V(I)$ for some ideal $I$ of $k[X]$.
  Since $k[X]$ is a PID, $I = (F)$ for some $F \in k[X]$.
  \begin{enumerate}
  \item[(a)]
    If $F = 0$, then $I = (0)$ and $X = V(0) = \mathbb{A}^1(k)$.

  \item[(b)]
    If $F \neq 0$, then $F(X) = 0$ has finitely many roots in $k$,
    say $P_1, \ldots, P_m \in k$.
    Hence,
    \[
      X = V(I) = V(F) = \{ F(P) = 0 : P \in k \}
      = \{ P_1, \ldots, P_m \}
    \]
    is a finite subsets of $X$.
  \end{enumerate}
  By (a)(b), the result is established.
\end{enumerate}
$\Box$\\



\emph{Notes.}
\begin{enumerate}
\item[(1)]
  By the Hilbert basis theorem, $k[X]$ is Noetherian as $k$ is Noetherian.
  Hence, for any algebraic subset $X = V(I)$ of $\mathbb{A}^1(k)$,
  we can write $I = (F_1, \cdots, F_m)$.
  Note that
  \[
    V(I) = V(F_1) \cap \cdots \cap V(F_m).
  \]
  Now apply the same argument to get the same conclusion.

\item[(2)]
  Suppose $k = \overline{k}$.
  $\mathbb{A}^1(k)$ is irreducible, because its only proper closed subsets are finite,
  yet it is infinite
  (because $k$ is algebraically closed, hence infinite).

\end{enumerate}



%%%%%%%%%%%%%%%%%%%%%%%%%%%%%%%%%%%%%%%%%%%%%%%%%%%%%%%%%%%%%%%%%%%%%%%%%%%%%%%%
%%%%%%%%%%%%%%%%%%%%%%%%%%%%%%%%%%%%%%%%%%%%%%%%%%%%%%%%%%%%%%%%%%%%%%%%%%%%%%%%


\end{document}