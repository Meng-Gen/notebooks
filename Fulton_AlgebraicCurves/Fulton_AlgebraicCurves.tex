\documentclass{article}
\usepackage{amsfonts}
\usepackage{amsmath}
\usepackage{amssymb}
\usepackage{hyperref}
\usepackage[none]{hyphenat}
\usepackage{mathrsfs}
\parindent=0pt



\title{\textbf{Solutions to the book: \\\emph{Fulton, Algebraic Curves}}}
\author{Meng-Gen Tsai \\ plover@gmail.com}



\begin{document}
\maketitle
\tableofcontents



%%%%%%%%%%%%%%%%%%%%%%%%%%%%%%%%%%%%%%%%%%%%%%%%%%%%%%%%%%%%%%%%%%%%%%%%%%%%%%%%
%%%%%%%%%%%%%%%%%%%%%%%%%%%%%%%%%%%%%%%%%%%%%%%%%%%%%%%%%%%%%%%%%%%%%%%%%%%%%%%%



\newpage
\section*{Chapter 1: Affine Algebraic Sets \\}
\addcontentsline{toc}{section}{Chapter 1: Affine Algebraic Sets}



%%%%%%%%%%%%%%%%%%%%%%%%%%%%%%%%%%%%%%%%%%%%%%%%%%%%%%%%%%%%%%%%%%%%%%%%%%%%%%%%



\subsection*{1.1. Algebraic Preliminaries \\}
\addcontentsline{toc}{subsection}{1.1. Algebraic Preliminaries}



\subsubsection*{Problem 1.1.*}
\addcontentsline{toc}{subsubsection}{Problem 1.1.*}
\emph{Let $R$ be a domain.}
\begin{enumerate}
\item[(a)]
  \emph{If $f$, $g$ are forms of degree $r$, $s$ respectively in $R[x_1,\ldots,x_n]$,
  show that $fg$ is a form of degree $r+s$.}

\item[(b)]
  \emph{Show that any factor of a form in $R[x_1,\ldots,x_n]$ is also a form. } \\
\end{enumerate}

\emph{Proof of (a).}
\begin{enumerate}
\item[(1)]
  Write
  \begin{align*}
    f &= \sum_{(i)} a_{(i)} x^{(i)}, \\
    g &= \sum_{(j)} b_{(j)} x^{(j)},
  \end{align*}
  where $\sum_{(i)}$ is the summation over $(i) = (i_1,\ldots,i_n)$ with $i_1+\cdots+i_n = r$
  and $\sum_{(j)}$ is the summation over $(j) = (j_1,\ldots,j_n)$ with $j_1+\cdots+j_n = s$.

\item[(2)]
  Hence,
  \begin{align*}
    fg
    &= \sum_{(i)} \sum_{(j)} a_{(i)}b_{(j)} x^{(i)}x^{(j)} \\
    &= \sum_{(i),(j)} a_{(i)}b_{(j)} x^{(k)}
  \end{align*}
  where $(k) = (i_1+j_1,\ldots,i_n+j_n)$ with $(i_1+j_1)+\cdots+(i_n+j_n) = r+s$.
  Each $x^{(k)}$ is the form of degree $r+s$ and $a_{(i)}b_{(j)} \in R$.
  Hence $fg$ is a form of degree $r+s$.
\end{enumerate}
$\Box$\\



\emph{Proof of (b).}
\begin{enumerate}
\item[(1)]
  Given any form $f \in R[x_1,\ldots,x_n]$, and write $f = gh$.
  \emph{It suffices to show that $g$ is a form as well.}
  (So does $h$.)

\item[(2)]
  Write
  \[
    g = g_0+\cdots+g_r,
    \qquad
    h = h_0+\cdots+h_s
  \]
  where $g_r \neq 0$ and $h_s \neq 0$.
  So
  \[
    f = gh = g_0h_0 + \cdots + g_r h_s.
  \]
  Since $R$ is a domain, $R[x_1,\ldots,x_n]$ is a domain and thus $g_r h_s \neq 0$.
  The maximality of $r$ and $s$ implies that $\deg f = r+s$.
  Therefore, by the maximality of $r+s$,
  $f = g_r h_s$, or $g = g_r$, or $g$ is a form.
\end{enumerate}
$\Box$\\\\



%%%%%%%%%%%%%%%%%%%%%%%%%%%%%%%%%%%%%%%%%%%%%%%%%%%%%%%%%%%%%%%%%%%%%%%%%%%%%%%%



\subsubsection*{Problem 1.2.*}
\addcontentsline{toc}{subsubsection}{Problem 1.2.*}
\emph{Let $R$ be a UFD,
$K$ the quotient field of $R$.
Show that every element $z$ of $K$ may be written $z = a/b$,
where $a, b \in R$ have no common factors;
this representative is unique up to units of $R$.} \\

\emph{Proof.}
\begin{enumerate}
\item[(1)]
  \emph{Show that every element $z$ of $K$ may be written $z=a/b$,
  where $a$, $b\in R$ have no common factors.}
  Given any $z = a/b \in K$ where $a, b\in R$.
  Write
  \begin{align*}
    a &= p_1 \cdots p_n, \\
    b &= q_1 \cdots q_m
  \end{align*}
  where all $p_1, \ldots, p_n, q_1, \ldots, q_m$ are irreducible in $R$.
  (It is possible since $R$ is a UFD.)
  For each $i$, suppose $p_i \mid q_j$ for some $i, j$.
  Write $q_j = p_i u$ for some $u \in R$.
  By the irreducibility of $p_i$ and $q_j$, $u$ is a unit.
  So
  \[
    z
    = \frac{a}{b}
    = \frac{p_1 \cdots \widehat{p_i} \cdots p_n}{q_1 \cdots \widehat{q_j} \cdots q_m}
    = \frac{p_1 \cdots \widehat{p_i} \cdots p_n}{u q_1 \cdots \widehat{q_j} \cdots q_m}.
  \]
  Continue this method we can write $z = \frac{a'}{b'}$ where $a'$ and $b'$ have no common factors.

\item[(2)]
  Write $z = a/b = a'/b'$ where
  \begin{enumerate}
  \item[(a)]
    $a, b, a', b' \in R$,

  \item[(b)]
    $a$ and $b$ have no common factors,

  \item[(c)]
    $a'$ and $b'$ have no common factors.
  \end{enumerate}
  Write
  \begin{align*}
    a &= p_1 \cdots p_n, \\
    b &= q_1 \cdots q_m, \\
    a' &= p'_1 \cdots p'_{n'}, \\
    b' &= q'_1 \cdots q'_{m'}
  \end{align*}
  where all $p_i, q_j, p'_{i'}, q'_{j'}$ are irreducible in $R$.
  As $z = a/b = a'/b'$, $ab' = a'b$ or
  \[
    p_1 \cdots p_n q'_1 \cdots q'_{m'}
    = p'_1 \cdots p'_{n'} q_1 \cdots q_m.
  \]

\item[(3)]
  For $i = 1$, $p_1 = u_1 p'_{i'}$ for some unit $u_1 \in R$
  since $a$ and $b$ have no common factors and all $p_1, q_j, p'_{i'}$ are irreducible.
  Hence
  \[
    u_1 \widehat{p_1} p_2 \cdots p_n q'_1 \cdots q'_{m'}
    = p'_1 \cdots \widehat{p'_{i'}} \cdots p'_{n'} q_1 \cdots q_m.
  \]
  Continue this method,
  we have $n \leq n'$ and all $p_1, \ldots, p_n$ are canceled.

\item[(4)]
  Conversely, we can apply the argument in (3) to $i' = 1, \ldots n'$ to conclude that $n' \leq n$.
  Therefore, $n = n'$ and
  \[
    \underbrace{u_1 \cdots u_n}_{\text{a unit in $R$}} q'_1 \cdots q'_{m'} = q_1 \cdots q_m.
  \]
  Hence, $b = ub'$ where $u = u_1 \cdots u_n$ is a unit in $R$.
  Similarly, $a = va'$ where $v$ is a unit in $R$.
  So the representative of $z \in K$ is unique up to units of $R$.
\end{enumerate}
$\Box$\\\\



%%%%%%%%%%%%%%%%%%%%%%%%%%%%%%%%%%%%%%%%%%%%%%%%%%%%%%%%%%%%%%%%%%%%%%%%%%%%%%%%



\subsubsection*{Problem 1.3.*}
\addcontentsline{toc}{subsubsection}{Problem 1.3.*}
\emph{Let $R$ be a PID. Let $\mathfrak{p}$ be a nonzero, proper, prime ideal in $R$.}
\begin{enumerate}
\item[(a)]
  \emph{Show that $\mathfrak{p}$ is generated by an irreducible element.}

\item[(b)]
  \emph{Show that $\mathfrak{p}$ is maximal.} \\
\end{enumerate}



\emph{Proof of (a).}
\begin{enumerate}
\item[(1)]
  Let $\mathfrak{p} = (a)$ be a nonzero, proper, prime ideal in $R$.
  It suffices to show that $a$ is irreducible.

\item[(2)]
  Suppose $a = bc$.
  By the primality of $\mathfrak{p}$, $b \in \mathfrak{p}$ or $c \in \mathfrak{p}$.
  Suppose $b \in \mathfrak{p} = (a)$. (The case $c \in \mathfrak{p}$ is similar.)
  Then there is a $d \in R$ such that $b = ad$.
  Hence, $a = bc = adc$ or $(1-dc)a = 0$.

\item[(3)]
  Since $R$ is a domain, $1 = dc$ or $a = 0$.
  $a = 0$ implies that $\mathfrak{p} = (0)$ is a zero ideal, contrary to the assumption.
  Therefore, $1 = dc$, or $c$ is a unit, or $a$ is irreducible.
\end{enumerate}
$\Box$\\



\emph{Proof of (b).}
\begin{enumerate}
\item[(1)]
  Given any ideal $I = (b)$ of $R$ containing $\mathfrak{p} = (a)$.
  As the generator $a$ of $\mathfrak{p}$ is in $\mathfrak{p} \subseteq I$,
  there is some $c \in R$ such that $a = bc$.
  By the irreducibility of $a$ (in (a)), $b$ is a unit or $c$ is a unit.

\item[(2)]
  $b$ is a unit implies that $I = R$.
  $c$ is a unit implies that $I = \mathfrak{p}$.
  In any case, we conclude that $\mathfrak{p}$ is maximal.
\end{enumerate}
$\Box$\\\\



%%%%%%%%%%%%%%%%%%%%%%%%%%%%%%%%%%%%%%%%%%%%%%%%%%%%%%%%%%%%%%%%%%%%%%%%%%%%%%%%



\subsubsection*{Problem 1.4.*}
\addcontentsline{toc}{subsubsection}{Problem 1.4.*}
\emph{Let $k$ be an infinite field,
$f \in k[x_1,\ldots,x_n]$.
Suppose $f(a_1,\ldots,a_n) = 0$ for all $a_1, \ldots, a_n \in k$.
Show that $f = 0$.
(Hint: Write
\[
  f = \sum f_i x_n^{i},
  \qquad
  f_i \in k[x_1,\ldots,x_{n-1}].
\]
Use induction on $n$,
and the fact that $f(a_1, \ldots, a_{n-1}, x_n)$
has only a finite number of roots if any $f_i(a_1, \ldots, a_{n-1}) \neq 0$.)} \\



\emph{Proof.}
\begin{enumerate}
\item[(1)]
  Induction on $n$.
  The case $n = 1$.
  (Reductio ad absurdum)
  If there were a nonzero $f \in k[x_1]$ such that $f(a) = 0$ for all $a \in k$.
  Note that $f$ has at most $\deg f < \infty$ roots,
  contrary to the infinity of $k$.

\item[(2)]
  Assume that the conclusion holds for $n - 1$, then for any $f \in k[x_1,\ldots,x_n]$
  we can write
  \[
    f = \sum f_i x_n^{i},
    \qquad
    f_i \in k[x_1,\ldots,x_{n-1}]
  \]
  as $f \in (k[x_1,\ldots,x_{n-1}])[x_n]$.
  Suppose $f(a_1,\ldots,a_n) = 0$ for all $a_1, \ldots, a_n \in k$.
  For fixed $a_1, \ldots, a_{n-1}$,
  the polynomial $f(a_1, \ldots, a_{n-1}, x_n) \in k[x_n]$ has all distinct roots in
  an infinite field $k$.
  By (1), $f(a_1, \ldots, a_{n-1}, x_n) = 0 \in k[x_n]$,
  or each $f_i(a_1, \ldots, a_{n-1}) = 0$.
  As all $a_1, \ldots, a_{n-1}$ run over $k$,
  we can apply the induction hypothesis
  each $f_i(x_1, \ldots, x_{n-1}) = 0 \in k[x_1,\ldots,x_{n-1}]$.
  Hence, $f = 0 \in k[x_1,\ldots,x_{n}]$.
\end{enumerate}
$\Box$\\



\emph{Note.}
If $k$ is a finite field of order $q = p^k$,
then the polynomial $f(x) = x^q - x$ has $q$ distinct roots in $k$. \\\\



%%%%%%%%%%%%%%%%%%%%%%%%%%%%%%%%%%%%%%%%%%%%%%%%%%%%%%%%%%%%%%%%%%%%%%%%%%%%%%%%



\subsubsection*{Problem 1.5.*}
\addcontentsline{toc}{subsubsection}{Problem 1.5.*}
\emph{Let $k$ be any field.
Show that there are an infinitely number of irreducible monic polynomials in $k[x]$.
(Hint: Suppose $f_1,\ldots,f_n$ were all of them, and factor $f_1\cdots f_n+1$ into irreducible factors.)} \\

\emph{Proof (Due to Euclid).}
\begin{enumerate}
\item[(1)]
  If
  $f_1, \ldots, f_n$ were all irreducible monic polynomials, then
  we consider
  \[
    g = f_1 \cdots f_n + 1 \in k[x].
  \]
  So there is an irreducible monic polynomial $f = f_i$ dividing $g$ for some $i$
  since
  \[
    \deg g = \deg f_1 + \cdots + \deg f_n \geq 1
  \]
  and $k[x]$ is a UFD.

\item[(2)]
  However, $f$ would divide the difference
  \[
    g - f_1 \cdots f_{i-1} f_i f_{i+1} \cdots f_n = 1,
  \]
  contrary to $\deg f_i \geq 1$.
\end{enumerate}
$\Box$\\\\



%%%%%%%%%%%%%%%%%%%%%%%%%%%%%%%%%%%%%%%%%%%%%%%%%%%%%%%%%%%%%%%%%%%%%%%%%%%%%%%%



\subsubsection*{Problem 1.6.*}
\addcontentsline{toc}{subsubsection}{Problem 1.6.*}
\emph{Show that any algebraically closed field is infinite.
(Hint: The irreducible monic polynomials are $x - a$, $a \in k$.)} \\

\emph{Proof (Due to Euclid).}
\begin{enumerate}
\item[(1)]
  Let $k$ be an algebraically closed field.
  If $a_1, \ldots, a_n$ were all elements in $k$, then
  we consider a monic polynomials
  \[
    f(x) = (x - a_1) \cdots (x - a_n) + 1 \in k[x].
  \]

\item[(2)]
  Since $k$ is algebraically closed,
  there is an element $a \in k$ such that $f(a) = 0$.
  By assumption, $a = a_i$ for some $1 \leq i \leq n$,
  and thus $f(a) = f(a_i) = 1$, contrary to the fact that
  a field is a commutative ring where $0 \neq 1$ and all nonzero elements are invertible.
\end{enumerate}
$\Box$\\\\



%%%%%%%%%%%%%%%%%%%%%%%%%%%%%%%%%%%%%%%%%%%%%%%%%%%%%%%%%%%%%%%%%%%%%%%%%%%%%%%%



\subsubsection*{Problem 1.7.*}
\addcontentsline{toc}{subsubsection}{Problem 1.7.*}
\emph{Let $k$ be a field, $f \in k[x_1, \ldots, x_n]$, $a_1, \ldots, a_n \in k$.}
\begin{enumerate}
\item[(a)]
  \emph{Show that}
  \[
    f = \sum \lambda_{(i)} (x_1-a_1)^{i_1} \cdots (x_n-a_n)^{i_n},
    \qquad
    \lambda_{(i)} \in k.
  \]

\item[(b)]
  \emph{If $f(a_1, \ldots, a_n) = 0$,
  show that $f = \sum_{i=1}^n (x_i-a_i) g_i$ for some (not unique) $g_i$ in $k[x_1, \ldots, x_n]$.} \\
\end{enumerate}



\emph{Proof of (a).}
\begin{enumerate}
\item[(1)]
  Regard $k[x_1, \ldots, x_n]$ as $(k[x_1, \ldots, x_{n-1}])[x_n]$.
  Since $(k[x_1, \ldots, x_{n-1}])[x_n]$ is a Euclidean domain with a function
  \[
    f \in (k[x_1, \ldots, x_{n-1}])[x_n] \mapsto \deg_{x_n} f \in \mathbb{Z}_{\geq 0}
  \]
  satisfying the division-with-remainder property.

\item[(2)]
  Apply the division algorithm for $f$ and nonzero $x_n-a_n$
  to produce a quotient $q$ and remainder $r$ with
  $f = (x_n-a_n) q + r$ and either $r = 0$ or $\deg_{x_n}(r) < \deg_{x_n} (x_n-a_n) = 1$.
  That is, $r \in k[x_1, \ldots, x_{n-1}]$ is a constant in $(k[x_1, \ldots, x_{n-1}])[x_n]$.
  Continue this process to get that $f$ is of the form
  \[
    f = \sum_{i_n} f_{i_n} (x_n - a_n)^{i_n}
  \]
  where $f_{i_n} \in k[x_1, \ldots, x_{n-1}]$.

\item[(3)]
  Use the same argument in (2) for each $f_{i_n} \in k[x_1, \ldots, x_{n-1}]$, we have
  \begin{align*}
    f_{i_n}
    &= \sum_{i_{n-1}} \underbrace{f_{i_n,i_{n-1}}}_{\in k[x_1, \ldots, x_{n-2}]}
      (x_{n-1} - a_{n-1})^{i_{n-1}} \\
    f_{i_n,i_{n-1}}
    &=
    \sum_{i_{n-2}} \underbrace{f_{i_n,i_{n-1},i_{n-2}}}_{\in k[x_1, \ldots, x_{n-3}]}
      (x_{n-2} - a_{n-2})^{i_{n-2}}, \\
    & \cdots \\
    f_{i_n,\ldots,i_{2}}
    &= \sum_{i_1} \underbrace{f_{i_n,\ldots,i_1}}_{\in k} (x_1 - a_1)^{i_1}.
  \end{align*}
  Note that $f_{i_n,\ldots,i_1} \in k$, we can write
  \[
    f = \sum \lambda_{(i)} (x_1-a_1)^{i_1} \cdots (x_n-a_n)^{i_n},
    \qquad
    \lambda_{(i)} \in k.
  \]
  by replacing all $f_{i_n,\ldots,i_k}$ by $f_{i_n,\ldots,i_{k-1}}$
  for $k = n, n-1, \ldots, 2$.

\item[(4)]
  Or use the induction on $n$.
\end{enumerate}
$\Box$\\



\emph{Proof of (b).}
\begin{enumerate}
\item[(1)]
  Write
  \[
    f = \sum \lambda_{(i)} (x_1-a_1)^{i_1} \cdots (x_n-a_n)^{i_n},
    \qquad
    \lambda_{(i)} \in k
  \]
  by (a).

\item[(2)]
  As $f(a_1, \cdots, a_n) = 0$,
  $\lambda_{(i)} = 0$ if all $i_1, \ldots, i_n$ are zero, that it,
  there is no nonzero constant term in the representation of $f$.
  Hence, for each term
  \[
    f_{(i)} : = \lambda_{(i)} (x_1-a_1)^{i_1} \cdots (x_n-a_n)^{i_n}
  \]
  with $\lambda_{(i)} \neq 0$,
  there exists one $i_k > 0$ for some $1 \leq k \leq n$.
  So we can write
  \[
    f_{(i)}
    =
    (x_k-a_k)
      \underbrace{
        (\lambda_{(i)} (x_1-a_1)^{i_1} \cdots (x_k-a_k)^{i_k-1} \cdots (x_n-a_n)^{i_n})}_{
          := g_{(i)} \in k[x_1,\ldots,x_n]}.
  \]
  Note that the expression of $f_{(i)}$ is not unique since
  there may exist more than one $i_k > 0$ as $1 \leq k \leq n$.

\item[(3)]
  Now we iterate each nonzero term in $f$, apply the factorization in (2),
  and then group by each $x_k-a_k$.
  Therefore, we can write
  \[
    f = \sum_{i=1}^{n}(x_i-a_i)g_i
  \]
  for some $g_1 \in k[x_1, \ldots, x_n]$.

\item[(4)]
  The expression of $f$ is not unique.
  For example, take $f(x,y) = x^2 + 2xy + y^2 \in k[x,y]$.
  As $f(0,0) = 0$, we can write
  \begin{align*}
    f(x,y)
    &= x \cdot \underbrace{(x+2y)}_{g_1} + y \cdot \underbrace{y}_{g_2}, \text{ or } \\
    &= x \cdot \underbrace{(x+y)}_{g_1} + y \cdot \underbrace{(x+y)}_{g_2}, \text{ or } \\
    &= x \cdot \underbrace{x}_{g_1} + y \cdot \underbrace{(2x+y)}_{g_2}.
  \end{align*}
\end{enumerate}
$\Box$\\\\



%%%%%%%%%%%%%%%%%%%%%%%%%%%%%%%%%%%%%%%%%%%%%%%%%%%%%%%%%%%%%%%%%%%%%%%%%%%%%%%%
%%%%%%%%%%%%%%%%%%%%%%%%%%%%%%%%%%%%%%%%%%%%%%%%%%%%%%%%%%%%%%%%%%%%%%%%%%%%%%%%



\subsection*{1.2. Affine Space and Algebraic Sets \\}
\addcontentsline{toc}{subsection}{1.2. Affine Space and Algebraic Sets}



\subsubsection*{Problem 1.8.*}
\addcontentsline{toc}{subsubsection}{Problem 1.8.*}
\emph{Show that the algebraic subsets of $\mathbf{A}^1(k)$ are just the finite subsets, together
with $\mathbf{A}^1(k)$ itself.} \\

\emph{Proof.}
\begin{enumerate}
\item[(1)]
  \emph{Show that $k[x]$ is a PID if $k$ is a field.}
  \begin{enumerate}
  \item[(a)]
    Let $I$ be an ideal of $k[x]$.

  \item[(b)]
    If $I = \{0\}$ then $I = (0)$ and $I$ is principal.

  \item[(c)]
    If $I \neq \{0\}$, then take $f$ to be a polynomial of minimal degree in $I$.
    It suffices to show that $I = (f)$.
    Clearly, $(f) \subseteq I$ since $I$ is an ideal.
    Conversely, for any $g \in I$,
    \[
      g(x) = f(x)h(x) + r(x)
    \]
    for some $h, r \in k[x]$ with $r = 0$ or $\deg r < \deg f$ (as $k[x]$ is a Euclidean domain).
    Now as
    \[
      r = g - fh \in I,
    \]
    $r = 0$ (otherwise contrary to the minimality of $f$),
    we have $g = fh \in (f)$ for all $g \in I$.
  \end{enumerate}

\item[(2)]
  Let $Y$ be an algebraic subset of $\mathbf{A}^1(k)$,
  say $Y = V(I)$ for some ideal $I$ of $k[x]$.
  Since $k[x]$ is a PID, $I = (f)$ for some $f \in k[x]$.
  \begin{enumerate}
  \item[(a)]
    If $f = 0$, then $I = (0)$ and $Y = V(0) = \mathbf{A}^1(k)$.

  \item[(b)]
    If $f \neq 0$, then $f(x) = 0$ has finitely many roots in $k$,
    say $a_1, \ldots, a_m \in k$.
    Hence,
    \[
      Y = V(I) = V(f) = \{ f(a) = 0 : a \in k \}
      = \{ a_1, \ldots, a_m \}
    \]
    is a finite subsets of $\mathbf{A}^1(k)$.
  \end{enumerate}
  By (a)(b), the result is established.
\end{enumerate}
$\Box$\\



\emph{Notes.}
\begin{enumerate}
\item[(1)]
  By the Hilbert basis theorem, $k[x]$ is Noetherian as $k$ is Noetherian.
  Hence, for any algebraic subset $Y = V(I)$ of $\mathbf{A}^1(k)$,
  we can write $I = (f_1, \cdots, f_m)$.
  Note that
  \[
    Y = V(I) = V(f_1) \cap \cdots \cap V(f_m).
  \]
  Now apply the same argument to get the same conclusion.

\item[(2)]
  Suppose $k = \overline{k}$.
  $\mathbf{A}^1(k)$ is irreducible, because its only proper closed subsets are finite,
  yet it is infinite
  (because $k$ is algebraically closed, hence infinite). \\
\end{enumerate}



%%%%%%%%%%%%%%%%%%%%%%%%%%%%%%%%%%%%%%%%%%%%%%%%%%%%%%%%%%%%%%%%%%%%%%%%%%%%%%%%



\subsubsection*{Problem 1.9.}
\addcontentsline{toc}{subsubsection}{Problem 1.9.}
\emph{If $k$ is a finite field, show that every subset of $\mathbf{A}^{n}(k)$ is algebraic.} \\

\emph{Proof.}
\begin{enumerate}
\item[(1)]
  Every subset of $\mathbf{A}^{n}(k)$ is finite since
  $|\mathbf{A}^{n}(k)| = |k|^n$ is finite.

\item[(2)]
  Note that $V(x_1-a_1,\ldots,x_n-a_n) = \{ (a_1,\ldots,a_n) \} \subseteq \mathbf{A}^{n}(k)$
  (property (5) in this section)
  and any finite union of algebraic sets is algebraic (property (4) in this section).
  Thus, every subset of $\mathbf{A}^{n}(k)$ is algebraic (by (1)).
\end{enumerate}
$\Box$\\\\



%%%%%%%%%%%%%%%%%%%%%%%%%%%%%%%%%%%%%%%%%%%%%%%%%%%%%%%%%%%%%%%%%%%%%%%%%%%%%%%%



\subsubsection*{Problem 1.10.}
\addcontentsline{toc}{subsubsection}{Problem 1.10.}
\emph{Give an example of a countable collection of algebraic sets whose union is not
algebraic.} \\

\emph{Proof.}
\begin{enumerate}
\item[(1)]
  Let $k = \mathbb{Q}$ be an infinite field.
  $V(x-a) = \{ a \}$ is an algebraic sets for all $a \in \mathbb{Q}$.
  In particular, $V(x-a) = \{ a \}$ is algebraic for all $a \in \mathbb{Z}$.

\item[(2)]
  Note that
  \[
    Y := \bigcup_{a \in \mathbb{Z}} V(x-a) = \mathbb{Z}
  \]
  is a countable union of algebraic sets.
  Since $Y$ is a proper subset of $k = \mathbb{Q}$,
  it cannot be algebraic by Problem 1.8.
\end{enumerate}
$\Box$\\\\



%%%%%%%%%%%%%%%%%%%%%%%%%%%%%%%%%%%%%%%%%%%%%%%%%%%%%%%%%%%%%%%%%%%%%%%%%%%%%%%%



\subsubsection*{Problem 1.11.}
\addcontentsline{toc}{subsubsection}{Problem 1.11.}
\emph{Show that the following are algebraic sets:}
\begin{enumerate}
\item[(a)]
  $\{ (t,t^2,t^3) \in \mathbf{A}^{3}(k) : t \in k \}$;

\item[(b)]
  $\{ (\cos(t),\sin(t)) \in \mathbf{A}^{2}(\mathbb{R}) : t \in \mathbb{R} \}$;

\item[(c)]
  \emph{the set of points in $\mathbf{A}^{2}(\mathbb{R})$
  whose polar coordinates $(r,\theta)$ satisfy the equation $r = \sin(\theta)$.} \\
\end{enumerate}



\emph{Proof of (a).}
\begin{enumerate}
\item[(1)]
  The twisted cubic curve
  \[
    Y = \{ (t,t^2,t^3) \in \mathbf{A}^3(k) : t \in k \}
    =
    V(x^2-y) \cap V(x^3-z)
  \]
  is algebraic.
  We say that $Y$ is given by the parametric representation $x=t$, $y=t^2$, $z=t^3$.

\item[(2)]
  The generators for the ideal $I(Y)$ are $x^2-y$ and $x^3-z$.

\item[(3)]
  $Y$ is an affine variety of dimension $1$.

\item[(4)]
  The affine coordinate ring $A(Y)$ is isomorphic to a polynomial ring in one variable over $k$.
\end{enumerate}
$\Box$\\



\emph{Proof of (b).}
The circle
\[
  \{(\cos(t),\sin(t)) \in \mathbf{A}^2(\mathbb{R}) : t \in \mathbb{R} \} = V(x^2-y^2-1)
\]
is algebraic.
$\Box$\\



\emph{Proof of (c).}
The circle
\[
  \{ (r,\theta) : r = \sin(\theta) \} = V(x^2+y^2-y)
\]
is algebraic again.
$\Box$\\\\



%%%%%%%%%%%%%%%%%%%%%%%%%%%%%%%%%%%%%%%%%%%%%%%%%%%%%%%%%%%%%%%%%%%%%%%%%%%%%%%%



\subsubsection*{Problem 1.15.*}
\addcontentsline{toc}{subsubsection}{Problem 1.15.*}
\emph{Let $V \subseteq \mathbf{A}^n(k)$, $W \subseteq \mathbf{A}^m(k)$ be algebraic sets.
Show that
\[
  V \times W
  = \{(a_1,\ldots,a_n,b_1,\ldots,b_m) : (a_1,\ldots,a_n) \in V, (b_1,\ldots,b_m) \in W \}
\]
is an algebraic set in $\mathbf{A}^{n+m}(k)$.
It is called the \textbf{product} of $V$ and $W$.} \\

\emph{Proof.}
\begin{enumerate}
\item[(1)]
  Write
  \begin{align*}
    V &= V(S_V) = \{ P \in \mathbf{A}^n(k) : f(P) = 0 \: \forall f \in S_V \} \\
    W &= V(S_W) = \{ Q \in \mathbf{A}^m(k) : g(Q) = 0 \: \forall g \in S_W \},
  \end{align*}
  where $S_V \subseteq k[x_1,\ldots,x_n]$ and $S_W \subseteq k[y_1,\ldots,y_m]$.
  It suffices to show that
  \[
    V \times W = V(S),
  \]
  where
  $S \subseteq k[x_1,\ldots,x_n,y_1,\ldots,y_m]$ is the union of $S_V$ and $S_W$.

\item[(2)]
  Here we can identify $S_V$ with the subset of
  $k[x_1,\ldots,x_n,y_1,\ldots,y_m]$
  by noting that
  \[
    k[x_1,\ldots,x_n]
    \hookrightarrow (k[y_1,\ldots,y_m])[x_1,\ldots,x_n]
    = k[x_1,\ldots,x_n,y_1,\ldots,y_m].
  \]
  Here we regard $k$ as a subring of $k[y_1,\ldots,y_m]$.
  Similar treatment to $S_W$.

\item[(3)]
  By construction, $V \times W \subseteq V(S)$.
  Conversely, given any $(P,Q) \in V(S) \subseteq \mathbf{A}^{n+m}(k)$,
  we have $h(P,Q) = 0$ for all $h \in S = S_V \cup S_W$ (by (2)).
  By construction, $f(P) = 0$ for all $f \in S_V$ since $f$ only involve $x_1,\ldots,x_n$.
  Hence, $P \in V$. Similarly, $Q \in W$. Therefore, $(P,Q) \in V \times W$.
\end{enumerate}
$\Box$\\\\



%%%%%%%%%%%%%%%%%%%%%%%%%%%%%%%%%%%%%%%%%%%%%%%%%%%%%%%%%%%%%%%%%%%%%%%%%%%%%%%%
%%%%%%%%%%%%%%%%%%%%%%%%%%%%%%%%%%%%%%%%%%%%%%%%%%%%%%%%%%%%%%%%%%%%%%%%%%%%%%%%



\subsection*{1.3. The Ideal of a Set of Points \\}
\addcontentsline{toc}{subsection}{1.3. The Ideal of a Set of Points}



\subsubsection*{Problem 1.18.*}
\addcontentsline{toc}{subsubsection}{Problem 1.18.*}
\emph{Let $I$ be an ideal in a ring $R$.
If $a^n \in I$, $b^m \in I$, show that $(a+b)^{n+m} \in I$.
Show that $\mathrm{rad}(I)$ is an ideal, in fact a radical ideal.
Show that any prime ideal is radical.} \\

\emph{Proof.}
\begin{enumerate}
\item[(1)]
  \emph{Show that $(a+b)^{n+m} \in I$ if $a^n \in I$, $b^m \in I$.}
  By the binomial theorem,
  \[
    (a+b)^{n+m}=\sum_{i=0}^{n+m} a^i b^{n+m-i}.
  \]
  For each term $a^i b^{n+m-i}$, either $i \geq n$ holds or $n+m-i \geq m$ holds,
  and thus $a^i b^{n+m-i} \in I$ (since $a^n \in I$, $b^m \in I$ and $I$ is an ideal).
  Hence, the result is established.

\item[(2)]
  \emph{Show that $\mathrm{rad}(I)$ is an ideal.}
  \begin{enumerate}
  \item[(a)]
    $0 \in \mathrm{rad}(I)$ since $0 = 0^{1} \in I$ for any ideal in $R$.

  \item[(b)]
    $(a+b)^{n+m} \in I$ if $a^n \in I$, $b^m \in I$ by (1).

  \item[(c)]
    $(-a)^{2n} = (a^n)^2 \in I$ if $a^n \in I$ (since $I$ is an ideal).

  \item[(d)]
    $(ra)^n = r^n a^n \in I$ if $a^n \in I$ and $r \in R$ (since $I$ is an ideal and $R$ is commutative).
  \end{enumerate}

\item[(3)]
  \emph{Show that $\mathrm{rad}(\mathrm{rad}(I)) = \mathrm{rad}(I)$.}
  It suffices to show $\mathrm{rad}(\mathrm{rad}(I)) \subseteq \mathrm{rad}(I)$.
  Given any $a \in \mathrm{rad}(\mathrm{rad}(I))$.
  By definition $a^n \in \mathrm{rad}(I)$ for some positive integer $n$.
  Again by definition $(a^n)^m = a^{nm} \in I$ for some positive integer $m$.
  As $nm$ is a postive integer, $a \in \mathrm{rad}(I)$.

\item[(4)]
  \emph{Show that every prime ideal $\mathfrak{p}$ is radical.}
  Given any $a \in \mathrm{rad}(\mathfrak{p})$, that is,
  $a^n \in \mathfrak{p}$ for some positive integer.
  Write $a^n = a a^{n-1}$ if $n > 1$.
  By the primality of $\mathfrak{p}$, $a \in \mathfrak{p}$ or $a^{n-1} \in \mathfrak{p}$.
  If $a \in \mathfrak{p}$, we are done.
  If $a^{n-1} \in \mathfrak{p}$,
  we continue this descending argument (or the mathematical induction)
  until the power of $a$ is equal to $1$.
  Hence $\mathfrak{p}$ is radical.
\end{enumerate}
$\Box$\\\\



%%%%%%%%%%%%%%%%%%%%%%%%%%%%%%%%%%%%%%%%%%%%%%%%%%%%%%%%%%%%%%%%%%%%%%%%%%%%%%%%



\subsubsection*{Problem PLACEHOLDER}
\addcontentsline{toc}{subsubsection}{Problem PLACEHOLDER}
\emph{PLACEHOLDER} \\

\emph{Proof.}
\begin{enumerate}
\item[(1)]
  PLACEHOLDER
\end{enumerate}



%%%%%%%%%%%%%%%%%%%%%%%%%%%%%%%%%%%%%%%%%%%%%%%%%%%%%%%%%%%%%%%%%%%%%%%%%%%%%%%%



\subsection*{1.4. The Hilbert Basis Theorem \\}
\addcontentsline{toc}{subsection}{1.4. The Hilbert Basis Theorem}



%%%%%%%%%%%%%%%%%%%%%%%%%%%%%%%%%%%%%%%%%%%%%%%%%%%%%%%%%%%%%%%%%%%%%%%%%%%%%%%%



\subsection*{1.5. Irreducible Components of an Algebraic Set \\}
\addcontentsline{toc}{subsection}{1.5. Irreducible Components of an Algebraic Set}



%%%%%%%%%%%%%%%%%%%%%%%%%%%%%%%%%%%%%%%%%%%%%%%%%%%%%%%%%%%%%%%%%%%%%%%%%%%%%%%%



\subsection*{1.6. Algebraic Subsets of the Plane \\}
\addcontentsline{toc}{subsection}{1.6. Algebraic Subsets of the Plane}



%%%%%%%%%%%%%%%%%%%%%%%%%%%%%%%%%%%%%%%%%%%%%%%%%%%%%%%%%%%%%%%%%%%%%%%%%%%%%%%%



\subsection*{1.7. Hilbert's Nullstellensatz \\}
\addcontentsline{toc}{subsection}{1.7. Hilbert's Nullstellensatz}



%%%%%%%%%%%%%%%%%%%%%%%%%%%%%%%%%%%%%%%%%%%%%%%%%%%%%%%%%%%%%%%%%%%%%%%%%%%%%%%%



\subsection*{1.8. Modules; Finiteness Conditions \\}
\addcontentsline{toc}{subsection}{1.8. Modules; Finiteness Conditions}



\subsubsection*{Problem 1.41.*}
\addcontentsline{toc}{subsubsection}{Problem 1.41.*}
\emph{If $S$ is module-finite over $R$, then $S$ is ring-finite over $R$.} \\

\emph{Proof.}
\begin{enumerate}
\item[(1)]
  $S = \sum R s_i$ for some $s_1, \ldots, s_n \in S$
  since $S$ is module-finite over $R$.

\item[(2)]
  Let $I$ be the minimal subset of $\{ s_1, \ldots, s_n \}$ which also spans $S$,
  say $\{ t_1, \ldots, t_m \}$ with $m \leq n$.
  Clearly we can write
  \[
    S = R[t_1, \ldots, t_m],
  \]
  that is, $S$ is ring-finite over $R$.

\item[(3)]
  The converse is not true (Problem 1.42).
\end{enumerate}
$\Box$\\\\



%%%%%%%%%%%%%%%%%%%%%%%%%%%%%%%%%%%%%%%%%%%%%%%%%%%%%%%%%%%%%%%%%%%%%%%%%%%%%%%%



\subsubsection*{Problem 1.42.}
\addcontentsline{toc}{subsubsection}{Problem 1.42.}
\emph{Show that $S = R[x]$ (the ring of polynomials in one variable)
is ring-finite over $R$, but not module-finite.} \\

\emph{Proof.}
\begin{enumerate}
\item[(1)]
  $S = R[x]$ is ring-finite over $R$ by definition (as $x \in S$).

\item[(2)]
  (Reductio ad absurdum)
  If $S = \sum R s_i$ for some $s_1, \ldots, s_n \in S$ were module-finite over $R$.
  Any element $s \in \sum R s_i$ is of degree
  \[
    \deg s \leq \max_{1 \leq i \leq n} \deg s_i := m.
  \]
  So that $x^{m+1} \in S = R[x]$ but not in $\sum R s_i$,
  which is absurd.
\end{enumerate}
$\Box$\\\\



%%%%%%%%%%%%%%%%%%%%%%%%%%%%%%%%%%%%%%%%%%%%%%%%%%%%%%%%%%%%%%%%%%%%%%%%%%%%%%%%



\subsubsection*{Problem 1.43.* (WIP)}
\addcontentsline{toc}{subsubsection}{Problem 1.43.* (WIP)}
\emph{If $L$ is ring-finite over $K$ ($K$, $L$ fields)
then $L$ is a finitely generated field extension of $K$.} \\

\emph{Proof.}
\begin{enumerate}
\item[(1)]
  $L = K[v_1, \cdots, v_n]$ for some $v_i \in L$.
  To show $L = K[v_1, \cdots, v_n] = K(v_1, \cdots, v_n)$,
  it suffices to show that all $v_i$ are algebraic over $L$.

\item[(2)]
\end{enumerate}
$\Box$\\\\



%%%%%%%%%%%%%%%%%%%%%%%%%%%%%%%%%%%%%%%%%%%%%%%%%%%%%%%%%%%%%%%%%%%%%%%%%%%%%%%%



\subsection*{1.9. Integral Elements \\}
\addcontentsline{toc}{subsection}{1.9. Integral Elements}



%%%%%%%%%%%%%%%%%%%%%%%%%%%%%%%%%%%%%%%%%%%%%%%%%%%%%%%%%%%%%%%%%%%%%%%%%%%%%%%%



\subsection*{1.10. Field Extensions \\}
\addcontentsline{toc}{subsection}{1.10. Field Extensions}



%%%%%%%%%%%%%%%%%%%%%%%%%%%%%%%%%%%%%%%%%%%%%%%%%%%%%%%%%%%%%%%%%%%%%%%%%%%%%%%%
%%%%%%%%%%%%%%%%%%%%%%%%%%%%%%%%%%%%%%%%%%%%%%%%%%%%%%%%%%%%%%%%%%%%%%%%%%%%%%%%



\newpage
\section*{Chapter 2: Affine Varieties \\}
\addcontentsline{toc}{section}{Chapter 2: Affine Varieties}



\subsection*{2.1. Coordinate Rings \\}
\addcontentsline{toc}{subsection}{2.1. Coordinate Rings}



\subsubsection*{Problem 2.1.*}
\addcontentsline{toc}{subsubsection}{Problem 2.1.*}
\emph{Show that the map which associates to each
$f \in k[x_1,\ldots,x_n]$ a polynomial function in $\mathscr{F}(V,k)$
is a ring homomorphism whose kernel is $I(V)$.} \\

\emph{Proof.}
\begin{enumerate}
\item[(1)]
  Define a map $\alpha: k[x_1,\ldots,x_n] \to \mathscr{F}(V,k)$.
  Every polynomial $f \in k[x_1,\ldots,x_n]$ defines a function
  from $V$ to $k$ by
  \[
    \alpha(f)(a_1,\ldots,a_n) = f(a_1,\ldots,a_n)
  \]
  for all $(a_1,\ldots,a_n) \in V$.

\item[(2)]
  $\alpha$ is a ring homomorphism by construction in (1).

\item[(3)]
  \emph{Show that $\mathrm{ker}(\alpha) = I(V)$.}
  In fact,
  given any $f \in k[x_1,\ldots,x_n]$, we have
  $\alpha(f) = 0$ (sending all $a \in V$ to $0 \in k$)
  if and only if $f(a) = 0$ for all $a \in V$
  if and only if $f \in I(V)$.

\item[(4)]
  Hence $k[x_1,\ldots,x_n]/I(V) = \Gamma(V) \hookrightarrow \mathscr{F}(V,k)$
  is an injective homomorphism.
\end{enumerate}
$\Box$\\\\



%%%%%%%%%%%%%%%%%%%%%%%%%%%%%%%%%%%%%%%%%%%%%%%%%%%%%%%%%%%%%%%%%%%%%%%%%%%%%%%%



\subsubsection*{Problem PLACEHOLDER}
\addcontentsline{toc}{subsubsection}{Problem PLACEHOLDER}
\emph{PLACEHOLDER} \\

\emph{Proof.}
\begin{enumerate}
\item[(1)]
  PLACEHOLDER
\end{enumerate}



%%%%%%%%%%%%%%%%%%%%%%%%%%%%%%%%%%%%%%%%%%%%%%%%%%%%%%%%%%%%%%%%%%%%%%%%%%%%%%%%



\subsection*{2.2. Polynomial Maps \\}
\addcontentsline{toc}{subsection}{2.2. Polynomial Maps}



%%%%%%%%%%%%%%%%%%%%%%%%%%%%%%%%%%%%%%%%%%%%%%%%%%%%%%%%%%%%%%%%%%%%%%%%%%%%%%%%



\subsection*{2.3. Coordinate Changes \\}
\addcontentsline{toc}{subsection}{2.3. Coordinate Changes}



%%%%%%%%%%%%%%%%%%%%%%%%%%%%%%%%%%%%%%%%%%%%%%%%%%%%%%%%%%%%%%%%%%%%%%%%%%%%%%%%



\subsection*{2.4. Rational Functions and Local Rings \\}
\addcontentsline{toc}{subsection}{2.4. Rational Functions and Local Rings}



%%%%%%%%%%%%%%%%%%%%%%%%%%%%%%%%%%%%%%%%%%%%%%%%%%%%%%%%%%%%%%%%%%%%%%%%%%%%%%%%



\subsection*{2.5. Discrete Valuation Rings \\}
\addcontentsline{toc}{subsection}{2.5. Discrete Valuation Rings}



%%%%%%%%%%%%%%%%%%%%%%%%%%%%%%%%%%%%%%%%%%%%%%%%%%%%%%%%%%%%%%%%%%%%%%%%%%%%%%%%



\subsection*{2.6. Forms \\}
\addcontentsline{toc}{subsection}{2.6. Forms}



%%%%%%%%%%%%%%%%%%%%%%%%%%%%%%%%%%%%%%%%%%%%%%%%%%%%%%%%%%%%%%%%%%%%%%%%%%%%%%%%



\subsection*{2.7. Direct Products of Rings \\}
\addcontentsline{toc}{subsection}{2.7. Direct Products of Rings}



%%%%%%%%%%%%%%%%%%%%%%%%%%%%%%%%%%%%%%%%%%%%%%%%%%%%%%%%%%%%%%%%%%%%%%%%%%%%%%%%



\subsection*{2.8. Operations with Ideals \\}
\addcontentsline{toc}{subsection}{2.8. Operations with Ideals}



\subsubsection*{Problem 2.39.*}
\addcontentsline{toc}{subsubsection}{Problem 2.39.*}
\emph{Prove the following relations among ideals $I_i$, $J$ in a ring $R$:} \\
\begin{enumerate}
\item[(a)]
  $(I_1 + I_2) J = I_1 J + I_2 J$.

\item[(b)]
  $(I_1 \cdots I_N)^n = I_1^n \cdots I_N^n$. \\
\end{enumerate}



\emph{Proof of (a).}
\begin{enumerate}
\item[(1)]
  Note that $(I_1 + I_2) J$ and $I_1 J + I_2 J$ are ideals.

\item[(2)]
  \emph{Show that $(I_1 + I_2) J \subseteq I_1 J + I_2 J$.}
  Given any
  \[
    (x_{1} + x_{2}) y \in (I_1 + I_2) J
  \]
  where $x_{i} \in I_i$ and $y \in J$.
  It suffices to show that $(x_{1} + x_{2}) y \in I_1 J + I_2 J$ (by (1)).
  In fact,
  \[
    (x_{1} + x_{2}) y = x_{1} y + x_{2} y \in I_1 J + I_2 J.
  \]

\item[(3)]
  \emph{Show that $(I_1 + I_2) J \supseteq I_1 J + I_2 J$.}
  Given any
  \[
    x_{1} y_{1} + x_{2} y_{2} \in I_1 J + I_2 J
  \]
  where $x_{i} \in I_i$ and $y_{i} \in J$.
  It suffices to show that $x_{1} y_{1} + x_{2} y_{2} \in (I_1 + I_2) J$ (by (1)).
  In fact,
  \[
    x_{1} y_{1} + x_{2} y_{2}
    = (x_{1}+\underbrace{0}_{\in I_2}) y_{1} + (\underbrace{0}_{\in I_1}+x_{2}) y_{2}
    \in (I_1 + I_2) J
  \]
  since $(I_1 + I_2) J$ is an ideal.
\end{enumerate}
$\Box$\\



\emph{Proof of (b).}
\begin{enumerate}
\item[(1)]
  Note that $(I_1 \cdots I_N)^n$ and $I_1^n \cdots I_N^n$ are ideals.

\item[(2)]
  \emph{Show that $(I_1 \cdots I_N)^n \subseteq I_1^n \cdots I_N^n$.}
  Given any
  \[
    x = x_1 \cdots x_n
  \]
  where $x_i \in I_1 \cdots I_N$.
  It suffices to show that $x \in I_1^n \cdots I_N^n$ (by (1)).
  For each $x_i \in I_1 \cdots I_N$, write
  \[
    x_i = \sum_{j(i)} x_{j(i),1} \cdots x_{j(i),N}
  \]
  where $x_{j(i),k} \in I_k$ for $1 \leq k \leq N$.
  Hence
  \begin{align*}
    x
    &= x_1 \cdots x_n \\
    &= \left(\sum_{j(1)} x_{j(1),1} \cdots x_{j(1),N}\right)
      \cdots
      \left(\sum_{j(n)} x_{j(n),1} \cdots x_{j(n),N}\right) \\
    &= \sum_{j(1),\ldots,j(n)} (x_{j(1),1} \cdots x_{j(1),N})
      \cdots (x_{j(n),1} \cdots x_{j(n),N}) \\
    &= \sum_{j(1),\ldots,j(n)}
      (\underbrace{x_{j(1),1} \cdots x_{j(n),1}}_{\in I_1^n})
      \cdots
      (\underbrace{x_{j(1),N} \cdots x_{j(n),N}}_{\in I_N^n}) \\
    &\in I_1^n \cdots I_N^n.
  \end{align*}

\item[(3)]
  \emph{Show that $(I_1 \cdots I_N)^n \supseteq I_1^n \cdots I_N^n$.}
  Given any
  \[
    x = x_1 \cdots x_N \in I_1^n \cdots I_N^n
  \]
  where $x_i \in I_i^n$ ($1 \leq i \leq N$).
  It suffices to show that
  $x \in (I_1 \cdots I_N)^n$ (by (1)).
  For each $x_i \in I_i^n$, write
  \[
    x_i = \sum_{j(i)} x_{j(i),1} \cdots x_{j(i),n}
  \]
  where $x_{j(i),k} \in I_i$ for $1 \leq k \leq n$.
  Hence
  \begin{align*}
    x
    &= x_1 \cdots x_N \\
    &= \left(\sum_{j(1)} x_{j(1),1} \cdots x_{j(1),n}\right)
      \cdots
      \left(\sum_{j(N)} x_{j(N),1} \cdots x_{j(N),n}\right) \\
    &= \sum_{j(1),\ldots,j(N)} (x_{j(1),1} \cdots x_{j(1),n})
      \cdots (x_{j(N),1} \cdots x_{j(N),n}) \\
    &= \sum_{j(1),\ldots,j(N)}
      (\underbrace{x_{j(1),1} \cdots x_{j(N),1}}_{\in I_1 \cdots I_N})
      \cdots
      (\underbrace{x_{j(1),n} \cdots x_{j(N),n}}_{\in I_1 \cdots I_N}) \\
    &\in (I_1 \cdots I_N)^n.
  \end{align*}
\end{enumerate}
$\Box$\\\\



%%%%%%%%%%%%%%%%%%%%%%%%%%%%%%%%%%%%%%%%%%%%%%%%%%%%%%%%%%%%%%%%%%%%%%%%%%%%%%%%



\subsubsection*{Problem 2.41.*}
\addcontentsline{toc}{subsubsection}{Problem 2.41.*}
\emph{Let $I$, $J$ be ideals in $R$.
Suppose $I$ is finitely generated and $I \subseteq \mathrm{rad}(J)$.
Show that $I^n \subseteq J$ for some $n$.} \\

\emph{Proof.}
\begin{enumerate}
\item[(1)]
  Let $I$ be generated by $x_1,\ldots,x_m \in I$.
  As $I \subseteq \mathrm{rad}(J)$, there are integers $n_i > 0$
  such that $x_i^{n_i} \in J$.

\item[(2)]
  Let $N = n_1 + \cdots + n_m$.
  Given any $x = \sum_{i=1}^{m} r_i x_i \in I$,
  so
  \begin{align*}
    x^N
    &= \left( \sum_{i=1}^{m} r_i x_i \right)^{N} \\
    &= \sum_{k_1 + \cdots + k_m = N} {N \choose k_1,\ldots,k_m}
      r_1^{k_1} x_1^{k_1} \cdots r_m^{k_m} x_m^{k_m}.
  \end{align*}

\item[(3)]
  Note that for each term there is some $j$ such that $k_j \geq n_j$.
  Hence,
  \begin{align*}
    & \: x_j^{k_j} = x_j^{k_j-n_j} x_j^{n_j} \in J
      &\text{($J$ is an ideal)} \\
    \Longrightarrow& \:
    r_1^{k_1} x_1^{k_1} \cdots r_m^{k_m} x_m^{k_m} \in J \text{ for each term }
      &\text{($J$ is an ideal)} \\
    \Longrightarrow& \:
    x^N \in J.
      &\text{($J$ is an ideal)} \\
    \Longrightarrow& \:
      I^N \subseteq J.
  \end{align*}
\end{enumerate}
$\Box$\\\\



\textbf{Supplement.}
\emph{(Exercise 1.13 in the textbook:
Eisenbud, Commutative Algebra with a View Toward Algebraic Geometry.)}
\emph{Suppose that $I$ is an ideal in a commutative ring.
Show that if $\mathrm{rad}(I)$ is finitely generated,
then for some integer $N$ we have $(\mathrm{rad}(I))^N \subseteq I$.
Conclude that in a Noetherian ring the ideals $I$ and $J$ have the same radical
iff there is some integer $N$ such that $I^N \subseteq J$ and $J^N \subseteq I$.
Use the Nullstellensatz to deduce that if $I, J \subseteq S = k[x_1,\ldots,x_n]$
are ideals and $k$ is algebraically closed,
then $Z(I) = Z(J)$ iff $I^N \subseteq J$ and $J^N \subseteq I$ for some $N$.} \\

\emph{Proof.}
\begin{enumerate}
  \item[(1)]
  \emph{Show that if $\mathrm{rad}(I)$ is finitely generated,
  then for some integer $N$ we have $(\mathrm{rad}(I))^N \subseteq I$.}
  Say $x_1, \ldots, x_m \in \mathrm{rad}(I)$ generate $\mathrm{rad}(I)$.
  \begin{enumerate}
    \item[(a)]
    For each $i$, there exists an integer $n_i > 0$ such that $x_i^{n_i} \in I$
    (since $\mathrm{rad}(I)$ is radical).
    \item[(b)]
    Let $N = n_1 + \cdots + n_m$.
    Given any $x = \sum_{i=1}^{m} r_i x_i \in \mathrm{rad}(I)$,
    so
    \begin{align*}
      x^N
      &= \left( \sum_{i=1}^{m} r_i x_i \right)^{N} \\
      &= \sum_{k_1 + \cdots + k_m = N} {N \choose k_1,\ldots,k_m}
        r_1^{k_1} x_1^{k_1} \cdots r_m^{k_m} x_m^{k_m}.
    \end{align*}
    \item[(c)]
    Note that for each term there is some $j$ such that $k_j \geq n_j$.
    Hence,
    \begin{align*}
      & \: x_j^{k_j} = x_j^{k_j-n_j} x_j^{n_j} \in I
        &\text{($I$ is an ideal)} \\
      \Longrightarrow& \:
      r_1^{k_1} x_1^{k_1} \cdots r_m^{k_m} x_m^{k_m} \in I \text{ for each term }
        &\text{($I$ is an ideal)} \\
      \Longrightarrow& \:
      x^N \in I.
        &\text{($I$ is an ideal)} \\
      \Longrightarrow& \:
        (\mathrm{rad}(I))^N \subseteq I.
    \end{align*}
  \end{enumerate}
  \item[(2)]
  \emph{Show that in a Noetherian ring the ideals $I$ and $J$ have the same radical
  iff there is some integer $N$ such that $I^N \subseteq J$ and $J^N \subseteq I$.}
  \begin{enumerate}
    \item[(a)]
    $(\Longrightarrow)$
    Since in a Noetherian ring every ideal is finitely generated,
    $\mathrm{rad}(I)$ and $\mathrm{rad}(J)$ are finitely generated.
    By (1), there is a common integer $N$ such that
    \[
      (\mathrm{rad}(I))^N \subseteq I \:\: \text{ and } \:\:
      (\mathrm{rad}(J))^N \subseteq J.
    \]
    Note that $I^N \subseteq (\mathrm{rad}(I))^N$ and $J^N \subseteq (\mathrm{rad}(J))^N$.
    Since $\mathrm{rad}(I)$ = $\mathrm{rad}(J)$ by assumption,
    \begin{align*}
      I^N &\subseteq (\mathrm{rad}(I))^N = (\mathrm{rad}(J))^N \subseteq J, \\
      J^N &\subseteq (\mathrm{rad}(J))^N = (\mathrm{rad}(I))^N \subseteq I.
    \end{align*}
    \item[(b)]
    $(\Longleftarrow)$
    It suffices to show that $\mathrm{rad}(I) \subseteq \mathrm{rad}(J)$.
    $\mathrm{rad}(J) \subseteq \mathrm{rad}(I)$ is similar.
    Given any $x \in \mathrm{rad}(I)$, there is an integer $M > 0$ such that
    $x^M \in I$.
    Hence $x^{MN} \in I^N \subseteq J$, or $x \in \mathrm{rad}(J)$.
  \end{enumerate}
  \item[(3)]
  \emph{Show that if $I, J \subseteq S = k[x_1,\ldots,x_n]$
  are ideals and $k$ is algebraically closed,
  then $Z(I) = Z(J)$ iff $I^N \subseteq J$ and $J^N \subseteq I$ for some $N$.}
  Note that $S$ is Noetherian and we can apply part (2).
  By the Nullstellensatz, $Z(I) = Z(J)$ iff $\mathrm{rad}(I) = \mathrm{rad}(J)$
  iff $I^N \subseteq J$ and $J^N \subseteq I$ for some $N$.
\end{enumerate}
$\Box$ \\\\



%%%%%%%%%%%%%%%%%%%%%%%%%%%%%%%%%%%%%%%%%%%%%%%%%%%%%%%%%%%%%%%%%%%%%%%%%%%%%%%%



\subsection*{2.9. Ideals with a Finite Number of Zeros \\}
\addcontentsline{toc}{subsection}{2.9. Ideals with a Finite Number of Zeros}



%%%%%%%%%%%%%%%%%%%%%%%%%%%%%%%%%%%%%%%%%%%%%%%%%%%%%%%%%%%%%%%%%%%%%%%%%%%%%%%%



\subsection*{2.10. Quotient Modules and Exact Sequences \\}
\addcontentsline{toc}{subsection}{2.10. Quotient Modules and Exact Sequences}



\subsubsection*{Problem 2.51.}
\addcontentsline{toc}{subsubsection}{Problem 2.51.}
\emph{Let
\[
  0 \longrightarrow V_1 \longrightarrow \cdots \longrightarrow V_n \longrightarrow 0
\]
be an exact sequence of finite-dimensional vector spaces.
Show that $\sum (-1)^i \dim(V_i) = 0$.} \\

\emph{Proof (Proposition 7 in this section).}
\begin{enumerate}
\item[(1)]
  For $i = 0,\ldots,n$,
  by the rank-nullity theorem for a linear transformation
  $\varphi_{i}: V_{i} \to V_{i+1}$, we have
  \[
    \dim V_{i} = \dim \mathrm{im}(\varphi_{i}) + \dim \mathrm{ker}(\varphi_{i}).
  \]
  (Here $V_0 = V_{n+1} := 0$ by convention.)

\item[(2)]
  By the exactness of the sequence, we have
  \begin{enumerate}
  \item[(a)]
    $\mathrm{im}(\varphi_{i}) = \mathrm{ker}(\varphi_{i+1})$ for $i = 0,\ldots, n-1$.
    In particular, $\mathrm{ker}(\varphi_{1}) = \mathrm{im}(\varphi_{0}) = 0$.

  \item[(b)]
    $\mathrm{ker}(\varphi_{n}) = V_n$.
  \end{enumerate}
  Hence,
  \begin{align*}
    \sum_{i=1}^{n-1} (-1)^i \dim(V_i)
    &= \sum_{i=1}^{n-1} (-1)^i \dim \mathrm{im}(\varphi_{i})
      + \sum_{i=1}^{n-1} (-1)^i \dim \mathrm{ker}(\varphi_{i}) \\
    &= \sum_{i=1}^{n-1} (-1)^i \dim \mathrm{ker}(\varphi_{i+1})
      + \sum_{i=1}^{n-1} (-1)^i \dim \mathrm{ker}(\varphi_{i}) \\
    &= (-1)^{n-1} \dim \underbrace{\mathrm{ker}(\varphi_{n})}_{=V_n}
      + (-1)^1 \dim \underbrace{\mathrm{ker}(\varphi_{1})}_{= 0} \\
    &= -(-1)^n \dim V_n,
  \end{align*}
  or $\sum (-1)^i \dim(V_i) = 0$.
\end{enumerate}
$\Box$\\\\



%%%%%%%%%%%%%%%%%%%%%%%%%%%%%%%%%%%%%%%%%%%%%%%%%%%%%%%%%%%%%%%%%%%%%%%%%%%%%%%%



\subsection*{2.11. Free Modules \\}
\addcontentsline{toc}{subsection}{2.11. Free Modules}



%%%%%%%%%%%%%%%%%%%%%%%%%%%%%%%%%%%%%%%%%%%%%%%%%%%%%%%%%%%%%%%%%%%%%%%%%%%%%%%%
%%%%%%%%%%%%%%%%%%%%%%%%%%%%%%%%%%%%%%%%%%%%%%%%%%%%%%%%%%%%%%%%%%%%%%%%%%%%%%%%



\newpage
\section*{Chapter 3: Local Properties of Plane Curves \\}
\addcontentsline{toc}{section}{Chapter 3: Local Properties of Plane Curves}



\subsection*{3.1. Multiple Points and Tangent Lines \\}
\addcontentsline{toc}{subsection}{3.1. Multiple Points and Tangent Lines}



\subsubsection*{Problem PLACEHOLDER}
\addcontentsline{toc}{subsubsection}{Problem PLACEHOLDER}
\emph{PLACEHOLDER} \\

\emph{Proof.}
\begin{enumerate}
\item[(1)]
  PLACEHOLDER
\end{enumerate}
$\Box$\\\\



%%%%%%%%%%%%%%%%%%%%%%%%%%%%%%%%%%%%%%%%%%%%%%%%%%%%%%%%%%%%%%%%%%%%%%%%%%%%%%%%



\subsection*{3.2. Multiplicities and Local Rings \\}
\addcontentsline{toc}{subsection}{3.2. Multiplicities and Local Rings}



%%%%%%%%%%%%%%%%%%%%%%%%%%%%%%%%%%%%%%%%%%%%%%%%%%%%%%%%%%%%%%%%%%%%%%%%%%%%%%%%



\subsection*{3.3. Intersection Numbers \\}
\addcontentsline{toc}{subsection}{3.3. Intersection Numbers}



%%%%%%%%%%%%%%%%%%%%%%%%%%%%%%%%%%%%%%%%%%%%%%%%%%%%%%%%%%%%%%%%%%%%%%%%%%%%%%%%
%%%%%%%%%%%%%%%%%%%%%%%%%%%%%%%%%%%%%%%%%%%%%%%%%%%%%%%%%%%%%%%%%%%%%%%%%%%%%%%%



\newpage
\section*{Chapter 4: Projective Varieties \\}
\addcontentsline{toc}{section}{Chapter 4: Projective Varieties}



\subsection*{4.1. Projective Space \\}
\addcontentsline{toc}{subsection}{4.1. Projective Space}



\subsubsection*{Problem PLACEHOLDER}
\addcontentsline{toc}{subsubsection}{Problem PLACEHOLDER}
\emph{PLACEHOLDER} \\

\emph{Proof.}
\begin{enumerate}
\item[(1)]
  PLACEHOLDER
\end{enumerate}
$\Box$\\\\



%%%%%%%%%%%%%%%%%%%%%%%%%%%%%%%%%%%%%%%%%%%%%%%%%%%%%%%%%%%%%%%%%%%%%%%%%%%%%%%%



\subsection*{4.2. Projective Algebraic Sets \\}
\addcontentsline{toc}{subsection}{4.2. Projective Algebraic Sets}



%%%%%%%%%%%%%%%%%%%%%%%%%%%%%%%%%%%%%%%%%%%%%%%%%%%%%%%%%%%%%%%%%%%%%%%%%%%%%%%%



\subsection*{4.3. Affine and Projective Varieties \\}
\addcontentsline{toc}{subsection}{4.3. Affine and Projective Varieties}



%%%%%%%%%%%%%%%%%%%%%%%%%%%%%%%%%%%%%%%%%%%%%%%%%%%%%%%%%%%%%%%%%%%%%%%%%%%%%%%%



\subsection*{4.4. Multiprojective Space \\}
\addcontentsline{toc}{subsection}{4.4. Multiprojective Space}



%%%%%%%%%%%%%%%%%%%%%%%%%%%%%%%%%%%%%%%%%%%%%%%%%%%%%%%%%%%%%%%%%%%%%%%%%%%%%%%%
%%%%%%%%%%%%%%%%%%%%%%%%%%%%%%%%%%%%%%%%%%%%%%%%%%%%%%%%%%%%%%%%%%%%%%%%%%%%%%%%



\newpage
\section*{Chapter 5: Projective Plane Curves\\}
\addcontentsline{toc}{section}{Chapter 5: Projective Plane Curves}



\subsection*{5.1. Definitions \\}
\addcontentsline{toc}{subsection}{5.1. Definitions}



\subsubsection*{Problem PLACEHOLDER}
\addcontentsline{toc}{subsubsection}{Problem PLACEHOLDER}
\emph{PLACEHOLDER} \\

\emph{Proof.}
\begin{enumerate}
\item[(1)]
  PLACEHOLDER
\end{enumerate}
$\Box$\\\\



%%%%%%%%%%%%%%%%%%%%%%%%%%%%%%%%%%%%%%%%%%%%%%%%%%%%%%%%%%%%%%%%%%%%%%%%%%%%%%%%



\subsection*{5.2. Linear Systems of Curves \\}
\addcontentsline{toc}{subsection}{5.2. Linear Systems of Curves}



%%%%%%%%%%%%%%%%%%%%%%%%%%%%%%%%%%%%%%%%%%%%%%%%%%%%%%%%%%%%%%%%%%%%%%%%%%%%%%%%



\subsection*{5.3. B\'ezout's Theorem  \\}
\addcontentsline{toc}{subsection}{5.3. B\'ezout's Theorem}



%%%%%%%%%%%%%%%%%%%%%%%%%%%%%%%%%%%%%%%%%%%%%%%%%%%%%%%%%%%%%%%%%%%%%%%%%%%%%%%%



\subsection*{5.4. Multiple Points \\}
\addcontentsline{toc}{subsection}{5.4. Multiple Points}



%%%%%%%%%%%%%%%%%%%%%%%%%%%%%%%%%%%%%%%%%%%%%%%%%%%%%%%%%%%%%%%%%%%%%%%%%%%%%%%%



\subsection*{5.5. Max Noether's Fundamental Theorem \\}
\addcontentsline{toc}{subsection}{5.5. Max Noether's Fundamental Theorem}



%%%%%%%%%%%%%%%%%%%%%%%%%%%%%%%%%%%%%%%%%%%%%%%%%%%%%%%%%%%%%%%%%%%%%%%%%%%%%%%%



\subsection*{5.6. Applications of Noether's Theorem \\}
\addcontentsline{toc}{subsection}{5.6. Applications of Noether's Theorem}



%%%%%%%%%%%%%%%%%%%%%%%%%%%%%%%%%%%%%%%%%%%%%%%%%%%%%%%%%%%%%%%%%%%%%%%%%%%%%%%%
%%%%%%%%%%%%%%%%%%%%%%%%%%%%%%%%%%%%%%%%%%%%%%%%%%%%%%%%%%%%%%%%%%%%%%%%%%%%%%%%



\newpage
\section*{Chapter 6: Varieties, Morphisms, and Rational Maps \\}
\addcontentsline{toc}{section}{Chapter 6: Varieties, Morphisms, and Rational Maps}

\subsection*{6.1. The Zariski Topology \\}
\addcontentsline{toc}{subsection}{6.1. The Zariski Topology}

\subsection*{6.2. Varieties \\}
\addcontentsline{toc}{subsection}{6.2. Varieties}

\subsection*{6.3. Morphisms of Varieties \\}
\addcontentsline{toc}{subsection}{6.3. Morphisms of Varieties}

\subsection*{6.4. Products and Graphs \\}
\addcontentsline{toc}{subsection}{6.4. Products and Graphs}

\subsection*{6.5. Algebraic Function Fields and Dimension of Varieties \\}
\addcontentsline{toc}{subsection}{6.5. Algebraic Function Fields and Dimension of Varieties}

\subsection*{6.6. Rational Maps \\}
\addcontentsline{toc}{subsection}{6.6. Rational Maps}



%%%%%%%%%%%%%%%%%%%%%%%%%%%%%%%%%%%%%%%%%%%%%%%%%%%%%%%%%%%%%%%%%%%%%%%%%%%%%%%%
%%%%%%%%%%%%%%%%%%%%%%%%%%%%%%%%%%%%%%%%%%%%%%%%%%%%%%%%%%%%%%%%%%%%%%%%%%%%%%%%



\newpage
\section*{Chapter 7: Resolution of Singularities \\}
\addcontentsline{toc}{section}{Chapter 7: Resolution of Singularities}



\subsection*{7.1. Rational Maps of Curves \\}
\addcontentsline{toc}{subsection}{7.1. Rational Maps of Curves}



\subsubsection*{Problem PLACEHOLDER}
\addcontentsline{toc}{subsubsection}{Problem PLACEHOLDER}
\emph{PLACEHOLDER} \\

\emph{Proof.}
\begin{enumerate}
\item[(1)]
  PLACEHOLDER
\end{enumerate}
$\Box$\\\\



%%%%%%%%%%%%%%%%%%%%%%%%%%%%%%%%%%%%%%%%%%%%%%%%%%%%%%%%%%%%%%%%%%%%%%%%%%%%%%%%



\subsection*{7.2. Blowing up a Point in $\mathbf{A}^{2}$ \\}
\addcontentsline{toc}{subsection}{7.2. Blowing up a Point in $\mathbf{A}^{2}$}



%%%%%%%%%%%%%%%%%%%%%%%%%%%%%%%%%%%%%%%%%%%%%%%%%%%%%%%%%%%%%%%%%%%%%%%%%%%%%%%%



\subsection*{7.3. Blowing up a Point in $\mathbf{P}^{2}$ \\}
\addcontentsline{toc}{subsection}{7.3. Blowing up a Point in $\mathbf{P}^{2}$}



%%%%%%%%%%%%%%%%%%%%%%%%%%%%%%%%%%%%%%%%%%%%%%%%%%%%%%%%%%%%%%%%%%%%%%%%%%%%%%%%



\subsection*{7.4. Quadratic Transformations \\}
\addcontentsline{toc}{subsection}{7.4. Quadratic Transformations}



%%%%%%%%%%%%%%%%%%%%%%%%%%%%%%%%%%%%%%%%%%%%%%%%%%%%%%%%%%%%%%%%%%%%%%%%%%%%%%%%



\subsection*{7.5. Nonsingular Models of Curves \\}
\addcontentsline{toc}{subsection}{7.5. Nonsingular Models of Curves}



%%%%%%%%%%%%%%%%%%%%%%%%%%%%%%%%%%%%%%%%%%%%%%%%%%%%%%%%%%%%%%%%%%%%%%%%%%%%%%%%
%%%%%%%%%%%%%%%%%%%%%%%%%%%%%%%%%%%%%%%%%%%%%%%%%%%%%%%%%%%%%%%%%%%%%%%%%%%%%%%%



\newpage
\section*{Chapter 8: Riemann-Roch Theorem \\}
\addcontentsline{toc}{section}{Chapter 8: Riemann-Roch Theorem}




\subsection*{8.1. Divisors \\}
\addcontentsline{toc}{subsection}{8.1. Divisors}



\subsubsection*{Problem PLACEHOLDER}
\addcontentsline{toc}{subsubsection}{Problem PLACEHOLDER}
\emph{PLACEHOLDER} \\

\emph{Proof.}
\begin{enumerate}
\item[(1)]
  PLACEHOLDER
\end{enumerate}
$\Box$\\\\



%%%%%%%%%%%%%%%%%%%%%%%%%%%%%%%%%%%%%%%%%%%%%%%%%%%%%%%%%%%%%%%%%%%%%%%%%%%%%%%%



\subsection*{8.2. The Vector Spaces $L(D)$ \\}
\addcontentsline{toc}{subsection}{8.1. The Vector Spaces $L(D)$}



%%%%%%%%%%%%%%%%%%%%%%%%%%%%%%%%%%%%%%%%%%%%%%%%%%%%%%%%%%%%%%%%%%%%%%%%%%%%%%%%



\subsection*{8.3. Riemann's Theorem \\}
\addcontentsline{toc}{subsection}{8.1. Riemann's Theorem}



%%%%%%%%%%%%%%%%%%%%%%%%%%%%%%%%%%%%%%%%%%%%%%%%%%%%%%%%%%%%%%%%%%%%%%%%%%%%%%%%



\subsection*{8.4. Derivations and Differentials \\}
\addcontentsline{toc}{subsection}{8.1. Derivations and Differentials}



%%%%%%%%%%%%%%%%%%%%%%%%%%%%%%%%%%%%%%%%%%%%%%%%%%%%%%%%%%%%%%%%%%%%%%%%%%%%%%%%



\subsection*{8.5. Canonical Divisors \\}
\addcontentsline{toc}{subsection}{8.1. Canonical Divisors}



%%%%%%%%%%%%%%%%%%%%%%%%%%%%%%%%%%%%%%%%%%%%%%%%%%%%%%%%%%%%%%%%%%%%%%%%%%%%%%%%



\subsection*{8.6. Riemann-Roch Theorem \\}
\addcontentsline{toc}{subsection}{8.6. Riemann-Roch Theorem}



%%%%%%%%%%%%%%%%%%%%%%%%%%%%%%%%%%%%%%%%%%%%%%%%%%%%%%%%%%%%%%%%%%%%%%%%%%%%%%%%
%%%%%%%%%%%%%%%%%%%%%%%%%%%%%%%%%%%%%%%%%%%%%%%%%%%%%%%%%%%%%%%%%%%%%%%%%%%%%%%%



\end{document}