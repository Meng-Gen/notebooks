\documentclass{article}
\usepackage{amsfonts}
\usepackage{amsmath}
\usepackage{amssymb}
\usepackage{hyperref}
\usepackage[none]{hyphenat}
\usepackage{mathrsfs}
\parindent=0pt



\title{\textbf{Solutions to Algebraic Curves}}
\author{Meng-Gen Tsai \\ plover@gmail.com}



\begin{document}
\maketitle
\tableofcontents



%%%%%%%%%%%%%%%%%%%%%%%%%%%%%%%%%%%%%%%%%%%%%%%%%%%%%%%%%%%%%%%%%%%%%%%%%%%%%%%%
%%%%%%%%%%%%%%%%%%%%%%%%%%%%%%%%%%%%%%%%%%%%%%%%%%%%%%%%%%%%%%%%%%%%%%%%%%%%%%%%



\newpage
\section*{Chapter 1: Affine Algebraic Sets \\}
\addcontentsline{toc}{section}{Chapter 1: Affine Algebraic Sets}



%%%%%%%%%%%%%%%%%%%%%%%%%%%%%%%%%%%%%%%%%%%%%%%%%%%%%%%%%%%%%%%%%%%%%%%%%%%%%%%%



\subsection*{1.1. Algebraic Preliminaries \\}
\addcontentsline{toc}{subsection}{1.1. Algebraic Preliminaries}



\subsubsection*{Problem 1.1.*}
\addcontentsline{toc}{subsubsection}{Problem 1.1.*}
\emph{Let $R$ be a domain.}
\begin{enumerate}
\item[(a)]
  \emph{If $f$, $g$ are forms of degree $r$, $s$ respectively in $R[x_1,\ldots,x_n]$,
  show that $fg$ is a form of degree $r+s$.}

\item[(b)]
  \emph{Show that any factor of a form in $R[x_1,\ldots,x_n]$ is also a form. } \\
\end{enumerate}

\emph{Proof of (a).}
\begin{enumerate}
\item[(1)]
  Write
  \begin{align*}
    f &= \sum_{(i)} a_{(i)} x^{(i)}, \\
    g &= \sum_{(j)} b_{(j)} x^{(j)},
  \end{align*}
  where $\sum_{(i)}$ is the summation over $(i) = (i_1,\ldots,i_n)$ with $i_1+\cdots+i_n = r$
  and $\sum_{(j)}$ is the summation over $(j) = (j_1,\ldots,j_n)$ with $j_1+\cdots+j_n = s$.

\item[(2)]
  Hence,
  \begin{align*}
    fg
    &= \sum_{(i)} \sum_{(j)} a_{(i)}b_{(j)} x^{(i)}x^{(j)} \\
    &= \sum_{(i),(j)} a_{(i)}b_{(j)} x^{(k)}
  \end{align*}
  where $(k) = (i_1+j_1,\ldots,i_n+j_n)$ with $(i_1+j_1)+\cdots+(i_n+j_n) = r+s$.
  Each $x^{(k)}$ is the form of degree $r+s$ and $a_{(i)}b_{(j)} \in R$.
  Hence $fg$ is a form of degree $r+s$.
\end{enumerate}
$\Box$\\



\emph{Proof of (b).}
\begin{enumerate}
\item[(1)]
  Given any form $f \in R[x_1,\ldots,x_n]$, and write $f = gh$.
  \emph{It suffices to show that $g$ is a form as well.}
  (So does $h$.)

\item[(2)]
  Write
  \[
    g = g_0+\cdots+g_r,
    \qquad
    h = h_0+\cdots+h_s
  \]
  where $g_r \neq 0$ and $h_s \neq 0$.
  So
  \[
    f = gh = g_0h_0 + \cdots + g_r h_s.
  \]
  Since $R$ is a domain, $R[x_1,\ldots,x_n]$ is a domain and thus $g_r h_s \neq 0$.
  The maximality of $r$ and $s$ implies that $\deg f = r+s$.
  Therefore, by the maximality of $r+s$,
  $f = g_r h_s$, or $g = g_r$, or $g$ is a form.
\end{enumerate}
$\Box$\\\\



%%%%%%%%%%%%%%%%%%%%%%%%%%%%%%%%%%%%%%%%%%%%%%%%%%%%%%%%%%%%%%%%%%%%%%%%%%%%%%%%



\subsubsection*{Problem 1.5.*}
\addcontentsline{toc}{subsubsection}{Problem 1.5.*}
\emph{Let $k$ be any field.
Show that there are an infinitely number of irreducible monic polynomials in $k[x]$.
(Hint: Suppose $f_1,\ldots,f_n$ were all of them, and factor $f_1\cdots f_n+1$ into irreducible factors.)} \\

\emph{Proof (Due to Euclid).}
\begin{enumerate}
\item[(1)]
  If
  $f_1, \ldots, f_n$ were all irreducible monic polynomials, then
  we consider
  \[
    g = f_1 \cdots f_n + 1 \in k[x].
  \]
  So there is an irreducible monic polynomial $f = f_i$ dividing $g$ for some $i$
  since
  \[
    \deg g = \deg f_1 + \cdots + \deg f_n \geq 1.
  \]

\item[(2)]
  However, $f$ would divide the difference
  \[
    g - f_1 \cdots f_{i-1} f_i f_{i+1} \cdots f_n = 1,
  \]
  contrary to $\deg f_i \geq 1$.
\end{enumerate}
$\Box$\\\\



%%%%%%%%%%%%%%%%%%%%%%%%%%%%%%%%%%%%%%%%%%%%%%%%%%%%%%%%%%%%%%%%%%%%%%%%%%%%%%%%



\subsubsection*{Problem 1.6.*}
\addcontentsline{toc}{subsubsection}{Problem 1.6.*}
\emph{Show that any algebraically closed field is infinite.
(Hint: The irreducible monic polynomials are $x - a$, $a \in k$.)} \\

\emph{Proof (Due to Euclid).}
\begin{enumerate}
\item[(1)]
  Let $k$ be an algebraically closed field.
  If $a_1, \ldots, a_n$ were all elements in $k$, then
  we consider a monic polynomials
  \[
    f(x) = (x - a_1) \cdots (x - a_n) + 1 \in k[x].
  \]

\item[(2)]
  Since $k$ is algebraically closed,
  there is an element $a \in k$ such that $f(a) = 0$.
  By assumption, $a = a_i$ for some $1 \leq i \leq n$,
  and thus $f(a) = f(a_i) = 1$, contrary to the fact that
  a field is a commutative ring where $0 \neq 1$ and all nonzero elements are invertible.
\end{enumerate}
$\Box$\\\\



%%%%%%%%%%%%%%%%%%%%%%%%%%%%%%%%%%%%%%%%%%%%%%%%%%%%%%%%%%%%%%%%%%%%%%%%%%%%%%%%
%%%%%%%%%%%%%%%%%%%%%%%%%%%%%%%%%%%%%%%%%%%%%%%%%%%%%%%%%%%%%%%%%%%%%%%%%%%%%%%%



\subsection*{1.2. Affine Space and Algebraic Sets \\}
\addcontentsline{toc}{subsection}{1.2. Affine Space and Algebraic Sets}



%%%%%%%%%%%%%%%%%%%%%%%%%%%%%%%%%%%%%%%%%%%%%%%%%%%%%%%%%%%%%%%%%%%%%%%%%%%%%%%%



\subsubsection*{Problem 1.8.*}
\addcontentsline{toc}{subsubsection}{Problem 1.8.*}
\emph{Show that the algebraic subsets of $\mathbb{A}^1(k)$ are just the finite subsets, together
with $\mathbb{A}^1(k)$ itself.} \\

\emph{Proof.}
\begin{enumerate}
\item[(1)]
  \emph{Show that $k[x]$ is a PID if $k$ is a field.}
  \begin{enumerate}
  \item[(a)]
    Let $I$ be an ideal of $k[x]$.

  \item[(b)]
    If $I = \{0\}$ then $I = (0)$ and $I$ is principal.

  \item[(c)]
    If $I \neq \{0\}$, then take $f$ to be a polynomial of minimal degree in $I$.
    It suffices to show that $I = (f)$.
    Clearly, $(f) \subseteq I$ since $I$ is an ideal.
    Conversely, for any $g \in I$,
    \[
      g(x) = f(x)h(x) + r(x)
    \]
    for some $h, r \in k[x]$ with $r = 0$ or $\deg r < \deg f$.
    Now as
    \[
      r = g - fh \in I,
    \]
    $r = 0$ (otherwise contrary to the minimality of $f$),
    we have $g = fh \in (f)$ for all $g \in I$.
  \end{enumerate}

\item[(2)]
  Let $X$ be an algebraic subset of $\mathbb{A}^1(k)$,
  say $X = V(I)$ for some ideal $I$ of $k[x]$.
  Since $k[x]$ is a PID, $I = (f)$ for some $f \in k[x]$.
  \begin{enumerate}
  \item[(a)]
    If $f = 0$, then $I = (0)$ and $X = V(0) = \mathbb{A}^1(k)$.

  \item[(b)]
    If $f \neq 0$, then $f(x) = 0$ has finitely many roots in $k$,
    say $a_1, \ldots, a_m \in k$.
    Hence,
    \[
      X = V(I) = V(f) = \{ f(a) = 0 : a \in k \}
      = \{ a_1, \ldots, a_m \}
    \]
    is a finite subsets of $X$.
  \end{enumerate}
  By (a)(b), the result is established.
\end{enumerate}
$\Box$\\



\emph{Notes.}
\begin{enumerate}
\item[(1)]
  By the Hilbert basis theorem, $k[x]$ is Noetherian as $k$ is Noetherian.
  Hence, for any algebraic subset $X = V(I)$ of $\mathbb{A}^1(k)$,
  we can write $I = (f_1, \cdots, f_m)$.
  Note that
  \[
    V(I) = V(f_1) \cap \cdots \cap V(f_m).
  \]
  Now apply the same argument to get the same conclusion.

\item[(2)]
  Suppose $k = \overline{k}$.
  $\mathbb{A}^1(k)$ is irreducible, because its only proper closed subsets are finite,
  yet it is infinite
  (because $k$ is algebraically closed, hence infinite). \\
\end{enumerate}



%%%%%%%%%%%%%%%%%%%%%%%%%%%%%%%%%%%%%%%%%%%%%%%%%%%%%%%%%%%%%%%%%%%%%%%%%%%%%%%%
%%%%%%%%%%%%%%%%%%%%%%%%%%%%%%%%%%%%%%%%%%%%%%%%%%%%%%%%%%%%%%%%%%%%%%%%%%%%%%%%


\end{document}