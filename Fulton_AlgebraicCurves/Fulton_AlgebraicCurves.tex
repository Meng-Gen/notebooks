\documentclass{article}
\usepackage{amsfonts}
\usepackage{amsmath}
\usepackage{amssymb}
\usepackage{hyperref}
\usepackage[none]{hyphenat}
\usepackage{mathrsfs}
\usepackage{mathtools}
\usepackage{physics}
\parindent=0pt



\title{\textbf{Solutions to the book: \\\emph{Fulton, Algebraic Curves}}}
\author{Meng-Gen Tsai \\ plover@gmail.com}



\begin{document}
\maketitle
\tableofcontents



%%%%%%%%%%%%%%%%%%%%%%%%%%%%%%%%%%%%%%%%%%%%%%%%%%%%%%%%%%%%%%%%%%%%%%%%%%%%%%%%
%%%%%%%%%%%%%%%%%%%%%%%%%%%%%%%%%%%%%%%%%%%%%%%%%%%%%%%%%%%%%%%%%%%%%%%%%%%%%%%%



\newpage
\section*{Chapter 1: Affine Algebraic Sets \\}
\addcontentsline{toc}{section}{Chapter 1: Affine Algebraic Sets}



\subsection*{1.1. Algebraic Preliminaries \\}
\addcontentsline{toc}{subsection}{1.1. Algebraic Preliminaries}



\subsubsection*{Problem 1.1.*}
\addcontentsline{toc}{subsubsection}{Problem 1.1.*}
\emph{Let $R$ be a domain.}
\begin{enumerate}
\item[(a)]
  \emph{If $f$, $g$ are forms of degree $r$, $s$ respectively in $R[x_1,\ldots,x_n]$,
  show that $fg$ is a form of degree $r+s$.}

\item[(b)]
  \emph{Show that any factor of a form in $R[x_1,\ldots,x_n]$ is also a form. } \\
\end{enumerate}

\emph{Proof of (a).}
\begin{enumerate}
\item[(1)]
  Write
  \begin{align*}
    f &= \sum_{(i)} a_{(i)} x^{(i)}, \\
    g &= \sum_{(j)} b_{(j)} x^{(j)},
  \end{align*}
  where $\sum_{(i)}$ is the summation over $(i) = (i_1,\ldots,i_n)$ with $i_1+\cdots+i_n = r$
  and $\sum_{(j)}$ is the summation over $(j) = (j_1,\ldots,j_n)$ with $j_1+\cdots+j_n = s$.

\item[(2)]
  Hence,
  \begin{align*}
    fg
    &= \sum_{(i)} \sum_{(j)} a_{(i)}b_{(j)} x^{(i)}x^{(j)} \\
    &= \sum_{(i),(j)} a_{(i)}b_{(j)} x^{(k)}
  \end{align*}
  where $(k) = (i_1+j_1,\ldots,i_n+j_n)$ with $(i_1+j_1)+\cdots+(i_n+j_n) = r+s$.
  Each $x^{(k)}$ is the form of degree $r+s$ and $a_{(i)}b_{(j)} \in R$.
  Hence $fg$ is a form of degree $r+s$.
\end{enumerate}
$\Box$ \\



\emph{Proof of (b).}
\begin{enumerate}
\item[(1)]
  Given any form $f \in R[x_1,\ldots,x_n]$, and write $f = gh$.
  \emph{It suffices to show that $g$ is a form as well.}
  (So does $h$.)

\item[(2)]
  Write
  \[
    g = g_0+\cdots+g_r,
    \qquad
    h = h_0+\cdots+h_s
  \]
  where $g_r \neq 0$ and $h_s \neq 0$.
  So
  \[
    f = gh = g_0h_0 + \cdots + g_r h_s.
  \]
  Since $R$ is a domain, $R[x_1,\ldots,x_n]$ is a domain and thus $g_r h_s \neq 0$.
  The maximality of $r$ and $s$ implies that $\deg f = r+s$.
  Therefore, by the maximality of $r+s$,
  $f = g_r h_s$, or $g = g_r$, or $g$ is a form.
\end{enumerate}
$\Box$ \\\\



%%%%%%%%%%%%%%%%%%%%%%%%%%%%%%%%%%%%%%%%%%%%%%%%%%%%%%%%%%%%%%%%%%%%%%%%%%%%%%%%



\subsubsection*{Problem 1.2.*}
\addcontentsline{toc}{subsubsection}{Problem 1.2.*}
\emph{Let $R$ be a UFD,
$K$ the quotient field of $R$.
Show that every element $z$ of $K$ may be written $z = a/b$,
where $a, b \in R$ have no common factors;
this representative is unique up to units of $R$.} \\

\emph{Proof.}
\begin{enumerate}
\item[(1)]
  \emph{Show that every element $z$ of $K$ may be written $z=a/b$,
  where $a$, $b\in R$ have no common factors.}
  Given any $z = a/b \in K$ where $a, b\in R$.
  Write
  \begin{align*}
    a &= p_1 \cdots p_n, \\
    b &= q_1 \cdots q_m
  \end{align*}
  where all $p_1, \ldots, p_n, q_1, \ldots, q_m$ are irreducible in $R$.
  (It is possible since $R$ is a UFD.)
  For each $i$, suppose $p_i \mid q_j$ for some $i, j$.
  Write $q_j = p_i u$ for some $u \in R$.
  By the irreducibility of $p_i$ and $q_j$, $u$ is a unit.
  So
  \[
    z
    = \frac{a}{b}
    = \frac{p_1 \cdots \widehat{p_i} \cdots p_n}{q_1 \cdots \widehat{q_j} \cdots q_m}
    = \frac{p_1 \cdots \widehat{p_i} \cdots p_n}{u q_1 \cdots \widehat{q_j} \cdots q_m}.
  \]
  Continue this method we can write $z = \frac{a'}{b'}$ where $a'$ and $b'$ have no common factors.

\item[(2)]
  Write $z = a/b = a'/b'$ where
  \begin{enumerate}
  \item[(a)]
    $a, b, a', b' \in R$,

  \item[(b)]
    $a$ and $b$ have no common factors,

  \item[(c)]
    $a'$ and $b'$ have no common factors.
  \end{enumerate}
  Write
  \begin{align*}
    a &= p_1 \cdots p_n, \\
    b &= q_1 \cdots q_m, \\
    a' &= p'_1 \cdots p'_{n'}, \\
    b' &= q'_1 \cdots q'_{m'}
  \end{align*}
  where all $p_i, q_j, p'_{i'}, q'_{j'}$ are irreducible in $R$.
  As $z = a/b = a'/b'$, $ab' = a'b$ or
  \[
    p_1 \cdots p_n q'_1 \cdots q'_{m'}
    = p'_1 \cdots p'_{n'} q_1 \cdots q_m.
  \]

\item[(3)]
  For $i = 1$, $p_1 = u_1 p'_{i'}$ for some unit $u_1 \in R$
  since $a$ and $b$ have no common factors and all $p_1, q_j, p'_{i'}$ are irreducible.
  Hence
  \[
    u_1 \widehat{p_1} p_2 \cdots p_n q'_1 \cdots q'_{m'}
    = p'_1 \cdots \widehat{p'_{i'}} \cdots p'_{n'} q_1 \cdots q_m.
  \]
  Continue this method,
  we have $n \leq n'$ and all $p_1, \ldots, p_n$ are canceled.

\item[(4)]
  Conversely, we can apply the argument in (3) to $i' = 1, \ldots n'$ to conclude that $n' \leq n$.
  Therefore, $n = n'$ and
  \[
    \underbrace{u_1 \cdots u_n}_{\text{a unit in $R$}} q'_1 \cdots q'_{m'} = q_1 \cdots q_m.
  \]
  Hence, $b = ub'$ where $u = u_1 \cdots u_n$ is a unit in $R$.
  Similarly, $a = va'$ where $v$ is a unit in $R$.
  So the representative of $z \in K$ is unique up to units of $R$.
\end{enumerate}
$\Box$ \\\\



%%%%%%%%%%%%%%%%%%%%%%%%%%%%%%%%%%%%%%%%%%%%%%%%%%%%%%%%%%%%%%%%%%%%%%%%%%%%%%%%



\subsubsection*{Problem 1.3.*}
\addcontentsline{toc}{subsubsection}{Problem 1.3.*}
\emph{Let $R$ be a PID. Let $\mathfrak{p}$ be a nonzero, proper, prime ideal in $R$.}
\begin{enumerate}
\item[(a)]
  \emph{Show that $\mathfrak{p}$ is generated by an irreducible element.}

\item[(b)]
  \emph{Show that $\mathfrak{p}$ is maximal.} \\
\end{enumerate}



\emph{Proof of (a).}
\begin{enumerate}
\item[(1)]
  Let $\mathfrak{p} = (a)$ be a nonzero, proper, prime ideal in $R$.
  It suffices to show that $a$ is irreducible.

\item[(2)]
  Suppose $a = bc$.
  By the primality of $\mathfrak{p}$, $b \in \mathfrak{p}$ or $c \in \mathfrak{p}$.
  Suppose $b \in \mathfrak{p} = (a)$. (The case $c \in \mathfrak{p}$ is similar.)
  Then there is a $d \in R$ such that $b = ad$.
  Hence, $a = bc = adc$ or $(1-dc)a = 0$.

\item[(3)]
  Since $R$ is a domain, $1 = dc$ or $a = 0$.
  $a = 0$ implies that $\mathfrak{p} = (0)$ is a zero ideal, contrary to the assumption.
  Therefore, $1 = dc$, or $c$ is a unit, or $a$ is irreducible.
\end{enumerate}
$\Box$ \\



\emph{Proof of (b).}
\begin{enumerate}
\item[(1)]
  Given any ideal $I = (b)$ of $R$ containing $\mathfrak{p} = (a)$.
  As the generator $a$ of $\mathfrak{p}$ is in $\mathfrak{p} \subseteq I$,
  there is some $c \in R$ such that $a = bc$.
  By the irreducibility of $a$ (in (a)), $b$ is a unit or $c$ is a unit.

\item[(2)]
  $b$ is a unit implies that $I = R$.
  $c$ is a unit implies that $I = \mathfrak{p}$.
  In any case, we conclude that $\mathfrak{p}$ is maximal.
\end{enumerate}
$\Box$ \\\\



%%%%%%%%%%%%%%%%%%%%%%%%%%%%%%%%%%%%%%%%%%%%%%%%%%%%%%%%%%%%%%%%%%%%%%%%%%%%%%%%



\subsubsection*{Problem 1.4.*}
\addcontentsline{toc}{subsubsection}{Problem 1.4.*}
\emph{Let $k$ be an infinite field,
$f \in k[x_1,\ldots,x_n]$.
Suppose $f(a_1,\ldots,a_n) = 0$ for all $a_1, \ldots, a_n \in k$.
Show that $f = 0$.
(Hint: Write
\[
  f = \sum f_i x_n^{i},
  \qquad
  f_i \in k[x_1,\ldots,x_{n-1}].
\]
Use induction on $n$,
and the fact that $f(a_1, \ldots, a_{n-1}, x_n)$
has only a finite number of roots if any $f_i(a_1, \ldots, a_{n-1}) \neq 0$.)} \\



\emph{Proof.}
\begin{enumerate}
\item[(1)]
  Induction on $n$.
  The case $n = 1$.
  (Reductio ad absurdum)
  If there were a nonzero $f \in k[x_1]$ such that $f(a) = 0$ for all $a \in k$.
  Note that $f$ has at most $\deg f < \infty$ roots,
  contrary to the infinity of $k$.

\item[(2)]
  Assume that the conclusion holds for $n - 1$, then for any $f \in k[x_1,\ldots,x_n]$
  we can write
  \[
    f = \sum f_i x_n^{i},
    \qquad
    f_i \in k[x_1,\ldots,x_{n-1}]
  \]
  as $f \in (k[x_1,\ldots,x_{n-1}])[x_n]$.
  Suppose $f(a_1,\ldots,a_n) = 0$ for all $a_1, \ldots, a_n \in k$.
  For fixed $a_1, \ldots, a_{n-1}$,
  the polynomial $f(a_1, \ldots, a_{n-1}, x_n) \in k[x_n]$ has all distinct roots in
  an infinite field $k$.
  By (1), $f(a_1, \ldots, a_{n-1}, x_n) = 0 \in k[x_n]$,
  or each $f_i(a_1, \ldots, a_{n-1}) = 0$.
  As all $a_1, \ldots, a_{n-1}$ run over $k$,
  we can apply the induction hypothesis
  each $f_i(x_1, \ldots, x_{n-1}) = 0 \in k[x_1,\ldots,x_{n-1}]$.
  Hence, $f = 0 \in k[x_1,\ldots,x_{n}]$.
\end{enumerate}
$\Box$ \\



\emph{Note.}
If $k$ is a finite field of order $q = p^k$,
then the polynomial $f(x) = x^q - x$ has $q$ distinct roots in $k$. \\\\



%%%%%%%%%%%%%%%%%%%%%%%%%%%%%%%%%%%%%%%%%%%%%%%%%%%%%%%%%%%%%%%%%%%%%%%%%%%%%%%%



\subsubsection*{Problem 1.5.*}
\addcontentsline{toc}{subsubsection}{Problem 1.5.*}
\emph{Let $k$ be any field.
Show that there are an infinitely number of irreducible monic polynomials in $k[x]$.
(Hint: Suppose $f_1,\ldots,f_n$ were all of them, and factor $f_1\cdots f_n+1$ into irreducible factors.)} \\

\emph{Proof (Due to Euclid).}
\begin{enumerate}
\item[(1)]
  If
  $f_1, \ldots, f_n$ were all irreducible monic polynomials, then
  we consider
  \[
    g = f_1 \cdots f_n + 1 \in k[x].
  \]
  So there is an irreducible monic polynomial $f = f_i$ dividing $g$ for some $i$
  since
  \[
    \deg g = \deg f_1 + \cdots + \deg f_n \geq 1
  \]
  and $k[x]$ is a UFD.

\item[(2)]
  However, $f$ would divide the difference
  \[
    g - f_1 \cdots f_{i-1} f_i f_{i+1} \cdots f_n = 1,
  \]
  contrary to $\deg f_i \geq 1$.
\end{enumerate}
$\Box$ \\\\



%%%%%%%%%%%%%%%%%%%%%%%%%%%%%%%%%%%%%%%%%%%%%%%%%%%%%%%%%%%%%%%%%%%%%%%%%%%%%%%%



\subsubsection*{Problem 1.6.*}
\addcontentsline{toc}{subsubsection}{Problem 1.6.*}
\emph{Show that any algebraically closed field is infinite.
(Hint: The irreducible monic polynomials are $x - a$, $a \in k$.)} \\

\emph{Proof (Due to Euclid).}
\begin{enumerate}
\item[(1)]
  Let $k$ be an algebraically closed field.
  If $a_1, \ldots, a_n$ were all elements in $k$, then
  we consider a monic polynomials
  \[
    f(x) = (x - a_1) \cdots (x - a_n) + 1 \in k[x].
  \]

\item[(2)]
  Since $k$ is algebraically closed,
  there is an element $a \in k$ such that $f(a) = 0$.
  By assumption, $a = a_i$ for some $1 \leq i \leq n$,
  and thus $f(a) = f(a_i) = 1$, contrary to the fact that
  a field is a commutative ring where $0 \neq 1$ and all nonzero elements are invertible.
\end{enumerate}
$\Box$ \\\\



%%%%%%%%%%%%%%%%%%%%%%%%%%%%%%%%%%%%%%%%%%%%%%%%%%%%%%%%%%%%%%%%%%%%%%%%%%%%%%%%



\subsubsection*{Problem 1.7.*}
\addcontentsline{toc}{subsubsection}{Problem 1.7.*}
\emph{Let $k$ be a field, $f \in k[x_1, \ldots, x_n]$, $a_1, \ldots, a_n \in k$.}
\begin{enumerate}
\item[(a)]
  \emph{Show that}
  \[
    f = \sum \lambda_{(i)} (x_1-a_1)^{i_1} \cdots (x_n-a_n)^{i_n},
    \qquad
    \lambda_{(i)} \in k.
  \]

\item[(b)]
  \emph{If $f(a_1, \ldots, a_n) = 0$,
  show that $f = \sum_{i=1}^n (x_i-a_i) g_i$ for some (not unique) $g_i$ in $k[x_1, \ldots, x_n]$.} \\
\end{enumerate}



\emph{Proof of (a).}
\begin{enumerate}
\item[(1)]
  Regard $k[x_1, \ldots, x_n]$ as $(k[x_1, \ldots, x_{n-1}])[x_n]$.
  Since $(k[x_1, \ldots, x_{n-1}])[x_n]$ is a Euclidean domain with a function
  \[
    f \in (k[x_1, \ldots, x_{n-1}])[x_n] \mapsto \deg_{x_n} f \in \mathbb{Z}_{\geq 0}
  \]
  satisfying the division-with-remainder property.

\item[(2)]
  Apply the division algorithm for $f$ and nonzero $x_n-a_n$
  to produce a quotient $q$ and remainder $r$ with
  $f = (x_n-a_n) q + r$ and either $r = 0$ or $\deg_{x_n}(r) < \deg_{x_n} (x_n-a_n) = 1$.
  That is, $r \in k[x_1, \ldots, x_{n-1}]$ is a constant in $(k[x_1, \ldots, x_{n-1}])[x_n]$.
  Continue this process to get that $f$ is of the form
  \[
    f = \sum_{i_n} f_{i_n} (x_n - a_n)^{i_n}
  \]
  where $f_{i_n} \in k[x_1, \ldots, x_{n-1}]$.

\item[(3)]
  Use the same argument in (2) for each $f_{i_n} \in k[x_1, \ldots, x_{n-1}]$, we have
  \begin{align*}
    f_{i_n}
    &= \sum_{i_{n-1}} \underbrace{f_{i_n,i_{n-1}}}_{\in k[x_1, \ldots, x_{n-2}]}
      (x_{n-1} - a_{n-1})^{i_{n-1}} \\
    f_{i_n,i_{n-1}}
    &=
    \sum_{i_{n-2}} \underbrace{f_{i_n,i_{n-1},i_{n-2}}}_{\in k[x_1, \ldots, x_{n-3}]}
      (x_{n-2} - a_{n-2})^{i_{n-2}}, \\
    & \cdots \\
    f_{i_n,\ldots,i_{2}}
    &= \sum_{i_1} \underbrace{f_{i_n,\ldots,i_1}}_{\in k} (x_1 - a_1)^{i_1}.
  \end{align*}
  Note that $f_{i_n,\ldots,i_1} \in k$, we can write
  \[
    f = \sum \lambda_{(i)} (x_1-a_1)^{i_1} \cdots (x_n-a_n)^{i_n},
    \qquad
    \lambda_{(i)} \in k.
  \]
  by replacing all $f_{i_n,\ldots,i_k}$ by $f_{i_n,\ldots,i_{k-1}}$
  for $k = n, n-1, \ldots, 2$.

\item[(4)]
  Or use the induction on $n$.
\end{enumerate}
$\Box$ \\



\emph{Proof of (b).}
\begin{enumerate}
\item[(1)]
  Write
  \[
    f = \sum \lambda_{(i)} (x_1-a_1)^{i_1} \cdots (x_n-a_n)^{i_n},
    \qquad
    \lambda_{(i)} \in k
  \]
  by (a).

\item[(2)]
  As $f(a_1, \cdots, a_n) = 0$,
  $\lambda_{(i)} = 0$ if all $i_1, \ldots, i_n$ are zero, that it,
  there is no nonzero constant term in the representation of $f$.
  Hence, for each term
  \[
    f_{(i)} : = \lambda_{(i)} (x_1-a_1)^{i_1} \cdots (x_n-a_n)^{i_n}
  \]
  with $\lambda_{(i)} \neq 0$,
  there exists one $i_k > 0$ for some $1 \leq k \leq n$.
  So we can write
  \[
    f_{(i)}
    =
    (x_k-a_k)
      \underbrace{
        (\lambda_{(i)} (x_1-a_1)^{i_1} \cdots (x_k-a_k)^{i_k-1} \cdots (x_n-a_n)^{i_n})}_{
          := g_{(i)} \in k[x_1,\ldots,x_n]}.
  \]
  Note that the expression of $f_{(i)}$ is not unique since
  there may exist more than one $i_k > 0$ as $1 \leq k \leq n$.

\item[(3)]
  Now we iterate each nonzero term in $f$, apply the factorization in (2),
  and then group by each $x_k-a_k$.
  Therefore, we can write
  \[
    f = \sum_{i=1}^{n}(x_i-a_i)g_i
  \]
  for some $g_1 \in k[x_1, \ldots, x_n]$.

\item[(4)]
  The expression of $f$ is not unique.
  For example, take $f(x,y) = x^2 + 2xy + y^2 \in k[x,y]$.
  As $f(0,0) = 0$, we can write
  \begin{align*}
    f(x,y)
    &= x \cdot \underbrace{(x+2y)}_{g_1} + y \cdot \underbrace{y}_{g_2}, \text{ or } \\
    &= x \cdot \underbrace{(x+y)}_{g_1} + y \cdot \underbrace{(x+y)}_{g_2}, \text{ or } \\
    &= x \cdot \underbrace{x}_{g_1} + y \cdot \underbrace{(2x+y)}_{g_2}.
  \end{align*}
\end{enumerate}
$\Box$ \\\\



%%%%%%%%%%%%%%%%%%%%%%%%%%%%%%%%%%%%%%%%%%%%%%%%%%%%%%%%%%%%%%%%%%%%%%%%%%%%%%%%
%%%%%%%%%%%%%%%%%%%%%%%%%%%%%%%%%%%%%%%%%%%%%%%%%%%%%%%%%%%%%%%%%%%%%%%%%%%%%%%%



\subsection*{1.2. Affine Space and Algebraic Sets \\}
\addcontentsline{toc}{subsection}{1.2. Affine Space and Algebraic Sets}



\subsubsection*{Problem 1.8.*}
\addcontentsline{toc}{subsubsection}{Problem 1.8.*}
\emph{Show that the algebraic subsets of $\mathbf{A}^1(k)$ are just the finite subsets, together
with $\mathbf{A}^1(k)$ itself.} \\

\emph{Proof.}
\begin{enumerate}
\item[(1)]
  \emph{Show that $k[x]$ is a PID if $k$ is a field.}
  \begin{enumerate}
  \item[(a)]
    Let $I$ be an ideal of $k[x]$.

  \item[(b)]
    If $I = \{0\}$ then $I = (0)$ and $I$ is principal.

  \item[(c)]
    If $I \neq \{0\}$, then take $f$ to be a polynomial of minimal degree in $I$.
    It suffices to show that $I = (f)$.
    Clearly, $(f) \subseteq I$ since $I$ is an ideal.
    Conversely, for any $g \in I$,
    \[
      g(x) = f(x)h(x) + r(x)
    \]
    for some $h, r \in k[x]$ with $r = 0$ or $\deg r < \deg f$ (as $k[x]$ is a Euclidean domain).
    Now as
    \[
      r = g - fh \in I,
    \]
    $r = 0$ (otherwise contrary to the minimality of $f$),
    we have $g = fh \in (f)$ for all $g \in I$.
  \end{enumerate}

\item[(2)]
  Let $Y$ be an algebraic subset of $\mathbf{A}^1(k)$,
  say $Y = V(I)$ for some ideal $I$ of $k[x]$.
  Since $k[x]$ is a PID, $I = (f)$ for some $f \in k[x]$.
  \begin{enumerate}
  \item[(a)]
    If $f = 0$, then $I = (0)$ and $Y = V(0) = \mathbf{A}^1(k)$.

  \item[(b)]
    If $f \neq 0$, then $f(x) = 0$ has finitely many roots in $k$,
    say $a_1, \ldots, a_m \in k$.
    Hence,
    \[
      Y = V(I) = V(f) = \{ f(a) = 0 : a \in k \}
      = \{ a_1, \ldots, a_m \}
    \]
    is a finite subsets of $\mathbf{A}^1(k)$.
  \end{enumerate}
  By (a)(b), the result is established.
\end{enumerate}
$\Box$ \\



\emph{Notes.}
\begin{enumerate}
\item[(1)]
  By the Hilbert basis theorem, $k[x]$ is Noetherian as $k$ is Noetherian.
  Hence, for any algebraic subset $Y = V(I)$ of $\mathbf{A}^1(k)$,
  we can write $I = (f_1, \cdots, f_m)$.
  Note that
  \[
    Y = V(I) = V(f_1) \cap \cdots \cap V(f_m).
  \]
  Now apply the same argument to get the same conclusion.

\item[(2)]
  Suppose $k = \overline{k}$.
  $\mathbf{A}^1(k)$ is irreducible, because its only proper closed subsets are finite,
  yet it is infinite
  (because $k$ is algebraically closed, hence infinite). \\
\end{enumerate}



%%%%%%%%%%%%%%%%%%%%%%%%%%%%%%%%%%%%%%%%%%%%%%%%%%%%%%%%%%%%%%%%%%%%%%%%%%%%%%%%



\subsubsection*{Problem 1.9.}
\addcontentsline{toc}{subsubsection}{Problem 1.9.}
\emph{If $k$ is a finite field, show that every subset of $\mathbf{A}^{n}(k)$ is algebraic.} \\

\emph{Proof.}
\begin{enumerate}
\item[(1)]
  Every subset of $\mathbf{A}^{n}(k)$ is finite since
  $|\mathbf{A}^{n}(k)| = |k|^n$ is finite.

\item[(2)]
  Note that $V(x_1-a_1,\ldots,x_n-a_n) = \{ (a_1,\ldots,a_n) \} \subseteq \mathbf{A}^{n}(k)$
  (Property (5) in \S 1.2)
  and any finite union of algebraic sets is algebraic (Property (4) in \S 1.2).
  Thus, every subset of $\mathbf{A}^{n}(k)$ is algebraic (by (1)).
\end{enumerate}
$\Box$ \\\\



%%%%%%%%%%%%%%%%%%%%%%%%%%%%%%%%%%%%%%%%%%%%%%%%%%%%%%%%%%%%%%%%%%%%%%%%%%%%%%%%



\subsubsection*{Problem 1.10.}
\addcontentsline{toc}{subsubsection}{Problem 1.10.}
\emph{Give an example of a countable collection of algebraic sets whose union is not
algebraic.} \\

\emph{Proof.}
\begin{enumerate}
\item[(1)]
  Let $k = \mathbb{Q}$ be an infinite field.
  $V(x-a) = \{ a \}$ is an algebraic sets for all $a \in \mathbb{Q}$.
  In particular, $V(x-a) = \{ a \}$ is algebraic for all $a \in \mathbb{Z}$.

\item[(2)]
  Note that
  \[
    Y := \bigcup_{a \in \mathbb{Z}} V(x-a) = \mathbb{Z}
  \]
  is a countable union of algebraic sets.
  Since $Y$ is a proper subset of $k = \mathbb{Q}$,
  it cannot be algebraic by Problem 1.8.
\end{enumerate}
$\Box$ \\\\



%%%%%%%%%%%%%%%%%%%%%%%%%%%%%%%%%%%%%%%%%%%%%%%%%%%%%%%%%%%%%%%%%%%%%%%%%%%%%%%%



\subsubsection*{Problem 1.11.}
\addcontentsline{toc}{subsubsection}{Problem 1.11.}
\emph{Show that the following are algebraic sets:}
\begin{enumerate}
\item[(a)]
  $\{ (t,t^2,t^3) \in \mathbf{A}^{3}(k) : t \in k \}$;

\item[(b)]
  $\{ (\cos(t),\sin(t)) \in \mathbf{A}^{2}(\mathbb{R}) : t \in \mathbb{R} \}$;

\item[(c)]
  \emph{the set of points in $\mathbf{A}^{2}(\mathbb{R})$
  whose polar coordinates $(r,\theta)$ satisfy the equation $r = \sin(\theta)$.} \\
\end{enumerate}



\emph{Proof of (a).}
\begin{enumerate}
\item[(1)]
  The twisted cubic curve
  \[
    Y = \{ (t,t^2,t^3) \in \mathbf{A}^3(k) : t \in k \}
    =
    V(x^2-y) \cap V(x^3-z)
  \]
  is algebraic.
  We say that $Y$ is given by the parametric representation $x=t$, $y=t^2$, $z=t^3$.

\item[(2)]
  The generators for the ideal $I(Y)$ are $x^2-y$ and $x^3-z$.

\item[(3)]
  $Y$ is an affine variety of dimension $1$.

\item[(4)]
  The affine coordinate ring $A(Y)$ is isomorphic to a polynomial ring in one variable over $k$.
\end{enumerate}
$\Box$ \\



\emph{Proof of (b).}
The circle
\[
  \{(\cos(t),\sin(t)) \in \mathbf{A}^2(\mathbb{R}) : t \in \mathbb{R} \} = V(x^2-y^2-1)
\]
is algebraic.
$\Box$ \\



\emph{Proof of (c).}
The circle
\[
  \{ (r,\theta) : r = \sin(\theta) \} = V(x^2+y^2-y)
\]
is algebraic again.
$\Box$ \\\\



%%%%%%%%%%%%%%%%%%%%%%%%%%%%%%%%%%%%%%%%%%%%%%%%%%%%%%%%%%%%%%%%%%%%%%%%%%%%%%%%



\subsubsection*{Problem 1.12.}
\addcontentsline{toc}{subsubsection}{Problem 1.12.}
\emph{Suppose $C$ is an affine plane curve,
and $L$ is a line in $\mathbf{A}^2(k)$,
$L \not\subseteq C$.
Suppose $C = V(f)$, $f \in k[x,y]$ a polynomial of degree $n$.
Show that $L \cap C$ is a finite set of no more than $n$ points.
(Hint: Suppose $L = V(y - (ax + b))$, and consider
$f(x,ax +b) \in k[x]$.)} \\

\emph{Proof.}
\begin{enumerate}
\item[(1)]
  Say $L = V(y - (ax + b))$ be a line in $\mathbf{A}^2(k)$.
  (The case $L = V(x - (ay + b))$ is similar.)

\item[(2)]
  Note that $L \not\subseteq C$ implies that $(y - (ax + b)) \nmid f$.
  Hence, the polynomial
  \[
    g: x \mapsto f(x,ax + b) \in k[x]
  \]
  is nonzero and $\deg g \leq n$.
  Therefore, the number of roots of $g$ in $k$ is no more than $n$.

\item[(3)]
  Hence,
  \begin{align*}
    L \cap C
    &= V(y - (ax + b)) \cap V(f) \\
    &= \{ (x,y) \in \mathbf{A}^2(k) : y = ax + b \text{ and } f(x,y) = 0 \} \\
    &= \{ (x,y) \in \mathbf{A}^2(k) : f(x,ax + b) = 0 \}
  \end{align*}
  is finite of no more than $n$ points.
\end{enumerate}
$\Box$ \\\\



%%%%%%%%%%%%%%%%%%%%%%%%%%%%%%%%%%%%%%%%%%%%%%%%%%%%%%%%%%%%%%%%%%%%%%%%%%%%%%%%



\subsubsection*{Problem 1.13.}
\addcontentsline{toc}{subsubsection}{Problem 1.13.}
\emph{Show that each of the following sets is not algebraic:}
\begin{enumerate}
\item[(a)]
  $\{ (x,y) \in \mathbf{A}^{2}(\mathbb{R}) : y = \sin(x) \}$.

\item[(b)]
  \emph{$\{ (z,w) \in \mathbf{A}^{2}(\mathbb{C}) : |z|^2 + |w|^2 = 1 \}$,
  where $|x+iy|^2 = x^2 + y^2$ for $x, y \in \mathbb{R}$.}

\item[(c)]
  $\{ (\cos(t), \sin(t), t) \in \mathbf{A}^3(\mathbb{R}) : t \in \mathbb{R} \}$. \\
\end{enumerate}



\emph{Proof of (a).}
\begin{enumerate}
\item[(1)]
  (Reductio ad absurdum)
  If
  \[
    Y := \{ (x,y) \in \mathbf{A}^{2}(\mathbb{R}) : y = \sin(x) \}
  \]
  were algebraic,
  then there is a subset $S$ of $\mathbb{R}[x,y]$ such that
  \[
    Y = V(S) = \bigcap_{f \in S} V(f).
  \]

\item[(2)]
  $S \neq \varnothing$ since $Y \neq \mathbf{A}^{2}(\mathbb{R})$.
  ($(89,64) \in \mathbf{A}^{2}(\mathbb{R})-Y$.)

\item[(3)]
  Take a fixed line $L = V(y)$ in $\mathbf{A}^{2}(\mathbb{R})$.
  For each affine curve $f \in S$, we have
  \[
    V(f) \cap L
    \supseteq
    \bigcap_{f \in S} V(f) \cap L
    = Y \cap L
    = \{ (n\pi,0) \in \mathbf{A}^{2}(\mathbb{R}) : n \in \mathbb{Z} \},
  \]
  which is infinite.
  By problem 1.12, $y \mid f$.
  As $f$ runs over $S$, $Y \subseteq V(y) = L$,
  contradicts that $\left(0,\frac{\pi}{2}\right) \in L-Y$.
\end{enumerate}
$\Box$ \\\\



\emph{Proof of (b).}
\begin{enumerate}
\item[(1)]
  Similar to (a).
  (Reductio ad absurdum)
  If
  \[
    Y := \{ (x,y) \in \mathbf{A}^{2}(\mathbb{C}) : |x|^2 + |y|^2 = 1 \}
  \]
  were algebraic,
  then there is a subset $S$ of $\mathbb{C}[x,y]$ such that
  \[
    Y = V(S) = \bigcap_{f \in S} V(f).
  \]

\item[(2)]
  $S \neq \varnothing$ since $Y \neq \mathbf{A}^{2}(\mathbb{C})$.
  ($(89,64) \in \mathbf{A}^{2}(\mathbb{C})-Y$.)

\item[(3)]
  Take a fixed line $L = V(x)$ in $\mathbf{A}^{2}(\mathbb{C})$.
  For each affine curve $f \in S$, we have
  \[
    V(f) \cap L
    \supseteq
    \bigcap_{f \in S} V(f) \cap L
    = Y \cap L
    = \{ (0,y) \in \mathbf{A}^{2}(\mathbb{C}) : |y| = 1 \},
  \]
  which is infinite (since $Y$ contains a unit circle in the complex plane).
  By problem 1.12, $x \mid f$.
  As $f$ runs over $S$, $Y \subseteq V(x) = L$,
  contradicts that the origin $(0,0) \in L-Y$.
\end{enumerate}
$\Box$ \\



\emph{Proof of (c).}
\begin{enumerate}
\item[(1)]
  Similar to (a) and (b).

\item[(2)]
  \emph{Suppose $C$ is an affine plane curve,
  and $L$ is a line in $\mathbf{A}^3(k)$,
  $L \not\subseteq C$.
  Suppose $C = V(f)$, $f \in k[x,y,z]$ a polynomial of degree $n$.
  Show that $L \cap C$ is a finite set of no more than $n$ points.}
  The proof is similar to Problem 1.12.
  \begin{enumerate}
  \item[(a)]
    Say $L = V(y - (ax + b), z - (cx + d))$ be a line in $\mathbf{A}^3(k)$.

  \item[(b)]
    Note that $L \not\subseteq C$ implies that
    $(y - (ax + b)) \nmid f$ and $(z - (cx + d)) \nmid f$.
    Hence, the polynomial
    \[
      g: x \mapsto f(x,ax + b, cx + d) \in k[x]
    \]
    is nonzero and $\deg g \leq n$.
    Therefore, the number of roots of $g$ in $k$ is no more than $n$.

  \item[(c)]
    Hence,
    \begin{align*}
      L \cap C
      &= V(y - (ax + b), z - (cx + d)) \cap V(f) \\
      &= \{ (x,y) \in \mathbf{A}^2(k) : y = ax + b, z = cx + d \text{ and } f(x,y) = 0 \} \\
      &= \{ (x,y) \in \mathbf{A}^2(k) : f(x, ax + b, cx + d) = 0 \}
    \end{align*}
    is finite of no more than $n$ points.
  \end{enumerate}

\item[(3)]
  (Reductio ad absurdum)
  If
  \[
    Y := \{ (\cos(t), \sin(t), t) \in \mathbf{A}^3(\mathbb{R}) : t \in \mathbb{R} \}
  \]
  were algebraic,
  then there is a subset $S$ of $\mathbb{R}[x,y,z]$ such that
  \[
    Y = V(S) = \bigcap_{f \in S} V(f).
  \]

\item[(4)]
  $S \neq \varnothing$ since $Y \neq \mathbf{A}^{3}(\mathbb{R})$.
  ($(1989,6,4) \in \mathbf{A}^{3}(\mathbb{R})-Y$.)

\item[(5)]
  Take a fixed line $L = V(x-1,y)$ in $\mathbf{A}^{3}(\mathbb{R})$.
  For each affine curve $f \in S$, we have
  \[
    V(f) \cap L
    \supseteq
    \bigcap_{f \in S} V(f) \cap L
    = Y \cap L
    = \{ (1,0,2n\pi) \in \mathbf{A}^{3}(\mathbb{R}) : n \in \mathbb{Z} \},
  \]
  which is infinite.
  By (2), $(x-1) \mid f$ and $y \mid f$.
  As $f$ runs over $S$, $Y \subseteq V(x-1,y) = L$,
  contradicts that $(1,0,\pi) \in L-Y$.
\end{enumerate}
$\Box$ \\

\textbf{Supplement.}
\emph{A circular disk of radius $1$ in the plane $xy$ rolls without slipping
along the $x$ axis.
The figure described by a point of the circumference of of the disk is
called a \textbf{cycloid}.
The parametrized curve $\alpha: \mathbb{R} \to \mathbb{R}^2$
is
\begin{equation*}
  \begin{cases}
     x = t - \sin t \\
     y = 1 - \cos t.
  \end{cases}
\end{equation*}
The cycloid is not algebraic (as (a)).} \\\\



%%%%%%%%%%%%%%%%%%%%%%%%%%%%%%%%%%%%%%%%%%%%%%%%%%%%%%%%%%%%%%%%%%%%%%%%%%%%%%%%



\subsubsection*{Problem 1.14.*}
\addcontentsline{toc}{subsubsection}{Problem 1.14.*}
\emph{Let $f$ be a nonconstant polynomial in $k[x_1, \ldots, x_n]$,
$k$ algebraically closed.
Show that $\mathbf{A}^{n}(k) - V(f)$ is infinite if $n \geq 1$,
and $V(f)$ is infinite if $n \geq 2$.
Conclude that the complement of any proper algebraic set is infinite.
(Hint: See Problem 1.4.)} \\

\emph{Proof.}
\begin{enumerate}
\item[(1)]
  \emph{Show that $\mathbf{A}^{n}(k) - V(f)$ is infinite if $n \geq 1$.}
  Since $f$ is a nonconstant polynomial in $k[x_1, \ldots, x_n]$,
  we may assume that $\deg_{x_n}(f) > 0$.
  Hence
  \[
    x_n \mapsto f(1,\ldots,1,x_n)
  \]
  is a nonconstant polynomial of degree $\deg_{x_n}(f) > 0$ in $k[x_n]$.
  So $f$ has finitely many roots in $k$, say $\xi_1, \ldots, \xi_m$ ($m \geq 0$).
  Hence,
  \[
    (1,\ldots,1,x_n) \neq 0
  \]
  whenever $x_n \neq \xi_m$.
  Such subset in $\mathbf{A}^{1}(k)$ is infinite since $k = \overline{k}$ (Problem 1.6).
  Therefore,
  \begin{align*}
    \mathbf{A}^{n}(k) - V(f)
    &=
    \{ (a_1,\ldots,a_n) \in \mathbf{A}^{n}(k) : f(a_1,\ldots,a_n) \neq 0 \} \\
    &\supseteq
    \{ a_n  \in \mathbf{A}^{1}(k) : f(1,\ldots,1,x_n) \neq 0 \}
  \end{align*}
  is infinite.

\item[(2)]
  \emph{Show that $V(f)$ is infinite if $n \geq 2$.}
  \begin{enumerate}
  \item[(a)]
    Similar to (1).
    Since $f$ is a nonconstant polynomial in $k[x_1, \ldots, x_n]$,
    we may assume that $m := \deg_{x_n}(f) > 0$.
    Write
    \[
      f = \sum_{i=0}^{m} f_i(x_1,\ldots,x_{n-1}) x_n^{i}.
    \]
    Note that each $f_i$ is well-defined since $n \geq 2$.

  \item[(b)]
    If $f_n$ is constant in $k[x_1, \ldots, x_{n-1}]$,
    then $f_n$ is nonzero (since $m > 0$) or $V(f_n) = \varnothing$.
    If $f_n$ is nonconstant in $k[x_1, \ldots, x_{n-1}]$,
    then the set $\mathbf{A}^{n-1}(k) - V(f_n)$ is infinite by (1).
    In any case,
    \[
      \mathbf{A}^{n-1}(k) - V(f_n)
    \]
    is infinite.

  \item[(c)]
    For each $P = (a_1,\ldots,a_{n-1}) \in \mathbf{A}^{n-1}(k) - V(f_n)$,
    \[
      g_P: x_n \mapsto f(P,x_n) = f(a_1,\ldots,a_{n-1},x_n)
    \]
    defines a polynomial in $k[x_n]$ of degree $m > 0$.
    Since $k = \overline{k}$, $g_P$ has at least one root $Q \in k$.
    Hence
    \[
      V(f) \supseteq
      \{ (P,Q) \in \mathbf{A}^{n}(k) : P \in \mathbf{A}^{n-1}(k) - V(f_n), g_P(Q) = 0 \}
    \]
    is infinite since the set $\mathbf{A}^{n-1}(k) - V(f_n)$ is infinite.
  \end{enumerate}
  \emph{Note.}
  It is not true if $k \neq \overline{k}$.
  For example, $V(x^2+y^2+1) = \varnothing$ in $\mathbf{A}^{2}(\mathbb{R})$.

\item[(3)]
  Note that
  \[
    \mathbf{A}^{n}(k) - V(S)
    = \mathbf{A}^{n}(k) - \bigcap_{f \in S} V(f)
    = \bigcup_{f \in S}( \mathbf{A}^{n}(k) - V(f) ).
  \]
  Thus the complement of any proper algebraic set is infinite by (1).
\end{enumerate}
$\Box$ \\\\



%%%%%%%%%%%%%%%%%%%%%%%%%%%%%%%%%%%%%%%%%%%%%%%%%%%%%%%%%%%%%%%%%%%%%%%%%%%%%%%%



\subsubsection*{Problem 1.15.*}
\addcontentsline{toc}{subsubsection}{Problem 1.15.*}
\emph{Let $V \subseteq \mathbf{A}^n(k)$, $W \subseteq \mathbf{A}^m(k)$ be algebraic sets.
Show that
\[
  V \times W
  = \{(a_1,\ldots,a_n,b_1,\ldots,b_m) : (a_1,\ldots,a_n) \in V, (b_1,\ldots,b_m) \in W \}
\]
is an algebraic set in $\mathbf{A}^{n+m}(k)$.
It is called the \textbf{product} of $V$ and $W$.} \\

\emph{Proof.}
\begin{enumerate}
\item[(1)]
  Write
  \begin{align*}
    V &= V(S_V) = \{ P \in \mathbf{A}^n(k) : f(P) = 0 \: \forall f \in S_V \} \\
    W &= V(S_W) = \{ Q \in \mathbf{A}^m(k) : g(Q) = 0 \: \forall g \in S_W \},
  \end{align*}
  where $S_V \subseteq k[x_1,\ldots,x_n]$ and $S_W \subseteq k[y_1,\ldots,y_m]$.
  It suffices to show that
  \[
    V \times W = V(S),
  \]
  where
  $S \subseteq k[x_1,\ldots,x_n,y_1,\ldots,y_m]$ is the union of $S_V$ and $S_W$.

\item[(2)]
  Here we can identify $S_V$ with the subset of
  $k[x_1,\ldots,x_n,y_1,\ldots,y_m]$
  by noting that
  \[
    k[x_1,\ldots,x_n]
    \hookrightarrow (k[y_1,\ldots,y_m])[x_1,\ldots,x_n]
    = k[x_1,\ldots,x_n,y_1,\ldots,y_m].
  \]
  Here we regard $k$ as a subring of $k[y_1,\ldots,y_m]$.
  Similar treatment to $S_W$.

\item[(3)]
  By construction, $V \times W \subseteq V(S)$.
  Conversely, given any $(P,Q) \in V(S) \subseteq \mathbf{A}^{n+m}(k)$,
  we have $h(P,Q) = 0$ for all $h \in S = S_V \cup S_W$ (by (2)).
  By construction, $f(P) = 0$ for all $f \in S_V$ since $f$ only involve $x_1,\ldots,x_n$.
  Hence, $P \in V$. Similarly, $Q \in W$. Therefore, $(P,Q) \in V \times W$.
\end{enumerate}
$\Box$ \\\\



%%%%%%%%%%%%%%%%%%%%%%%%%%%%%%%%%%%%%%%%%%%%%%%%%%%%%%%%%%%%%%%%%%%%%%%%%%%%%%%%
%%%%%%%%%%%%%%%%%%%%%%%%%%%%%%%%%%%%%%%%%%%%%%%%%%%%%%%%%%%%%%%%%%%%%%%%%%%%%%%%



\subsection*{1.3. The Ideal of a Set of Points \\}
\addcontentsline{toc}{subsection}{1.3. The Ideal of a Set of Points}



\subsubsection*{Problem 1.16.*}
\addcontentsline{toc}{subsubsection}{Problem 1.16.*}
\emph{Let $V, W$ be algebraic sets in $\mathbf{A}^n(k)$.
Show that $V = W$ if and only if $I(V) = I(W)$.} \\

\emph{Proof.}
\begin{enumerate}
\item[(1)]
  (Proof of Property (6) in \S 1.3.)
  \emph{Show that if $X \subseteq Y$, then $I(X) \supseteq I(Y)$.}
  If $f \in I(Y)$ then $f(P) = 0$ for all $P \in Y$.
  So $f(P) = 0$ for all $P \in X \subseteq Y$ or $f \in I(X)$.

\item[(2)]
  (Proof of Property (8) in \S 1.3.)
  \emph{$I(V(S)) \supseteq S$ for any set $S$ of polynomials;
  $V(I(X)) \supseteq X$ for any set $X$ of points.}
  \begin{enumerate}
  \item[(a)]
    If $f \in S$ then $f$ vanishes on $V(S)$,
    hence $f \in IV(S)$.

  \item[(b)]
    If $P \in X$ then every polynomial in $I(X)$ vanishes at $P$,
    so $P$ belongs to the zero set of $I(X)$.
  \end{enumerate}

\item[(3)]
  (Proof of Property (9) in \S 1.3.)
  \emph{$V(I(V(S))) = V(S)$ for any set $S$ of polynomials,
  and $I(V(I(X))) = I(X)$ for any set $X$ of points.
  So if $V$ is an algebraic set, $V = V(I(V))$,
  and if $I$ is the ideal of an algebraic set, $I = I(V(I))$.}
  \begin{enumerate}
  \item[(a)]
    In each case, it suffices to show that the left side is a subset of the right side.
    (by Properties (6)(8) in \S 1.3).

  \item[(b)]
    If $P \in V(S)$ then $f(P) = 0$ for all $f \in I(V(S))$, so $P \in V(I(V(S)))$.

  \item[(c)]
    If $f \in I(X)$ then $f(P) = 0$ for all $P \in V(I(X))$.
    Thus $f$ vanishes on $V(I(X))$, so $f \in I(V(I(X)))$.
  \end{enumerate}

\item[(4)]
  \emph{Show that $V = W$ if and only if $I(V) = I(W)$.}
  \begin{enumerate}
  \item[(a)]
    By Property (6) in \S 1.3, $I(V) \supseteq I(W)$ if $V \subseteq W$
    and $I(V) \subseteq I(W)$ if $V \supseteq W$.
    Thus, $I(V) = I(W)$ if $V = W$.

  \item[(b)]
    Conversely,
    $I(V) = I(W)$ implies that $V(I(V)) = V(I(W))$
    by Property (3) in \S 1.2 and similar argument in (a).
    By Property (9) in \S 1.3, $V(I(V)) = V$ and $V(I(W)) = W$.
    Thus, $V = W$.
  \end{enumerate}
\end{enumerate}
$\Box$ \\\\



%%%%%%%%%%%%%%%%%%%%%%%%%%%%%%%%%%%%%%%%%%%%%%%%%%%%%%%%%%%%%%%%%%%%%%%%%%%%%%%%



\subsubsection*{Problem 1.17.*}
\addcontentsline{toc}{subsubsection}{Problem 1.17.*}
\begin{enumerate}
\item[(a)]
  \emph{Let $V$ be an algebraic set in $\mathbf{A}^n(k)$,
  $P \in\mathbf{A}^n(k)$ a point not in $V$.
  Show that there is a polynomial $f \in k[x_1,\ldots,x_n]$ such that $f(Q) = 0$
  for all $Q \in V$, but $f(P) = 1$. (Hint: $I(V) \neq I(V \cup \{P\})$.)}

\item[(b)]
  \emph{Let $P_1, \ldots, P_r$ be distinct points in $\mathbf{A}^n(k)$,
  not in an algebraic set $V$.
  Show that there are polynomials $f_1, \ldots, f_r \in I(V)$
  such that $f_i(P_j) = 0$ if $i \neq j,$ and $f_i(P_i) = 1$.
  (Hint: Apply (a) to the union of $V$ and all but one point.)}

\item[(c)]
  \emph{With $P_1, \ldots, P_r$ and $V$ as in (b),
  and $a_{ij} \in k$ for $1 \leq i,j \leq r$,
  show that there are $g_i \in I(V)$ with $g_i(P_j) = a_{ij}$ for all $i$ and $j$.
  (Hint: Consider $\sum_{j} a_{ij} f_j$.)} \\
\end{enumerate}



\emph{Proof of (a).}
\begin{enumerate}
\item[(1)]
  Since $I(V) \supsetneq I(V \cup \{P\})$ (by Problem 1.16),
  there is a polynomial $f \in k[x_1,\ldots,x_n]$ such that $f(Q) = 0$ for all $Q \in V$, but $f(P) \neq 0$.

\item[(2)]
  Since $k$ is a field, $(f(P))^{-1} \in k$.
  Consider the polynomial $(f(P))^{-1} f \in k[x_1,\ldots,x_n]$.
  It is well-defined.
  Also, $((f(P))^{-1} f)(Q) = (f(P))^{-1}f(Q) = 0$ for all $Q \in V$,
  but $(f(P))^{-1} f)(P) = (f(P))^{-1} f(P) = 1$.
\end{enumerate}
$\Box$ \\



\emph{Proof of (b).}
\begin{enumerate}
\item[(1)]
  For $1 \leq i \leq$,
  define
  \begin{align*}
    W &= V \cup \{ P_1, \ldots, P_r \} \\
    W_i &= V \cup \{ P_1, \ldots, \widehat{P_i}, \ldots, P_r \}.
  \end{align*}
  Here $W = W_i \cup \{P_i\} \neq W_i$.

\item[(2)]
  By (a), there is a polynomial $f_i \in k[x_1,\ldots,x_n]$ such that $f_i(Q) = 0$
  for all $Q \in W_i$, but $f_i(P_i) = 1$.
  Here $f_i \in I(V)$ and $f_i(P_j) = \delta_{ij}$
  where $\delta_{ij}$ is the Kronecker delta.
\end{enumerate}
$\Box$ \\



\emph{Proof of (c).}
\begin{enumerate}
\item[(1)]
  For each $1 \leq i \leq r$,
  define
  \[
    g_i = \sum_{j} a_{ij} f_j \in k[x_1,\ldots,x_n].
  \]

\item[(2)]
  $g_i \in I(V)$ since $g_i$ is a linear combination of $f_j$ and $I(V)$ is an ideal.

\item[(3)]
  Also,
  \[
    g_i(P_j)
    = \sum_{j'} a_{ij'} f_{j'}(P_j)
    = \sum_{j'} a_{ij'} \delta_{j'j}
    = a_{ij}.
  \]
\end{enumerate}
$\Box$ \\\\



%%%%%%%%%%%%%%%%%%%%%%%%%%%%%%%%%%%%%%%%%%%%%%%%%%%%%%%%%%%%%%%%%%%%%%%%%%%%%%%%



\subsubsection*{Problem 1.18.*}
\addcontentsline{toc}{subsubsection}{Problem 1.18.*}
\emph{Let $I$ be an ideal in a ring $R$.
If $a^n \in I$, $b^m \in I$, show that $(a+b)^{n+m} \in I$.
Show that $\mathrm{rad}(I)$ is an ideal, in fact a radical ideal.
Show that any prime ideal is radical.} \\

\emph{Proof.}
\begin{enumerate}
\item[(1)]
  \emph{Show that $(a+b)^{n+m} \in I$ if $a^n \in I$, $b^m \in I$.}
  By the binomial theorem,
  \[
    (a+b)^{n+m}=\sum_{i=0}^{n+m} a^i b^{n+m-i}.
  \]
  For each term $a^i b^{n+m-i}$, either $i \geq n$ holds or $n+m-i \geq m$ holds,
  and thus $a^i b^{n+m-i} \in I$ (since $a^n \in I$, $b^m \in I$ and $I$ is an ideal).
  Hence, the result is established.

\item[(2)]
  \emph{Show that $\mathrm{rad}(I)$ is an ideal.}
  \begin{enumerate}
  \item[(a)]
    $0 \in \mathrm{rad}(I)$ since $0 = 0^{1} \in I$ for any ideal in $R$.

  \item[(b)]
    $(a+b)^{n+m} \in I$ if $a^n \in I$, $b^m \in I$ by (1).

  \item[(c)]
    $(-a)^{2n} = (a^n)^2 \in I$ if $a^n \in I$ (since $I$ is an ideal).

  \item[(d)]
    $(ra)^n = r^n a^n \in I$ if $a^n \in I$ and $r \in R$ (since $I$ is an ideal and $R$ is commutative).
  \end{enumerate}

\item[(3)]
  \emph{Show that $\mathrm{rad}(\mathrm{rad}(I)) = \mathrm{rad}(I)$.}
  It suffices to show $\mathrm{rad}(\mathrm{rad}(I)) \subseteq \mathrm{rad}(I)$.
  Given any $a \in \mathrm{rad}(\mathrm{rad}(I))$.
  By definition $a^n \in \mathrm{rad}(I)$ for some positive integer $n$.
  Again by definition $(a^n)^m = a^{nm} \in I$ for some positive integer $m$.
  As $nm$ is a postive integer, $a \in \mathrm{rad}(I)$.

\item[(4)]
  \emph{Show that every prime ideal $\mathfrak{p}$ is radical.}
  Given any $a \in \mathrm{rad}(\mathfrak{p})$, that is,
  $a^n \in \mathfrak{p}$ for some positive integer.
  Write $a^n = a a^{n-1}$ if $n > 1$.
  By the primality of $\mathfrak{p}$, $a \in \mathfrak{p}$ or $a^{n-1} \in \mathfrak{p}$.
  If $a \in \mathfrak{p}$, we are done.
  If $a^{n-1} \in \mathfrak{p}$,
  we continue this descending argument (or the mathematical induction)
  until the power of $a$ is equal to $1$.
  Hence $\mathfrak{p}$ is radical.
\end{enumerate}
$\Box$ \\\\



%%%%%%%%%%%%%%%%%%%%%%%%%%%%%%%%%%%%%%%%%%%%%%%%%%%%%%%%%%%%%%%%%%%%%%%%%%%%%%%%



\subsubsection*{Problem 1.19.}
\addcontentsline{toc}{subsubsection}{Problem 1.19.}
\emph{Show that $I = (x^2+1) \subseteq \mathbb{R}[x]$ is a radical (even a prime) ideal,
but $I$ is not the ideal of any set in $\mathbf{A}^1(\mathbb{R})$.} \\

\emph{Proof.}
\begin{enumerate}
\item[(1)]
  \emph{Show that $I = (x^2+1)$ is a prime ideal in $\mathbb{R}[x]$.}
  Given any $fg \in I$.
  It suffices to show that $f \in I$ or $g \in I$.
  By definition of $I$,
  there is a polynomial $h \in \mathbb{R}[x]$ such that $fg = (x^2+1)h$.
  So $(x^2+1) \mid f$ or $(x^2+1) \mid g$
  since $x^2+1$ is irreducible in a unique factorization domain $\mathbb{R}[x]$.
  Therefore, $f \in I$ or $g \in I$.

\item[(2)]
  \emph{Show that $I$ is not the ideal of any set in $\mathbf{A}^1(\mathbb{R})$.}
  Since $x^2+1$ has no roots in $\mathbb{R}$,
  $I$ cannot be the ideal of any nonempty set in $\mathbf{A}^1(\mathbb{R})$.
  Besides,
  $I(\varnothing) = (1) \neq (x^2+1)$.
\end{enumerate}
$\Box$ \\\\



%%%%%%%%%%%%%%%%%%%%%%%%%%%%%%%%%%%%%%%%%%%%%%%%%%%%%%%%%%%%%%%%%%%%%%%%%%%%%%%%



\subsubsection*{Problem 1.20.*}
\addcontentsline{toc}{subsubsection}{Problem 1.20.*}
\emph{Show that for any ideal $I$ in $k[x_1,\ldots,x_n]$,
$V(I) = V(\mathrm{rad}(I))$, and $\mathrm{rad}(I) \subseteq I(V(I))$.} \\

\emph{Proof.}
\begin{enumerate}
\item[(1)]
  \emph{Show that $V(I) = V(\mathrm{rad}(I))$.}
  Since $I \subseteq \mathrm{rad}(I)$,
  it suffices to show that $V(I) \subseteq V(\mathrm{rad}(I))$.
  Given any $P \in V(I)$.
  For any $f \in \mathrm{rad}(I)$, $f^n \in I$ for some positive integer $n > 0$.
  Note that
  \[
    0 = (f^n)(P) = f(P)^n
  \]
  since $f^n \in I$ and $P \in V(I)$.
  As $k$ is a domain, $f(P)^n = 0$ implies $f(P) = 0$. So $P \in V(\mathrm{rad}(I))$.

\item[(2)]
  By Properties (6)(8) in \S 1.3,
  \[
    I(V(I)) = I(V(\mathrm{rad}(I))) \supseteq \mathrm{rad}(I).
  \]
\end{enumerate}
$\Box$ \\

\emph{Note.}
\begin{enumerate}
\item[(1)]
  By the Hilbert's Nullstellensatz, $I(V(I)) = \mathrm{rad}(I)$ if $k = \overline{k}$.

\item[(2)]
  Take $I = (x^2+1)$ as an ideal in $\mathbb{R}[x]$.
  Note that $I(V(I))= I(\varnothing) = (1)$ and $\mathrm{rad}(I) = I = (x^2+1)$.
  So the equality in $\mathrm{rad}(I) \subsetneq I(V(I))$ might not hold
  if $k \neq \overline{k}$.
  (See Problem 1.19.) \\\\
\end{enumerate}



%%%%%%%%%%%%%%%%%%%%%%%%%%%%%%%%%%%%%%%%%%%%%%%%%%%%%%%%%%%%%%%%%%%%%%%%%%%%%%%%



\subsubsection*{Problem 1.21.*}
\addcontentsline{toc}{subsubsection}{Problem 1.21.*}
\emph{Show that
$I=(x_1-a_1, \ldots, x_n-a_n) \subseteq k[x_1,\ldots,x_n]$ is a maximal ideal,
and that the natural homomorphism from $k$ to $k[x_1,\ldots,x_n]/I$ is an isomorphism.} \\

\emph{Proof.}
\begin{enumerate}
\item[(1)]
  \emph{Show that $I$ is a maximal ideal.}
  Suppose that $J$ is an ideal such that $J \supsetneq I$.
  Take any $f \in J - I$.
  By Problem 1.7(a),
  \[
    f = \sum \lambda_{(i)} (x_1-a_1)^{i_1} \cdots (x_n - a_n)^{i_n}.
  \]
  As $f \not\in I$, there is a nonzero constant term in $f$, say $\lambda \in k - \{0\}$.
  Note that $f - \lambda \in I \subsetneq J$.
  Hence,
  \[
    \lambda = f - (f - \lambda) \in J
  \]
  since $J$ is an ideal.
  As $\lambda \neq 0$, $J = k[x_1,\ldots,x_n]$ is not a proper ideal containing $I$.

\item[(2)]
  Let $\varphi: k \to k[x_1,\ldots,x_n]/I$ be the natural homomorphism.
  (That is, $\varphi: \lambda \to \lambda + I \in k[x_1,\ldots,x_n]/I$.)

\item[(3)]
  \emph{Show that $\varphi$ is surjective.}
  Given any $f + I \in k[x_1,\ldots,x_n]/I$.
  By Problem 1.7(a),
  \[
    f = \sum \lambda_{(i)} (x_1-a_1)^{i_1} \cdots (x_n - a_n)^{i_n}.
  \]
  So
  \begin{align*}
    f + I
    &= \sum \lambda_{(i)} (x_1-a_1)^{i_1} \cdots (x_n - a_n)^{i_n} + I \\
    &= \left(f(a_1,\ldots,a_n)
      + \sum_{\text{nonconstant}} \lambda_{(i)} (x_1-a_1)^{i_1} \cdots (x_n - a_n)^{i_n} \right) + I \\
    &= f(a_1,\ldots,a_n) + I.
  \end{align*}
  (Here the summation over all nonconstant terms is in $I$.)
  Hence
  \[
    \varphi: f(a_1,\ldots,a_n) \in k \mapsto f + I.
  \]

\item[(4)]
  \emph{Show that $\varphi$ is injective.}
  $\ker(\varphi) = \{ \lambda \in k : \lambda \in I \} = k \cap I = \{0\}$
  since $I$ is a proper ideal.

\item[(5)]
  By (2)(3)(4), $\varphi: k \to k[x_1,\ldots,x_n]/(x_1-a_1,\ldots,x_n-a_n)$
  is an isomorphism.
\end{enumerate}
$\Box$ \\\\



%%%%%%%%%%%%%%%%%%%%%%%%%%%%%%%%%%%%%%%%%%%%%%%%%%%%%%%%%%%%%%%%%%%%%%%%%%%%%%%%
%%%%%%%%%%%%%%%%%%%%%%%%%%%%%%%%%%%%%%%%%%%%%%%%%%%%%%%%%%%%%%%%%%%%%%%%%%%%%%%%



\subsection*{1.4. The Hilbert Basis Theorem \\}
\addcontentsline{toc}{subsection}{1.4. The Hilbert Basis Theorem}



\subsubsection*{Problem 1.22.* (Correspondence theorem for rings)}
\addcontentsline{toc}{subsubsection}{Problem 1.22.* (Correspondence theorem for rings)}
\emph{Let $I$ be an ideal in a ring $R$, $\pi: R \to R/I$ the natural homomorphism.}
\begin{enumerate}
\item[(a)]
  \emph{Show that for every ideal $J'$ of $R/I$,
  $\pi^{-1}(J') = J$ is an ideal of $R$ containing $I$,
  and for every ideal $J$ of $R$ containing $I$,
  $\pi(J) = J'$ is an ideal of $R/I$.
  This sets up a natural one-to-one correspondence between
  $\{ \text{ideals of $R/I$} \}$ and $\{ \text{ideals of $R$ that contain $I$} \}$.}

\item[(b)]
  \emph{Show that $J'$ is a radical ideal if and only if $J$ is radical.
  Similarly for prime and maximal ideals.}

\item[(c)]
  \emph{Show that $J'$ is finitely generated if $J$ is.
  Conclude that $R/I$ is Noetherian if $R$ is Noetherian.
  Any ring of the form $k[x_1,\ldots,x_n]/I$ is Noetherian.} \\
\end{enumerate}



\emph{Proof of (a).}
\begin{enumerate}
\item[(1)]
  \emph{Show that for every ideal $J'$ of $R/I$,
  $\pi^{-1}(J') = J$ is an ideal of $R$ containing $I$.}
  \begin{enumerate}
  \item[(a)]
    \emph{Show that $J$ contains $I$.}
    Note that $\pi^{-1}(0) = I \subseteq \pi^{-1}(J') = J$.
    So $J$ contains $I$. In particular, $J \neq \varnothing$ since $I \neq \varnothing$.

  \item[(b)]
    \emph{Show that $J$ is a additive subgroup of $R$.}
    It suffices to show that
    $a - b \in J$
    for any $a \in J$ and $b \in J$.
    Actually,
    \[
      \pi(a - b) = \pi(a) - \pi(b) \in J'
    \]
    implies $a - b \in \pi^{-1}(J') = J$.

  \item[(c)]
    \emph{Show that for every $r \in R$ and every $a \in J$,
    the product $ra \in J$.}
    In fact,
    \[
      \pi(ra) = \pi(r) \pi(a) \in J'
    \]
    implies $ra \in \pi^{-1}(J') = J$.
  \end{enumerate}

\item[(2)]
  \emph{Show that for every ideal $J$ of $R$ containing $I$,
  $\pi(J) = J'$ is an ideal of $R/I$.}
  \begin{enumerate}
  \item[(a)]
    \emph{Show that $J'$ is nonempty.}
    Note that $\pi(a) = 0 \in \pi(I) \subseteq \pi(J) = J'$ for any $a \in I$.
    So $J'$ is nonempty since $J$ is nonempty.

  \item[(b)]
    \emph{Show that $J'$ is a additive subgroup of $R/I$.}
    It suffices to show that
    $\pi(a) - \pi(b) \in J'$ for any $\pi(a) \in J'$, $\pi(b) \in J'$, $a \in J$ and $b \in J$.
    It is trivial since
    \[
      \pi(a) - \pi(b) = \pi(a - b) \in \pi(J) = J',
    \]
    $\pi$ is a ring homomorphism and $J$ is an ideal.

  \item[(c)]
    \emph{Show that for every $\pi(r) \in R/I$ ($r \in R$) and every $\pi(a) \in J'$ ($a \in J$),
    the product $\pi(r)\pi(a) \in J'$.}
    It is trivial since
    \[
      \pi(r)\pi(a) = \pi(ra) \in \pi(J) = J',
    \]
    $\pi$ is a ring homomorphism and $J$ is an ideal.
  \end{enumerate}

\item[(3)]
  By (1)(2), we setup the correspondence between
  \[
    \{ \text{ideals of $R/I$} \}
    \longleftrightarrow
    \{ \text{ideals of $R$ that contain $I$} \}.
  \]
  Note that this correspondence preserves the subset relation,
  and thus this correspondence is one-to-one.
\end{enumerate}
$\Box$ \\



\emph{Proof of (b).}
\begin{enumerate}
\item[(1)]
  \emph{Show that $J'$ is radical if $J$ is radical.}
  It suffices to show that $(a + I)^n = a^n + I \in J'$ implies that $a + I \in J'$.
  Note that
  \[
    (a + I)^n = a^n + I \in J'
  \]
  implies that $a^n \in J$ or $a \in J$ since $J$ is radical.
  Hence $a + I \in J/I = J'$.

\item[(2)]
  \emph{Show that $J$ is radical if $J'$ is radical.}
  It suffices to show that $a^n \in J$ implies that $a \in J$.
  Note that
  \[
    \pi(a^n) = \pi(a)^n \in J'
  \]
  implies that $\pi(a) \in J'$ since $J'$ is radical.
  $a \in \pi^{-1}(J') = J$.

\item[(3)]
  \emph{Show that $J'$ is prime if $J$ is prime.}
  It suffices to show that
  $(a + I)(b + I) = ab + I \in J'$ implies that $a + I \in J'$ or $b + I \in J'$.
  Note that
  \[
    (a + I)(b + I) = ab + I \in J'
  \]
  implies that $ab \in J$. So $a \in J$ or $b \in J$ by the primality of $J$.
  Hence $a + I \in J'$ or $b + I \in J'$.

\item[(4)]
  \emph{Show that $J$ is prime if $J'$ is prime.}
  It suffices to show that $ab \in J$ implies that $a \in J$ or $b \in J$.
  Note that
  \[
    \pi(ab) = \pi(a)\pi(b) \in J'
  \]
  implies that $\pi(a) \in J'$ or $\pi(b) \in J'$ by the primality of $J'$.
  So $a \in \pi^{-1}(J') = J$ or $b \in \pi^{-1}(J') = J$.

\item[(5)]
  \emph{Show that $J'$ is maximal if $J$ is maximal.}
  Suppose $\mathfrak{m}$ is an ideal containing $J'$.
  By (a), $\pi^{-1}(\mathfrak{m})$ is an ideal containing $J$.
  So $\pi^{-1}(\mathfrak{m}) = J$ or $\pi^{-1}(\mathfrak{m}) = R$ by the maximality of $J$.
  Hence, $\mathfrak{m} = \pi(J) = J'$ or $\mathfrak{m} = \pi(R) = R/I$.

\item[(6)]
  \emph{Show that $J$ is maximal if $J'$ is maximal.}
  Suppose $\mathfrak{m}$ is an ideal containing $J$.
  By (a), $\pi(\mathfrak{m})$ is an ideal containing $J'$.
  So $\pi(\mathfrak{m}) = J'$ or $\pi(\mathfrak{m}) = R/I$ by the maximality of $J'$.
  Hence, $\mathfrak{m} = \pi^{-1}(J') = J$ or $\mathfrak{m} = \pi^{-1}(R/I) = R$.
\end{enumerate}
$\Box$ \\



\emph{Note.}
\begin{enumerate}
\item[(1)]
  Note that
  \[
    R/J \cong (R/I)/(J/I)
  \]
  if $J$ is an ideal of $R$ such that $I \subseteq J$.

\item[(2)]
  Hence, $J$ is prime iff $R/J \cong (R/I)/(J/I)$ is a domain iff $J/I$ is prime.

\item[(3)]
  Also, $J$ is maximal iff $R/J \cong (R/I)/(J/I)$ is a field iff $J/I$ is maximal. \\
\end{enumerate}



\emph{Proof of (c).}
\begin{enumerate}
\item[(1)]
  \emph{Show that $J'$ is finitely generated if $J$ is.}
  Suppose $J$ is generated by $a_1, \ldots, a_m$.
  It suffices to show that $J'$ is generated by
  \[
    a_1 + I, \ldots, a_m + I \in J/I.
  \]
  Given any $a + I \in J'$ where $a \in J$.
  Write $a = \sum_{1 \leq i \leq m} r_i a_i$ for some $r_i \in R$.
  Then
  \[
    a + I = \sum r_i a_i + I = \sum (r_i + I)(a_i + I)
  \]
  is generated by $a_1 + I, \ldots, a_m + I$.

\item[(2)]
  \emph{Show that that $R/I$ is Noetherian if $R$ is Noetherian.}
  Note that $R$ is an ideal of itself.

\item[(3)]
  \emph{Show that any ring of the form $k[x_1,\ldots,x_n]/I$ is Noetherian.}
  By the corollary to the Hilbert basis theorem,
  $k[x_1,\ldots,x_n]$ is Noetherian.
  By (2), the ring $k[x_1,\ldots,x_n]/I$ is Noetherian.
\end{enumerate}
$\Box$ \\\\



%%%%%%%%%%%%%%%%%%%%%%%%%%%%%%%%%%%%%%%%%%%%%%%%%%%%%%%%%%%%%%%%%%%%%%%%%%%%%%%%
%%%%%%%%%%%%%%%%%%%%%%%%%%%%%%%%%%%%%%%%%%%%%%%%%%%%%%%%%%%%%%%%%%%%%%%%%%%%%%%%



\subsection*{1.5. Irreducible Components of an Algebraic Set \\}
\addcontentsline{toc}{subsection}{1.5. Irreducible Components of an Algebraic Set}



\subsubsection*{Problem 1.23.}
\addcontentsline{toc}{subsubsection}{Problem 1.23.}
\emph{Give an example of a collection of ideals $\mathscr{S}$ ideals in a Noetherian ring
such that no maximal member of $\mathscr{S}$ is a maximal ideal.} \\

\emph{Proof.}
\begin{enumerate}
\item[(1)]
  Let $R$ be any Noetherian ring.
  Let $\mathscr{S}$ be any collection of ideals containing $R$ itself.
  Then the only maximal member of $\mathscr{S}$ is $R$, which is not a maximal ideal.

\item[(2)]
  Or let $R$ be any Noetherian ring and $R$ is not a field.
  ($R = k[x_1,\ldots,k_n]$ where $k$ is a field for example.)
  Let $\mathscr{S} = \{ (0) \}$.
  Then the only maximal member of $\mathscr{S}$ is $(0)$, which is not maximal
  since $R$ is not a field.
\end{enumerate}
$\Box$ \\\\



%%%%%%%%%%%%%%%%%%%%%%%%%%%%%%%%%%%%%%%%%%%%%%%%%%%%%%%%%%%%%%%%%%%%%%%%%%%%%%%%



\subsubsection*{Problem 1.24.}
\addcontentsline{toc}{subsubsection}{Problem 1.24.}
\emph{Show that every proper ideal in a Noetherian ring is contained in a maximal ideal.
(Hint: If $I$ is the ideal, apply the lemma to $\{\text{proper ideals that contain $I$}\}$.)} \\

\emph{Proof.}
\begin{enumerate}
\item[(1)]
  Say $I$ be any proper ideal in a Noetherian ring.
  Let
  \[
    \mathscr{S} = \{\text{proper ideals that contain $I$}\}.
  \]
  Apply the lemma to $\mathscr{S}$ to get that
  $\mathscr{S}$ has a maximal member $\mathfrak{m} \in \mathscr{S}$.

\item[(2)]
  \emph{Show that $\mathfrak{m}$ is maximal.}
  Since $\mathfrak{m} \in \mathscr{S}$, $\mathfrak{m}$ is a proper ideal in $R$.
  Suppose $\mathfrak{m}' \supseteq \mathfrak{m}$
  is a proper ideal containing $\mathfrak{m}$.
  As $\mathfrak{m}$ contains $I$,
  $\mathfrak{m}'$ also contains $I$ or $\mathfrak{m}' \in \mathscr{S}$.
  By the maximality of $\mathfrak{m}$, $\mathfrak{m}' \subseteq \mathfrak{m}$.
  So $\mathfrak{m}' = \mathfrak{m}$.
\end{enumerate}
$\Box$ \\\\



%%%%%%%%%%%%%%%%%%%%%%%%%%%%%%%%%%%%%%%%%%%%%%%%%%%%%%%%%%%%%%%%%%%%%%%%%%%%%%%%



\subsubsection*{Problem 1.25.}
\addcontentsline{toc}{subsubsection}{Problem 1.25.}
\begin{enumerate}
\item[(a)]
  \emph{Show that $V(y - x^2) \subseteq \mathbf{A}^2(\mathbb{C})$ is irreducible,
  in fact, $I(V(y - x^2))=(y - x^2)$.}

\item[(b)]
  \emph{Decompose $V(y^4 - x^2, y^4 - x^2 y^2 + xy^2 - x^3) \subseteq \mathbf{A}^2(\mathbb{C})$
  into irreducible components.} \\
\end{enumerate}



\emph{Proof of (a).}
\begin{enumerate}
\item[(1)]
  Let $I = (y - x^2)$ be an ideal of $\mathbb{C}[x,y]$.
  Since $\mathbb{C}$ is algebraically closed,
  \[
    I(V(I)) = \mathrm{rad}(I)
  \]
  by the Hilbert's Nullstellensatz.
  It suffices to show that $I$ is prime,
  or to show that $y - x^2$ is prime.
  Since $\mathbb{C}[x,y]$ is a UFD, it suffices to show that $y - x^2$ is irreducible.

\item[(2)]
  \emph{Show that $y - x^2$ is irreducible in $\mathbb{C}[x,y]$.}
  Write
  \[
    y - x^2 \in (\mathbb{C}[y])[x].
  \]
  Note that $\mathbb{C}[y]$ is a UFD and $y$ is the constant term.
  If we can show that $y$ is prime in $\mathbb{C}[y]$, then by the Eisenstein's criterion
  we can say $y - x^2$ is irreducible in $(\mathbb{C}[y])[x]$.

\item[(3)]
  As $\mathbb{C}[y]/(y) \cong \mathbb{C}$ is a field or a domain,
  $(y)$ is maximal or prime.
  Hence, $y - x^2$ is irreducible.

\item[(4)]
  Or apply Corollary 1 to Proposition 2 in the next section to (2)(3).
\end{enumerate}
$\Box$ \\



\emph{Proof of (b).}
\begin{enumerate}
\item[(1)]
  Write
  \begin{align*}
    Y
    :=& \: V(y^4 - x^2, y^4 - x^2 y^2 + xy^2 - x^3) \\
    =& \: V((y^2 - x)(y^2 + x), (y^2 - x^2)(y^2 + x)) \\
    =& \: V(y^2 + x) \cup V(y^2 - x, y^2 - x^2) \\
    =& \: V(y^2 + x) \cup V(y^2 - x, x(x - 1)) \\
    =& \: V(y^2 + x) \cup V(x, y) \cup V(y + 1, x - 1) \cup V(y - 1, x - 1).
  \end{align*}

\item[(2)]
  Here $V(y^2 + x)$ is irreducible as (a).
  Besides,
  $V(x, y)$, $V(y + 1, x - 1)$ and $V(y - 1, x - 1)$ are irreducible
  since all corresponding ideals are maximal
  (by the Hilbert's Nullstellensatz and Problem 1.21).
\end{enumerate}
$\Box$ \\\\



%%%%%%%%%%%%%%%%%%%%%%%%%%%%%%%%%%%%%%%%%%%%%%%%%%%%%%%%%%%%%%%%%%%%%%%%%%%%%%%%



\subsubsection*{Problem 1.26.}
\addcontentsline{toc}{subsubsection}{Problem 1.26.}
\emph{Show that $f = y^2 + x^2(x-1)^2 \in \mathbb{R}[x,y]$ is an irreducible polynomial,
but $V(f)$ is reducible.} \\

\emph{Proof.}
\begin{enumerate}
\item[(1)]
  \emph{Show that $f$ is an irreducible polynomial.}
  \begin{enumerate}
  \item[(a)]
    Suppose
    \[
      f = (f_2(x) y^2 + f_1(x) y + f_0(x)) \cdot g(x)
    \]
    for some $f_i(x), g(x) \in \mathbb{R}[x]$.
    So
    \[
      f_2(x)g(x) = 1,
      \qquad
      f_1(x)g(x) = 0,
      \qquad
      f_0(x)g(x) = x^2(x-1)^2.
    \]
    Hence,
    \[
      f_2(x) y^2 + f_1(x) y + f_0(x) = uf,
      \qquad
      g(x) = u^{-1},
    \]
    where $u$ is a unit in $\mathbb{R}$.

  \item[(b)]
    Suppose
    \[
      f = (f_1(x) y + f_0(x)) \cdot (g_1(x) y + g_0(x))
    \]
    for some $f_i(x), g_j(x) \in \mathbb{R}[x]$.
    So
    \begin{align*}
      f_1(x)g_1(x) &= 1, \\
      f_1(x)g_0(x) + f_0(x)g_1(x) &= 0, \\
      f_0(x)g_0(x) &= x^2(x-1)^2.
    \end{align*}
    So $f_1(x) = u$, $g_1(x) = u^{-1}$ for some unit $u \in \mathbb{R}$.
    Hence,
    \[
      u^2 g_0(x)^2 = -x^2(x-1)^2,
    \]
    which is absurd since $\mathbb{R}$ is not algebraically closed.

  \item[(c)]
    By (a)(b), $f$ is irreducible in $\mathbb{R}[x,y]$.
  \end{enumerate}

\item[(2)]
  \emph{Show that $V(f)$ is reducible.}
  $V(f) = \{ (0,0), (1,0) \} = V(x,y) \cup V(x-1,y)$.
  Here $V(x,y)$ and $V(x-1,y)$ are all proper algebraic sets in $V(f)$.
\end{enumerate}
$\Box$ \\\\



%%%%%%%%%%%%%%%%%%%%%%%%%%%%%%%%%%%%%%%%%%%%%%%%%%%%%%%%%%%%%%%%%%%%%%%%%%%%%%%%



\subsubsection*{Problem 1.27.}
\addcontentsline{toc}{subsubsection}{Problem 1.27.}
\emph{Let $V, W$ be algebraic sets in $\mathbf{A}^n(k)$ with $V \subseteq W$.
Show that each irreducible component of $V$ is contained in some irreducible component of $W$.} \\

\emph{Proof.}
\begin{enumerate}
\item[(1)]
  Write two decompositions of $V, W$ into irreducible components as
  \begin{align*}
    V &= V_1 \cup \cdots \cup V_r, \\
    W &= W_1 \cup \cdots \cup W_s, \\
  \end{align*}

\item[(2)]
  For each irreducible component $V_i$ of $V$,
  consider $V_i \cap W$:
  \[
    V_i \cap W = (V_i \cap W_1) \cup \cdots \cup (V_i \cap W_s).
  \]
  By the irreducibility of $V_i$,
  there is only one $j$ such that $V_i \cap W_j = V_i$
  and other intersections are empty.
  Therefore, each irreducible component $V_i$ is contained in
  some irreducible component $W_j$ of $W$.
\end{enumerate}
$\Box$ \\\\



%%%%%%%%%%%%%%%%%%%%%%%%%%%%%%%%%%%%%%%%%%%%%%%%%%%%%%%%%%%%%%%%%%%%%%%%%%%%%%%%



\subsubsection*{Problem 1.28.}
\addcontentsline{toc}{subsubsection}{Problem 1.28.}
\emph{If $V = V_1 \cup \cdots \cup V_r$ is the decomposition
of an algebraic set into irreducible components,
show that $V_i \not\subseteq \bigcup_{j \neq i} V_j$.} \\

\emph{Proof.}
\begin{enumerate}
\item[(1)]
  (Reductio ad absurdum)
  If
  \[
    V_i \subseteq \bigcup_{j \neq i} V_j
  \]
  for some $i$, then
  \[
    V = V_1 \cup \cdots \cup \widehat{V_i} \cup \cdots \cup V_r
  \]
  is another decomposition of an algebraic set into irreducible components.

\item[(2)]
  By Theorem 2 in \S 1.5, the number of irreducible components is unique determined,
  contrary to the assumption and (1).
\end{enumerate}
$\Box$ \\\\



%%%%%%%%%%%%%%%%%%%%%%%%%%%%%%%%%%%%%%%%%%%%%%%%%%%%%%%%%%%%%%%%%%%%%%%%%%%%%%%%



\subsubsection*{Problem 1.29.*}
\addcontentsline{toc}{subsubsection}{Problem 1.29.*}
\emph{Show that $\mathbf{A}^n(k)$ is irreducible if $k$ is infinite.} \\

\emph{Proof.}
\begin{enumerate}
\item[(1)]
  (Reductio ad absurdum)
  If $\mathbf{A}^n(k)$ were reducible,
  then $\mathbf{A}^n(k) = V_1 \cup V_2$ where $V_1, V_2$ are algebraic sets in $\mathbf{A}^n(k)$,
  $V_1$ and $V_2$ are nonempty and proper in $\mathbf{A}^n(k)$.

\item[(2)]
  Take $P_i \in V_i$ for $i = 1, 2$.
  By Problem 1.17,
  there are two polynomials $f_1, f_2 \in k[x_1,\ldots,x_n]$
  such that $f_i(Q) = 0$ for all $Q \in V_i$ and $f_1(P_2) = f_2(P_1) = 1$.

\item[(3)]
  By construction,
  $(f_1 f_2)(a_1,\ldots,a_n) = 0$ for any $a_1,\ldots,a_n \in k$.
  As $k$ is infinite, $f_1 f_2 = 0$ by Problem 1.4.
  Since $k[x_1,\ldots,x_n]$ is a domain, $f_1 = 0$ or $f_2 = 0$,
  contrary to $f_1(P_2) = f_2(P_1) \neq 0$.
\end{enumerate}
$\Box$ \\

\emph{Note.}
  $\mathbf{A}^n(k)$ is reducible if $k$ is finite. \\\\



%%%%%%%%%%%%%%%%%%%%%%%%%%%%%%%%%%%%%%%%%%%%%%%%%%%%%%%%%%%%%%%%%%%%%%%%%%%%%%%%
%%%%%%%%%%%%%%%%%%%%%%%%%%%%%%%%%%%%%%%%%%%%%%%%%%%%%%%%%%%%%%%%%%%%%%%%%%%%%%%%



\subsection*{1.6. Algebraic Subsets of the Plane \\}
\addcontentsline{toc}{subsection}{1.6. Algebraic Subsets of the Plane}



\subsubsection*{Problem 1.30.}
\addcontentsline{toc}{subsubsection}{Problem 1.30.}
\emph{Let $k = \mathbb{R}$.}
\begin{enumerate}
\item[(a)]
  \emph{Show that $I(V(x^2+y^2+1))=(1)$.}

\item[(b)]
  \emph{Show that every algebraic subset of $\mathbf{A}^2(\mathbb{R})$ is equal to $V(f)$
  for some $f \in \mathbb{R}[x,y]$.}
\end{enumerate}
\emph{This indicates why we usually require that $k$ be algebraically closed.} \\



\emph{Proof of (a).}
$I(V(x^2+y^2+1)) = I(\varnothing) = (1)$
since $x^2+y^2+1 \geq 1$ is never zero for any $x, y \in \mathbb{R}$.
$\Box$ \\



\emph{Proof of (b).}
\begin{enumerate}
\item[(1)]
  Given any algebraic subset $V$ of $\mathbf{A}^2(\mathbb{R})$.
  $V = V(1)$ if $V = \varnothing$.
  $V = V(0)$ if $V = \mathbf{A}^2(\mathbb{R})$.
  Now suppose $V$ is a nonempty proper algebraic subset $V$ of $\mathbf{A}^2(\mathbb{R})$.
  Write $V = V_1 \cup \cdots \cup V_m$,
  where each $V_i$ is irreducible.
  Here $V_i \neq \varnothing$ and $V_i \neq \mathbf{A}^2(\mathbb{R})$ for all $i$.

\item[(2)]
  As $k = \mathbb{R}$ is infinite,
  Corollary 2 to Proposition 2 implies that each $V_i$ is either a point
  or an irreducible plane curves $V(f_i)$,
  where $f_i$ is an irreducible polynomial and $V(f_i)$ is infinite.

\item[(3)]
  If $V_i = \{ (a_i,b_i) \}$ is a point, then define
  \[
    f_i(x,y) = (x-a_i)^2 + (x-b_i)^2.
  \]
  By the property of $\mathbb{R}$, $V_i = V(f_i)$.

\item[(4)]
  Define $f = f_1 \cdots f_m \in \mathbb{R}[x,y]$.
  Hence,
  \begin{align*}
    V
    &= V_1 \cup \cdots \cup V_m \\
    &= V(f_1) \cup \cdots \cup V(f_m) \\
    &= V(f_1 \cdots f_m) \\
    &= V(f).
  \end{align*}
\end{enumerate}
$\Box$ \\\\



%%%%%%%%%%%%%%%%%%%%%%%%%%%%%%%%%%%%%%%%%%%%%%%%%%%%%%%%%%%%%%%%%%%%%%%%%%%%%%%%



\subsubsection*{Problem 1.31.}
\addcontentsline{toc}{subsubsection}{Problem 1.31.}
\begin{enumerate}
\item[(a)]
  \emph{Find the irreducible components of $V(y^2-xy-x^2y+x^3)$ in $\mathbf{A}^2(\mathbb{R})$,
  and also in $\mathbf{A}^2(\mathbb{C})$.}

\item[(b)]
  \emph{Do the same for $V(y^2 - x(x^2-1))$, and for $V(x^3+x-x^2y-y)$.} \\
\end{enumerate}



\emph{Proof of (a).}
\begin{enumerate}
\item[(1)]
  Note that
  \begin{align*}
    V(y^2-xy-x^2y+x^3)
    &= V((y-x^2)(y - x)) \\
    &= V(y-x^2) \cup V(y - x).
  \end{align*}

\item[(2)]
  Note that $y-x^2$ and $y-x$ are irreducible in $\mathbb{C}[x,y]$
  and thus also in $\mathbb{R}[x,y]$
  by the similar argument in Problem 1.25(a).
  Also, $V(y-x^2)$ and $V(y - x)$ are infinite in $\mathbf{A}^2(\mathbb{R})$
  and thus also in $\mathbf{A}^2(\mathbb{C})$.

\item[(3)]
  Therefore,
  $V(y-x^2)$ and $V(y - x)$ are the irreducible components of $V(y^2-xy-x^2y+x^3)$
  in $\mathbf{A}^2(\mathbb{R})$ and also in $\mathbf{A}^2(\mathbb{C})$.
\end{enumerate}
$\Box$ \\



\emph{Outline of (b).}
\begin{enumerate}
\item[(1)]
  The elliptic curve $V(y^2 - x(x+1)(x-1))$ is irreducible over $\mathbf{A}^2(\mathbb{R})$.

\item[(2)]
  The elliptic curve $V(y^2 - x(x+1)(x-1))$ is irreducible over $\mathbf{A}^2(\mathbb{C})$.

\item[(3)]
  The irreducible component of $V(x^3+x-x^2y-y)$ over $\mathbf{A}^2(\mathbb{R})$ is $V(x - y)$.

\item[(4)]
  The irreducible components of $V(x^3+x-x^2y-y)$ over $\mathbf{A}^2(\mathbb{C})$ are
  $V(x + i)$, $V(x - i)$ and $V(x - y)$. \\
\end{enumerate}



\emph{Proof of (b).}
\begin{enumerate}
\item[(1)]
  Similar to Problem 1.25.
  To show $y^2 - x(x+1)(x-1)$ is irreducible in $\mathbb{C}[x,y]$,
  we write
  \[
    y^2 - x(x+1)(x-1) \in (\mathbb{C}[x])[y].
  \]
  Note that $\mathbb{C}[x]$ is a UFD and $-x(x+1)(x-1)$ is the constant term.
  As $\mathbb{C}[x]/(x) \cong \mathbb{C}$ is a domain, $(x)$ is prime.
  Clearly, $x \mid x(x+1)(x-1)$ but $x^2 \nmid x(x+1)(x-1)$.
  By the Eisenstein's criterion,
  we can say $y^2 - x(x+1)(x-1)$ is irreducible over $(\mathbb{C}[x])[y]$.

\item[(2)]
  Moreover, $V(y^2 - x(x+1)(x-1))$ is infinite over $\mathbf{A}^2(\mathbb{R})$
  and thus also over $\mathbf{A}^2(\mathbb{C})$.
  ($y = f(x) = \sqrt{x(x+1)(x-1)}$ is continuous and strictly increasing on $[1,\infty)$
  in the sense of calculus.
  As the measure of $[1,\infty)$ is $\infty$, the set $V(y^2 - x(x+1)(x-1))$ is infinite
  over $\mathbf{A}^2(\mathbb{R})$.)

\item[(3)]
  By Corollary 1 to Proposition 2,
  $V(y^2 - x(x^2-1))$ itself is irreducible over $\mathbf{A}^2(\mathbb{R})$
  or $\mathbf{A}^2(\mathbb{C})$.

\item[(4)]
  Consider $V(x^3+x-x^2y-y) \subseteq \mathbf{A}^2(\mathbb{R})$.
  \begin{align*}
    V(x^3+x-x^2y-y)
    &= V((x^2 + 1)(x - y)) \\
    &= V(x^2 + 1) \cup V(x - y) \\
    &= \varnothing \cup V(x - y) \\
    &= V(x - y).
  \end{align*}
  Here we use that fact that
  $x^2 + 1 = 0$ has no real solution $x \in \mathbb{R}$.
  Similar to (a), $V(x - y)$ is the only irreducible component of $V(x^3+x-x^2y-y)$
  in $\mathbf{A}^2(\mathbb{R})$.

\item[(5)]
  Consider $V(x^3+x-x^2y-y) \subseteq \mathbf{A}^2(\mathbb{C})$.
  \begin{align*}
    V(x^3+x-x^2y-y)
    &= V((x + i)(x - i)(x - y)) \\
    &= V(x + i) \cup V(x - i) \cup V(x - y).
  \end{align*}
  Similar to (a), $V(x \pm i)$ and $V(x - y)$ are the irreducible components of $V(x^3+x-x^2y-y)$
  in $\mathbf{A}^2(\mathbb{C})$.
\end{enumerate}
$\Box$ \\\\



%%%%%%%%%%%%%%%%%%%%%%%%%%%%%%%%%%%%%%%%%%%%%%%%%%%%%%%%%%%%%%%%%%%%%%%%%%%%%%%%
%%%%%%%%%%%%%%%%%%%%%%%%%%%%%%%%%%%%%%%%%%%%%%%%%%%%%%%%%%%%%%%%%%%%%%%%%%%%%%%%



\subsection*{1.7. Hilbert's Nullstellensatz \\}
\addcontentsline{toc}{subsection}{1.7. Hilbert's Nullstellensatz}



\subsubsection*{Problem 1.32.}
\addcontentsline{toc}{subsubsection}{Problem 1.32.}
\emph{Show that both theorems and all of the corollaries are false
if $k$ is not algebraically closed.} \\



\emph{Proof.}
\begin{enumerate}
\item[(1)]
  Weak Nullstellensatz:
  $I = (x^2+1)$ is a proper ideal in $\mathbb{R}[x]$ but $V(I) = \varnothing$.

\item[(2)]
  Hilbert's Nullstellensatz:
  Let $I = (y^2 + x^2(x-1)^2)$ be an ideal in $\mathbb{R}[x,y]$.
  Hence,
  \begin{align*}
    I(V(I))
    &= I(\{ (0,0), (1,0) \})
      &(\text{Problem 1.26.}) \\
    &= (x(x-1),y) \\
    &\neq I \\
    &= \mathrm{rad}(I).
  \end{align*}
  The last equality holds since $f$ is irreducible in a UFD $\mathbb{R}[x,y]$
  and thus $I$ is a prime ideal.

\item[(3)]
  Corollary 1: Same example in the case Hilbert's Nullstellensatz.
  If $I = (y^2 + x^2(x-1)^2)$ is a radical ideal in $\mathbb{R}[x,y]$.
  Then $I(V(I)) \neq I$.

\item[(4)]
  Corollary 2: Same example in the case Hilbert's Nullstellensatz.
  If $I = (y^2 + x^2(x-1)^2)$ is a prime ideal in $\mathbb{R}[x,y]$,
  then
  \[
    V(I) = \{ (0,0), (1,0) \} = V(x,y) \cup V(x-1,y)
  \]
  is reducible.
  Next, consider a prime ideal $J = (x^2+y^2)$ in $\mathbb{R}[x,y]$.
  (Use the same argument in Problem 1.26 to get the irreducibility of $x^2+y^2$.)
  $V(J) = \{ (0,0) \}$ is a point but $J$ is not a maximal ideal
  (since $J \subsetneq (x^2+y^2,x) \subsetneq (1)$).

\item[(5)]
  Corollary 3: Same example in Corollary 2.

\item[(6)]
  Corollary 4:
  Let $I = (x^2 + y^2)$ be an ideal in $\mathbb{R}[x,y]$.
  Then $V(I) = \{ (0,0) \}$ is a finite set.
  But $\mathbb{R}[x,y]/(x^2 + y^2)$ is an infinite dimensional vector space over $\mathbb{R}$.
  In fact, the monomials
  \[
    \{ \overline{x^m}, \overline{x^m y} : m = 0, 1, 2, \ldots \}
  \]
  is a basis for $\mathbb{R}[x,y]/(x^2 + y^2)$.
\end{enumerate}
$\Box$ \\\\



%%%%%%%%%%%%%%%%%%%%%%%%%%%%%%%%%%%%%%%%%%%%%%%%%%%%%%%%%%%%%%%%%%%%%%%%%%%%%%%%



\subsubsection*{Problem 1.33.}
\addcontentsline{toc}{subsubsection}{Problem 1.33.}

\begin{enumerate}
\item[(a)]
  \emph{Decompose $V(x^2+y^2-1, x^2-z^2-1) \subseteq \mathbf{A}^3(\mathbb{C})$
  into irreducible components.}

\item[(b)]
  \emph{Let $V = \{(t,t^2,t^3) \in \mathbf{A}^3(\mathbb{C}) : t \in \mathbb{C} \}$.
  Find $I(V)$, and show that $V$ is irreducible.} \\
\end{enumerate}



\emph{Proof of (a).}
\begin{enumerate}
\item[(1)]
  Write
  \begin{align*}
    &\: V(x^2+y^2-1, x^2-z^2-1) \\
    =&\: V(x^2+y^2-1, y^2+z^2) \\
    =&\: V(x^2+y^2-1, (y + iz)(y - iz)) \\
    =&\: V(x^2+y^2-1, y + iz) \cup V(x^2+y^2-1, y - iz).
  \end{align*}
  By the Hilbert's Nullstellensatz,
  it suffices to show that $(x^2+y^2-1, y + iz)$ and $(x^2+y^2-1, y - iz)$ are prime.

\item[(2)]
  \emph{Show that $I = (x^2+y^2-1, y + iz)$ is prime in $\mathbb{C}[x,y,z]$.}
  Note that
  \[
    \mathbb{C}[x,y,z]/I \cong \mathbb{C}[x,y]/(x^2+y^2-1)
  \]
  is a ring isomorphism defined by
  \[
    f(x,y,z) + I \mapsto f(x,y,-iy) + (x^2+y^2-1).
  \]
  (Use the similar argument in (b) to prove it is indeed an isomorphism.)
  So it suffices to show that
  \[
    x^2+y^2-1 \in \mathbb{C}[x,y]
  \]
  is irreducible.
  (Thus, $\mathbb{C}[x,y]/(x^2+y^2-1) \cong \mathbb{C}[x,y,z]/I$ is a domain, or $I$ is prime.)
  We can use the similar argument in Problem 1.31 (b) to show
  $x^2+y^2-1 = y^2 + (x+1)(x-1)$ is irreducible
  as showing the irreducibility of $y^2 - x(x+1)(x-1)$.

\item[(3)]
  Similarly, $I = (x^2+y^2-1, y - iz)$ is prime.
  Therefore,
  the irreducible components of $V(x^2+y^2-1, x^2-z^2-1)$
  are $V(x^2+y^2-1, y + iz)$ and $V(x^2+y^2-1, y - iz)$.
\end{enumerate}
$\Box$ \\



\emph{Proof of (b).}
\begin{enumerate}
\item[(1)]
  Write
  \[
    V = \{ (t,t^2,t^3) \in \mathbf{A}^3(\mathbb{C}) : t \in \mathbb{C} \}
    =
    V(x^2-y, x^3-z).
  \]
  Let $I = (x^2-y, x^3-z)$ in $\mathbb{C}[x,y,z]$.
  By the Hilbert's Nullstellensatz,
  $I(V) = \mathrm{rad}(I)$.
  So it suffices to show that $I = (x^2-y, x^3-z)$ is prime
  (and thus $V$ is irreducible).

\item[(2)]
  \emph{Show that
  \[
    \mathbb{C}[x,y,z]/I \cong \mathbb{C}[t]
  \]
  is a domain, and thus $I = (x^2-y, x^3-z)$ is a prime ideal.}
  \begin{enumerate}
  \item[(a)]
    Define a ring homomorphism $\alpha: \mathbb{C}[x,y,z]/I \to \mathbb{C}[t]$
    by
    \[
      \alpha: f(x,y,z) + I \mapsto f(t,t^2,t^3).
    \]
    $\alpha$ is well-defined since $\alpha((x^2-y)+I) = 0$ and $\alpha((x^3-z)+I) = 0$.

  \item[(b)]
    \emph{Show that $\alpha$ is surjective.}
    \[
      \alpha: g(x) + I \in \mathbb{C}[x,y,z]/I \mapsto g(t) \in \mathbb{C}[t]
    \]
    for any $g(t)$.

  \item[(c)]
    \emph{Show that $\alpha$ is injective.}
    Suppose $\alpha(f(x,y,z) + I) = 0$.
    Write
    \begin{align*}
      f(x,y,z) + I
      &= \sum_{(i)} \lambda_{(i)} x^{i_1} (y-x^2)^{i_2} (z-x^3)^{i_3} + I \\
      &= \sum_{i} \lambda_{i} x^{i} + I.
    \end{align*}
    So
    \[
      0
      = \alpha(f(x,y,z)+I)
      = \alpha\left(\sum_{i} \lambda_{i} x^{i}+I\right)
      = \sum_{i} \lambda_{i} t^{i}.
    \]
    Hence, $\ker(\alpha) = I$.
  \end{enumerate}
\end{enumerate}
$\Box$ \\\\



%%%%%%%%%%%%%%%%%%%%%%%%%%%%%%%%%%%%%%%%%%%%%%%%%%%%%%%%%%%%%%%%%%%%%%%%%%%%%%%%



\subsubsection*{Problem 1.34.}
\addcontentsline{toc}{subsubsection}{Problem 1.34.}
\emph{Let $R$ be a UFD.}
\begin{enumerate}
\item[(a)]
  \emph{Show that a monic polynomial of degree two or three in $R[x]$ is irreducible
  if and only if it has no root in $R$.}

\item[(b)]
  \emph{$x^2-a \in R[x]$ is irreducible if and only if $a$ is not a square in $R$.} \\
\end{enumerate}



\emph{Proof of (a).}
\begin{enumerate}
\item[(1)]
  It is equivalent to show that
  a monic polynomial of degree two or three in $R[x]$ is reducible
  if and only if it has one root in $R$.

\item[(2)]
  Suppose $f$ is reducible of degree $2$ or $3$.
  Then there exist nonconstant monic polynomials $g, h \in R[x]$ such that $f = gh$.
  By
  \[
    \deg(g) + \deg(h) = \deg(f) = 2 \text{ or } 3,
  \]
  we may assume that $\deg(g) = 1$. (Otherwise $g$ or $h$ will be a constant polynomial.)
  Say $g(x) = x - a$ where $a \in R$.
  Now
  \[
    f(a) = g(a)h(a) = 0
  \]
  implies that $a \in R$ is a root of $f$.

\item[(3)]
  Conversely, if $a \in R$ is a root of $f$, then
  apply the same argument in Problem 1.7 we can write
  \[
    f = (x - a)g
  \]
  for some $g \in R[x]$.
  Here $\deg(g) \geq 1$ since $\deg(f) = 1 + \deg(g) \geq 2$.
  Therefore, $f$ is reducible.
\end{enumerate}
$\Box$ \\



\emph{Proof of (b).}
  By (a),
  $x^2-a \in R[x]$ is reducible $\Longleftrightarrow$
  $x^2-a$ has one root $\alpha \in R$ $\Longleftrightarrow$
  $a = \alpha^2$ is a square in $R$ for some $\alpha \in R$.
$\Box$ \\\\



%%%%%%%%%%%%%%%%%%%%%%%%%%%%%%%%%%%%%%%%%%%%%%%%%%%%%%%%%%%%%%%%%%%%%%%%%%%%%%%%



\subsubsection*{Problem 1.35.}
\addcontentsline{toc}{subsubsection}{Problem 1.35.}
\emph{Show that $V(y^2-x(x-1)(x-\lambda)) \subseteq \mathbf{A}^2(k)$
is an irreducible curve for any algebraically closed field $k$,
and any $\lambda \in k$.} \\

\emph{Proof.}
\begin{enumerate}
\item[(1)]
  By the Hilbert's Nullstellensatz, it suffices to show that
  \[
    I = (y^2-x(x-1)(x-\lambda))
  \]
  is a prime ideal in $k[x,y]$, or show that
  \[
    y^2-x(x-1)(x-\lambda)
  \]
  is irreducible (since $k[x,y]$ is a UFD).

\item[(2)]
  By Problem 1.34(b),
  $y^2-x(x-1)(x-\lambda) \in (\mathbb{C}[x])[y]$ is irreducible
  if $x(x-1)(x-\lambda)$ is not a square in $\mathbb{C}[x]$.
  Note that every square in $\mathbb{C}[x]$ is of even degree.
  So $x(x-1)(x-\lambda)$ cannot be a square in $\mathbb{C}[x]$
  since $\deg(x(x-1)(x-\lambda)) = 3$ is odd.
\end{enumerate}
$\Box$ \\

\emph{Note.}
  $V(y^2-x(x-1)(x-\lambda))$ is the elliptic curve as Problem 1.31. \\\\



%%%%%%%%%%%%%%%%%%%%%%%%%%%%%%%%%%%%%%%%%%%%%%%%%%%%%%%%%%%%%%%%%%%%%%%%%%%%%%%%



\subsubsection*{Problem 1.36.}
\addcontentsline{toc}{subsubsection}{Problem 1.36.}
\emph{Let $I = (y^2-x^2, y^2+x^2) \subseteq \mathbb{C}[x,y]$.
Find $V(I)$ and $\dim_{\mathbb{C}}(\mathbb{C}[x,y]/I)$.} \\

\emph{Proof.}
\begin{enumerate}
\item[(1)]
  Clearly, $V(I) = \{ (0,0) \}$ is a finite set.
  By Corollary 4 to the Hilbert's Nullstellensatz,
  \[
    \dim_{\mathbb{C}}(\mathbb{C}[x,y]/I) < \infty.
  \]
  In fact, $\dim_{\mathbb{C}}(\mathbb{C}[x,y]/I) = 4$.

\item[(2)]
  Given any $f + I \in \mathbb{C}[x,y]/I$ where $f \in \mathbb{C}[x,y]$.
  Write
  \[
    f(x,y) = \sum_{i} f_i(x) y^i
  \]
  where $f_i(x) = \sum_{j} a_{ij} x^{j} \in \mathbb{C}[x]$.
  Note that
  \begin{align*}
    x^2 &= \frac{1}{2}(y^2+x^2) - \frac{1}{2}(y^2-x^2) \in I, \\
    y^2 &= \frac{1}{2}(y^2+x^2) + \frac{1}{2}(y^2-x^2) \in I.
  \end{align*}
  So
  \begin{align*}
    f(x,y) + I
    &= \sum_{i} f_i(x) y^i + I \\
    &= f_0(x) + f_1(x) y + I \\
    &= \sum_{j} a_{0j} x^{j} + \left(\sum_{j} a_{1j} x^{j}\right) y + I \\
    &= a_{00} + a_{01} x + a_{10} y + a_{11} xy + I
  \end{align*}
  is generated by $\mathscr{B} = \{ \overline{1}, \overline{x}, \overline{y}, \overline{xy} \}$.

\item[(3)]
  Note that $\mathscr{B}$ is a basis
  since any linear combination of elements in $\mathscr{B}$ is not in $I$.
  Therefore,
  \[
    \dim_{\mathbb{C}}(\mathbb{C}[x,y]/I) = |\mathscr{B}| = 4.
  \]
\end{enumerate}
$\Box$ \\\\



%%%%%%%%%%%%%%%%%%%%%%%%%%%%%%%%%%%%%%%%%%%%%%%%%%%%%%%%%%%%%%%%%%%%%%%%%%%%%%%%



\subsubsection*{Problem 1.37.*}
\addcontentsline{toc}{subsubsection}{Problem 1.37.*}
\emph{Let $K$ be any field, $f \in K[x]$ a polynomial of degree $n > 0$.
Show that the residues $\overline{1}, \overline{x}, \ldots, \overline{x}^{n-1}$
form a basis for $K[x]/(f)$ over $K$.} \\

\emph{Proof.}
\begin{enumerate}
\item[(1)]
  \emph{Show that every element in $K[x]/(f)$ is generated by
  $\mathscr{B} = \{ \overline{1}, \overline{x}, \ldots, \overline{x}^{n-1} \}$.}
  Given any $\overline{g} \in K[x]/(f)$ with $g \in K[x]$.
  By the division-with-remainder property of $K[x]$,
  there are some polynomials $q, r \in K[x]$ such that
  \[
    g = fq + r
  \]
  where $r = 0$ or $\deg(r) < n$ if $r \neq 0$.
  Therefore,
  \[
    g + (f) = fq + r + (f) = r + (f).
  \]
  Note that $r + (f)$ is generated by $\mathscr{B}$.

\item[(2)]
  \emph{Show that $\mathscr{B}$ is a basis for $K[x]/(f)$ over $K$.}
  Suppose
  \[
    a_0 + a_1 x + \cdots + a_{n-1} x^{n-1} \in (f)
  \]
  for $a_1, \ldots, a_{n-1} \in K$.
  We can regard any linear combination of $\{ 1, x, \ldots, x^{n-1} \}$
  as a polynomial $r(x)$ in $K[x]$.
  $r \in (f)$ implies that there exists a polynomial $g \in K[x]$
  such that $r = fg$.
  If $g \neq 0$, then $\deg(r) = \deg(f) + \deg(g) \geq n$, which is impossible.
  So $g = 0$ and thus $r = fg = 0 \in K[x]$.
  Therefore, $a_0 = a_1 = \cdots = a_{n-1} = 0 \in K$ and
  \[
    \dim_{K}(K[x]/(f)) = \deg(f).
  \]
\end{enumerate}
$\Box$ \\\\



%%%%%%%%%%%%%%%%%%%%%%%%%%%%%%%%%%%%%%%%%%%%%%%%%%%%%%%%%%%%%%%%%%%%%%%%%%%%%%%%



\subsubsection*{Problem 1.38.*}
\addcontentsline{toc}{subsubsection}{Problem 1.38.*}
\emph{Let $R = k[x_1,\ldots,x_n]$, $k$ algebraically closed, $V = V(I)$.
Show that there is a natural one-to-one correspondence
between algebraic subsets of $V$ and radical ideals in $k[x_1,\ldots,x_n]/I$,
and that irreducible algebraic sets (resp. points) correspond to prime ideals (resp. maximal ideals).
(See Problem 1.22.)} \\



\emph{Proof.}
\begin{enumerate}
\item[(1)]
  Given any algebraic subset $W$ of $V$.
  By the Hilbert's Nullstellensatz,
  \[
    I(W) \supseteq I(V) = \mathrm{rad}(I) \supseteq I.
  \]

\item[(2)]
  By Corollary 1 to the Hilbert's Nullstellensatz and Problem 1.22(b),
  we have a one-to-one correspondence such that
  \begin{align*}
    &\: \{ \text{algebraic subsets of $V$} \} \\
    \longleftrightarrow&\:
    \{ \text{radical ideals containing $I$} \} \\
    \longleftrightarrow&\:
    \{ \text{radical ideals of $k[x_1,\ldots,x_n]/I$} \}.
  \end{align*}

\item[(3)]
  Again
  by Corollary 2 to the Hilbert's Nullstellensatz and Problem 1.22(b),
  we have a one-to-one correspondence such that
  \begin{align*}
    &\: \{ \text{irreducible algebraic subsets (resp. points) of $V$} \} \\
    \longleftrightarrow&\:
    \{ \text{prime (resp. maximal) ideals containing $I$} \} \\
    \longleftrightarrow&\:
    \{ \text{prime (resp. maximal) ideals of $k[x_1,\ldots,x_n]/I$} \}.
  \end{align*}
\end{enumerate}
$\Box$ \\\\



%%%%%%%%%%%%%%%%%%%%%%%%%%%%%%%%%%%%%%%%%%%%%%%%%%%%%%%%%%%%%%%%%%%%%%%%%%%%%%%%



\subsubsection*{Problem 1.39.}
\addcontentsline{toc}{subsubsection}{Problem 1.39.}
\begin{enumerate}
\item[(a)]
  \emph{Let $R$ be a UFD, and let $\mathfrak{p} = (t)$ be a principal proper prime ideal.
  Show that there is no prime ideal $\mathfrak{q}$
  such that $0 \subsetneq \mathfrak{q} \subsetneq \mathfrak{p}$.}

\item[(b)]
  \emph{Let $V = V(f)$ be irreducible hypersurface in $\mathbf{A}^n$.
  Show that there is no irreducible algebraic set $W$ such that
  $V \subsetneq W \subsetneq \mathbf{A}^n$.} \\
\end{enumerate}



\emph{Proof of (a).}
\begin{enumerate}
\item[(1)]
  (Reductio ad absurdum)
  Suppose that $\mathfrak{q}$ were a prime ideal in $R$ such that
  $0 \subsetneq \mathfrak{q} \subsetneq \mathfrak{p}$.

\item[(2)]
  \emph{Show that there is an irreducible element in $\mathfrak{q}$.}
  Given any $q \in \mathfrak{q}$. Since $\mathfrak{q}$ is proper,
  we can write
  \[
    q = q_1 \cdots q_n
  \]
  as a product of irreducible elements in a UFD.
  Since $\mathfrak{q}$ is prime, there is one irreducible element $q_i \in \mathfrak{q}$.

\item[(3)]
  Now $q_i \in \mathfrak{q} \subseteq \mathfrak{p} = (t)$.
  So $q_i = ut$ for some $u \in R$.
  By the irreducibility of $q_i$, $u$ is a unit or $t$ is a unit.
  If $u$ is a unit, then
  \[
    (t) = (q_i) \subseteq \mathfrak{q} \subseteq \mathfrak{p} = (t).
  \]
  So $\mathfrak{q} = \mathfrak{p}$, which is absurd.
  If $t$ is a unit, then $\mathfrak{p} = (1)$, contrary to the primality of $\mathfrak{p}$.
\end{enumerate}
$\Box$ \\



\emph{Proof of (b).}
\begin{enumerate}
\item[(1)]
  We might assume that $k = \overline{k}$.
  By Corollary 3 to the Hilbert's Nullstellensatz and the irreducibility of $V(f)$,
  there are an irreducible polynomial $g \in k[x_1,\ldots,x_n]$ and an integer $m > 0$
  such that
  \[
    f = g^m,
  \]
  and
  \[
    I(V(f)) = (g).
  \]

\item[(2)]
  (Reductio ad absurdum)
  Suppose that there were an irreducible algebraic set $W$ such that
  $V \subsetneq W \subsetneq \mathbf{A}^n$.
  Then by Corollary 3 to the Hilbert's Nullstellensatz again,
  \[
    (g) = I(V(f)) \supsetneq I(W) \supsetneq (1) \in k[x_1,\ldots,x_n].
  \]
  Here $(g) = I(V(f))$ and $I(W)$ are all prime.

\item[(3)]
  Note that $(g)$ is a principal proper prime ideal in a UFD $k[x_1,\ldots,x_n]$.
  By (a), such ideal $I(W)$ cannot be prime, which is absurd.
\end{enumerate}
$\Box$ \\\\



%%%%%%%%%%%%%%%%%%%%%%%%%%%%%%%%%%%%%%%%%%%%%%%%%%%%%%%%%%%%%%%%%%%%%%%%%%%%%%%%



\subsubsection*{Problem 1.40.}
\addcontentsline{toc}{subsubsection}{Problem 1.40.}
\emph{Let $I = (x^2-y^3, y^2-z^3) \subseteq k[x,y,z]$.
Define $\alpha: k[x,y,z] \to k[t]$ by
$\alpha(x) = t^9$, $\alpha(y) = t^6$, $\alpha(z) = t^4$.}

\begin{enumerate}
\item[(a)]
  \emph{Show that every element of $k[x,y,z]/I$ is the residue of an element
  $a + xb + yc + xyd$, for some $a, b, c, d \in k[z]$.}

\item[(b)]
  \emph{If $f = a + xb + yc + xyd$, $a, b, c, d \in k[z]$ and $\alpha(f) = 0$,
  compare like powers of $t$ to conclude that $f = 0$.}

\item[(c)]
  \emph{Show that $\ker(\alpha) = I$, so $I$ is prime,
  $V(I)$ is irreducible, and $I(V(I)) = I$.} \\
\end{enumerate}



\emph{Proof of (a).}
\begin{enumerate}
\item[(1)]
  Take any element $\overline{f} \in k[x,y,z]/I$ where $f \in k[x,y,z]$.
  Regard $f \in (k[y,z])[x]$,
  By the division-with-remainder property of $(k[y,z])[x]$,
  \[
    f = (x^2 - y^3)q + r
  \]
  where $q, r \in (k[y,z])[x]$ and $r = 0$ or $\deg_{x}(r) < 2$.
  In any case,
  $r = x r_1 + r_0$ for some $r_1, r_0 \in k[y,z]$.

\item[(2)]
  Apply the same argument to (1),
  we have
  \begin{align*}
    r_0 &= (y^2 - z^3)q_0 + y c + a \\
    r_1 &= (y^2 - z^3)q_1 + y d + b
  \end{align*}
  where $q_0, q_1 \in k[y,z]$ and $a, b, c, d \in k[z]$.

\item[(3)]
  By $\overline{r_0} = \overline{y c} + \overline{a}$
  and $\overline{r_1} = \overline{y d} + \overline{b}$,
  \begin{align*}
    \overline{f}
    &= \overline{r} \\
    &= \overline{x} \overline{r_1} + \overline{r_0} \\
    &= \overline{x} (\overline{y d} + \overline{b}) + (\overline{y c} + \overline{a}) \\
    &= \overline{a} + \overline{b} \cdot \overline{x}
      + \overline{c} \cdot \overline{y} + \overline{d} \cdot \overline{xy}.
  \end{align*}
\end{enumerate}
$\Box$ \\



\emph{Proof of (b).}
  As $0 = \alpha(f) = a + ct^6 + bt^9 + dt^{15} \in k[t]$,
  $a = b = c = d = 0 \in k$.
$\Box$ \\



\emph{Proof of (c).}
\begin{enumerate}
\item[(1)]
  $I \subseteq \ker(\alpha)$ is trivial.

\item[(2)]
  \emph{Show that $\ker(\alpha) \subseteq I$.}
  Take any $f \in \ker(\alpha)$, or $\alpha(f) = 0$.
  By (a), $f = r + f_1$ where $f_1 \in I$ and $r = a + bx + cy + dxy \in k[x,y,z]$
  for some $a, b, c, d \in k[z]$.
  Note that $\alpha$ is a ring homomorphism.
  Therefore,
  \[
    0 = \alpha(f) = \alpha(r + f_1) = \alpha(r) + \alpha(g) = \alpha(r).
  \]
  By (b), $r = 0 \in k[x,y,z]$ and thus $f = f_1 \in I$.

\item[(3)]
  Therefore,
  \[
    \alpha: k[x,y,z]/(x^2-y^3, y^2-z^3) \hookrightarrow k[t]
  \]
  is injective.
\end{enumerate}
$\Box$ \\\\



%%%%%%%%%%%%%%%%%%%%%%%%%%%%%%%%%%%%%%%%%%%%%%%%%%%%%%%%%%%%%%%%%%%%%%%%%%%%%%%%
%%%%%%%%%%%%%%%%%%%%%%%%%%%%%%%%%%%%%%%%%%%%%%%%%%%%%%%%%%%%%%%%%%%%%%%%%%%%%%%%



\subsection*{1.8. Modules; Finiteness Conditions \\}
\addcontentsline{toc}{subsection}{1.8. Modules; Finiteness Conditions}



\subsubsection*{Problem 1.41.*}
\addcontentsline{toc}{subsubsection}{Problem 1.41.*}
\emph{If $S$ is module-finite over $R$, then $S$ is ring-finite over $R$.} \\

\emph{Proof.}
\begin{enumerate}
\item[(1)]
  Write $S = \sum R s_i$ for some $s_1, \ldots, s_n \in S$
  since $S$ is module-finite over $R$.

\item[(2)]
  \emph{Show that $\sum R s_i = R[s_1,\ldots,s_n]$.}
  $\sum R s_i \subseteq R[s_1,\ldots,s_n]$ is trivial.
  Conversely, take any $v \in R[s_1,\ldots,s_n]$.
  Write
  \[
    v = \sum_{(j)} \overbrace{
      \underbrace{a_{(j)}}_{\in R}
      \underbrace{s_1^{j_1} \cdots s_n^{j_n}}_{\in S = \sum R s_i}}^{\in \sum R s_i}
  \]
  Here each term $a_{(i)} s_1^{i_1} \cdots s_n^{i_n}$ is in $\sum R s_i$.
  As $\sum R s_i$ is an $R$-module,
  \[
    v = \sum_{(i)} a_{(i)} s_1^{i_1} \cdots s_n^{i_n} \in \sum R s_i.
  \]
\end{enumerate}
$\Box$ \\



\emph{Note.}
  The converse is not true (by Problem 1.42). \\\\



%%%%%%%%%%%%%%%%%%%%%%%%%%%%%%%%%%%%%%%%%%%%%%%%%%%%%%%%%%%%%%%%%%%%%%%%%%%%%%%%



\subsubsection*{Problem 1.42.}
\addcontentsline{toc}{subsubsection}{Problem 1.42.}
\emph{Show that $S = R[x]$ (the ring of polynomials in one variable)
is ring-finite over $R$, but not module-finite.} \\

\emph{Proof.}
\begin{enumerate}
\item[(1)]
  $S = R[x]$ is ring-finite over $R$ by definition (as $x \in S$).

\item[(2)]
  (Reductio ad absurdum)
  If $S = \sum R s_i$ for some $s_1, \ldots, s_n \in S$ were module-finite over $R$.
  Any element $s \in \sum R s_i$ is of degree
  \[
    \deg s \leq \max_{1 \leq i \leq n} \deg s_i := m.
  \]
  So that $x^{m+1} \in S = R[x]$ but not in $\sum R s_i$,
  which is absurd.
\end{enumerate}
$\Box$ \\\\



%%%%%%%%%%%%%%%%%%%%%%%%%%%%%%%%%%%%%%%%%%%%%%%%%%%%%%%%%%%%%%%%%%%%%%%%%%%%%%%%



\subsubsection*{Problem 1.43.*}
\addcontentsline{toc}{subsubsection}{Problem 1.43.*}
\emph{If $L$ is ring-finite over $K$ ($K$, $L$ fields)
then $L$ is a finitely generated field extension of $K$.} \\

\emph{Proof.}
\begin{enumerate}
\item[(1)]
  $L = K[v_1, \cdots, v_n]$ for some $v_i \in L$ since $L$ is ring-finite over $K$.

\item[(2)]
  Apply Proposition 4 in \S 1.10,
  $L$ is module-finite (and hence algebraic) over $K$,
  that is, $L = K[v_1, \cdots, v_n] = K(v_1, \cdots, v_n)$
  is a finitely generated field extension of $K$.
\end{enumerate}
$\Box$ \\\\



%%%%%%%%%%%%%%%%%%%%%%%%%%%%%%%%%%%%%%%%%%%%%%%%%%%%%%%%%%%%%%%%%%%%%%%%%%%%%%%%



\subsubsection*{Problem 1.44.*}
\addcontentsline{toc}{subsubsection}{Problem 1.44.*}
\emph{Show that $L = K(x)$ (the field of rational functions in one variable)
is a finitely generated field extension of $K$, but $L$ is not ring-finite over $K$.
(Hint: If $L$ were ring-finite over $K$,
a common denominator of ring generators would be an element $b \in K[x]$ such that
for all $z \in L$, $b^n z \in K[x]$ for some $n$;
but let $z = 1/c$, where $c$ doesn't divide $b$ (Problem 1.5).)} \\

\emph{Proof.}
\begin{enumerate}
\item[(1)]
  (Reductio ad absurdum)
  Suppose that $L$ were ring-finite over $K$.
  Write $L = K[v_1,\ldots,v_m]$
  where $v_1,\ldots,v_m \in L = K(x)$.
  Let $b \in K[x]$ be a common denominator of ring generators $v_1, \ldots, v_m$.
  (So that all $bv_i \in K[x]$.)
  Therefore, for any $z \in L = K[v_1,\ldots,v_m]$, there is an integer $n > 0$
  such that $b^n z \in K[x]$.

\item[(2)]
  Consider $z = 1/c \in K(x)$, where $c \in K[x]$ doesn't divide $b$.
  The existence of $c$ is guaranteed by Problem 1.5.
  Hence, for any integer $n > 0$
  \[
    b^n z = b^n/c
  \]
  is never in $K[x]$ by the construction of $c$, which is absurd.
\end{enumerate}
$\Box$ \\\\



%%%%%%%%%%%%%%%%%%%%%%%%%%%%%%%%%%%%%%%%%%%%%%%%%%%%%%%%%%%%%%%%%%%%%%%%%%%%%%%%



\subsubsection*{Problem 1.45.*}
\addcontentsline{toc}{subsubsection}{Problem 1.45.*}
\emph{Let $R$ be a subring of $S$, $S$ a subring of $T$.}
\begin{enumerate}
\item[(a)]
  \emph{If $S = \sum R v_i$, $T = \sum S w_j$, show that $T = \sum R v_i w_j$.}

\item[(b)]
  \emph{If $S = R[v_1,\ldots,v_n]$, $T = S[w_1,\ldots,w_m]$,
  show that $T = R[v_1,\ldots,v_n,w_1,\ldots,w_m]$.}

\item[(c)]
  \emph{If $R$, $S$, $T$ are fields, and $S = R(v_1,\ldots,v_n)$, $T = S(w_1,\ldots,w_m)$,
  show that $T = R(v_1,\ldots,v_n,w_1,\ldots,w_m)$.}
\end{enumerate}
\emph{So each of the three finiteness conditions is a transitive relation.} \\



\emph{Proof of (a).}
\begin{enumerate}
\item[(1)]
  \emph{Show that $T \subseteq \sum R v_i w_j$.}
  Given any $t \in T = \sum S w_j$.
  There are some $s_j \in S$ such that $t = \sum_{j} s_j w_j$.
  As $s_j \in S = \sum R v_i$, there are some $r_{ij} \in R$ such that $s_j = \sum_{i} r_{ij} v_i$.
  Hence,
  \[
    t
    = \sum_{j} s_j w_j
    = \sum_{j} \left(\sum_{i} r_{ij} v_i\right) w_j
    = \sum_{i,j} r_{ij} v_i w_j
    \in \sum R v_i w_j.
  \]

\item[(2)]
  \emph{Show that $T \supseteq \sum R v_i w_j$.}
  Take any $\sum r_{ij} v_i w_j \in \sum R v_i w_j$.
  \[
    \sum r_{ij} v_i w_j
    = \sum_{j} \left(\sum_{i} r_{ij} v_i\right) w_j
    \in \sum_{j} S w_j = T.
  \]

\end{enumerate}
$\Box$ \\



\emph{Proof of (b).}
\begin{enumerate}
\item[(1)]
  Note that $R[x_1,\cdots,x_m]$ is canonically isomorphic to $R[x_1,\cdots,x_{m-1}][x_m]$.
  Hence $R[x_1,\cdots,x_m]$ is isomorphic to $R[x_1][x_2] \cdots [x_m]$.

\item[(2)]
  Hence,
  \begin{align*}
    T
    &= S[w_1,\ldots,w_m] \\
    &= R[v_1,\ldots,v_n][w_1,\ldots,w_m] \\
    &= R[v_1,\ldots,v_n][w_1] \cdots [w_m] \\
    &= R[v_1] \cdots [v_n][w_1] \cdots [w_m] \\
    &= R[v_1,\ldots,v_n,w_1,\ldots,w_m].
  \end{align*}
\end{enumerate}
$\Box$ \\



\emph{Proof of (c).}
\begin{enumerate}
\item[(1)]
  By (b),
  $R(v_1,\ldots,v_n)$ is canonically isomorphic to $R(v_1,\ldots,v_{n-1})(v_n)$.
  Hence $R(v_1,\ldots,v_n)$ is isomorphic to $R(v_1) \cdots (v_n)$.
  To see this, note that $R[x_1,\cdots,x_m] \cong R[x_1,\cdots,x_{m-1}][x_m]$
  implies that
  \[
    R(x_1,\cdots,x_m)
    \cong R[x_1,\cdots,x_{m-1}](x_m)
    \hookrightarrow R(x_1,\cdots,x_{m-1})(x_m).
  \]
  Conversely,
  for any $a/b \in R(x_1,\cdots,x_{m-1})(x_m)$
  where
  \begin{align*}
    a &= \sum_{i} a_i x_m^{i} \in R(x_1,\cdots,x_{m-1})[x_m], \\
    b &= \sum_{j} b_j x_m^{j} \in R(x_1,\cdots,x_{m-1})[x_m]
  \end{align*}
  and $b \neq 0$, there is a nonzero polynomial $c \in R[x_1,\cdots,x_{m-1}]$
  such that all $c a_i$ and $c b_j$ are in $R[x_1,\cdots,x_{m-1}]$.
  Hence,
  \begin{align*}
    \frac{a}{b}
    &= \frac{\sum_{i} a_i x_m^{i}}{\sum_{j} b_j x_m^{j}} \\
    &= \frac{c\sum_{i} a_i x_m^{i}}{c\sum_{j} b_j x_m^{j}} \\
    &= \frac{\sum_{i} ca_i x_m^{i}}{\sum_{j} cb_j x_m^{j}} \\
    &\in R[x_1,\cdots,x_{m-1}](x_m).
  \end{align*}

\item[(2)]
  Hence,
  \begin{align*}
    T
    &= S(w_1,\ldots,w_m) \\
    &= R(v_1,\ldots,v_n)(w_1,\ldots,w_m) \\
    &= R(v_1,\ldots,v_n)(w_1) \cdots (w_m) \\
    &= R(v_1) \cdots (v_n)(w_1) \cdots (w_m) \\
    &= R(v_1,\ldots,v_n,w_1,\ldots,w_m).
  \end{align*}
\end{enumerate}
$\Box$ \\\\



%%%%%%%%%%%%%%%%%%%%%%%%%%%%%%%%%%%%%%%%%%%%%%%%%%%%%%%%%%%%%%%%%%%%%%%%%%%%%%%%
%%%%%%%%%%%%%%%%%%%%%%%%%%%%%%%%%%%%%%%%%%%%%%%%%%%%%%%%%%%%%%%%%%%%%%%%%%%%%%%%



\subsection*{1.9. Integral Elements \\}
\addcontentsline{toc}{subsection}{1.9. Integral Elements}



\subsubsection*{Problem 1.46.* (Transitivity of integral extensions)}
\addcontentsline{toc}{subsubsection}{Problem 1.46.* (Transitivity of integral extensions)}
\emph{Let $R$ be a subring of $S$, $S$ a subring of (a domain) $T$.
If $S$ is integral over $R$, and $T$ is integral over $S$,
show that $T$ is integral over $R$.
(Hint: Let $z \in T$, so we have
$z^n + a_1 z^{n-1} + \cdots + a_n = 0$, $a_i \in S$.
Then $R[a_1, \ldots, a_n, z]$ is module-finite over $R$.)} \\

\emph{Proof (Hint).}
\begin{enumerate}
\item[(1)]
  Let $z \in T$, so we have
  $z^n + a_1 z^{n-1} + \cdots + a_n = 0$, $a_i \in S$.
  Therefore, $z$ is integral over $R[a_1, \ldots, a_n]$,
  or $R[a_1, \ldots, a_n, z]$ is module-finite over $R[a_1, \ldots, a_n]$.

\item[(2)]
  \emph{Show that $R[a_1, \ldots, a_n]$ is module-finite over $R$ if all $a_i \in S$.}
  Note that
  \begin{align*}
    & \text{$a_1$ is integral over $R$}, \\
    & \text{$a_2$ is integral over $R[a_1] \supseteq R$}, \\
    & \ldots \\
    & \text{$a_n$ is integral over $R[a_1,\ldots,a_{n-1}]$}.
  \end{align*}
  By Proposition 3,
  \begin{align*}
    & \text{$R[a_1]$ is module-finite over $R$}, \\
    & \text{$R[a_1][a_2]$ is module-finite over $R[a_1]$}, \\
    & \ldots \\
    & \text{$R[a_1,\ldots,a_{n-1}][a_n]$ is module-finite over $R[a_1,\ldots,a_{n-1}]$}.
  \end{align*}
  Also note that $R[a_1,\ldots,a_i] = R[a_1,\ldots,a_{i-1}][a_i]$ if $i > 1$.
  By the transitive relation of the module-finiteness (Problem 1.45),
  $R[a_1, \ldots, a_n]$ is module-finite over $R$.

\item[(3)]
  Again by the transitive relation of the module-finiteness (Problem 1.45),
  $R[a_1, \ldots, a_n, z]$ is module-finite over $R$.
  Hence, $R[a_1, \ldots, a_n, z]$ is a subring of $T$ containing $R[z]$
  which is module-finite over $R$.
  By Proposition 3, $z$ is integral over $R$.
\end{enumerate}
$\Box$ \\\\



%%%%%%%%%%%%%%%%%%%%%%%%%%%%%%%%%%%%%%%%%%%%%%%%%%%%%%%%%%%%%%%%%%%%%%%%%%%%%%%%



\subsubsection*{Problem 1.47.*}
\addcontentsline{toc}{subsubsection}{Problem 1.47.*}
\emph{Suppose (a domain) $S$ is ring-finite over $R$.
Show that $S$ is module-finite over $R$ if and only if $S$ is integral over $R$.} \\

\emph{Proof.}
\begin{enumerate}
\item[(1)]
  Write $S = R[v_1,\ldots,v_m]$ for some $v_i \in S$.

\item[(2)]
  Suppose that $S$ is integral over $R$.
  Then all $v_i$ are integral over $R$.
  Use the same argument in Problem 1.46, we have
  \[
    S = R[v_1, \ldots, v_n]
  \]
  is module-finite over $R$.

\item[(3)]
  Conversely, suppose that $S$ is module-finite over $R$.
  Take any $v \in S$.
  Write $v = \sum_{i} r_i v_i \in S$ since $S$ is module-finite over $R$.
  Note that $S = R[v_1,\ldots,v_m]$ is a subring of $S$ itself containing $R[v]$
  which is module-finite over $R$.
  By Proposition 3, $v$ is integral over $R$.
\end{enumerate}
$\Box$ \\\\



%%%%%%%%%%%%%%%%%%%%%%%%%%%%%%%%%%%%%%%%%%%%%%%%%%%%%%%%%%%%%%%%%%%%%%%%%%%%%%%%



\subsubsection*{Problem 1.48.*}
\addcontentsline{toc}{subsubsection}{Problem 1.48.*}
\emph{Let $L$ be a field, $k$ an algebraically closed subfield of $L$.}

\begin{enumerate}
\item[(a)]
  \emph{Show that any element of $L$ that is algebraic over $k$ is already in $k$.}

\item[(b)]
  \emph{An algebraically closed field has no module-finite field extensions except itself.} \\
\end{enumerate}



\emph{Proof of (a).}
\begin{enumerate}
\item[(1)]
  Let $\alpha \in L$ be algebraic over $k$.
  Then there is a nonzero polynomial $f(x) \in k[x]$ with $f(\alpha) = 0$.
  Note that $\deg f \geq 1$.

\item[(2)]
  Since $k$ is algebraically closed,
  every polynomial is a product of first degree polynomials, say
  \[
    f(x) = c(x - \alpha_1) \cdots (x - \alpha_m)
  \]
  where $c \in k-\{0\}$ and $\alpha_1, \ldots, \alpha_m \in k$.
  As $f(\alpha) = 0$, $\alpha = \alpha_i \in k$ for some $1 \leq i \leq m$.
  Hence, $\alpha \in L$ is algebraic over $k$ implies that $\alpha \in k$.
\end{enumerate}
$\Box$ \\



\emph{Proof of (b).}
\begin{enumerate}
\item[(1)]
  Suppose that $L$ is module-finite field extensions of
  an algebraically closed field $k$.

\item[(2)]
  By Problem 1.41, $L$ is ring-finite over $k$.
  By Problem 1.47, $L$ is integral or algebraic over $k$ (since $k$ is a field).
  By (a), $L = k$.
\end{enumerate}
$\Box$ \\\\



%%%%%%%%%%%%%%%%%%%%%%%%%%%%%%%%%%%%%%%%%%%%%%%%%%%%%%%%%%%%%%%%%%%%%%%%%%%%%%%%



\subsubsection*{Problem 1.49.*}
\addcontentsline{toc}{subsubsection}{Problem 1.49.*}
\emph{Let $K$ be a field, $L = K(x)$ the field of rational functions in one variable over $K$.} \\
\begin{enumerate}
\item[(a)]
  \emph{Show that any element of $L$ that is integral over $K[x]$ is already in $K[x]$.
  (Hint: If $z^n + a_1 z^{n-1} + \cdots + a_n = 0$, write $z = f/g$,
  $f, g$ relatively prime. Then $f^n + a_1 f^{n-1}g + \cdots + a_n g^n = 0$,
  So $g$ divides $f$.)}

\item[(b)]
  \emph{Show that there is no nonzero element $f \in K[x]$ such that for
  every $z \in L$, $f^n z$ is integral over $K[x]$ for some $n > 0$. (Hint: See Problem 1.44.)} \\
\end{enumerate}



\emph{Proof of (a).}
\begin{enumerate}
\item[(1)]
  Note that $0$ is integral over $K[x]$ and $0 \in K[x]$ trivially.

\item[(2)]
  Now we take any nonzero element $z \in L = K(x)$ which is integral over $K[x]$.
  So $z^n + a_1 z^{n-1} + \cdots + a_n = 0$ for some $a_1, \ldots, a_n \in K[x]$
  and $a_n \neq 0$ (since $z \neq 0$).

\item[(3)]
  Write $z = f/g$, $f, g$ relatively prime in $K[x]$.
  Then
  \[
    f^n + a_1 f^{n-1}g + \cdots + a_n g^n = 0 \in K[x].
  \]
  Since $a_n \neq 0$, $g \mid f^n$ or $g \mid f$ or $g = 1 \in K$.
  Therefore, $z = f \in K[x]$.
\end{enumerate}
$\Box$ \\



\emph{Proof of (b).}
\begin{enumerate}
\item[(1)]
  (Reductio ad absurdum)
  Suppose there were a nonzero element $f \in K[x]$ such that for
  every $z \in L$, $f^n z$ is integral over $K[x]$ for some $n > 0$.

\item[(2)]
  Let $z = 1/g \in K(x)$, where $g$ is an irreducible polynomial not dividing $f$.
  The existence of $g$ is guaranteed by Problem 1.5.

\item[(3)]
  By the hypothesis in (1), there is an integer $n > 0$ such that
  $f^n z$ is integral over $K[x]$.
  By (a), $f^n z = f^n/g$ is also in $K[x]$. So $g \mid f^n$ or $g \mid f$,
  which is absurd.
\end{enumerate}
$\Box$ \\\\



%%%%%%%%%%%%%%%%%%%%%%%%%%%%%%%%%%%%%%%%%%%%%%%%%%%%%%%%%%%%%%%%%%%%%%%%%%%%%%%%



\subsubsection*{Problem 1.50.*}
\addcontentsline{toc}{subsubsection}{Problem 1.50.*}
\emph{Let $K$ be a subfield of a field $L$.}
\begin{enumerate}
\item[(a)]
  \emph{Show that the set of elements of $L$ that are
  algebraic over $K$ is a subfield of $L$ containing $K$.
  (Hint: If $v^n + a_1 v^{n-1} + \cdots + a_n = 0$, and $a_n \neq 0$,
  then $v(v^{n-1} + \cdots + a_{n-1}) = -a_n$.)}

\item[(b)]
  \emph{Suppose $L$ is module-finite over $K$, and $K \subseteq R \subseteq L$, $R$ a subring of $L$.
  Show that $R$ is a field.} \\
\end{enumerate}



\emph{Proof of (a).}
\begin{enumerate}
\item[(1)]
  Let $R$ be the set of elements of $L$ that are algebraic over $K$.
  By Corollary to Proposition 3, $R$ is a subring of $L$ containing $K$.
  (Note that $K$ is a field.)
  So it suffices to show that $v^{-1} \in R$ if $v \in R - \{0\}$.

\item[(2)]
  Since $v$ is algebraic over $K$,
  we can write
  \[
    v^n + a_1 v^{n-1} + \cdots + a_n = 0
  \]
  for some $a_1, \ldots, a_n \in K$ and $a_n \neq 0$.
  So
  \[
    \left(v^{-1}\right)^n
      + \underbrace{\frac{a_{n-1}}{a_n}}_{\in K} \left(v^{-1}\right)^{n-1}
      + \cdots
      + \underbrace{\frac{a_1}{a_n}}_{\in K} \left(v^{-1}\right)
      + \underbrace{\frac{1}{a_n}}_{\in K} = 0,
  \]
  or $v^{-1}$ is integral over $K$. Hence, $v^{-1} \in R$.
\end{enumerate}
$\Box$ \\



\emph{Proof of (b).}
\begin{enumerate}
\item[(1)]
  By Problem 1.47, $L$ is algebraic over $K$.
  Hence, $R$ is algebraic over $K$.

\item[(2)]
  To show that $R$ is a field,
  it suffices to show that $v^{-1} \in R$ if $v \in R - \{0\}$.
  Since $v$ is algebraic over $K$,
  we can write
  \[
    v^n + a_1 v^{n-1} + \cdots + a_n = 0
  \]
  for some $a_1, \ldots, a_n \in K$ and $a_n \neq 0$.
  So
  \[
    v \left(- \underbrace{\frac{1}{a_n}}_{\in K \subseteq R} \underbrace{v^{n-1}}_{\in R}
      - \cdots
      - \underbrace{\frac{a_{n-1}}{a_n}}_{\in K \subseteq R} \right) = 1.
  \]
  Here $v^{-1} = \left(- \frac{1}{a_n} v^{n-1} - \cdots - \frac{a_{n-1}}{a_n} \right)$
  is the inverse of $v$ in $R$ (since $R$ is a ring containing $K$).
\end{enumerate}
$\Box$ \\\\



%%%%%%%%%%%%%%%%%%%%%%%%%%%%%%%%%%%%%%%%%%%%%%%%%%%%%%%%%%%%%%%%%%%%%%%%%%%%%%%%
%%%%%%%%%%%%%%%%%%%%%%%%%%%%%%%%%%%%%%%%%%%%%%%%%%%%%%%%%%%%%%%%%%%%%%%%%%%%%%%%



\subsection*{1.10. Field Extensions \\}
\addcontentsline{toc}{subsection}{1.10. Field Extensions}



\subsubsection*{Problem 1.51.*}
\addcontentsline{toc}{subsubsection}{Problem 1.51.*}
\emph{Let $K$ be a field, $f \in K[x]$ an irreducible monic polynomial of degree $n > 0$.}
\begin{enumerate}
\item[(a)]
  \emph{Show that $L = K[x]/(f)$ is a field,
  and if $\overline{x}$ is the residue of $x$ in $L$, then $f(\overline{x}) = 0$.}

\item[(b)]
  \emph{Suppose $L'$ is a field extension of $K$, $y \in L'$ such that $f(y) = 0$.
  Show that the homomorphism from $K[x]$ to $L'$
  that takes $x$ to $y$ induces an isomorphism of $L$ with $K(y)$.}

\item[(c)]
  \emph{With $L'$, $y$ as in (b), suppose $g \in K[x]$ and $g(y) = 0$.
  Show that $f$ divides $g$.}

\item[(d)]
  \emph{Show that $f = (x - \overline{x})f_1$, $f_1 \in L[x]$.} \\
\end{enumerate}



\emph{Proof of (a).}
\begin{enumerate}
\item[(1)]
  $(f)$ is a prime ideal in a UFD $K[x]$ since $f$ is irreducible.
  Note that $K[x]$ is also a PID, $(f)$ is maximal (Problem 1.3).
  Hence $L = K[x]/(f)$ is a field.

\item[(2)]
  $f(\overline{x}) = f(x) + (f(x)) = (f(x)) = \overline{0}$.
\end{enumerate}
$\Box$ \\



\emph{Proof of (b).}
\begin{enumerate}
\item[(1)]
  Let $\alpha: K[x] \to L'$ be a homomorphism
  defined by
  \[
    \alpha\left(\sum a_i x^i\right) = \sum a_i y^i
  \]
  where $a_i \in K$.
  $\mathrm{im}(\alpha) = K(y)$ clearly.

\item[(2)]
  Note that $\mathrm{ker}(\alpha)$ is an ideal containing $(f)$ since $\alpha(f) = 0$.
  $\mathrm{ker}(\alpha)$ is proper since $\alpha(1) = 1 \neq 0$.
  By the maximality of $(f)$, $\mathrm{ker}(\alpha) = (f)$.

\item[(3)]
  Hence, $\alpha$ induces an isomorphism of $L$ with $K(y)$:
  \[
    L = K[x]/(f) \cong K(y) \hookrightarrow L'.
  \]
\end{enumerate}
$\Box$ \\



\emph{Proof of (c).}
By (b), $g \in \mathrm{ker}(\alpha) = (f)$. So $f \mid g$.
$\Box$ \\



\emph{Proof of (d).}
\begin{enumerate}
\item[(1)]
  By (a), $\overline{x} \in L$ is a root of $f \in L[x]$ (by embedding $K[x]$ in $L[x]$).

\item[(2)]
  Since $L$ is a field, by Problem 1.7(b)
  we have
  \[
    f = (x - \overline{x})f_1
  \]
  for some $f_1 \in L[x]$.
\end{enumerate}
$\Box$ \\\\



%%%%%%%%%%%%%%%%%%%%%%%%%%%%%%%%%%%%%%%%%%%%%%%%%%%%%%%%%%%%%%%%%%%%%%%%%%%%%%%%



\subsubsection*{Problem 1.52.* (Splitting fields)}
\addcontentsline{toc}{subsubsection}{Problem 1.52.* (Splitting fields)}
\emph{Let $K$ be a field, $f \in K[x]$.
Show that there is a field $L$ containing $K$ such that
$f = \prod_{i=1}^{n} (x - x_i) \in L[x]$.
(Hint: Use Problem 1.51(d) and induction on the degree.)
$L$ is called a \textbf{splitting field} of $F$.} \\

\emph{Proof.}
\begin{enumerate}
\item[(1)]
  Let $p(x) \in K[x]$ be an irreducible factor of $f(x) \in K[x]$,
  and let $L'$ be the field $K[x]/(p(x))$ (by Problem 1.51(a)).

\item[(2)]
  Then we might regard $K$ as a subfield of $L'$
  by sending $a \in K$ to $\overline{a} = a + (p(x)) \in L'$.

\item[(3)]
  By Problem 1.51(a), $\overline{x}$ is a root of $p \in L'$;
  therefore is a root of $f$.

\item[(4)]
  Induction on $n$. By (1)(2)(3), there is a field $L' \supseteq K$
  such that $L'$ contains a root $\overline{x}$ of $f(x)$,
  say $f(x) = (x - \overline{x}) f_1(x)$ over $L'[x]$ (by Problem 1.51(d)).
  By induction, there is a field $L \supseteq L'$ such that
  $f_1$ splits over $L$.
  Hence, $f$ splits over $L$.
\end{enumerate}
$\Box$ \\\\



%%%%%%%%%%%%%%%%%%%%%%%%%%%%%%%%%%%%%%%%%%%%%%%%%%%%%%%%%%%%%%%%%%%%%%%%%%%%%%%%



\subsubsection*{Problem 1.53.*}
\addcontentsline{toc}{subsubsection}{Problem 1.53.*}
\emph{Suppose $K$ is a field of characteristic zero,
$f$ an irreducible monic polynomial in $K[x]$ of degree $n > 0$.
Let $L$ be a splitting field of $f$,
so $f = \prod_{i=1}^{n} (x - x_i)$, $x_i \in L$.
Show that the $x_i$ are distinct.
(Hint: Apply Problem 1.51(c) to $g = f_x$;
if $(x - \overline{x})^2$ divides $f$, then $g(\overline{x}) = 0$.)} \\



\emph{Proof.}
\begin{enumerate}
\item[(1)]
  Since $f \in K[x]$ is irreducible over $K$,
  $\gcd(f, f_x)$ is $1$ or $f$.
  As $\mathrm{char}(K) = 0$, $\deg(f_x) = \deg(f) - 1$.
  So $f$ does not divide $f_x$ or $\gcd(f, f_x) = 1$.
  Hence, there are polynomials $g, h \in K[x]$ such that
  \[
    1 = f g + f_x h.
  \]
  This equation is also true in $L[x]$.

\item[(2)]
  Note that
  \begin{align*}
    f &= \prod_{i=1}^{n} (x - x_i) \in L[x], \\
    f_x &= \sum_{i=1}^{n} (x - x_1) \cdots \widehat{(x - x_i)} \cdots (x - x_n) \in L[x].
  \end{align*}
  If $\overline{x}$ were a multiple root of $f$,
  then $f(\overline{x}) = f_x(\overline{x}) = 0$.
  By (1),
  \[
    1 = f(\overline{x}) g(\overline{x}) + f_x(\overline{x}) h(\overline{x}) = 0,
  \]
  which is absurd.
\end{enumerate}
$\Box$ \\\\



%%%%%%%%%%%%%%%%%%%%%%%%%%%%%%%%%%%%%%%%%%%%%%%%%%%%%%%%%%%%%%%%%%%%%%%%%%%%%%%%



\subsubsection*{Problem 1.54.*}
\addcontentsline{toc}{subsubsection}{Problem 1.54.*}
\emph{Let $R$ be a domain with quotient field $K$,
and let $L$ be a finite algebraic extension of $K$.}
\begin{enumerate}
\item[(a)]
  \emph{For any $v \in L$, show that there is a nonzero $a \in R$ such that
  $av$ is integral over $R$.}

\item[(b)]
  \emph{Show that there is a basis $v_1, \ldots, v_n$ for $L$ over $K$ (as a vector space)
  such that each $v_i$ is integral over $R$.} \\
\end{enumerate}



\emph{Proof of (a).}
\begin{enumerate}
\item[(1)]
  Take any $v \in L$, which is algebraic over $K$.
  Write
  \[
    v^n + a_1 v^{n-1} + \cdots + a_n = 0
  \]
  for some $a_1, \ldots, a_n \in K$ and $a_n \neq 0$.
  Since $K$ is the quotient field of $R$,
  there is a common denominator $a \in R$ of $a_1, \ldots, a_n$.
  Here $a \neq 0$ and $aa_i \in R$ for all $1 \leq i \leq n$.

\item[(2)]
  Hence,
  \begin{align*}
    & \: a^n v^n + a^n a_1 v^{n-1} + \cdots + a^n a_n = 0 \\
    \Longleftrightarrow & \:
    (av)^n
      + \underbrace{(aa_1)}_{\in R} (av)^{n-1}
      + \underbrace{a(aa_2)}_{\in R} (av)^{n-2}
      + \cdots
      + \underbrace{a^{n-1}(aa_n)}_{\in R} = 0.
  \end{align*}
  $av$ is integral over $R$.
\end{enumerate}
$\Box$ \\



\emph{Proof of (b).}
\begin{enumerate}
\item[(1)]
  Since $L$ be a finite algebraic extension of $K$,
  there exists a basis
  \[
    \{ w_1, \ldots, w_n \}
  \]
  for $L$ over $K$ (as a vector space).

\item[(2)]
  For each $w_i \in L$, there is a nonzero $a_i \in R$ such that
  $a_i w_i$ is integral over $R$ (by (a)).
  So it suffices to show that
  \[
    \{ a_1 w_1, \ldots, a_n w_n \}
  \]
  is also a basis for $L$ over $K$.

\item[(3)]
  Suppose
  \[
    0 = \sum_{i} \alpha_i (a_i w_i) = \sum_{i} (\alpha_i a_i) w_i
  \]
  for some $\alpha_1, \ldots, \alpha_n \in K$.
  Since $\{ w_1, \ldots, w_n \}$ is a basis,
  $\alpha_i a_i = 0$ for all $i$, or $\alpha_i = 0$ for all $i$ (since all $a_i \neq 0$).
  Hence $\{ a_1 w_1, \ldots, a_n w_n \}$ is linearly independent.

\item[(4)]
  Also,
  for any $w \in L$, we can write
  \begin{align*}
    w
    &= \underbrace{\beta_1}_{\in K} w_1 + \cdots + \underbrace{\beta_n}_{\in K} w_n \\
    &= \underbrace{\frac{\beta_1}{a_1}}_{\in K} (a_1 w_1) + \cdots
      + \underbrace{\frac{\beta_n}{a_n}}_{\in K} (a_n w_n)
  \end{align*}
  as a linear combination of $\{ a_1 w_1, \ldots, a_n w_n \}$ over $K$.
\end{enumerate}
$\Box$ \\\\



%%%%%%%%%%%%%%%%%%%%%%%%%%%%%%%%%%%%%%%%%%%%%%%%%%%%%%%%%%%%%%%%%%%%%%%%%%%%%%%%
%%%%%%%%%%%%%%%%%%%%%%%%%%%%%%%%%%%%%%%%%%%%%%%%%%%%%%%%%%%%%%%%%%%%%%%%%%%%%%%%
%%%%%%%%%%%%%%%%%%%%%%%%%%%%%%%%%%%%%%%%%%%%%%%%%%%%%%%%%%%%%%%%%%%%%%%%%%%%%%%%
%%%%%%%%%%%%%%%%%%%%%%%%%%%%%%%%%%%%%%%%%%%%%%%%%%%%%%%%%%%%%%%%%%%%%%%%%%%%%%%%



\newpage
\section*{Chapter 2: Affine Varieties \\}
\addcontentsline{toc}{section}{Chapter 2: Affine Varieties}



\subsection*{2.1. Coordinate Rings \\}
\addcontentsline{toc}{subsection}{2.1. Coordinate Rings}



\subsubsection*{Problem 2.1.*}
\addcontentsline{toc}{subsubsection}{Problem 2.1.*}
\emph{Show that the map which associates to each
$f \in k[x_1,\ldots,x_n]$ a polynomial function in $\mathscr{F}(V,k)$
is a ring homomorphism whose kernel is $I(V)$.} \\

\emph{Proof.}
\begin{enumerate}
\item[(1)]
  Define a map $\alpha: k[x_1,\ldots,x_n] \to \mathscr{F}(V,k)$.
  Every polynomial $f \in k[x_1,\ldots,x_n]$ defines a function
  from $V$ to $k$ by
  \[
    \alpha(f)(a_1,\ldots,a_n) = f(a_1,\ldots,a_n)
  \]
  for all $(a_1,\ldots,a_n) \in V$.

\item[(2)]
  $\alpha$ is a ring homomorphism by construction in (1).

\item[(3)]
  \emph{Show that $\mathrm{ker}(\alpha) = I(V)$.}
  In fact,
  given any $f \in k[x_1,\ldots,x_n]$, we have
  $\alpha(f) = 0$ (sending all $a \in V$ to $0 \in k$)
  if and only if $f(a) = 0$ for all $a \in V$
  if and only if $f \in I(V)$.

\item[(4)]
  Hence,
  \[
    k[x_1,\ldots,x_n]/I(V) = \Gamma(V)
    \cong
    \{ \text{polynomial functions in $\mathscr{F}(V,k)$} \}
  \]
  as a ring isomorphism.
\end{enumerate}
$\Box$ \\\\



%%%%%%%%%%%%%%%%%%%%%%%%%%%%%%%%%%%%%%%%%%%%%%%%%%%%%%%%%%%%%%%%%%%%%%%%%%%%%%%%



\subsubsection*{Problem 2.2.*}
\addcontentsline{toc}{subsubsection}{Problem 2.2.*}
\emph{Let $V \subseteq \mathbf{A}^n$ be a variety.
A \textbf{subvariety} of $V$ is a variety $W \subseteq \mathbf{A}^n$ that is contained in $V$.
Show that there is a natural one-to-one correspondence between
algebraic subsets (resp. subvarieties, resp. points) of $V$
and radical ideals (resp. prime ideals, resp. maximal ideals) of $\Gamma(V)$.
(See Problems 1.22, 1.38.)} \\



\emph{Proof.}
Repeat Problem 1.38
by replacing $k[x_1,\ldots,x_n]/I$ by $\Gamma(V)$.
$\Box$ \\\\



%%%%%%%%%%%%%%%%%%%%%%%%%%%%%%%%%%%%%%%%%%%%%%%%%%%%%%%%%%%%%%%%%%%%%%%%%%%%%%%%



\subsubsection*{Problem 2.3.*}
\addcontentsline{toc}{subsubsection}{Problem 2.3.*}
\emph{Let $W$ be a subvariety of a variety $V$,
and let $I_V(W)$ be the ideal of $\Gamma(V)$ corresponding to $W$.}
\begin{enumerate}
\item[(a)]
  \emph{Show that every polynomial function on $V$ restricts to a polynomial function on $W$.}

\item[(b)]
  \emph{Show that the map from $\Gamma(V)$ to $\Gamma(W)$ defined in part (a)
  is a surjective homomorphism with kernel $I_V(W)$,
  so that $\Gamma(W)$ is isomorphic to $\Gamma(V)/I_V(W)$.} \\
\end{enumerate}



\emph{Proof of (a).}
\begin{enumerate}
\item[(1)]
  Given any polynomial function $f \in \mathscr{F}(V,k)$ on $V$.
  There is a polynomial $g \in k[x_1,\ldots,x_n]$
  such that $f(P) = g(P)$ for all $P \in V \supseteq W$; thus $f(P) = g(P)$ for all $P \in W$,
  or $f|_W$ is a polynomial function on $W$.

\item[(2)]
  The map $\alpha: \{ \text{polynomial functions in $\mathscr{F}(V,k)$} \}
  \to \{ \text{polynomial functions in $\mathscr{F}(W,k)$} \}$ in (1) is defined by
  \[
    \alpha(f) = f|_{W}.
  \]
  It is a ring homomorphism.
\end{enumerate}
$\Box$ \\



\emph{Proof of (b).}
\begin{enumerate}
\item[(1)]
  Identify $\Gamma(V)$ (resp. $\Gamma(W)$) with
  the set of all polynomial functions in $\mathscr{F}(V,k)$
  (resp. in $\mathscr{F}(W,k)$) by Problem 2.1.
  The map
  \[
    \alpha: \Gamma(V) \to \Gamma(W)
  \]
  is defined by
  \[
    \alpha(f + I(V)) = f + I(W).
  \]
  It is well-defined by (a).

\item[(2)]
  \emph{Show that $\alpha$ is surjective.}
  For any $f + I(W) \in \Gamma(W)$, take $f + I(V) \in \Gamma(V)$
  and then $\alpha(f + I(V)) = f + I(W)$.
  (The choice of $f + I(V)$ depends on the representation of $f + I(W)$
  and thus might not be unique.)

\item[(3)]
  \emph{Show that $\ker(\alpha) = I_V(W)$, and thus $\Gamma(W) \cong \Gamma(V)/I_V(W)$.}
  Since $\alpha$ is a surjective homomorphism,
  \begin{align*}
    \ker(\alpha)
    &= \Gamma(V)/\Gamma(W) \\
    &= (k[x_1,\ldots,x_n]/I(V))/(k[x_1,\ldots,x_n]/I(W)) \\
    &= I(W)/I(V) \\
    &= I_V(W).
  \end{align*}
\end{enumerate}
$\Box$ \\\\



%%%%%%%%%%%%%%%%%%%%%%%%%%%%%%%%%%%%%%%%%%%%%%%%%%%%%%%%%%%%%%%%%%%%%%%%%%%%%%%%



\subsubsection*{Problem 2.4.*}
\addcontentsline{toc}{subsubsection}{Problem 2.4.*}
\emph{Let $V \subseteq \mathbf{A}^n$ be a nonempty variety.
Show that the following are equivalent:}
\begin{enumerate}
\item[(i)]
  \emph{$V$ is a point.}
\item[(ii)]
  \emph{$\Gamma(V) = k$.}
\item[(iii)]
  \emph{$\dim_k \Gamma(V) < \infty$.} \\
\end{enumerate}



\emph{Proof.}
\begin{enumerate}
\item[(1)]
  (i) $\Longrightarrow$ (ii).
  By Corollary 2 to the Hilbert's Nullstellensatz in \S 1.7,
  $V = \{ (a_1,\ldots,a_n) \}$ corresponds to the maximal ideal
  \[
    I(V) = (x_1 - a_1, \ldots, x_n - a_n)
  \]
  in $k[x_1,\ldots,x_n]$.
  Hence,
  \[
    \Gamma(V) = k[x_1,\ldots,x_n]/(x_1 - a_1, \ldots, x_n - a_n) \cong k
  \]
  (by Problem 1.24).

\item[(2)]
  (ii) $\Longrightarrow$ (iii).
  $\dim_k(\Gamma(V)) = \dim_k(k) = 1 < \infty$.

\item[(3)]
  (iii) $\Longrightarrow$ (i).
  By Corollary 4 to the Hilbert's Nullstellensatz in \S 1.7,
  $V$ is a finite set of points in $\mathbf{A}^n$.
  Since $V$ is a nonempty variety, $V$ is exactly a point.
\end{enumerate}
$\Box$ \\\\



%%%%%%%%%%%%%%%%%%%%%%%%%%%%%%%%%%%%%%%%%%%%%%%%%%%%%%%%%%%%%%%%%%%%%%%%%%%%%%%%



\subsubsection*{Problem 2.5.}
\addcontentsline{toc}{subsubsection}{Problem 2.5.}
\emph{Let $f$ be an irreducible polynomial in $k[x,y]$,
and suppose $f$ is monic in $y$:
$f = y^n + a_1(x) y^{n-1} + \cdots + a_n(x)$, with $n > 0$.
Let $V = V(f) \subseteq \mathbf{A}^2$.
Show that the natural homomorphism from $k[x]$ to $\Gamma(V) = k[x,y]/(f)$ is one-to-one,
so that $k[x]$ may be regarded as a subring of $\Gamma(V)$;
show that the residues $\overline{1}, \overline{y}, \ldots, \overline{y}^{n-1}$
generate $\Gamma(V)$ over $k[x]$ as a module.} \\



\emph{Proof.}
\begin{enumerate}
\item[(1)]
  $\Gamma(V) = k[x,y]/(f)$ is well-defined since $f$ is irreducible.
  Define a ring homomorphism $\alpha: k[x] \to \Gamma(V) = k[x,y]/(f)$
  by
  \[
    \alpha: g(x) \mapsto g(x) + (f(x,y)).
  \]

\item[(2)]
  \emph{Show that $\alpha$ is one-to-one.}
  If there were a nonzero polynomial $g \in k[x]$ such that $\alpha(g) = 0$,
  then $g = fh$ for some nonzero polynomial $h \in k[x,y]$.
  Hence
  \[
    0 = \deg_y(g) = \deg_y(f) + \deg_y(h) \geq n > 0,
  \]
  which is absurd. Therefore, $\alpha$ is one-to-one.
  Hence $k[x]$ may be regarded as a subring of $\Gamma(V)$,
  and thus the multiplication in $\Gamma(V)$
  makes $\Gamma(V)$ a $k[x]$-module.

\item[(3)]
  Given any $g(x,y) + (f(x,y)) \in k[x,y]/(f)$ where $g \in k[x,y] = (k[x])[y]$.
  By the division-with-remainder property of $(k[x])[y]$,
  \[
    g = fq + r
  \]
  for some $q, r \in (k[x])[y]$ and
  \[
    r = r_1(x) y^{n-1} + \cdots + r_n(x)
  \]
  where $r_1, \ldots, r_n \in k[x]$.
  Hence
  \begin{align*}
    g + (f)
    &= fq + r + (f) \\
    &= r + (f) \\
    &= r_1(x) y^{n-1} + \cdots + r_n(x) + (f) \\
    &= \underbrace{r_1(x)}_{\in k[x]} \overline{y}^{n-1} + \cdots
      + \underbrace{r_n(x)}_{\in k[x]} \overline{1},
  \end{align*}
  which means that
  the residues $\overline{1}, \overline{y}, \ldots, \overline{y}^{n-1}$
   generate $\Gamma(V)$ over $k[x]$ as a module.
\end{enumerate}
$\Box$ \\\\



%%%%%%%%%%%%%%%%%%%%%%%%%%%%%%%%%%%%%%%%%%%%%%%%%%%%%%%%%%%%%%%%%%%%%%%%%%%%%%%%
%%%%%%%%%%%%%%%%%%%%%%%%%%%%%%%%%%%%%%%%%%%%%%%%%%%%%%%%%%%%%%%%%%%%%%%%%%%%%%%%



\subsection*{2.2. Polynomial Maps \\}
\addcontentsline{toc}{subsection}{2.2. Polynomial Maps}



\subsubsection*{Problem 2.6.*}
\addcontentsline{toc}{subsubsection}{Problem 2.6.*}
\emph{Let $\varphi: V \to W$, $\psi: W \to Z$.
Show that $\widetilde{\psi \circ \varphi} = \widetilde{\varphi} \circ \widetilde{\psi}$.
Show that the composition of polynomial maps is a polynomial map.} \\



\emph{Proof.}
\begin{enumerate}
\item[(1)]
  \emph{Show that $\widetilde{\psi \circ \varphi} = \widetilde{\varphi} \circ \widetilde{\psi}$.}
  It is equivalent to show that
  \[
    (\widetilde{\psi \circ \varphi})(f) = (\widetilde{\varphi} \circ \widetilde{\psi})(f)
  \]
  for all $f \in \mathscr{F}(Z,k)$.
  In fact,
  \begin{align*}
    (\widetilde{\psi \circ \varphi})(f)
    &= f \circ \psi \circ \varphi, \\
    (\widetilde{\varphi} \circ \widetilde{\psi})(f)
    &= \widetilde{\varphi} ( \widetilde{\psi}(f) )
    = \widetilde{\varphi} ( f \circ \psi )
    = f \circ \psi \circ \varphi.
  \end{align*}

\item[(2)]
  \emph{Show that the composition of polynomial maps is a polynomial map.}
  Say $V \subseteq \mathbf{A}^n, W \subseteq \mathbf{A}^m, Z \subseteq \mathbf{A}^r$.
  Since $\varphi$ (resp. $\psi$) is a polynomial map,
  there are polynomials $t_1,\ldots,t_m \in k[x_1,\ldots,x_n]$
  (resp. $s_1,\ldots,s_r \in k[x_1,\ldots,x_m]$)
  such that
  \begin{align*}
    \varphi(P)
    &= (t_1(P),\ldots,t_m(P)) \\
    \psi(Q)
    &= (s_1(Q),\ldots,s_r(Q))
  \end{align*}
  for all $P \in V$ (resp. $Q \in W$).
  Hence the composition $\psi \circ \varphi$ is
  \begin{align*}
    (\psi \circ \varphi)(P)
    =& \: \psi(\varphi(P)) \\
    =& \: \psi(t_1(P),\ldots,t_m(P)) \\
    =& \: (s_1(t_1(P),\ldots,t_m(P)), \ldots, s_r(t_1(P),\ldots,t_m(P))).
  \end{align*}
  So there are polynomials $y_1,\ldots,y_r \in k[x_1,\ldots,x_n]$
  defined by
  \[
    y_i(P) = s_i(t_1(P),\ldots,t_m(P))
  \]
  for all $(a_1,\ldots,a_n) \in \mathbf{A}^n$
  such that
  \[
    (\psi \circ \varphi)(P)
    = (y_1(P),\ldots,y_r(P)).
  \]
  (Note that the composition of polynomials is a polynomials.)
  Hence $\psi \circ \varphi$ is a polynomial map.
\end{enumerate}
$\Box$ \\\\



%%%%%%%%%%%%%%%%%%%%%%%%%%%%%%%%%%%%%%%%%%%%%%%%%%%%%%%%%%%%%%%%%%%%%%%%%%%%%%%%



\subsubsection*{Problem 2.7.*}
\addcontentsline{toc}{subsubsection}{Problem 2.7.*}
\emph{If $\varphi: V \to W$ is a polynomial map,
and $X$ is an algebraic subset of $W$,
show that $\varphi^{-1}(X)$ is an algebraic subset of $V$.
If $\varphi^{-1}(X)$ is irreducible,
and $X$ is contained in the image of $\varphi$, show that $X$ is irreducible.
This gives a useful test for irreducibility.} \\



\emph{Proof.}
\begin{enumerate}
\item[(1)]
  \emph{Show that $\varphi^{-1}(X) = V(\widetilde{\varphi}(I(X)))$ is algebraic.}
  \begin{align*}
    P \in \varphi^{-1}(X)
    &\Longleftrightarrow
    \varphi(P) \in X \\
    &\Longleftrightarrow
    f(\varphi(P)) = 0 \: \forall f \in I(X) \\
    &\Longleftrightarrow
    \widetilde{\varphi}(f)(P) = 0 \: \forall f \in I(X) \\
    &\Longleftrightarrow
    g(P) = 0 \: \forall g \in \widetilde{\varphi}(I(X)) \\
    &\Longleftrightarrow
    P \in V(\widetilde{\varphi}(I(X))).
  \end{align*}
  Also note that $\widetilde{\varphi}(I(X))$ is an ideal in $k[x_1,\ldots,x_n]$
  since $\varphi$ is a polynomial map.

\item[(2)]
  \emph{If $\varphi^{-1}(X)$ is irreducible,
  and $X$ is contained in the image of $\varphi$, show that $X$ is irreducible.}
  (Reductio ad absurdum)
  Suppose that $X$ were reducible or $I(X)$ were not prime.
  So that there exist two polynomials $f_1, f_2 \not\in I(X)$ but $f_1 f_2 \in I(X)$.
  By definition of $I(X)$, there exist two points $P_1, P_2 \in X$
  such that $f_i(P_i) \neq 0$ for $i = 1, 2$.

\item[(3)]
  Since $X$ is contained in the image of $\varphi$,
  there are two corresponding points $Q_1, Q_2 \in \varphi^{-1}(X)$ such that
  $\varphi(Q_i) = P_i$.
  So $\widetilde{\varphi}(f_i)(Q_i) = f_i(P_i) \neq 0$,
  or $\widetilde{\varphi}(f_i) \not\in I(\varphi^{-1}(X))$.
  However
  \[
    \widetilde{\varphi}(f_1)\widetilde{\varphi}(f_2)
    = \widetilde{\varphi}(f_1 f_2)
    \in I(\varphi^{-1}(X))
  \]
  since $f_1 f_2 \in I(X)$,
  contrary to the primality of $I(\varphi^{-1}(X))$.
\end{enumerate}
$\Box$ \\\\



%%%%%%%%%%%%%%%%%%%%%%%%%%%%%%%%%%%%%%%%%%%%%%%%%%%%%%%%%%%%%%%%%%%%%%%%%%%%%%%%



\subsubsection*{Problem 2.8.}
\addcontentsline{toc}{subsubsection}{Problem 2.8.}
\begin{enumerate}
\item[(a)]
  \emph{Show that $\{ (t,t^2,t^3) \in \mathbf{A}^3(k) : t \in k \}$ is an affine variety.}

\item[(b)]
  \emph{Show that $V(xz-y^2, yz - x^3, z^2-x^2y) \subseteq \mathbf{A}^3(\mathbb{C})$ is a variety.
  (Hint: $y^3-x^4$, $z^3-x^5$, $z^4-y^5 \in I(V)$.
  Find a polynomial map from $\mathbf{A}^1(\mathbb{C})$ onto $V$.)} \\
\end{enumerate}



\emph{Proof of (a).}
\begin{enumerate}
\item[(1)]
  Let $Y := \{ (t,t^2,t^3) \in \mathbf{A}^3(k) : t \in k \}$ be the twisted cubic curve.
  By Problem 2.7, it suffices to show that
  there is a polynomial map from $\mathbf{A}^1(k)$ onto $Y$.
  Here we use the fact that $\mathbf{A}^1(k)$ is irreducible as $k = \overline{k}$ is infinite
  (by Problem 1.29).

\item[(2)]
  Define a mapping $\varphi$ from $\mathbf{A}^1(k)$ to $Y$ by
  $\varphi(t) = (t,t^2,t^3) \in Y$.
  $\varphi$ is a polynomial map. Also, $\varphi$ is surjective.
\end{enumerate}
$\Box$ \\

\emph{Note.}
  Also see Problems 1.11 and 1.33 (for the case $k = \mathbb{C}$). \\



\emph{Proof of (b).}
\begin{enumerate}
\item[(1)]
  We prove for any algebraically closed field $k$.

\item[(2)]
  Write
  \begin{align*}
    V &= V(xz-y^2, yz - x^3, z^2-x^2y), \\
    Y &= \{ (t^3,t^4,t^5) \in \mathbf{A}^3(k) : t \in k \}.
  \end{align*}
  We want to show that $Y = V$. $Y \subseteq V$ is trivial.
  Now given any $(x,y,z) \in V$.
  If $x = 0$, then $y = z = 0$. So $(x,y,z) = (0,0,0) \in Y$.
  If $x \neq 0$, define
  \[
    t = \frac{y}{x} \in k.
  \]
  Hence,
  \begin{align*}
    t^3 &= \frac{y^3}{x^3} = \frac{y(xz)}{x^3} = \frac{yz}{x^2} = \frac{x^3}{x^2} = x, \\
    t^4 &= tx = y, \\
    t^5 &= ty = \frac{y^2}{x} = \frac{xz}{x} = z.
  \end{align*}

\item[(3)]
  Same as (a).
  Define a mapping $\varphi$ from $\mathbf{A}^1(k)$ to $Y = V$
  by $\varphi(t) = (t^3,t^4,t^5) \in Y = V$.
\end{enumerate}
$\Box$ \\



\emph{Note.}
\begin{enumerate}
\item[(1)]
  We don't use the hint.

\item[(2)]
  In fact, it is easy to show that
  \[
    Y = V(y^3-x^4, z^3-x^5, z^4-y^5).
  \]

\item[(3)]
  $I(V)$ is a prime ideal of height $2$ in $k[x,y,z]$
  which cannot be generated by $2$ elements.
  We say $V$ is \textbf{not a local complete intersection}. \\\\
\end{enumerate}



%%%%%%%%%%%%%%%%%%%%%%%%%%%%%%%%%%%%%%%%%%%%%%%%%%%%%%%%%%%%%%%%%%%%%%%%%%%%%%%%



\subsubsection*{Problem 2.9.*}
\addcontentsline{toc}{subsubsection}{Problem 2.9.*}
\emph{Let $\varphi: V \to W$ be a polynomial map of affine varieties,
$V' \subseteq V$, $W' \subseteq W$ subvarieties.
Suppose $\varphi(V') \subseteq W'$.}
\begin{enumerate}
\item[(a)]
  \emph{Show that $\widetilde{\varphi} (I_W(W')) \subseteq I_V(V')$ (see Problems 2.3).}

\item[(b)]
  \emph{Show that the restriction of $\varphi$ gives a polynomial map from $V'$ to $W'$.} \\
\end{enumerate}



\emph{Proof of (a).}
\begin{enumerate}
\item[(1)]
  It suffices to show that $f \in I_V(V')$
  for any $f = \widetilde{\varphi}(g) \in \widetilde{\varphi} (I_W(W'))$
  for some $g \in I_W(W')$.

\item[(2)]
  To show $f \in I_V(V')$, it suffices to show that $f(P) = 0$ for all $P \in \varphi(V')$.
  In fact,
  \[
    f(P) = \widetilde{\varphi}(g)(P) = g(\varphi(P)) = 0
  \]
  since $\varphi(V') \subseteq W'$ and $g \in I_W(W')$.
\end{enumerate}
$\Box$ \\



\emph{Proof of (b).}
\begin{enumerate}
\item[(1)]
  Similar to Problem 2.3.

\item[(2)]
  Since $\varphi$ is a polynomial map,
  there are polynomials $t_1,\ldots,t_m \in k[x_1,\ldots,x_n]$
  such that
  \[
    \varphi(P) = (t_1(P),\ldots,t_m(P)) \in W
  \]
  for all $P \in V$.
  So that $\varphi|_{V'}: V' \to \varphi(V') \subseteq W'$ is also a polynomial map
  which is equipped with the same polynomials $t_1,\ldots,t_m$
  such that
  \[
    \varphi(P) = (t_1(P),\ldots,t_m(P)) \in W' \subseteq W
  \]
  for all $P \in V' \subseteq V$.
  (Note that both $V'$ and $W'$ are affine varieties.)
\end{enumerate}
$\Box$ \\\\



%%%%%%%%%%%%%%%%%%%%%%%%%%%%%%%%%%%%%%%%%%%%%%%%%%%%%%%%%%%%%%%%%%%%%%%%%%%%%%%%



\subsubsection*{Problem 2.10.*}
\addcontentsline{toc}{subsubsection}{Problem 2.10.*}
\emph{Show that the \textbf{projection map} $\mathrm{pr}: \mathbf{A}^n \to \mathbf{A}^r$,
$n \geq r$, defined by $\mathrm{pr}(a_1,\ldots,a_n) = (a_1,\ldots,a_r)$ is a polynomial map.} \\

\emph{Proof.}
\begin{enumerate}
\item[(1)]
  Define $t_i \in k[x_1,\ldots,x_n]$ by $t_i(x_1,\ldots,x_n) = x_i$
  for $i = 1, \ldots, r$.

\item[(2)]
  Clearly,
  \[
    \mathrm{pr}(P) = (t_1(P),\ldots,t_r(P))
  \]
  for $P = (a_1,\ldots,a_n) \in \mathbf{A}^n$,
  and thus $\mathrm{pr}$ is a polynomial map.
\end{enumerate}
$\Box$ \\\\



%%%%%%%%%%%%%%%%%%%%%%%%%%%%%%%%%%%%%%%%%%%%%%%%%%%%%%%%%%%%%%%%%%%%%%%%%%%%%%%%



\subsubsection*{Problem 2.11.}
\addcontentsline{toc}{subsubsection}{Problem 2.11.}
\emph{Let $f \in \Gamma(V)$, $V$ a variety $\subseteq \mathbf{A}^n$.
Define
\begin{align*}
  G(f) &= \{ (a_1,\ldots,a_n,a_{n+1}) \in \mathbf{A}^{n+1} \\
  &\qquad : (a_1,\ldots,a_n) \in V \text{ and } a_{n+1} = f (a_1,\ldots,a_n) \},
\end{align*}
the \textbf{graph} of $f$.
Show that $G(f)$ is an affine variety,
and that the map $(a_1,\ldots,a_n) \mapsto (a_1,\ldots,a_n,f(a_1,\ldots,a_n))$
defines an isomorphism of $V$ with $G(f)$. (Projection gives the inverse.)} \\



\emph{Proof.}
\begin{enumerate}
\item[(1)]
  Define $I = I(V)$ as an ideal in $k[x_1,\ldots,x_n]$.
  Note that
  \[
    G(f) = V\underbrace{(I, x_{n+1} - f)}_{:= J}.
  \]
  Here we can view $I$ as an ideal of $k[x_1,\ldots,x_n,x_{n+1}]$.

\item[(2)]
  To show that $G(f)$ is an affine variety, it suffices to show that
  \[
    I(G(f)) = I(V(J)) = \mathrm{rad}(J)
  \]
  is prime (by Proposition 1 in \S 1.5 and the Hilbert's Nullstellensatz in \S 1.7).
  Suppose $gh \in I(G(f)) = \mathrm{rad}(J)$. Write
  \begin{align*}
    g
    &= \sum_{i} g_i x_{n+1}^i
    = \sum_{i} g_i (\underbrace{(x_{n+1} - f)}_{\in J} + f)^i, \\
    h
    &= \sum_{j} h_j x_{n+1}^j
    = \sum_{j} h_j (\underbrace{(x_{n+1} - f)}_{\in J} + f)^j
  \end{align*}
  where $g_i, h_j \in k[x_1,\ldots,x_n]$.

\item[(3)]
  Hence
  \begin{align*}
    \mathrm{rad}(J)
    &= gh + \mathrm{rad}(J)
      &(gh \in \mathrm{rad}(J)) \\
    &= (g + \mathrm{rad}(J))(h + \mathrm{rad}(J)) \\
    &= \left( \sum_{i} g_i f^i + \mathrm{rad}(J) \right)
      \left( \sum_{j} h_j f^j + \mathrm{rad}(J) \right)
      &(x_{n+1} - f \in J) \\
    &= \left( \sum_{i} g_i f^i \right)
      \left( \sum_{j} h_j f^j \right) + \mathrm{rad}(J)
  \end{align*}
  or
  \[
    \underbrace{\left( \sum_{i} g_i f^i \right)^N
      \left( \sum_{j} h_j f^j \right)^N}_{\in k[x_1,\ldots,x_n]}
    \in J = (I, x_{n+1} - f)
  \]
  for some positive integer $N$.
  So that $\left( \sum_{i} g_i f^i \right)^N \left( \sum_{j} h_j f^j \right)^N \in I$.

\item[(4)]
  Since $I = I(V)$ is a prime ideal,
  we might get $\sum_{i} g_i f^i \in I \subseteq \mathrm{rad}(J)$.
  (The case $\sum_{j} h_j f^j$ is similar.)
  Hence $\mathrm{rad}(J) = I(G(f))$ is a prime ideal, or $G(f)$ is irreducible.

\item[(5)]
  As $G(f)$ is an affine variety,
  the map $\alpha: V \to G(f)$ defined by
  \[
    \alpha: (a_1,\ldots,a_n) \mapsto (a_1,\ldots,a_n,f(a_1,\ldots,a_n))
  \]
  is a polynomial map. (Here $t_1 = x_1, \ldots, t_n = x_n$ and $t_{n+1} = f$.)

\item[(6)]
  By Problem 2.10, the projection map $\mathrm{pr}$ is a polynomial map.
  Also note that $\mathrm{pr} \circ \alpha = 1_{V}$
  and $\alpha \circ \mathrm{pr} = 1_{G(f)}$.
  Therefore, $V \cong G(f)$ as an affine variety isomorphism.
\end{enumerate}
$\Box$ \\\\



%%%%%%%%%%%%%%%%%%%%%%%%%%%%%%%%%%%%%%%%%%%%%%%%%%%%%%%%%%%%%%%%%%%%%%%%%%%%%%%%



\subsubsection*{Problem 2.12.}
\addcontentsline{toc}{subsubsection}{Problem 2.12.}
\begin{enumerate}
\item[(a)]
  \emph{Let $\varphi: \mathbf{A}^{1} \to V = V(y^2-x^3) \subseteq \mathbf{A}^{2}$
  be defined by $\varphi(t) = (t^2, t^3)$.
  Show that although $\varphi$ is a one-to-one, onto polynomial map,
  $\varphi$ is not an isomorphism.
  (Hint: $\widetilde{\varphi}(\Gamma(V)) = k[t^2,t^3] \subsetneq k[t] = \Gamma(\mathbf{A}^{1})$.)}

\item[(b)]
  \emph{Let $\varphi: \mathbf{A}^{1} \to V = V(y^2 - x^2(x+1))$
  be defined by $\varphi(t) = (t^2-1, t(t^2-1))$.
  Show that $\varphi$ is one-to-one and onto, except that $\varphi(\pm 1) = (0, 0)$.} \\
\end{enumerate}



\emph{Proof of (a).}
\begin{enumerate}
\item[(1)]
  Similar to Problem 2.8(a), $\varphi$ is a polynomial map.

\item[(2)]
  Similar to Problem 2.8(a) again,
  \[
    V = V(y^2-x^3) = \{ (t^2,t^3) \in \mathbf{A}^2(k) : t \in k \}.
  \]
  Hence the map $\varphi: t \mapsto (t^2, t^3)$ is surjective.

\item[(3)]
  \emph{Show that $\varphi$ is injective.}
  Suppose $(t^2, t^3) = (s^2, s^3)$ for some $t, s \in k$.
  If $t = 0$, then $s = 0$.
  If $t \neq 0$, then $t = \frac{t^3}{t^2} = \frac{s^3}{s^2} = s$.
  In any case, $t = s$ whenever $(t^2, t^3) = (s^2, s^3)$.

\item[(4)]
  \emph{Show that $\varphi$ is not an isomorphism.}
  It suffices to show that $\widetilde{\varphi}(\Gamma(V)) \subsetneq \Gamma(\mathbf{A}^{1})$
  by Proposition 1.
  For any $f \in \Gamma(V)$,
  \[
    \widetilde{\varphi}(f)(t) = (f \circ \varphi)(t) = f(t^2,t^3) \in k[t^2,t^3].
  \]
  Hence,
  \[
    \widetilde{\varphi}(\Gamma(V)) \subseteq k[t^2,t^3] \subsetneq k[t] = \Gamma(\mathbf{A}^{1}).
  \]
  (Here note that $t \not\in k[t^2,t^3]$ but $t \in k[t]$.)
\end{enumerate}
$\Box$ \\



\emph{Proof of (b).}
\begin{enumerate}
\item[(1)]
  \emph{Write
  \[
    Y = \{ (t^2-1, t(t^2-1)) \in \mathbf{A}^2(k) : t \in k \}.
  \]
  Show that $Y = V$.}
  Similar to Problem 2.8(a).
  It suffices to show that $(x,y) \in Y$ for any $(x,y) \in V$.
  If $x = 0$, then $y = 0$ or $(x,y) = (0,0) \in Y$ whenever $t = \pm 1$.
  (In fact, $(0,0) = (t^2-1, t(t^2-1))$ iff $t^2-1 = 0$ iff $t = \pm 1$ in any field.)
  If $x \neq 0$, define
  \[
    t = \frac{y}{x} \in k.
  \]
  So $y = tx$ and thus
  \[
    0 = y^2 - x^2(x+1) = t^2 x^2 - x^2(x+1) = x^2(t^2 - (x+1)).
  \]
  Since $x \neq 0$ and $k$ is a field, we have
  \[
    t^2 - (x+1) = 0
    \Longleftrightarrow x = t^2 - 1.
  \]
  Hence, $y = tx = t(t^2 - 1)$ and therefore $(x,y) \in Y$.

\item[(2)]
  By (1), $\varphi$ is surjective and $\varphi(\pm 1) = (0,0)$.

\item[(3)]
  \emph{Show that $\varphi$ is injective except that $\varphi(\pm 1) = (0, 0)$.}
  Given $t, s \in k$.
  It suffices to show that $t = s$
  whenever $(t^2-1, t(t^2-1)) = (s^2-1, s(s^2-1)) \neq (0,0)$.
  In fact, by assumption we have $t^2-1 = s^2-1 \neq 0$ by assumption.
  Therefore,
  \[
    t = \frac{t(t^2-1)}{t^2-1} = \frac{s(s^2-1)}{s^2-1} = s.
  \]
\end{enumerate}
$\Box$ \\\\



%%%%%%%%%%%%%%%%%%%%%%%%%%%%%%%%%%%%%%%%%%%%%%%%%%%%%%%%%%%%%%%%%%%%%%%%%%%%%%%%



\subsubsection*{Problem 2.13.}
\addcontentsline{toc}{subsubsection}{Problem 2.13.}
\emph{Let $V = V(x^2-y^3, y^2-z^3) \subseteq \mathbf{A}^3$ as in Problem 1.40,
$\overline{\alpha}: \Gamma(V) \to k[t]$ induced by the homomorphism $\alpha$ of that problem.}
\begin{enumerate}
\item[(a)]
  \emph{What is the polynomial map $f$ from $\mathbf{A}^1$ to $V$ such that
  $\widetilde{f} = \overline{\alpha}$?}

\item[(b)]
  \emph{Show that $f$ is one-to-one and onto, but not an isomorphism.} \\
\end{enumerate}



\emph{Proof of (a).}
\begin{enumerate}
\item[(1)]
  \emph{Write
  \[
    Y = \{ (t^9, t^6, t^4) \in \mathbf{A}^3(k) : t \in k \}.
  \]
  Show that $Y = V$.}
  Similar to Problem 2.8(a).
  It suffices to show that $(x,y,z) \in Y$ for any $(x,y,z) \in V$.
  If $x = 0$, then $y = z = 0$ or $(x,y,z) = (0,0,0) \in Y$ by taking $t = 0$.
  If $x \neq 0$, define
  \[
    t = \frac{yz}{x} \in k.
  \]
  Hence,
  \begin{align*}
    t^9 &= \frac{y^9 z^9}{x^9} = \frac{y^{15}}{x^9} = \frac{x^{10}}{x^9} = x, \\
    t^6
    &= \frac{y^6 z^6}{x^6} = \frac{y^5 z^6}{x^6} y = \frac{y^9}{x^6} y = \frac{x^6}{x^6} y = y, \\
    t^4
    &= \frac{y^4 z^4}{x^4} = \frac{y^4 z^3}{x^4} z = \frac{y^6}{x^4} z = \frac{x^4}{x^4} z = z.
  \end{align*}

\item[(2)]
  Define a mapping $f: \mathbf{A}^1 \to \mathbf{A}^3$ by
  \[
    f: t \mapsto (t^9, t^6, t^4).
  \]
  $f$ is a polynomial map by construction.
  By (1), $f: \mathbf{A}^1 \to f(\mathbf{A}^1) = V$
  and thus $\widetilde{f}=\overline{\alpha}$ by the definition of $\alpha$.
\end{enumerate}
$\Box$ \\



\emph{Proof of (b).}
\begin{enumerate}
\item[(1)]
  Similar to Problem 2.12(a).

\item[(2)]
  $f$ is surjective by the proof of (a).

\item[(3)]
  \emph{Show that $f$ is injective.}
  Suppose $(t^9, t^6, t^4) = (s^9, s^6, s^4)$ for some $t, s \in k$.
  If $t = 0$, then $s = 0$.
  If $t \neq 0$, then $t = \frac{t^6 t^4}{t^9} = \frac{s^6 s^4}{s^9} = s$.
  In any case, $t = s$ whenever $(t^9, t^6, t^4) = (s^9, s^6, s^4)$.

\item[(4)]
  \emph{Show that $f$ is not an isomorphism.}
  It suffices to show that $\widetilde{f}(\Gamma(V)) \subsetneq \Gamma(\mathbf{A}^{1})$
  by Proposition 1.
  For any $g \in \Gamma(V)$,
  \[
    \widetilde{f}(g)(t) = (g \circ f)(t) = g(t^9,t^6,t^4) \in k[t^4,t^6,t^9].
  \]
  Hence,
  \[
    \widetilde{\varphi}(\Gamma(V)) \subseteq k[t^4,t^6,t^9] \subsetneq k[t] = \Gamma(\mathbf{A}^{1}).
  \]
  (Here note that $t \not\in k[t^4,t^6,t^9]$ but $t \in k[t]$.)
\end{enumerate}
$\Box$ \\\\



%%%%%%%%%%%%%%%%%%%%%%%%%%%%%%%%%%%%%%%%%%%%%%%%%%%%%%%%%%%%%%%%%%%%%%%%%%%%%%%%
%%%%%%%%%%%%%%%%%%%%%%%%%%%%%%%%%%%%%%%%%%%%%%%%%%%%%%%%%%%%%%%%%%%%%%%%%%%%%%%%



\subsection*{2.3. Coordinate Changes \\}
\addcontentsline{toc}{subsection}{2.3. Coordinate Changes}



\subsubsection*{Problem 2.14.* (Linear subvariety)}
\addcontentsline{toc}{subsubsection}{Problem 2.14.* (Linear subvariety)}
\emph{A set $V \subseteq \mathbf{A}^n(k)$
is called a \textbf{linear subvariety} of $\mathbf{A}^n(k)$ if
$V = V(f_1,\ldots,f_r)$ for some polynomials $f_i$ of degree $1$.}

\begin{enumerate}
\item[(a)]
  \emph{Show that if $t$ is an affine change of coordinates on $\mathbf{A}^n(k)$,
  then $V^{t}$ is also a linear subvariety of $\mathbf{A}^n(k)$.}

\item[(b)]
  \emph{If $V \neq \varnothing$,
  show that there is an affine change of coordinates $t$ of $\mathbf{A}^n$
  such that $V^t = V(x_{m+1},\ldots,x_n)$.
  (Hint: use induction on $r$.)
  So $V$ is a variety.}

\item[(c)]
  \emph{Show that the $m$ that appears in part (b) is independent of the choice of $t$.
  It is called the \textbf{dimension} of $V$.
  Then $V$ is then isomorphic (as a variety) to $\mathbf{A}^m(k)$.
  (Hint: Suppose there were an affine change of coordinates $t$ such that
  $V(x_{m+1},\ldots,x_n)^{t} = V(x_{s+1},\ldots,x_n)$, $m < s$;
  show that $t_{m+1},\ldots,t_n$ would be dependent.)} \\
\end{enumerate}



\emph{Proof of (a).}
\begin{enumerate}
\item[(1)]
  Say $t = (t_1,\ldots,t_n)$ is an affine change of coordinates,
  and $V = V(f_1,\ldots,f_r)$ for some polynomials $f_i$ of degree $1$.

\item[(2)]
  \emph{Show that $V$ is a variety and thus $I(V) = (f_1, \ldots, f_r)$
  by the Hilbert's Nullstellensatz.}
  $V(f_1, \ldots, f_r)$ is the set of all solutions of the system of linear equations:
  \begin{align*}
    f_1 &= a_{11} x_1 + \cdots + a_{1n} x_n - b_1 = 0, \\
    &\cdots \\
    f_r &= a_{r1} x_1 + \cdots + a_{rn} x_n - b_r = 0.
  \end{align*}
  Write $Ax = b$ and $V = V(Ax = b)$ where
  \[
    A =
    \underbrace{\begin{pmatrix}
    a_{11} & \cdots & a_{1n} \\
    \vdots & \ddots & \vdots \\
    a_{r1} & \cdots & a_{rn}
    \end{pmatrix}}_{\in \mathsf{M}_{r \times n}(k)},
    \qquad
    x =
    \underbrace{\begin{pmatrix}
    x_1 \\
    \vdots \\
    x_{n}
    \end{pmatrix}}_{\in \mathsf{M}_{n \times 1}(k)},
    \qquad
    b =
    \underbrace{\begin{pmatrix}
    b_1 \\
    \vdots \\
    b_{r}
    \end{pmatrix}}_{\in \mathsf{M}_{r \times 1}(k)}.
  \]

\item[(3)]
  The Gaussian elimination in linear algebra says that
  $(A|b)$ has the same solutions as its reduced row echelon form $(A'|b')$,
  that is, $V(Ax = b) = V(A'x = b')$.

\item[(4)]
  If $V(f_1, \ldots, f_r) = \varnothing$, nothing to do.
  If $V(f_1, \ldots, f_r) \neq \varnothing$, then
  \[
    V(f_1, \ldots, f_r) = V(g_1, \ldots, g_m)
  \]
  where $m = \mathrm{rank}(A)$ is the number of nonzero rows in $A'$ ($m \leq r, n$)
  and $g_i = a'_{i1} x_1 + \cdots + a'_{in} x_n - b'_i$ for $1 \leq i \leq m$.
  ($a'_{ij}$ is the entry of the matrix $A'$.)

\item[(5)]
  Now given any $f + I(V) \in k[x_1,\ldots,x_n]/I(V)$,
  we replace the leading term $x_{i_1}$ of $g_1$ by $x_{i_1} - g_1$ to get
  \[
    f + I(V)
    = \: f(x_1, \cdots,
      \underbrace{x_{i_1} - g_1}_{\text{$i_1$th position}}, \cdots, x_n) + I(V)
    := \: f_1 + I(V)
  \]
  where $f_1 \in k[x_1,\ldots,\widehat{x_{i_1}}\ldots,x_n]$.
  Continue this process to replace each leading term $x_{i_j}$ of $g_j$ by $x_{i_j} - g_j$ to get
  one by one to get
  \begin{align*}
    f + I(V) &= f_1 + I(V)
      \text{ where } f_1 \in k[x_1,\ldots,\widehat{x_{i_1}}\ldots,x_n]. \\
    & \cdots \\
    f_{m-1} + I(V) &= f_m + I(V)
      \text{ where } f_m \in k[x_1,\ldots,\widehat{x_{i_1}},\ldots,\widehat{x_{i_m}}\ldots,x_n].
  \end{align*}
  Hence, a routine shows that there is a ring isomorphism
  \[
    \alpha: k[x_1,\ldots,x_n]/I(V) \to
    \underbrace{k[x_1,\ldots,\widehat{x_{i_1}},\ldots,\widehat{x_{i_m}}\ldots,x_n]}_{\text{a domain}}
  \]
  sending $f$ to $f_m$.
  Therefore, $V$ is a variety.

\item[(6)]
  As $I(V) = (f_1, \ldots, f_r)$,
  $I(V)^{t} = (f_1^{t}, \ldots, f_r^{t})$ where each $f_i^{t}$ is linear.
  Thus $V^{t} = V(I(V)^{t}) = V(f_1^{t}, \ldots, f_r^{t})$
  is also a linear subvariety of $\mathbf{A}^n(k)$.
\end{enumerate}
$\Box$ \\



\emph{Proof of (b).}
\begin{enumerate}
\item[(1)]
  Suppose $A \in \mathsf{M}_{r \times n}(k)$ is of rank $n - m$.
  Linear algebra says that there exist invertible matrices
  $B \in \mathsf{M}_{r \times r}(k)$ and $C \in \mathsf{M}_{n \times n}(k)$
  such that $D = BAC$, where
  \[
    D = BAC =
    \underbrace{\begin{pmatrix}
    O_1 & O_2 \\
    O_3 & I_{n-m}
    \end{pmatrix}}_{\in \mathsf{M}_{r \times n}(k)}
  \]
  in which $I_{n-m} \in \mathsf{M}_{(n-m) \times (n-m)}(k)$ is the identity matrix
  and $O_1$, $O_2$, and $O_3$ are zero matrices.

\item[(2)]
  Let $t'$ be the linear map corresponding to the matrix $C$.
  So
  \begin{align*}
    V^{t'}
    &= V(Ax = b)^{t'} \\
    &= V(ACx = b) \\
    &= V(BACx = Bb)
      &(\text{$B$: invertible}) \\
    &= V(Dx = Bb) \\
    &= V(-\beta_1, \cdots, -\beta_m, x_{m+1} - \beta_{m+1}, \cdots, x_n - \beta_n)
      &(V \neq \varnothing) \\
    &= V(x_{m+1} - \beta_{m+1}, \cdots, x_n - \beta_n)
  \end{align*}
  where $Bb =
  \underbrace{\begin{pmatrix}
    \beta_1 \\
    \vdots \\
    \beta_n
  \end{pmatrix}}_{\in \mathsf{M}_{n \times 1}(k)}$.

\item[(3)]
  Let $t''$ be the translation corresponding to the matrix $Bb$.
  Let $t = t'' \circ t'$ be the desired affine change of coordinates.
  Therefore,
  \begin{align*}
    V^{t}
    &= (V^{t'})^{t''} \\
    &= V(x_{m+1} - \beta_{m+1}, \cdots, x_n - \beta_n)^{t''} \\
    &= V(x_{m+1}, \cdots, x_n).
  \end{align*}
\end{enumerate}
$\Box$ \\



\emph{Proof of (c).}
\begin{enumerate}
\item[(1)]
  Linear algebra says that
  the rank of any matrix is uniquely determined.
  Therefore, $\dim(V) = n - \mathrm{rank}(A|b) = n - \mathrm{rank}(A'|b')$ is uniquely determined.

\item[(2)]
  $V$ is then isomorphic to $\mathbf{A}^m(k)$ as a variety.
\end{enumerate}
$\Box$ \\\\



%%%%%%%%%%%%%%%%%%%%%%%%%%%%%%%%%%%%%%%%%%%%%%%%%%%%%%%%%%%%%%%%%%%%%%%%%%%%%%%%



\subsubsection*{Problem 2.15.* (Line)}
\addcontentsline{toc}{subsubsection}{Problem 2.15.* (Line)}
\emph{Let $P = (a_1, \ldots, a_n)$, $Q = (b_1, \ldots, b_n)$ be distinct points of $\mathbf{A}^{n}$.
The \textbf{line} through $P$ and $Q$ is defined to be
$\{ ( a_1 + s(b_1-a_1), \ldots, a_n + s(b_n-a_n) ) : s \in k \}$.}
\begin{enumerate}
\item[(a)]
  \emph{Show that if $L$ is the line through $P$ and $Q$, and $t$ is an affine change of coordinates,
  then $t(L)$ is the line through $t(P)$ and $t(Q)$.}

\item[(b)]
  \emph{Show that a line is a linear subvariety of dimension $1$,
  and that a linear subvariety of dimension $1$ is the line through any two of its points.}

\item[(c)]
  \emph{Show that, in $\mathbf{A}^{2}$, a line is the same thing as a hyperplane.}

\item[(d)]
  \emph{Let $P, P' \in \mathbf{A}^{2}$, $L_1$, $L_2$ two distinct lines through $P$,
  $L'_1$, $L'_2$ distinct lines through $P'$.
  Show that there is an affine change of coordinates $t$ of $\mathbf{A}^{2}$
  such that $t(P) = P'$ and $t(L_i) = L'_i$, $i = 1, 2$.} \\
\end{enumerate}



\emph{Proof of (a).}
\begin{enumerate}
\item[(1)]
  Write $t = (t_1,\ldots,t_n)$ as
  \[
    t_i = \sum_{j} c_{ij} x_j + c_{i0}.
  \]
  Take any point $P_{s} = ( a_1 + s(b_1-a_1), \ldots, a_n + s(b_n-a_n) ) \in L$
  for some $s \in k$.
  (In particular, $P_0 = P$ and $P_1 = Q$.)

\item[(2)]
  As
  \begin{align*}
    t_i(P_{s})
    =& \: \sum_{j} c_{ij}(a_j + s(b_j-a_j)) + c_{i0} \\
    =& \: \left( \sum_{j} c_{ij} a_j + c_{i0} \right) \\
      & \: + s \left[ \left( \sum_{j} c_{ij} b_j + c_{i0} \right)
        - \left( \sum_{j} c_{ij} a_j + c_{i0} \right) \right] \\
    =& t_i(P) + s(t_i(Q) - t_i(P)),
  \end{align*}
  we have
  \begin{align*}
    t(L) =& \: \{ (t_1(P) + s(t_1(Q) - t_1(P)), \ldots, t_n(P) + s(t_n(Q) - t_n(P))) \\
      & \: \qquad : s \in k\}.
  \end{align*}
  Moreover, $t(P) \in t(L)$ as $s = 0$ and $t(Q) \in t(L)$ as $s = 1$.
  Therefore, $t(L)$ is the line through $t(P)$ and $t(Q)$.
\end{enumerate}
$\Box$ \\



\emph{Proof of (b).}
\begin{enumerate}
\item[(1)]
  Note that $a_{\alpha} \neq b_{\alpha}$ for some $1 \leq \alpha \leq n$ since $P \neq Q$.
  Write
  \[
    L
    =
    V\left( x_i = a_i + \frac{x_{\alpha}-a_{\alpha}}{b_{\alpha}-a_{\alpha}}(b_i - a_i)
      : 1 \leq i \leq n \right).
  \]
  (Here we solve $s = \frac{x_{\alpha}-a_{\alpha}}{b_{\alpha}-a_{\alpha}}$
  and then replace $s$ in the equation $x_i = a_i + t(b_i - b_i)$.)
  By Problem 2.14, $L$ is a linear subvariety.

\item[(2)]
  Note that
  \begin{align*}
    n - \dim(L)
    &= \text{the rank of the corresponding augmented matrix $(A'|b')$} \\
    &= \text{the maximal number of the linearly independent rows of $(A'|b')$} \\
    &= n - 1,
  \end{align*}
  which is uniquely determined.
  Therefore, $\dim(V) = 1$.

\item[(3)]
  Conversely, $\dim(V) = 1$ implies that $\mathrm{rank}(A'|b') = n - 1$.
  So all leading terms are all $x_i$ except only one $x_j$ for some $j$.
  Hence $V$ is of the form
  \[
    V
    = (x_i + a_{ij} x_j = b_i)
  \]
  for $1 \leq i \leq n$ and $i \neq j$.
  So
  \begin{align*}
    V
    &= \{ (b_1 - a_{1j}s, \ldots,
      \underbrace{s}_{\text{$j$th position}}, \ldots, b_n - a_{nj}s) : s \in k \} \\
    &= \{ (b_1 + s((b_1-a_{1j}) - b_1), \ldots,
      \underbrace{0 + s(1 - 0)}_{\text{$j$th position}}, \ldots, \\
      & \: \qquad (b_n + s((b_n-a_{nj}) - b_n) : s \in k \}
  \end{align*}
  is a line passing
  \begin{align*}
    P &= (b_1, \ldots, 0, \ldots, b_n) \\
    Q &= (b_1-a_{1j}, \ldots, 1, \ldots, b_n-a_{nj})
  \end{align*}
  with $P \neq Q$ (since they are different in the $j$th position).
  (Here we can change $P$ and $Q$ to any two different points on $V$.)
\end{enumerate}
$\Box$ \\



\emph{Proof of (c).}
\begin{enumerate}
\item[(1)]
  A line $L \subseteq \mathbf{A}^{2}$ is $V(x+ay=b)$ or $V(x+ay=b)$ by (b).
  In any case, $L$ is a hyperplane in $\mathbf{A}^{2}$.

\item[(2)]
  Conversely, given any hyperplane $V = V(ax + by + c = 0) \subseteq \mathbf{A}^{2}$
  where $a$ and $b$ are not all zero.
  Might assume that $a \neq 0$. (The case $b \neq 0$ is similar.)
  So
  \[
    V = \left\{ \left(-\frac{c}{a}-\frac{b}{a}s, s\right) : s \in k \right\}
  \]
  is a line passing $\left(-\frac{c}{a}, 0\right)$ and $\left(-\frac{c+b}{a}, 1\right)$.
\end{enumerate}
$\Box$ \\



\emph{Proof of (d).}
\begin{enumerate}
\item[(1)]
  It suffices to show that there is a bijective affine change of coordinates $t$ of $\mathbf{A}^{2}$
  such that $t(P) = (0,0)$, $t(L_1) = V(x = 0)$ and $t(L_2) = V(y = 0)$.
  Write $P = (p_1, p_2)$ and $L_i = a_i x + b_i y + c_i$ for $i = 1, 2$.

\item[(2)]
  Let $t'' = (t''_1, t''_2)$ be a translation defined by
  \[
    \begin{pmatrix}
      t''_1 \\
      t''_2
    \end{pmatrix}
    = \begin{pmatrix}
        x - p_1 \\
        y - p_2
      \end{pmatrix}.
  \]
  So ${L_1}^{t''} = a_1 x + b_1 y$ and ${L_2}^{t''} = a_2 x + b_2 y$.
  Let
  \[
    A = \begin{pmatrix}
      a_1 & b_1 \\
      a_2 & b_2
    \end{pmatrix}
  \]
  and $t' = (t'_1, t'_2)$ be a linear map defined by
  \[
    \begin{pmatrix}
      t'_1 \\
      t'_2
    \end{pmatrix}
    = \frac{1}{\det(A)}
      \begin{pmatrix}
        b_2 & -b_1 \\
        -a_2 & a_1
      \end{pmatrix}
      \begin{pmatrix}
        x \\
        y
      \end{pmatrix}.
  \]
  ($t'$ is well-defined since $L_1$ and $L_2$ are distinct lines and thus $\det(A) \neq 0$.)
  Write $t = (t_1, t_2) = t' \circ t''$.
  So
  \[
    \begin{pmatrix}
      t_1 \\
      t_2
    \end{pmatrix}
    = \frac{1}{\det(A)}
      \begin{pmatrix}
        b_2 & -b_1 \\
        -a_2 & a_1
      \end{pmatrix}
      \begin{pmatrix}
        x - p_1 \\
        y - p_2
      \end{pmatrix}
  \]
  and
  \begin{align*}
    L_1^{t} = ({L_1}^{t''})^{t'} = x \\
    L_2^{t} = ({L_2}^{t''})^{t'} = y.
  \end{align*}

\item[(3)]
  Conversely, define an affine change of coordinates $s = (s_1, s_2)$ of $\mathbf{A}^{2}$
  by
  \[
    \begin{pmatrix}
      s_1 \\
      s_2
    \end{pmatrix}
    = \begin{pmatrix}
      a_1 x + b_1 y + c_1 \\
      a_2 x + b_2 y + c_2
    \end{pmatrix}
  \]
  so that $x^{s} = L_1$ and $y^{s} = L_2$.

\item[(4)]
  By (2)(3), the statement in (1) is established.
\end{enumerate}
$\Box$ \\\\



%%%%%%%%%%%%%%%%%%%%%%%%%%%%%%%%%%%%%%%%%%%%%%%%%%%%%%%%%%%%%%%%%%%%%%%%%%%%%%%%



\subsubsection*{Problem 2.16.}
\addcontentsline{toc}{subsubsection}{Problem 2.16.}
\emph{Let $k = \mathbb{C}$. Give $\mathbf{A}^n(\mathbb{C}) = \mathbb{C}^n$ the usual topology
(obtained by identifying $\mathbb{C}$ with $\mathbb{R}^2$,
and hence $\mathbb{C}^n$ with $\mathbb{R}^{2n}$).
Recall that a topological space X is path-connected if
for any $P, Q \in X$, there is a continuous mapping $\gamma: [0, 1] \to X$
such that $\gamma(0) = P$, $\gamma(1) = Q$.}
\begin{enumerate}
\item[(a)]
  \emph{Show that $\mathbb{C} - S$ is path-connected for any finite set $S$.}

\item[(b)]
  \emph{Let $V$ be an algebraic set in $\mathbf{A}^n(\mathbb{C})$.
  Show that $\mathbf{A}^n(\mathbb{C}) - V$ is path-connected.
  (Hint: If $P, Q \in \mathbf{A}^n(\mathbb{C}) - V$, let $L$ be the line through $P$ and $Q$.
  Then $L \cap V$ is finite, and $L$ is isomorphic to $\mathbf{A}^1(\mathbb{C})$.)} \\
\end{enumerate}



\emph{Proof of (a).}
\begin{enumerate}
\item[(1)]
  Regard $\mathbb{C}^n$ as $\mathbb{R}^{2n}$.
  Given any $P, Q \in \mathbb{C}^n - S$.
  Write $S := \{ P_1, \ldots, P_m \} \subseteq \mathbb{C}^n$.

\item[(2)]
  Let $L_{P_i}$ (resp. $L'_{P_i}$) be a line passing $P$ (resp. $Q$) and $P_i$
  for every $P_i \in S$.
  (It is well-defined since $P, Q$ are not in $S$.)
  So $\mathscr{C} := \{ L_{P_i} : P_i \in S \}$
  (resp. $\mathscr{C}' := \{ L'_{P_i} : P_i \in S \}$)
  is a collection of finitely many lines.

\item[(3)]
  Consider a unit sphere $\mathbb{S}^{2n-1}(P)$ centered at $P$.
  \[
    \left( \bigcup_{L_i \in \mathscr{C}} L_i \right) \bigcap \mathbb{S}^{2n-1}(P)
  \]
  is again a finite set (of order $\leq 2|S| = 2m$).
  Since $\mathbb{S}^{2n-1}$ is infinite,
  we can always take a line $L$ passing $P$ and some point in $\mathbb{S}^{2n-1}(P)$
  where $L \cap S = \varnothing$.

\item[(4)]
  Similarly, we take a line $L'$ passing $Q$ and some point in $\mathbb{S}^{2n-1}(Q)$
  where $L' \cap S = \varnothing$ and $L' \cap L \neq \varnothing$.
  (There are only two points in $\mathbb{S}^{2n-1}(Q)$ such that $L' \cap L = \varnothing$.
  Note that $\mathbb{S}^{2n-1}$ is infinite.)

\item[(5)]
  Take any point $A \in L' \cap L$.
  So there is a path from $P$ to $A$ (on a segment containted in $L$)
  and then to $Q$ (on a segment containted in $L'$).
  Therefore, $\mathbb{C}^n - S$ is path-connected.
\end{enumerate}
$\Box$ \\



\emph{Proof of (b).}
\begin{enumerate}
\item[(1)]
  Given any $P, Q \in \mathbf{A}^n(\mathbb{C}) - V$.
  Let $L$ be the line through $P$ and $Q$.
  To show $\mathbf{A}^n(\mathbb{C}) - V$ is path-connected,
  it suffices to show that
  \[
    L - V = L - (V \cap L)
  \]
  is path-connected.

\item[(2)]
  Similar to Problem 1.12, we have $V \cap L$ is finite.
  In fact, write $V = (f_1, \ldots, f_r)$ and
  $L = \{ ( a_1 + t(b_1-a_1), \ldots, a_n + t(b_n-a_n) ) : t \in k \}$
  where $P = (a_1, \ldots, a_n)$ and $Q = (b_1, \ldots, b_n)$.
  Then
  \begin{align*}
    V \cap L
    &= \bigcap_{i} (V(f_i) \cap L) \\
    &= \bigcap_{i} \{ f_i(a_1 + t(b_1-a_1), \ldots, a_n + t(b_n-a_n)) = 0 : t \in k \} \\
    &= \bigcap_{i} \text{ finite set } \\
    &= \text{finite set}.
  \end{align*}
  Here $f_i(a_1 + t(b_1-a_1), \ldots, a_n + t(b_n-a_n))$ is a nonzero polynomial
  since $P, Q \not\in V(f_i)$.

\item[(3)]
  Note that the path-connectedness is a topological invariant under homeomorphisms.
  Since any line is homeomorphic to $\mathbb{C}^1$,
  $L - V$ is homeomorphic to $\mathbb{C}^1 - S$ for some finite set $S$
  (by (2)).
  By (a), $L - V$ is path-connected and so is $\mathbf{A}^n(\mathbb{C}) - V$.
\end{enumerate}
$\Box$ \\\\



%%%%%%%%%%%%%%%%%%%%%%%%%%%%%%%%%%%%%%%%%%%%%%%%%%%%%%%%%%%%%%%%%%%%%%%%%%%%%%%%
%%%%%%%%%%%%%%%%%%%%%%%%%%%%%%%%%%%%%%%%%%%%%%%%%%%%%%%%%%%%%%%%%%%%%%%%%%%%%%%%



\subsection*{2.4. Rational Functions and Local Rings \\}
\addcontentsline{toc}{subsection}{2.4. Rational Functions and Local Rings}



\subsubsection*{Problem 2.17.}
\addcontentsline{toc}{subsubsection}{Problem 2.17.}
\emph{Let $V = V(y^2 - x^2(x+1)) \subseteq \mathbf{A}^{2}$,
and $\overline{x}, \overline{y}$ the residues of $x, y$ in $\Gamma(V)$;
let $z = \overline{y}/\overline{x} \in k(V)$.
Find the pole sets of $z$ and of $z^2$.} \\



\emph{Proof.}
\begin{enumerate}
\item[(1)]
  \emph{Show that the pole set of $z$ is $\{(0,0)\}$.}
  \begin{enumerate}
  \item[(a)]
    Since $V$ is irreducible by Problem 2.12(b), $V$ is a variety.
    Note that the pole set of $z$ is
    \[
      \bigcap_{z = \overline{f}/\overline{g}} V(\overline{g}).
    \]

  \item[(b)]
    By (a), $\{ (0,0) \}$ contains the the pole set of $z$.
    (As the denominator $x = 0$, we solve the equation $y^2 - x^2(x+1) = 0$ to get $y = 0$.)

  \item[(c)]
    (Reductio ad absurdum)
    If $(0,0)$ were not a pole, then there were $\overline{f}, \overline{g} \in \Gamma(V)$
    such that $z = \overline{y}/\overline{x} = \overline{f}/\overline{g}$ where $\overline{g}(0,0) \neq 0$.
    So
    \begin{align*}
      & \: z = \overline{y}/\overline{x} = \overline{f}/\overline{g} \\
      \Longrightarrow & \:
      \overline{xf} = \overline{yg} \\
      \Longrightarrow & \:
      xf - yg \in (y^2 - x^2(x+1)) \\
      \Longrightarrow & \:
      xf - yg = h(y^2 - x^2(x+1)) \text{ for some $h$} \\
      \Longrightarrow & \:
      y(g + hy) = x(f + hx(x+1)) \in (x) \\
      \Longrightarrow & \:
      g + hy \in (x) \\
     \Longrightarrow & \:
      g(0,0) = 0,
    \end{align*}
    which is absurd.
  \end{enumerate}

\item[(2)]
\emph{Show that the pole set of $z^2$ is empty.}
  Note that
  $z^2 = \overline{y^2}/\overline{x^2} = \overline{x+1}$.
  So the pole set of $z^2$ is
  \[
    \bigcap_{z^2 = \overline{f}/\overline{g}} V(\overline{g}) = \varnothing.
  \]
\end{enumerate}
$\Box$ \\\\



%%%%%%%%%%%%%%%%%%%%%%%%%%%%%%%%%%%%%%%%%%%%%%%%%%%%%%%%%%%%%%%%%%%%%%%%%%%%%%%%



\subsubsection*{Problem 2.18.}
\addcontentsline{toc}{subsubsection}{Problem 2.18.}
\emph{Let $\mathscr{O}_P(V)$ be the local ring of a variety $V$ at a point $P$.
Show that there is a natural one-to-one correspondence between the prime ideals in $\mathscr{O}_P(V)$
and the subvarieties of $V$ that pass through $P$.
(Hint: If $I$ is prime in $\mathscr{O}_P(V)$,
$I \cap \Gamma(V)$ is prime in $\Gamma(V)$,
and $I$ is generated by $I \cap \Gamma(V)$; use Problem 2.2.)} \\



\emph{Proof.}
\begin{enumerate}
\item[(1)]
  Write $P = (a_1, \ldots, a_n)$ and $\mathfrak{m} := (x_1-a_1, \ldots, x_n - a_n)$.
  It suffices to show that
  there is a natural one-to-one correspondence between the prime ideals in $\mathscr{O}_P(V)$
  and prime ideals in $\Gamma(V)$ which is contained in $I(V(P)) = \mathfrak{m}$
  by Problem 2.2.

\item[(2)]
  If $\mathfrak{p}$ is prime in $\mathscr{O}_P(V)$,
  $\mathfrak{p} \cap \Gamma(V)$ is prime in $\Gamma(V)$
  since $\Gamma(V)$ is a subring of $\mathscr{O}_P(V)$.
  Note that $\mathfrak{p} \subseteq \mathfrak{m}_P(V)$
  and thus
  \[
    \mathfrak{p} \cap \Gamma(V)
    \subseteq \mathfrak{m}_P(V) \cap \Gamma(V)
    = (x_1-a_1, \ldots, x_n - a_n).
  \]

\item[(3)]
  Conversely,
  if $\mathfrak{q}$ is prime in $\Gamma(V)$ which is contained in
  $\mathfrak{m}$
  then we need to show that
  $\mathfrak{p} := \mathfrak{q}\mathscr{O}_P(V)$ is prime in $\mathscr{O}_P(V)$.

\item[(4)]
  Note that $\mathfrak{p}$ is proper
  (since $\mathfrak{q} \subseteq \mathfrak{m}$).
  Suppose $\frac{a}{b} \frac{c}{d} \in \mathfrak{p}$ with $b(P) \neq 0$ and $d(P) \neq 0$.
  Hence
  \[
    ac = \frac{a}{b} \frac{c}{d} \cdot (bd)
    \in \mathfrak{p} \cap \Gamma(V)
    = \mathfrak{q}.
  \]
  By the primality of $\mathfrak{q}$, might assume that $a \in \mathfrak{q}$.
  (The case $c \in \mathfrak{q}$ is the same.)
  So that $\frac{a}{b} = a \cdot \frac{1}{b} \in \mathfrak{q}\mathscr{O}_P(V) = \mathfrak{p}$.
  Therefore,
  $\mathfrak{p}$ is prime.
\end{enumerate}
$\Box$ \\\\



%%%%%%%%%%%%%%%%%%%%%%%%%%%%%%%%%%%%%%%%%%%%%%%%%%%%%%%%%%%%%%%%%%%%%%%%%%%%%%%%



\subsubsection*{Problem 2.19.}
\addcontentsline{toc}{subsubsection}{Problem 2.19.}
\emph{Let $f$ be a rational function on a variety $V$.
Let
\[
  U = \{ P \in V : \text{$f$ is defined at $P$} \}.
\]
Then $f$ defines a function from $U$ to $k$.
Show that this function determines $f$ uniquely.
So a rational function may be considered as a type of function,
but only on the complement of an algebraic subset of $V$, not on $V$ itself.} \\



\emph{Proof.}
\begin{enumerate}
\item[(1)]
  Write $f = a/b \in k(V)$ with $b(P) \neq 0$.
  Define $f: U \to k$ by $f: P \mapsto f(P) = a(P)/b(P)$.

\item[(2)]
  \emph{Show that this function is well-defined.}
  Given any $P \in U$.
  Suppose that $f = a/b = c/d$ with $b(P) \neq 0$ and $d(P) \neq 0$.
  So, $ad = bc \in \Gamma(V)$ implies that $a(P) d(P) = b(P) c(P)$.
  So, $a(P)/b(P) = c(P)/d(P)$ (since $b(P) \neq 0$ and $d(P) \neq 0$).
  Therefore, $f: U \to k$ is well-defined.
\end{enumerate}
$\Box$ \\\\



%%%%%%%%%%%%%%%%%%%%%%%%%%%%%%%%%%%%%%%%%%%%%%%%%%%%%%%%%%%%%%%%%%%%%%%%%%%%%%%%



\subsubsection*{Problem 2.20. (Quadric surface)}
\addcontentsline{toc}{subsubsection}{Problem 2.20. (Quadric surface)}
\emph{Let
\[
  V = V(xw-yz) \subseteq \mathbf{A}^{4}(k),
\]
and
\[
  \Gamma(V) = k[x,y,z,w]/(xw-yz).
\]
Let $\overline{x},\overline{y},\overline{z},\overline{w}$ be the residues of $x, y, z, w$ in $\Gamma(V)$.
Then
\[
  \overline{x}/\overline{y} = \overline{z}/\overline{w} = f \in k(V)
\]
is defined at
$P = (x,y,z,w) \in V$ if $y \neq 0$ or $w \neq 0$.
Show that it is impossible to write $f = a/b$,
where $a,b \in \Gamma(V)$, and $b(P) \neq 0$ for every $P$ where $f$ is defined.
Show that the pole set of $f$ is exactly $\{ (x,y,z,w) : y = 0 \text{ and } w = 0 \}$.} \\



\emph{Proof.}
\begin{enumerate}
\item[(1)]
   Note that the pole set of $f$ is
  \[
    \bigcap_{f = \overline{a}/\overline{b}} V(\overline{b})
    \subseteq \{ (x,y,z,w) \in \mathbf{A}^{4}(k) : y = w = 0 \}.
  \]

\item[(2)]
  \emph{Show that $f$ is not defined at the origin $O = (0,0,0,0)$.}
  (Reductio ad absurdum)
  Suppose $f = \overline{x}/\overline{y} = \overline{a}/\overline{b}$
  with $\overline{b}(O) \neq 0$.
  So $\overline{bx} - \overline{ay} = 0$, or
  \[
    bx - ay = g(xw - yz)
  \]
  for some $g \in k[x,y,z,w]$.
  Take $1$-forms on the both sides to get
  \[
    b(O) x - a(O) y = 0 \in k[x,y,z,w],
  \]
  or $a(O) = b(O) = 0$, which is absurd.

\item[(3)]
  \emph{Show that it is impossible to write $f = \overline{a}/\overline{b}$,
  where $\overline{a},\overline{b} \in \Gamma(V)$,
  and $\overline{b}(P) \neq 0$ for every $P$ where $f$ is defined.}
  (Reductio ad absurdum)
  Consider the polynomial
  \[
    \beta(y,w) := b(0,y,0,w) \in k[y,w].
  \]
  $\beta$ is not a constant polynomial since $V(\beta) = \{ (0,0) \}$ by (1)(2).
  By Problem 1.14, $V(\beta) = \{ (0,0) \}$ is infinite, which is absurd.

\item[(4)]
  \emph{Show that the pole set of $f$ is exactly $\{ (x,y,z,w) : y = 0 \text{ and } w = 0 \}$.}
  (Reductio ad absurdum)
  Given any $P = (x_0,0,z_0,0) \in \{ (x,y,z,w) : y = 0 \text{ and } w = 0 \}$.
  Suppose $f = \overline{x}/\overline{y} = \overline{a}/\overline{b}$
  where $\overline{b}(P) \neq 0$.
  Similar to (2),
  \[
    bx - ay = g(xw - yz)
  \]
  for some $g \in k[x,y,z,w]$.
  So
  \[
    b(P) x_0 = b(P) x_0 - a(P) \cdot 0 = g(P) (x_0 \cdot 0 - 0 \cdot z_0) = 0.
  \]
  As $b(P) \neq 0$, $x_0 = 0$.
  Similarly, $z_0 = 0$ by noting that $bz - aw = h(xw - yz)$ for some $h \in k[x,y,z,w]$.
  Hence $P = (0,0,0,0)$, contrary to (2).
\end{enumerate}
$\Box$ \\

\emph{Note.}
  It is equal to the Segre embedding of $\mathbf{P}^1 \times \mathbf{P}^1$
  in $\mathbf{P}^3$, for suitable choice of coordinates. \\\\



%%%%%%%%%%%%%%%%%%%%%%%%%%%%%%%%%%%%%%%%%%%%%%%%%%%%%%%%%%%%%%%%%%%%%%%%%%%%%%%%



\subsubsection*{Problem 2.21.*}
\addcontentsline{toc}{subsubsection}{Problem 2.21.*}
\emph{Let $\varphi: V \to W$ be a polynomial map of affine varieties,
$\widetilde{\varphi}: \Gamma(W) \to \Gamma(V)$ the induced map on coordinate rings.
Suppose $P \in V$, $\varphi(P) = Q$.
Show that $\widetilde{\varphi}$ extends uniquely to a ring homomorphism
(also written $\widetilde{\varphi}$) from $\mathscr{O}_Q(W)$ to $\mathscr{O}_P(V)$.
(Note that $\widetilde{\varphi}$ may not extend to all of $k(W)$.)
Show that $\widetilde{\varphi}(\mathfrak{m}_Q(W)) \subseteq \mathfrak{m}_P(V)$.} \\



\emph{Proof.}
\begin{enumerate}
\item[(1)]
  Define $\widetilde{\varphi}: \mathscr{O}_Q(W) \to \mathscr{O}_P(V)$ by
  \[
    \widetilde{\varphi}: a/b
    \mapsto \widetilde{\varphi}(a)/\widetilde{\varphi}(b).
  \]
  It is well-defined since $b(Q) \neq 0$ implies that
  \[
    \widetilde{\varphi}(b)(P) = b(\varphi(P)) = b(Q) \neq 0.
  \]

\item[(2)]
  Note that $\widetilde{\varphi}$ may not extend to all of $k(W)$
  since $\widetilde{\varphi}: k(W) \to k(V)$
  might not be well-defined if $\widetilde{\varphi}(b) = 0$ for all $b \in \Gamma(W)$.

\item[(3)]
  \emph{Show that $\widetilde{\varphi}(\mathfrak{m}_Q(W)) \subseteq \mathfrak{m}_P(V)$}.
  Take any $a/b \in \mathfrak{m}_Q(W)$ with $a(Q) = 0$ and $b(Q) \neq 0$.
  As
  \[
    \widetilde{\varphi}(a)(P) = a(\varphi(P)) = a(Q) = 0,
  \]
  we have
  $\widetilde{\varphi}(a/b) \in \mathfrak{m}_P(V)$.
\end{enumerate}
$\Box$ \\\\



%%%%%%%%%%%%%%%%%%%%%%%%%%%%%%%%%%%%%%%%%%%%%%%%%%%%%%%%%%%%%%%%%%%%%%%%%%%%%%%%



\subsubsection*{Problem 2.22.*}
\addcontentsline{toc}{subsubsection}{Problem 2.22.*}
\emph{Let $t: \mathbf{A}^{n} \to \mathbf{A}^{n}$ be an affine change of coordinates, $t(P) = Q$.
Show that
$\widetilde{t}: \mathscr{O}_Q(\mathbf{A}^{n}) \to \mathscr{O}_P(\mathbf{A}^{n})$ is an isomorphism.
Show that $\widetilde{t}$ induces an isomorphism from
$\mathscr{O}_Q(V)$ to $\mathscr{O}_P(V^{t})$ if $P \in V^{t}$,
for $V$ a subvariety of $\mathbf{A}^{n}$.} \\



\emph{Proof.}
\begin{enumerate}
\item[(1)]
  Since $\widetilde{t}: \Gamma(\mathbf{A}^{n}) \to \Gamma(\mathbf{A}^{n})$ is a ring isomorphism,
  it extends uniquely to a ring isomorphism (also written $\widetilde{t}$)
  from $\mathscr{O}_Q(\mathbf{A}^{n})$ to $\mathscr{O}_P(\mathbf{A}^{n})$
  by Problem 2.21.

\item[(2)]
  Note that $\mathscr{O}_Q(V) \hookrightarrow \mathscr{O}_Q(\mathbf{A}^{n})$,
  $\mathscr{O}_P(V^{t}) \hookrightarrow \mathscr{O}_P(\mathbf{A}^{n})$,
  and $\widetilde{t}(\mathscr{O}_Q(V)) = \mathscr{O}_P(V^{t})$,
  $\widetilde{t}: \mathscr{O}_Q(V) \to \mathscr{O}_P(V^{t})$ is an isomorphism.
\end{enumerate}
$\Box$ \\\\



%%%%%%%%%%%%%%%%%%%%%%%%%%%%%%%%%%%%%%%%%%%%%%%%%%%%%%%%%%%%%%%%%%%%%%%%%%%%%%%%
%%%%%%%%%%%%%%%%%%%%%%%%%%%%%%%%%%%%%%%%%%%%%%%%%%%%%%%%%%%%%%%%%%%%%%%%%%%%%%%%



\subsection*{2.5. Discrete Valuation Rings \\}
\addcontentsline{toc}{subsection}{2.5. Discrete Valuation Rings}



\subsubsection*{Problem 2.23.*}
\addcontentsline{toc}{subsubsection}{Problem 2.23.*}
\emph{Show that the order function on $K$ is independent of the choice of uniformizing parameter.} \\

\emph{Proof.}
\begin{enumerate}
\item[(1)]
  \emph{Show that a uniformizing parameter is unique up to a unit.}
  Suppose $t$ and $t'$ are two uniformizing parameters
  for a discrete valuation ring $R$ with the quotient field $K$.
  Since $R$ is a DVR, the maximal ideal is
  \[
    \mathfrak{m} = (t) = (s).
  \]
  As $s \in (t)$, there is an element $a \in R$ such that $s = at$.
  As $s$ is irreducible (by the maximality of $\mathfrak{m}$),
  $a$ is a unit or $t$ is a unit (which is impossible).
  Hence $s = at$ for some unit $a \in R$.

\item[(2)]
  For any $z \in K$, write
  \[
    z = ut^n = v{s}^{m}
  \]
  for some units $u, v$ and integers $n \geq m$. (The case $n \leq m$ is similar.)
  Replace $s = at$ to get $u t^n = v a^m t^m$.
  So $t^{n - m} = u^{-1}v a^m$ is a unit.
  Hence, $m = n$, or the order function on $K$ is independent of the choice of uniformizing parameter.
\end{enumerate}
$\Box$ \\\\



%%%%%%%%%%%%%%%%%%%%%%%%%%%%%%%%%%%%%%%%%%%%%%%%%%%%%%%%%%%%%%%%%%%%%%%%%%%%%%%%



\subsubsection*{Problem 2.24.*}
\addcontentsline{toc}{subsubsection}{Problem 2.24.*}
\emph{Let $V = \mathbf{A}^{1}$, $\Gamma(V) = k[x]$, $K = k(V) = k(x)$.}
\begin{enumerate}
\item[(a)]
  \emph{For each $a \in k = V$, show that $\mathscr{O}_{a}(V)$ is a DVR
  with uniformizing parameter $t = x - a$.}

\item[(b)]
  \emph{Show that $\mathscr{O}_{\infty} = \{ f/g \in k(x) : \deg(g) \geq \deg(f) \}$
  is also a DVR, with uniformizing parameter $t = 1/x$.} \\\\
\end{enumerate}



\emph{Proof of (a).}
\begin{enumerate}
\item[(1)]
  By Proposition 7 in \S 2.4,
  $\mathscr{O}_{a}(V)$ is a (Noetherian) local domain.
  It suffices to show that $t = x - a$ is an irreducible element in $\mathscr{O}_{a}(V)$
  such that every nonzero $z \in \mathscr{O}_{a}(V)$ might be written uniquely
  in the form $z = ut^n$, $u$ a unit in $\mathscr{O}_{a}(V)$, $n$ a nonnegative integer
  (by Proposition 4).

\item[(2)]
  Write $z = f/g \in \mathscr{O}_{a}(V)$ where $g(a) \neq 0$.
  By Problem 1.7,
  \[
    f = \sum_{i=0}^{\deg(f)} \lambda_i (x-a)^{i}.
  \]
  Let $n$ be the smallest integer such that $\lambda_n \neq 0$.
  (Such $n$ is existed since $z$ or $f$ is nonzero.)
  Hence, $f = f_1 (x-a)^n$
  where $f_1 = \sum_{i=n}^{\deg(f)} \lambda_i (x-a)^{i-n} \neq 0$ and $f_1(a) = \lambda_n \neq 0$.
  So
  \[
    z = f/g = (f_1/g) (x-a)^n.
  \]
  Here $f_1/g$ is a unit in $\mathscr{O}_{a}(V)$.
  Besides, it is easy to show that $n$ is unique by the similar argument in Problem 2.23.
  Hence, $\mathscr{O}_{a}(V)$ is a DVR with uniformizing parameter $t = x - a$.
\end{enumerate}
$\Box$ \\



\emph{Proof of (b).}
\begin{enumerate}
\item[(1)]
  \emph{Show that $\mathscr{O}_{\infty}$ is a subring of $k(x)$.}
  Clearly, $1 = 1/1 \in \mathscr{O}_{\infty}$.
  Also, given any $f = a/b, g = c/d \in \mathscr{O}_{\infty}$.
  So
  \begin{align*}
    f - g &= a/b - c/d = \frac{ad - bc}{bd} \in \mathscr{O}_{\infty} \\
    fg &= a/b \cdot c/d = \frac{ac}{bd} \in \mathscr{O}_{\infty}
  \end{align*}
  since
  \begin{align*}
    \deg(ad - bc)
    &\leq \max(\deg(ad), \deg(bc)) \\
    &\leq \max(\deg(a)+\deg(d), \deg(b)+\deg(c)) \\
    &\leq \max(\deg(b)+\deg(d), \deg(b)+\deg(d)) \\
    &\leq \deg(b) + \deg(d) \\
    &\leq \deg(bd)
  \end{align*}
  and
  \[
    \deg(ac) = \deg(a) + \deg(c) \leq \deg(b) + \deg(d) = \deg(bd).
  \]
  (Here we define $\deg(0) = -\infty$ by convention.)
  By the subring test, $\mathscr{O}_{\infty}$ is a subring of $k(x)$.

\item[(2)]
  \emph{Show that $\mathscr{O}_{\infty}$ is a DVR.}
  Clearly $\mathscr{O}_{\infty}$ is not a field
  since $1/x \in \mathscr{O}_{\infty}$ but $x = x/1 \not\in \mathscr{O}_{\infty}$.
  Let $t = 1/x$ be an irreducible element of $\mathscr{O}_{\infty}$.
  ($\deg(x) = 1$ implies the irreducibility of $t$.)
  Now for any nonzero $f/g \in \mathscr{O}_{\infty}$, write
  \[
    f/g = ((f x^n)/g) (1/x^n) = ((f x^n)/g) t^n
  \]
  where $n := \deg(g) - \deg(f) \geq 0$.
  Note that $\deg(fx^n) = \deg(f) + n = \deg(g)$.
  So $(f x^n)/g$ is a unit since the inverse $g/(f x^n)$ is also in $\mathscr{O}_{\infty}$.
  Besides, it is easy to show that $n$ is unique by the similar argument in Problem 2.23.
  Hence, $\mathscr{O}_{\infty}$ is a DVR.
\end{enumerate}
$\Box$ \\

\emph{Note.}
\begin{enumerate}
\item[(1)]
  The quotient field of $\mathscr{O}_{\infty}$ is $K = k(V) = k(x)$.

\item[(2)]
  The set of units in $\mathscr{O}_{\infty}(V)$ is
  $\{ f/g \in k(x) : \deg(g) = \deg(f) \}$.

\item[(3)]
  The maximal ideal of $\mathscr{O}_{\infty}(V)$ is
  $\{ f/g \in k(x) : \deg(g) > \deg(f) \}$. \\\\
\end{enumerate}



%%%%%%%%%%%%%%%%%%%%%%%%%%%%%%%%%%%%%%%%%%%%%%%%%%%%%%%%%%%%%%%%%%%%%%%%%%%%%%%%



\subsubsection*{Problem 2.25. ($p$-adic integers)}
\addcontentsline{toc}{subsubsection}{Problem 2.25. ($p$-adic integers)}
\emph{Let $p \in \mathbb{Z}$ be a prime number.
Show that
\[
  \{ r \in \mathbb{Q} : \text{$r = a/b$, $a,b \in \mathbb{Z}$, $p$ doesn't divide $b$} \}
\]
is a DVR with quotient field $\mathbb{Q}$.} \\



\emph{Proof.}
\begin{enumerate}
\item[(1)]
  Let
  \[
    \mathbb{Z}_p
    = \{ r \in \mathbb{Q} : r = a/b, \: a,b \in \mathbb{Z}, \: p \nmid b \}
  \]
  be the set of all $p$-adic integers.

\item[(2)]
  \emph{Show that $\mathbb{Z}_p$ is a subring of $\mathbb{Q}$.}
  Clearly, $1 = 1/1 \in \mathbb{Z}_p$ (since $p \nmid 1$).
  Also, given any $r = a/b, s = c/d \in \mathbb{Z}_p$.
  So
  \begin{align*}
    r - s &= a/b - c/d = \frac{ad - bc}{bd} \in \mathbb{Z}_p \\
    rs &= a/b \cdot c/d = \frac{ac}{bd} \in \mathbb{Z}_p
  \end{align*}
  since $p \nmid b$, $p \nmid d$ and $p$ is a prime number.
  By the subring test, $\mathbb{Z}_p$ is a subring of $\mathbb{Q}$.

\item[(3)]
  Note that $\mathbb{Z}_p \subseteq \mathbb{Q}$ is a domain
  and $\mathbb{Z}_p$ is not a field (since $p = p/1 \in \mathbb{Z}_p$
  but $p^{-1} = 1/p \not\in \mathbb{Z}_p$).

\item[(4)]
  Let $t = p$ be an irreducible element in $\mathbb{Z}_p$.
  For the irreducibility of $t = p$, we write $p = a/b \cdot c/d = \frac{ac}{bd}$
  where $p \nmid b$, $p \nmid d$. So $pbd = ac$ or
  \[
    1 = \mathrm{ord}_p(ac) = \mathrm{ord}_p(a) + \mathrm{ord}_p(c).
  \]
  Here $\mathrm{ord}_p: \mathbb{Z} \to \mathbb{Z}_{\geq 0}$
  is defined by $\mathrm{ord}_p(a) = n$ where
  $n$ is the largest number such that $p^n$ divides $a$,
  that is, $p^n \mid a$ and $p^{n+1} \nmid a$.
  So $(\mathrm{ord}_p(a), \mathrm{ord}_p(c)) = (0,1)$ or $(1,0)$.
  Hence, $a/b$ or $c/d$ is a unit in $\mathbb{Z}_p$, or $p$ is irreducible in $\mathbb{Z}_p$.

\item[(5)]
  For any nonzero $r = a/b \in \mathbb{Z}_p$, $a \neq 0$ can be written as
  $a = p^n c$ for some nonnegative integer $n$ and $c \in \mathbb{Z}^{+}$ uniquely.
  Hence
  \[
    r = a/b = (c/b) p^n = (c/b) t^n,
  \]
  where $c/b$ is a unit and $n$ is a nonnegative integer.
  Besides, it is easy to show that $n$ is unique by the similar argument in Problem 2.23.
  By Proposition 4, $\mathbb{Z}_p$ is a DVR.

\item[(6)]
  \emph{Show that the quotient field of $\mathbb{Z}_p$ is $\mathbb{Q}$.}
  It suffices to show that $r$ is in the quotient field of $\mathbb{Z}_p$
  if $r \in \mathbb{Q} - \mathbb{Z}_p$.
  Note that $r \neq 0$.
  Write $r = a/b$ with $\gcd(a,b) = 1$.
  As $r \not\in \mathbb{Z}_p$, $p \mid b$ and $p \nmid a$.
  Therefore, $1/r = b/a \in \mathbb{Z}_p$, or $r$ is in the quotient field of $\mathbb{Z}_p$.
\end{enumerate}
$\Box$ \\

\emph{Note.}
\begin{enumerate}
\item[(1)]
  $p\mathbb{Z}_p$ is the maximal ideal of $\mathbb{Z}_p$.

\item[(2)]
  The residue field $\mathbb{Z}_p/p\mathbb{Z}_p \cong \mathbb{Z}/p\mathbb{Z}$. \\\\
\end{enumerate}



%%%%%%%%%%%%%%%%%%%%%%%%%%%%%%%%%%%%%%%%%%%%%%%%%%%%%%%%%%%%%%%%%%%%%%%%%%%%%%%%



\subsubsection*{Problem 2.26.*}
\addcontentsline{toc}{subsubsection}{Problem 2.26.*}
\emph{Let $R$ be a DVR with quotient field $K$; let $\mathfrak{m}$ be the maximal ideal of $R$.}
\begin{enumerate}
\item[(a)]
  \emph{Show that if $z \in K$, $z \not\in R$, then $z^{-1} \in \mathfrak{m}$.}

\item[(b)]
  \emph{Suppose $R \subseteq S \subseteq K$, and $S$ is also a DVR.
  Suppose the maximal ideal of $S$ contains $\mathfrak{m}$.
  Show that $S = R$.} \\
\end{enumerate}



\emph{Proof of (a).}
\begin{enumerate}
\item[(1)]
  Suppose $t$ is one uniformizing parameter for $R$.
  If $z \in K - R$,
  then we can write $z = ut^{-n}$ for some unit $u \in R$ and $n \in \mathbb{Z}^{+}$.

\item[(2)]
  Hence,
  \[
    z^{-1} = u^{-1} t^n.
  \]
  Since $u^{-1}$ is a unit in $R$ and $n > 0$, $z^{-1} \in \mathfrak{m}$.
\end{enumerate}
$\Box$ \\



\emph{Proof of (b).}
\begin{enumerate}
\item[(1)]
  (Reductio ad absurdum)
  Suppose $z \in S - R \subseteq K - R$.
  By (a), $z^{-1} \in \mathfrak{m}$.
  So $z^{-1}$ is in the maximal ideal $\mathfrak{m}'$ of $S$ containing $\mathfrak{m}$.

\item[(2)]
  As $\mathfrak{m}'$ is an ideal,
  $1 = z \cdot z^{-1} \in \mathfrak{m}'$, which is absurd.
  Therefore, $S = R$.
\end{enumerate}
$\Box$ \\\\



%%%%%%%%%%%%%%%%%%%%%%%%%%%%%%%%%%%%%%%%%%%%%%%%%%%%%%%%%%%%%%%%%%%%%%%%%%%%%%%%



\subsubsection*{Problem 2.27.}
\addcontentsline{toc}{subsubsection}{Problem 2.27.}
\emph{Show that the DVR's of Problem 2.24 are
the only DVR's with quotient field $k(x)$ that contain $k$.
Show that those of Problem 2.25 are the only DVR's with quotient field $\mathbb{Q}$.} \\



\emph{Proof (Problem 2.26).}
\begin{enumerate}
\item[(1)]
  \emph{Show that $\mathscr{O}_{a}(V)$ and $\mathscr{O}_{\infty}$
  are the only DVR's with quotient field $k(x)$ that contain $k$.}
  \begin{enumerate}
  \item[(a)]
    Let $k \subseteq R \subsetneq k(x)$ be a DVR with quotient field $k(x)$,
    $\mathfrak{m}$ be the unique maximal ideal of $R$.
    $\mathfrak{m} \neq (0)$ and the set of units in $R$ is $R - \mathfrak{m}$.

  \item[(b)]
    There are two possible cases: $x \in R$ or $x \not\in R$.

  \item[(c)]
    Suppose $x \in R$. So $R$ contains $k[x]$ as a subring.
    Consider the subset
    \[
      S := \{ x - a \in k[x] : a \in k \} \cap \mathfrak{m}
      \subseteq \mathfrak{m}.
    \]
    Suppose there were two distinct elements $x - a, x - b \in S$.
    Then $1 \in \mathfrak{m}$, contrary to the maximality of $\mathfrak{m}$.
    Suppose $S = \varnothing$, then every $x - a$ is a unit in $R$.
    Since $k = \overline{k}$, $R = k(x)$ is a field, which is absurd.
    Hence, there is only one $x - a \in \mathfrak{m}$ for one unique $a \in k$
    and other $x - b$ with $b \neq a$ is a unit in $R$.
    Thus, $R \supseteq \mathscr{O}_{a}(V)$
    and $\mathfrak{m}$ contains $(x - a)\mathscr{O}_{a}(V)$,
    which is the maximal ideal of $\mathscr{O}_{a}(V)$.
    By Problem 2.26, $R = \mathscr{O}_{a}(V)$.

  \item[(d)]
    If $x \not\in R$, then $x - a \not\in R$ whenever $a \in k \subseteq R$.
    Hence $(x-a)^{-1} \in \mathfrak{m}$ whenever $a \in k$ by Problem 2.26(a).
    Next, given any $f/g \in \mathscr{O}_{\infty}$, by $k = \overline{k}$ we have
    \[
      f/g = \underbrace{u}_{\in k}
        \underbrace{\frac{x-\alpha_1}{x-\beta_1}}_{\in R} \cdots
        \underbrace{\frac{x-\alpha_n}{x-\beta_n}}_{\in R}
        \underbrace{\frac{1}{x-\beta_{n+1}}}_{\in \mathfrak{m}} \cdots
        \underbrace{\frac{1}{x-\beta_{m}}}_{\in \mathfrak{m}},
    \]
    where $n := \deg(f)$, $m := \deg(g)$ and $n \leq m$.
    Here
    \[
      \frac{x-\alpha_i}{x-\beta_i}
      = \underbrace{1}_{\in k}
        + \underbrace{\frac{\beta_i-\alpha_i}{x-\beta_i}}_{\in \mathfrak{m} \subseteq R} \in R.
    \]
    Therefore, $R \supseteq \mathscr{O}_{\infty}$ and $\mathfrak{m}$
    contains the maximal ideal $x^{-1} \mathscr{O}_{\infty}$ of $\mathscr{O}_{\infty}$.
    By Problem 2.26, $R = \mathscr{O}_{\infty}$.
  \end{enumerate}

\item[(2)]
  \emph{Show that $\mathbb{Z}_p$ are the only DVR's with quotient field $\mathbb{Q}$.}
  \begin{enumerate}
  \item[(a)]
    Let $R \subsetneq \mathbb{Q}$ be a DVR with quotient field $\mathbb{Q}$,
    $\mathfrak{m}$ be the unique maximal ideal of $R$.
    $\mathfrak{m} \neq (0)$ and the set of units in $R$ is $R - \mathfrak{m}$.

  \item[(b)]
    Note that $R \subseteq \mathbb{Q}$ contains $\mathbb{Z}$ as a subring.
    Consider the subset
    \[
      S := \{ p \in \mathbb{Z} : \text{$p$ is a prime number} \} \cap \mathfrak{m}
      \subseteq \mathfrak{m}.
    \]

  \item[(c)]
    Suppose there were two distinct prime integers $p, q \in S$.
    By the B\'ezout's identity, there exist integers $a$ and $b$ such that $pa + qb = 1$.
    $1 \in \mathfrak{m}$, contrary to the maximality of $\mathfrak{m}$.

  \item[(d)]
    Suppose no prime integer were in $S$, then every prime integer is a unit in $R$.
    By the fundamental theorem of arithmetic, $R = \mathbb{Q}$ is a field, which is absurd.

  \item[(e)]
    By (c)(d), $p \in \mathfrak{m}$ for one unique prime $p \in \mathbb{Z}$.
    Thus, $R \supseteq \mathbb{Z}_p$ by the definition of $\mathbb{Z}_p$
    and $\mathfrak{m}$ contains $p\mathbb{Z}_p$,
    which is the maximal ideal of $\mathbb{Z}_p$.
    By Problem 2.26, $R = \mathbb{Z}_p$.
  \end{enumerate}
\end{enumerate}
$\Box$ \\\\



%%%%%%%%%%%%%%%%%%%%%%%%%%%%%%%%%%%%%%%%%%%%%%%%%%%%%%%%%%%%%%%%%%%%%%%%%%%%%%%%



\subsubsection*{Problem 2.28.*}
\addcontentsline{toc}{subsubsection}{Problem 2.28.*}
\emph{An order function on a field $K$ is a function $\varphi$
from $K$ onto $\mathbb{Z} \cup \{ \infty \}$, satisfying:}
\begin{enumerate}
\item[(i)]
  \emph{$\varphi(a) = \infty$ if and only if $a = 0$.}

\item[(ii)]
  \emph{$\varphi(ab) = \varphi(a) + \varphi(b)$.}

\item[(iii)]
  \emph{$\varphi(a + b) \geq \min(\varphi(a), \varphi(b))$.}
\end{enumerate}
\emph{Show that $R = \{z \in K : \varphi(z) \geq 0 \}$ is a DVR
with maximal ideal $\mathfrak{m} = \{ z \in K : \varphi(z) > 0 \}$, and quotient field $K$.
Conversely, show that if $R$ is a DVR with quotient field $K$,
then the function $\mathrm{ord}: K \to \mathbb{Z} \cup \{ \infty \}$ is an order function on $K$.
Giving a DVR with quotient field $K$ is equivalent to defining an order function on $K$.} \\



\emph{Proof.}
\begin{enumerate}
\item[(1)]
  \emph{Show that $\varphi(1) = 0$.}
  Note that $\varphi(1) = \varphi(1 \cdot 1) = \varphi(1) + \varphi(1)$ by (ii).
  By Property (i) of $\varphi$,
  we cancel $\varphi(1) \in \mathbb{Z}$ on the both side to get $\varphi(1) = 0$.

\item[(2)]
  \emph{Show that $\varphi(-z) = \varphi(z)$ for all $z \in K$,
  and $\varphi(z^{-1}) = -\varphi(z)$ for all $z \in K - \{0\}$.}
  Note that $\varphi(-1) = 0$
  since $0 = \varphi(1) = \varphi((-1) \cdot (-1)) = \varphi(-1) + \varphi(-1)$ (by (1)).
  Therefore,
  \[
    \varphi(-z) = \varphi((-1) \cdot z) = \varphi(-1) + \varphi(z) = \varphi(z).
  \]
  Besides,
  \[
    0 = \varphi(1) = \varphi(z z^{-1}) = \varphi(z) + \varphi(z^{-1})
  \]
  if $z \neq 0$. So $\varphi(z^{-1}) = -\varphi(z)$ if $z \neq 0$.

\item[(3)]
  \emph{Show that $R = \{z \in K : \varphi(z) \geq 0 \}$ is a ring.}
  \begin{enumerate}
  \item[(a)]
    $R \neq \varnothing$ since $1 \in R$.

  \item[(b)]
    If $a, b \in R$, then
    \[
      \varphi(a-b)
      \geq \min(\varphi(a), \varphi(-b))
      = \min(\varphi(a), \varphi(b))
      \geq 0
    \]
    (by (2)), or $a - b \in R$.

  \item[(c)]
    If $a, b \in R$, then $\varphi(ab) = \varphi(a) + \varphi(b) \geq 0$.
  \end{enumerate}
  By the subring test, $R$ is a subring of $K$.

\item[(4)]
  \emph{Show that $\{ z \in K - \{0\}: \varphi(z) = 0 \}$ is the set of all units in $R$.}
  Given any $z \in K - \{0\}$, we have
  \[
    0 = \varphi(z) + \varphi(z^{-1})
  \]
  (by (2)).
  Hence $z$ is a unit in $R$ iff
  $z, z^{-1} \in R$ iff $\varphi(z) = \varphi(z^{-1}) = 0$.

\item[(5)]
  \emph{Show that $\mathfrak{m} = \{ z \in K : \varphi(z) > 0 \}$ is a maximal ideal of $R$.}
  \begin{enumerate}
  \item[(a)]
    If $a, b \in \mathfrak{m}$, then $\varphi(a+b) \geq \min(\varphi(a), \varphi(b)) > 0$.

  \item[(b)]
    If $a \in \mathfrak{m}$ and $r \in R$,
    then $\varphi(ra) = \varphi(r) + \varphi(a) \geq \varphi(a) > 0$.

  \item[(c)]
    By (a)(b), $\mathfrak{m}$ is an ideal of $R$.

  \item[(d)]
    Note that each proper ideal in $R$ does not have any unit, that is,
    such proper ideal is contained in $\{ z \in K : \varphi(z) > 0 \} = \mathfrak{m}$ exactly (by (4)).
    Therefore, $\mathfrak{m}$ is maximal.
    (Such maximal ideal $\mathfrak{m}$ is unique and thus $R$ is a local ring.)
  \end{enumerate}

\item[(6)]
  \emph{Show that $R$ is a DVR.}
  It suffices to show that there is an irreducible element $t \in R$
  such that every nonzero $z \in R$ may be written uniquely in the form $z = ut^n$,
  $u$ a unit in $R$, $n$ a nonnegative integer.
  Since $\varphi$ is surjective, there is an element $t \in R$
  such that $\varphi(t) = 1$.
  Note that $t \neq 0$ and irreducible (by using Property (ii) of $\varphi$).
  Hence for any nonzero $z \in R$ with $n := \varphi(z) \in \mathbb{Z}$ and $n \geq 0$,
  the order of $z t^{-n} \in K$ is
  \[
    \varphi(z t^{-n}) = \varphi(z) - n \varphi(t) = n - n \cdot 1 = 0
  \]
  (by (2)).
  That is, $z t^{-n} = u$ is a unit in $R$ (by (4)).
  Hence $z = ut^n$ for some unit $u \in R$
  and nonnegative integer $n$.
  Note that $n$ is uniquely determined by $\varphi(z)$.
  By Proposition 4, $R$ is a DVR.

\item[(7)]
  \emph{Show that the quotient field of $R$ is $K$.}
  Since $R$ is a DVR, the quotient field of $R$ is contained in $K$.
  Conversely, given any $z \in K$.
  If $\varphi(z) \geq 0$, then $z \in R \subseteq K$.
  If $\varphi(z) < 0$, then $\varphi(z^{-1}) = -\varphi(z) > 0$ or $z^{-1} \in R$.
  Hence $z = 1/z^{-1} \in K$ is in the quotient field of $R$.

\item[(8)]
  \emph{Show that giving a DVR with quotient field $K$
  is equivalent to defining an order function on $K$.}
  It suffices to show that $\mathrm{ord}(\cdot)$ on $K$ defines an order function $\varphi$ on $K$.
  By Problem 2.29,
  it suffices to show that
  \[
    \mathrm{ord}(a + b) \geq \min(\mathrm{ord}(a), \mathrm{ord}(b))
  \]
  if $\mathrm{ord}(a) = \mathrm{ord}(b) := n$.
  Write $a = ut^n, b = vt^n$ where $u, v$ are units in $R$.
  Hence,
  \begin{align*}
    \mathrm{ord}(a + b)
    &= \mathrm{ord}(ut^n + vt^n) \\
    &= \mathrm{ord}((u+v)t^n) \\
    &= \mathrm{ord}(u+v) + n \\
    &\geq n
      &(u+v \in R) \\
    &= \min(\mathrm{ord}(a), \mathrm{ord}(b)).
  \end{align*}
\end{enumerate}
$\Box$ \\\\



%%%%%%%%%%%%%%%%%%%%%%%%%%%%%%%%%%%%%%%%%%%%%%%%%%%%%%%%%%%%%%%%%%%%%%%%%%%%%%%%



\subsubsection*{Problem 2.29.*}
\addcontentsline{toc}{subsubsection}{Problem 2.29.*}
\emph{Let $R$ be a DVR with quotient field $K$, $\mathrm{ord}$ the order function on $K$.}
\begin{enumerate}
\item[(a)]
  \emph{If $\mathrm{ord}(a) < \mathrm{ord}(b)$,
  show that $\mathrm{ord}(a + b) = \mathrm{ord}(a)$.}
\item[(b)]
  \emph{If $a_1, \ldots, a_n \in K$, and for some $i$,
  $\mathrm{ord}(a_i) < \mathrm{ord}(a_j)$ (all $j \neq i$),
  then $a_1 + \cdots + a_n \neq 0$.} \\
\end{enumerate}



\emph{Proof of (a).}
\begin{enumerate}
\item[(1)]
  Let $t$ be a uniformizing parameter for $R$.
  Given any $a, b \in K$.
  Write $a = ut^n, b = vt^m$ where $u, v$ are units in $R$ and $n, m$ are integers.

\item[(2)]
  Since $\mathrm{ord}(a) < \mathrm{ord}(b)$, $n < m$.
  Hence,
  \[
    a + b = (u + vt^{m-n}) t^n.
  \]
  To show that $\mathrm{ord}(a + b) = \mathrm{ord}(a) = n$,
  it suffices to show that $u + vt^{m-n}$ is a unit in $R$.

\item[(3)]
  (Reductio ad absurdum)
  Suppose that $u + vt^{m-n}$ were not a unit.
  Since $R$ is local, the maximal ideal $(t)$ contains all nonunit elements in $R$.
  Hence, $u + vt^{m-n} \in (t)$. As $m - n > 0$, $vt^{m-n} \in (t)$
  and thus a unit $u \in (t)$, contrary to the maximality of $(t)$.
\end{enumerate}
$\Box$ \\



\emph{Proof of (b).}
\begin{enumerate}
\item[(1)]
  Might assume that $\mathrm{ord}(a_1) < \mathrm{ord}(a_j)$ (all $j \neq 1$).
  In particular, $\mathrm{ord}(a_1) < \infty$.

\item[(2)]
  Similar to (a).
  Let $t$ be a uniformizing parameter for $R$.
  Write $a_i = u_i t^{m_i}$ where $u_i$ are units in $R$ and $m_i$ are integers.
  ($i = 1, \ldots, n$.)
  Since $\mathrm{ord}(a_1) < \mathrm{ord}(a_j)$ (all $j \neq 1$), $m_1 < m_j$.
  Hence,
  \[
    a_1 + \cdots + a_n
    = (u_1
      + \underbrace{u_2 t^{m_2 - m_1} + \cdots + u_n t^{m_n - m_1}}_{\in (t)}) t^{m_1}.
  \]
  So $u_1 + u_2 t^{m_2 - m_1} + \cdots + u_n t^{m_n - m_1}$ is a unit in $R$.

\item[(3)]
  By (1)(2),
  \[
    \mathrm{ord}(a_1 + \cdots + a_n) = \mathrm{ord}(a_1) < \infty,
  \]
  or $a_1 + \cdots + a_n \neq 0$
  (since $\mathrm{ord}$ is an order function on $K$).
\end{enumerate}
$\Box$ \\\\



%%%%%%%%%%%%%%%%%%%%%%%%%%%%%%%%%%%%%%%%%%%%%%%%%%%%%%%%%%%%%%%%%%%%%%%%%%%%%%%%



\subsubsection*{Problem 2.30.*}
\addcontentsline{toc}{subsubsection}{Problem 2.30.*}
\emph{Let $R$ be a DVR with maximal ideal $\mathfrak{m}$, and quotient field $K$,
and suppose a field $k$ is a subring of $R$,
and that the composition $k \to R \to R/\mathfrak{m}$ is an isomorphism of $k$ with $R/\mathfrak{m}$
(as for example in Problem 2.24).
Verify the following assertions:}
\begin{enumerate}
\item[(a)]
  \emph{For any $z \in R$, there is a unique $\lambda \in k$ such that
  $z - \lambda \in \mathfrak{m}$.}

\item[(b)]
  \emph{Let $t$ be a uniformizing parameter for $R$, $z \in R$.
  Then for any $n \geq 0$
  there are unique $\lambda_0,\lambda_1,\ldots,\lambda_n \in k$ and $z_n \in R$ such that
  \[
    z = \lambda_0 + \lambda_1 t + \lambda_2 t^2 + \cdots + \lambda_n t^n + z_n t^{n+1}.
  \]
  (Hint: For uniqueness use Problem 2.29; for existence use (a) and induction.)} \\
\end{enumerate}



\emph{Proof of (a).}
\begin{enumerate}
\item[(1)]
  Note that
  \[
    k \xrightarrow{i} R \xrightarrow{\pi} R/\mathfrak{m}
  \]
  is an isomorphism.

\item[(2)]
  For $z + \mathfrak{m} \in R/\mathfrak{m}$,
  there exists the unique $\lambda \in k$ such that
  \[
    z + \mathfrak{m}
    = \pi(i(\lambda))
    = \pi(\lambda)
    = \lambda + \mathfrak{m}.
  \]
  So $z - \lambda \in \mathfrak{m}$ for one unique $\lambda \in k$.
\end{enumerate}
$\Box$ \\



\emph{Proof of (b).}
\begin{enumerate}
\item[(1)]
  Note that
  \[
    \mathfrak{m} = \{ z \in K : \mathrm{ord}(z) > 0 \}.
  \]
  By (a),
  \[
    z = \lambda_0 + \underbrace{t z_0}_{\in \mathfrak{m}}
  \]
  for one unique $\lambda_0 \in k$ and $z_0 \in R$.
  Continue this process or by induction, we have the expression
  \[
    z = \lambda_0 + \lambda_1 t + \lambda_2 t^2 + \cdots + \lambda_n t^n + z_n t^{n+1}.
  \]

\item[(2)]
  For the uniqueness,
  suppose
  \begin{align*}
    0 = \lambda_0 + \lambda_1 t + \lambda_2 t^2 + \cdots + \lambda_n t^n + z_n t^{n+1}.
  \end{align*}
  Note that
  \begin{equation*}
    \mathrm{ord}(\lambda_i t^i) =
    \begin{cases}
      \infty
        &(\lambda_i = 0) \\
      i
        &(\lambda_i \neq 0)
    \end{cases}
  \end{equation*}
  since every nonzero element in $k$ is a unit in $k \subseteq R$.
  Also, $\mathrm{ord}(z_n t^{n+1}) = \infty$ if $z_n = 0$;
  $\mathrm{ord}(z_n t^{n+1}) \geq n+1$ if $z_n \neq 0$.

\item[(3)]
  Suppose $i_0$ is the smallest integer such that $\lambda_{i_0} \neq 0$,
  then $\mathrm{ord}(\lambda_{i_0} t^{i_0}) = i_0 < \mathrm{ord}(\lambda_{j} t^{j})$
  if $i_0 \neq j$ and $\mathrm{ord}(\lambda_{i_0} t^{i_0}) = i_0 < n+1 \leq \mathrm{ord}(z_n t^{n+1})$.
  By Problem 2.29(b), such $i_0$ does not exist.
  Hence all $\lambda_i = 0$.
  So as $R$ is a domain, $z_n$ is also equal to $0$.
  Therefore, the uniqueness is established.
\end{enumerate}
$\Box$ \\\\



%%%%%%%%%%%%%%%%%%%%%%%%%%%%%%%%%%%%%%%%%%%%%%%%%%%%%%%%%%%%%%%%%%%%%%%%%%%%%%%%



\subsubsection*{Problem 2.31. (Formal power series)}
\addcontentsline{toc}{subsubsection}{Problem 2.31. (Formal power series)}
\emph{Let $k$ be a field.
The ring of \textbf{formal power series} over $k$,
written $k[[x]]$,
is defined to be
\[
  \left\{ \sum_{i=0}^{\infty} a_i x^i : a_i \in k \right\}.
\]
(As with polynomials,
a rigorous definition is best given in terms of sequences
$(a_0,a_1,\ldots)$ of elements in $k$;
here we allow an infinite number of nonzero terms.)
Define the sum by
\[
  \sum a_i x^i + \sum b_i x^i = \sum (a_i + b_i) x^i,
\]
and the (Cauchy) product by
\[
  \left( \sum a_i x^i \right)\left( \sum b_i x^i \right) = \sum c_i x^i,
\]
where $c_i = \sum_{j+k=i} a_j b_k$.
Show that $k[[x]]$ is a ring containing $k[x]$ as a subring.
Show that $k[[x]]$ is a DVR with uniformizing parameter $x$.
Its quotient field is denoted $k((x))$.} \\



\emph{Proof.}
\begin{enumerate}
\item[(1)]
  Two formal power series $\sum a_i x^i$ and $\sum b_i x^i$ in $k[[x]]$
  are considered equal if $a_i = b_i$ for all integers $i \geq 0$.

\item[(2)]
  The zero element in $k[[x]]$ is $0 = \sum_{i=0}^{\infty} 0 x^i$, and the multiplicative identity is
  \[
    1 = 1 + 0 x + \cdots + 0 x^n + \cdots.
  \]
  Hence, $k[[x]]$ is a ring (by a tedious argument).
  Moreover, $k[[x]]$ is a domain (again by a tedious argument).

\item[(3)]
  \emph{Show that $k[[x]] \supseteq k[x]$.}
  In fact, for any $f = \sum_{i=0}^{n} a_i x^i \in k[x]$, we can write
  \[
    f = a_0 + a_1 x + \cdots + a_n x^n + 0 x^{n+1} + \cdots \in k[[x]].
  \]

\item[(4)]
  \emph{Show that $f = \sum_{i=0}^{\infty} a_i x^i$ is a unit in $k[[x]]$
  if and only if $a_0 \neq 0$.}
  Suppose $g = \sum_{i=0}^{\infty} b_i x^i \in k[[x]]$ such that $fg = 1$.
  Then
  \begin{align*}
    1 &= a_0 b_0, \\
    0 &= \sum_{j=0}^{k} a_j b_{k-j}.
  \end{align*}
  So $f$ is not a unit in $k[[x]]$ if $a_0 = 0$.
  Now if $a_0 \neq 0$ then $b_0 := a_0^{-1} \in k$.
  Then by observing that
  \begin{align*}
    0 = \sum_{j=0}^{k} a_j b_{k-j}
    &\Longleftrightarrow
    a_0 b_k = -\sum_{j=1}^{k} a_j b_{k-j} \\
    &\Longleftrightarrow
    b_k = -b_0 \sum_{j=1}^{k} a_j b_{k-j},
  \end{align*}
  we can solve $b_1, b_2, \ldots$ by induction, and $(b_0, b_1, \ldots)$ gives
  the existence of $g \in k[[x]]$.

\item[(5)]
  By (4), $k[[x]]$ is not a field since $x \in k[[x]]$ but $x^{-1} \not\in k[[x]]$.
  Let $t = x$ be an irreducible element in $k[[x]]$.
  ($\deg(x) = 1$ implies the irreducibility of $t$.)
  Hence every nonzero $f \in k[[x]]$ can be written uniquely in
  the form
  \[
    f = u x^n
  \]
  where $n$ is the smallest integer such that $a_{n} \neq 0$.
  By (4),
  \[
    u = a_n + a_{n+1} x + \cdots
  \]
  is a unit in $k[[x]]$ as $a_n \neq 0$.
  Besides, it is easy to show that $n$ is unique by the similar argument in Problem 2.23.
  Therefore, $k[[x]]$ is a DVR with uniformizing parameter $x$.
\end{enumerate}
$\Box$ \\\\



%%%%%%%%%%%%%%%%%%%%%%%%%%%%%%%%%%%%%%%%%%%%%%%%%%%%%%%%%%%%%%%%%%%%%%%%%%%%%%%%



\subsubsection*{Problem 2.32. (Power series expansion)}
\addcontentsline{toc}{subsubsection}{Problem 2.32. (Power series expansion)}
\emph{Let $R$ be a DVR satisfying the conditions of Problem 2.30.
Any $z \in R$ then determines a power series $\sum \lambda_i x^i$,
if $\lambda_0, \lambda_1, \ldots$ are determined as in Problem 2.30(b).}
\begin{enumerate}
\item[(a)]
  \emph{Show that the map $z \to \sum \lambda_i x^i$
  is a one-to-one ring homomorphism of $R$ into $k[[x]]$.
  We often write $z = \sum \lambda_i t^i$,
  and call this the \textbf{power series expansion} of $z$ in terms of $t$.}

\item[(b)]
  \emph{Show that the homomorphism extends to a homomorphism of $K$ into $k((x))$,
  and that the order function on $k((x))$ restricts to that on $K$.}

\item[(c)]
  \emph{Let $a = 0$ in Problem 2.24, $t = x$.
  Find the power series expansion of $z = (1-x)^{-1}$ and of
  $(1-x)(1+x^2)^{-1}$ in terms of $t$.} \\
\end{enumerate}



\emph{Proof of (a).}
\begin{enumerate}
\item[(1)]
  Define the map $\alpha: R \to k[[x]]$ by
  \[
    \alpha: z \mapsto \sum_{i=0}^{\infty} \lambda_i x^i
  \]
  where $\lambda_i$ are determined as in Problem 2.30(b).

\item[(2)]
  \emph{Show that $\alpha$ is well-defined and one-to-one.}
  Write
  \[
    \alpha(z)
    = \sum_{i=0}^{\infty} \lambda_i x^i
    = \sum_{i=0}^{\infty} \lambda'_i x^i.
  \]
  If there were $\lambda_n \neq \lambda'_n$ for some $n$,
  then Problem 2.30(b) implies that two expressions of $z$
  \begin{align*}
    z
    &= \lambda_0 + \lambda_1 t + \cdots + \lambda_n t^n + z_n t^{n+1} \\
    &= \lambda'_0 + \lambda'_1 t + \cdots + \lambda'_n t^n + z'_n t^{n+1}
  \end{align*}
  are the same. That is, $\lambda_n = \lambda'_n$, which is absurd.
  Hence, $\alpha$ is well-defined.
  Also,
  $0 = 0 + 0 t + 0 t^2 + \cdots + 0 t^n + 0 t^{n+1}$
  implies that $\alpha$ is one-to-one.

\item[(3)]
  \emph{Show that $\alpha$ is addition preserving.}
  Given $a, b \in R$. By Problem 2.30(b),
  \[
    a+b
    = \lambda_0 + \lambda_1 t + \cdots + \lambda_n t^n + c_n t^{n+1}
  \]
  and
  \begin{align*}
    a
    &= \mu_0 + \mu_1 t + \cdots + \mu_n t^n + a_n t^{n+1} \\
    b
    &= \nu_0 + \nu_1 t + \cdots + \nu_n t^n + b_n t^{n+1}
  \end{align*}
  for any integer $n \geq 0$.
  So
  \[
    a+b
    = \underbrace{(\mu_0+\nu_0)}_{\in k}
      + \underbrace{(\mu_1+\nu_1)}_{\in k} t + \cdots
      + \underbrace{(\mu_n+\nu_n)}_{\in k} t^n
      + \underbrace{(a_n+b_n)}_{\in R} t^{n+1}.
  \]
  Since the expression of $a+b$ is unique (by Problem 2.30(b)),
  \[
    \lambda_i = \mu_i + \nu_i
  \]
  for all $i = 0, 1, \ldots, n$.
  Since $n$ is arbitrary, $\lambda_i = \mu_i + \nu_i$ is true for all nonnegative integers.
  Hence, $\alpha(a+b) = \alpha(a) + \alpha(b)$.

\item[(4)]
  \emph{Show that $\alpha$ is multiplication preserving.}
  Given $a, b \in R$. By Problem 2.30(b),
  \[
    ab
    = \lambda_0 + \lambda_1 t + \cdots + \lambda_n t^n + c_n t^{n+1}
  \]
  and
  \begin{align*}
    a
    &= \mu_0 + \mu_1 t + \cdots + \mu_n t^n + a_n t^{n+1} \\
    b
    &= \nu_0 + \nu_1 t + \cdots + \nu_n t^n + b_n t^{n+1}
  \end{align*}
  for any integer $n \geq 0$.
  So
  \begin{align*}
    ab
    =& \: \underbrace{(\mu_0 \nu_0)}_{\in k}
      + \underbrace{(\mu_1 \nu_0 + \mu_0 \nu_1)}_{\in k} t + \cdots \\
      &+ \underbrace{(\mu_n \nu_0 + \mu_{n-1}\nu_1 + \cdots + \mu_1\nu_{n-1} + \mu_0\nu_n)}_{\in k} t^n \\
      &+ \underbrace{(\text{other terms})}_{\in R} t^{n+1}.
  \end{align*}
  Since the expression of $a+b$ is unique (by Problem 2.30(b)),
  \[
    \lambda_i = \sum_{j+k=i} \mu_j \nu_k
  \]
  for all $i = 0, 1, \ldots, n$.
  Since $n$ is arbitrary, $\lambda_i = \sum_{j+k=i} \mu_j + \nu_k$ is true for all nonnegative integers.
  Hence, $\alpha(ab) = \alpha(a)\alpha(b)$.

\item[(5)]
  \emph{Show that $\alpha$ is multiplicative identity preserving.}
  Note that
  \[
    1
    = \underbrace{1}_{\in k}
      + \underbrace{0}_{\in k} t + \cdots
      + \underbrace{0}_{\in k} t^n
      + \underbrace{0}_{\in k} t^{n+1}
  \]
  for every nonnegative integer $n$.
  Hence $\alpha: 1 \mapsto 1 \in k[[x]]$.

\item[(6)]
  By (3)(4)(5), $\alpha$ is a ring homomorphism.
\end{enumerate}
$\Box$ \\



\emph{Proof of (b).}
\begin{enumerate}
\item[(1)]
  Define the mapping $\beta$ from $K$ to $k((x))$
  by
  \[
    \beta: a/b \mapsto \alpha(a)/\alpha(b)
  \]
  where $a, b \in R$ and $b \neq 0$.

\item[(2)]
  $\beta$ is well-defined since:
  \begin{enumerate}
  \item[(a)]
    $\alpha(b) \neq 0$ if $b \neq 0$ by the injectivity of $\alpha$.

  \item[(b)]
    The value of $\beta(a/b)$ is independent of the choice of $a/b \in K$
    since $\alpha$ is a ring homomorphism.
  \end{enumerate}

\item[(3)]
  Also, $\beta$ is a ring homomorphism since $\alpha$ is a ring homomorphism.

\item[(4)]
  To show that the order function on $k((x))$ restricts to that on $K$,
  it suffices to show that
  \[
    \mathrm{ord}_{R}(z) = \mathrm{ord}_{k[[x]]}(\alpha(z)).
  \]
  In fact,
  \begin{align*}
    m := \mathrm{ord}_{R}(z)
    &\Longleftrightarrow
    z = \lambda_m t^m + \cdots + \lambda_n t^n + z_n t^{n+1} \text{ with } \lambda_m \neq 0 \\
    &\Longleftrightarrow
    \alpha(z) = \lambda_m x^m + \cdots \text{ with } \lambda_m \neq 0 \\
    &\Longleftrightarrow
    \mathrm{ord}_{k[[x]]}(\alpha(z)) = m.
  \end{align*}
\end{enumerate}
$\Box$ \\



\emph{Proof of (c).}
\begin{enumerate}
\item[(1)]
  In calculus we have
  \[
    (1-x)^{-1} = 1 + x + x^2 + \cdots = \sum_{i=0}^{\infty} x^i
  \]
  for $|x| < 1$.
  In the ring of formal power series $k[[x]]$, $1-x$ is a unit
  (by (4) in the proof of Problem 2.31) and satisfies
  \[
    (1-x)\left( \sum_{i=0}^{\infty} x^i \right) = 1 \in k[[x]].
  \]
  Hence, the power expansion of $(1-x)^{-1}$ is
  \[
    (1-x)^{-1} = \sum_{i=0}^{\infty} x^i \in k((x)).
  \]

\item[(2)]
  Note that $1+x^2$ is a unit in $k[[x]]$ and satisfies
  \[
    (1+x^2)\left( \sum_{i=0}^{\infty} (-1)^i x^{2i} \right)
    = 1 \in k[[x]].
  \]
  Hence, the power expansion of $(1-x)(1+x^2)^{-1}$ is
  \begin{align*}
    (1 - x)\left( \sum_{i=0}^{\infty} (-1)^i x^{2i} \right)
    &= \left( \sum_{i=0}^{\infty} (-1)^i x^{2i} \right)
      - x\left( \sum_{i=0}^{\infty} (-1)^i x^{2i} \right) \\
    &= \sum_{i=0}^{\infty} (-1)^i x^{2i} + \sum_{i=0}^{\infty} (-1)^{i+1} x^{2i+1} \\
    &= \sum_{i=0}^{\infty} (-1)^i x^{i} \in k[[x]].
  \end{align*}
\end{enumerate}
$\Box$ \\\\



%%%%%%%%%%%%%%%%%%%%%%%%%%%%%%%%%%%%%%%%%%%%%%%%%%%%%%%%%%%%%%%%%%%%%%%%%%%%%%%%
%%%%%%%%%%%%%%%%%%%%%%%%%%%%%%%%%%%%%%%%%%%%%%%%%%%%%%%%%%%%%%%%%%%%%%%%%%%%%%%%



\subsection*{2.6. Forms \\}
\addcontentsline{toc}{subsection}{2.6. Forms}



\subsubsection*{Problem 2.33.}
\addcontentsline{toc}{subsubsection}{Problem 2.33.}
\emph{Factor $y^3 - 2xy^2 + 2x^2y + x^3$
into linear factors in $\mathbb{C}[x,y]$.} \\



\emph{Proof.}
\begin{enumerate}
\item[(1)]
  Let $f(x,y) = y^3 - 2xy^2 + 2x^2y + x^3$.
  Then $f_*(x) = 1 - 2x + 2x^3 + x^3$.

\item[(2)]
  Solve $f_*(x) = 0$ over $\mathbb{C}$ by WolframAlpha (a computational knowledge engine) to get
  \begin{align*}
    \alpha_1
    &= - \frac{2}{3} - \frac{10}{3} \sqrt[3]{\frac{2}{79 - 3\sqrt{249}}}
      - \frac{1}{3} \sqrt[3]{\frac{79 - 3\sqrt{249}}{2}} \\
    \alpha_2
    &= - \frac{2}{3} + \frac{5}{3}(1 - \sqrt{3}i) \sqrt[3]{\frac{2}{79 - 3\sqrt{249}}}
      + \frac{1}{6}(1 + \sqrt{3}i) \sqrt[3]{\frac{79 - 3\sqrt{249}}{2}} \\
    \alpha_3
    &= - \frac{2}{3} + \frac{5}{3}(1 + \sqrt{3}i) \sqrt[3]{\frac{2}{79 - 3\sqrt{249}}}
      + \frac{1}{6}(1 - \sqrt{3}i) \sqrt[3]{\frac{79 - 3\sqrt{249}}{2}}.
  \end{align*}
  So $f_*(x) = (x - \alpha_1)(x - \alpha_2)(x - \alpha_3)$.

\item[(3)]
  Hence,
  \begin{align*}
    f(x,y)
    =& \: (f_*)^{*} \\
    =& \: ((x - \alpha_1)(x - \alpha_2)(x - \alpha_3))^{*} \\
    =& \: (x - \alpha_1 y)(x - \alpha_2 y)(x - \alpha_3 y).
  \end{align*}
\end{enumerate}
$\Box$ \\

\emph{Note.}
  If $f(x,y) = y^3 - 2xy^2 + 2x^2y + 4x^3$, then
  \[
    f(x,y) = (x - \alpha_1 y)(x - \alpha_2 y)(x - \alpha_3 y)
  \]
  where
  \begin{align*}
    \alpha_1
    &= - \frac{1}{6} - \frac{7}{6} \sqrt[3]{\frac{1}{37 - 3\sqrt{114}}}
      - \frac{1}{6} \sqrt[3]{37 - 3\sqrt{114}} \\
    \alpha_2
    &= - \frac{1}{6} + \frac{7}{12}(1 - \sqrt{3}i) \sqrt[3]{\frac{1}{37 - 3\sqrt{114}}}
      + \frac{1}{12}(1 + \sqrt{3}i) \sqrt[3]{37 - 3\sqrt{114}} \\
    \alpha_3
    &= - \frac{1}{6} + \frac{7}{12}(1 + \sqrt{3}i) \sqrt[3]{\frac{1}{37 - 3\sqrt{114}}}
      + \frac{1}{12}(1 - \sqrt{3}i) \sqrt[3]{37 - 3\sqrt{114}}.
  \end{align*} \\



%%%%%%%%%%%%%%%%%%%%%%%%%%%%%%%%%%%%%%%%%%%%%%%%%%%%%%%%%%%%%%%%%%%%%%%%%%%%%%%%



\subsubsection*{Problem 2.34.}
\addcontentsline{toc}{subsubsection}{Problem 2.34.}
\emph{Suppose $f, g \in k[x_1,\ldots,x_n]$ are forms of degree $r$, $r + 1$ respectively,
with no common factors ($k$ a field). Show that $f+g$ is irreducible.} \\



\emph{Proof.}
\begin{enumerate}
\item[(1)]
  Suppose $f + g = rs \in k[x_1,\ldots,x_n]$.
  Proposition 5 implies that
  \[
    (f + g)^{*} = (rs)^{*}
    \Longrightarrow
    x_{n+1} f + g = r^{*} s^{*}.
  \]
  Note that $\deg_{x_{n+1}}(x_{n+1} f + g) = 1$.
  So $\deg_{x_{n+1}}(r^{*}) = 0$ or $\deg_{x_{n+1}}(s^{*}) = 0$.
  Might assume $\deg_{x_{n+1}}(r^{*}) = 0$. (The case $\deg_{x_{n+1}}(s^{*}) = 0$ is similar.)

\item[(2)]
  Since $\deg_{x_{n+1}}(r^{*}) = 0$, $r^{*} \mid f$ and $r^{*} \mid g$.
  Note that $\deg_{x_{n+1}}(r^{*}) = 0$ implies that $r^{*} = r$ is a form in $k[x_1,\ldots,x_n]$.
  Hence $r$ is a common factor of $f$ and $g$, or $r$ is a constant in $k[x_1,\ldots,x_n]$.
  So $f+g$ is irreducible.
\end{enumerate}
$\Box$ \\\\



%%%%%%%%%%%%%%%%%%%%%%%%%%%%%%%%%%%%%%%%%%%%%%%%%%%%%%%%%%%%%%%%%%%%%%%%%%%%%%%%



\subsubsection*{Problem 2.35.*}
\addcontentsline{toc}{subsubsection}{Problem 2.35.*}
\begin{enumerate}
\item[(a)]
  \emph{Show that there are $d + 1$ monomials of degree $d$ in $R[x,y]$,
  and $1 + 2 + \cdots + (d+1) = \frac{(d+1)(d+2)}{2}$ monomials of degree $d$ in $R[x,y,z]$.}

\item[(b)]
  \emph{Let $V(d,n) = \{\text{forms of degree $d$ in $k[x_1,\ldots,x_n]$} \}$, $k$ a field.
  Show that $V(d,n)$ is a vector space over $k$,
  and that the monomials of degree $d$ form a basis.
  So $\dim V(d,1) = 1$; $\dim V(d,2) = d+1$; $\dim V(d,3) = \frac{(d+1)(d+2)}{2}$.}

\item[(c)]
  \emph{Let $\ell_1, \ell_2, \ldots$ and $m_1, m_2, \ldots$
  be sequences of nonzero linear forms in $k[x,y]$,
  and assume no $\ell_i = \lambda m_j$, $\lambda \in k$.
  Let $A_{ij} = \ell_1 \ell_2 \cdots \ell_i m_1 m_2 \cdots m_j$, $i, j \geq 0$ ($A_{00} = 1$).
  Show that $\{ A_{ij} : i+j=d \}$ forms a basis for $V(d,2)$.} \\
\end{enumerate}



\emph{Proof of (a).}
\begin{enumerate}
\item[(1)]
  All monomials of degree $d$ in $R[x,y]$ are
  \[
    x^d, x^{d-1} y, \cdots, x y^{d-1}, y^d,
  \]
  or of the form $x^i y^j$ with $i, j \geq 0$ and $i + j = d$.
  So there are $d + 1$ monomials of degree $d$ in $R[x,y]$.

\item[(2)]
  Similar to (1), all monomials of degree $d$ in $R[x,y]$ are
  of the form $x^i y^j z^k$ with $i, j, k \geq 0$ and $i + j + k = d$.
  By the \href{https://en.wikipedia.org/wiki/Stars_and_bars_%28combinatorics%29}
  {stars and bars (combinatorics)} method,
  there are
  \[
    {d+3-1 \choose 3-1} = \frac{(d+2)(d+1)}{2}
  \]
  monomials of degree $d$ in $R[x,y,z]$.
\end{enumerate}
$\Box$ \\



\emph{Proof of (b).}
\begin{enumerate}
\item[(1)]
  To show $V(d,n)$ is a vector space,
  it suffices to show that $V(d,n)$ is a subspace of $k[x_1,\ldots,x_n]$
  since $k[x_1,\ldots,x_n]$ is a vector space over $k$.

\item[(2)]
  Note that $0 \in V(d,n)$ is nonempty.
  For any $f, g \in V(d,n)$ and $a, b \in k$, we have $af + bg \in V(d,n)$.
  Hence $V(d,n)$ is subspace.

\item[(3)]
  Let
  \[
    \mathscr{B}
    = \{ x_1^{i_1} \cdots x_n^{i_n}
      : i_1, \ldots, i_n \geq 0, i_1 + \cdots + i_n = d \}.
  \]
  $\mathscr{B}$ is an independent set, and $\mathscr{B}$ generates $V(d,n)$.
  So $\mathscr{B}$ is a basis for $V(d,n)$.

\item[(4)]
  Similar to (a),
  \[
    \dim_k V(d,n) = |\mathscr{B}| = {d+n-1 \choose n-1}
  \]
  by the \href{https://en.wikipedia.org/wiki/Stars_and_bars_%28combinatorics%29}
  {stars and bars (combinatorics)} method.
  In particular,
  $\dim V(d,1) = 1$; $\dim V(d,2) = d+1$; $\dim V(d,3) = \frac{(d+1)(d+2)}{2}$.
\end{enumerate}
$\Box$ \\



\emph{Proof of (c).}
\begin{enumerate}
\item[(1)]
  \emph{Show that $\mathscr{B}' := \{ A_{ij} : i+j=d \}$ is an independent set.}
  (Reductio ad absurdum)
  Suppose that there were a nontrivial linear combination of $A_{ij}$ such that
  \[
    \sum_{i+j=d} c_{ij} A_{ij} = 0.
  \]

\item[(2)]
  Let $p$ be the smallest index $i$ such that $c_{ij} \neq 0$.
  Write $q := d - p$.
  So
  \begin{align*}
    & \: c_{pq} A_{pq}
    = -\sum_{\substack{i+j=d \\ i \neq p, j \neq q}} c_{ij} A_{ij}
    = -\sum_{\substack{i+j=d \\ i > p, j < q}} c_{ij} A_{ij} \\
    \Longleftrightarrow& \:
    A_{pq} = -\sum_{\substack{i+j=d \\ i > p, j < q}} \frac{c_{ij}}{c_{pq}} A_{ij} \\
    \Longleftrightarrow& \:
    \ell_1 \cdots \ell_p m_1 \cdots m_q
      = -\sum_{\substack{i+j=d \\ i > p, j < q}} \frac{c_{ij}}{c_{pq}}
        \ell_1 \cdots \ell_p \ell_{p+1} \cdots \ell_i m_1 \cdots m_j \\
    \Longleftrightarrow& \:
    m_1 \cdots m_q
      = - \ell_{p+1} \sum_{\substack{i+j=d \\ i > p, j < q}} \frac{c_{ij}}{c_{pq}}
        \underbrace{\ell_{p+2} \cdots \ell_i}_{:= 1 \text{ if } i = p+1} m_1 \cdots m_j \\
    \Longleftrightarrow& \:
    \ell_{p+1} \mid m_1 \cdots m_q.
  \end{align*}

  Since all $\ell_i, m_j$ are linear forms,
  $\ell_{p+1} \mid m_j$ for some $1 \leq j \leq q$,
  which is absurd since no $\ell_i = \lambda m_j$, $\lambda \in k$.
  Therefore, $\mathscr{B}'$ is an independent set.

\item[(3)]
  Since
  \[
    |\mathscr{B}'| = d+1 = \dim_{k} V(d,2),
  \]
  $\mathscr{B}'$ is also a basis for $V(d,2)$.
\end{enumerate}
$\Box$ \\\\



%%%%%%%%%%%%%%%%%%%%%%%%%%%%%%%%%%%%%%%%%%%%%%%%%%%%%%%%%%%%%%%%%%%%%%%%%%%%%%%%



\subsubsection*{Problem 2.36.}
\addcontentsline{toc}{subsubsection}{Problem 2.36.}
\emph{With the above notation, show that
\[
  \dim V(d,n) = {d+n-1 \choose n-1},
\]
the binomial coefficient.} \\



\emph{Proof.}
See the proof of Problem 2.35(b).
$\Box$ \\\\



%%%%%%%%%%%%%%%%%%%%%%%%%%%%%%%%%%%%%%%%%%%%%%%%%%%%%%%%%%%%%%%%%%%%%%%%%%%%%%%%
%%%%%%%%%%%%%%%%%%%%%%%%%%%%%%%%%%%%%%%%%%%%%%%%%%%%%%%%%%%%%%%%%%%%%%%%%%%%%%%%



\subsection*{2.7. Direct Products of Rings \\}
\addcontentsline{toc}{subsection}{2.7. Direct Products of Rings}



\subsubsection*{Problem 2.37.}
\addcontentsline{toc}{subsubsection}{Problem 2.37.}
\emph{What are the additive and multiplicative identities in $\bigtimes R_i$?
Is the map from $R_i$ to $\bigtimes R_i$ taking $a_i$ to $(0,\ldots,a_i,\ldots,0)$ a ring homomorphism?} \\

\emph{Proof.}
\begin{enumerate}
\item[(1)]
  $(0, \ldots, 0)$ is the additive identity in $\bigtimes R_i$.

\item[(2)]
  $(1, \ldots, 1)$ is the multiplicative identity in $\bigtimes R_i$.

\item[(3)]
  The map $\alpha: R_i \to \bigtimes R_i$ taking $a_i$ to $(0,\ldots,a_i,\ldots,0)$
  is not a ring homomorphism
  since
  \[
    \alpha(1) = (0,\ldots,1,\ldots,0) \neq (1, \ldots, 1),
  \]
  or $\alpha$ is not multiplicative identity preserving
  (if $R_j$ is not the zero ring for some $j \neq i$).
\end{enumerate}
$\Box$ \\\\



%%%%%%%%%%%%%%%%%%%%%%%%%%%%%%%%%%%%%%%%%%%%%%%%%%%%%%%%%%%%%%%%%%%%%%%%%%%%%%%%



\subsubsection*{Problem 2.38.*}
\addcontentsline{toc}{subsubsection}{Problem 2.38.*}
\emph{Show that if $k \subseteq R_i$,
and each $R_i$ is finite-dimensional over $k$, then
$\dim\left(\bigtimes R_i\right) = \sum \dim(R_i)$.} \\

\emph{Proof.}
\begin{enumerate}
\item[(1)]
  In the terminology of linear algebra,
  $\bigtimes R_i$ is the direct sum $\bigoplus R_i$ of $R_i$.

\item[(2)]
  Hence,
  \[
    \dim_{k} \left(\bigoplus R_i\right) = \sum \dim_{k}(R_i).
  \]
\end{enumerate}
$\Box$ \\\\



%%%%%%%%%%%%%%%%%%%%%%%%%%%%%%%%%%%%%%%%%%%%%%%%%%%%%%%%%%%%%%%%%%%%%%%%%%%%%%%%
%%%%%%%%%%%%%%%%%%%%%%%%%%%%%%%%%%%%%%%%%%%%%%%%%%%%%%%%%%%%%%%%%%%%%%%%%%%%%%%%



\subsection*{2.8. Operations with Ideals \\}
\addcontentsline{toc}{subsection}{2.8. Operations with Ideals}



\subsubsection*{Problem 2.39.*}
\addcontentsline{toc}{subsubsection}{Problem 2.39.*}
\emph{Prove the following relations among ideals $I_i$, $J$ in a ring $R$:} \\
\begin{enumerate}
\item[(a)]
  $(I_1 + I_2) J = I_1 J + I_2 J$.

\item[(b)]
  $(I_1 \cdots I_N)^n = I_1^n \cdots I_N^n$. \\
\end{enumerate}



\emph{Proof of (a).}
\begin{enumerate}
\item[(1)]
  Note that $(I_1 + I_2) J$ and $I_1 J + I_2 J$ are ideals.

\item[(2)]
  \emph{Show that $(I_1 + I_2) J \subseteq I_1 J + I_2 J$.}
  Given any
  \[
    (x_{1} + x_{2}) y \in (I_1 + I_2) J
  \]
  where $x_{i} \in I_i$ and $y \in J$.
  It suffices to show that $(x_{1} + x_{2}) y \in I_1 J + I_2 J$ (by (1)).
  In fact,
  \[
    (x_{1} + x_{2}) y = x_{1} y + x_{2} y \in I_1 J + I_2 J.
  \]

\item[(3)]
  \emph{Show that $(I_1 + I_2) J \supseteq I_1 J + I_2 J$.}
  Given any
  \[
    x_{1} y_{1} + x_{2} y_{2} \in I_1 J + I_2 J
  \]
  where $x_{i} \in I_i$ and $y_{i} \in J$.
  It suffices to show that $x_{1} y_{1} + x_{2} y_{2} \in (I_1 + I_2) J$ (by (1)).
  In fact,
  \[
    x_{1} y_{1} + x_{2} y_{2}
    = (x_{1}+\underbrace{0}_{\in I_2}) y_{1} + (\underbrace{0}_{\in I_1}+x_{2}) y_{2}
    \in (I_1 + I_2) J
  \]
  since $(I_1 + I_2) J$ is an ideal.
\end{enumerate}
$\Box$ \\



\emph{Proof of (b).}
\begin{enumerate}
\item[(1)]
  Note that $(I_1 \cdots I_N)^n$ and $I_1^n \cdots I_N^n$ are ideals.

\item[(2)]
  \emph{Show that $(I_1 \cdots I_N)^n \subseteq I_1^n \cdots I_N^n$.}
  Given any
  \[
    x = x_1 \cdots x_n
  \]
  where $x_i \in I_1 \cdots I_N$.
  It suffices to show that $x \in I_1^n \cdots I_N^n$ (by (1)).
  For each $x_i \in I_1 \cdots I_N$, write
  \[
    x_i = \sum_{j(i)} x_{j(i),1} \cdots x_{j(i),N}
  \]
  where $x_{j(i),k} \in I_k$ for $1 \leq k \leq N$.
  Hence
  \begin{align*}
    x
    &= x_1 \cdots x_n \\
    &= \left(\sum_{j(1)} x_{j(1),1} \cdots x_{j(1),N}\right)
      \cdots
      \left(\sum_{j(n)} x_{j(n),1} \cdots x_{j(n),N}\right) \\
    &= \sum_{j(1),\ldots,j(n)} (x_{j(1),1} \cdots x_{j(1),N})
      \cdots (x_{j(n),1} \cdots x_{j(n),N}) \\
    &= \sum_{j(1),\ldots,j(n)}
      (\underbrace{x_{j(1),1} \cdots x_{j(n),1}}_{\in I_1^n})
      \cdots
      (\underbrace{x_{j(1),N} \cdots x_{j(n),N}}_{\in I_N^n}) \\
    &\in I_1^n \cdots I_N^n.
  \end{align*}

\item[(3)]
  \emph{Show that $(I_1 \cdots I_N)^n \supseteq I_1^n \cdots I_N^n$.}
  Given any
  \[
    x = x_1 \cdots x_N \in I_1^n \cdots I_N^n
  \]
  where $x_i \in I_i^n$ ($1 \leq i \leq N$).
  It suffices to show that
  $x \in (I_1 \cdots I_N)^n$ (by (1)).
  For each $x_i \in I_i^n$, write
  \[
    x_i = \sum_{j(i)} x_{j(i),1} \cdots x_{j(i),n}
  \]
  where $x_{j(i),k} \in I_i$ for $1 \leq k \leq n$.
  Hence
  \begin{align*}
    x
    &= x_1 \cdots x_N \\
    &= \left(\sum_{j(1)} x_{j(1),1} \cdots x_{j(1),n}\right)
      \cdots
      \left(\sum_{j(N)} x_{j(N),1} \cdots x_{j(N),n}\right) \\
    &= \sum_{j(1),\ldots,j(N)} (x_{j(1),1} \cdots x_{j(1),n})
      \cdots (x_{j(N),1} \cdots x_{j(N),n}) \\
    &= \sum_{j(1),\ldots,j(N)}
      (\underbrace{x_{j(1),1} \cdots x_{j(N),1}}_{\in I_1 \cdots I_N})
      \cdots
      (\underbrace{x_{j(1),n} \cdots x_{j(N),n}}_{\in I_1 \cdots I_N}) \\
    &\in (I_1 \cdots I_N)^n.
  \end{align*}
\end{enumerate}
$\Box$ \\\\



%%%%%%%%%%%%%%%%%%%%%%%%%%%%%%%%%%%%%%%%%%%%%%%%%%%%%%%%%%%%%%%%%%%%%%%%%%%%%%%%



\subsubsection*{Problem 2.40.* (Chinese remainder theorem)}
\addcontentsline{toc}{subsubsection}{Problem 2.40.* (Chinese remainder theorem)}
\begin{enumerate}
\item[(a)]
  \emph{Suppose $I, J$ are comaximal ideals in $R$.
  Show that $I + J^2 = R$. Show that $I^m$ and $J^n$ are comaximal for all $m, n$.}

\item[(b)]
  \emph{Suppose $I_1, \ldots, I_N$ are ideals in $R$,
  and $I_i$ and $J_i = \cap_{j \neq i} I_j$ are comaximal for all $i$.
  Show that
  \[
    I_1^{n} \cap \cdots \cap I_N^{n}
    = (I_1 \cdots I_N)^{n}
    = (I_1 \cap \cdots \cap I_N)^{n}
  \]
  for all $n$.} \\
\end{enumerate}



\emph{Proof of (a).}
\begin{enumerate}
\item[(1)]
  It suffices to show that $I^m + J^n = R$.

\item[(2)]
  Since $I^m + J^n \subseteq R$ is always true,
  it suffices to show that $I^m + J^n \supseteq R$.
  In fact,
  \begin{align*}
    R
    &= R^{m+n-1}
      &(1 \in R) \\
    &= (I+J)^{m+n-1}
      &(\text{$I, J$ are comaximal}) \\
    &= \sum_{i=0}^{m+n-1} I^i J^{m+n-1-i}
      &(\text{Problem 2.39}) \\
    &\subseteq I^m + J^n
  \end{align*}
  for all positive integers $m, n$.
  (If $m = 0$ or $n = 0$, then nothing to prove.)
\end{enumerate}
$\Box$ \\



\emph{Proof of (b).}
\begin{enumerate}
\item[(1)]
  \emph{Show that $I_i$ and $I_j$ are comaximal if $i \neq j$.}
  Note that
  \[
    R = I_i + J_i \subseteq I_i + I_j \subseteq R
  \]
  if $i \neq j$.

\item[(2)]
  \emph{If $I_i$ is comaximal to $I_j$ and $I_{j'}$.
  Show that $I_i$ is also comaximal to $I_j I_{j'}$.}
  \begin{align*}
    R
    &= (I_i + I_j)(I_i + I_{j'}) \\
    &= I_i(I_i + I_j + I_{j'}) + I_j I_{j'}
      &(\text{Problem 2.39(a)}) \\
    &\subseteq I_i + I_j I_{j'} \subseteq R.
  \end{align*}

\item[(3)]
  By (2), it is easy to get that $I_i$ and $\prod_{j \neq i} I_j$
  are comaximal by induction on the number of $I_j$ for $j \neq i$.

\item[(4)]
  \emph{Show that $I_1 \cdots I_N = I_1 \cap \cdots \cap I_N$.}
  Induction on $N$.
  \begin{align*}
    I_1 \cap \cdots \cap I_N
    &= I_1 \cap (I_2 \cap \cdots \cap I_N) \\
    &= I_1 \cap (I_2 \cdots I_N)
      &(\text{Induction hypothesis}) \\
    &= I_1 \cdot (I_2 \cdots I_N)
      &(\text{(3)}) \\
    &= I_1 \cdots I_N.
  \end{align*}

\item[(5)]
  Note that $I_i^n$ and $I_j^n$ are comaximal if $i \neq j$ by (a).
  We can apply the same argument in (2)(3)(4) to show that
  \[
    I_1^{n} \cdots I_N^{n} = I_1^{n} \cap \cdots \cap I_N^{n}.
  \]

\item[(6)]
  Therefore,
  \begin{align*}
    (I_1 \cap \cdots \cap I_N)^{n}
    &= (I_1 \cdots I_N)^{n}
      &((4)) \\
    &= I_1^{n} \cdots I_N^{n}
      &(\text{Problem 2.39(b)}) \\
    &= I_1^{n} \cap \cdots \cap I_N^{n}
      &((5)).
  \end{align*}
\end{enumerate}
$\Box$ \\\\



%%%%%%%%%%%%%%%%%%%%%%%%%%%%%%%%%%%%%%%%%%%%%%%%%%%%%%%%%%%%%%%%%%%%%%%%%%%%%%%%



\subsubsection*{Problem 2.41.*}
\addcontentsline{toc}{subsubsection}{Problem 2.41.*}
\emph{Let $I$, $J$ be ideals in $R$.
Suppose $I$ is finitely generated and $I \subseteq \mathrm{rad}(J)$.
Show that $I^n \subseteq J$ for some $n$.} \\

\emph{Proof.}
\begin{enumerate}
\item[(1)]
  Let $I$ be generated by $x_1,\ldots,x_m \in I$.
  As $I \subseteq \mathrm{rad}(J)$, there are integers $n_i > 0$
  such that $x_i^{n_i} \in J$.

\item[(2)]
  Let $N = n_1 + \cdots + n_m$.
  Given any $x = \sum_{i=1}^{m} r_i x_i \in I$,
  so
  \begin{align*}
    x^N
    &= \left( \sum_{i=1}^{m} r_i x_i \right)^{N} \\
    &= \sum_{k_1 + \cdots + k_m = N} {N \choose k_1,\ldots,k_m}
      r_1^{k_1} x_1^{k_1} \cdots r_m^{k_m} x_m^{k_m}.
  \end{align*}

\item[(3)]
  Note that for each term there is some $j$ such that $k_j \geq n_j$.
  Hence,
  \begin{align*}
    & \: x_j^{k_j} = x_j^{k_j-n_j} x_j^{n_j} \in J
      &\text{($J$ is an ideal)} \\
    \Longrightarrow& \:
    r_1^{k_1} x_1^{k_1} \cdots r_m^{k_m} x_m^{k_m} \in J \text{ for each term }
      &\text{($J$ is an ideal)} \\
    \Longrightarrow& \:
    x^N \in J.
      &\text{($J$ is an ideal)} \\
    \Longrightarrow& \:
      I^N \subseteq J.
  \end{align*}
\end{enumerate}
$\Box$ \\\\



\textbf{Supplement.}
\emph{(Exercise 1.13 in the textbook:
Eisenbud, Commutative Algebra with a View Toward Algebraic Geometry.)}
\emph{Suppose that $I$ is an ideal in a commutative ring.
Show that if $\mathrm{rad}(I)$ is finitely generated,
then for some integer $N$ we have $(\mathrm{rad}(I))^N \subseteq I$.
Conclude that in a Noetherian ring the ideals $I$ and $J$ have the same radical
iff there is some integer $N$ such that $I^N \subseteq J$ and $J^N \subseteq I$.
Use the Nullstellensatz to deduce that if $I, J \subseteq S = k[x_1,\ldots,x_n]$
are ideals and $k$ is algebraically closed,
then $Z(I) = Z(J)$ iff $I^N \subseteq J$ and $J^N \subseteq I$ for some $N$.} \\

\emph{Proof.}
\begin{enumerate}
  \item[(1)]
  \emph{Show that if $\mathrm{rad}(I)$ is finitely generated,
  then for some integer $N$ we have $(\mathrm{rad}(I))^N \subseteq I$.}
  Say $x_1, \ldots, x_m \in \mathrm{rad}(I)$ generate $\mathrm{rad}(I)$.
  \begin{enumerate}
    \item[(a)]
    For each $i$, there exists an integer $n_i > 0$ such that $x_i^{n_i} \in I$
    (since $\mathrm{rad}(I)$ is radical).
    \item[(b)]
    Let $N = n_1 + \cdots + n_m$.
    Given any $x = \sum_{i=1}^{m} r_i x_i \in \mathrm{rad}(I)$,
    so
    \begin{align*}
      x^N
      &= \left( \sum_{i=1}^{m} r_i x_i \right)^{N} \\
      &= \sum_{k_1 + \cdots + k_m = N} {N \choose k_1,\ldots,k_m}
        r_1^{k_1} x_1^{k_1} \cdots r_m^{k_m} x_m^{k_m}.
    \end{align*}
    \item[(c)]
    Note that for each term there is some $j$ such that $k_j \geq n_j$.
    Hence,
    \begin{align*}
      & \: x_j^{k_j} = x_j^{k_j-n_j} x_j^{n_j} \in I
        &\text{($I$ is an ideal)} \\
      \Longrightarrow& \:
      r_1^{k_1} x_1^{k_1} \cdots r_m^{k_m} x_m^{k_m} \in I \text{ for each term }
        &\text{($I$ is an ideal)} \\
      \Longrightarrow& \:
      x^N \in I.
        &\text{($I$ is an ideal)} \\
      \Longrightarrow& \:
        (\mathrm{rad}(I))^N \subseteq I.
    \end{align*}
  \end{enumerate}
  \item[(2)]
  \emph{Show that in a Noetherian ring the ideals $I$ and $J$ have the same radical
  iff there is some integer $N$ such that $I^N \subseteq J$ and $J^N \subseteq I$.}
  \begin{enumerate}
    \item[(a)]
    $(\Longrightarrow)$
    Since in a Noetherian ring every ideal is finitely generated,
    $\mathrm{rad}(I)$ and $\mathrm{rad}(J)$ are finitely generated.
    By (1), there is a common integer $N$ such that
    \[
      (\mathrm{rad}(I))^N \subseteq I \:\: \text{ and } \:\:
      (\mathrm{rad}(J))^N \subseteq J.
    \]
    Note that $I^N \subseteq (\mathrm{rad}(I))^N$ and $J^N \subseteq (\mathrm{rad}(J))^N$.
    Since $\mathrm{rad}(I)$ = $\mathrm{rad}(J)$ by assumption,
    \begin{align*}
      I^N &\subseteq (\mathrm{rad}(I))^N = (\mathrm{rad}(J))^N \subseteq J, \\
      J^N &\subseteq (\mathrm{rad}(J))^N = (\mathrm{rad}(I))^N \subseteq I.
    \end{align*}
    \item[(b)]
    $(\Longleftarrow)$
    It suffices to show that $\mathrm{rad}(I) \subseteq \mathrm{rad}(J)$.
    $\mathrm{rad}(J) \subseteq \mathrm{rad}(I)$ is similar.
    Given any $x \in \mathrm{rad}(I)$, there is an integer $M > 0$ such that
    $x^M \in I$.
    Hence $x^{MN} \in I^N \subseteq J$, or $x \in \mathrm{rad}(J)$.
  \end{enumerate}
  \item[(3)]
  \emph{Show that if $I, J \subseteq S = k[x_1,\ldots,x_n]$
  are ideals and $k$ is algebraically closed,
  then $Z(I) = Z(J)$ iff $I^N \subseteq J$ and $J^N \subseteq I$ for some $N$.}
  Note that $S$ is Noetherian and we can apply part (2).
  By the Nullstellensatz, $Z(I) = Z(J)$ iff $\mathrm{rad}(I) = \mathrm{rad}(J)$
  iff $I^N \subseteq J$ and $J^N \subseteq I$ for some $N$.
\end{enumerate}
$\Box$ \\\\



%%%%%%%%%%%%%%%%%%%%%%%%%%%%%%%%%%%%%%%%%%%%%%%%%%%%%%%%%%%%%%%%%%%%%%%%%%%%%%%%



\subsubsection*{Problem 2.42.* (Isomorphism theorems for rings)}
\addcontentsline{toc}{subsubsection}{Problem 2.42.* (Isomorphism theorems for rings)}
\begin{enumerate}
\item[(a)]
  \emph{Let $I \subseteq J$ be ideals in a ring $R$.
  Show that there is a natural ring homomorphism from $R/I$ onto $R/J$.}

\item[(b)]
  \emph{Let $I$ be an ideal in a ring $R$, $R$ a subring of a ring $S$.
  Show that there is a natural ring homomorphism from $R/I$ to $S/IS$.} \\
\end{enumerate}



\emph{Proof of (a).}
\begin{enumerate}
\item[(1)]
  Define a map $\alpha: R/I \to R/J$ by $\alpha(r + I) = r + J$.

\item[(2)]
  \emph{Show that $\alpha$ is well-defined.}
  If $a+I = b+I$,
  then $a-b \in I \subseteq J$ or $a+J = b+J$.
  Hence, $\alpha(a+I) = a+J = b+J = \alpha(b+I)$.

\item[(3)]
  \emph{Show that $\alpha$ is a surjective homomorphism.}
  \begin{enumerate}
  \item[(a)]
    \emph{$\alpha$ is addition preserving.}
    \begin{align*}
      \alpha((a+I) + (b+I))
      &= \alpha(a+b + I) \\
      &= a+b + J \\
      &= (a+J) + (b+J) \\
      &= \alpha(a+I) + \alpha(b+I).
    \end{align*}

  \item[(b)]
    \emph{$\alpha$ is multiplication preserving.}
    \begin{align*}
      \alpha((a+I)(b+I))
      &= \alpha(ab + I) \\
      &= ab + J \\
      &= (a+J)(b+J) \\
      &= \alpha(a+I)\alpha(b+I).
    \end{align*}

  \item[(c)]
    \emph{$\alpha$ is multiplicative identity preserving.}
    $\alpha(1+I) = 1+J$.

  \item[(d)]
    $\alpha$ is surjective since for any $a+J \in R/J$
    there is an element $a+I \in R/I$ such that $\alpha(a+I) = a+J$.
  \end{enumerate}

\item[(4)]
  Note that $\ker(\alpha) = J/I$. So $(R/I)/(J/I) \cong R/J$.
\end{enumerate}
$\Box$ \\



\emph{Proof of (b).}
\begin{enumerate}
\item[(1)]
  $I$ is not necessary an ideal of $S$; $IS$ an ideal of $S$ (and thus $S/IS$ is well-defined).

\item[(2)]
  Define a map $\alpha: R/I \to S/IS$ by $\alpha(r + I) = r + IS$.
  Note that $I \subseteq IS$ as a subset in $S$.
  Apply the same argument in (a),
  $\alpha$ is well-defined
  and $\alpha$ is a surjective homomorphism.

\item[(3)]
    Note that $\ker(\alpha) = (R \cap SI)/I$.
    So $(R/I)/((R \cap SI)/I) \cong S/IS$.
\end{enumerate}
$\Box$ \\\\



%%%%%%%%%%%%%%%%%%%%%%%%%%%%%%%%%%%%%%%%%%%%%%%%%%%%%%%%%%%%%%%%%%%%%%%%%%%%%%%%



\subsubsection*{Problem 2.43.*}
\addcontentsline{toc}{subsubsection}{Problem 2.43.*}
\emph{Let $P = (0, \ldots, 0) \in \mathbf{A}^{n}$,
$\mathscr{O} = \mathscr{O}_P(\mathbf{A}^{n})$,
$\mathfrak{m} = \mathfrak{m}_P(\mathbf{A}^{n})$.
Let $I = (x_1, \ldots, x_n) \subseteq k[x_1,\ldots,x_n]$ be the ideal generated by $x_1, \ldots, x_n$.
Show that $I\mathscr{O} = \mathfrak{m}$,
so $I^r \mathscr{O} = \mathfrak{m}^r$ for all $r$.} \\



\emph{Proof.}
\begin{enumerate}
\item[(1)]
  By the defintion
  \[
    \mathfrak{m} = \{ f \in \mathscr{O} : f(P) = 0 \},
  \]
  $I\mathscr{O} \subseteq \mathfrak{m}$.
  Conversely, by Problem 1.7(b) we have $I\mathscr{O} \supseteq \mathfrak{m}$.

\item[(2)]
  By Problem 2.39(b),
  \[
    \mathfrak{m}^r
    = (I\mathscr{O})^r
    = I^r \mathscr{O}^r
    = I^r \mathscr{O}.
  \]
  Here $\mathscr{O}^r = \mathscr{O}$ since $1 \in \mathscr{O}$.
\end{enumerate}
$\Box$ \\\\



%%%%%%%%%%%%%%%%%%%%%%%%%%%%%%%%%%%%%%%%%%%%%%%%%%%%%%%%%%%%%%%%%%%%%%%%%%%%%%%%



\subsubsection*{Problem 2.44.*}
\addcontentsline{toc}{subsubsection}{Problem 2.44.*}
\emph{Let $V$ be a variety in $\mathbf{A}^{n}$,
$I = I(V) \subseteq k[x_1,\ldots,x_n]$, $P \in V$,
and let $J$ be an ideal of $k[x_1,\ldots,x_n]$ that contains $I$.
Let $J'$ be the image of $J$ in $\Gamma(V)$.
Show that there is a natural homomorphism $\varphi$ from
$\mathscr{O}_P(\mathbf{A}^{n}) / J\mathscr{O}_P(\mathbf{A}^{n})$
to $\mathscr{O}_P(V) / J'\mathscr{O}_P(V)$,
and that $\varphi$ is an isomorphism.
In particular, $\mathscr{O}_P(\mathbf{A}^{n}) / I\mathscr{O}_P(\mathbf{A}^{n})$
is isomorphic to $\mathscr{O}_P(V)$.} \\



\emph{Proof.}
\begin{enumerate}
\item[(1)]
  Define $\varphi$ from
  $\mathscr{O}_P(\mathbf{A}^{n})/J \mathscr{O}_P(\mathbf{A}^{n})$
  to $\mathscr{O}_P(V) / J'\mathscr{O}_P(V)$ by
  \[
    \varphi: a/b + J \mathscr{O}_P(\mathbf{A}^{n})
    \mapsto \overline{a}/\overline{b} + J' \mathscr{O}_P(V).
  \]
  It is well-defined since $\varphi(J \mathscr{O}_P(\mathbf{A}^{n})) = J' \mathscr{O}_P(V)$
  and $b(P) \neq 0$ implies that $\overline{b}(P) \neq 0$.

\item[(2)]
  Note that $V$ is a subvariety of $\mathbf{A}^{n}$.
  So $\varphi: \Gamma(\mathbf{A}^{n}) \to \Gamma(V)$ is a ring homomorphism by Problem 2.3
  and then $\varphi$ extends uniquely to a ring homomorphism
  by using the similar argument in Problem 2.21.

\item[(3)]
  $\varphi$ is surjective since
  $\mathscr{O}_P(\mathbf{A}^{n}) \hookrightarrow \mathscr{O}_P(V)$
  and $\varphi(J \mathscr{O}_P(\mathbf{A}^{n})) = J' \mathscr{O}_P(V)$.
  $\varphi$ is injective since $\varphi(J \mathscr{O}_P(\mathbf{A}^{n})) = J' \mathscr{O}_P(V)$.
  Hence $\varphi: \mathscr{O}_P(\mathbf{A}^{n})/J \to \mathscr{O}_P(\mathbf{A}^{n})$
  is isomorphic.
  In particular,
  $\mathscr{O}_P(\mathbf{A}^{n}) / I\mathscr{O}_P(\mathbf{A}^{n}) \cong \mathscr{O}_P(V)$
  (by taking $J = I$ and noting that $J' = I' = 0$).
\end{enumerate}
$\Box$ \\\\



%%%%%%%%%%%%%%%%%%%%%%%%%%%%%%%%%%%%%%%%%%%%%%%%%%%%%%%%%%%%%%%%%%%%%%%%%%%%%%%%



\subsubsection*{Problem 2.45.*}
\addcontentsline{toc}{subsubsection}{Problem 2.45.*}
\emph{Show that ideals $I, J \subseteq k[x_1,\ldots,x_n]$ ($k$ algebraically closed) are comaximal
if and only if $V(I) \cap V(J) = \varnothing$.} \\



\emph{Proof.}
\begin{enumerate}
\item[(1)]
  \emph{Show that $V(I) \cap V(J) = V(I + J)$.}
  \begin{align*}
    P \in V(I) \cap V(J)
    &\Longleftrightarrow
    f(P) = 0 \: \forall f \in I \text{ and } g(P) = 0 \: \forall g \in J \\
    &\Longleftrightarrow
    f(P) = 0 \: \forall f \in I + J \\
    &\Longleftrightarrow
    P \in V(I + J).
  \end{align*}

\item[(2)]
  Hence,
  \begin{align*}
    \varnothing = V(I) \cap V(J)
    &\Longleftrightarrow
    \varnothing = V(I + J)
      &((1)) \\
    &\Longleftrightarrow
    I + J = k[x_1,\ldots,x_n]
      &(\text{Weak Nullstellensatz}) \\
    &\Longleftrightarrow
    \text{$I$ and $J$ are comaximal}.
  \end{align*}
\end{enumerate}
$\Box$ \\\\



%%%%%%%%%%%%%%%%%%%%%%%%%%%%%%%%%%%%%%%%%%%%%%%%%%%%%%%%%%%%%%%%%%%%%%%%%%%%%%%%



\subsubsection*{Problem 2.46.*}
\addcontentsline{toc}{subsubsection}{Problem 2.46.*}
\emph{Let $I = (x,y) \subseteq k[x,y]$.
Show that}
\[
  \dim_{k}(k[x,y]/I^n) = 1 + 2 + \cdots + n = \frac{n(n+1)}{2}.
\]

\emph{Proof.}
\begin{enumerate}
\item[(1)]
  The set
  \[
    \mathscr{B} = \{ x^i y^j + I^n : i, j \in \mathbb{Z}, i, j \geq 0, i + j < n \}
  \]
  generates $k[x,y]/I^n$ as a $k$-vector space.
  Besides, each nonzero element in $I^n$ has the degree $\geq n$,
  and thus $\mathscr{B}$ is an independent set.
  Therefore, $\mathscr{B}$ is a basis for $k[x,y]/I^n$.

\item[(2)]
  Hence,
  \[
    \dim_{k}(k[x,y]/I^n)
    = |\mathscr{B}|
    = 1 + 2 + \cdots + n
    = \frac{n(n+1)}{2}.
  \]
\end{enumerate}
$\Box$ \\\\



%%%%%%%%%%%%%%%%%%%%%%%%%%%%%%%%%%%%%%%%%%%%%%%%%%%%%%%%%%%%%%%%%%%%%%%%%%%%%%%%
%%%%%%%%%%%%%%%%%%%%%%%%%%%%%%%%%%%%%%%%%%%%%%%%%%%%%%%%%%%%%%%%%%%%%%%%%%%%%%%%



\subsection*{2.9. Ideals with a Finite Number of Zeros \\}
\addcontentsline{toc}{subsection}{2.9. Ideals with a Finite Number of Zeros}



\subsubsection*{Problem 2.47.}
\addcontentsline{toc}{subsubsection}{Problem 2.47.}
\emph{Suppose $R$ is a ring containing $k$, and $R$ is finite dimensional over $k$.
Show that $R$ is isomorphic to a direct product of local rings.} \\



\emph{Proof.}
\begin{enumerate}
\item[(1)]
  Let $\{ v_1, \ldots, v_n \}$ be a basis for $R$ over $k$ (as a vector space).
  Define a $k$-module homomorphism $\alpha: k[x_1,\ldots,x_n] \to R$
  by $\alpha(x_i) = v_i$.
  Clearly, $\alpha$ is surjective and thus
  \[
    R \cong k[x_1,\ldots,x_n]/\ker(\alpha)
  \]
  as a $k$-module isomorphism.
  Note that $\ker(\alpha)$ is an ideal of $k[x_1,\ldots,x_n]$.

\item[(2)]
  Write $I := \ker(\alpha)$.
  Hence,
  \[
    \dim_k(k[x_1,\ldots,x_n]/I) = \dim_k(R) < \infty.
  \]
  By Corollary 4 to the Hilbert's Nullstellensatz in \S 1.7,
  $V(I)$ is finite.

\item[(3)]
  Write $V(I) = \{ P_1, \ldots, P_N \}$ and $\mathscr{O}_i = \mathscr{O}_{P_i}(\mathbf{A}^n)$.
  By Proposition 6,
  \[
    R \cong k[x_1,\ldots,x_n]/I \cong \prod_{i=1}^{N} \mathscr{O}_i/I\mathscr{O}_i,
  \]
  which is isomorphic to a direct product of local rings.
\end{enumerate}
$\Box$ \\\\



%%%%%%%%%%%%%%%%%%%%%%%%%%%%%%%%%%%%%%%%%%%%%%%%%%%%%%%%%%%%%%%%%%%%%%%%%%%%%%%%
%%%%%%%%%%%%%%%%%%%%%%%%%%%%%%%%%%%%%%%%%%%%%%%%%%%%%%%%%%%%%%%%%%%%%%%%%%%%%%%%



\subsection*{2.10. Quotient Modules and Exact Sequences \\}
\addcontentsline{toc}{subsection}{2.10. Quotient Modules and Exact Sequences}



\subsubsection*{Problem 2.48.*}
\addcontentsline{toc}{subsubsection}{Problem 2.48.*}
\emph{Verify that for any $R$-module homomorphism $\varphi: M \to M'$,
$\ker(\varphi)$ and $\mathrm{im}(\varphi)$ are submodules of $M$ and $M'$ respectively.
Show that
\[
  0
  \to \ker(\varphi)
  \to M
  \xrightarrow{\varphi} \mathrm{im}(\varphi)
  \to 0
\]
is exact.} \\



\emph{Proof.}
\begin{enumerate}
\item[(1)]
  \emph{Show that $\ker(\varphi)$ is a subgroup of $M$.}
  It suffices to show that $a - b \in \ker(\varphi)$ for all $a, b \in \ker(\varphi)$.
  In fact, $\varphi(a - b) = \varphi(a) - \varphi(b) = 0 - 0 = 0$,
  or $a - b \in \ker(\varphi)$.

\item[(2)]
  \emph{Show that $\ker(\varphi)$ is a submodule of $M$.}
  By (1), it suffices to show that $ra \in \ker(\varphi)$ for all $r \in R$ and $a \in \ker(\varphi)$.
  In fact, $\varphi(ra) = r \cdot \varphi(a) = r \cdot 0 = 0$,
  or $ra \in \ker(\varphi)$.

\item[(3)]
  \emph{Show that $\mathrm{im}(\varphi)$ is a subgroup of $M'$.}
  It suffices to show that $a - b \in \mathrm{im}(\varphi)$ for all $a, b \in \mathrm{im}(\varphi)$.
  As $a, b \in \mathrm{im}(\varphi)$,
  there are two elements $a', b' \in M$ such that $\varphi(a') = a$ and $\varphi(b') = b$.
  So $\varphi(a' - b') = \varphi(a') - \varphi(b') = a - b$,
  or $a - b \in \mathrm{im}(\varphi)$.

\item[(4)]
  \emph{Show that $\mathrm{im}(\varphi)$ is a submodule of $M$.}
  By (3), it suffices to show that $ra \in \mathrm{im}(\varphi)$
  for all $r \in R$ and $a \in \mathrm{im}(\varphi)$.
  As $a \in \mathrm{im}(\varphi)$,
  there is one element $a' \in M$ such that $\varphi(a') = a$.
  So $\varphi(ra') = r\varphi(a') = ra$,
  or $ra \in \mathrm{im}(\varphi)$.

\item[(5)]
  \emph{Show that
  \[
    0
    \to \ker(\varphi)
    \xrightarrow{i} M
    \xrightarrow{\varphi} \mathrm{im}(\varphi)
    \to 0
  \]
  is exact.}
  Note that $\ker(\varphi) \xrightarrow{i} M$ is the natural inclusion
  and $M \xrightarrow{\varphi} \mathrm{im}(\varphi)$ is surjective.
  Also, it is trivial that $\mathrm{im}(i) = \ker(\varphi)$.
\end{enumerate}
$\Box$ \\\\




%%%%%%%%%%%%%%%%%%%%%%%%%%%%%%%%%%%%%%%%%%%%%%%%%%%%%%%%%%%%%%%%%%%%%%%%%%%%%%%%



\subsubsection*{Problem 2.49.*}
\addcontentsline{toc}{subsubsection}{Problem 2.49.*}
\begin{enumerate}
\item[(a)]
  (Factor theorem for modules)
  \emph{Let $N$ be a submodule of $M$, $\pi: M \to M/N$ the natural homomorphism.
  Suppose $\varphi: M \to M'$ is a homomorphism of $R$-modules, and $\varphi(N) = 0$.
  Show that there is a unique homomorphism $\overline{\varphi}: M/N \to M'$
  such that $\overline{\varphi} \circ \pi = \varphi$.}

\item[(b)]
  (Isomorphism theorems for modules)
  \emph{If $N$ and $P$ are submodules of a module $M$, with $P \subseteq N$,
  then there are natural homomorphisms from $M/P$ onto $M/N$ and from $N/P$ into $M/P$.
  Show that the resulting sequence
  \[
    0 \to N/P \to M/P \to M/N \to 0
  \]
  is exact.}

\item[(c)]
  \emph{Let $U \subseteq W \subseteq V$ be vector spaces,
  with $V/U$ finite-dimensional.
  Then $\dim V/U = \dim V/W + \dim W/U$.}

\item[(d)]
  \emph{If $J \subseteq I$ are ideals in a ring $R$,
  there is a natural exact sequence of $R$-modules:}
  \[
    0 \to I/J \to R/J \to R/I \to 0.
  \]

\item[(e)]
  \emph{If $\mathscr{O}$ is a local ring with maximal ideal $\mathfrak{m}$,
  there is a natural exact sequence of $\mathscr{O}$-modules}
  \[
    0
    \to \mathfrak{m}^{n}/\mathfrak{m}^{n+1}
    \to \mathscr{O}/\mathfrak{m}^{n+1}
    \to \mathscr{O}/\mathfrak{m}^{n}
    \to
    0.
  \] \\
\end{enumerate}



\emph{Proof of (a).}
\begin{enumerate}
\item[(1)]
  Define $\overline{\varphi}: M/N \to M'$ by
  \[
    \overline{\varphi}(m + N) = \varphi(m).
  \]
  $\overline{\varphi}$ is well-defined since
  $m + N = n + N$ implies that $m - n \in N \subseteq \ker(\varphi)$.

\item[(2)]
  $\overline{\varphi}$ is a homomorphism of $R$-modules
  since $\varphi$ is a homomorphism of $R$-modules.

\item[(3)]
  $\overline{\varphi} \circ \pi = \varphi$ by construction.

\item[(4)]
  Suppose there is a homomorphism $\psi: M/N \to M'$
  such that $\psi \circ \pi = \varphi$.
  For any $m + N \in M/N$, we have
  \[
    \overline{\varphi}(m + N)
    = \varphi(m)
    = (\psi \circ \pi)(m)
    = \psi(\pi(m))
    = \psi(m + N).
  \]
  That is, $\psi = \overline{\varphi}$.
\end{enumerate}
$\Box$ \\



\emph{Proof of (b).}
\begin{enumerate}
\item[(1)]
  Define $\pi: M/P \twoheadrightarrow M/N$ by
  \[
    \pi: \underbrace{m + P}_{\in M/P} \mapsto \underbrace{m + N}_{\in M/N}.
  \]
  \begin{enumerate}
  \item[(a)]
    \emph{Show that $\pi$ is well-defined.}
    If $m + P = n + P \in M/P$, then $m - n \in P \subseteq N$ or $m + N = n + N \in M/N$.

  \item[(b)]
    $\pi$ is a module homomorphism since $M/N$ is a module.

  \item[(c)]
    $\pi$ is surjective by construction.
  \end{enumerate}

\item[(2)]
  Define $i: N/P \hookrightarrow M/P$ by
  \[
    i: \underbrace{m + P}_{\in N/P} \mapsto \underbrace{m + P}_{\in M/P}.
  \]
  \begin{enumerate}
  \item[(a)]
    \emph{Show that $i$ is well-defined.}
    If $m + P = n + P \in N/P$, then $m, n \in N \subseteq M$ and $m - n \in P$.
    So $m + P = n + P \in M/P$.

  \item[(b)]
    $i$ is a module homomorphism since $M/P$ is a module.

  \item[(c)]
    $i$ is injective by construction.
  \end{enumerate}

\item[(3)]
  To show that $0 \to N/P \to M/P \to M/N \to 0$ is exact,
  it suffices to show that
  $\ker(\pi) = \mathrm{im}(i) = N/P$ (by the injectivity of $i$).
  It is trivial since
  \[
    m + P \in \ker(\pi)
    \Longleftrightarrow m \in N
    \Longleftrightarrow m + P \in N/P.
  \]
\end{enumerate}
$\Box$ \\



\emph{Proof of (c).}
\begin{enumerate}
\item[(1)]
  By (b),
  \[
    0 \to W/U \to V/U \xrightarrow{\varphi} V/W \to 0
  \]
  is exact.

\item[(2)]
  By the rank-nullity theorem for a linear transformation,
  \[
    \dim V/U
    = \dim \mathrm{im}(\varphi) + \dim \mathrm{ker}(\varphi)
    = \dim V/W + \dim W/U.
  \]
\end{enumerate}
$\Box$ \\



\emph{Proof of (d).}
\begin{enumerate}
\item[(1)]
  Regard $R$ as a $R$-module and $I, J$ as submodules of a $R$-module $R$.

\item[(2)]
  As $J \subseteq I$, by (b) we have
  \[
    0 \to I/J \to R/J \to R/I \to 0.
  \]
\end{enumerate}
$\Box$ \\



\emph{Proof of (e).}
\begin{enumerate}
\item[(1)]
  Note that $\mathfrak{m}^{n+1} \subseteq \mathfrak{m}^{n}$ are ideals in a local ring $\mathscr{O}$.

\item[(2)]
  By (d), there is a natural exact sequence of $\mathscr{O}$-modules:
  \[
    0
    \to \mathfrak{m}^{n}/\mathfrak{m}^{n+1}
    \to \mathscr{O}/\mathfrak{m}^{n+1}
    \to \mathscr{O}/\mathfrak{m}^{n}
    \to
    0.
  \]
\end{enumerate}
$\Box$ \\\\



%%%%%%%%%%%%%%%%%%%%%%%%%%%%%%%%%%%%%%%%%%%%%%%%%%%%%%%%%%%%%%%%%%%%%%%%%%%%%%%%



\subsubsection*{Problem 2.50.*}
\addcontentsline{toc}{subsubsection}{Problem 2.50.*}
\emph{Let $R$ be a DVR satisfying the conditions of Problem 2.30.
Then $\mathfrak{m}^{n}/\mathfrak{m}^{n+1}$ is an $R$-module, and so also a $k$-module,
since $k \subseteq R$.}
\begin{enumerate}
\item[(a)]
  \emph{Show that $\dim_k(\mathfrak{m}^{n}/\mathfrak{m}^{n+1}) = 1$
  for all $n \geq 0$.}

\item[(b)]
  \emph{Show that $\dim_k(R/\mathfrak{m}^{n}) = n$ for all $n > 0$.}

\item[(c)]
  \emph{Let $z \in R$.
  Show that $\mathrm{ord}(z) = n$ if $(z) = \mathfrak{m}^{n}$,
  and hence that $\mathrm{ord}(z) = \dim_k(R/(z))$.} \\
\end{enumerate}



\emph{Proof of (a).}
\begin{enumerate}
\item[(1)]
  By Problem 2.49(e),
  \[
    0
    \to \mathfrak{m}^{n}/\mathfrak{m}^{n+1}
    \to R/\mathfrak{m}^{n+1}
    \to R/\mathfrak{m}^{n}
    \to 0
  \]
  is exact.

\item[(2)]
  By the rank-nullity theorem (Proposition 3),
  \begin{align*}
    \dim_k(\mathfrak{m}^{n}/\mathfrak{m}^{n+1})
    &= \dim_k(R/\mathfrak{m}^{n+1}) - \dim_k(R/\mathfrak{m}^{n}) \\
    &= (n+1) - n
      &((b)) \\
    &= 1.
  \end{align*}
\end{enumerate}
$\Box$ \\



\emph{Proof of (b).}
\begin{enumerate}
\item[(1)]
  Let $t$ be a uniformizing parameter for $R$, $z \in R$.
  By Problem 2.30(b), there are unique $\lambda_0, \ldots, \lambda_{n-1} \in k$
  and $z_{n-1} \in R$ such that
  \[
    z = \lambda_0 + \lambda_1 t + \cdots + \lambda_{n-1} t^{n-1} + z_{n-1} t^{n}.
  \]
  Hence we can define a map $\varphi: R/\mathfrak{m}^{n} \to k^n$
  by
  \[
    \varphi: z + \mathfrak{m}^{n} \mapsto (\lambda_0, \ldots, \lambda_{n-1}).
  \]

\item[(2)]
  $\varphi$ is well-defined by the uniqueness of the expression of $z$ in Problem 2.30(b).
  $\varphi$ is a $k$-module homomorphism and $\varphi$ is surjective (since $k \subseteq R$).
  $\varphi$ is injective by the uniqueness of the expression of $z$ in Problem 2.30(b).

\item[(3)]
  Hence, $R/\mathfrak{m}^{n} \cong k^n$ or $\dim_k(R/\mathfrak{m}^{n}) = n$ for $n > 0$.
  (It is also true for $n = 0$ since $\dim_k(\{0\}) = 0$.)
\end{enumerate}
$\Box$ \\



\emph{Proof of (c).}
\begin{enumerate}
\item[(1)]
  Note that $\mathfrak{m}^{n} = (t^n)$ as $\mathfrak{m} = (t)$
  where $t$ is a uniformizing parameter for $R$.

\item[(2)]
  Since $(z) = (t^n) = \mathfrak{m}^{n}$, $\mathrm{ord}(z) = n$ by Problem 2.28.
  (Here $\mathrm{ord}(z) \geq n$ by $z \in (t^n)$
  and $n \geq \mathrm{ord}(z)$ by $t^n \in (z)$.)

\item[(3)]
  Hence,
  \[
    \mathrm{ord}(z) = n = \dim_k(R/\mathfrak{m}^{n}) = \dim_k(R/(z))
  \]
  by (b).
\end{enumerate}
$\Box$ \\\\



%%%%%%%%%%%%%%%%%%%%%%%%%%%%%%%%%%%%%%%%%%%%%%%%%%%%%%%%%%%%%%%%%%%%%%%%%%%%%%%%



\subsubsection*{Problem 2.51.}
\addcontentsline{toc}{subsubsection}{Problem 2.51.}
\emph{Let
\[
  0 \longrightarrow V_1 \longrightarrow \cdots \longrightarrow V_n \longrightarrow 0
\]
be an exact sequence of finite-dimensional vector spaces.
Show that $\sum (-1)^i \dim(V_i) = 0$.} \\

\emph{Proof (Proposition 7 in \S 2.10).}
\begin{enumerate}
\item[(1)]
  For $i = 0,\ldots,n$,
  by the rank-nullity theorem for a linear transformation
  $\varphi_{i}: V_{i} \to V_{i+1}$, we have
  \[
    \dim V_{i} = \dim \mathrm{im}(\varphi_{i}) + \dim \mathrm{ker}(\varphi_{i}).
  \]
  (Here $V_0 = V_{n+1} := 0$ by convention.)

\item[(2)]
  By the exactness of the sequence, we have
  \begin{enumerate}
  \item[(a)]
    $\mathrm{im}(\varphi_{i}) = \mathrm{ker}(\varphi_{i+1})$ for $i = 0,\ldots, n-1$.
    In particular, $\mathrm{ker}(\varphi_{1}) = \mathrm{im}(\varphi_{0}) = 0$.

  \item[(b)]
    $\mathrm{ker}(\varphi_{n}) = V_n$.
  \end{enumerate}
  Hence,
  \begin{align*}
    \sum_{i=1}^{n-1} (-1)^i \dim(V_i)
    &= \sum_{i=1}^{n-1} (-1)^i \dim \mathrm{im}(\varphi_{i})
      + \sum_{i=1}^{n-1} (-1)^i \dim \mathrm{ker}(\varphi_{i}) \\
    &= \sum_{i=1}^{n-1} (-1)^i \dim \mathrm{ker}(\varphi_{i+1})
      + \sum_{i=1}^{n-1} (-1)^i \dim \mathrm{ker}(\varphi_{i}) \\
    &= (-1)^{n-1} \dim \underbrace{\mathrm{ker}(\varphi_{n})}_{=V_n}
      + (-1)^1 \dim \underbrace{\mathrm{ker}(\varphi_{1})}_{= 0} \\
    &= -(-1)^n \dim V_n,
  \end{align*}
  or $\sum (-1)^i \dim(V_i) = 0$.
\end{enumerate}
$\Box$ \\\\



%%%%%%%%%%%%%%%%%%%%%%%%%%%%%%%%%%%%%%%%%%%%%%%%%%%%%%%%%%%%%%%%%%%%%%%%%%%%%%%%



\subsubsection*{Problem 2.52.* (Isomorphism theorems for modules)}
\addcontentsline{toc}{subsubsection}{Problem 2.52.* (Isomorphism theorems for modules)}
\emph{Let $N, P$ be submodules of a module $M$.
Show that the subgroup
\[
  N + P = \{ n + p : n \in N, p \in P \}
\]
is a submodule of $M$.
Show that there is a natural $R$-module isomorphism of $N/(N \cap P)$ onto $(N+P)/P$.} \\



\emph{Proof.}
\begin{enumerate}
\item[(1)]
  \emph{Show that $N+P$ is a submodule of $M$.}
  Given any $n_1+p_1, n_2+p_2 \in N+P$,
  \[
    (n_1+p_1) + (n_2+p_2)
    = \underbrace{n_1+n_2}_{\in N} + \underbrace{p_1+p_2}_{\in P} \in N+P.
  \]
  Given any $n+p \in N+P$ and $r \in R$,
  \[
    r(n+p) = \underbrace{rn}_{\in N} + \underbrace{rp}_{\in P} \in N+P.
  \]
  Here we use the fact that $N$ and $P$ are modules.

\item[(2)]
  Define a module homomorphism $\varphi: N \to M/P$ by
  \[
    \varphi: m \mapsto m + P.
  \]
  $\ker(\varphi) = N \cap P$ and
  $\mathrm{im}(\varphi) = \{ m + P : m \in N \} = (N+P)/P$.
  By Problem 2.48,
  $\varphi$ induces a natural $R$-module isomorphism of
  $N/\ker(\varphi) = N/(N \cap P)$ onto $\mathrm{im}(\varphi) = (N+P)/P$
  (which is sending $m + (N \cap P)$ to $m + P$).


\end{enumerate}
$\Box$ \\\\



%%%%%%%%%%%%%%%%%%%%%%%%%%%%%%%%%%%%%%%%%%%%%%%%%%%%%%%%%%%%%%%%%%%%%%%%%%%%%%%%



\subsubsection*{Problem 2.53.*}
\addcontentsline{toc}{subsubsection}{Problem 2.53.*}
\emph{Let $V$ be a vector space, $W$ a subspace,
$T : V \to V$ a one-to-one linear map such that $T(W) \subseteq W$,
and assume $V/W$ and $W/T(W)$ are finite-dimensional.}
\begin{enumerate}
\item[(a)]
  \emph{Show that $T$ induces an isomorphism of $V/W$ with $T(V)/T(W)$.}

\item[(b)]
  \emph{Construct an isomorphism between $T(V)/(W \cap T(V))$ and $(W+T(V))/W$,
  and an isomorphism between $W/(W \cap T(V))$ and $(W+T(V))/T(V)$.}

\item[(c)]
  \emph{Use Problem 2.49(c) to show that
  $\dim V/(W+T(V)) = \dim (W \cap T(V))/T(W)$.}

\item[(d)]
  \emph{Conclude finally that $\dim V/T(V) = \dim W/T(W)$.} \\
\end{enumerate}



\emph{Proof of (a).}
\begin{enumerate}
\item[(1)]
  Define a map $\overline{T}: V/W \to T(V)/T(W)$ by
  \[
    \overline{T}: v + W \mapsto T(v) + T(W).
  \]

\item[(2)]
  \emph{Show that $\overline{T}$ is well-defined}.
  Suppose $u + W = v + W \in V/W$.
  So $u - v \in W$.
  So $T(u - v) = T(u) - T(v) \in T(W)$ (since $T$ is a linear map).
  Hence, $T(u) + T(W) = T(v) + T(W)$.

\item[(3)]
  $\overline{T}$ is a linear map since $T$ is a linear map.

\item[(4)]
  $\overline{T}$ is surjective by construction.
  Also, $\overline{T}$ is injective since $T$ is injective.
  Therefore, $\overline{T}: V/W \xrightarrow{\sim} T(V)/T(W)$ is isomorphic.
\end{enumerate}
$\Box$ \\



\emph{Proof of (b).}
\begin{enumerate}
\item[(1)]
  Put $N = T(V)$ and $P = W$ in Problem 2.52 to get
  \[
    T(V)/(W \cap T(V)) \cong (W+T(V))/W.
  \]

\item[(2)]
  Put $N = W$ and $P = T(V)$ in Problem 2.52 to get
  \[
    W/(W \cap T(V)) \cong (W+T(V))/T(V).
  \]
\end{enumerate}
$\Box$ \\



\emph{Proof of (c).}
\begin{enumerate}
\item[(1)]
  Note that $W \subseteq W+T(V) \subseteq V$ as vector spaces
  and $V/W$ is finite-dimensional.
  By Problem 2.49(c),
  \[
    \dim V/W = \dim V/(W+T(V)) + \dim (W+T(V))/W.
  \]

\item[(2)]
  Similarly,
  $T(V) \subseteq W \cap T(V) \subseteq T(W)$ as vector spaces
  and $T(V)/T(W)$ is finite-dimensional
  (since $V/T(W)$ is finite-dimensional and $T(V)/T(W)$ is a subspace of $V/T(W)$).
  Again by Problem 2.49(c),
  \[
    \dim T(V)/T(W) = \dim T(V)/(W \cap T(V)) + \dim (W \cap T(V))/T(W).
  \]

\item[(3)]
  By (a),
  \[
    \dim V/W = \dim T(V)/T(W).
  \]
  By (b),
  \[
    \dim (W+T(V))/W = \dim T(V)/(W \cap T(V)).
  \]
  Hence, the result is established by (1)(2).
\end{enumerate}
$\Box$ \\



\emph{Proof of (d).}
\begin{enumerate}
\item[(1)]
  Note that $V/T(V)$ is finite-dimensional.
  By Problem 2.49(c),
  \[
    \dim V/T(V) = \dim V/(W+T(V)) + \dim (W+T(V))/T(V).
  \]

\item[(2)]
  Similarly,
  \[
    \dim W/T(W) = \dim W/(W \cap T(V)) + \dim (W \cap T(V))/T(W).
  \]

\item[(3)]
  By (b),
  \[
    \dim (W+T(V))/T(V) = \dim W/(W \cap T(V)).
  \]
  By (c),
  \[
    \dim V/(W+T(V)) = \dim (W \cap T(V))/T(W).
  \]
  Hence, the result is established by (1)(2).
\end{enumerate}
$\Box$ \\\\



%%%%%%%%%%%%%%%%%%%%%%%%%%%%%%%%%%%%%%%%%%%%%%%%%%%%%%%%%%%%%%%%%%%%%%%%%%%%%%%%
%%%%%%%%%%%%%%%%%%%%%%%%%%%%%%%%%%%%%%%%%%%%%%%%%%%%%%%%%%%%%%%%%%%%%%%%%%%%%%%%



\subsection*{2.11. Free Modules \\}
\addcontentsline{toc}{subsection}{2.11. Free Modules}



\subsubsection*{Problem 2.54.}
\addcontentsline{toc}{subsubsection}{Problem 2.54.}
\emph{What does $M$ being free on $m_1, \ldots, m_n$ say in terms of the elements of $M$?} \\



\emph{Proof.}
\begin{enumerate}
\item[(1)]
  Any element $m \in M$ can be written uniquely as
  \[
    m = \sum_{i=1}^{n} r_i m_i
  \]
  for some $r_i \in R$ (which is analogous to the vector space).

\item[(2)]
  The number of members in a basis for $M$ is called the \textbf{rank} of $M$.
  That is, $n = \mathrm{rank}(M)$.
\end{enumerate}
$\Box$ \\\\



%%%%%%%%%%%%%%%%%%%%%%%%%%%%%%%%%%%%%%%%%%%%%%%%%%%%%%%%%%%%%%%%%%%%%%%%%%%%%%%%



\subsubsection*{Problem 2.55.}
\addcontentsline{toc}{subsubsection}{Problem 2.55.}
\emph{Let $f = x^n + a_1 x^{n-1} + \cdots + a_n$ be a monic polynomial in $R[x]$.
Show that $R[x]/(f)$ is a free $R$-module with basis
$\overline{1}, \overline{x}, \ldots, \overline{x}^{n-1}$,
where $\overline{x}$ is the residue of $x$.} \\



\emph{Proof.}
\begin{enumerate}
\item[(1)]
  Given any $\overline{g} \in R[x]/(f)$
  where
  \[
    g = b_0 x^m + b_1 x^{m-1} + \cdots + b_m \in R[x],
  \]
  it suffices to show that
  $\overline{g}$ is a linear combination of
  \[
    \mathscr{B} := \{ \overline{1}, \overline{x}, \ldots, \overline{x}^{n-1} \}.
  \]

\item[(2)]
  By the division-with-remainder property of $R[x]$,
  \[
    g = fq + r
  \]
  where $q, r \in R[x]$ with $r = c_0 x^{n-1} + \cdots + c_{n-1}$.
  Hence,
  \[
    \overline{g}
    = \overline{f} \overline{q} + \overline{r}
    = \overline{r}
    = c_0 \overline{x}^{n-1} + \cdots + c_{n-1} \overline{1}
  \]
  is a linear combination of $\mathscr{B}$.
\end{enumerate}
$\Box$ \\\\



%%%%%%%%%%%%%%%%%%%%%%%%%%%%%%%%%%%%%%%%%%%%%%%%%%%%%%%%%%%%%%%%%%%%%%%%%%%%%%%%



\subsubsection*{Problem 2.56.}
\addcontentsline{toc}{subsubsection}{Problem 2.56.}
\emph{Show that a subset $X$ of a module $M$ generates $M$ if and only if
the homomorphism $M_X \to M$ is onto.
Every module is isomorphic to a quotient of a free module.} \\



\emph{Proof.}
\begin{enumerate}
\item[(1)]
  If $X$ generates $M$, then for any $m \in M$
  we can write
  \[
    m = \sum_{x \in X} a_{x} x
  \]
  as a finite sum where $a_{x} \in R$ and $x \in X \subseteq M$.
  Define $\varphi_{x} \in M_X$ by $\varphi_{x}(y) = \delta_{xy}$ where
  $\delta_{xy}$ is the Kronecker delta.
  Hence, the homomorphism $\alpha: M_X \to M$ maps
  the finite sum $\varphi := \sum_{x \in X} a_x \varphi_{x}$ to
  $\sum_{x \in X} a_{x} x = m$.

\item[(2)]
  Conversely, if the homomorphism $\alpha: M_X \to M$ is onto,
  then for any $m \in M$ there is a finite sum $\varphi := \sum_{x \in X} a_x \varphi_{x}$
  such that $\alpha(\varphi) = m$.
  Hence,
  \[
    m
    = \alpha(\varphi)
    = \alpha\left( \sum_{x \in X} a_x \varphi_{x} \right)
    = \sum_{x \in X} a_x x
  \]
  is generated by $X$.

\item[(3)]
  Let
  \[
    F = \bigoplus_{m \in M} R
  \]
  be a free module.
  Define a map $\varphi: F \to M$ by
  \[
    \varphi:
    (0, \ldots, 0, \underbrace{1}_{\text{$m$th position}}, 0, \ldots 0)
    \mapsto m.
  \]
  $\varphi$ is well-defined.
  $\varphi$ is a module homomorphism.
  $\varphi$ is surjective.
  Hence
  \[
    M \cong F/ker(\varphi)
  \]
  is isomorphic to a quotient of a free module.
\end{enumerate}
$\Box$ \\\\



%%%%%%%%%%%%%%%%%%%%%%%%%%%%%%%%%%%%%%%%%%%%%%%%%%%%%%%%%%%%%%%%%%%%%%%%%%%%%%%%
%%%%%%%%%%%%%%%%%%%%%%%%%%%%%%%%%%%%%%%%%%%%%%%%%%%%%%%%%%%%%%%%%%%%%%%%%%%%%%%%
%%%%%%%%%%%%%%%%%%%%%%%%%%%%%%%%%%%%%%%%%%%%%%%%%%%%%%%%%%%%%%%%%%%%%%%%%%%%%%%%
%%%%%%%%%%%%%%%%%%%%%%%%%%%%%%%%%%%%%%%%%%%%%%%%%%%%%%%%%%%%%%%%%%%%%%%%%%%%%%%%



\newpage
\section*{Chapter 3: Local Properties of Plane Curves \\}
\addcontentsline{toc}{section}{Chapter 3: Local Properties of Plane Curves}



\subsection*{3.1. Multiple Points and Tangent Lines \\}
\addcontentsline{toc}{subsection}{3.1. Multiple Points and Tangent Lines}



\subsubsection*{Problem 3.1.}
\addcontentsline{toc}{subsubsection}{Problem 3.1.}
\emph{Prove that in the above examples $P = (0,0)$
is the only multiple point on the curves
$c = y^2 - x^3$,
$d = y^2 - x^3 - x^2$,
$e = (x^2+y^2)^2 + 3x^2y - y^3$, and
$f = (x^2+y^2)^3 - 4x^2y^2$.} \\



\emph{Proof.}
\begin{enumerate}
\item[(1)]
  \begin{align*}
    \frac{\partial c}{\partial x} &= -3x^2 = 0 \\
    \frac{\partial c}{\partial y} &= 2y = 0
  \end{align*}
  implies that $(x,y) = (0,0)$.
  Note that
  $c(0,0) = \frac{\partial f}{\partial c}(0,0) = \frac{\partial c}{\partial y}(0,0) = 0$.
  So $(x,y) = (0,0)$ is the only multiple point on $c$.

\item[(2)]
  \begin{align*}
    \frac{\partial d}{\partial x} &= -3x^2 - 2x = 0 \\
    \frac{\partial d}{\partial y} &= 2y = 0
  \end{align*}
  implies that $(x,y) = (0,0) \in d$ is the only multiple point on $d$.
  (Note that $(x,y) = \left( -\frac{2}{3},0 \right) \not\in d$.)

\item[(3)]
  \begin{align*}
    \frac{\partial e}{\partial x} &= 4x(x^2+y^2) + 6xy = 0 \\
    \frac{\partial e}{\partial y} &= 4y(x^2+y^2) + 3x^2 - 3y^2 = 0
  \end{align*}
  implies that $x = 0$ or $4(x^2+y^2) + 6y = 0$.
  \begin{enumerate}
  \item[(a)]
    $x = 0$ implies that $(x,y) = (0,0)$ or $(0,1)$.
    Note that $(0,1)$ is a simple point (since $\frac{\partial e}{\partial y}(0,1) = 1$).
  \item[(b)]
    $4(x^2+y^2) + 6y = 0$ implies that $x^2+y^2 = -\frac{3y}{2}$ and thus
    \begin{align*}
      0
      &= 4y(x^2+y^2) + 3x^2 - 3y^2 \\
      &= 4y\left( -\frac{3y}{2} \right) + 3x^2 - 3y^2 \\
      &= 3(x^2 - 3y^2).
    \end{align*}
    $(x,y) \in e$ implies that
    \begin{align*}
      0
      &= (x^2+y^2)^2 + 3x^2y - y^3 \\
      &= (3y^2+y^2)^2 + 3(3y^2)y - y^3 \\
      &= 8y^3(2y+1).
    \end{align*}
    So $(x,y) = (0,0)$ or $\left( \pm\frac{\sqrt{3}}{2}, -\frac{1}{2} \right)$.
    Note that $\frac{\partial e}{\partial x}\left( \pm\frac{\sqrt{3}}{2}, -\frac{1}{2} \right) \neq 0$.
  \end{enumerate}
  Therefore, $(x,y) = (0,0)$ is the only multiple point on $e$.

\item[(4)]
  \begin{align*}
    \frac{\partial f}{\partial x} &= 6x(x^2+y^2)^2 - 8xy^2 = x(6(x^2+y^2)^2 - 8y^2) = 0 \\
    \frac{\partial f}{\partial y} &= 6y(x^2+y^2)^2 - 8x^2y = y(6(x^2+y^2)^2 - 8x^2) = 0
  \end{align*}
  implies that $(x,y) = (0,0)$
  or $6(x^2+y^2)^2 = 8x^2 = 8y^2$.
  $6(x^2+y^2)^2 = 8x^2 = 8y^2$ implies that $x^2 = y^2$.
  So $(x,y) \in f$ implies that $x^2 = y^2 = \frac{1}{2}$,
  contrary that $6 = 6(x^2+y^2)^2 \neq 8x^2 = 4$.
  Therefore, $(x,y) = (0,0)$ is the only multiple point on $f$.
\end{enumerate}
$\Box$ \\\\



%%%%%%%%%%%%%%%%%%%%%%%%%%%%%%%%%%%%%%%%%%%%%%%%%%%%%%%%%%%%%%%%%%%%%%%%%%%%%%%%



\subsubsection*{Problem 3.2.}
\addcontentsline{toc}{subsubsection}{Problem 3.2.}
\emph{Find the multiple points, and the tangent lines at the multiple points,
for each of the following curves:}
\begin{enumerate}
\item[(a)]
  \emph{$y^3 - y^2 + x^3 - x^2 + 3xy^2 + 3x^2y + 2xy$.}

\item[(b)]
  \emph{$x^4 + y^4 - x^2y^2$.}

\item[(c)]
  \emph{$x^3 + y^3 - 3x^2 - 3y^2 + 3xy + 1$.}

\item[(d)]
  \emph{$y^2 + (x^2 - 5)(4x^4 - 20x^2 + 25)$.}
\end{enumerate}
\emph{Sketch the part of the curve in (d)
that is contained in $\mathbf{A}^{2}(\mathbb{R}) \subseteq \mathbf{A}^{2}(\mathbb{C})$.} \\



\emph{Proof of (a).}
\begin{enumerate}
\item[(1)]
  Let $f = y^3 - y^2 + x^3 - x^2 + 3xy^2 + 3x^2y + 2xy \in k[x,y]$.
  So
  \begin{align*}
    \frac{\partial f}{\partial x} &= 3x^2 + 6xy + 3y^2 - 2x + 2y = 0 \\
    \frac{\partial f}{\partial y} &= 3x^2 + 6xy + 3y^2 + 2x - 2y = 0
  \end{align*}
  implies that
  \begin{align*}
    6(x+y)^2 &= 0 \\
    -4(x - y) &= 0
  \end{align*}
  Note that
  $f(0,0) = \frac{\partial f}{\partial x}(0,0) = \frac{\partial f}{\partial y}(0,0) = 0$.
  Hence, $(x,y) = (0,0)$ is the only multiple point on $f$.

\item[(2)]
  Write $f = (y^3 + 3xy^2 + 3x^2y + x^3) + (-x^2 + 2xy - y^2)$.
  The tangent lines at $(x,y) = (0,0)$ is the linear factors of
  $-x^2 + 2xy - y^2 = -(x-y)^2$.
  Hence, the line $x - y = 0$ is the only tangent line at $(x,y) = (0,0)$
  of the multiplicity $= 2$.
\end{enumerate}
$\Box$ \\



\emph{Proof of (b).}
\begin{enumerate}
\item[(1)]
  Let $f = x^4 + y^4 - x^2y^2 \in k[x,y]$.
  So
  \begin{align*}
    \frac{\partial f}{\partial x} &= 4x^3 - 2xy^2 = 0 \\
    \frac{\partial f}{\partial y} &= 4y^3 - 2x^2y = 0
  \end{align*}
  implies that $(x,y) = (0,0)$.
  Note that
  $f(0,0) = \frac{\partial f}{\partial x}(0,0) = \frac{\partial f}{\partial y}(0,0) = 0$.
  Hence, $(x,y) = (0,0)$ is the only multiple point on $f$.

\item[(2)]
  The tangent lines at $(x,y) = (0,0)$ is the linear factors of
  $x^4 + y^4 - x^2y^2$.
  Hence, there are four distinct tangent lines
  \[
    x \pm \sqrt{\frac{1 \pm \sqrt{-3}}{2}} y
  \]
  at $(x,y) = (0,0)$.
  Each tangent line is simple.
\end{enumerate}
$\Box$ \\



\emph{Proof of (c).}
\begin{enumerate}
\item[(1)]
  Let $f = x^3 + y^3 - 3x^2 - 3y^2 + 3xy + 1 \in k[x,y]$.
  So
  \begin{align*}
    \frac{\partial f}{\partial x} &= 3(x^2 - 2x + y) = 0 \\
    \frac{\partial f}{\partial y} &= 3(y^2 - 2y + x) = 0
  \end{align*}
  implies that $(x-y)(x+y-3) = 0$.

\item[(2)]
  The case $x-y = 0$. Take $x = y$ in $3x^2 - 6x + 3y = 0$ to get
  $(x,y) = (1,1), (0,0)$.
  $(x,y) = (1,1)$ is a multiple point on $f$ since
  $f(1,1) = \frac{\partial f}{\partial x}(1,1) = \frac{\partial f}{\partial y}(1,1) = 0$.
  $(x,y) = (0,0)$ is impossible since $f(0,0) = 1 \neq 0$.

\item[(3)]
  The case $x+y-3 = 0$. Take $x = -y + 3$ in $f$ to get $1 = 0$, which is absurd.

\item[(4)]
  By (2)(3), the only multiple point on $f$ is $(x,y) = (1,1)$.

\item[(5)]
  Let $t(x,y) = (x+1,y+1)$.
  Then
  \[
    f^{t} = f(x+1,y+1) = x^3 + y^3 + 3xy.
  \]
  The tangent lines at $(x,y) = (1,1)$ is the linear factors of
  $x^3 + y^3 + 3xy$.
  Hence, there are two distinct simple tangent lines $x$ and $y$ at $(x,y) = (1,1)$.
  \end{enumerate}
$\Box$ \\



\emph{Proof of (d).}
\begin{enumerate}
\item[(1)]
  Let $f = y^2 + (x^2 - 5)(4x^4 - 20x^2 + 25) \in k[x,y]$.
  So
  \begin{align*}
    \frac{\partial f}{\partial x} &= 2x(2x^2-5)(6x^2-25) = 0 \\
    \frac{\partial f}{\partial y} &= 2y = 0
  \end{align*}
  implies that there are only two multiple points
  \[
    (x,y) = \left( \pm\sqrt{\frac{5}{2}}, 0 \right)
  \]
  on $f$.

\item[(2)]
  Let $t(x,y) = \left( x + \sqrt{\frac{5}{2}}, y \right)$.
  Then
  \begin{align*}
    f^{t}
    &= f\left( x + \sqrt{\frac{5}{2}}, y \right) \\
    &= 4 x^6 + 12\sqrt{10} x^5 + 110 x^4 + 20\sqrt{10} x^3 - 100 x^2 + y^2.
  \end{align*}
  The tangent lines at $(x,y) = \left( \sqrt{\frac{5}{2}}, 0 \right)$ is the linear factors of
  $-100 x^2 + y^2 = -(10x+y)(10x-y)$.
  Hence, there are two distinct simple tangent lines
  \[
    10x \pm y
  \]
  at $(x,y) = \left( \sqrt{\frac{5}{2}}, 0 \right)$.

\item[(3)]
  Similarly, there are also two distinct simple tangent lines
  \[
    10x \pm y
  \]
  at $(x,y) = \left( -\sqrt{\frac{5}{2}}, 0 \right)$.
  \end{enumerate}
$\Box$ \\\\



%%%%%%%%%%%%%%%%%%%%%%%%%%%%%%%%%%%%%%%%%%%%%%%%%%%%%%%%%%%%%%%%%%%%%%%%%%%%%%%%



\subsubsection*{Problem 3.3.}
\addcontentsline{toc}{subsubsection}{Problem 3.3.}
\emph{If a curve $f$ of degree $n$ has a point $P$ of multiplicity $n$,
show that $f$ consists of $n$ lines through $P$ (not necessarily distinct).} \\



\emph{Proof.}
\begin{enumerate}
\item[(1)]
  Might assume that $P = (0,0)$.
  (Note that any translation of $f$ preserves the degree of $f$.)

\item[(2)]
  Write
  \[
    f = f_m + f_{m+1} + \cdots + f_n
  \]
  where $f_i$ is a form in $k[x,y]$.
  Since $m$ is the multiplicity of $f$ at $P$, $m = n$.
  Hence, $f$ is a form in two variables of degree $n$,
  or $f$ consists of $n$ lines through $P$.
\end{enumerate}
$\Box$ \\\\



%%%%%%%%%%%%%%%%%%%%%%%%%%%%%%%%%%%%%%%%%%%%%%%%%%%%%%%%%%%%%%%%%%%%%%%%%%%%%%%%



\subsubsection*{Problem 3.4.}
\addcontentsline{toc}{subsubsection}{Problem 3.4.}
\emph{Let $P$ be a double point on a curve $f$.
Show that $P$ is a node if and only if}
\[
  \frac{\partial^2 f}{\partial x \partial y}(P)^2
  \neq
  \frac{\partial^2 f}{\partial x^2}(P)
  \cdot
  \frac{\partial^2 f}{\partial y^2}(P).
\]



\emph{Proof.}
\begin{enumerate}
\item[(1)]
  Might assume that $P = (0,0)$ is a double point on $f$.
  Write
  \[
    f = f_2 + f_3 + \cdots + f_n \in k[x,y]
  \]
  (where $f_i$ is a form in $k[x,y]$),
  and
  \[
    f_2 = ax^2 + bxy + cy^2 \in k[x,y].
  \]

\item[(2)]
  $P$ is a node if and only if the discriminant
  \[
    b^2 - 4ac \neq 0.
  \]
  Note that
  \begin{align*}
    \frac{\partial^2 f}{\partial x \partial y}(P)
    &= b, \\
    \frac{\partial^2 f}{\partial x^2}(P)
    &= 2a, \\
    \frac{\partial^2 f}{\partial y^2}(P)
    &= 2c.
  \end{align*}
  Hence,
  $P$ is a node if and only if
  \[
    b^2 - 4ac
    = b^2 - (2a)(2c)
    = \frac{\partial^2 f}{\partial x \partial y}(P)^2
      - \frac{\partial^2 f}{\partial x^2}(P) \cdot \frac{\partial^2 f}{\partial y^2}(P)
    \neq 0.
  \]
\end{enumerate}
$\Box$ \\\\



%%%%%%%%%%%%%%%%%%%%%%%%%%%%%%%%%%%%%%%%%%%%%%%%%%%%%%%%%%%%%%%%%%%%%%%%%%%%%%%%



\subsubsection*{Problem 3.5.}
\addcontentsline{toc}{subsubsection}{Problem 3.5.}
\emph{($\mathrm{char}(k) = 0$)
Show that $m_P(f)$ is the smallest integer $m$ such that
for some $i + j = m$,
\[
  \frac{\partial^{m} f}{\partial x^i \partial y^j}(P) \neq 0.
\]
Find an explicit description for the leading form for $f$ at $P$ in
terms of these derivatives.} \\



\emph{Proof.}
\begin{enumerate}
\item[(1)]
  Might assume that $P = (0,0)$.
  Write $f = f_0 + f_1 + \cdots + f_n$ where $f_i$ is a form in $k[x,y]$.
  Consider any form $f_m$ of $f$.
  Write
  \[
    f_m = c_{m} x^m + c_{m-1} x^{m-1} y + \cdots + c_{0} y^m
  \]
  where $c_i \in k$ and not all $c_i$ are zero.

\item[(2)]
  Similar to Problem 3.4,
  $\frac{\partial^{m} f}{\partial x^i \partial y^j}(P) = c_i i!j!$.
  Here $i+j = m$.
  Hence,
  \[
    c_i
    = \frac{1}{i!j!} \frac{\partial^{m} f}{\partial x^i \partial y^j}(P)
    = \frac{1}{m!} {m \choose i} \frac{\partial^{m} f}{\partial x^i \partial y^j}(P)
  \]
  and thus
  \[
    f_m
    = \frac{1}{m!} \sum_{i = 0}^{m}
    {m \choose i} \frac{\partial^{m} f}{\partial x^i \partial y^j}(P) x^i y^j.
  \]

\item[(3)]
  Suppose $m$ is the smallest integer such that
  for some $i + j = m$,
  \[
    \frac{\partial^{m} f}{\partial x^i \partial y^j}(P) \neq 0.
  \]
  Then $f_0 = f_1 = \cdots = f_{m-1} = 0$ and $f_m \neq 0$ in $k[x,y]$ by (2).
  Therefore, $m = m_P(f)$.
  The explicit description for the leading form $f_m$ for $f$ at $P$
  is already stated in (2).
\end{enumerate}
$\Box$ \\\\



%%%%%%%%%%%%%%%%%%%%%%%%%%%%%%%%%%%%%%%%%%%%%%%%%%%%%%%%%%%%%%%%%%%%%%%%%%%%%%%%



\subsubsection*{Problem 3.6.}
\addcontentsline{toc}{subsubsection}{Problem 3.6.}
\emph{Irreducible curves with given tangent lines $\ell_i$ of multiplicity $r_i$
may be constructed as follows: if $\sum r_i = m$, let
$f = \prod \ell_i^{r_i} + f_{m+1}$,
where $f_{m+1}$ is chosen to make $f$ irreducible (see Problem 2.34).} \\



\emph{Proof.}
\begin{enumerate}
\item[(1)]
  Let $f_m = \prod \ell_i^{r_i} \in k[x,y]$.
  Problem 1.4 implies that there exists a point $P = (a,b)$ such that
  $f_m(P) \neq 0$ since $f_m \neq 0 \in k[x,y]$.

\item[(2)]
  Let $\ell: bx - ay = 0$ and $f_{m+1} = \ell^{m+1}$.
  Since $(a,b) \neq (0,0)$, $\deg(\ell) = 1$.
  Also, $\ell_i \nmid \ell$ by (1).
  Hence, $f_m$ and $f_{m+1}$ have no common factors.
  By Problem 2.34, $f = f_{m} + f_{m+1}$ is irreducible.
\end{enumerate}
$\Box$ \\\\



%%%%%%%%%%%%%%%%%%%%%%%%%%%%%%%%%%%%%%%%%%%%%%%%%%%%%%%%%%%%%%%%%%%%%%%%%%%%%%%%



\subsubsection*{Problem 3.7.}
\addcontentsline{toc}{subsubsection}{Problem 3.7.}
\begin{enumerate}
\item[(a)]
  \emph{Show that the real part of the curve $e$ of the examples is
  the set of points in $\mathbf{A}^2(\mathbb{R})$ whose
  polar coordinates $(r,\theta)$ satisfy the equation $r = -\sin(3\theta)$.
  Find the polar equation for the curve $f$.}

\item[(b)]
  \emph{If $n$ is an odd integer $\geq 1$,
  show that the equation $r = \sin(n\theta)$ defines
  the real part of a curve of degree $n+1$ with an ordinary $n$-tuple point at $(0,0)$.
  (Use the fact that $\sin(n\theta) = \mathrm{im}(e^{in\theta})$ to get the equation;
  note that rotation by $\frac{\pi}{n}$ is a linear transformation that takes the curve into itself.)}

\item[(c)]
  \emph{Analyze the singularities that arise by looking at
  $r^2 = \sin^2(n\theta)$, $n$ even.}

\item[(d)]
  \emph{Show that the curves constructed in (b) and (c) are all irreducible in $\mathbf{A}^2(\mathbb{C})$.
  (Hint: Make the polynomials homogeneous with respect to a variable $z$, and use \S 2.6.)} \\
\end{enumerate}



\emph{Proof of (a).}
\begin{enumerate}
\item[(1)]
  De Moivre's theorem implies that
  \begin{align*}
    \cos(n\theta)+i \sin(n\theta)
    =& \: (\cos\theta + i\sin\theta)^n \\
    =& \: \sum_{k=0}^{n} {n \choose k} (\cos\theta)^{n-k} i^{k} (\sin\theta)^{k} \\
    =& \: \sum_{k=0}^{\lfloor \frac{n}{2} \rfloor}
      (-1)^k {n \choose 2k} (\cos\theta)^{n-2k} (\sin\theta)^{2k} \\
      &+ i \sum_{k=0}^{\lfloor \frac{n-1}{2} \rfloor}
      (-1)^k {n \choose 2k+1} (\cos\theta)^{n-2k-1} (\sin\theta)^{2k+1}.
  \end{align*}
  Hence,
  \[
    \sin(n\theta)
    = \sum_{k=0}^{\lfloor \frac{n-1}{2} \rfloor}
      (-1)^k {n \choose 2k+1} (\cos\theta)^{n-2k-1} (\sin\theta)^{2k+1}.
  \]
  In particular,
  \[
    \sin(3\theta)
    = 3 \cos^{2}\theta \sin\theta - \sin^{3}\theta.
  \]

\item[(2)]
  $r = -\sin(3\theta)$ implies that
  \begin{align*}
    r^4
    &= r^3(-\sin(3\theta)) \\
    &= r^3 (-3 \cos^{2}\theta \sin\theta + \sin^{3}\theta) \\
    &= -3 (r\cos\theta)^2 (r\sin\theta) + (r\sin\theta)^3.
  \end{align*}
  Hence, $r = -\sin(3\theta)$ implies that
  \[
    (x^2+y^2)^2 = -3x^2 y + y^3
  \]
  in $\mathbf{A}^2(\mathbb{R})$.

\item[(3)]
  As $f(x,y) = (x^2+y^2)^3 - 4x^2y^2$,
  $f(r\cos\theta,r\sin\theta) = 0$ implies that
  \[
    r^6
    = 4r^4 \cos^2\theta\sin^2\theta
    = r^4 \sin^2 2\theta
  \]
  or
  \[
    r^2 = \sin^2 2\theta.
  \]
\end{enumerate}
$\Box$ \\



\emph{Proof of (b).}
\begin{enumerate}
\item[(1)]
  By (a), $r = \sin(n\theta)$ with odd $n \geq 1$
  implies that
  \begin{align*}
    & \: r
    = \sin(n\theta)
    = \sum_{k=0}^{\frac{n-1}{2}}
      (-1)^k {n \choose 2k+1} (\cos\theta)^{n-2k-1} (\sin\theta)^{2k+1} \\
    \Longrightarrow& \:
    r^{n+1}
    = \sum_{k=0}^{\frac{n-1}{2}}
      (-1)^k {n \choose 2k+1} (r\cos\theta)^{n-2k-1} (r\sin\theta)^{2k+1} \\
    \Longrightarrow& \:
    (x^2+y^2)^{\frac{n+1}{2}}
    = \sum_{k=0}^{\frac{n-1}{2}}
      (-1)^k {n \choose 2k+1} x^{n-2k-1} y^{2k+1}.
  \end{align*}
  Hence, $r = \sin(n\theta)$ defines the real part of a curve
  \[
    \alpha: (x^2+y^2)^{\frac{n+1}{2}}
    - \sum_{k=0}^{\frac{n-1}{2}}
      (-1)^k {n \choose 2k+1} x^{n-2k-1} y^{2k+1}
  \]
  of degree $n+1$.

\item[(2)]
  Note that $(0,0) \in \alpha$.
  Since
  \begin{align*}
    \frac{\partial \alpha}{\partial x}
    =& \:
    (n+1)(x^2+y^2)^{\frac{n-1}{2}}x \\
    &- \sum_{k=0}^{\frac{n-3}{2}}
      (-1)^k (n-2k-1) {n \choose 2k+1} x^{n-2k-2} y^{2k+1} \\
    \frac{\partial \alpha}{\partial y}
    =& \:
    (n+1)(x^2+y^2)^{\frac{n-1}{2}}y \\
    &- \sum_{k=0}^{\frac{n-1}{2}}
      (-1)^k (2k+1) {n \choose 2k+1} x^{n-2k-1} y^{2k},
  \end{align*}
  $\frac{\partial \alpha}{\partial x}(0,0) = \frac{\partial \alpha}{\partial y}(0,0) = 0$.
  $(0,0)$ is a multiple point.

\item[(3)]
  The tangents at $(0,0)$ are the linear factors of
  \[
    \sum_{k=0}^{\frac{n-1}{2}} (-1)^k {n \choose 2k+1} x^{n-2k-1} y^{2k+1}.
  \]
  Clearly, $y$ is a tangent line at $(0,0)$.
  Note that rotation by $\frac{\pi}{n}$ is a linear transformation that takes the curve into itself.
  Hence, all tangents at $(0,0)$ are
  \[
    \ell_k: \sin\frac{k\pi}{n} x - \cos\frac{k\pi}{n} y
  \]
  for $k = 0, 1, \ldots, n-1$.
  All $\ell_k$ are pairwise distinct, and thus $(0,0)$ is an ordinary $n$-tuple point.
\end{enumerate}
$\Box$ \\



\emph{Proof of (c).}
\begin{enumerate}
\item[(1)]
  Similar to (b),
  $r^2 = \sin^2(n\theta)$ defines the real part of a curve
  \[
    \beta: (x^2+y^2)^{n+1}
    - \left( \sum_{k=0}^{\frac{n-2}{2}} (-1)^k {n \choose 2k+1} x^{n-2k-1} y^{2k+1} \right)^2
  \]
  of degree $2n+2$.

\item[(2)]
  Note that
  \[
    \beta(0,0)
    = \frac{\partial \beta}{\partial x}(0,0)
    = \frac{\partial \beta}{\partial y}(0,0)
    = 0.
  \]
  Hence, $(0,0)$ is a multiple point.

\item[(3)]
  Similar to (b), all tangents at $(0,0)$ are
  \[
    \ell_k: \sin\frac{k\pi}{n} x - \cos\frac{k\pi}{n} y
  \]
  of multiplicity $= 2$ for $k = 0, 1, \ldots, n-1$.
\end{enumerate}
$\Box$ \\



\emph{Proof of (d).}
\begin{enumerate}
\item[(1)]
  The case $n$ is odd.
  \begin{enumerate}
  \item[(a)]
    Consider
    \[
      \alpha: (x^2+y^2)^{\frac{n+1}{2}}
      - \sum_{k=0}^{\frac{n-1}{2}}
        (-1)^k {n \choose 2k+1} x^{n-2k-1} y^{2k+1}.
    \]

  \item[(b)]
    Since
    \begin{align*}
      \alpha_{n+1}
      :=& \: (x^2+y^2)^{\frac{n+1}{2}} \\
      =& \: (x + iy)^{\frac{n+1}{2}} (x - iy)^{\frac{n+1}{2}} \\
      \alpha_{n}
      =& \: \sum_{k=0}^{\frac{n-1}{2}} (-1)^k {n \choose 2k+1} x^{n-2k-1} y^{2k+1} \\
      =& \: c \prod_{k=0}^{n-1} \left( \sin\frac{k\pi}{n} x - \cos\frac{k\pi}{n} y \right)
    \end{align*}
    (with some $c \in \mathbb{C}$),
    $\alpha_{n+1}$ and $\alpha_{n}$ have no common factors in $\mathbb{C}[x,y]$.
    By Problem 2.34, $\alpha$ is irreducible.
  \end{enumerate}

\item[(2)]
  The case $n$ is even.
  \begin{enumerate}
  \item[(a)]
    Consider
    \[
      \beta: \underbrace{(x^2+y^2)^{n+1}}_{:= \beta_{2n+2}}
      \underbrace{-\left( \sum_{k=0}^{\frac{n-2}{2}}
        (-1)^k {n \choose 2k+1} x^{n-2k-1} y^{2k+1} \right)^2}_{:= \beta_{2n}}.
    \]

  \item[(b)]
    Similar to the proof of Problem 2.34,
    suppose $\beta = \beta_{2n} + \beta_{2n+2} = rs \in \mathbb{C}[x,y]$.
    So
    \[
      (\beta_{2n} + \beta_{2n+2})^{*} = (rs)^{*}
      \Longrightarrow
      z^2 \beta_{2n} + \beta_{2n+2} = r^{*} s^{*}.
    \]
    Note that $\deg_{z}(z^2 \beta_{2n} + \beta_{2n+2}) = 2$.
    So $\deg_{z}(r^{*}) = 0, 1, 2$.

  \item[(c)]
    The case $\deg_{z}(r^{*}) = 0, 2$ is similar to the proof of Problem 2.34
    because $\beta_{2n}$ and $\beta_{2n+2}$ have no common factors in $\mathbb{C}[x,y]$.

  \item[(d)]
    The case $\deg_{z}(r^{*}) = 1$. (So $\deg_{z}(s^{*}) = 1$.)
    Write $r = r_p + r_{p+1}$ and $s = s_q + s_{q+1}$.
    Hence, $\beta = rs$ implies that
    \begin{align*}
      r_p s_q
      &= -\left( \sum_{k=0}^{\frac{n-2}{2}}
        (-1)^k {n \choose 2k+1} x^{n-2k-1} y^{2k+1} \right)^2 \\
      r_p s_{q+1} + r_{p+1}s_q
      &= 0 \\
      r_{p+1} s_{q+1}
      &= (x^2+y^2)^{n+1} = (x+iy)^{n+1}(x-iy)^{n+1}.
    \end{align*}
    Since $n+1$ is odd and
    $x \pm iy \nmid
    -\left( \sum_{k=0}^{\frac{n-2}{2}} (-1)^k {n \choose 2k+1} x^{n-2k-1} y^{2k+1} \right)^2$,
    $r_p s_{q+1} + r_{p+1}s_q = 0$ implies that $r_p = s_q = 0$, which is absurd.

  \item[(e)]
    By (b)(c)(d), $\beta$ is irreducible over $\mathbb{C}$.
  \end{enumerate}
\end{enumerate}
$\Box$ \\\\



%%%%%%%%%%%%%%%%%%%%%%%%%%%%%%%%%%%%%%%%%%%%%%%%%%%%%%%%%%%%%%%%%%%%%%%%%%%%%%%%



\subsubsection*{Problem 3.8.}
\addcontentsline{toc}{subsubsection}{Problem 3.8.}
\emph{Let $t: \mathbf{A}^2 \to \mathbf{A}^2$ be a polynomial map, $t(Q) = P$.}
\begin{enumerate}
\item[(a)]
  \emph{Show that $m_Q(f^t) \geq m_P(f)$.}

\item[(b)]
  \emph{Let $t = (t_1, t_2)$, and define
  \[
    J_Qt = \left( \frac{\partial t_i}{\partial x_j}(Q) \right)
  \]
  to be the \textbf{Jacobian matrix} of $t$ at $Q$.
  Show that $m_Q(f^t) = m_P(f)$ if $J_Qt$ is invertible.}

\item[(c)]
  \emph{Show that the converse of (b) is false:
  let $t = (x^2, y)$, $f = y - x^2$, $P = Q = (0, 0)$.} \\
\end{enumerate}



\emph{Proof of (a).}
\begin{enumerate}
\item[(1)]
  Might assume that $P = Q = (0, 0)$.
  Write $t = (t_1, t_2)$ and thus $(0,0) = t(0,0) = (t_1(0,0), t_2(0,0))$.
  So there is no nonzero constant term in $t_i$ ($i = 1, 2$).

\item[(2)]
  Might assume that $f \neq 0$ (since there is nothing to prove when $f = 0$).
  Write $f = f_m + f_{m+1} + \cdots + f_n$ where $f_i$ is a form in $k[x,y]$
  and $f_m \neq 0$.
  So,
  \[
    f(t_1(x,y),t_2(x,y))
    = f_m(t_1(x,y),t_2(x,y)) + \cdots + f_n(t_1(x,y),t_2(x,y))
  \]
  has the multiplicity $= \infty$ or $\geq m = m_P(f)$.
  In any case, $m_Q(f^t) \geq m_P(f)$.

\end{enumerate}
$\Box$ \\



\emph{Proof of (b).}
\begin{enumerate}
\item[(1)]
  Might assume that $P = Q = (0, 0)$.
  Since $J_Qt$ is invertible,
  \begin{align*}
    \begin{pmatrix}
      \frac{\partial t_1}{\partial x}(Q) & \frac{\partial t_1}{\partial y}(Q)
    \end{pmatrix}
    &\neq 0, \\
    \begin{pmatrix}
      \frac{\partial t_2}{\partial x}(Q) & \frac{\partial t_2}{\partial y}(Q)
    \end{pmatrix}
    &\neq 0
  \end{align*}
  (as vectors).
  Hence $t_1$ (resp. $t_2$) has the multiplicity $= 1$.

\item[(2)]
  Define
  \[
    s = (s_1, s_2): \mathbf{A}^2 \to \mathbf{A}^2
  \] be a polynomial map such that $s_i$ is the linear term of $t_i$.
  Note that $J_Q{s} = J_Q{t}$ is invertible and $m_Q(f^s) = m_Q(f^t)$ for any $f$.

\item[(3)]
  \emph{Show that $m_Q(f^s) = m_Q(f^t)$ for any $f \in k[x,y]$.}
  Might assume that $f \neq 0$ (since there is nothing to prove when $f = 0$).
  Write $f = f_m + f_{m+1} + \cdots + f_n$ where $f_i$ is a form in $k[x,y]$
  and $f_m \neq 0$.
  Since
  \[
    m_{Q}(f_i^t) = m_{Q}(f_i^s) = i \text{ or } \infty,
  \]
  $m_Q(f^s) = m_Q(f^t)$.

\item[(4)]
  Since $J_Q{s}$ is invertible,
  $s^{-1}$ is also a polynomial map with an invertible Jacobian matrix $J_Q{s^{-1}}$.
  By (a),
  \[
    m_Q(f^s) \geq m_P(f) = m_P((f^s)^{s^{-1}}) = m_Q(f^s)
  \]
  or $m_Q(f^s) = m_P(f)$.
  Therefore, $m_Q(f^t) = m_Q(f^s) = m_P(f)$.
\end{enumerate}
$\Box$ \\



\emph{Proof of (c).}
  $m_P(f) = 1$ and $m_Q(f^t) = 1$ since $f^t = y - x^4$.
  However,
  \[
    J_Q{t}
    =
    \begin{pmatrix}
      0 & 0 \\
      0 & 1
    \end{pmatrix}
  \]
  is not invertible.
$\Box$ \\\\



%%%%%%%%%%%%%%%%%%%%%%%%%%%%%%%%%%%%%%%%%%%%%%%%%%%%%%%%%%%%%%%%%%%%%%%%%%%%%%%%



\subsubsection*{Problem 3.9.}
\addcontentsline{toc}{subsubsection}{Problem 3.9.}
\emph{Let $f \in k[x_1,\ldots,x_n]$ define a hypersurface $V(f) \subseteq \mathbf{A}^{n}$.
Let $P \in \mathbf{A}^{n}$.}
\begin{enumerate}
\item[(a)]
  \emph{Define the multiplicity $m_P(f)$ of $f$ at $P$.}

\item[(b)]
  \emph{If $m_P(f) = 1$, define the tangent hyperplane to $f$ at $P$.}

\item[(c)]
  \emph{Examine $f = x^2 + y^2 - z^2$, $P = (0,0,0)$.
  Is it possible to define tangent hyperplanes at multiple points?} \\
\end{enumerate}



\emph{Proof of (a).}
\begin{enumerate}
\item[(1)]
  Let $P = (0,\ldots,0)$.
  Write $f = f_m + f_{m+1} + \cdots + f_n$,
  where $f_i$ is a form in $k[x_1,\ldots,x_n]$ of degree $i$, $f_m \neq 0$.
  We define $m$ to be the multiplicity of $f$ at $P = (0,\ldots,0)$, write $m = m_P(f)$.

\item[(2)]
  To extend these definitions to a point $P = (a_1,\ldots,a_n) \neq (0,\ldots,0)$,
  let $t$ be the translation that takes $(0,\ldots,0)$ to $P$, i.e.,
  \[
    t(x_1,\ldots,x_n) = (x_1+a_1,\ldots,x_n+a_n).
  \]
  Then
  \[
    f^{t} = f(x_1+a_1,\ldots,x_n+a_n).
  \]
  Define $m_P(f)$ to be $m_{(0,\ldots,0)}(f^t)$, i.e., write
  $f^t = g_m + g_{m+1} + \cdots$, $g_i$ forms, $g_m \neq 0$,
  and let $m = m_P(f)$.
\end{enumerate}
$\Box$ \\



\emph{Proof of (b).}
\begin{enumerate}
\item[(1)]
  Let $P = (0,\ldots,0)$.
  Write $f = f_m + f_{m+1} + \cdots + f_n$,
  where $f_i$ is a form in $k[x_1,\ldots,x_n]$ of degree $i$, $f_m \neq 0$.
  If $m = 1$, then $f_m = f_1$ is a hyperplane.
  Hence, we can define the tangent hyperplane to $f$ at $P$ to be $f_1$.

\item[(2)]
  Similar to (a),
  the tangent hyperplane to $f$ at $P = (a_1,\ldots,a_n) \neq (0,\ldots,0)$ is $g_1$
  where
  \[
    f^t = g_1 + g_2 + \cdots
  \]
  and $t(x_1,\ldots,x_n) = (x_1+a_1,\ldots,x_n+a_n)$.
\end{enumerate}
$\Box$ \\



\emph{Proof of (c).}
\begin{enumerate}
\item[(1)]
  No.

\item[(2)]
  \emph{Show that $x^2 + y^2 - z^2$ is irreducible over $k$.}
  (Reductio ad absurdum)
  Suppose $x^2 + y^2 - z^2$ were reducible.
  By Problem 1.1, we can write
  \[
    x^2 + y^2 - z^2 = (a_1 x + a_2 y + a_3 z)(b_1 x + b_2 y + b_3 z)
  \]
  for some $a_i, b_i \in k$ ($i = 1, 2, 3$).
  Expanding out the right hand side and comparing coefficients to get
  \begin{align*}
    & a_1 b_1 = a_2 b_2 = -a_3 b_3 = 1 \\
    & a_1 b_2 + a_2 b_1 = a_2 b_3 + a_3 b_2 = a_3 b_1 + a_1 b_3 = 0.
  \end{align*}
  So $a_i, b_i \neq 0$ for all $i$ and
  \[
    b_1
    = \frac{-a_1}{a_3} b_3
    = \frac{-a_1}{a_3} \cdot \frac{-a_3}{a_2} b_2
    = \frac{-a_1}{a_3} \cdot \frac{-a_3}{a_2} \cdot \frac{-a_2}{a_1} b_1 = -b_1.
  \]
  Hence, $b_1 = 0$, which is absurd.

\item[(3)]
  Since $x^2 + y^2 - z^2$ is irreducible over any field $k$ with $\mathrm{char}(k) = 0$,
  it is impossible to define tangent hyperplanes at $(0,0,0)$.
\end{enumerate}
$\Box$ \\\\



%%%%%%%%%%%%%%%%%%%%%%%%%%%%%%%%%%%%%%%%%%%%%%%%%%%%%%%%%%%%%%%%%%%%%%%%%%%%%%%%



\subsubsection*{Problem 3.10.}
\addcontentsline{toc}{subsubsection}{Problem 3.10.}
\emph{Show that an irreducible plane curve has only a finite number of multiple points.
Is this true for hypersurfaces?} \\



\emph{Proof.}
\begin{enumerate}
\item[(1)]
  Let $f \in k[x,y]$ be an irreducible plane curve, and
  \[
    V
    = V\left( f, \frac{\partial f}{\partial x}, \frac{\partial f}{\partial y} \right)
  \]
  be the set of the multiple (singular) points of $f$.
  Moreover, $V$ is an algebraic set.

\item[(2)]
  Suppose $\frac{\partial f}{\partial x} \neq 0$
  (since not all $\frac{\partial f}{\partial x}$ and $\frac{\partial f}{\partial y}$ are zero).
  It is nothing to do if $\frac{\partial f}{\partial x}$ is a nonzero constant.
  Suppose $\deg(\frac{\partial f}{\partial x}) \geq 1$.
  Since $f$ is irreducible and $\deg\left( \frac{\partial f}{\partial x} \right) = \deg f - 1$,
  $f$ and $\frac{\partial f}{\partial x}$ have no common factors.
  By Proposition 2 in \S 1.6,
  $V\left( f, \frac{\partial f}{\partial x} \right)$ is a finite set.
  Hence,
  $V \subseteq V\left( f, \frac{\partial f}{\partial x} \right)$ is finite
  as a subset of a finite set.

\item[(3)]
  The conclusion is not true for hypersurfaces when $n \geq 3$.
  Consider $f = x_2^2 - x_1^3 \in k[x_1,\ldots,x_n]$.
  The set of the multiple (singular) points of $f$ is
  \[
    \{ (0,0,a_3,\ldots,a_n) : a_3,\ldots,a_n \in k \},
  \]
  which is infinite as $k$ is infinite.
\end{enumerate}
$\Box$ \\\\



%%%%%%%%%%%%%%%%%%%%%%%%%%%%%%%%%%%%%%%%%%%%%%%%%%%%%%%%%%%%%%%%%%%%%%%%%%%%%%%%



\subsubsection*{Problem 3.11. (Tangent space)}
\addcontentsline{toc}{subsubsection}{Problem 3.11. (Tangent space)}
\emph{Let $V \subseteq \mathbf{A}^{n}$ be an affine variety, $P \in V$.
The \textbf{tangent space} $T_P(V)$ is defined to be
\[
  \left\{
    (v_1, \ldots, v_n) \in \mathbf{A}^{n} :
    \sum_i \frac{\partial g}{\partial x_i}(P) v_i = 0 \:\: \forall g \in I(V)
  \right\}.
\]
If $V = V(f)$ is a hypersurface, $f$ irreducible, show that
\[
  T_P(V)
  =
  \left\{
    (v_1, \ldots, v_n) \in \mathbf{A}^{n} :
    \sum_i \frac{\partial f}{\partial x_i}(P) v_i = 0
  \right\}.
\]
How does the dimension of $T_P(V)$ relate to the multiplicity of $f$ at $P$?} \\



\emph{Proof.}
\begin{enumerate}
\item[(1)]
  By the Hilbert's Nullstellensatz,
  the irreducibility of $f$ implies that $I(V) = I(V(f)) = (f)$.

\item[(2)]
  Let
  \[
    W
    =
    \left\{
      (v_1, \ldots, v_n) \in \mathbf{A}^{n} :
      \sum_i \frac{\partial f}{\partial x_i}(P) v_i = 0
    \right\}.
  \]
  $W \supseteq T_P(V)$ is true since $f \in I(V) = (f)$.

\item[(3)]
  \emph{Show that $W \subseteq T_P(V)$.}
  Given any $(v_1, \ldots, v_n) \in W$.
  Now for any $g \in I(V) = (f)$, there exists a $h \in k[x_1,\ldots,x_n]$ such that $g = fh$.
  Hence,
  \begin{align*}
    \sum_i \frac{\partial g}{\partial x_i}(P) v_i
    &= \sum_i \frac{\partial (fh)}{\partial x_i}(P) v_i \\
    &= \sum_i \left(
        \frac{\partial (f)}{\partial x_i}(P) h(P)
        + \underbrace{f(P)}_{= 0} \frac{\partial h}{\partial x_i}(P)
      \right)
      v_i \\
    &= h(P) \underbrace{\sum_i \frac{\partial f}{\partial x_i}(P) v_i}_{= 0} \\
    &= 0
  \end{align*}
  implies that $(v_1, \ldots, v_n) \in T_P(V)$.

\item[(4)]
  By definition of $T_P(V)$,
  \begin{equation*}
    \dim_k(T_P(V)) =
    \begin{cases}
      n - 1 & \text{if $m_P(f) = 1$} \\
      n & \text{if $m_P(f) > 1$}. \\
    \end{cases}
  \end{equation*}
\end{enumerate}
$\Box$ \\\\



%%%%%%%%%%%%%%%%%%%%%%%%%%%%%%%%%%%%%%%%%%%%%%%%%%%%%%%%%%%%%%%%%%%%%%%%%%%%%%%%
%%%%%%%%%%%%%%%%%%%%%%%%%%%%%%%%%%%%%%%%%%%%%%%%%%%%%%%%%%%%%%%%%%%%%%%%%%%%%%%%



\subsection*{3.2. Multiplicities and Local Rings \\}
\addcontentsline{toc}{subsection}{3.2. Multiplicities and Local Rings}



\subsubsection*{Problem 3.12. (Flex)}
\addcontentsline{toc}{subsubsection}{Problem 3.12. (Flex)}
\emph{A simple point $P$ on a curve $f$ is called a \textbf{flex} if $\mathrm{ord}^{f}_P(L) \geq 3$,
where $L$ is the tangent to $f$ at $P$.
The flex is called \textbf{ordinary} if $\mathrm{ord}_P(L) = 3$,
a \textbf{higher} flex otherwise.}
\begin{enumerate}
\item[(a)]
  \emph{Let $f = y - x^n$.
  For which $n$ does $f$ have a flex at $P = (0, 0)$, and what kind of flex?}

\item[(b)]
  \emph{Suppose $P = (0, 0)$, $L = y$ is the tangent line,
  $f = y + a x^2 + \cdots$.
  Show that $P$ is a flex on $f$ if and only if $a = 0$.
  Give a simple criterion for calculating $\mathrm{ord}^{f}_P(y)$,
  and therefore for determining if $P$ is a higher flex.} \\
\end{enumerate}



\emph{Proof of (a).}
\begin{enumerate}
\item[(1)]
  When $n = 0$ or $1$, the tangent line $L$ to $f$ at any point is $L = f$ itself.
  So
  \[
    \mathrm{ord}^{f}_P(L)
    = \mathrm{ord}^{f}_P(f)
    = \mathrm{ord}^{f}_P(0)
    = \infty.
  \]
  $P$ is a higher flex.

\item[(2)]
  When $n > 1$, the tangent line $L$ to $f$ at $P = (0,0)$ is $L = y$.
  So
  \[
    \mathrm{ord}^{f}_P(L)
    = \mathrm{ord}^{f}_P(y)
    = \mathrm{ord}^{f}_P(x^n)
    = n.
  \]
  Here $x$ is a uniformizing parameter for $\mathscr{O}_P(f)$
  since the line $x$ is not tangent to $f$ (Theorem 1).
  Hence, $P$ is a flex if $n \geq 3$,
  $P$ is an ordinary flex if $n = 3$, and
  $P$ is a higher flex if $n > 3$.
\end{enumerate}
$\Box$ \\



\emph{Proof of (b).}
\begin{enumerate}
\item[(1)]
  Since $y$ is the tangent line, $\mathrm{ord}^{f}_P(y) \geq 2$.
  By Problem 2.29(a),
  \[
    \mathrm{ord}^{f}_P(y)
    = \mathrm{ord}^{f}_P(ax^2 + \cdots)
    = 2
  \]
  if and only if $a \neq 0$.
  Hence, $P$ is flex iff $\mathrm{ord}^{f}_P(y) \geq 3$ iff $a = 0$.

\item[(2)]
  In general,
  \[
    \mathrm{ord}^{f}_P(y)
    = \mathrm{ord}^{f}_P(ax^2 + \cdots)
    = m_P(ax^2 + \cdots)
    = m_P(f - y).
  \]
  Hence, $P$ is a higher flex if $f - y$ has no nonzero form of degree $3$.
\end{enumerate}
$\Box$ \\\\



%%%%%%%%%%%%%%%%%%%%%%%%%%%%%%%%%%%%%%%%%%%%%%%%%%%%%%%%%%%%%%%%%%%%%%%%%%%%%%%%



\subsubsection*{Problem 3.13.*}
\addcontentsline{toc}{subsubsection}{Problem 3.13.*}
\emph{With the notation of Theorem 2, and $\mathfrak{m} = \mathfrak{m}_P(f)$,
show that $\dim_k(\mathfrak{m}^n/\mathfrak{m}^{n+1}) = n+1$ for $0 \leq n < m_P(f)$.
In particular, $P$ is a simple point if and only if $\dim_k(\mathfrak{m}/\mathfrak{m}^2) = 1$;
otherwise $\dim_k(\mathfrak{m}/\mathfrak{m}^2) = 2$.} \\



\emph{Proof.}
\begin{enumerate}
\item[(1)]
  From the exact sequence
  \[
    0
    \to \mathfrak{m}^{n}/\mathfrak{m}^{n+1}
    \to \mathscr{O}/\mathfrak{m}^{n+1}
    \to \mathscr{O}/\mathfrak{m}^{n}
    \to
    0,
  \]
  it suffices to show that
  \[
    \dim_k(\mathscr{O}/\mathfrak{m}^{n}) = \frac{n(n+1)}{2}
  \]
  as $0 \leq n < m_P(f)$. (Problem 2.49 and Proposition 7 in \S 2.10.)

\item[(2)]
  We may assume that $P = (0,0)$. Similar to the proof of Theorem 2,
  we are reduced to calculating the dimension of $k[x,y]/(I^n,f)$.
  Let $m = m_P(f)$.
  By the definition of $m$,
  \[
    f \in {\underbrace{(x,y)}_{= I }}^{m} = I^{m}.
  \]
  So if $0 \leq n < m_P(f)$, then $f \in I^{m} \subseteq I^n$ and thus $(I^n,f) = I^n$.
  Therefore,
  \begin{align*}
    \dim_k(\mathscr{O}/\mathfrak{m}^{n})
    &= \dim_k(k[x,y]/(I^n,f)) \\
    &= \dim_k(k[x,y]/I^n) \\
    &= \frac{n(n+1)}{2}.
      &(\text{Problem 2.46})
  \end{align*}
  So
  \[
    \dim_k(\mathfrak{m}^n/\mathfrak{m}^{n+1})
    = \dim_k(\mathscr{O}/\mathfrak{m}^{n+1}) - \dim_k(\mathscr{O}/\mathfrak{m}^{n})
    = n + 1.
  \]

\item[(3)]
  $P$ is a simple point if $m = m_P(f) = 1$ by definition.
  Note that
  \begin{equation*}
    \dim_k(\mathfrak{m}/\mathfrak{m}^2) =
    \begin{cases}
      1 & \text{if $m = 1$ (Theorem 1)} \\
      2 & \text{if $m > 1$ ((2))}. \\
    \end{cases}
  \end{equation*}
  Therefore,
  $P$ is a simple iff $m = 1$ iff $\dim_k(\mathfrak{m}/\mathfrak{m}^2) = 1$.
\end{enumerate}
$\Box$ \\\\



%%%%%%%%%%%%%%%%%%%%%%%%%%%%%%%%%%%%%%%%%%%%%%%%%%%%%%%%%%%%%%%%%%%%%%%%%%%%%%%%



\subsubsection*{Problem 3.14.}
\addcontentsline{toc}{subsubsection}{Problem 3.14.}
\emph{Let $V = V(x^2-y^3, y^2-z^3) \subseteq \mathbf{A}^3$,
$P = (0, 0, 0)$, $\mathfrak{m} = \mathfrak{m}_P(V)$.
Find $\dim_k(\mathfrak{m}/\mathfrak{m}^2)$. (See Problem 1.40.)} \\



\emph{Proof.}
\begin{enumerate}
\item[(1)]
  $\mathfrak{m} = (x, y, z)$.

\item[(2)]
  Write $\mathscr{O} = \mathscr{O}_P(V)$.
  By Problem 1.40(a),
  every element of $\mathscr{O}$ is of the form
  \[
    \overline{a}
    + \overline{x}\overline{b}
    + \overline{y}\overline{c}
    + \overline{x}\overline{y}\overline{d}
  \]
  for some $a, b, c, d \in k[z]$.

\item[(3)]
  By (1)(2),
  $\{ \overline{1} \}$ (resp. $\{ \overline{1}, \overline{z}, \overline{y}, \overline{z} \}$)
  is a basis for $\mathscr{O}/\mathfrak{m}$ (resp. $\mathscr{O}/\mathfrak{m}^2$).
  Hence, $\dim_k(\mathscr{O}/\mathfrak{m}) = 1$ (resp. $\dim_k(\mathscr{O}/\mathfrak{m}^2) = 4$).
  Therefore,
  \[
    \dim_k(\mathfrak{m}/\mathfrak{m}^2)
    = \dim_k(\mathscr{O}/\mathfrak{m}^2) - \dim_k(\mathscr{O}/\mathfrak{m})
    = 3
  \]
  by Proposition 7 in \S 2.10.

\item[(4)]
  By Theorem I.5.1 in \emph{Robin Hartshorne, Algebraic Geometry},
  \[
    3
    = \dim_k(\mathfrak{m}/\mathfrak{m}^2) + \mathrm{rank} J
    = \dim_k(\mathfrak{m}/\mathfrak{m}^2).
  \]
  Here the Jacobian matrix of $V$ at $P$
  \[
    J
    = \left[ \frac{\partial f_i}{\partial x_j}(P) \right]
    =
    \begin{pmatrix}
      2x & -3y^2 & 0 \\
      0 & 2y & -3z^2
    \end{pmatrix}_{P=(0,0,0)}
    = 0
  \]
  has rank zero.
\end{enumerate}
$\Box$ \\\\



%%%%%%%%%%%%%%%%%%%%%%%%%%%%%%%%%%%%%%%%%%%%%%%%%%%%%%%%%%%%%%%%%%%%%%%%%%%%%%%%



\subsubsection*{Problem 3.15.}
\addcontentsline{toc}{subsubsection}{Problem 3.15.}
\begin{enumerate}
\item[(a)]
  \emph{Let $\mathscr{O} = \mathscr{O}_P(\mathbf{A}^2)$ for some $P \in \mathbf{A}^2$,
  $\mathfrak{m} = \mathfrak{m}_P(\mathbf{A}^2)$.
  Calculate $\chi(n) = \dim_k(\mathscr{O}/\mathfrak{m}^n)$.}

\item[(b)]
  \emph{Let $\mathscr{O} = \mathscr{O}_P(\mathbf{A}^r(k))$.
  Show that $\chi(n)$ is a polynomial of degree $r$ in $n$,
  with leading coefficient $\frac{1}{r!}$ (see Problem 2.36).} \\
\end{enumerate}



\emph{Proof of (a).}
  Might assume that $P = (0,0)$.
  By Problem 2.46, the Hilbert-Samuel polynomial is
  \[
    \chi(n)
    = \dim_k(\mathscr{O}/\mathfrak{m}^n)
    = \dim_k(k[x,y]/(x,y)^n)
    = \frac{n(n+1)}{2}.
  \]
$\Box$ \\



\emph{Proof of (b).}
\begin{enumerate}
\item[(1)]
  Might assume that $P = (0, \ldots, 0)$.
  Similar to (a),
  \[
    \chi(n)
    = \dim_k(\mathscr{O}/\mathfrak{m}^n)
    = \dim_k(k[x_1,\ldots,x_r]/(x_1,\ldots,x_r)^n).
  \]

\item[(2)]
  Since
  \[
    \mathscr{B}
    = \{ x_1^{i_1} x_2^{i_2} \cdots x_r^{i_r} : i_1 + \cdots + i_r < n \}
  \]
  is a basis for $k[x_1,\ldots,x_r]/(x_1,\ldots,x_r)^n$,
  \[
    \dim_k(k[x_1,\ldots,x_r]/(x_1,\ldots,x_r)^n) = \abs{ \mathscr{B} }.
  \]

\item[(3)]
  By the \href{https://en.wikipedia.org/wiki/Stars_and_bars_%28combinatorics%29}
  {stars and bars (combinatorics)} method (as Problem 2.35(b)),
  \begin{align*}
    \abs{ \mathscr{B} }
    =& \: \abs{ \{ (i_1, \ldots i_r) : i_1 + \cdots + i_r \leq n \} } \\
      & \: - \abs{ \{ (i_1, \ldots i_r) : i_1 + \cdots + i_r = n \} } \\
    =& \: \abs{ \{ (i_1, \ldots i_r, j) : i_1 + \cdots + i_r + j = n \} } \\
      & \: - \abs{ \{ (i_1, \ldots i_r) : i_1 + \cdots + i_r = n \} } \\
    =& \: { n+r \choose r } - { n+r-1 \choose r-1 } \\
    =& \: { n+r-1 \choose r } \\
    =& \: \frac{1}{r!} (n+r-1)(n+r-2) \cdots (n+1)(n).
  \end{align*}
  So
  \[
    \chi(n)
    = \frac{1}{r!} (n+r-1)(n+r-2) \cdots (n+1)(n)
  \]
  is a polynomial of degree $r$ in $n$, with leading coefficient $\frac{1}{r!}$.

\item[(4)]
  By Problem 2.36, we can also deduce that
  \begin{align*}
    \dim_k(k[x_1,\ldots,x_r]/(x_1,\ldots,x_r)^n)
    &= \sum_{i=0}^{n-1} \dim_k V(i,r) \\
    &= \sum_{i=0}^{n-1} { i+r-1 \choose r-1 } \\
    &= { n+r-1 \choose r }
  \end{align*}
  by the Pascal's identity.
\end{enumerate}
$\Box$ \\\\



%%%%%%%%%%%%%%%%%%%%%%%%%%%%%%%%%%%%%%%%%%%%%%%%%%%%%%%%%%%%%%%%%%%%%%%%%%%%%%%%



\subsubsection*{Problem 3.16.}
\addcontentsline{toc}{subsubsection}{Problem 3.16.}
\emph{Let $f \in k[x_1,\ldots,x_r]$ define a hypersurface in $\mathbf{A}^r$.
Write $f = f_m + f_{m+1} + \cdots$,
and let $m = m_P(f)$ where $P = (0, \ldots, 0)$.
Suppose $f$ is irreducible,
and let $\mathscr{O} = \mathscr{O}_P(V(f))$,
$\mathfrak{m}$ its maximal ideal.
Show that $\chi(n) = \dim_k(\mathscr{O}/\mathfrak{m}^n)$ is a polynomial of degree $r-1$
for sufficiently large $n$,
and that the leading coefficient of $\chi$ is $\frac{m_P(f)}{(r-1)!}$.
Can you find a definition for the multiplicity of a local ring
that makes sense in all the cases you know?} \\



\emph{Proof.}
\begin{enumerate}
\item[(1)]
  Similar to the proof of Theorem 2.
  By Problem 2.43,
  \[
    \mathfrak{m}^n = I^n \mathscr{O}
  \]
  where $I = (x_1, \ldots, x_r) \subseteq k[x_1,\ldots,x_r]$.
  Since $V(I^n) = \{P\}$,
  \[
    k[x_1,\ldots,x_r]/(I^n,f)
    \cong
    \mathscr{O}_P(\mathbf{A}^r)/(I^n,f)\mathscr{O}_P(\mathbf{A}^r)
    \cong
    \mathscr{O}/I^n \mathscr{O}
    \cong
    \mathscr{O}/\mathfrak{m}^n
  \]
  (Corollary 2 to Proposition 6 in \S 2.9 and Problem 2.44).

\item[(2)]
  So we are reduced to calculating the dimension of $k[x_1,\ldots,x_r]/(I^n,f)$.
  As $n \geq m = m_P(f)$,
  there is a natural ring homomorphism
  \[
    \varphi: k[x_1,\ldots,x_r]/I^n \to k[x_1,\ldots,x_r]/(I^n,f)
  \]
  and a $k$-linear map
  \[
    \psi: k[x_1,\ldots,x_r]/I^{n-m} \to k[x_1,\ldots,x_r]/I^n
  \]
  defined by $\overline{g} \mapsto \overline{fg}$.
  It is easy to verify that the sequence
  \[
    0
    \to k[x_1,\ldots,x_r]/I^{n-m}
    \xrightarrow{\psi} k[x_1,\ldots,x_r]/I^{n}
    \xrightarrow{\varphi} k[x_1,\ldots,x_r]/(I^{n},f)
    \to 0
  \]
  is exact.

\item[(3)]
  By Problem 3.15,
  \begin{align*}
    & \: \dim_k(k[x_1,\ldots,x_r]/(I^{n},f)) \\
    =& \:
    {n+r-1 \choose r} - {n-m+r-1 \choose r} \\
    =& \:
    \frac{1}{r!} \left( (n+r-1) \cdots n - (n-m+r-1) \cdots (n-m) \right) \\
    =& \:
    \frac{1}{r!} \left( n^{r-1}(rm) + \cdots \right) \\
    =& \:
    \frac{m}{(r-1)!} n^{r-1} + \cdots.
  \end{align*}
  Therefore, $\chi(n) = \dim_k(\mathscr{O}/\mathfrak{m}^n)$ is a polynomial of degree $r-1$
  for $n \geq m$,
  and that the leading coefficient of $\chi$ is $\frac{m_P(f)}{(r-1)!}$.

\item[(4)]
  It is reasonable to define the multiplicity of a Noetherian local ring $\mathscr{O}$ by
  \[
    (d!) \cdot (\text{leading coefficient of } \chi(n))
  \]
  for sufficiently large $n$,
  where $d$ is the dimension (or Krull dimension) of $\mathscr{O}$.
  (Note that the dimension of a hypersurface in $\mathbf{A}^r$ is $r-1$.)
\end{enumerate}
$\Box$ \\\\



%%%%%%%%%%%%%%%%%%%%%%%%%%%%%%%%%%%%%%%%%%%%%%%%%%%%%%%%%%%%%%%%%%%%%%%%%%%%%%%%
%%%%%%%%%%%%%%%%%%%%%%%%%%%%%%%%%%%%%%%%%%%%%%%%%%%%%%%%%%%%%%%%%%%%%%%%%%%%%%%%



\subsection*{3.3. Intersection Numbers \\}
\addcontentsline{toc}{subsection}{3.3. Intersection Numbers}



\subsubsection*{Problem 3.17.}
\addcontentsline{toc}{subsubsection}{Problem 3.17.}
\emph{Find the intersection numbers of various pairs of curves}
\begin{enumerate}
\item[(a)]
  $a = y - x^2$

\item[(b)]
  $b = y^2 - x^3 + x$

\item[(c)]
  $c = y^2 - x^3$

\item[(d)]
  $d = y^2 - x^3 - x^2$

\item[(e)]
  $e = (x^2+y^2)^2 + 3x^2 y - y^3$

\item[(f)]
  $f = (x^2+y^2)^3 - 4x^2 y^2$
\end{enumerate}
\emph{at the point $P = (0, 0)$.} \\



\emph{Proof.}
\begin{enumerate}
\item[(1)]
  Note that Example in \S 3.3 shows that $I(P, e \cap f) = 14$.
  Also,
  \begin{align*}
    I(P, a \cap b) &= m_P(a) m_P(b) = 1 \cdot 1 = 1 \\
    I(P, a \cap d) &= m_P(a) m_P(d) = 1 \cdot 2 = 2 \\
    I(P, b \cap c) &= m_P(b) m_P(c) = 1 \cdot 2 = 2 \\
    I(P, b \cap d) &= m_P(b) m_P(d) = 1 \cdot 2 = 2 \\
    I(P, b \cap e) &= m_P(b) m_P(e) = 1 \cdot 3 = 3 \\
    I(P, c \cap d) &= m_P(c) m_P(d) = 2 \cdot 2 = 4 \\
    I(P, d \cap e) &= m_P(d) m_P(e) = 2 \cdot 3 = 6 \\
    I(P, d \cap f) &= m_P(d) m_P(f) = 2 \cdot 4 = 8
  \end{align*}
  by Property (5).

\item[(2)]
  \emph{Show that $I(P, a \cap c) = 3$.}
  \begin{align*}
    I(P, a \cap c)
    &= I(P, a \cap (c + (-x^2-y)a))
      & \text{(Property (7))} \\
    &= I(P, a \cap x^3(x - 1)) \\
    &= 3I(P, a \cap x) + I(P, a \cap (x - 1))
      & \text{(Property (6))} \\
    &= 3I(P, a \cap x)
      & \text{(Property (2))} \\
    &= 3.
      & \text{(Property (5))}
  \end{align*}

\item[(3)]
  \emph{Show that $I(P, a \cap e) = 4$.}
  \begin{align*}
    I(P, a \cap e)
    &= I(P, a \cap (e + (x^2+2y^2+4y)a))
      & \text{(Property (7))} \\
    &= I(P, a \cap y^2(y^2+y+4)) \\
    &= 2I(P, a \cap y) + I(P, a \cap (y^2+y+4))
      & \text{(Property (6))} \\
    &= 2I(P, a \cap y)
      & \text{(Property (2))} \\
    &= 2I(P, (a-y) \cap y)
      & \text{(Property (7))} \\
    &= 2I(P, (-x^2) \cap y) \\
    &= 4.
      & \text{(Property (5))}
  \end{align*}

\item[(4)]
  \emph{Show that $I(P, a \cap f) = 6$.}
  Similar to (3).
  Let
  \[
    q_4 = x^4+3x^2y^2+3y^4+x^2y+3y^3-3y^2.
  \]
  So
  \begin{align*}
    I(P, a \cap f)
    =& \: I(P,
      a \cap \underbrace{(f + q_4 a)}_{\text{(Property (7))}}) \\
    =& \: I(P, a \cap y^3(y^3+3y^2+3y-3)) \\
    =& \: \overbrace{3 \underbrace{I(P, a \cap y)}_{= 2 \text{ (by (3))}}
      + \underbrace{I(P, a \cap (y^3+3y^2+3y-3))}_{
        = 0 \text{ (Property (2))}}}^{\text{(Property (6)})} \\
    =& \: 6.
  \end{align*}

\item[(5)]
  \emph{Show that $I(P, b \cap f) = 6$.}
  Similar to (3).
  Let
  \[
    q_5 = -x^6 -3x^5-x^3y^2 -x^4-3x^2y^2-y^4 +3x^3+xy^2 + 3x^2.
  \]
  So
  \begin{align*}
    & \: I(P, b \cap f) \\
    =& \: I(P,
      b \cap \underbrace{(f + q_5 b)}_{\text{(Property (7))}}) \\
    =& \: I(P, b \cap x^3(x^6+3x^5-5x^3-4x^2+3x+3)) \\
    =& \: \overbrace{3 \underbrace{I(P, b \cap x)}_{= 2}
      + \underbrace{I(P, b \cap (x^6+3x^5-5x^3-4x^2+3x+3))}_{
        = 0 \text{ (Property (2))}}}^{\text{(Property (6)})} \\
    =& \: 6.
  \end{align*}
  Here
  \[
    I(P, b \cap x) = I(P, (b + (x^2-1)x) \cap x)= I(P, y^2 \cap x) = 2.
  \]

\item[(6)]
  \emph{Show that $I(P, c \cap e) = 7$.}
  Similar to (3).
  \begin{align*}
    I(P, c \cap e)
    &= I(P, c \cap \underbrace{(e + (-x^3-2x^2-y^2+y)c)}_{\text{(Property (7))}}) \\
    &= I(P, c \cap x^2(x^4 + 2x^3 + x^2 - xy + 3y)) \\
    &= \overbrace{2 \underbrace{I(P, c \cap x)}_{= 2 \text{ (Property (5))}}
      + I(P, c \cap (\underbrace{x^4 + 2x^3 + x^2 - xy + 3y}_{:= h_6}))}^{\text{(Property (6)})} \\
    &= 4 + I(P, c \cap h_6).
  \end{align*}
  Here
  \begin{align*}
    & \: I(P, c \cap h_6) \\
    =& \: I(P, (3c) \cap h_6) \\
    =& \: I(P, (3c - y h_6) \cap h_6) \\
    =& \: I(P, (-x^4y - 2x^3y \underbrace{-3x^3-x^2y+xy^2})
      \cap (x^4 + 2x^3 + x^2 - xy + \underbrace{3y})) \\
    =& \: 3 \cdot 1 \\
    =& \: 3
  \end{align*}
  by Properties (5) and (7).
  Therefore, $I(P, c \cap e) = 4 + 3 = 7$.

\item[(7)]
  \emph{Show that $I(P, c \cap f) = 10$.}
  Similar to (5).
  Let
  \[
    q_7 = -x^6 -3x^5-x^3y^2 -3x^4-3x^2y^2-y^4 + 4x^2.
  \]
  So
  \begin{align*}
    & \: I(P, c \cap f) \\
    =& \: I(P,
      c \cap \underbrace{(f + q_7 c)}_{\text{(Property (7))}}) \\
    =& \: I(P, c \cap x^5(x^4+3x^3+3x^2+x-4) \\
    =& \: \overbrace{5 \underbrace{I(P, c \cap x)}_{= 2}
      + \underbrace{I(P, c \cap (x^4+3x^3+3x^2+x-4))}_{
        = 0 \text{ (Property (2))}}}^{\text{(Property (6)})} \\
    =& \: 10.
  \end{align*}
\end{enumerate}
$\Box$ \\\\



%%%%%%%%%%%%%%%%%%%%%%%%%%%%%%%%%%%%%%%%%%%%%%%%%%%%%%%%%%%%%%%%%%%%%%%%%%%%%%%%



\subsubsection*{Problem 3.18.}
\addcontentsline{toc}{subsubsection}{Problem 3.18.}
\emph{Give a proof of Property (8) that uses only Properties (1)-(7).} \\



Recall Properties (1)-(8):
\begin{enumerate}
\item[(1)]
  $I(P, f \cap g)$ is a nonnegative integer for any $f$, $g$,
  and $P$ such that $f$ and $g$ intersect properly at $P$.
  $I(P, f \cap g) = \infty$ if $f$ and $g$ do not intersect properly at $P$.

\item[(2)]
  $I(P, f \cap g) = 0$ if and only if $P \not\in f \cap g$.
  $I(P, f \cap g)$ depends only on the components of $f$ and $g$ that pass through $P$.

\item[(3)]
  If $t$ is an affine change of coordinates on $\mathbf{A}^2$,
  and $t(Q) = P$, then
  \[
    I(P, f \cap g) = I(Q, f^t \cap g^t).
  \]

\item[(4)]
  $I(P, f \cap g) = I(P, g \cap f)$.

\item[(5)]
  $I(P, f \cap g) \geq m_P(f) m_P(g)$,
  with equality occurring if and only if $f$ and $g$ have not tangent lines in common at $P$.

\item[(6)]
  If $f = \prod f_i^{r_i}$, and $g = \prod g_j^{s_j}$, then
  \[
    I(P, f \cap g) = \sum_{i, j} r_i s_j I(P, f_i \cap g_j).
  \]

\item[(7)]
  $I(P, f \cap g) = I(P, f \cap (g+af))$ for any $a \in k[x,y]$.

\item[(8)]
  If $P$ is a simple point on $f$, then $I(P, f \cap g) = \mathrm{ord}_P^{f}(g)$. \\
\end{enumerate}



\emph{Proof.}
\begin{enumerate}
\item[(1)]
  Might assume that $f$ is irreducible.
  There is nothing to prove if $\mathrm{ord}_P^{f}(g) = \infty$.
  Might assume that $n = \mathrm{ord}_P^{f}(g) < \infty$.

\item[(2)]
  Similar to the proof of Theorem 1 in \S 3.2,
  Property (3) implies that we might assume that $P = (0,0)$,
  that $y$ is the tangent line,
  and that $x$ is one line through $P$ which is not tangent to $f$ at $P$.
  Here $x$ is a uniformizing parameter for $\mathscr{O}_P(f)$ (Theorem 1 in \S 3.2).

\item[(3)]
  By definition,
  \[
    g = ux^n
  \]
  for some unit in $\mathscr{O}_P(f)$.
  Write $u = \frac{a}{b}$ where $a, b \in k[x,y]$, $a(P) \neq 0$ and $b(P) \neq 0$.
  Hence,
  \[
    bg = ax^n + cf
  \]
  in $k[x,y]$ for some $c \in k[x,y]$.

\item[(4)]
  Therefore,
  \begin{align*}
    I(P, f \cap g)
    &= I(P, f \cap bg) - I(P, f \cap b)
      &(\text{Property (6)}) \\
    &= I(P, f \cap bg) - 0
      &(\text{Property (2)}) \\
    &= I(P, f \cap (ax^n + cf))
      &(\text{Step (3)}) \\
    &= I(P, f \cap ax^n)
      &(\text{Property (7)}) \\
    &= I(P, f \cap a) + n I(P, f \cap x)
      &(\text{Property (6)}) \\
    &= 0 + n I(P, f \cap x)
      &(\text{Property (2)}) \\
    &= n.
      &(\text{Property (5)})
  \end{align*}
\end{enumerate}
$\Box$ \\\\



%%%%%%%%%%%%%%%%%%%%%%%%%%%%%%%%%%%%%%%%%%%%%%%%%%%%%%%%%%%%%%%%%%%%%%%%%%%%%%%%



\subsubsection*{Problem 3.19.*}
\addcontentsline{toc}{subsubsection}{Problem 3.19.*}
\emph{A line $L$ is tangent to a curve $f$ at a point $P$ if and only if
$I(P, f \cap L) > m_P(f)$.} \\



\emph{Proof.}
\begin{enumerate}
\item[(1)]
  Note that $m_P(L) = 1$ and the only tangent line of $L$ is itself.

\item[(2)]
  By Property (5),
  \begin{align*}
    & \: I(P, f \cap L) > m_P(f) = m_P(f) m_P(L) \\
    \Longleftrightarrow & \:
      \text{$f$ and $L$ have one common tangent line at $P$} \\
    \Longleftrightarrow & \:
      \text{$L$ is tangent to a curve $f$ at $P$}.
      &\text{(By (1))}
  \end{align*}
\end{enumerate}
$\Box$ \\\\



%%%%%%%%%%%%%%%%%%%%%%%%%%%%%%%%%%%%%%%%%%%%%%%%%%%%%%%%%%%%%%%%%%%%%%%%%%%%%%%%



\subsubsection*{Problem 3.20.}
\addcontentsline{toc}{subsubsection}{Problem 3.20.}
\emph{If $P$ is a simple point on $f$,
then $I(P, f \cap (g+h)) \geq \min\{ I(P, f \cap g), I(P, f \cap h) \}$.
Give an example to show that this may be false if $P$ is not simple on $f$.} \\



\emph{Proof.}
\begin{enumerate}
\item[(1)]
  \begin{align*}
    I(P, f \cap (g+h))
    &= \mathrm{ord}_P^f(g+h)
      &(\text{Property (8)}) \\
    &\geq \min\left\{ \mathrm{ord}_P^f(g), \mathrm{ord}_P^f(h) \right\}
      &(\text{Problem 2.28}) \\
    &= \min\{ I(P, f \cap g), I(P, f \cap h) \}.
      &(\text{Property (8)})
  \end{align*}

\item[(2)]
  Pick $P = (0,0)$, $f = (x^2+y^2)^3 - 4x^2y^2$, $g = x$ and $h = y$.
  By Property (5),
  $I(P, f \cap (g+h)) = 4$, $I(P, f \cap g) > 4$ and $I(P, f \cap h) > 4$.
  So
  \[
    I(P, f \cap (g+h)) < \min\{ I(P, f \cap g), I(P, f \cap h) \}
  \]
  for such example.
\end{enumerate}
$\Box$ \\\\



%%%%%%%%%%%%%%%%%%%%%%%%%%%%%%%%%%%%%%%%%%%%%%%%%%%%%%%%%%%%%%%%%%%%%%%%%%%%%%%%



\subsubsection*{Problem 3.21.}
\addcontentsline{toc}{subsubsection}{Problem 3.21.}
\emph{Let $f$ be an affine plane curve.
Let $L$ be a line that is not a component of $f$.
Suppose $L = \{(a+tb, c+td) : t \in k \}$.
Define $g(t) = f(a+tb, c+td)$.
Factor $g(t) = \prod (t - \lambda_i)^{e_i}$, $\lambda_i$ distinct.
Show that there is a natural one-to-one correspondence between
the $\lambda_i$ and the points $P_i \in L \cap f$.
Show that under this correspondence, $I(P_i, L \cap f) = e_i$.
In particular, $\sum I(P, L \cap f) \leq \deg(f)$.} \\



\emph{Proof.}
\begin{enumerate}
\item[(1)]
  \emph{Show that there is a natural one-to-one correspondence between
  the $\lambda_i$ and the points $P_i \in L \cap f$.}
  \begin{align*}
    P_i \in L \cap f
    &\Longleftrightarrow
    P_i \in L \text{ and } P \in f \\
    &\Longleftrightarrow
    \exists \lambda \in k \text{ such that }
    0 = f(P_i) = f(a+\lambda b, c+\lambda d) = g(\lambda) \\
    &\Longleftrightarrow
    \text{$\lambda \in k$ is a root of $g(t) = \prod (t - \lambda_i)^{e_i}$} \\
    &\Longleftrightarrow
    \lambda = \lambda_i \in k \text{ for some } i.
  \end{align*}

\item[(2)]
  \emph{Show that $I(P_i, L \cap f) = e_i$.}
  By Property (3),
  we may suppose $P_i = (0,0)$ and $L = y$.
  Write
  \[
    f(x,y) = f_0(x) + f_1(x)y + f_2(x)y^2 + \cdots \in (k[x])[y]
  \]
  where $f_j \in k[x]$.
  Note that $f_0(x) = f(x,0) = g(x)$.
  So
  \begin{align*}
    I(P_i, L \cap f)
    &= I(P_i, y \cap (f_0(x) + f_1(x)y + \cdots)) \\
    &= I(P_i, y \cap f_0(x))
      &(\text{Property (7)}) \\
    &= I(P_i, y \cap g(x)) \\
    &= I\left( P_i, y \cap \prod (x - \lambda_i)^{e_i} \right) \\
    &= \sum_j e_j I(P_j, y \cap (x - \lambda_j))
      &(\text{Property (6)}) \\
    &= e_i I(P_i, y \cap x)
      &(\text{Property (2)}) \\
    &= e_i.
      &(\text{Property (5)})
  \end{align*}
  Here $\lambda_i = 0$ by the correspondence of (1).

\item[(3)]
  In particular,
  \[
    \sum_i I(P_i, L \cap f)
    = \sum_i e_i
    = \deg(g(x))
    = \deg(f(x,0))
    \leq \deg(f(x,y)).
  \]
\end{enumerate}
$\Box$ \\\\



%%%%%%%%%%%%%%%%%%%%%%%%%%%%%%%%%%%%%%%%%%%%%%%%%%%%%%%%%%%%%%%%%%%%%%%%%%%%%%%%



\subsubsection*{Problem 3.22. (Cusp)}
\addcontentsline{toc}{subsubsection}{Problem 3.22. (Cusp)}
\emph{Suppose $P$ is a double point on a curve $f$,
and suppose $f$ has only one tangent $L$ at $P$.}
\begin{enumerate}
\item[(a)]
  \emph{Show that $I(P, f \cap L) \geq 3$.
  The curve $f$ is said to have an (ordinary) \textbf{cusp} at $P$ if $I(P, f \cap L) = 3$.}

\item[(b)]
  \emph{Suppose $P = (0,0)$, and $L = y$.
  Show that $P$ is a cusp if and only if
  $\frac{\partial^3 f}{\partial x^3}(P) \neq 0$. Give some examples.}

\item[(c)]
  \emph{Show that if $P$ is a cusp on $f$,
  then $f$ has only one component passing through $P$.} \\
\end{enumerate}



Might assume that $\mathrm{char}(k) = 0$. \\



\emph{Proof of (a).}
  Since $I(P, f \cap L) > m_P(f) = 2$ (Problem 3.19),
  $I(P, f \cap L) \geq 3$ (Property (1)).
$\Box$ \\



\emph{Proof of (b).}
\begin{enumerate}
\item[(1)]
  By assumption,
  \[
    f = y^2 + f_3 + f_4 + \cdots
  \]
  where $f_i$ is a form in $k[x,y]$.

\item[(2)]
  Hence, $P$ is a cusp of $f$
  if and only if
  \begin{align*}
    3
    &= I(P, f \cap y) \\
    &= I(P, (y^2 + f_3 + f_4 + \cdots) \cap y) \\
    &= I(P, (f_3 + f_4 + \cdots) \cap y) \\
    &\geq m_P(f_3 + f_4 + \cdots) m_P(y) \\
    &\geq 3.
  \end{align*}
  Here the equality is occuring if and only if
  $y$ is not a tangent line of $f_3 + f_4 + \cdots$ (Property (5)).

\item[(3)]
  Note that
  \begin{align*}
    & \: \text{$y$ is not a tangent line of $f_3 + f_4 + \cdots$} \\
    \Longleftrightarrow & \:
    y \nmid f_3 \\
    \Longleftrightarrow & \:
    \frac{\partial^3 f}{\partial x^3}(P) \neq 0.
      &(\text{Problem 3.5})
  \end{align*}

\item[(4)]
  Examples: $y^2 = x^3$, $y^2 = -x^2 y^2 + x^3$ and so on.
\end{enumerate}
$\Box$ \\



\emph{Proof of (c).}
\begin{enumerate}
\item[(1)]
  Might assume $P = (0,0)$ and $L = y$ by Property (3).

\item[(2)]
  Given $f = gh$.
  Write
  \begin{align*}
    f &= y^2 + \text{higher terms} \\
    g &= g_r + \text{higher terms} \\
    h &= h_s + \text{higher terms}
  \end{align*}
  where $g_r$ (resp. $h_s$) is a form of degree $r$ (resp. $s$) in $k[x,y]$.
  So $f = gh$ implies that
  \begin{align*}
    y^2 + \text{higher terms}
    &= (g_r + \text{higher terms})(h_s + \text{higher terms}) \\
    &= (g_r h_s) + \text{higher terms}.
  \end{align*}
  Hence $y^2 = g_r h_s$. In particular, $2 = r+s$.

\item[(3)]
  If $y \mid g_r$ and $y \mid h_s$, then
  \begin{align*}
    I(P, g \cap L) > m_P(g) m_P(L) = r \Longrightarrow& I(P, g \cap L) \geq r+1 \\
    I(P, h \cap L) > m_P(h) m_P(L) = s \Longrightarrow& I(P, g \cap L) \geq s+1.
  \end{align*}
  So,
  \begin{align*}
    3
    &= I(P, f \cap L) \\
    &= I(P, g \cap L) + I(P, h \cap L) \\
    &\geq (r+1) + (s+1) \\
    &= 4,
  \end{align*}
  which is absurd.
  So we might assume that $g_r = 1$ and $h_s = y^2$.
  $f = gh$ implies that $g = 1 + g_1 + \cdots$ is not passing through $P$.
  Hence the conclusion is established.
\end{enumerate}
$\Box$ \\\\



%%%%%%%%%%%%%%%%%%%%%%%%%%%%%%%%%%%%%%%%%%%%%%%%%%%%%%%%%%%%%%%%%%%%%%%%%%%%%%%%



\subsubsection*{Problem 3.23. (Hypercusp)}
\addcontentsline{toc}{subsubsection}{Problem 3.23. (Hypercusp)}
\emph{A point $P$ on a curve $f$ is called a \textbf{hypercusp} if $m_P(f) > 1$,
$f$ has only one tangent line $L$ at $P$,
and $I(P, L \cap f) = m_P(f) + 1$.
Generalize the results of the preceding problem to this case.} \\



\emph{Generalization.}
\begin{enumerate}
\item[(a)]
  \emph{$I(P, f \cap L) \geq m_P(f) + 1$.}

\item[(b)]
  \emph{Suppose $P = (0,0)$, $L = y$ and $m = m_P(f)$.
  Then $P$ is a hypercusp if and only if
  $\frac{\partial^{m+1} f}{\partial x^{m+1}}(P) \neq 0$. Give some examples.}

\item[(c)]
  \emph{If $P$ is a hypercusp on $f$,
  then $f$ has only one component passing through $P$.}
\end{enumerate}
The proof is almost the same as Problem 3.22 by replacing $2$ by $m_P(f)$.
$\Box$ \\\\



%%%%%%%%%%%%%%%%%%%%%%%%%%%%%%%%%%%%%%%%%%%%%%%%%%%%%%%%%%%%%%%%%%%%%%%%%%%%%%%%



\subsubsection*{Problem 3.24.*}
\addcontentsline{toc}{subsubsection}{Problem 3.24.*}
\emph{The object of this problem is
to find a property of the local ring $\mathscr{O}_P(f)$ that
determines whether or not $P$ is an ordinary multiple point on $f$.
Let $f$ be an irreducible plane curve, $P = (0, 0)$, $\mathfrak{m} = \mathfrak{m}_P(f) > 1$.
Let $m = m_P(f)$.
For $g \in k[x,y]$ or $\in \Gamma(f)$,
denote its residue in $\mathfrak{m}/\mathfrak{m}^2$ by $\overline{g}$.}

\begin{enumerate}
\item[(a)]
  \emph{Show that the map from $V = \{ \text{forms of degree $1$ in $k[x,y]$} \}$ to
  $\mathfrak{m}/\mathfrak{m}^2$ taking $ax+by$ to $\overline{ax+by}$
  is an isomorphism of vector spaces (see Problem 3.13). }

\item[(b)]
  \emph{Suppose $P$ is an ordinary multiple point, with tangents $L_1, \ldots, L_m$.
  Show that $I(P, f \cap L_i) > m$ and
  $\overline{L_i} \neq \lambda \overline{L_j}$ for all $i \neq j$, and all $\lambda \in k$.}

\item[(c)]
  \emph{Suppose there are $g_1, \ldots, g_m \in k[x,y]$
  such that $I(P, f \cap g_i) > m$ and
  $\overline{g_i} \neq \lambda \overline{g_j}$ for all $i \neq j$, and all $\lambda \in k$.
  Show that $P$ is an ordinary multiple point on $f$.
  (Hint: Write $g_i = L_i + \text{higher terms} \in k[x,y]$.
  $\overline{L_i} = \overline{g_i} \neq 0$,
  and $L_i$ is the tangent to $g_i$,
  so $L_i$ is tangent to $f$ by Property (5) of intersection numbers.
  Thus $f$ has $m$ tangents at $P$.)}

\item[(d)]
  \emph{Show that $P$ is an ordinary multiple point on $f$ if and only if
  there are $g_1, \ldots, g_m \in \mathfrak{m}$ such that
  $\overline{g_i} \neq \lambda \overline{g_j}$ for all $i \neq j$, $\lambda \in k$,
  and} \\
  \[
    \dim_k \mathscr{O}_P(f)/(g_i) > m.
  \]
\end{enumerate}



\emph{Proof of (a).}
\begin{enumerate}
\item[(1)]
  $\mathscr{B} = \{ x, y \}$ is a basis for $V$ as a $k$-vector space.

\item[(2)]
  $\mathscr{B}' = \{ \overline{x}, \overline{y} \}$ is a basis for $\mathfrak{m}/\mathfrak{m}^2$.
  as a $k$-vector space.

\item[(3)]
  By (1)(2), we can define a canonical isomorphism
  \[
    \alpha: V \to \mathfrak{m}/\mathfrak{m}^2
  \]
  by sending $\mathscr{B}$ to $\mathscr{B}'$, that is,
  $\alpha(x) = \overline{x}$ and $\alpha(y) = \overline{y}$.
\end{enumerate}
$\Box$ \\



\emph{Proof of (b).}
\begin{enumerate}
\item[(1)]
  Write
  \[
    f = \prod_{i=1}^{m} L_i + \text{higher terms}.
  \]
  Problem 3.19 says that $I(P, f \cap L_i) > m_P(f) = m$.

\item[(2)]
  Since $P$ is an ordinary multiple point on $f$,
  $L_i$ and $L_j$ are linearly independent in $V$ in the sense of (a).
  Hence, $\alpha(L_i) = \overline{L_i}$ and $\alpha(L_j) = \overline{L_j}$
  are linearly independent in $\mathfrak{m}/\mathfrak{m}^2$
  (since $\alpha$ is an isomorphism).
  The conclusion holds.
\end{enumerate}
$\Box$ \\



\emph{Proof of (c).}
\begin{enumerate}
\item[(1)]
  Write $g_i = L_i + \text{higher terms} \in k[x,y]$.
  Here $g_i$ has no constant term since $P \in g_i$ by the defintion of intersection numbers.

\item[(2)]
  Pick $\lambda = 0 \in k$ and $j \neq i$.
  ($m > 1$ implies the existence of $j$.)
  So
  \[
    \alpha(L_i) = \overline{L_i} = \overline{g_i} \neq \lambda \overline{g_j} = 0
  \]
  or $L_i \neq 0$ (since $\alpha$ is an isomorphism).

\item[(3)]
  Hence, $L_i$ is the tangent to $g_i$.
  So Property (5) implies that
  \[
    I(P, f \cap g_i)
    \geq m_P(f) m_P(g_i)
    = m.
  \]
  Note that $L_i$ is the only tangent line of $g_i$.
  By Property (5),
  the assumption $I(P, f \cap g_i) > m$ implies that $L_i$ is tangent to $f$.

\item[(4)]
  Note that
  $\overline{g_i} \neq \lambda \overline{g_j}$ for all $i \neq j$, and all $\lambda \in k$.
  So $\overline{L_i} \neq \lambda \overline{L_j}$ for all $i \neq j$, and all $\lambda \in k$.
  Since $\alpha$ is an isomorphism, all $L_i$ are linearly independent
  and thus $f$ has $m$ tangents $L_1, \ldots, L_m$ at $P$.
  Therefore, $P$ is an ordinary multiple point.
\end{enumerate}
$\Box$ \\



\emph{Proof of (d).}
\begin{enumerate}
\item[(1)]
  Note that
  \[
    \dim_k \mathscr{O}_P(f)/(g_i)
    = \dim_k \mathscr{O}_P(\mathbf{A}^2)/(f, g_i)
    = I(P, f \cap g_i)
  \]
  (by Problem 2.44).

\item[(2)]
  ($\Longrightarrow$)
  Suppose that $P$ is an ordinary multiple point, with tangents $L_1, \ldots, L_m$.
  By (b),
  \[
    \dim_k \mathscr{O}_P(f)/(L_i) = I(P, f \cap L_i) > m
  \]
  and
  $\overline{L_i} \neq \lambda \overline{L_j}$ for all $i \neq j$, and all $\lambda \in k$.
  Take
  \[
    g_i = L_i + I(f) \in \Gamma(f) \subseteq \mathscr{O}_P(f).
  \]
  Since $g_i \in \mathfrak{m}$ (by $L_i(P) = 0$) and $\overline{g_i} = \overline{L_i}$,
  the conclusion is proved.

\item[(3)]
  ($\Longleftarrow$)
  Suppose that there are $g_1, \ldots, g_m \in \mathfrak{m}$ such that
  $\overline{g_i} \neq \lambda \overline{g_j}$,
  and $\dim_k \mathscr{O}_P(f)/(g_i) > m$.
  For each $i = 1, \ldots, m$,
  we take $g'_i + I(f) = g_i \in \mathfrak{m}$ for some $g'_i \in k[x,y]$.
  Although $g'_i$ is not uniquely determined by $g_i$,
  $\overline{g'_i} = \overline{g_i}$ and thus $\overline{g'_i} \neq \lambda \overline{g'_j}$.
  By (1),
  \[
    \dim_k \mathscr{O}_P(f)/(g_i)
    = \dim_k \mathscr{O}_P(f)/(g_i')
    = I(P, f \cap g_i') > m
  \]
  Hence, by (c) $f$ has $m$ tangents at $P$.
\end{enumerate}
$\Box$ \\\\



%%%%%%%%%%%%%%%%%%%%%%%%%%%%%%%%%%%%%%%%%%%%%%%%%%%%%%%%%%%%%%%%%%%%%%%%%%%%%%%%
%%%%%%%%%%%%%%%%%%%%%%%%%%%%%%%%%%%%%%%%%%%%%%%%%%%%%%%%%%%%%%%%%%%%%%%%%%%%%%%%
%%%%%%%%%%%%%%%%%%%%%%%%%%%%%%%%%%%%%%%%%%%%%%%%%%%%%%%%%%%%%%%%%%%%%%%%%%%%%%%%
%%%%%%%%%%%%%%%%%%%%%%%%%%%%%%%%%%%%%%%%%%%%%%%%%%%%%%%%%%%%%%%%%%%%%%%%%%%%%%%%



\newpage
\section*{Chapter 4: Projective Varieties \\}
\addcontentsline{toc}{section}{Chapter 4: Projective Varieties}



\subsection*{4.1. Projective Space \\}
\addcontentsline{toc}{subsection}{4.1. Projective Space}



\subsubsection*{Problem 4.1.}
\addcontentsline{toc}{subsubsection}{Problem 4.1.}
\emph{What points in $\mathbf{P}^{2}$ do not belong to two of the three sets $U_1$, $U_2$, $U_3$?} \\



\emph{Proof.}
\begin{enumerate}
\item[(1)]
  The point $[1:0:0]$ does not belong to $U_2$ and $U_3$.

\item[(2)]
  The point $[0:1:0]$ does not belong to $U_3$ and $U_1$.

\item[(3)]
  The point $[0:0:1]$ does not belong to $U_1$ and $U_2$.
\end{enumerate}
$\Box$ \\\\



%%%%%%%%%%%%%%%%%%%%%%%%%%%%%%%%%%%%%%%%%%%%%%%%%%%%%%%%%%%%%%%%%%%%%%%%%%%%%%%%



\subsubsection*{Problem 4.2.*}
\addcontentsline{toc}{subsubsection}{Problem 4.2.*}
\emph{Let $f \in k[x_1,\ldots,x_{n+1}]$ ($k$ infinite).
Write $f = \sum f_i$, $f_i$ a form of degree $i$.
Let $P \in \mathbf{P}^{n}(k)$, and suppose $f(x_1,\ldots,x_{n+1}) = 0$
for every choice of homogeneous coordinates $(x_1,\ldots,x_{n+1})$ for $P$.
Show that each $f_i(x_1,\ldots,x_{n+1}) = 0$ for all homogeneous coordinates for $P$.
(Hint: consider
\[
  g(\lambda) = f(\lambda x_1, \ldots, \lambda x_{n+1}) = \sum \lambda^i f_i(x_1,\ldots,x_{n+1})
\]
for fixed $(x_1,\ldots,x_{n+1})$.)} \\



\emph{Proof.}
\begin{enumerate}
\item[(1)]
  Consider
  \[
    g(\lambda)
    = f(\lambda x_1, \ldots, \lambda x_{n+1})
    = \sum \lambda^i f_i(x_1,\ldots,x_{n+1})
  \]
  for fixed $(x_1,\ldots,x_{n+1})$.
  $g(\lambda)$ is a polynomial in $k[\lambda]$.

\item[(2)]
  Since $g(\lambda) = f(\lambda x_1, \ldots, \lambda x_{n+1}) = 0$ for all $\lambda \in k - \{ 0 \}$,
  $g(\lambda) = 0$ has infinitely many solutions in $k$.
  Similar to Problem 1.4, $g = 0 \in k[\lambda]$,
  that is,
  each $f_i(x_1,\ldots,x_{n+1}) = 0$ for all homogeneous coordinates for $P$.
\end{enumerate}
$\Box$ \\\\



%%%%%%%%%%%%%%%%%%%%%%%%%%%%%%%%%%%%%%%%%%%%%%%%%%%%%%%%%%%%%%%%%%%%%%%%%%%%%%%%



\subsubsection*{Problem 4.3.}
\addcontentsline{toc}{subsubsection}{Problem 4.3.}
\begin{enumerate}
\item[(a)]
  \emph{Show that the definitions of this section carry over without change to the
  case where $k$ is an arbitrary field.}

\item[(b)]
  \emph{If $k_0$ is a subfield of $k$, show that $\mathbf{P}^{n}(k_0)$
  may be identified with a subset of $\mathbf{P}^{n}(k)$.} \\
\end{enumerate}



\emph{Proof of (a).}
Note that a field is a commutative ring where $0 \neq 1$ and all nonzero elements are invertible.
Hence the definitions in this section are well-defined for any field $k$.
$\Box$ \\



\emph{Proof of (b).}
Note that $0 \in k_0$ and $0 \in k$.
So any point $P \in \mathbf{P}^{n}(k_0)$ is also in $\mathbf{P}^{n}(k)$
since $P \neq (0, \ldots, 0)$
and
\[
  \{ (\lambda x_1, \ldots, \lambda x_{n+1}) : \lambda \in k_0 \}
  \subseteq
  \{ (\lambda x_1, \ldots, \lambda x_{n+1}) : \lambda \in k \}
\]
as a subset.
$\Box$ \\\\



%%%%%%%%%%%%%%%%%%%%%%%%%%%%%%%%%%%%%%%%%%%%%%%%%%%%%%%%%%%%%%%%%%%%%%%%%%%%%%%%
%%%%%%%%%%%%%%%%%%%%%%%%%%%%%%%%%%%%%%%%%%%%%%%%%%%%%%%%%%%%%%%%%%%%%%%%%%%%%%%%



\subsection*{4.2. Projective Algebraic Sets \\}
\addcontentsline{toc}{subsection}{4.2. Projective Algebraic Sets}



\subsubsection*{Problem 4.4.*}
\addcontentsline{toc}{subsubsection}{Problem 4.4.*}
\emph{Let $I$ be a homogeneous ideal in $k[x_1,\ldots,x_{n+1}]$.
Show that $I$ is prime if and only if the following condition is satisfied:
for any forms $f, g \in k[x_1,\ldots,x_{n+1}]$,
if $fg \in I$, then $f \in I$ or $g \in I$.} \\



\emph{Proof.}
\begin{enumerate}
\item[(1)]
  ($\Longrightarrow$) Trivial.

\item[(2)]
  ($\Longleftarrow$) Suppose that $f, g \in k[x_1,\ldots,x_{n+1}]$ and $fg \in I$.
  Write $f = \sum_{i=0}^{r} f_i$ (resp. $g = \sum_{j=0}^{s} g_j$),
  $f_i$ a form of degree $i$ (resp. $g_j$ a form of degree $j$).
  Induction on the $\deg(fg) = r + s$.

\item[(3)]
  When $r + s = 0$, nothing to do.

\item[(4)]
  Assume that the result is true for smaller values of $r + s$.
  Then the highest homogeneous component $f_r g_s$ of $fg$ is also in $I$
  since $I$ is homogeneous.
  By assumption, $f_r \in I$ or $g_s \in I$, and might say that $f_r \in I$.
  Therefore,
  \[
    (f - f_r)g \in I.
  \]
  By the induction hypothesis, $f - f_r \in I$ or $g \in I$.
  Hence, $f = (f - f_r) + f_r \in I$ or $g \in I$.

\item[(5)]
  Therefore, (3)(4) implies that $I$ is prime.
\end{enumerate}
$\Box$ \\\\



%%%%%%%%%%%%%%%%%%%%%%%%%%%%%%%%%%%%%%%%%%%%%%%%%%%%%%%%%%%%%%%%%%%%%%%%%%%%%%%%



\subsubsection*{Problem 4.5.}
\addcontentsline{toc}{subsubsection}{Problem 4.5.}
\emph{If $I$ is a homogeneous ideal, show that $\mathrm{rad}(I)$ is also homogeneous.} \\



\emph{Proof.}
\begin{enumerate}
\item[(1)]
  Given any $f = \sum_{i=0}^{r} f_i \in \mathrm{rad}(I)$, $f_i$ a form of degree $i$.
  It suffices to show that each $f_i \in \mathrm{rad}(I)$.
  Note that $f^{m} \in I$ for some $m > 0$.

\item[(2)]
  The highest homogeneous component ${f_r}^m$ of $f^{m}$ is also in $I$
  since $I$ is homogeneous.
  Hence, $f_r \in \mathrm{rad}(I)$.
  Again note that $f - f_r \in \mathrm{rad}(I)$ and $\deg(f - f_r) < r$.
  Continue this process (or by induction), and we have
  $f_{r-1}, \ldots, f_0 \in \mathrm{rad}(I)$.
\end{enumerate}
$\Box$ \\\\



%%%%%%%%%%%%%%%%%%%%%%%%%%%%%%%%%%%%%%%%%%%%%%%%%%%%%%%%%%%%%%%%%%%%%%%%%%%%%%%%



\subsubsection*{Problem 4.6.}
\addcontentsline{toc}{subsubsection}{Problem 4.6.}
\emph{State and prove the projective analogues of properties (1)-(10) of Chapter 1,
Sections 2 and 3.} \\

\emph{Statements.}
\begin{enumerate}
\item[(1)]
  \emph{If $I$ is the homogeneous ideal in $k[x_1, \ldots, x_{n+1}]$ generated by a set of forms $S$,
  then $V(S) = V(I)$; so every algebraic set is equal to $V(I)$ for some homogeneous ideal $I$.}

\item[(2)]
  \emph{If $\{ I_\alpha \}$ is any collection of homogeneous ideals,
  then $V(\cup_{\alpha} I_\alpha) = \cap_{\alpha} V(I_\alpha)$;
  so the intersection of any collection of algebraic sets is an algebraic set.}

\item[(3)]
  \emph{If $I \subseteq J$, then $V(I) \supseteq V(J)$.}

\item[(4)]
  \emph{$V(fg) = V(f) \cup V(g)$ for any forms $f$, $g$;
  $V(I) \cup V(J) = V(\{ fg : f \in I, g \in J \})$;
  so any finite union of algebraic sets is an algebraic set.}

\item[(5)]
  \emph{$V(0) = \mathbf{P}^{n}(k)$;
  $V(I) = \varnothing$ if $I$ contains all forms of degree $\geq N$ for some $N$;
  $V(a_i x_j - a_j x_i) = \{ [a_1 : \cdots : a_{n+1}] \}$
  for $[a_1 : \cdots : a_{n+1}] \in \mathbf{P}^{n}(k)$.
  So any finite subset of $\mathbf{P}^{n}(k)$ is an algebraic set.}

\item[(6)]
  \emph{If $X \subseteq Y$ are nonempty, then $I(X) \supseteq I(Y)$.}

\item[(7)]
  \emph{$I(\varnothing) = k[x_1, \ldots, x_{n+1}]$;
  $I(\mathbf{P}^{n}(k)) = (0)$ if $k$ is an infinite field;}

\item[(8)]
  \emph{$I(V(S)) \supseteq S$ for any set $S$ of forms;
  $V(I(X)) \supseteq X$ for any set $X$ of points.}

\item[(9)]
  \emph{$V(I(V(S))) = V(S)$ for any set $S$ of forms,
  and $I(V(I(X))) = I(X)$ for any set $X$ of points.
  So if $V$ is an algebraic set, $V = V(I(V))$,
  and if $I$ is the homogeneous ideal of an algebraic set, $I = I(V(I))$.}

\item[(10)]
  \emph{$I(X)$ is a radical ideal for any nonempty $X \subseteq \mathbf{P}^{n}(k)$.} \\
\end{enumerate}



\emph{Proof.}
  Proposition 1 and the projective Nullstellensatz give all.
$\Box$ \\\\



%%%%%%%%%%%%%%%%%%%%%%%%%%%%%%%%%%%%%%%%%%%%%%%%%%%%%%%%%%%%%%%%%%%%%%%%%%%%%%%%



\subsubsection*{Problem 4.7.}
\addcontentsline{toc}{subsubsection}{Problem 4.7.}
\emph{Show that each irreducible component of a cone is also a cone.} \\



\emph{Proof.}
\begin{enumerate}
\item[(1)]
  Let $V$ is an algebraic set in $\mathbf{P}^n$, and $C(V)$ be the cone over $V$.
  Might assume that $V \neq \varnothing$.

\item[(2)]
  \emph{Show that $V$ is irreducible if and only if $C(V)$ is irreducible.}
  $V$ is irreducible
  if and only if $I_p(V) = I_a(C(V))$ is prime
  if and only if $C(V)$ is irreducible.

\item[(3)]
  Let $V = V_1 \cup \cdots \cup V_r$ be the decomposition of an algebraic set
  into irreducible components.
  Note that
  \[
    C(V_1 \cup \cdots \cup V_r) = C(V_1) \cup \cdots \cup C(V_r)
  \]
  (by the definition of the cone).
  Here each $C(V_i)$ is irreducible (by (2)).
  By Theorem 2 in \S 1.5, each irreducible component of $C(V)$
  must be one of $C(V_i)$, which is also a cone.
\end{enumerate}
$\Box$ \\\\



%%%%%%%%%%%%%%%%%%%%%%%%%%%%%%%%%%%%%%%%%%%%%%%%%%%%%%%%%%%%%%%%%%%%%%%%%%%%%%%%



\subsubsection*{Problem 4.8.}
\addcontentsline{toc}{subsubsection}{Problem 4.8.}
\emph{Let $V = \mathbf{P}^{1}$, $\Gamma_h(V) = k[x,y]$.
Let $t = x/y \in k(V)$, and show that $k(V) = k(t)$.
Show that there is a natural one-to-one correspondence between the points of $\mathbf{P}^{1}$
and the DVR’s with quotient field $k(V)$ that contain $k$ (see Problem 2.27);
which DVR corresponds to the point at infinity?} \\



\emph{Proof.}
\begin{enumerate}
\item[(1)]
  \emph{Show that $k(V) = k(t)$.}
  Given any $f/g \in k(V)$ where $f, g \in \Gamma_h(V)$ are of the same degree.
  Then
  \[
    f(x,y)/g(x,y) = f(t,1)/g(t,1) \in k(t).
  \]
  Conversely, given any $f/g \in k(t)$,
  \[
    \frac{f(t)}{g(t)} = \frac{f(x/y)}{g(x/y)} = \frac{y^d f(x/y)}{y^d g(x/y)} \in k(V)
  \]
  where $d = \max\{ \deg(f), \deg(g) \}$.

\item[(2)]
  Note that $k = \overline{k}$.
  By Problem 2.27,
  the DVR's with quotient field $k(V) = k(t)$ are
  \[
    \text{$\mathscr{O}_a(\mathbf{A}^1)$ where $a \in \mathbf{A}^1 = k$ and
    $\mathscr{O}_{\infty}(\mathbf{A}^1)$},
  \]
  which correspond to
  \[
    \mathbf{P}^1(k) = \{ [a:1] : a \in k \} \cup \{ [1:0] \}.
  \]
  In particular, the DVR $\mathscr{O}_{\infty}(\mathbf{A}^1)$ corresponds
  to the point at infinity ($= [1:0]$).
\end{enumerate}
$\Box$ \\\\



%%%%%%%%%%%%%%%%%%%%%%%%%%%%%%%%%%%%%%%%%%%%%%%%%%%%%%%%%%%%%%%%%%%%%%%%%%%%%%%%



\subsubsection*{Problem 4.9.*}
\addcontentsline{toc}{subsubsection}{Problem 4.9.*}
\emph{Let $I$ be a homogeneous ideal in $k[x_1, \ldots, x_{n+1}]$,
and
\[
  \Gamma = k[x_1, \ldots, x_{n+1}]/I.
\]
Show that the forms of degree $d$ in $\Gamma$ form a finite-dimensional vector space over $k$.} \\



\emph{Proof.}
\begin{enumerate}
\item[(1)]
  Write $R = k[x_1, \ldots, x_{n+1}]$.
  For $R$ (resp. $\Gamma$),
  define $R_{(d)}$ (resp. $\Gamma_{(d)}$) to be the corresponding homogeneous component of degree $d$.
  Consider a natural homomorphism
  \[
    \alpha: R_{(d)} \to R_{(d)}/I \cong \Gamma_{(d)}
  \]
  by $\alpha(h) = \overline{h}$ for any form $h$ of degree $d$.

\item[(2)]
  $\Gamma_{(d)}$ can be regarded as a subspace of $R_{(d)}$ since $\alpha$ is surjective.
  Since $R_{(d)}$ is finite-dimensional with $\dim R_{(d)} = {d+n-1 \choose n-1}$,
  $\Gamma_{(d)}$ is finite-dimensional by linear algebra.
\end{enumerate}
$\Box$ \\\\



%%%%%%%%%%%%%%%%%%%%%%%%%%%%%%%%%%%%%%%%%%%%%%%%%%%%%%%%%%%%%%%%%%%%%%%%%%%%%%%%



\subsubsection*{Problem 4.10.}
\addcontentsline{toc}{subsubsection}{Problem 4.10.}
\emph{Let $R = k[x,y,z]$, $f \in R$ an irreducible form of degree $n$,
$V = V(f) \subseteq \mathbf{P}^2$, and $\Gamma = \Gamma_h(V)$.}
\begin{enumerate}
\item[(a)]
  \emph{Construct an exact sequence
  \[
    0
    \to R
    \xrightarrow{\psi} R
    \xrightarrow{\varphi} \Gamma
    \to 0
  \]
  where $\psi$ is multiplication by $f$.}

\item[(b)]
  \emph{Show that
  \[
    \dim_k\{ \text{forms of degree $d$ in $\Gamma$} \} = dn - \frac{n(n-3)}{2}
  \]
  if $d > n$.} \\
\end{enumerate}



\emph{Proof of (a).}
\begin{enumerate}
\item[(1)]
  $\psi$ is defined by $\psi(g) = fg$ and
  $\varphi$ is naturally defined by $\varphi(h) = \overline{h}$.
  $\psi$ and $\varphi$ are well-defined and homomorphisms of vector space.

\item[(2)]
  $\mathrm{im}(\psi) = \ker(\varphi) = (f) = I(f)$
  (since $f$ is irreducible).
  $\psi$ is injective since $R = k[x,y,z]$ is a domain.
  $\varphi$ is surjective trivially.
  Hence, the sequence of vector spaces over $k$ is exact.
\end{enumerate}
$\Box$ \\



\emph{Proof of (b).}
\begin{enumerate}
\item[(1)]
  The exact sequence
  \[
    0
    \to R
    \xrightarrow{\psi} R
    \xrightarrow{\varphi} \Gamma
    \to 0
  \]
  induces the exact sequence
  \[
    0
    \to R_{(d-n)}
    \xrightarrow{\psi} R_{(d)}
    \xrightarrow{\varphi} \Gamma_{(d)}
    \to 0,
  \]
  where $*_{(d)}$ (resp. $*_{(d-n)}$) denotes
  the corresponding homogeneous component of degree $d$ (resp. $d-n$).

\item[(2)]
  By Problem 2.36,
  \[
    \dim_k R_{(d-n)} = {d-n+2 \choose 2},
    \qquad
    \dim_k R_{(d)} = {d+2 \choose 2}.
  \]

\item[(3)]
  Since $R_{(d)}$ is finite-dimensional, $\Gamma_{(d)}$ is also finite-dimensional
  (by regarding $\Gamma_{(d)}$ as a subspace of $R_{(d)}$).
  Proposition 7 in \S 2.9 shows that
  \begin{align*}
    \dim_k \Gamma_{(d)}
    &= \dim_k R_{(d)} - \dim_k R_{(d-n)} \\
    &= {d+2 \choose 2} - {d-n+2 \choose 2} \\
    &= dn - \frac{n^2}{2} + \frac{3n}{2}.
  \end{align*}
\end{enumerate}
$\Box$ \\\\



%%%%%%%%%%%%%%%%%%%%%%%%%%%%%%%%%%%%%%%%%%%%%%%%%%%%%%%%%%%%%%%%%%%%%%%%%%%%%%%%



\subsubsection*{Problem 4.11.* (Linear subvariety)}
\addcontentsline{toc}{subsubsection}{Problem 4.11.* (Linear subvariety)}
\emph{A set $V \subseteq \mathbf{P}^{n}(k)$
is called a \textbf{linear subvariety} of $\mathbf{P}^{n}(k)$ if
$V = V(h_1, \cdots, h_r)$, where each $h_i$ is a form of degree $1$.}
\begin{enumerate}
\item[(a)]
  \emph{Show that if $t$ is a projective change of coordinates,
  then $V^{t} = t^{-1}(V)$ is also a linear subvariety.}

\item[(b)]
  \emph{Show that there is a projective change of coordinates $t$ of $\mathbf{P}^{n}$ such that
  $V^t = V(x_{m+2}, \cdots, x_{n+1})$, so $V$ is a variety.}

\item[(c)]
  \emph{Show that the $m$ that appears in part (b) is independent of the choice of $t$.
  It is called the \textbf{dimension} of $V$ ($m = -1$ if $V = \varnothing$).} \\
\end{enumerate}



\emph{Proof of (a).}
\begin{enumerate}
\item[(1)]
  Say $t = (t_1,\ldots,t_{n+1})$ is a projective change of coordinates,
  and $V = V(h_1, \cdots, h_r)$, where each $h_i$ is a form of degree $1$.

\item[(2)]
  \emph{Show that $V$ is a variety and thus $I(V) = (h_1, \ldots, h_r)$
  by the projective Nullstellensatz.}
  $V$ is the set of all non-trivial solutions of the system of linear equations:
  \begin{align*}
    h_1 &= a_{1,1} x_1 + \cdots + a_{1,n+1} x_{n+1} = 0, \\
    &\cdots \\
    h_r &= a_{r,1} x_1 + \cdots + a_{r,n+1} x_{n+1} = 0.
  \end{align*}
  (Here we identify $[x_1 : \cdots : x_{n+1}] \in \mathbf{P}^{n}$.)
  Write $Ax = 0$ and thus $V = V(Ax = 0)$, where
  \[
    A =
    \underbrace{\begin{pmatrix}
    a_{1,1} & \cdots & a_{1,n+1} \\
    \vdots & \ddots & \vdots \\
    a_{r,1} & \cdots & a_{r,n+1}
    \end{pmatrix}}_{\in \mathsf{M}_{r \times (n+1)}(k)},
    \qquad
    x =
    \underbrace{\begin{pmatrix}
    x_1 \\
    \vdots \\
    x_{n+1}
    \end{pmatrix}}_{\in \mathsf{M}_{(n+1) \times 1}(k)}.
  \]

\item[(3)]
  The Gaussian elimination in linear algebra says that
  $(A|0)$ has the same solutions as its reduced row echelon form $(A'|0)$,
  that is, $V(Ax = 0) = V(A'x = 0)$.

\item[(4)]
  If $V(h_1, \ldots, h_r) = \varnothing$, nothing to do.
  If $V(h_1, \ldots, h_r) \neq \varnothing$, then
  \[
    V(h_1, \ldots, h_r) = V(g_1, \ldots, g_{m+1})
  \]
  where $m+1 = \mathrm{rank}(A)$ is the number of nonzero rows in $A'$ ($m+1 \leq r, n+1$)
  and $g_i = a'_{i,1} x_1 + \cdots + a'_{i,n+1} x_{n+1}$ for $1 \leq i \leq m+1$.
  ($a'_{i,j}$ is the entry of the matrix $A'$.)

\item[(5)]
  Now given any $f + I(V) \in k[x_1,\ldots,x_{n+1}]/I(V)$,
  we replace the leading term $x_{i_1}$ of $g_1$ by $x_{i_1} - g_1$ to get
  \[
    f + I(V)
    = \: f(x_1, \cdots,
      \underbrace{x_{i_1} - g_1}_{\text{$i_1$th position}}, \cdots, x_{n+1}) + I(V)
    := \: f_1 + I(V)
  \]
  where $f_1 \in k[x_1,\ldots,\widehat{x_{i_1}}\ldots,x_{n+1}]$.
  Continue this process to replace each leading term $x_{i_j}$ of $g_j$ by $x_{i_j} - g_j$ to get
  one by one to get
  \begin{align*}
    f + I(V) &= f_1 + I(V),
      f_1 \in k[x_1,\ldots,\widehat{x_{i_1}}\ldots,x_{n+1}]. \\
    & \cdots \\
    f_{m} + I(V) &= f_{m+1} + I(V),
      f_{m+1} \in k[x_1,\ldots,\widehat{x_{i_1}},\ldots,\widehat{x_{i_{m+1}}}\ldots,x_{n+1}].
  \end{align*}
  Hence, a routine shows that there is a ring isomorphism
  \[
    \alpha: k[x_1,\ldots,x_{n+1}]/I(V) \to
    \underbrace{
      k[x_1,\ldots,\widehat{x_{i_1}},\ldots,\widehat{x_{i_{m+1}}}\ldots,x_{n+1}]
    }_{\text{a domain}}
  \]
  sending $f$ to $f_{m+1}$.
  Therefore, $V$ is a variety.

\item[(6)]
  As $I(V) = (h_1, \ldots, h_r)$,
  $I(V)^{t} = (h_1^{t}, \ldots, h_r^{t})$ where each $h_i^{t}$ is a form of degree $1$.
  Thus $V^{t} = V(I(V)^{t}) = V(h_1^{t}, \ldots, h_r^{t})$
  is also a linear subvariety of $\mathbf{P}^n(k)$.
\end{enumerate}
$\Box$ \\



\emph{Proof of (b).}
\begin{enumerate}
\item[(1)]
  Suppose $A \in \mathsf{M}_{r \times (n+1)}(k)$ is of rank $(n+1) - (m+1) = n-m$.
  Linear algebra says that there exist invertible matrices
  $B \in \mathsf{M}_{r \times r}(k)$ and $C \in \mathsf{M}_{(n+1) \times (n+1)}(k)$
  such that $D = BAC$, where
  \[
    D = BAC =
    \underbrace{\begin{pmatrix}
    O_1 & O_2 \\
    O_3 & I_{n-m}
    \end{pmatrix}}_{\in \mathsf{M}_{r \times (n+1)}(k)}
  \]
  in which $I_{n-m} \in \mathsf{M}_{(n-m) \times (n-m)}(k)$ is the identity matrix
  and $O_1$, $O_2$, and $O_3$ are zero matrices.

\item[(2)]
  Let $t'$ be the linear map corresponding to the matrix $C$.
  So
  \begin{align*}
    V^{t'}
    &= V(Ax = 0)^{t'} \\
    &= V(ACx = 0) \\
    &= V(BACx = 0)
      &(\text{$B$: invertible}) \\
    &= V(Dx = 0) \\
    &= V(0, \cdots, 0, x_{m+2}, \cdots, x_{n+1})
      &(V \neq \varnothing) \\
    &= V(x_{m+2}, \cdots, x_{n+1}).
  \end{align*}
\end{enumerate}
$\Box$ \\



\emph{Proof of (c).}
  Linear algebra says that
  the rank of any matrix is uniquely determined.
  Therefore, $m = n - \mathrm{rank}(A)$ is uniquely determined.
$\Box$ \\\\



%%%%%%%%%%%%%%%%%%%%%%%%%%%%%%%%%%%%%%%%%%%%%%%%%%%%%%%%%%%%%%%%%%%%%%%%%%%%%%%%



\subsubsection*{Problem 4.12.*}
\addcontentsline{toc}{subsubsection}{Problem 4.12.*}
\emph{Let $H_1, \ldots, H_m$ be hyperplanes in $\mathbf{P}^n$, $m \leq n$.
Show that} \\
\[
  H_1 \cap H_2 \cap \cdots \cap H_m \neq \varnothing.
\]


\emph{Proof.}
\begin{enumerate}
\item[(1)]
  Let
  \[
    H_i: a_{i,1} x_1 + \cdots + a_{i,n+1} x_{n+1} = 0
  \]
  for $i = 1, 2, \ldots, m$.

\item[(2)]
  View (1) as the system of linear equations.
  Let the coefficient matrix $A$ be
  \[
    A
    =
    \begin{pmatrix}
      a_{1,1} & a_{1,2} & \cdots & a_{1,n+1} \\
      a_{2,1} & a_{2,2} & \cdots & a_{2,n+1} \\
      \vdots & \vdots & \ddots & \vdots \\
      a_{m,1} & a_{m,2} & \cdots & a_{m,n+1}
    \end{pmatrix}.
  \]
  Note that $\rank(A) \leq \min\{ m, n+1 \} = m$.
  The rank-nullity theorem shows that
  \[
    \dim_k \ker(A) = (n+1) - \rank(A) \geq (n+1)-m \geq 1.
  \]
  Hence, there is a nonzero solution of $\bigcap_{i=1}^{m} H_i$,
  or $\bigcap_{i=1}^{m} H_i \neq \varnothing \in \mathbf{P}^n$.
\end{enumerate}
$\Box$ \\\\



%%%%%%%%%%%%%%%%%%%%%%%%%%%%%%%%%%%%%%%%%%%%%%%%%%%%%%%%%%%%%%%%%%%%%%%%%%%%%%%%



\subsubsection*{Problem 4.13.* (Line)}
\addcontentsline{toc}{subsubsection}{Problem 4.13.* (Line)}
\emph{Let $P = [a_1 : \cdots : a_{n+1}]$, $Q = [b_1 : \cdots : b_{n+1}]$
be distinct points of $\mathbf{P}^n$.
The \textbf{line} $L$ through $P$ and $Q$ is defined by
\[
  L = \{ [ \lambda a_1 + \mu b_1 : \cdots : \lambda a_{n+1} + \mu b_{n+1} ] :
    \lambda, \mu \in k, \lambda \neq 0 \text{ or } \mu \neq 0 \}.
\]
Prove the projective analogue of Problem 2.15.}

\begin{enumerate}
\item[(a)]
  \emph{Show that if $L$ is the line through $P$ and $Q$,
  and $t$ is a projective change of coordinates,
  then $t(L)$ is the line through $t(P)$ and $t(Q)$.}

\item[(b)]
  \emph{Show that a line is a linear subvariety of dimension $1$,
  and that a linear subvariety of dimension $1$ is the line through any two of its points.}

\item[(c)]
  \emph{Show that, in $\mathbf{P}^{2}$, a line is the same thing as a hyperplane.}

\item[(d)]
  \emph{Let $P, P' \in \mathbf{P}^{2}$, $L_1$, $L_2$ two distinct lines through $P$,
  $L'_1$, $L'_2$ distinct lines through $P'$.
  Show that there is a projective change of coordinates $t$ of $\mathbf{P}^{2}$
  such that $t(P) = P'$ and $t(L_i) = L'_i$, $i = 1, 2$.} \\
\end{enumerate}



\emph{Proof of (a).}
\begin{enumerate}
\item[(1)]
  Write $t = (t_1, \ldots, t_{n+1})$ as
  \[
    t_i = \sum_{j} c_{ij} x_j.
  \]
  Given any point
  $P_{\lambda,\mu}
  = [ \lambda a_1 + \mu b_1 : \cdots : \lambda a_{n+1} + \mu b_{n+1} ] \in L$
  for some not all zeros $\lambda, \mu \in k$.
  (In particular, $P_{1,0} = P$ and $P_{0,1} = Q$.)

\item[(2)]
  As
  \begin{align*}
    t_i(P_{\lambda,\mu})
    =& \: \sum_{j} c_{ij}(\lambda a_j + \mu b_j) \\
    =& \: \lambda \sum_{j} c_{ij}a_j + \mu \sum_{j} c_{ij}b_j \\
    =& \: \lambda t_i(P) + \mu t_i(Q),
  \end{align*}
  we have
  \begin{align*}
    t(L) = &\: \{ [ \lambda t_1(P) + \mu t_1(Q) : \cdots :
        \lambda t_{n+1}(P) + \mu t_{n+1}(Q) ] \\
      & \qquad : \lambda, \mu \in k, \lambda \neq 0 \text{ or } \mu \neq 0 \}.
  \end{align*}
  Moreover, $t(P) \in t(L)$ as $(\lambda,\mu) = (1,0)$,
  $t(Q) \in t(L)$ as $(\lambda,\mu) = (0,1)$,
  and $t(P) \neq t(Q)$ (since $P \neq Q$ and $t$ is a projective change of coordinates.)
  Therefore, $t(L)$ is the line through $t(P)$ and $t(Q)$.
\end{enumerate}
$\Box$ \\



\emph{Proof of (b).}
\begin{enumerate}
\item[(1)]
  First, write $L$ as the system of equations
  \[
    x_i = \lambda a_i + \mu b_i
  \]
  ($i = 1, \ldots, n+1$)
  where $\lambda, \mu \in k, \lambda \neq 0 \text{ or } \mu \neq 0$.
  Since $P \neq Q \in \mathbf{P}^n$, there exist $1 \leq \alpha, \beta \leq n$
  such that $a_\alpha b_\beta - a_\beta b_\alpha \neq 0$.
  So we can solve $\lambda$ and $\mu$ in terms of $x_\alpha$ and $x_\beta$ by Cramer's rule,
  say
  \[
    \lambda = \frac{x_\alpha b_\beta - x_\beta b_\alpha}{a_\alpha b_\beta - a_\beta b_\alpha},
    \qquad
    \mu = \frac{a_\alpha x_\beta - a_\beta x_\alpha}{a_\alpha b_\beta - a_\beta b_\alpha}.
  \]

\item[(2)]
  Define
  \begin{align*}
    V
    &= V\left(
      x_i = \frac{x_\alpha b_\beta - x_\beta b_\alpha}{a_\alpha b_\beta - a_\beta b_\alpha} a_i
        + \frac{a_\alpha x_\beta - a_\beta x_\alpha}{a_\alpha b_\beta - a_\beta b_\alpha} b_i
      : 1 \leq i \leq n+1 \right) \\
    &= V\left(
        \begin{vmatrix}
          x_i      & a_i      & b_i      \\
          x_\alpha & a_\alpha & b_\alpha \\
          x_\beta  & a_\beta  & b_\beta
        \end{vmatrix} = 0 : 1 \leq i \leq n+1
      \right).
  \end{align*}
  By construction, $L = V$ is a linear subvariety in $\mathbf{P}^n$. (See Problem 4.11.)

\item[(3)]
  Might assume that $\alpha = n$ and $\beta = n+1$.
  View
  \[
    \begin{vmatrix}
      x_i      & a_i      & b_i      \\
      x_\alpha & a_\alpha & b_\alpha \\
      x_\beta  & a_\beta  & b_\beta
    \end{vmatrix} = 0, i = 1, \ldots, n+1
  \]
  as the system of linear equations.
  Write $A$ as the corresponding coefficient matrix.
  $A$ is a reduced row echelon form of rank $(n+1)-2 = n-1$
  So $\dim(V) = n - \mathrm{rank}(A) = n - (n-1) = 1$.

\item[(4)]
  Conversely, $\dim(V) = 1$ implies that $\mathrm{rank}(A'|0) = n - 1$.
  So all leading terms are all $x_i$ except two $x_\alpha, x_\beta$ for some $\alpha \neq \beta$,
  might say $\alpha = n$ and $\beta = n+1$.
  Hence $V$ is of the form
  \[
    V
    = (x_i + a_{i} x_n + b_{i} x_{n+1} = 0)
  \]
  for $1 \leq i \leq n-1$.
  So
  \begin{align*}
    V
    = & \: \{ [-a_1 \lambda - b_1 \mu : \ldots : -a_{n-1} \lambda - b_{n-1} \mu :
      \lambda : \mu ] : \\
      & \qquad \lambda, \mu \in k, \lambda \neq 0 \text{ or } \mu \neq 0 \}
  \end{align*}
  is a line passing two different points
  \begin{align*}
    P &= [ -a_1 : \ldots : -a_{n-1} : 1 : 0 ] \\
    Q &= [ -b_1 : \ldots : -b_{n-1} : 0 : 1 ].
  \end{align*}
\end{enumerate}
$\Box$ \\



\emph{Proof of (c).}
\begin{enumerate}
\item[(1)]
  By part (b), a line $L \subseteq \mathbf{P}^{2}$ is
  \begin{align*}
    & \: V((a_2 b_3 - b_2 a_3) x + (a_3 b_1 - a_1 b_3) y + (a_1 b_2 - a_2 b_1) z = 0) \\
    = & \: V\left(
      \begin{vmatrix}
        x & a_1 & b_1 \\
        y & a_2 & b_2 \\
        z & a_3 & b_3
      \end{vmatrix} = 0
    \right),
  \end{align*}
  which is also a plane in $\mathbf{P}^{2}$.

\item[(2)]
  Conversely, given any plane
  \[
    V = V(ax + by + cz = 0) \subseteq \mathbf{P}^{2}
  \]
  where $a, b, c$ are not all zero.
  Might assume that $a \neq 0$. (Other cases are similar.)
  So
  \[
    V
    =
    \left\{
      \left[ -\frac{b}{a}\lambda-\frac{c}{a}\mu : \lambda : \mu \right] \in \mathbf{P}^{2}
      : \lambda, \mu \in k, \lambda \neq 0 \text{ or } \mu \neq 0
    \right\}
  \]
  is a line passing
  $P = \left[ -\frac{b}{a} : 1 : 0 \right] \in \mathbf{P}^{2}$ and
  $Q = \left[ -\frac{c}{a} : 0 : 1 \right] \in \mathbf{P}^{2}$.
\end{enumerate}
$\Box$ \\



\emph{Proof of (d).}
\begin{enumerate}
\item[(1)]
  Take one point $P_i \in L_i$ (resp. $P'_i \in L'_i$) other than $P$ (resp. $P'$) for $i = 1, 2$.
  It is possible since every line is passing two distinct points.

\item[(2)]
  By Problem 4.15, $P_1 \not\in L_2$ (resp. $P'_1 \not\in L'_2$) and
  $P_2 \not\in L_1$ (resp. $P'_2 \not\in L'_1$).

\item[(3)]
  By Problem 4.14,
  there is a unique projective change of coordinates $t: \mathbf{P}^2 \to \mathbf{P}^2$
  such that $t(P) = P'$, $t(P_1) = P'_1$ and $t(P_2) = P'_2$.

\item[(4)]
  Hence, the line $t(L_i)$ (by part (a)) and the line $L'_i$ are both passing $P'$ and $P'_i$
  for $i = 1, 2$.
  Since $P' \neq P'_i$ by construction, Problem 4.15 implies that $t(L_i) = L'_i$.
\end{enumerate}
$\Box$ \\\\



%%%%%%%%%%%%%%%%%%%%%%%%%%%%%%%%%%%%%%%%%%%%%%%%%%%%%%%%%%%%%%%%%%%%%%%%%%%%%%%%



\subsubsection*{Problem 4.14.*}
\addcontentsline{toc}{subsubsection}{Problem 4.14.*}
\emph{Let $P_1, P_2, P_3$ (resp. $Q_1, Q_2, Q_3$) be three points in $\mathbf{P}^2$
not lying on a line.
Show that there is a projective change of coordinates
$t: \mathbf{P}^2 \to \mathbf{P}^2$ such that $t(P_i) = Q_i$, $i = 1, 2, 3$.
Extend this to $n+1$ points in $\mathbf{P}^{n}$, not lying on a hyperplane.} \\



\emph{Proof.}
\begin{enumerate}
\item[(1)]
  Write
  \begin{align*}
    P_i &= [a_{i1} : a_{i2} : a_{i3}] \in \mathbf{P}^2(k) \\
    Q_i &= [b_{i1} : b_{i2} : b_{i3}] \in \mathbf{P}^2(k)
  \end{align*}
  for $i = 1, 2, 3$.

\item[(2)]
  Define
  \begin{align*}
    A
    &=
    \begin{pmatrix}
      P_1 & P_2 & P_3
    \end{pmatrix}
    =
    \begin{pmatrix}
      a_{11} & a_{21} & a_{31} \\
      a_{12} & a_{22} & a_{32} \\
      a_{13} & a_{23} & a_{33}
    \end{pmatrix} \\
    B
    &=
    \begin{pmatrix}
      Q_1 & Q_2 & Q_3
    \end{pmatrix}
    =
    \begin{pmatrix}
      b_{11} & b_{21} & b_{31} \\
      b_{12} & b_{22} & b_{32} \\
      b_{13} & b_{23} & b_{33}
    \end{pmatrix}.
  \end{align*}
  Note that
  $A$ (resp. $B$) is depending on the representations of $P_i$ (resp. $Q_i$)
  up to a nonzero constant in $k$.

\item[(3)]
  Here $A$ (resp. $B$) is invertible since
  $P_1, P_2, P_3$ (resp. $Q_1, Q_2, Q_3$) are not lying on a line.
  Define $t: \mathbf{P}^2 \to \mathbf{P}^2$ by sending $P = [x:y:z] \in \mathbf{P}^2$
  to
  \[
    t(P)
    = BA^{-1}P
    =
    \begin{pmatrix}
      b_{11} & b_{21} & b_{31} \\
      b_{12} & b_{22} & b_{32} \\
      b_{13} & b_{23} & b_{33}
    \end{pmatrix}
    \begin{pmatrix}
      a_{11} & a_{21} & a_{31} \\
      a_{12} & a_{22} & a_{32} \\
      a_{13} & a_{23} & a_{33}
    \end{pmatrix}^{-1}
    \begin{pmatrix}
      x \\
      y \\
      z
    \end{pmatrix}.
  \]
  Note that the matrix $BA^{-1} \in \mathsf{GL}_3(k)$
  depends on the representations of $P_i$ (resp. $Q_i$)
  up to a nonzero constant in $k$.
  Hence, $t$ is a well-defined map from $\mathbf{P}^2$ to $\mathbf{P}^2$.
  Besides, $t$ is a projective change of coordinates mapping $P_i$ to $Q_i$
  ($i = 1, 2, 3$).

\item[(4)]
  \emph{Generalization.
  Let $P_i$ (resp. $Q_i$) be $n+1$ points in $\mathbf{P}^n$ ($i = 1, \ldots, n+1$)
  not lying on a hyperplane.
  Then there is a projective change of coordinates
  $t: \mathbf{P}^n \to \mathbf{P}^n$ such that $t(P_i) = Q_i$ ($i = 1, \ldots, n+1$).}
  The proof is the same except replacing $2$ by $n$.
\end{enumerate}
$\Box$ \\\\



%%%%%%%%%%%%%%%%%%%%%%%%%%%%%%%%%%%%%%%%%%%%%%%%%%%%%%%%%%%%%%%%%%%%%%%%%%%%%%%%



\subsubsection*{Problem 4.15.*}
\addcontentsline{toc}{subsubsection}{Problem 4.15.*}
\emph{Show that any two distinct lines in $\mathbf{P}^2$ intersect in one point.} \\



\emph{Proof.}
\begin{enumerate}
\item[(1)]
  Let
  \begin{align*}
    L_1 &: a_1 x + b_1 y + c_1 z = 0 \\
    L_2 &: a_2 x + b_2 y + c_2 z = 0
  \end{align*}
  be two distinct lines in $\mathbf{P}^2$.

\item[(2)]
  View (1) as the system of linear equations.
  Let the coefficient matrix $A$ be
  \[
    A
    =
    \begin{pmatrix}
      a_1 & b_1 & c_1 \\
      a_2 & b_2 & c_2
    \end{pmatrix}.
  \]
  Since $L_1$ and $L_2$ are distinct, $\rank(A) = 2$.
  The rank-nullity theorem shows that
  \[
    \dim_k \ker(A) = 3 - \rank(A) = 1.
  \]

\item[(3)]
  Might take a basis $\{ (x_0, y_0, z_0) \}$ for $\ker(A)$.
  Here $(x_0, y_0, z_0) \neq 0$ and
  any other nonzero solutions of $L_1 \cap L_2$ is of the form
  $(\lambda x_0, \lambda y_0, \lambda z_0)$ ($\lambda \neq 0$).
  Therefore,
  \[
    L_1 \cap L_2 = \{ [x_0:y_0:z_0] \} \in \mathbf{P}^2.
  \]
\end{enumerate}
$\Box$ \\\\



%%%%%%%%%%%%%%%%%%%%%%%%%%%%%%%%%%%%%%%%%%%%%%%%%%%%%%%%%%%%%%%%%%%%%%%%%%%%%%%%



\subsubsection*{Problem 4.16.*}
\addcontentsline{toc}{subsubsection}{Problem 4.16.*}
\emph{Let $L_1, L_2, L_3$ (resp. $M_1, M_2, M_3$) be lines in $\mathbf{P}^{2}(k)$ that
do not all pass through a point.
Show that there is a projective change of coordinates: $t: \mathbf{P}^{2} \to \mathbf{P}^{2}$
such that $t(L_i) = M_i$.
(Hint: Let $P_{ij} = L_i \cap L_j$, $Q_{ij} = M_i \cap M_j$, $i \neq j$,
and apply Problem 4.14.)
Extend this to $n+1$ hyperplanes in $\mathbf{P}^{n}$,
not passing through a point.} \\



\emph{Proof.}
\begin{enumerate}
\item[(1)]
  Let $P_{ij} = L_i \cap L_j$ (resp. $Q_{ij} = M_i \cap M_j$), $i \neq j$.
  $P_{ij}$ (resp. $Q_{ij}$) is uniquely determined by
  $L_i$ and $L_j$ (resp. $M_i$ and $M_j$) (Problem 4.15).
  Also, $P_{ij} \neq P_{i'j'}$ (resp. $Q_{ij} \neq Q_{i'j'}$)
  if $\{ i,j \} \neq \{ i',j' \}$ (as sets) by assumption.

\item[(2)]
  Problem 4.14 shows that
  there is a projective change of coordinates $t: \mathbf{P}^{2} \to \mathbf{P}^{2}$
  such that $t(P_{ij}) = Q_{ij}$, $i \neq j$.
  Similar to the argument in Problem 4.13(d), we conclude that $t(L_i)$ = $M_i$.

\item[(3)]
  \emph{Show that we can extend this to $n+1$ hyperplanes in $\mathbf{P}^{n}$,
  not passing through a point.}
  We cannot apply Steps (1)(2) to the generalized case $\mathbf{P}^{n}$.
  Instead, we can apply the proof of Problem 4.14.

\item[(4)]
  Let $E_i$ (resp. $F_i$) ($i=1,\ldots,n+1$) be $(n+1)$ hyperplanes in $\mathbf{P}^{n}$
  that do not passing through a point.
  Write
  \begin{align*}
    E_i &: a_{i,1} x_1 + \cdots + a_{i,n+1} = 0 \in \mathbf{P}^n(k) \\
    F_i &: b_{i,1} x_1 + \cdots + b_{i,n+1} = 0 \in \mathbf{P}^n(k)
  \end{align*}
  for $i = 1, \ldots, n+1$.

\item[(5)]
  View $E_i$ (resp. $F_i$) as the system of linear equations.
  Let the coefficient matrix $A$ (resp. $B$) be
    \begin{align*}
    A
    &=
    \begin{pmatrix}
      a_{1,1}   & \cdots & a_{n+1,1} \\
      \vdots    & \ddots & \vdots \\
      a_{1,n+1} & \cdots & a_{n+1,n+1}
    \end{pmatrix} \\
    B
    &=
    \begin{pmatrix}
      b_{1,1}   & \cdots & b_{n+1,1} \\
      \vdots    & \ddots & \vdots \\
      b_{1,n+1} & \cdots & b_{n+1,n+1}
    \end{pmatrix}.
  \end{align*}
  Note that
  $A$ (resp. $B$) is depending on the representations of $E_i$ (resp. $F_i$)
  up to a nonzero constant in $k$.

\item[(6)]
  Here $A$ (resp. $B$) is invertible since $E_i$ (resp. $F_i$) are not passing through a point.
  Define $t: \mathbf{P}^{n} \to \mathbf{P}^{n}$ by sending
  $P = [x_1 : \cdots : x_{n+1}] \in \mathbf{P}^{n}$
  to
  \begin{align*}
    t(P)
    &= BA^{-1}P \\
    &=
    \begin{pmatrix}
      b_{1,1}   & \cdots & b_{n+1,1} \\
      \vdots    & \ddots & \vdots \\
      b_{1,n+1} & \cdots & b_{n+1,n+1}
    \end{pmatrix}
    \begin{pmatrix}
      a_{1,1}   & \cdots & a_{n+1,1} \\
      \vdots    & \ddots & \vdots \\
      a_{1,n+1} & \cdots & a_{n+1,n+1}
    \end{pmatrix}^{-1}
    \begin{pmatrix}
      x_1 \\
      \vdots \\
      x_{n+1}
    \end{pmatrix}.
  \end{align*}
  Note that the matrix $BA^{-1} \in \mathsf{GL}_{n+1}(k)$
  depends on the representations of $E_i$ (resp. $F_i$)
  up to a nonzero constant in $k$.
  Hence, $t$ is a well-defined map from $\mathbf{P}^{n}$ to $\mathbf{P}^{n}$.
  Besides, $t$ is a projective change of coordinates mapping $E_i$ to $F_i$
  ($i = 1, \ldots, n+1$).
\end{enumerate}
$\Box$ \\



\emph{Note.}
  It is the duality of Problem 4.14.
  See Problem 4.18 for more details. \\\\



%%%%%%%%%%%%%%%%%%%%%%%%%%%%%%%%%%%%%%%%%%%%%%%%%%%%%%%%%%%%%%%%%%%%%%%%%%%%%%%%



\subsubsection*{Problem 4.17.*}
\addcontentsline{toc}{subsubsection}{Problem 4.17.*}
\emph{Let $z$ be a rational function on a projective variety $V$.
Show that the pole set of $z$ is an algebraic subset of $V$.} \\



\emph{Proof.}
\begin{enumerate}
\item[(1)]
  Similar to the proof of Proposition 2 in \S 2.4.
  For $g \in k[x_1, \ldots, x_{n+1}]$,
  denote the residue of $g$ in $\Gamma_h(V)$ by $\overline{g}$.

\item[(2)]
  Let
  \[
    J_z = \{ g \in k[x_1, \ldots, x_{n+1}] : \overline{g}z \in \Gamma_h(V) \}.
  \]
  $J_z$ is an ideal in $k[x_1, \ldots, x_{n+1}]$ containing $I(V)$,
  and the points of $V(J_z)$ are exactly those points where $z$ is not defined
  if $J_z$ is homogeneous.

\item[(3)]
  \emph{Show that $J_z$ is homogeneous by the homogeneous property of $I(V)$.}
  Given any $g = g_r + g_{r+1} + \cdots \in J_z$,
  where $g_i$ is a form of $k[x_1, \ldots, x_{n+1}]$.
  Write $z = a/b$ for some form $a, b \in \Gamma_h(V)$ of the same degree.
  So $\overline{g}z = \overline{g} a/b \in \Gamma_h(V)$.
  So there is $f = f_s + f_{s+1} + \cdots \in k[x_1, \ldots, x_{n+1}]$
  (where $f_i$ is a form of $k[x_1, \ldots, x_{n+1}]$)
  such that $ga - fb \in I(V)$.
  Since $I(V)$ is homogeneous, each form $g_i a - f_i b$ is in $I(V)$.
  So $\overline{g_i} z = \overline{f_i} \in \Gamma_h(V)$ for each $i$
  (since $\Gamma_h(V)$ is homogeneous), that is, $g_i \in J_z$ for each $i$.
\end{enumerate}
$\Box$ \\\\



%%%%%%%%%%%%%%%%%%%%%%%%%%%%%%%%%%%%%%%%%%%%%%%%%%%%%%%%%%%%%%%%%%%%%%%%%%%%%%%%



\subsubsection*{Problem 4.18. (Duality)}
\addcontentsline{toc}{subsubsection}{Problem 4.18. (Duality)}
\emph{Let $H = V\left( \sum a_i x_i \right)$ be a hyperplane in $\mathbf{P}^n$.
Note that $(a_1, \ldots, a_{n+1})$ is determined by $H$ up to a constant.}
\begin{enumerate}
\item[(a)]
  \emph{Show that assigning $[a_1 : \cdots : a_{n+1}] \in \mathbf{P}^n$ to $H$
  sets up a natural one-to-one correspondence between
  $\{ \text{hyperplanes in $\mathbf{P}^n$} \}$ and $\mathbf{P}^n$.
  If $P \in \mathbf{P}^n$,
  let $P^{*}$ be the corresponding hyperplane;
  if $H$ is a hyperplane, $H^{*}$ denotes the corresponding point.}

\item[(b)]
  \emph{Show that $P^{**} = P$, $H^{**} = H$.
  Show that $P \in H$ if and only if $H^{*} \in P^{*}$.}
\end{enumerate}
\emph{This is the well-known \textbf{duality} of the projective space.} \\



\emph{Proof of (a).}
\begin{enumerate}
\item[(1)]
  Define $\alpha: \{ \text{hyperplanes} \} \to \mathbf{P}^n$
  (resp. $\beta: \mathbf{P}^n \to \{ \text{hyperplanes} \}$) by
  \begin{align*}
    \alpha &: V\left( \sum a_i x_i \right) \mapsto [a_1 : \cdots : a_{n+1}], \\
    \beta &: [a_1 : \cdots : a_{n+1}] \mapsto V\left( \sum a_i x_i \right).
  \end{align*}

\item[(2)]
  As $H = V\left( \sum a_i x_i \right)$ is a hyperplane,
  the corresponding
  \[
    \alpha(H) = [a_1 : \cdots : a_{n+1}] \in \mathbf{P}^n
  \]
  and thus $\alpha$ is well-defined.
  Similarly, $\beta$ is well-defined.

\item[(3)]
  Note that both $\alpha \circ \beta$ and $\beta \circ \alpha$ are identity maps.
  $\alpha$ (resp. $\beta$) is an isomorphism, that is,
  there is a natural one-to-one correspondence between
  \[
    \{ \text{hyperplanes in $\mathbf{P}^n$} \} \longleftrightarrow \mathbf{P}^n.
  \]
\end{enumerate}
$\Box$ \\



\emph{Proof of (b).}
\begin{enumerate}
\item[(1)]
  We've showed that $P^{**} = P$, $H^{**} = H$ in (a).
  It is suffices to show that $P \in H$ iff $H^{*} \in P^{*}$.

\item[(2)]
  Write $H = V\left( \sum a_i x_i \right) \subseteq \mathbf{P}^n$ and
  $P = [b_1 : \cdots : b_{n+1}] \in \mathbf{P}^n$.
  Hence,
  \begin{align*}
    P \in H
    &\Longleftrightarrow
      a_1 b_1 + \cdots + a_{n+1} b_{n+1} = 0 \\
    &\Longleftrightarrow
      b_1 a_1 + \cdots + b_{n+1} a_{n+1} = 0 \\
    &\Longleftrightarrow
      H^{*} \in P^{*}.
  \end{align*}
\end{enumerate}
$\Box$ \\\\



%%%%%%%%%%%%%%%%%%%%%%%%%%%%%%%%%%%%%%%%%%%%%%%%%%%%%%%%%%%%%%%%%%%%%%%%%%%%%%%%
%%%%%%%%%%%%%%%%%%%%%%%%%%%%%%%%%%%%%%%%%%%%%%%%%%%%%%%%%%%%%%%%%%%%%%%%%%%%%%%%



\subsection*{4.3. Affine and Projective Varieties \\}
\addcontentsline{toc}{subsection}{4.3. Affine and Projective Varieties}



\subsubsection*{Problem 4.19.*}
\addcontentsline{toc}{subsubsection}{Problem 4.19.*}
\emph{If $I = (f)$ is the ideal of an affine hypersurface,
show that $I^{*} = (f^{*})$.} \\



\emph{Proof.}
\begin{enumerate}
\item[(1)]
  Note that $I^{*}$ is the ideal in $k[x_1,\ldots,x_{n+1}]$
  generated by $\{ g^{*} : g \in I = (f) \}$.
  In particular, $f^{*} \in I^{*}$. Thus $(f^{*}) \subseteq I^{*}$.

\item[(2)]
  Conversely,
  given any $g = \sum a_i (h_i f)^{*} \in I^{*}$,
  where $a_i, h_i \in k[x_1,\ldots,x_{n+1}]$.
  Thus,
  \[
    g
    = \sum_i a_i (h_i f)^{*}
    = \sum_i a_i h_i^{*} f^{*}
    = \left( \sum_i a_i h_i^{*} \right) f^{*} \in (f^{*})
  \]
  (by Proposition 5 in \S 2.6).
\end{enumerate}
$\Box$ \\\\



%%%%%%%%%%%%%%%%%%%%%%%%%%%%%%%%%%%%%%%%%%%%%%%%%%%%%%%%%%%%%%%%%%%%%%%%%%%%%%%%



\subsubsection*{Problem 4.20.}
\addcontentsline{toc}{subsubsection}{Problem 4.20.}
\emph{Let $V = V(y-x^2, z-x^3) \subseteq \mathbf{A}^{3}$. Prove:}
\begin{enumerate}
\item[(a)]
  \emph{$I(V) = (y-x^2, z-x^3)$.}

\item[(b)]
  \emph{$zw-xy \in I(V)^{*} \subseteq k[x,y,z,w]$,
  but $zw-xy \not\in I((y-x^2)^{*}, (z-x^3)^{*})$.
  So if $I(V) = (f_1, \ldots, f_r)$,
  it does not follow that $I(V)^{*} = (f_1^{*}, \ldots, f_r^{*})$.} \\
\end{enumerate}



\emph{Proof of (a).}
  By Problems 1.11 and 2.8, $V$ is an affine variety.
  Thus $I(V) = (y-x^2, z-x^3)$ is a prime ideal.
$\Box$ \\



\emph{Proof of (b).}
\begin{enumerate}
\item[(1)]
  Since $z - xy = (z-x^3) - x(y-x^2) \in I(V)$,
  $zw - xy = (z - xy)^{*} \in I(V)^{*}$.

\item[(2)]
  Suppose $zw - xy \in I((y-x^2)^{*}, (z-x^3)^{*}) = I(yw-x^2, zw^2-x^3)$.
  Write $zw - xy = f \cdot (yw-x^2) + g \cdot (zw^2-x^3)$ for some $f,g \in k[x,y,z,w]$.
  Then $\deg_{x}(zw - xy) = 1$
  but $\deg_{x}(f \cdot (yw-x^2) + g \cdot (zw^2-x^3))$ cannot be 1, which is absurd.
\end{enumerate}
$\Box$ \\\\



%%%%%%%%%%%%%%%%%%%%%%%%%%%%%%%%%%%%%%%%%%%%%%%%%%%%%%%%%%%%%%%%%%%%%%%%%%%%%%%%



\subsubsection*{Problem 4.21.}
\addcontentsline{toc}{subsubsection}{Problem 4.21.}
\emph{Show that if $V \subseteq W \subseteq \mathbf{P}^{n}$ are varieties,
and $V$ is a hypersurface, then $W = V$ or $W = \mathbf{P}^{n}$
(see Problem 1.39).} \\



\emph{Proof.}
\begin{enumerate}
\item[(1)]
  Write $V = V(f)$ for some irreducible form $f \in k[x_1,\ldots,x_{n+1}]$.
  So $I(V) = (f)$ by the projective Nullstellensatz.

\item[(2)]
  Note that $I(W)$ is a prime ideal such that
  \[
    (f) \supseteq I(W) \supseteq (0).
  \]
  By Problem 1.39, $I(W) = (f)$ or $(0)$.
  Thus, $W = V$ or $W = \mathbf{P}^{n}$.
\end{enumerate}
$\Box$ \\\\



%%%%%%%%%%%%%%%%%%%%%%%%%%%%%%%%%%%%%%%%%%%%%%%%%%%%%%%%%%%%%%%%%%%%%%%%%%%%%%%%



\subsubsection*{Problem 4.22.*}
\addcontentsline{toc}{subsubsection}{Problem 4.22.*}
\emph{Suppose $V$ is a variety in $\mathbf{P}^{n}$ and $V \supseteq H_{\infty}$.
Show that $V = \mathbf{P}^{n}$ or $V = H_{\infty}$.
If $V = \mathbf{P}^{n}$, $V_{*} = \mathbf{A}^{n}$,
while if $V = H_{\infty}$, $V_{*} = \varnothing$.} \\



\emph{Proof.}
\begin{enumerate}
\item[(1)]
  Note that $H_{\infty} = V(x_{n+1})$ is a hypersurface.
  By Problem 4.21, $V = \mathbf{P}^{n}$ or $V = H_{\infty}$.

\item[(2)]
  If $V = \mathbf{P}^{n}$, then $I = I(V) = (0)$.
  So $V_{*} = V(I_{*}) = V(0) = \mathbf{A}^{n}$.

\item[(3)]
  If $V = H_{\infty}$, then $I = I(V) = (x_{n+1})$.
  So $V_{*} = V(I_{*}) = V(1) = \varnothing$.
\end{enumerate}
$\Box$ \\\\



%%%%%%%%%%%%%%%%%%%%%%%%%%%%%%%%%%%%%%%%%%%%%%%%%%%%%%%%%%%%%%%%%%%%%%%%%%%%%%%%



\subsubsection*{Problem 4.23.*}
\addcontentsline{toc}{subsubsection}{Problem 4.23.*}
\emph{Describe all subvarieties in $\mathbf{P}^{1}$ and in $\mathbf{P}^{2}$.} \\



\emph{Proof.}
\begin{enumerate}
\item[(1)]
  \emph{Show that all subvarieties in $\mathbf{P}^{1}$ are
  $\varnothing$, $\mathbf{P}^{1}$, and single points.}
  (Also compare to Problem 1.8.)
  \begin{enumerate}
  \item[(a)]
    $\varnothing$, single points and $\mathbf{P}^{1}$ are all varieties.

  \item[(b)]
    Let $V$ be a nonempty subvariety in $\mathbf{P}^{1}$.
    Write $\{[a:b]\} = V(bx - ay) \subseteq V$ for some point $[a:b] \in \mathbf{P}^{1}$.
    So $V(bx - ay) \subseteq V \subseteq \mathbf{P}^{1}$.

  \item[(c)]
    Since $V(bx - ay)$ is a hypersurface and $bx-ay$ is irreducible,
    Problem 4.21 implies that
    $V = V(bx - ay)$ (a single point) or $V = \mathbf{P}^{1}$ itself.
  \end{enumerate}

\item[(2)]
  \emph{Show that all subvarieties in $\mathbf{P}^{2}$ are
  $\varnothing$, $\mathbf{P}^{2}$, single points, and hypersurfaces $V(f)$,
  where $f$ is an irreducible form.}
  Similar to Corollary 2 of Proposition 2 in \S 1.6.
  \begin{enumerate}
  \item[(a)]
    Let $V$ be a subvariety in $\mathbf{P}^{2}$.
    If $V$ is finite or $I(V) = (0)$, $V$ is of one required type.

  \item[(b)]
    Otherwise $I(V)$ contains a non-constant form $f$;
    since $I(V)$ is prime, we may assume $f$ is irreducible.

  \item[(c)]
    Hence, $I(V) = (f)$; for if $g \in I(V)$, $g \not\in (f)$,
    then $V \subseteq V(f,g)$ is finite by Problem 5.7,
    which is a projective analogue of Proposition 2 in \S 1.6.
  \end{enumerate}
\end{enumerate}
$\Box$ \\\\



%%%%%%%%%%%%%%%%%%%%%%%%%%%%%%%%%%%%%%%%%%%%%%%%%%%%%%%%%%%%%%%%%%%%%%%%%%%%%%%%



\subsubsection*{Problem 4.24.*}
\addcontentsline{toc}{subsubsection}{Problem 4.24.*}
\emph{Let $P = [0 : 1 : 0] \in \mathbf{P}^{2}(k)$.
Show that the lines through $P$ consist of the following:}
\begin{enumerate}
\item[(a)]
  \emph{The vertical lines
  $L_{\lambda} = V(x-\lambda z) = \{ [\lambda:t:1] : t \in k \} \cup \{ P \}$.}

\item[(b)]
  \emph{The line at infinity $L_{\infty} = V(z)
  = \{ [x:y:0] : x,y \in k, x \neq 0 \text{ or } y \neq 0 \}$.} \\
\end{enumerate}



\emph{Proof.}
\begin{enumerate}
\item[(1)]
  Let $L$ be one line passing through $P$ and $Q = [b_1 : b_2 : b_3] \neq P$.
  Thus,
  \begin{align*}
    L
    &= \{ [ \mu b_1 : \lambda + \mu b_2 : \mu b_3 ] :
        \lambda, \mu \in k, \lambda \neq 0 \text{ or } \mu \neq 0 \} \\
    &= \{ [ \mu b_1 : \lambda + \mu b_2 : \mu b_3 ] :
        \lambda, \mu \in k, \mu \neq 0 \} \cup \{ P \} \\
    &= \{ [ b_1 : t + b_2 : b_3 ] : t \in k \} \cup \{ P \}
      &(t := \lambda/\mu)
  \end{align*}
  Consider two cases: $b_3 \neq 0$ and $b_3 = 0$.

\item[(2)]
  $b_3 \neq 0$.
  So
  \begin{align*}
    L
    &= \{ [ b_1 : t + b_2 : b_3 ] : t \in k \} \cup \{ P \} \\
    &= \left\{
        \left[ \frac{b_1}{b_3} : \frac{t}{b_3} + \frac{b_2}{b_3} : 1 \right] : t \in k
      \right\} \cup \{ P \} \\
    &= \{ [ \lambda : t : 1 ] : t \in k \} \cup \{ P \}
      &\left( \frac{b_1}{b_3} \mapsto \lambda,  \frac{t+b_2}{b_3} \mapsto t \right) \\
    &= L_{\lambda}.
  \end{align*}

\item[(3)]
  $b_3 = 0$.
  So
  \begin{align*}
    L
    &= \{ [ b_1 : t + b_2 : 0 ] : t \in k \} \cup \{ P \} \\
    &= \{ [ x : y : 0 ] : y \in k \} \cup \{ P \}
      &(b_1 \mapsto x, t + b_2 \mapsto y) \\
    &= \{ [ x : y : 0 ] : y \in k, x \neq 0 \} \cup \{ P \} \\
    &= \{ [ x : y : 0 ] : y \in k, x \neq 0 \text{ or } y \neq 0 \} \\
    &= L_{\infty}.
  \end{align*}

\end{enumerate}
$\Box$ \\\\



%%%%%%%%%%%%%%%%%%%%%%%%%%%%%%%%%%%%%%%%%%%%%%%%%%%%%%%%%%%%%%%%%%%%%%%%%%%%%%%%



\subsubsection*{Problem 4.25.*}
\addcontentsline{toc}{subsubsection}{Problem 4.25.*}
\emph{Let $P = [x : y : z] \in \mathbf{P}^{2}$.}
\begin{enumerate}
\item[(a)]
  \emph{Show that $\{ (a,b,c) \in \mathbf{A}^{3} : ax + by + cz = 0 \}$
  is a hyperplane in $\mathbf{A}^{3}$.}

\item[(b)]
  \emph{Show that for any finite set of points in $\mathbf{P}^{2}$,
  there is a line not passing through any of them.} \\
\end{enumerate}



\emph{Proof of (a).}
\begin{enumerate}
\item[(1)]
  Let $V = \{ (a,b,c) \in \mathbf{A}^{3} : ax + by + cz = 0 \}$.

\item[(2)]
  $V$ is well-defined and the form $ax + by + cz$ is of degree one
  since not all $a, b, c$ are zero.
  Hence $V$ is a hyperplane (plane) in $\mathbf{A}^{3}$.
\end{enumerate}
$\Box$ \\



\emph{Proof of (b).}
\begin{enumerate}
\item[(1)]
  Let $S$ be any finite set of points in $\mathbf{P}^2(k)$.
  Write $S = (S \cap H_{\infty}) \cap (S - H_{\infty})$.

\item[(2)]
  $S \cap H_{\infty}$ is a finite subset of
  $H_{\infty} = \mathbf{P}^{1}(k) \supseteq \mathbf{A}^{1}(k) = k$.
  Since $k = \overline{k}$ is infinite (Problem 1.6),
  there is one point $[\beta:1:0] \in H_{\infty} - S$.
  Here $\beta \in k$ is uniquely determined.

\item[(3)]
  $S - H_{\infty}$ is finite too.
  Write
  $S - H_{\infty} = \{ [ \lambda_1 : \mu_1 : 1 ], \ldots, [ \lambda_r : \mu_r : 1 ] \}$.
  Here $\lambda_i$ and $\mu_i$ are all uniquely determined ($i = 1, \ldots, r$).
  Take any $\gamma \in k$ such that
  $\gamma \not\in\{ \lambda_1 - \beta\mu_1, \ldots,  \lambda_r - \beta\mu_r \}$.
  (It is possible since $k$ is infinite.)

\item[(4)]
  Let
  \[
    L = V(x - \beta y - \gamma z)
  \]
  be a line in $\mathbf{P}^{2}$.
  By construction, $L \cap S = \varnothing$.
\end{enumerate}
$\Box$ \\\\



%%%%%%%%%%%%%%%%%%%%%%%%%%%%%%%%%%%%%%%%%%%%%%%%%%%%%%%%%%%%%%%%%%%%%%%%%%%%%%%%
%%%%%%%%%%%%%%%%%%%%%%%%%%%%%%%%%%%%%%%%%%%%%%%%%%%%%%%%%%%%%%%%%%%%%%%%%%%%%%%%



\subsection*{4.4. Multiprojective Space \\}
\addcontentsline{toc}{subsection}{4.4. Multiprojective Space}



\subsubsection*{Problem 4.26.*}
\addcontentsline{toc}{subsubsection}{Problem 4.26.*}
\emph{}
\begin{enumerate}
\item[(a)]
  \emph{Define maps
  $\varphi_{i,j}: \mathbf{A}^{n+m}
  \to
  U_i \times U_j
  \subseteq
  \mathbf{P}^{n} \times \mathbf{P}^{m}$.
  Using $\varphi_{n+1,m+1}$, define the
  ``biprojective closure'' of an algebraic set in $\mathbf{A}^{n+m}$.
  Prove an analogue of Proposition 3 of \S 4.3.}

\item[(b)]
  \emph{Generalize part (a) to maps
  \[
    \varphi:
    \mathbf{A}^{n_1} \times \cdots \times \mathbf{A}^{n_r} \times \mathbf{A}^{m}
    \to
    \mathbf{P}^{n_1} \times \cdots \times \mathbf{P}^{n_r} \times \mathbf{A}^{m}.
  \]
  Show that this sets up a correspondence between
  \[
    \{ \text{nonempty affine varieties in $\mathbf{A}^{n_1 + \cdots + n_r + m}$} \}
  \]
  and
  \begin{align*}
    & \{ \text{varieties in }
      \mathbf{P}^{n_1} \times \cdots \times \mathbf{P}^{n_r} \times \mathbf{A}^{m} \\
    & \qquad
      \text{ that intersect }
      U_{n_1+1} \times \cdots \times U_{n_r+1} \times \mathbf{A}^{m} \}.
  \end{align*}
  Show that this correspondence preserves function fields and local rings.} \\
\end{enumerate}



\emph{Proof of (a).}
\begin{enumerate}
\item[(1)]
  Define maps
  $\varphi_{i,j}: \mathbf{A}^{n+m}
  \to
  U_i \times U_j
  \subseteq
  \mathbf{P}^{n} \times \mathbf{P}^{m}$
  by
  \begin{align*}
    &
    \varphi_{i,j}:
    (x_1, \ldots, x_{n}, y_1, \ldots, y_{m})
    \mapsto \\
    &\qquad
      [x_1 : \ldots : x_{i-1} : 1 : x_{i} : \ldots : x_{n}] \times
      [y_1 : \ldots : y_{j-1} : 1 : y_{j} : \ldots : y_{m}].
  \end{align*}
  Actually, we can write
  $\varphi_{i,j}(x, y) = (\varphi_i(x), \varphi_j(y))$
  for $x = (x_1, \ldots, x_{n}) \in \mathbf{A}^{n}$ and
  $y = (y_1, \ldots, y_{m}) \in \mathbf{A}^{m}$.

\item[(2)]
  Given any algebraic set $V$ in $\mathbf{A}^{n+m}$.
  Let $I^{*}$ be the bihomogeneous ideal in
  $k[x,y] = k[x_1, \ldots, x_{n+1}, y_1, \ldots, y_{m+1}]$
  generated by $\{ f^{*} : f \in I \}$.
  Here
  \[
    f^{*}
    =
    x_{n+1}^{r} y_{m+1}^{s}
    f\left( \frac{x_1}{x_{n+1}}, \ldots, \frac{x_{n}}{x_{n+1}},
      \frac{y_1}{y_{m+1}}, \ldots, \frac{y_{m}}{y_{m+1}} \right)
  \]
  where $r$ (resp. $s$) is the degree of $f$ in $x$ (resp. $y$).
  Define $V^{*} = V(I^{*})$ as the ``biprojective closure.''

\item[(3)]
  Proposition 3 for multiprojective space.
  \begin{enumerate}
  \item[(i)]
    \emph{If $V \subseteq \mathbf{A}^{n+m}$,
    $\varphi_{n+1,m+1}(V) = V^{*} \cap (U_{n+1} \times U_{m+1})$
    and $(V^{*})_{*}= V$.}

  \item[(ii)]
    \emph{If $V \subseteq W \subseteq \mathbf{A}^{n+m}$,
    then $V^{*} \subseteq W^{*} \subseteq \mathbf{P}^{n} \times \mathbf{P}^{m}$.
    If $V \subseteq W \subseteq \mathbf{P}^{n} \times \mathbf{P}^{m}$,
    then $V_{*} \subseteq W_{*} \subseteq \mathbf{A}^{n+m}$.}

  \item[(iii)]
    \emph{If $V$ is irreducible in $\mathbf{A}^{n+m}$,
    then $V^{*}$ is irreducible in $\mathbf{P}^{n} \times \mathbf{P}^{m}$.}

  \item[(iv)]
    \emph{If $V = \bigcup_{i} V_i$ is the irreducible decomposition of $V$ in $\mathbf{A}^{n+m}$,
    then $V^{*} = \bigcup_{i} V_i^{*}$ is the irreducible decomposition of $V^{*}$
    in $\mathbf{P}^{n} \times \mathbf{P}^{m}$.}

  \item[(v)]
    \emph{If $V \subseteq \mathbf{A}^{n+m}$,
    then $V^{*}$ is the smallest algebraic set in $\mathbf{P}^{n} \times \mathbf{P}^{m}$
    that contains $\varphi_{n+1,m+1}(V)$.}

  \item[(vi)]
    \emph{If $V \subsetneq \mathbf{A}^{n+m}$,
    then no component of $V^{*}$ lies in or contains
    $H_{\infty} \times H_{\infty}$.}

  \item[(vii)]
    \emph{If $V \subseteq \mathbf{P}^{n} \times \mathbf{P}^{m}$,
    and no component of $V$ lies in or contains
    $(H_{\infty} \times \mathbf{P}^{m}) \cup (\mathbf{P}^{n} \times H_{\infty})$,
    then $V_{*} \subsetneq \mathbf{A}^{n+m}$ and $(V_{*})^{*} = V$.}
  \end{enumerate}

\item[(4)]
  \emph{Proof of Proposition 3 for multiprojective space.}
  \begin{enumerate}
  \item[(i)]
    It follows from Proposition 5 in \S 2.6.

  \item[(ii)]
    It is obvious.

  \item[(iii)]
    If $V \subseteq \mathbf{A}^{n+m}$, $I = I(V)$, then a biform $f$ belongs to
    $I^{*}$ if and only if $f_{*} \in I$.
    If $I$ is prime, it follows readily that $I^{*}$ is also prime.

  \item[(iv)]
    It follows from (ii)(iii)(v).

  \item[(v)]
    Suppose $W$ is an algebraic set in $\mathbf{P}^{n} \times \mathbf{P}^{m}$
    which contains $\varphi_{n+1,m+1}(V)$.
    If $f \in I(W)$, then $f_{*} \in I(V)$, so
    $f = x_{n+1}^{r} y_{m+1}^{s}(f_{*})^{*} \in I(V)^{*}$.
    Therefore $I(W) \subseteq I(V)^{*}$, so $W \supseteq V^{*}$, as desired.

  \item[(vi)]
    We may assume $V$ is irreducible.
    $V^{*} \not\subseteq H_{\infty} \times H_{\infty}$ by (i).
    If $V \subseteq H_{\infty} \times H_{\infty}$,
    then $I(V)^{*} \subseteq I(V^{*}) \subseteq I(H_{\infty} \times H_{\infty}) = (x_{n+1},y_{m+1})$.
    Buf if $0 \neq f \in I(V)$, then $f^{*} \in I(V)^{*}$, but $f^{*} \not\in (x_{n+1},y_{m+1})$.
    So $V^{*} \not\supseteq H_{\infty} \times H_{\infty}$.

  \item[(vii)]
    We may assume $V$ is irreducible.
    Since $\varphi_{n+1,m+1}(V_*) \subseteq V$,
    it suffices to show that $V \subseteq (V_{*})^{*}$,
    or that $I(V_{*})^{*} \subseteq I(V)$.
    Let $f \in I(V_{*})$.
    Then $f^N \in I(V)_{*}$ for some $N$ (Nullstellensatz),
    so $x_{n+1}^{t} y_{m+1}^{s} (f^N)^{*} \in I(V)$ for some $t, s$ (Proposition 5 in \S 2.6).
    But $I(V)$ is prime, and $x_{n+1}, y_{m+1} \not\in I(V)$
    since $V \not\subseteq (H_{\infty} \times \mathbf{P}^{m}) \cup (\mathbf{P}^{n} \times H_{\infty})$,
    so $f^{*} \in I(V)$, as desired.
  \end{enumerate}
\end{enumerate}
$\Box$ \\



\emph{Proof of (b).}
\begin{enumerate}
\item[(1)]
  Define maps
  $\varphi_{i_1,\ldots,i_r}:
    \mathbf{A}^{n_1} \times \cdots \times \mathbf{A}^{n_r} \times \mathbf{A}^{m}
    \to U_{i_1} \times \cdots \times U_{i_r} \times \mathbf{A}^{m}$
  by
  \[
    \varphi_{i_1,\ldots,i_r}:
    (\mathbf{x}_1, \ldots, \mathbf{x}_r, \mathbf{y})
    \mapsto
    (\varphi_{i_1}(\mathbf{x}_1), \cdots, \varphi_{i_r}(\mathbf{x}_r), \mathbf{y}),
  \]
  where $\mathbf{x}_i = (x_{i,1}, \ldots, x_{i,n_i}) \in \mathbf{A}^{n_i}$ and
  $\mathbf{y} = (y_1, \ldots, y_{m}) \in \mathbf{A}^{m}$.

\item[(2)]
  Applying the same argument in (a), we have that there is a correspondence between
  \[
    \{ \text{nonempty affine varieties in $\mathbf{A}^{n_1 + \cdots + n_r + m}$} \}
  \]
  and
  \begin{align*}
    & \{ \text{varieties in }
      \mathbf{P}^{n_1} \times \cdots \times \mathbf{P}^{n_r} \times \mathbf{A}^{m} \\
    & \qquad
      \text{ that intersect }
      U_{n_1+1} \times \cdots \times U_{n_r+1} \times \mathbf{A}^{m} \}.
  \end{align*}

\item[(3)]
  Let $V$ be a nonempty affine varieties in $\mathbf{A}^{n_1 + \cdots + n_r + m}$.
  Let $V^{*} \subseteq \mathbf{P}^{n_1} \times \cdots \times \mathbf{P}^{n_r} \times \mathbf{A}^{m}$
  be the multiprojective closure of $V$
  intersecting $U_{n_1+1} \times \cdots \times U_{n_r+1} \times \mathbf{A}^{m}$.

\item[(4)]
  If $\overline{f} \in \Gamma(V^{*})$ is a multi-form,
  we may define $\overline{f}_{*}$ as follows:
  take a multi-form $f \in k[\mathbf{x}_1,\ldots,\mathbf{x}_r,\mathbf{y}]$
  such whose $I(V^{*})$-residue in $\overline{f}$,
  and let $\overline{f}_{*}$ to be $I(V)$-residue of $f_{*}$
  (one checks that this is independent of the choice of $f$).

\item[(5)]
  Define a natural isomorphism $\alpha: k(V^{*}) \to k(V)$ by
  $\alpha(f/g) = f_{*}/g_{*}$,
  where $f, g$ are multi-forms of the same multi-degree on $V^{*}$.

\item[(6)]
  If $P \in V$, we may consider $P \in V^{*}$
  (by means of $\varphi_{n_1+1,\ldots,n_r+1}$) and then
  $\alpha$ induces an isomorphism of $\mathscr{O}_{P}(V^{*})$ with $\mathscr{O}_{P}(V)$.
\end{enumerate}
$\Box$ \\\\



%%%%%%%%%%%%%%%%%%%%%%%%%%%%%%%%%%%%%%%%%%%%%%%%%%%%%%%%%%%%%%%%%%%%%%%%%%%%%%%%



\subsubsection*{Problem 4.27.*}
\addcontentsline{toc}{subsubsection}{Problem 4.27.*}
\emph{Show that the pole set of a rational function on a variety in any multispace is
an algebraic subset.} \\



\emph{Proof.}
\begin{enumerate}
\item[(1)]
  Similar to Problem 4.17.
  Let $V$ be a variety in one multispace
  $\mathbf{P}^{n_1} \times \cdots \times \mathbf{P}^{n_r} \times \mathbf{A}^{m}$.
  Let $z$ be a rational function on $V$.

\item[(2)]
  Let
  \[
    J_z = \{ g \in k[\mathbf{x}_1, \ldots, \mathbf{x}_{r}, \mathbf{y}]
      : \overline{g}z \in \Gamma(V) \},
  \]
  where
  $\mathbf{x}_i = [x_{i,1} : \ldots : x_{i,n_i+1}] \in \mathbf{P}^{n_i}$ for $1 \leq i \leq r$,
  $\mathbf{y} = (y_1, \ldots, y_m) \in \mathbf{A}^{m}$,
  and $\overline{g}$ is the residue of $g$ in the multi-homogeneous coordinate ring $\Gamma(V)$.

\item[(3)]
  $J_z$ is an ideal in $k[\mathbf{x}_1, \ldots, \mathbf{x}_{r}, \mathbf{y}]$
  containing the multi-homogeneous ideal $I(V)$,
  and the points of $V(J_z)$ are exactly those points where $z$ is not defined
  if $J_z$ is multi-homogeneous.

\item[(4)]
  \emph{Show that $J_z$ is multi-homogeneous by the multi-homogeneous property of $I(V)$.}
  Induction on $r$ and then apply the proof of Problem 4.17.
\end{enumerate}
$\Box$ \\\\



%%%%%%%%%%%%%%%%%%%%%%%%%%%%%%%%%%%%%%%%%%%%%%%%%%%%%%%%%%%%%%%%%%%%%%%%%%%%%%%%



\subsubsection*{Problem 4.28.* (Segre embedding)}
\addcontentsline{toc}{subsubsection}{Problem 4.28.* (Segre embedding)}
\emph{For simplicity of notation,
in this problem we let $x_0, \ldots, x_n$ be coordinates for $\mathbf{P}^{n}$,
$y_0, \ldots, y_m$ coordinates for $\mathbf{P}^{m}$,
and $t_{00}, t_{01}, \ldots, t_{0m}, t_{10}, \ldots, t_{nm}$ coordinates $\mathbf{P}^{N}$,
where $N = (n+1)(m+1)-1 = n+m+nm$.
Define $S: \mathbf{P}^{n} \times \mathbf{P}^{m} \to \mathbf{P}^{N}$
by the formula:
$S([x_0 : \ldots : x_n], [y_0 : \ldots : y_m]) = [x_0 y_0 : x_0 y_1 : \ldots : x_n y_m]$.
$S$ is called the \textbf{Segre embedding} of $\mathbf{P}^{n} \times \mathbf{P}^{m}$
in $\mathbf{P}^{n+m+nm}$.}

\begin{enumerate}
\item[(a)]
  \emph{Show that $S$ is a well-defined, one-to-one mapping.}

\item[(b)]
  \emph{Show that if $W$ is an algebraic subset of $\mathbf{P}^{N}$,
  then $S^{-1}(W)$ is an algebraic subset of $\mathbf{P}^{n} \times \mathbf{P}^{m}$.}

\item[(c)]
  \emph{Let $V = V( \{ t_{ij}t_{k\ell}-t_{i\ell}t_{kj} : i,k = 0,\ldots,n; j,\ell = 0,\ldots,m \})
  \subseteq \mathbf{P}^{N}$.
  Show that $S(\mathbf{P}^{n} \times \mathbf{P}^{m}) = V$.
  In fact, $S(U_i \times U_j) = V \cap U_{ij}$,
  where $U_{ij} = \{ [t] : t_{ij} \neq 0 \}$.}

\item[(d)]
  \emph{Show that $V$ is a variety.} \\
\end{enumerate}



\emph{Proof of (a).}
\begin{enumerate}
\item[(1)]
  \emph{Show that $S$ is well-defined.}
  Given any non-zero $\lambda, \mu \in k$.
  \begin{align*}
    & \:
    S([\lambda x_0 : \ldots : \lambda x_n], [\mu y_0 : \ldots : \mu y_m]) \\
    = & \:
    [(\lambda x_0)(\mu y_0) : (\lambda x_0)(\mu y_1) : \ldots : (\lambda x_n)(\mu y_m)] \\
    = & \:
    [(\lambda\mu) x_0 y_0 : (\lambda\mu) x_0 y_1 : \ldots : (\lambda\mu) x_n y_m] \\
    = & \:
    [x_0 y_0 : x_0 y_1 : \ldots : x_n y_m]
      &(\lambda\mu \neq 0) \\
    = & \:
    S([x_0 : \ldots : x_n], [y_0 : \ldots : y_m]).
  \end{align*}
  So $S$ is independent of the choices of $[x_0 : \ldots : x_n]$ and $[y_0 : \ldots : y_m]$.
  Since there is an index $i$ (resp. $j$) such that $x_i \neq 0$ (resp. $y_j \neq 0$),
  $x_i y_j \neq 0$ and thus $\mathrm{im}(S) \subseteq \mathbf{P}^{N}$.

\item[(2)]
  \emph{Show that $S$ is one-to-one.}
  Given any $P = [x_0 y_0 : x_0 y_1 : \ldots : x_n y_m] \in \mathrm{im}(S) \subseteq \mathbf{P}^{N}$.
  Might assume that $x_0 = y_0 = 1$. So
  \begin{align*}
    P
    &= [x_0 y_0 : x_0 y_1 : \ldots : x_n y_m] \\
    &= [1 : y_1 : y_2 : \ldots : y_m : x_1 :\ldots : x_2 : \ldots : x_n : \ldots x_n y_m].
  \end{align*}
  By the expression of $P$,
  $x_1, \ldots, x_n$ and $y_1, \ldots, y_m$ are uniquely determined.
  Hence $[1 : x_1 : \ldots : x_n]$ and $[1 : y_1 : \ldots : y_m]$
  are are uniquely determined by $P$.
  $S$ is injective.
\end{enumerate}
$\Box$ \\



\emph{Proof of (b).}
\begin{enumerate}
\item[(1)]
  $S^{-1}(W) = \{ P \in \mathbf{P}^{n} \times \mathbf{P}^{m} : S(P) \in W \}$
  is the inverse image of $W$ under $S$.
  Write
  \[
    W = V(h_1, \ldots, h_r)
  \]
  where each $h_i \in k[t_{00}, \ldots, t_{nm}]$ is a form of degree $d_i$.

\item[(2)]
  Define the pullback $\widetilde{h_i} \in k[x_0, \ldots, x_n, y_0, \ldots, y_m]$ as
  \[
    \widetilde{h_i}(x_0, \ldots, x_n, y_0, \ldots, y_m)
    = h_i(x_0 y_0, \ldots, x_n y_m)
  \]
  for each $i$.
  Clearly, $\widetilde{h_i}$ is a biform of bidegree $(d_i,d_i)$.

\item[(3)]
  As
  \begin{align*}
  P \in S^{-1}(W)
  &\Longleftrightarrow
    S(P) \in W \\
  &\Longleftrightarrow
    h_i(S(P)) = 0 \: \forall \: h_i \\
  &\Longleftrightarrow
    \widetilde{h_i}(P) = 0 \: \forall \: \widetilde{h_i} \\
  &\Longleftrightarrow
    P \in V(\widetilde{h_1}, \ldots, \widetilde{h_r}),
  \end{align*}
  $S^{-1}(W) = V(\widetilde{h_1}, \ldots, \widetilde{h_r})$ is algebraic.
\end{enumerate}
$\Box$ \\



\emph{Proof of (c).}
\begin{enumerate}
\item[(1)]
  \emph{Show that $\mathrm{im}(S) = V$ is algebraic.}
  $\mathrm{im}(S) \subseteq V$ by the construction of $S$.
  Conversely, take
  $P = [a_{ij}]_{\substack{i = 0,\ldots,n \\ j = 0,\ldots,m}} \in V \subseteq \mathbf{P}^{N}$.
  So there is one entry $a_{k\ell} \neq 0$.
  So
  \begin{align*}
    P
    &= [a_{ij}]_{\substack{i = 0,\ldots,n \\ j = 0,\ldots,m}} \\
    &= [a_{ij}a_{k\ell}]_{\substack{i = 0,\ldots,n \\ j = 0,\ldots,m}} \\
    &= [a_{i\ell}a_{kj}]_{\substack{i = 0,\ldots,n \\ j = 0,\ldots,m}} \\
    &= S([a_{i\ell}]_{i = 0,\ldots,n}, [a_{kj}]_{j = 0,\ldots,m})
  \end{align*}
  is in $\mathrm{im}(S)$.

\item[(2)]
  In the proof of (1),
  we have $S(U_i \times U_j) = V \cap U_{ij}$ where $U_{ij} = \{ [t] : t_{ij} \neq 0 \}$
  (by setting $(k,\ell) \to (i,j)$).
\end{enumerate}
$\Box$ \\



\emph{Proof of (d).}
\begin{enumerate}
\item[(1)]
  Write $V = V_1 \cup V_2$ where $V_1, V_2$ are algebraic sets in $\mathbf{P}^{N}$.
  By (b)(c),
  \[
    \mathbf{P}^{n} \times \mathbf{P}^{m}
    = S^{-1}(V)
    = S^{-1}(V_1) \cup S^{-1}(V_2).
  \]

\item[(2)]
  Since $\mathbf{P}^{n} \times \mathbf{P}^{m}$ is irreducible (Problem 4.26),
  we might assume that $\mathbf{P}^{n} \times \mathbf{P}^{m} = S^{-1}(V_1)$.
  Note that $S(S^{-1}(W)) = W$ holds for any set $W \subseteq V$.
  Hence
  \[
    V_1 = S(S^{-1}(V_1)) = S(\mathbf{P}^{n} \times \mathbf{P}^{m}) = V,
  \]
  that is, $V$ is irreducible.
\end{enumerate}
$\Box$ \\\\



%%%%%%%%%%%%%%%%%%%%%%%%%%%%%%%%%%%%%%%%%%%%%%%%%%%%%%%%%%%%%%%%%%%%%%%%%%%%%%%%
%%%%%%%%%%%%%%%%%%%%%%%%%%%%%%%%%%%%%%%%%%%%%%%%%%%%%%%%%%%%%%%%%%%%%%%%%%%%%%%%
%%%%%%%%%%%%%%%%%%%%%%%%%%%%%%%%%%%%%%%%%%%%%%%%%%%%%%%%%%%%%%%%%%%%%%%%%%%%%%%%
%%%%%%%%%%%%%%%%%%%%%%%%%%%%%%%%%%%%%%%%%%%%%%%%%%%%%%%%%%%%%%%%%%%%%%%%%%%%%%%%



\newpage
\section*{Chapter 5: Projective Plane Curves\\}
\addcontentsline{toc}{section}{Chapter 5: Projective Plane Curves}



\subsection*{5.1. Definitions \\}
\addcontentsline{toc}{subsection}{5.1. Definitions}



\subsubsection*{Problem 5.1.*}
\addcontentsline{toc}{subsubsection}{Problem 5.1.*}
\emph{Let $f$ be a projective plane curve.
Show that a point $P$ is a multiple point of $f$ if and only if
$f(P)
= \frac{\partial f}{\partial x}(P)
= \frac{\partial f}{\partial y}(P)
= \frac{\partial f}{\partial z}(P) = 0$.} \\



\emph{Proof.}
\begin{enumerate}
\item[(1)]
  Let $P$ be a point in $f$.
  Might assume $P \in U_3$.
  By dehomogenizing $f$ with respect to $z$, $m_P(f) = m_P(f_{*})$.
  Thus,
  \begin{align*}
    & \: m_P(f) = m_P(f_{*}) > 1 \\
    \Longleftrightarrow & \:
    f_{*}(P)
    = \frac{\partial f_{*}}{\partial x}(P)
    = \frac{\partial f_{*}}{\partial y}(P)
    = 0 \\
    \Longleftrightarrow & \:
    f(P)
    = \frac{\partial f}{\partial x}(P)
    = \frac{\partial f}{\partial y}(P)
    = 0.
      &(\text{$f$ is a form})
  \end{align*}

\item[(2)]
  By Euler's Theorem in \S 1.1,
  $f(P) = \frac{\partial f}{\partial x}(P) = \frac{\partial f}{\partial y}(P) = 0$
  implies that $\frac{\partial f}{\partial z}(P) = 0$.
\end{enumerate}
$\Box$ \\\\



%%%%%%%%%%%%%%%%%%%%%%%%%%%%%%%%%%%%%%%%%%%%%%%%%%%%%%%%%%%%%%%%%%%%%%%%%%%%%%%%



\subsubsection*{Problem 5.2.}
\addcontentsline{toc}{subsubsection}{Problem 5.2.}
\emph{Show that the following curves are irreducible;
find their multiple points, and the multiplicities and tangents at the multiple points.}
\begin{enumerate}
\item[(a)]
  \emph{$xy^4 + yz^4 + xz^4$.}

\item[(b)]
  \emph{$x^2y^3 + x^2z^3 + y^2z^3$.}

\item[(c)]
  \emph{$y^2z - x(x-z)(x-\lambda z)$, $\lambda \in k$.}

\item[(d)]
  \emph{$x^n + y^n + z^n$, $n > 0$.} \\
\end{enumerate}



\emph{Proof of (a).}
\begin{enumerate}
\item[(1)]
  Let $f = xy^4 + yz^4 + xz^4$.
  It suffices to show that $f_{*} = x(y^4+1) + y$ is irreducible.
  View $f_{*} \in (k[y])[x]$ as a linear function in $x$.
  Since $y^4+1$ and $y$ have no common factors in $k[x]$,
  $f_{*}$ is irreducible in $(k[y])[x] = k[x,y]$.

\item[(2)]
  By solving the system of equations
  $f(P)
  = \frac{\partial f}{\partial x}(P)
  = \frac{\partial f}{\partial y}(P)
  = \frac{\partial f}{\partial z}(P) = 0$,
  there is only one multiple point $[1:0:0]$ of $f$.

\item[(3)]
  Since $P = [1:0:0]$, we dehomogenize $f$ w.r.t. $x$ to get
  \[
    m_P(f) = m_P(f_{*}) = m_{(0,0)}(yz^4 + y^4 + z^4) = 4.
  \]
  $P$ is an ordinary multiple point of $4$ distinct tangents
  $y \pm (-1)^{\frac{1}{4}}z$, $y \pm (-1)^{\frac{3}{4}}z$.
\end{enumerate}
$\Box$ \\



\emph{Proof of (b).}
\begin{enumerate}
\item[(1)]
  Let $f = x^2y^3 + x^2z^3 + y^2z^3$.
  It suffices to show that $f_{*} = x^2(y^3+1) + y^2$ is irreducible.
  View $f_{*} \in (k[y])[x]$ as a quadratic function in $x$.
  Note that $y^3+1$ and $y^2$ have no common factors in $k[x]$.
  We may write
  \[
    f_{*}(x,y) = (a(y) x + b(y))(c(y) x + d(y))
  \]
  if $f_{*}$ were reducible.
  So
  \begin{align*}
    ac &= y^3+1 \\
    ad+bc &= 0 \\
    bd &= y^2.
  \end{align*}
  There is no solutions for $a, b, c, d \in k[y]$.
  Hence, $f_{*}$ is irreducible in $(k[y])[x] = k[x,y]$.

\item[(2)]
  By solving the system of equations
  $f(P)
  = \frac{\partial f}{\partial x}(P)
  = \frac{\partial f}{\partial y}(P)
  = \frac{\partial f}{\partial z}(P) = 0$,
  we have
  $P = [1:0:0], [0:1:0], [0:0:1]$.

\item[(3)]
  Suppose $P = [1:0:0]$.
  We dehomogenize $f$ w.r.t. $x$ to get
  \[
    m_P(f) = m_P(f_{*}) = m_{(0,0)}(y^2z^3 + y^3 + z^3) = 3.
  \]
  $P$ is an ordinary multiple point of $3$ distinct tangents
  $y + z$, $y + (-1)^{\frac{1}{3}}z$, $y - (-1)^{\frac{2}{3}}z$.

\item[(4)]
  Suppose $P = [0:1:0]$.
  We dehomogenize $f$ w.r.t. $y$ to get
  \[
    m_P(f) = m_P(f_{*}) = m_{(0,0)}(x^2z^3 + z^3 + x^2) = 2.
  \]
  $x$ is the tangent to $f$ at $P$ of the multiplicity $= 2$.

\item[(5)]
  Suppose $P = [0:0:1]$.
  We dehomogenize $f$ w.r.t. $z$ to get
  \[
    m_P(f) = m_P(f_{*}) = m_{(0,0)}(x^2y^3 + x^2 + y^2) = 2.
  \]
  $P$ is an ordinary multiple point of $2$ distinct tangents
  $x \pm (-1)^{\frac{1}{2}}y$.
\end{enumerate}
$\Box$ \\



\emph{Proof of (c).}
\begin{enumerate}
\item[(1)]
  Note that $y^2 - x(x-1)(x-\lambda)$ is irreducible (Problem 1.35).
  Hence, the homogenizing form $f = y^2z - x(x-z)(x-\lambda z)$ of
  $y^2 - x(x-1)(x-\lambda)$ w.r.t. $z$ is also irreducible.

\item[(2)]
  By solving the system of equations
  $f(P)
  = \frac{\partial f}{\partial x}(P)
  = \frac{\partial f}{\partial y}(P)
  = \frac{\partial f}{\partial z}(P) = 0$,
  we have
  \begin{align*}
    P &= [0:0:1] \text{ if } \lambda = 0 \\
    P &= [1:0:1] \text{ if } \lambda = 1
  \end{align*}
  are all multiple points of $f$.

\item[(3)]
  Suppose $P = [0:0:1]$ with $\lambda = 0$.
  Then
  \[
    m_P(f) = m_P(f_{*}) = m_{(0,0)}(y^2 - x^2(x-1)) = 2.
  \]
  $y$ is the tangent to $f$ at $P$ of the multiplicity $= 2$.
  (The projective closure of $y$ is $y$.)

\item[(4)]
  Similarly, suppose $P = [1:0:1]$ with $\lambda = 1$.
  Then
  \begin{align*}
    m_P(f)
    &= m_P(f_{*}) \\
    &= m_{(1,0)}(y^2 - x(x-1)^2) \\
    &= m_{(0,0)}(y^2 - x^2(x+1)) \\
    &= 2.
  \end{align*}
  $y$ is the tangent to $f$ at $P$ of the multiplicity $= 2$.
\end{enumerate}
$\Box$ \\



\emph{Proof of (d).}
\begin{enumerate}
\item[(1)]
  Let $f = x^n + y^n + z^n$.
  It suffices to show that $f_{*} = x^n + y^n + 1$ is irreducible.
  Write
  \[
    f_{*} = x^n + (y^n + 1) \in (k[y])[x].
  \]
  Note that $k[y]$ is a UFD and $y^n + 1$ is separable.
  Thus by the Eisenstein's criterion, $f_{*}$ irreducible in $(k[y])[x] = k[x,y]$.

\item[(2)]
  By solving the system of equations
  $f(P)
  = \frac{\partial f}{\partial x}(P)
  = \frac{\partial f}{\partial y}(P)
  = \frac{\partial f}{\partial z}(P) = 0$,
  there are no multiple points on $f$ for $n \geq 1$.
\end{enumerate}
$\Box$ \\\\



%%%%%%%%%%%%%%%%%%%%%%%%%%%%%%%%%%%%%%%%%%%%%%%%%%%%%%%%%%%%%%%%%%%%%%%%%%%%%%%%



% http://people.math.sfu.ca/~kyeats/teaching/math818soln2.pdf

\subsubsection*{Problem 5.3.}
\addcontentsline{toc}{subsubsection}{Problem 5.3.}
\emph{Find all points of intersection of the following pairs of curves,
and the intersection numbers at these points:}
\begin{enumerate}
\item[(a)]
  \emph{$y^2z - x(x-2z)(x+z)$ and $y^2 + x^2 - 2xz$.}

\item[(b)]
  \emph{$(x^2+y^2)z + x^3 + y^3$ and $x^3 + y^3 - 2xyz$.}

\item[(c)]
  \emph{$y^5 - x(y^2-xz)^2$ and $y^4 + y^3z - x^2z^2$.}

\item[(d)]
  \emph{$(x^2+y^2)^2 + 3x^2yz - y^3z$ and $(x^2+y^2)^3 - 4x^2y^2z^2$.} \\
\end{enumerate}



\emph{Proof of (a).}
\begin{enumerate}
\item[(1)]
  Say $f = y^2z - x(x-2z)(x+z)$ and $g = y^2 + x^2 - 2xz$.
  By $f - zg = 0$, $x(x-2z)(x+2z) = 0$.
  So the intersection points in $\mathbf{P}^{2}$ are
  \[
    \underbrace{[0:0:1]}_{x = 0},
    \underbrace{[2:0:1]}_{x-2z = 0},
    \underbrace{[-2,\pm\sqrt{-8}:1]}_{x+2z = 0}.
  \]

\item[(2)]
  Suppose $P = [0:0:1]$.
  We dehomogenize $f$ and $g$ w.r.t. $z$ to get
  \begin{align*}
    I(P, f \cap g)
    &= I((0,0), f_{*} \cap g_{*}) \\
    &= I((0,0), (-x(x-2)(x+1) + y^2) \cap (y^2 + x^2 - 2x)) \\
    &= 2
  \end{align*}
  (since they have no common tangents).

\item[(3)]
  Suppose $P = [2:0:1]$.
  We dehomogenize $f$ and $g$ w.r.t. $z$ to get
  \begin{align*}
    I(P, f \cap g)
    &= I(P, f_{*} \cap g_{*}) \\
    &= I(P, (-x(x-2)(x+1) + y^2) \cap (y^2 + x^2 - 2x)) \\
    &= I((0,0), (-x(x+2)(x+3) + y^2) \cap (y^2 + x^2 + 2x)) \\
    &= 2.
  \end{align*}

\item[(4)]
  Suppose $P = [-2,\pm\sqrt{-8}:1]$.
  We dehomogenize $f$ and $g$ w.r.t. $z$ to get
  \begin{align*}
    I(P, f \cap g)
    &= I(P, f_{*} \cap g_{*}) \\
    &= I(P, (-x(x-2)(x+1) + y^2) \cap (y^2 + x^2 - 2x)) \\
    &= I((0,0), (-x^3 + 7x^2 + y^2 - 14x \pm 2\sqrt{-8}y) \cap \\
      & \qquad\qquad\qquad
      (y^2 + x^2 - 6x \pm 2\sqrt{-8}y)) \\
    &= 1.
  \end{align*}
  (Or by B\'ezout's Theorem,
  each intersection point has the intersection number $1$ in this case.)
\end{enumerate}
$\Box$ \\



\emph{Proof of (b).}
\begin{enumerate}
\item[(1)]
  Say $f = (x^2+y^2)z + x^3 + y^3$ and $g = x^3 + y^3 - 2xyz$.
  By $f - g = 0$, $(x+y)^2z = 0$.
  So the intersection points in $\mathbf{P}^{2}$ are
  \[
    \underbrace{[0:0:1]}_{x+y = 0},
    \underbrace{[1:-1:0], [1:\omega:0], [1:1-\omega:0]}_{z = 0}.
  \]
  where $\omega = (-1)^{\frac{1}{3}} \in k$.

\item[(2)]
  Suppose $P = [0:0:1]$.
  We dehomogenize $f$ and $g$ w.r.t. $z$ to get
  \begin{align*}
    I(P, f \cap g)
    &= I((0,0), f_{*} \cap g_{*}) \\
    &= I((0,0), (x^3+y^3 + x^2+y^2) \cap (x^3 + y^3 - 2xy)) \\
    &= 4.
  \end{align*}

\item[(3)]
  Suppose $P = [1:-1:0]$.
  We dehomogenize $f$ and $g$ w.r.t. $x$ to get
  \begin{align*}
    I(P, f \cap g)
    &= I(P, f_{*} \cap g_{*}) \\
    &= I(P, (f_{*} - g_{*}) \cap g_{*}) \\
    &= I(P, (y+1)^2z \cap (1+y^3 - 2yz)) \\
    &= I((0,0), y^2z \cap (y^3 - 3y^2-2yz + 3y+2z)) \\
    &= 3.
  \end{align*}

\item[(4)]
  Suppose $P = [1:\omega:0]$.
  Similar to (3), we dehomogenize $f$ and $g$ w.r.t. $x$ to get
  \begin{align*}
    I(P, f \cap g)
    &= I(P, (y+1)^2z \cap (1+y^3 - 2yz)) \\
    &= I((0,0), (y^2z + 2(1+\omega)yz + (1+\omega)^2z) \cap \\
      & \qquad\qquad\qquad
      (y^3 + 3\omega y^2 - 2yz + 3\omega^2 y - 2\omega z)) \\
    &= I((0,0), (y^2z + 2(1+\omega)yz + 3\omega z) \cap \\
      & \qquad\qquad\qquad
      (y^3 + 3\omega y^2 - 2yz + 3\omega^2 y - 2\omega z)) \\
    &= 1.
  \end{align*}

\item[(5)]
  Suppose $P = [1:1-\omega:0]$.
  We dehomogenize $f$ and $g$ w.r.t. $x$ to get
  \begin{align*}
    & \: I(P, f \cap g) \\
    =& \: I(P, (y+1)^2z \cap (1+y^3 - 2yz)) \\
    =& \: I((0,0), (y^2z + 2(2-\omega)yz + (2-\omega)^2z) \cap \\
      & \qquad\qquad\qquad
      (y^3 + 3(1-\omega)y^2 - 2yz + 3(1-\omega)^2 y - 2(1-\omega)z)) \\
    =& \: I((0,0), (y^2z + 2(2-\omega)yz + 3(1-\omega)z) \cap \\
      & \qquad\qquad\qquad
      (y^3 + 3(1-\omega)y^2 - 2yz - 3\omega y - 2(1-\omega)z)) \\
    =& \: 1.
  \end{align*}
\end{enumerate}
$\Box$ \\



\emph{Proof of (c).}
\begin{enumerate}
\item[(1)]
  Say $f = y^5 - x(y^2-xz)^2$ and $g = y^4 + y^3z - x^2z^2$.
  If $y = 0$, the intersection points in $\mathbf{P}^{2} - U_2$ are
  \[
    [0:0:1], [1:0:0].
  \]
  If $y \neq 0$, the intersection points in $2_3 \cong \mathbf{A}^2$ are
  the solutions of $f_{*} = 1 - x(1-xz)^2$ and $g_{*} = 1 + z - x^2z^2$:
  \[
    \left[ \frac{1}{2}:1:2 \pm 2\sqrt{2} \right].
  \]
  Note that there are only 4 different intersection points (B\'ezout's Theorem).

\item[(2)]
  Suppose $P = [0:0:1]$.
  We dehomogenize $f$ and $g$ w.r.t. $z$ to get
  \begin{align*}
    & \:I(P, f \cap g) \\
    =& \: I((0,0), f_{*} \cap g_{*}) \\
    =& \: I((0,0), (y^5 - x(y^2-x)^2) \cap (y^4 + y^3 - x^2)) \\
    =& \: I((0,0), ((y^5 - x(y^2-x)^2) - x(y^4 + y^3 - x^2)) \cap (y^4 + y^3 - x^2)) \\
    =& \: I((0,0), y^2(y^2-x)(2x-y) \cap (y^4 + y^3 - x^2)) \\
    =& \: I((0,0), y^2 \cap (y^4 + y^3 - x^2)) \\
      & + I((0,0), (y^2-x) \cap (y^4 + y^3 - x^2)) \\
      & + I((0,0), (2x-y) \cap (y^4 + y^3 - x^2)) \\
    =& \: 4 + I((0,0), (y^2-x) \cap (y^4 + y^3 - x^2)) + 2 \\
    =& \: 9.
  \end{align*}
  Here
  \begin{align*}
    & \: I((0,0), (y^2-x) \cap (y^4 + y^3 - x^2)) \\
    =& \: I((0,0), (y^2-x) \cap ((y^4 + y^3 - x^2) - x(y^2-x))) \\
    =& \: I((0,0), (y^2-x) \cap (y^4 -xy^2+y^3)) \\
    =& \: 3.
  \end{align*}

\item[(3)]
  Suppose $P = [1:0:0]$.
  We dehomogenize $f$ and $g$ w.r.t. $x$ to get
  \begin{align*}
    & \:I(P, f \cap g) \\
    =& \: I((0,0), f_{*} \cap g_{*}) \\
    =& \: I((0,0), (y^5 - (y^2-z)^2) \cap (y^4 + y^3z - z^2)) \\
    =& \: I((0,0), ((y^5 - (y^2-z)^2) - (y^4 + y^3z - z^2)) \cap (y^4 + y^3z - z^2)) \\
    =& \: I((0,0), y^2(y^2-z)(y-2) \cap (y^4 + y^3z - z^2)) \\
    =& \: I((0,0), y^2 \cap (y^4 + y^3z - z^2)) \\
      & + I((0,0), (y^2-z) \cap (y^4 + y^3z - z^2)) \\
      & + I((0,0), (y-2) \cap (y^4 + y^3z - z^2)) \\
    =& \: 4 + I((0,0), (y^2-z) \cap (y^4 + y^3z - z^2)) + 0 \\
    =& \: 4 + 5 + 0 \\
    =& \: 9.
  \end{align*}
  Here
  \begin{align*}
    & \: I((0,0), (y^2-z) \cap (y^4 + y^3z - z^2)) \\
    =& \: I((0,0), (y^2-z) \cap ((y^4 + y^3z - z^2)-z(y^2-z))) \\
    =& \: I((0,0), (y^2-z) \cap y^2(y^2 +yz - z)) \\
    =& \: I((0,0), (y^2-z) \cap y^2) + I((0,0), (y^2-z) \cap (y^2 + yz - z)) \\
    =& \: 2 + I((0,0), (y^2-z) \cap yz) \\
    =& \: 2 + I((0,0), (y^2-z) \cap y) + I((0,0), (y^2-z) \cap z) \\
    =& \: 2 + 1 + I((0,0), y^2 \cap z) \\
    =& \: 2 + 1 + 2 \\
    =& \: 5.
  \end{align*}

\item[(4)]
  Suppose $P = \left[ \frac{1}{2}:1:2+2\sqrt{2} \right]$.
  We dehomogenize $f$ and $g$ w.r.t. $y$ to get
  \begin{align*}
    & \:I(P, f \cap g) \\
    =& \: I(P, f_{*} \cap g_{*}) \\
    =& \: I(P, (1 - x(1-xz)^2) \cap (1 + z - x^2z^2)) \\
    =& \: I\left( (0,0),
      \left( -(6+2\sqrt{2})x-\sqrt{2}/2 \cdot z + \text{higher terms} \right) \bigcap \right.\\
      & \qquad\qquad\qquad
      \left. ( -(12+8\sqrt{2})x-\sqrt{2} z + \text{higher terms}) \right) \\
    =& \: 1.
  \end{align*}
  Similarly, $I(P, f \cap g) = 1$ as $P = \left[ \frac{1}{2}:1:2-2\sqrt{2} \right]$.
\end{enumerate}
$\Box$ \\



\emph{Proof of (d).}
\begin{enumerate}
\item[(1)]
  Say $f = (x^2+y^2)^2 + 3x^2yz - y^3z$ and $g = (x^2+y^2)^3 - 4x^2y^2z^2$.
  If $z = 0$, the intersection points in $\mathbf{P}^{2} - U_3$ are
  \[
    [1:\pm\sqrt{-1}:0].
  \]
  If $z \neq 0$, the intersection points in $U_3 \cong \mathbf{A}^2$ are
  the solutions of $f_{*} = (x^2+y^2)^2 + 3x^2y - y^3$ and $g_{*} = (x^2+y^2)^3 - 4x^2y^2$:
  \[
    [0:0:1],
    \left[ \pm\frac{\sqrt{5+2\sqrt{5}}}{4} : \frac{-\sqrt{5}}{4} : 1 \right],
    \left[ \pm\frac{\sqrt{5-2\sqrt{5}}}{4} : \frac{\sqrt{5}}{4} : 1 \right].
  \]
  (To find these solutions, we might solve
  $f_{*}(r\cos\theta,r\sin\theta) = g_{*}(r\cos\theta,r\sin\theta) = 0$ over $k = \mathbb{C}$.
  As $r = 0$, nothing to do.
  As $r > 0$, we get $\theta = \frac{m\pi}{5}$ for all integers $m$.)
  Note that there are only 7 different intersection points (B\'ezout's Theorem).

\item[(2)]
  Suppose $P = [1:\sqrt{-1}:0]$.
  We dehomogenize $f$ and $g$ w.r.t. $x$ to get
  \begin{align*}
    & \: I(P, f \cap g) \\
    =& \: I(P, f_{*} \cap g_{*}) \\
    =& \: I(P, ((1+y^2)^2 + 3yz - y^3z) \cap ((1+y^2)^3 - 4y^2z^2)) \\
    =& \: I(P, ((1+y^2)^2 + 3yz - y^3z) \cap (yz(y^4 - 2y^2 - 4yz - 3)) \\
    =& \: I(P, ((1+y^2)^2 + 3yz - y^3z) \cap y) \\
      & \: + I(P, ((1+y^2)^2 + 3yz - y^3z) \cap z) \\
      & \: + I(P, ((1+y^2)^2 + 3yz - y^3z) \cap (y^4 - 2y^2 - 4yz - 3)) \\
    =& \: 2I(P, (1+y^2) \cap z) \\
      & \: + I(P, ((1+y^2)^2 + 3yz - y^3z) \cap (y^4 - 2y^2 - 4yz - 3)) \\
    =& 3.
  \end{align*}
  Here
  \[
    I(P, (1+y^2) \cap z) = I((0,0), (y^2 + 2\sqrt{-1}y) \cap z) = 1
  \]
  and
  \begin{align*}
    & \: I(P, ((1+y^2)^2 + 3yz - y^3z) \cap (y^4 - 2y^2 - 4yz - 3)) \\
    =& \: I((0,0),
      (y^4-y^3z + 4\sqrt{-1}y^3-3\sqrt{-1}y^2z - 4y^2 + 6yz + 4\sqrt{-1} z)\cap \\
      & \qquad\qquad\qquad
      (y^4 + 4\sqrt{-1}y^3 - 8y^2-4yz -8\sqrt{-1}y-4\sqrt{-1}z)) \\
    =& \: 1.
  \end{align*}
  Similarly, $I(P, f \cap g) = 3$ if $P = [1:-\sqrt{-1}:0]$.

\item[(3)]
  Suppose $P = [0:0:1]$.
  We dehomogenize $f$ and $g$ w.r.t. $z$ to get
  \begin{align*}
    I(P, f \cap g)
    &= I((0,0), f_{*} \cap g_{*}) \\
    &= I((0,0), ((x^2+y^2)^2 + 3x^2y - y^3) \cap ((x^2+y^2)^3 - 4x^2y^2)) \\
    &= 14
  \end{align*}
  (Example in \S 3.3).

\item[(4)]
  Suppose $P = \left[ \frac{\sqrt{5+2\sqrt{5}}}{4} : \frac{-\sqrt{5}}{4} : 1 \right]$.
  We dehomogenize $f$ and $g$ w.r.t. $z$ to get
  \begin{align*}
    & \: I(P, f \cap g) \\
    =& \: I(P, f_{*} \cap g_{*}) \\
    =& \: I(P, ((x^2+y^2)^2 + 3x^2y - y^3) \cap ((x^2+y^2)^3 - 4x^2y^2)) \\
    =& \: I(P, ((x^2+y^2)^2 + 3x^2y - y^3) \cap y(-3x^4-2x^2y^2+y^4 - 4x^2y)) \\
    =& \: 4 I(P, x \cap y) \\
      &+ I(P, ((x^2+y^2)^2 + 3x^2y - y^3) \cap (-3x^4-2x^2y^2+y^4 - 4x^2y)) \\
    =& \: 1.
  \end{align*}
  Here
  \[
    I(P, x \cap y)
    = I\left( (0,0), 
      \left( x+\frac{\sqrt{5+2\sqrt{5}}}{4} \right)
      \bigcap \left( y + \frac{-\sqrt{5}}{4} \right) \right)
    = 0
  \]
  and
  \begin{align*}
    & \: I(P, ((x^2+y^2)^2 + 3x^2y - y^3) \cap (-3x^4-2x^2y^2+y^4 - 4x^2y)) \\
    =& \: I\left( (0,0), \left(
        \frac{(5+2\sqrt{5})\sqrt{5+2\sqrt{5}}}{8}x-\frac{5+2\sqrt{5}}{8}y + \text{higher terms}
      \right) \bigcap \right.\\
      & \qquad\qquad\qquad
      \left. \left(
        \frac{(-10+\sqrt{5})\sqrt{5+2\sqrt{5}}}{8}x-\frac{5+4\sqrt{5}}{8}y + \text{higher terms}
      \right) \right) \\
    =& \: 1.
  \end{align*}
  Similarly, $I(P, f \cap g) = 1$ for rest three cases.
\end{enumerate}
$\Box$ \\\\



%%%%%%%%%%%%%%%%%%%%%%%%%%%%%%%%%%%%%%%%%%%%%%%%%%%%%%%%%%%%%%%%%%%%%%%%%%%%%%%%



\subsubsection*{Problem 5.4.*}
\addcontentsline{toc}{subsubsection}{Problem 5.4.*}
\emph{Let $P$ be a simple point on $f$.
Show that the tangent line to $f$ at $P$ has the equation
$\frac{\partial f}{\partial x}(P) x +
\frac{\partial f}{\partial y}(P) y +
\frac{\partial f}{\partial z}(P) z = 0$.} \\



\emph{Proof.}
\begin{enumerate}
\item[(1)]
  Similar to Problem 5.1.
  Might assume $P = [x_0:y_0:1] \in U_3$.
  By dehomogenizing $f$ with respect to $z$,
  the tangent line to $f_{*}$ at $P$ is
  \begin{align*}
    L: 0
    &=
    \frac{\partial f}{\partial x}(P) (x-x_0) + \frac{\partial f}{\partial x}(P) (y-y_0) \\
    &=
    \frac{\partial f}{\partial x}(P) x + \frac{\partial f}{\partial x}(P) y
      - \left( \frac{\partial f}{\partial x}(P) x_0 + \frac{\partial f}{\partial x}(P) y_0 \right) \\
    &=
    \frac{\partial f}{\partial x}(P) x + \frac{\partial f}{\partial x}(P) y
      + \frac{\partial f}{\partial z}(P).
  \end{align*}
  (The last equation is due to Euler's Theorem in \S 1.1.)

\item[(2)]
  Hence,
  the projective closure of $L$, say
  \[
    L^{*}: 0 =
    \frac{\partial f}{\partial x}(P) x
      + \frac{\partial f}{\partial x}(P) y
      + \frac{\partial f}{\partial z}(P) z,
  \]
  is the tangent line to $f$ at $P$.
\end{enumerate}
$\Box$ \\\\



%%%%%%%%%%%%%%%%%%%%%%%%%%%%%%%%%%%%%%%%%%%%%%%%%%%%%%%%%%%%%%%%%%%%%%%%%%%%%%%%



\subsubsection*{Problem 5.5.*}
\addcontentsline{toc}{subsubsection}{Problem 5.5.*}
\emph{Let $P = [0:1:0]$, $f$ a curve of degree $n$,
$f = \sum f_i(x,z)y^{n-i}$, $f_i$ a form of degree $i$.
Show that $m_P(f)$ is the smallest $m$ such that $f_m \neq 0$,
and the factors of $f_m$ determine the tangents to $f$ at $P$.} \\



\emph{Proof.}
\begin{enumerate}
\item[(1)]
  By dehomogenizing $f$ with respect to $y$,
  we have
  \[
    m_P(f)
    = m_P(f_{*})
    = m_{(0,0)}\left( \sum f_i(x,z) \right)
    = m
  \]
  where $m$ is the smallest integer such that $f_m \neq 0$.

\item[(2)]
  The factors of $f_m$ determine the tangents to $f_{*}$ at $P$.
  Note that each factor of $f_m$ is a form.
  Therefore, these factors of $f_m$ also determine the tangents to $f$ at $P$.
\end{enumerate}
$\Box$ \\\\



%%%%%%%%%%%%%%%%%%%%%%%%%%%%%%%%%%%%%%%%%%%%%%%%%%%%%%%%%%%%%%%%%%%%%%%%%%%%%%%%



\subsubsection*{Problem 5.6.*}
\addcontentsline{toc}{subsubsection}{Problem 5.6.*}
\emph{For any $f$, $P \in f$,
show that $m_{P}\left( \frac{\partial f}{\partial x} \right) \geq m_{P}(f)-1$.} \\



\emph{Proof.}
\begin{enumerate}
\item[(1)]
  If $P = [1:0:0]$, by dehomogenizing $f$ with respect to $x$
  we get
  \[
    m_{P}\left( \frac{\partial f}{\partial x} \right)
    = m_{P}\left( \frac{\partial f_{*}}{\partial x} \right)
    = m_{P}(0)
    = \infty.
  \]
  The result is established.

\item[(2)]
  Might assume $P \in U_3$.
  By dehomogenizing $f$ with respect to $z$, it suffices to show that
  \[
    m_{P}\left( \frac{\partial f_{*}}{\partial x} \right) \geq m_{P}(f_{*})-1.
  \]

\item[(3)]
  Write $m = m_{P}(f_{*})$ and $f_{*} = g_m + g_{m+1} + \cdots \in k[x,y]$
  where $g_{i} \neq 0$ is a form of degree $m$.
  Note that
  \[
    \frac{\partial f_{*}}{\partial x}
    = \frac{\partial g_m}{\partial x} + \frac{\partial g_{m+1}}{\partial x} + \cdots.
  \]
  So that
  $m_{P}\left( \frac{\partial f_{*}}{\partial x} \right) \geq m-1 = m_{P}(f_{*})-1$.
\end{enumerate}
$\Box$ \\\\



%%%%%%%%%%%%%%%%%%%%%%%%%%%%%%%%%%%%%%%%%%%%%%%%%%%%%%%%%%%%%%%%%%%%%%%%%%%%%%%%



\subsubsection*{Problem 5.7.*}
\addcontentsline{toc}{subsubsection}{Problem 5.7.*}
\emph{Show that two plane curves with no common components intersect in
a finite number of points.} \\



\emph{Proof.}
\begin{enumerate}
\item[(1)]
  Similar to Proposition 2 in \S 1.6.
  Let $f$ and $g$ be forms in $k[x,y,z]$ with no common components.
  In particular, $f \neq 0$ and $g \neq 0$.

\item[(2)]
  \emph{Show that the conclusion is true for a line $g$.}
  Similar to Problem 1.12.
  Let
  $\alpha: \mathbf{P}^1 \to \mathbf{P}^2$ be the map
  \[
    \alpha([\lambda : \mu]) = [\lambda a_1 + \mu b_1 : \lambda a_2 + \mu b_2 : \lambda a_3 + \mu b_3]
  \]
  and $g = \mathrm{im}(\alpha)$ be a line.
  Note that the points of $V(f, g)$ correspond to the zero of
  \[
    \widetilde{f}(\lambda, \mu)
    = f(\lambda a_1 + \mu b_1, \lambda a_2 + \mu b_2, \lambda a_3 + \mu b_3)
    = 0.
  \]
  in $\mathbf{P}^1$.
  $\widetilde{f}$ is a form in $k[\lambda,\mu]$ of degree $m := \deg(f)$
  since $f$ and $g$ have no common components.
  Hence, $\widetilde{f}$ has exactly $m$ zeros in $\mathbf{P}^1$ (with multiplicity).
  $V(f,g)$ has exactly $m$ zeros (with multiplicity). \\

  \emph{Note.} It is the same as Problem 5.12(b).

\item[(3)]
  \emph{Show that the conclusion is true for any $f, g$.}
  Note that $f$ and $g$ have no common factors in $k(x,y)[z]$.
  Since $k(x,y)[z]$ is a PID, $(f,g) = (1) \in k(x,y)[z]$.
  So $\alpha f + \beta g = 1$ for some $\alpha, \beta \in k(x,y)[z]$.
  Clearing denominators from $\alpha, \beta$ gives us
  \[
    af + bg = d
  \]
  where $d$ is a non-zero polynomial in $k[x,y]$,
  $a = d\alpha \in k[x,y,z]$ and $b = d\beta \in k[x,y,z]$.
  Taking suitable homogeneous parts of $a$, $b$ and $d$,
  we might assume $a$, $b$ and $d$ are all homogeneous.

\item[(4)]
  By Corollary to Proposition 5 in \S 2.6,
  $d$ is a product of linear forms, say $d = L_1 \cdots L_r$.
  A common zero of $f$ and $g$ is also a zero of $d$,
  and therefore it is a zero of $L_j$ for some $j$.
  So it suffices to show that $V = (f,g) = V(f,g,L_j)$ is finite for each $j$.

\item[(5)]
  (Reductio ad absurdum)
  If not, there were one $j$ such that both $V(f,L_j)$ and $V(g,L_j)$ were infinite.
  By (2), $L_j \mid f$ and $L_j \mid g$.
  $f$ and $g$ have one common factor, which is absurd.
\end{enumerate}
$\Box$ \\\\



%%%%%%%%%%%%%%%%%%%%%%%%%%%%%%%%%%%%%%%%%%%%%%%%%%%%%%%%%%%%%%%%%%%%%%%%%%%%%%%%



\subsubsection*{Problem 5.8.*}
\addcontentsline{toc}{subsubsection}{Problem 5.8.*}
\emph{Let $f$ be an irreducible curve.}
\begin{enumerate}
\item[(a)]
  \emph{Show that
  $\frac{\partial f}{\partial x}$,
  $\frac{\partial f}{\partial y}$,
  or $\frac{\partial f}{\partial z} \neq 0$.}

\item[(b)]
  \emph{Show that $f$ has only a finite number of multiple points.} \\
\end{enumerate}



\emph{Proof of (a).}
  If $\frac{\partial f}{\partial x}$,
  $\frac{\partial f}{\partial y}$,
  and $\frac{\partial f}{\partial z}$ are all zero,
  $f$ is a constant, which is not a projective plane curve.
$\Box$ \\



\emph{Proof of (b).}
\begin{enumerate}
\item[(1)]
  Let
  \[
    V
    = V\left(
      f,
      \frac{\partial f}{\partial x},
      \frac{\partial f}{\partial y},
      \frac{\partial f}{\partial z}
    \right)
  \]
  be the set of the multiple points of $f$ (Problem 5.1).
  Moreover, $V$ is an algebraic set.

\item[(2)]
  Suppose $\frac{\partial f}{\partial x} \neq 0$
  (since not all $\frac{\partial f}{\partial x}$,
  $\frac{\partial f}{\partial y}$ and
  $\frac{\partial f}{\partial z}$ are zero).
  It is nothing to do if $\frac{\partial f}{\partial x}$ is a nonzero constant.
  Suppose $\deg\left( \frac{\partial f}{\partial x} \right) \geq 1$.
  Since $f$ is irreducible and $\deg\left( \frac{\partial f}{\partial x} \right) = \deg f - 1$,
  $f$ and $\frac{\partial f}{\partial x}$ have no common factors.
  By Problem 5.7,
  $V\left( f, \frac{\partial f}{\partial x} \right)$ is a finite set.
  Hence,
  $V$ is finite
  as a subset of a finite set $V\left( f, \frac{\partial f}{\partial x} \right)$.
\end{enumerate}
$\Box$ \\\\



%%%%%%%%%%%%%%%%%%%%%%%%%%%%%%%%%%%%%%%%%%%%%%%%%%%%%%%%%%%%%%%%%%%%%%%%%%%%%%%%



\subsubsection*{Problem 5.9. (Conic)}
\addcontentsline{toc}{subsubsection}{Problem 5.9. (Conic)}
\begin{enumerate}
\item[(a)]
  \emph{Let $f$ be an irreducible conic, $P = [0:1:0]$ a simple point on $f$,
  and $z=0$ the tangent line to $f$ at $P$.
  Show that $f = ayz - bx^2 - cxz - dz^2$, $a, b \neq 0$.
  Find a projective change of coordinates $t$ so that $f^{t} = yz - x^2 - c'xz - d'z^2$.
  Find $t'$ so that $(f^{t})^{t'} = yz - x^2$.
  ($t' = (x, y+c'x+d'z, z)$.) }

\item[(b)]
  \emph{Show that, up to projective equivalence,
  there is only one irreducible conic: $yz = x^2$.
  Any irreducible conic is nonsingular.} \\
\end{enumerate}



\emph{Proof of (a).}
\begin{enumerate}
\item[(1)]
  Given any conic
  \[
    f = ayz - bx^2 - cxz - dz^2
      + \alpha y^2 + \beta xy.
  \]

\item[(2)]
  $\alpha = 0$ since $P \in f$.

\item[(3)]
  \emph{Show that $a \neq 0$ and $\beta = 0$.}
  We dehomogenize $f$ w.r.t. $y$ to get
  $f_{*} = -bx^2 -cxz-dz^2 + az + \beta x$.
  Thus, the tangent line of $f_{*}$ at $(0,0)$ is $az + \beta x$.
  So $a \neq 0$ and $\beta = 0$ since $z$ is the tangent line to $f$ at $P$.

\item[(4)]
  \emph{Show that $b \neq 0$.}
  If $b = 0$,
  $f = ayz - cxz - dz^2 = z(ay - cx - dz)$ is reducible, which is absurd.

\item[(5)]
  Write
  \begin{align*}
    f
    =& \: ayz - bx^2 - cxz - dz^2 \\
    =& \: (\sqrt{a}y)(\sqrt{a}z)
      - (\sqrt{b}x)^2
      - \frac{c}{\sqrt{ab}} (\sqrt{b}x)(\sqrt{a}z)
      - \frac{d}{a} (\sqrt{a}z)^2.
  \end{align*}
  Apply a projective change of coordinates
  $t = \left( \frac{x}{\sqrt{b}}, \frac{y}{\sqrt{a}}, \frac{z}{\sqrt{a}} \right)$ to $f$:
  \[
    f^{t} = yz - x^2 - c'xz - d'z^2,
  \]
  where $c' = \frac{c}{\sqrt{ab}}$ and $d' = \frac{d}{a}$.

\item[(6)]
  Write
  \[
    f^{t}
    = yz - x^2 - c'xz - d'z^2
    = (y - c'x - d'z)z - x^2.
  \]
  Apply a projective change of coordinates $t' = (x, y+c'x+d'z, z)$ to $f^{t}$:
  \[
    (f^{t})^{t'} = yz - x^2.
  \]
\end{enumerate}
$\Box$ \\



\emph{Proof of (b).}
\begin{enumerate}
\item[(1)]
  Take one simple point $P_1 \in f$ and the corresponding tangent line $L$.
  (It is possible since Problem 5.8 and $|f| = \infty$.)
  Take $P_2 \in L - f$ and $P_3 \in f - L$.
  (It is possible since Problem 5.7 or 5.12 and both $|f|$ and $|L|$ are $\infty$.)

\item[(2)]
  Let $Q_1 = [0:1:0]$, $Q_2 = [1:0:0]$, $Q_3 = [0:0:1]$.
  Since $P_1, P_2, P_3$ (resp. $Q_1, Q_2, Q_3$) are not lying on a common line,
  there is a unique projective change of coordinates $t$ such that $t(P_i) = Q_i$
  ($i=1,2,3$) by Problem 4.14.

\item[(3)]
  Hence, $f$ is projectively equivalent to an irreducible conic $g$
  such that $P = [0:1:0]$ is a simple point on $g$ and
  $z=0$ is the tangent line to $g$ at $P$.
  By (a),
  \[
    f \sim g \sim (yz-x^2).
  \]

\item[(4)]
  Nonsingularity is preserved under projectively equivalence.
  Might assume that $f = yz-x^2$.
  $\frac{\partial f}{\partial x}(P)
  = \frac{\partial f}{\partial y}(P)
  = \frac{\partial f}{\partial z}(P)
  = 0$ implies that $P = [0:0:0]$, which is not in $\mathbf{P}^{2}$.
  Hence $f$ is nonsingular (Problem 5.1).

\end{enumerate}
$\Box$ \\\\



%%%%%%%%%%%%%%%%%%%%%%%%%%%%%%%%%%%%%%%%%%%%%%%%%%%%%%%%%%%%%%%%%%%%%%%%%%%%%%%%



\subsubsection*{Problem 5.10. (Cubic with a cusp)}
\addcontentsline{toc}{subsubsection}{Problem 5.10. (Cubic with a cusp)}
\emph{Let $f$ be an irreducible cubic, $P = [0:0:1]$ a cusp on $f$,
$y = 0$ the tangent line to $f$ at $P$.
Show that $f = ay^2z - bx^3 - cx^2y - dxy^2 - ey^3$.
Find projective changes of coordinates}
\begin{enumerate}
\item[(i)]
  \emph{to make $a = b = 1$;}

\item[(ii)]
  \emph{to make $c = 0$ (change $x$ to $x-\frac{c}{3}y$);}

\item[(iii)]
  \emph{to make $d = e = 0$ ($z$ to $z+dx+ey$).}
\end{enumerate}
\emph{Up to projective equivalence,
there is only one irreducible cubic with a cusp: $y^2z = x^3$.
It has no other singularities.} \\



\emph{Proof.}
\begin{enumerate}
\item[(1)]
  Given any cubic
  \begin{align*}
    f
    =& \: ay^2z - bx^3 - cx^2y - dxy^2 - ey^3 \\
      & + \alpha_1 z^3 + \alpha_2 x^2 z + \alpha_3 z^2 x + \alpha_4 z^2 y + \alpha_5 xyz.
  \end{align*}

\item[(2)]
  $\alpha_1 = 0$ since $P \in f$.

\item[(3)]
  \emph{Show that $b \neq 0$ and $\alpha_2 = \alpha_3 = 0$.}
  Since $P = [0:0:1]$ is a cusp on $f$ and $y = 0$ is the tangent line to $f$ at $P$,
  $I(f \cap y) = 3$.
  So
  \begin{align*}
    3
    &= I(P, f \cap y) \\
    &= I(P, (-bx^3 + \alpha_2 x^2 z + \alpha_3 z^2 x) \cap y) \\
    &= I(P, (-bx^3 + \alpha_2 x^2 + \alpha_3 x) \cap y)
      &(\text{Dehomogenize w.r.t. $z$})
  \end{align*}
  if and only if $b \neq 0$ and $\alpha_2 = \alpha_3 = 0$.

\item[(4)]
  \emph{Show that $a \neq 0$ and $\alpha_4 = \alpha_5 = 0$.}
  We dehomogenize $f$ w.r.t. $z$ to get
  \[
    f_{*}
    = -bx^3-cx^2y-dxy^2-ey^3 + ay^2 + \alpha_5 xy + \alpha_4 y.
  \]
  Since $(0,0)$ is a double point, $\alpha_4 = 0$.
  Since $f_{*}$ has only one tangent $y$ on $(0,0)$,
  \[
    ay^2 + \alpha_5 xy = y(ay + \alpha_5x)
  \]
  implies that $a \neq 0$ and $\alpha_5 = 0$.

\item[(5)]
  Therefore, $f = ay^2z - bx^3 - cx^2y - dxy^2 - ey^3$ with $a \neq 0$ and $b \neq 0$.
  Similar to Problem 5.9,
  \begin{enumerate}
  \item[(i)]
    Write
    \begin{align*}
      f
      =& \: ay^2z - bx^3 - cx^2y - dxy^2 - ey^3 \\
      =& \: (\sqrt[3]{a}y)^2(\sqrt[3]{a}z)
        - (\sqrt[3]{b}x)^3
        - \frac{c}{\sqrt[3]{ab^2}} (\sqrt[3]{b}x)^2(\sqrt[3]{a}y) \\
        &
        - \frac{d}{\sqrt[3]{a^2 b}} (\sqrt[3]{b}x)(\sqrt[3]{a}y)^2
        - \frac{e}{a} (\sqrt[3]{a}y)^3.
    \end{align*}
    Apply a projective change of coordinates
    $t_1 = \left( \frac{x}{\sqrt[3]{b}}, \frac{y}{\sqrt[3]{a}}, \frac{z}{\sqrt[3]{a}} \right)$
    to $f$:
    \[
      f^{t_1} = y^2z - x^3 - c_1 x^2y - d_1 xy^2 - e_1 y^3,
    \]
    where $c_1 = \frac{c}{\sqrt[3]{ab^2}}$ and
    $d_1 = \frac{d}{\sqrt[3]{a^2 b}}$ and
    $e_1 = \frac{e}{a}$.

  \item[(ii)]
    Write
    \begin{align*}
      f^{t_1}
      =& \: y^2z - x^3 - c_1 x^2y - d_1 xy^2 - e_1 y^3 \\
      =& \: y^2z - \left( x+\frac{c_1}{3}y \right)^3
        - \left( d_1-\frac{c_1^2}{3} \right) \left( x+\frac{c_1}{3}y \right)y^2 \\
        & \: - \left( e_1-\frac{c_1d_1}{3}+\frac{2c_1^3}{27} \right) y^3.
    \end{align*}
    Apply a projective change of coordinates
    $t_2 = \left( x-\frac{c_1}{3}y, y, z \right)$ to $f^{t_1}$:
    \[
      (f^{t_1})^{t_2} = y^2z - x^3 - d_2 xy^2 - e_2 y^3
    \]
    where $d_2 = d_1-\frac{c_1^2}{3}$ and $e_2 = e_1-\frac{c_1d_1}{3}+\frac{2c_1^3}{27}$.
    (Compare to Cardano's formula.)

  \item[(iii)]
    Write
    \begin{align*}
      (f^{t_1})^{t_2}
      &= y^2z - x^3 - d_2 xy^2 - e_2 y^3 \\
      &= y^2(z - d_2 x - e_2 y) - x^3.
    \end{align*}
    Apply a projective change of coordinates
    $t_3 = \left( x, y, z + d_2 x + e_2 y \right)$ to $(f^{t_1})^{t_2}$:
    \[
      ((f^{t_1})^{t_2})^{t_3} = y^2z - x^3.
    \]
  \end{enumerate}

\item[(6)]
  Given any irreducible cubic $f$ with a cusp.
  Apply the proof of Problem 5.9(b),
  $f$ is projectively equivalent to an irreducible cubic $g$
  such that $P = [0:0:1]$ is a cusp on $g$ and $y = 0$ is the tangent line to $g$ at $P$.
  By (5),
  \[
    f \sim g \sim (y^2 z-x^3).
  \]
  Similar to Problem 5.9(b),
  $y^2z - x^3$ has no other singularities except one point $P = [0:0:1]$.
\end{enumerate}
$\Box$ \\\\



%%%%%%%%%%%%%%%%%%%%%%%%%%%%%%%%%%%%%%%%%%%%%%%%%%%%%%%%%%%%%%%%%%%%%%%%%%%%%%%%



\subsubsection*{Problem 5.11. (Cubic with a node)}
\addcontentsline{toc}{subsubsection}{Problem 5.11. (Cubic with a node)}
\emph{Up to projective equivalence,
there is only one irreducible cubic with a node: $xyz = x^3+y^3$.
It has no other singularities.} \\



\emph{Proof.}
\begin{enumerate}
\item[(1)]
  \emph{Let $f$ be an irreducible cubic, $P = [0:0:1]$ a node on $f$,
  $x=0$, $y=0$ the tangent lines to $f$ at $P$.
  Show that $f = ax^3 + by^3 - cxyz - dx^2 y - e xy^2$, $a \neq 0$, $b \neq 0$, $c \neq 0$.}
  \begin{enumerate}
  \item[(a)]
    Given any cubic
    \begin{align*}
      f
      =& \: ax^3 + by^3 - cxyz - dx^2 y - e xy^2 \\
        & + \alpha_1 z^3 + \alpha_2 x^2 z + \alpha_3 y^2 z + \alpha_4 z^2 x + \alpha_5 z^2 y.
    \end{align*}

  \item[(b)]
    $\alpha_1 = 0$ since $P \in f$.

  \item[(c)]
    We dehomogenize $f$ w.r.t. $z$ to get
    \[
      f_{*}
      = \: ax^3 + by^3 - cxy - dx^2 y - e xy^2
        + \alpha_2 x^2 + \alpha_3 y^2 + \alpha_4 x + \alpha_5 y.
    \]
    Since $(0,0)$ is a double point, $\alpha_4 = \alpha_5 = 0$.
    Since $f_{*}$ have only two tangents $x, y$ on $(0,0)$,
    $- cxy + \alpha_2 x^2 + \alpha_3 y^2$
    implies that $c \neq 0$ and $\alpha_2 = \alpha_3 = 0$.

  \item[(d)]
    Now $f = ax^3 + by^3 - cxyz - dx^2 y - e xy^2$ with $c \neq 0$.
    If $a = 0$ or $b = 0$, $f$ is reducible, which is absurd.
    So $a \neq 0$ and $b \neq 0$.
  \end{enumerate}

\item[(2)]
  Hence,
  \[
    f
    \sim (x^3 + y^3 - xyz - dx^2 y - e xy^2)
    \sim (x^3 + y^3 - xyz)
  \]
  by
  $(x,y,z) \mapsto
  \left( \frac{1}{\sqrt[3]{a}}x, \frac{1}{\sqrt[3]{b}}y, \frac{\sqrt[3]{ab}}{c}z \right)$
  and
  $(x,y,z) \mapsto (x, y, z - dx - ey)$.

\item[(3)]
  Given any irreducible cubic $f$ with a node.
  Apply the proof of Problem 5.9(b),
  $f$ is projectively equivalent to an irreducible cubic $g$
  such that $P = [0:0:1]$ is a node on $g$ and $x, y$ are the tangent lines to $g$ at $P$.
  By (2),
  \[
    f \sim g \sim (x^3 + y^3 - xyz).
  \]
  Similar to Problem 5.9(b),
  $x^3 + y^3 - xyz$ has no other singularities except one point $P = [0:0:1]$.
\end{enumerate}
$\Box$ \\\\



%%%%%%%%%%%%%%%%%%%%%%%%%%%%%%%%%%%%%%%%%%%%%%%%%%%%%%%%%%%%%%%%%%%%%%%%%%%%%%%%



\subsubsection*{Problem 5.12.*}
\addcontentsline{toc}{subsubsection}{Problem 5.12.*}
\begin{enumerate}
\item[(a)]
  \emph{Assume $P = [0:1:0] \not\in f$, $f$ a curve of degree $n$.
  Show that}
  \[
    \sum_{P} I(P, f \cap x) = n.
  \]

\item[(b)]
  \emph{Show that if $f$ is a curve of degree $n$,
  $L$ a line not contained in $f$, then}
  \[
    \sum_{P} I(P, f \cap L) = n.
  \]
\end{enumerate}



\emph{Proof of (a).}
\begin{enumerate}
\item[(1)]
  Similar to Problem 5.7.
  The intersection points of $f$ and $x$ are the solutions of
  \[
    f(0,y,z) = 0.
  \]
  Write
  \[
    f(0,y,z) = c y^n + \cdots
  \]
  as a form in $k[y,z]$ of degree $n$.

\item[(2)]
  Note that $c \neq 0$ since $P = [0:1:0] \not\in f$.
  By Corollary to Proposition 5 in \S 2.6,
  \[
    f(0,y,z) = c \prod_{i}(y - \alpha_i z)^{r_i}
  \]
  where $\alpha_i \in k$ and $\sum r_i = n$.
  So all intersection points are $P_i = [0:\alpha_i:1]$
  with multiplicity $r_i$.

\item[(3)]
  Hence,
  \begin{align*}
    & \: \sum_{P} I(P, f \cap x) \\
    =& \: \sum_{i} I(P_i, f \cap x) \\
    =& \: \sum_{i} I(P_i, f(0,y,z) \cap x)
      &(\text{Property (7) in \S 3.3}) \\
    =& \: \sum_{i} \sum_{j} r_j I(P_i, (y - \alpha_j z) \cap x)
      &(\text{Property (6) in \S 3.3}) \\
    =& \: \sum_{i} \sum_{j} r_j I((0,\alpha_i), (y - \alpha_j) \cap x)
      &(\text{Dehomogenize w.r.t. $z$}) \\
    =& \: \sum_{i} \sum_{j} r_j I((0,0), (y + \alpha_i - \alpha_j) \cap x) \\
    =& \: \sum_{i} \sum_{j} r_j \delta_{ij} \\
    =& \: \sum_{i} r_i \\
    =& \: n.
  \end{align*}
\end{enumerate}
$\Box$ \\



\emph{Proof of (b).}
\begin{enumerate}
\item[(1)]
  Take any point $Q_1 \not\in f \cup L$
  (since $\mathbf{P}^2$ is irreducible).
  Suppose $L$ is a line passing $Q_2$ and $Q_3$, $Q_2 \neq Q_3$.

\item[(2)]
  By Problem 4.14,
  there is a projective change of coordinates $t: \mathbf{P}^2 \to \mathbf{P}^2$
  such that $t(Q_1) = [0:1:0]$ and $t(L) = x$.
  In particular, $t(Q_1) = [0:1:0] \not\in f^{t}$ (by the construction in (1)).

\item[(3)]
  Hence,
  \[
    \sum_{P} I(P, f \cap L)
    = \sum_{P} I(P, f^{t} \cap L^{t})
    = \sum_{P} I(P, f^{t} \cap x)
    = n.
  \]
\end{enumerate}
$\Box$ \\\\



%%%%%%%%%%%%%%%%%%%%%%%%%%%%%%%%%%%%%%%%%%%%%%%%%%%%%%%%%%%%%%%%%%%%%%%%%%%%%%%%



\subsubsection*{Problem 5.13.}
\addcontentsline{toc}{subsubsection}{Problem 5.13.}
\emph{Prove that an irreducible cubic is either nonsingular or
has at most one double point (a node or a cusp).
(Hint: Use Problem 5.12, where $L$ is a line through two
multiple points; or use Problems 5.10 and 5.11.)} \\



\emph{Proof (Problem 5.12).}
\begin{enumerate}
\item[(1)]
  (Reductio ad absurdum)
  Suppose there were a line $L$ passing through two multiple points $P, Q$
  of an irreducible cubic $f$.

\item[(2)]
  Note that
  \begin{align*}
    \sum I(P, f \cap L)
    &\geq I(P, f \cap L) + I(Q, f \cap L) \\
    &\geq m_{P}(f) m_{P}(L) + m_{Q}(f) m_{Q}(L) \\
    &\geq 2 \cdot 1 + 2 \cdot 1 \\
    &= 4.
  \end{align*}
  By Problem 5.12, $\sum I(P, f \cap L) = 3$, a contradiction.
  Hence $f$ has at most one multiple point.
  In particular, $f$ has at most one double point.
\end{enumerate}
$\Box$ \\



\emph{Proof (Problems 5.10 and 5.11).}
\begin{enumerate}
\item[(1)]
  Suppose one irreducible cubic $f$ has one multiple point $P$.
  It suffices to show that $P$ is a cusp or a node,
  or it suffices to show that $P$ is a double point.

\item[(2)]
  Up to projective equivalence, might assume that $P = [0:0:1]$.
  If $m_P(f) = m_P(f_{*}) = 3$, then $f_{*}$ is a form in $k[x,y]$ of degree $3$.
  Thus $f$ is in $k[x,y] \xhookrightarrow{} k[x,y,z]$, which is reducible.
  Hence $P$ is a double point.
  By Problems 5.10 and 5.11, $f$ has no other singularities.
\end{enumerate}
$\Box$ \\\\



%%%%%%%%%%%%%%%%%%%%%%%%%%%%%%%%%%%%%%%%%%%%%%%%%%%%%%%%%%%%%%%%%%%%%%%%%%%%%%%%



\subsubsection*{Problem 5.14.*}
\addcontentsline{toc}{subsubsection}{Problem 5.14.*}
\emph{Let $P_1,\ldots,P_n \in \mathbf{P}^{2}$.
Show that there are an infinite number of lines passing through $P_1$,
but not through $P_2, \ldots, P_n$.
If $P_1$ is a simple point on $f$,
we may take these lines transversal to $f$ at $P_1$.} \\



\emph{Proof.}
\begin{enumerate}
\item[(1)]
  Might assume that all $P_1,\ldots,P_n$ are distinct.
  Write a line $L_{ij}$ passing through $P_i$ and $P_j$.

\item[(2)]
  (Reductio ad absurdum)
  Suppose there were finite number of lines passing through $P_1$,
  but not through $P_2, \ldots, P_n$, say $M_1, \ldots, M_r$.
  Hence, we can write $\mathbf{P}^{2}$ as
  \[
    \mathbf{P}^{2}
    =
    L_{12} \cup \cdots \cup L_{1n} \cup M_1 \cup \cdots \cup M_r,
  \]
  contrary to the fact that $\mathbf{P}^{2}$ is irreducible.
\end{enumerate}
$\Box$ \\\\



%%%%%%%%%%%%%%%%%%%%%%%%%%%%%%%%%%%%%%%%%%%%%%%%%%%%%%%%%%%%%%%%%%%%%%%%%%%%%%%%



\subsubsection*{Problem 5.15.*}
\addcontentsline{toc}{subsubsection}{Problem 5.15.*}
\emph{Let $C$ be an irreducible projective plane curve,
$P_1,\ldots,P_n$ simple points on $C$, $m_1,\ldots,m_n$ integers.
Show that there is a $z \in k(C)$ with $\mathrm{ord}_{P_i}(z) = m_i$ for $i = 1,\ldots,n$.
(Hint: Take lines $L_i$ as in Problem 5.14 for $P_i$,
and a line $L_0$ not through any $P_j$,
and let $z = \prod L_i^{m_i} L_0^{-\sum m_i}$.)} \\



\emph{Proof.}
\begin{enumerate}
\item[(1)]
  Take lines $L_i$ not tangent to $C$ at $P_i$ as in Problem 5.14 for $P_i$.
  Since $\mathbf{P}^2 = V(0) \supsetneq V(C)$, we can pick a point $P$ outside $C$.
  Take one line $L_0$ through $P$ but not through any $P_j$ as in Problem 5.14.

\item[(2)]
  Define $z = \prod L_i^{m_i} / L_0^{\sum m_i} \in k(C)$ since
  $\prod L_i^{m_i}$ and $L_0^{\sum m_i}$ are in $\Gamma(C)$
  of the same degree $\sum m_i$.
  Note that $L_0(P_j) \neq 0$ for any $P_j$.
  So $\mathrm{ord}_{P_i}(z)$ is well-defined.
  Moreover,
  \[
    \mathrm{ord}_{P_i}(z)
    = \sum_j m_j \mathrm{ord}_{P_i}(L_j) - \left( \sum m_j \right) \mathrm{ord}_{P_i}(L_0)
    = m_i
  \]
  since $L_j$ ($j \neq i$) and $L_0$ are units in $k(C)$ and
  $L_i \ni P_i$ is not tangent to $C$.
\end{enumerate}
$\Box$ \\\\



%%%%%%%%%%%%%%%%%%%%%%%%%%%%%%%%%%%%%%%%%%%%%%%%%%%%%%%%%%%%%%%%%%%%%%%%%%%%%%%%



\subsubsection*{Problem 5.16.*}
\addcontentsline{toc}{subsubsection}{Problem 5.16.*}
\emph{Let $f$ be an irreducible curve in $\mathbf{P}^2$.
Suppose $I(P, f \cap z) = 1$, and $P \neq [1:0:0]$.
Show that $\frac{\partial f}{\partial x}(P) \neq 0$.
(Hint: If not, use Euler's Theorem to show that $\frac{\partial f}{\partial y}(P) = 0$;
but $z$ is not tangent to $f$ at $P$.)} \\



\emph{Proof.}
\begin{enumerate}
\item[(1)]
  Might write $P = [\alpha:1:0] \in \mathbf{P}^2$
  since $z(P) = 0$ (by $I(P, f \cap z) > 0$) and $P \neq [1:0:0]$.

\item[(2)]
  Since $I(P, f \cap z) = 1$, $m_P(f) = 1$ and $z$ is not tangent to $f$ at $P$.
  Thus, $P$ is a simple point of $f$.
  By Problem 5.4, the tangent line $L$ to $f$ at $P$ is
  \[
    \frac{\partial f}{\partial x}(P) x +
    \frac{\partial f}{\partial y}(P) y +
    \frac{\partial f}{\partial z}(P) z = 0.
  \]
  In particular, $P = [\alpha:1:0] \in L$, that is,
  \[
    \frac{\partial f}{\partial x}(P) \alpha +
    \frac{\partial f}{\partial y}(P) = 0.
  \]

\item[(3)]
  (Reductio ad absurdum)
  If $\frac{\partial f}{\partial x}(P) = 0$, $\frac{\partial f}{\partial y}(P) = 0$ by (2).
  Hence, the tangent line $L = z$, which is absurd.
\end{enumerate}
$\Box$ \\\\



%%%%%%%%%%%%%%%%%%%%%%%%%%%%%%%%%%%%%%%%%%%%%%%%%%%%%%%%%%%%%%%%%%%%%%%%%%%%%%%%
%%%%%%%%%%%%%%%%%%%%%%%%%%%%%%%%%%%%%%%%%%%%%%%%%%%%%%%%%%%%%%%%%%%%%%%%%%%%%%%%



\subsection*{5.2. Linear Systems of Curves \\}
\addcontentsline{toc}{subsection}{5.2. Linear Systems of Curves}



%%%%%%%%%%%%%%%%%%%%%%%%%%%%%%%%%%%%%%%%%%%%%%%%%%%%%%%%%%%%%%%%%%%%%%%%%%%%%%%%



\subsubsection*{Problem 5.18.}
\addcontentsline{toc}{subsubsection}{Problem 5.18.}
\emph{Show that there is only one conic passing through the five points
$[0:0:1]$, $[0:1:0]$, $[1:0:0]$, $[1:1:1]$, and $[1:2:3]$;
show that it is nonsingular.} \\



\emph{Proof.}
\begin{enumerate}
\item[(1)]
  Let $C: ax^2+bxy+cxz+dy^2+eyz+fz^2$ be the conic passing these five points.
  That is,
  \[
    \underbrace{\begin{pmatrix}
      0 & 0 & 0 & 0 & 0 & 1 \\
      0 & 0 & 0 & 1 & 0 & 0 \\
      1 & 0 & 0 & 0 & 0 & 0 \\
      1 & 1 & 1 & 1 & 1 & 1 \\
      1 & 2 & 3 & 4 & 6 & 9
    \end{pmatrix}}_{\text{say } A \in \mathsf{M}_{5 \times 6}(k)}
    \underbrace{\begin{pmatrix}
      a \\
      b \\
      c \\
      d \\
      e \\
      f
    \end{pmatrix}}_{\in \mathbf{P}^5(k)}
    = 0.
  \]

\item[(2)]
  The row echelon form $A'$ of $A$ is
  \[
    A'
    =
    \begin{pmatrix}
      1 & 0 & 0 & 0 & 0 & 0 \\
      0 & 1 & 0 & 0 & -3 & 0 \\
      0 & 0 & 1 & 0 & 4 & 0 \\
      0 & 0 & 0 & 1 & 0 & 0 \\
      0 & 0 & 0 & 0 & 0 & 1
    \end{pmatrix} \in \mathsf{M}_{5 \times 6}(k).
  \]
  $\rank(A') = 5$ and thus there is a unique solution in $\mathbf{P}^5$,
  say $[0:3:-4:0:1:0] \in \mathbf{P}^5$.
  Hence,
  \[
    C: 3xy - 4xz + yz = 0
  \]
  is only one conic passing through these five points.

\item[(3)]
  By solving the system of equations
  $C(P)
  = \frac{\partial C}{\partial x}(P)
  = \frac{\partial C}{\partial y}(P)
  = \frac{\partial C}{\partial z}(P) = 0$,
  $P = [0:0:0]$, which is absurd.
  Hence $C$ is nonsingular.
\end{enumerate}
$\Box$ \\\\



%%%%%%%%%%%%%%%%%%%%%%%%%%%%%%%%%%%%%%%%%%%%%%%%%%%%%%%%%%%%%%%%%%%%%%%%%%%%%%%%



\subsubsection*{Problem 5.19.}
\addcontentsline{toc}{subsubsection}{Problem 5.19.}
\emph{Consider the nine points
$[0:0:1]$, $[0:1:1]$, $[1:0:1]$, $[1:1:1]$, $[0:2:1]$,
$[2:0:1]$, $[1:2:1]$, $[2:1:1]$, and $[2:2:1] \in \mathbf{P}^2$ (Sketch).
Show that there are an infinite number of cubics passing through these points.} \\



\emph{Proof.}
\begin{enumerate}
\item[(1)]
  Let $C: a_1 x^3 + a_2 y^3 + a_3 z^3
    + a_4 x^2y + a_5 x^2z + a_6 y^2 x + a_7 y^2 z + a_8 z^2 x + a_9 z^2 y + a_{10} xyz$
    be the cubic passing these five points.
  That is,
  \[
    \underbrace{\begin{pmatrix}
      0 & 0 & 1 & 0 & 0 & 0 & 0 & 0 & 0 & 0 \\
      0 & 1 & 1 & 0 & 0 & 0 & 1 & 0 & 1 & 0 \\
      1 & 0 & 1 & 0 & 1 & 0 & 0 & 1 & 0 & 0 \\
      1 & 1 & 1 & 1 & 1 & 1 & 1 & 1 & 1 & 1 \\
      0 & 8 & 1 & 0 & 0 & 0 & 4 & 0 & 2 & 0 \\
      8 & 0 & 1 & 0 & 4 & 0 & 0 & 2 & 0 & 0 \\
      1 & 8 & 1 & 2 & 1 & 4 & 4 & 1 & 2 & 2 \\
      8 & 1 & 1 & 4 & 4 & 2 & 1 & 2 & 1 & 2 \\
      8 & 8 & 1 & 8 & 4 & 8 & 4 & 2 & 2 & 4
    \end{pmatrix}}_{\text{say } A \in \mathsf{M}_{9 \times 10}(k)}
    \underbrace{\begin{pmatrix}
      a_1 \\
      \vdots \\
      a_{10} \\
    \end{pmatrix}}_{\in \mathbf{P}^9(k)}
    = 0.
  \]

\item[(2)]
  The row echelon form $A'$ of $A$ is
  \[
    A'
    =
    \begin{pmatrix}
      1 & 0 & 0 & 0 & 0 & 0 & 0 & \frac{-1}{2} &            0 & 0 \\
      0 & 1 & 0 & 0 & 0 & 0 & 0 &            0 & \frac{-1}{2} & 0 \\
      0 & 0 & 1 & 0 & 0 & 0 & 0 &            0 &            0 & 0 \\
      0 & 0 & 0 & 1 & 0 & 0 & 0 &            0 &            0 & 0 \\
      0 & 0 & 0 & 0 & 1 & 0 & 0 &  \frac{3}{2} &            0 & 0 \\
      0 & 0 & 0 & 0 & 0 & 1 & 0 &            0 &            0 & 0 \\
      0 & 0 & 0 & 0 & 0 & 0 & 1 &            0 &  \frac{3}{2} & 0 \\
      0 & 0 & 0 & 0 & 0 & 0 & 0 &            0 &            0 & 1 \\
      0 & 0 & 0 & 0 & 0 & 0 & 0 &            0 &            0 & 0
    \end{pmatrix} \in \mathsf{M}_{9 \times 10}(k).
  \]
  $\rank(A') = 8 < 9$ and thus
  there are an infinite number of cubics passing through these points, say
  \[
    C: \lambda x(x-z)(x-2z) + \mu y(y-z)(y-2z),
  \]
  where $[\lambda:\mu] \in \mathbf{P}^2(k)$.
\end{enumerate}
$\Box$ \\\\



%%%%%%%%%%%%%%%%%%%%%%%%%%%%%%%%%%%%%%%%%%%%%%%%%%%%%%%%%%%%%%%%%%%%%%%%%%%%%%%%
%%%%%%%%%%%%%%%%%%%%%%%%%%%%%%%%%%%%%%%%%%%%%%%%%%%%%%%%%%%%%%%%%%%%%%%%%%%%%%%%



\subsection*{5.3. B\'ezout's Theorem \\}
\addcontentsline{toc}{subsection}{5.3. B\'ezout's Theorem}



\subsubsection*{Problem 5.20.}
\addcontentsline{toc}{subsubsection}{Problem 5.20.}
\emph{Check your answers of Problem 5.3 with B\'ezout's Theorem} \\



\emph{Proof.}
\begin{enumerate}
\item[(1)]
  $f = y^2z - x(x-2z)(x+z)$ and $g = y^2 + x^2 - 2xz$.
  \begin{align*}
    I([0:0:1], f \cap g) &= 2, \\
    I([2:0:1], f \cap g) &= 2, \\
    I([-2,\sqrt{-8}:1], f \cap g) &= 1, \\
    I([-2,-\sqrt{-8}:1], f \cap g) &= 1.
  \end{align*}
  Hence, $\sum_P I(P, f \cap g) = 2 + 2 + 1 + 1 = 6 = \deg(f) \deg(g)$.

\item[(2)]
  $f = (x^2+y^2)z + x^3 + y^3$ and $g = x^3 + y^3 - 2xyz$.
  \begin{align*}
    I([0:0:1], f \cap g) &= 4, \\
    I([1:-1:0], f \cap g) &= 3, \\
    I([1:\sqrt[3]{-1}:0], f \cap g) &= 1, \\
    I([1:1-\sqrt[3]{-1}:0], f \cap g) &= 1.
  \end{align*}
  Hence, $\sum_P I(P, f \cap g) = 4 + 3 + 1 + 1 = 9 = \deg(f) \deg(g)$.

\item[(3)]
  $f = y^5 - x(y^2-xz)^2$ and $g = y^4 + y^3z - x^2z^2$.
  \begin{align*}
    I([0:0:1], f \cap g) &= 9, \\
    I([1:0:0], f \cap g) &= 9, \\
    I\left(\left[ \frac{1}{2}:1:2+2\sqrt{2} \right], f \cap g\right) &= 1, \\
    I\left(\left[ \frac{1}{2}:1:2-2\sqrt{2} \right], f \cap g\right) &= 1.
  \end{align*}
  Hence, $\sum_P I(P, f \cap g) = 9 + 9 + 1 + 1 = 20 = \deg(f) \deg(g)$.

\item[(4)]
  $f = (x^2+y^2)^2 + 3x^2yz - y^3z$ and $g = (x^2+y^2)^3 - 4x^2y^2z^2$.
  \begin{align*}
    I([1:\sqrt{-1}:0], f \cap g) &= 3, \\
    I([1:-\sqrt{-1}:0], f \cap g) &= 3, \\
    I([0:0:1], f \cap g) &= 14, \\
    I\left(\left[ \frac{\sqrt{5+2\sqrt{5}}}{4} : \frac{-\sqrt{5}}{4} : 1 \right],
      f \cap g\right) &= 1, \\
    I\left(\left[ -\frac{\sqrt{5+2\sqrt{5}}}{4} : \frac{-\sqrt{5}}{4} : 1 \right],
      f \cap g\right) &= 1, \\
    I\left(\left[ \frac{\sqrt{5-2\sqrt{5}}}{4} : \frac{\sqrt{5}}{4} : 1 \right],
      f \cap g\right) &= 1, \\
    I\left(\left[ -\frac{\sqrt{5-2\sqrt{5}}}{4} : \frac{\sqrt{5}}{4} : 1 \right],
      f \cap g\right) &= 1.
  \end{align*}
  Hence, $\sum_P I(P, f \cap g) = 3 + 3 + 14 + 1 + 1 + 1 + 1 = 24 = \deg(f) \deg(g)$.
\end{enumerate}
$\Box$ \\\\



%%%%%%%%%%%%%%%%%%%%%%%%%%%%%%%%%%%%%%%%%%%%%%%%%%%%%%%%%%%%%%%%%%%%%%%%%%%%%%%%



% http://math.ucsd.edu/~doprea/145s08/145sol6.pdf

\subsubsection*{Problem 5.21.*}
\addcontentsline{toc}{subsubsection}{Problem 5.21.*}
\emph{Show that every nonsingular projective plane curve is irreducible.
Is this true for affine curves?} \\



\emph{Proof.}
\begin{enumerate}
\item[(1)]
  Let $f$ be a nonsingular projective plane curve.
  (Reductio ad absurdum)
  Suppose $f = gh$ were reducible.
  Then $g$ and $h$ intersects in $\mathbf{P}^2$ (B\'ezout's Theorem).

\item[(2)]
  \emph{Show that each intersection point $P$ is a singularity of $f$.}
  Note that $g(P) = h(P) = 0$ and the chain rule for derivatives.
  Thus,
  \[
    \frac{\partial f}{\partial x_i}(P)
    = g(P) \frac{\partial h}{\partial x_i}(P) + \frac{\partial g}{\partial x_i}(P) h(P) = 0
  \]
  for $i = 1,2,3$.
  $P$ is a singularity of $f$ (Problem 5.1).

\item[(3)]
  \emph{Show that it is not true for affine curves.}
  $V(x^2 - x) \subseteq \mathbf{A}^2$ is nonsingular but not irreducible.
\end{enumerate}
$\Box$ \\\\



%%%%%%%%%%%%%%%%%%%%%%%%%%%%%%%%%%%%%%%%%%%%%%%%%%%%%%%%%%%%%%%%%%%%%%%%%%%%%%%%



% http://people.math.sfu.ca/~kyeats/teaching/math818soln2.pdf

\subsubsection*{Problem 5.22.*}
\addcontentsline{toc}{subsubsection}{Problem 5.22.*}
\emph{Let $f$ be an irreducible curve of degree $n$.
Assume $\frac{\partial f}{\partial x} \neq 0$.
Apply Corollary 1 to $f$ and $\frac{\partial f}{\partial x}$,
and conclude that $\sum m_P(f)(m_P(f)-1) \leq n(n-1)$.
In particular, $f$ has at most $\frac{n(n-1)}{2}$ multiple points.
(See Problems 5.6, 5.8.)} \\



\emph{Proof.}
\begin{enumerate}
\item[(1)]
  There is nothing to do for $n = 1$.
  Might assume that $n > 1$.

\item[(2)]
  Since $f$ and $\frac{\partial f}{\partial x}$ have no common component (Problem 5.8),
  Corollary 1 to B\'ezout's Theorem gives us
  \[
    \sum m_P(f) m_P\left( \frac{\partial f}{\partial x} \right)
    \leq \deg(f) \deg\left( \frac{\partial f}{\partial x} \right)
    \leq n(n-1).
  \]
  By Problem 5.6, the conclusion is established.

\item[(3)]
  Then
  \begin{align*}
    n(n-1)
    &\geq \sum_{P} m_P(f)(m_P(f)-1) \\
    &\geq \sum_{m_P(f) > 1} 2(2-1) \\
    &\geq \sum_{m_P(f) > 1} 2 \\
    &= 2 \cdot (\text{number of multiple points}).
  \end{align*}
\end{enumerate}
$\Box$ \\\\



%%%%%%%%%%%%%%%%%%%%%%%%%%%%%%%%%%%%%%%%%%%%%%%%%%%%%%%%%%%%%%%%%%%%%%%%%%%%%%%%
%%%%%%%%%%%%%%%%%%%%%%%%%%%%%%%%%%%%%%%%%%%%%%%%%%%%%%%%%%%%%%%%%%%%%%%%%%%%%%%%



\subsection*{5.4. Multiple Points \\}
\addcontentsline{toc}{subsection}{5.4. Multiple Points}



%%%%%%%%%%%%%%%%%%%%%%%%%%%%%%%%%%%%%%%%%%%%%%%%%%%%%%%%%%%%%%%%%%%%%%%%%%%%%%%%
%%%%%%%%%%%%%%%%%%%%%%%%%%%%%%%%%%%%%%%%%%%%%%%%%%%%%%%%%%%%%%%%%%%%%%%%%%%%%%%%



\subsection*{5.5. Max Noether's Fundamental Theorem \\}
\addcontentsline{toc}{subsection}{5.5. Max Noether's Fundamental Theorem}



%%%%%%%%%%%%%%%%%%%%%%%%%%%%%%%%%%%%%%%%%%%%%%%%%%%%%%%%%%%%%%%%%%%%%%%%%%%%%%%%
%%%%%%%%%%%%%%%%%%%%%%%%%%%%%%%%%%%%%%%%%%%%%%%%%%%%%%%%%%%%%%%%%%%%%%%%%%%%%%%%



\subsection*{5.6. Applications of Noether's Theorem \\}
\addcontentsline{toc}{subsection}{5.6. Applications of Noether's Theorem}



%%%%%%%%%%%%%%%%%%%%%%%%%%%%%%%%%%%%%%%%%%%%%%%%%%%%%%%%%%%%%%%%%%%%%%%%%%%%%%%%
%%%%%%%%%%%%%%%%%%%%%%%%%%%%%%%%%%%%%%%%%%%%%%%%%%%%%%%%%%%%%%%%%%%%%%%%%%%%%%%%
%%%%%%%%%%%%%%%%%%%%%%%%%%%%%%%%%%%%%%%%%%%%%%%%%%%%%%%%%%%%%%%%%%%%%%%%%%%%%%%%
%%%%%%%%%%%%%%%%%%%%%%%%%%%%%%%%%%%%%%%%%%%%%%%%%%%%%%%%%%%%%%%%%%%%%%%%%%%%%%%%



\newpage
\section*{Chapter 6: Varieties, Morphisms, and Rational Maps \\}
\addcontentsline{toc}{section}{Chapter 6: Varieties, Morphisms, and Rational Maps}



\subsection*{6.1. The Zariski Topology \\}
\addcontentsline{toc}{subsection}{6.1. The Zariski Topology}



\subsubsection*{Problem 6.1.*}
\addcontentsline{toc}{subsubsection}{Problem 6.1.*}
\emph{Let $Z \subseteq Y \subseteq X$, $X$ a topological space.
Give $Y$ the induced topology.
Show that the topology induced by $Y$ on $Z$ is the same as that induced by $X$ on $Z$.} \\



\emph{Proof.}
\begin{enumerate}
\item[(1)]
  Take any open set $U$ in $Z$ whose topology is induced by $Y$.
  $U = Z \cap B$ for some open subset $B$ in $Y$.
  Since $Y$ is induced by $X$.
  $A = Y \cap A$ for some open subset $A$ in $X$.
  Hence,
  \[
    U = Z \cap B = Z \cap (Y \cap A) = Z \cap A,
  \]
  that is, $U$ is also an open set in $Z$ whose topology is induced by $X$.

\item[(2)]
  Conversely, take any open set $U$ in $Z$ whose topology is induced by $X$.
  $U = Z \cap A$ for some open subset $A$ in $X$.
  Hence
  \[
    U = Z \cap A = Z \cap (Y \cap A),
  \]
  where $Y \cap A$ is an open set in $Y$ whose topology is induced by $X$.
  So $U$ is also an open set in $Z$ whose topology is induced by $Y$.
\end{enumerate}
$\Box$ \\\\



%%%%%%%%%%%%%%%%%%%%%%%%%%%%%%%%%%%%%%%%%%%%%%%%%%%%%%%%%%%%%%%%%%%%%%%%%%%%%%%%



\subsubsection*{Problem 6.3.*}
\addcontentsline{toc}{subsubsection}{Problem 6.3.*}
\begin{enumerate}
\item[(a)]
  \emph{Let $V$ be an affine variety, $f \in \Gamma(V)$.
  Considering $f$ as a mapping from $V$ to $k = \mathbf{A}^{1}$,
  show that $f$ is continuous.}

\item[(b)]
  \emph{Show that any polynomial map of affine varieties is continuous.} \\
\end{enumerate}



\emph{Proof of (a).}
  It is a special case of (b) since $f$ itself is a polynomial map from
  $V$ to $\mathbf{A}^{1}$.
$\Box$ \\



\emph{Proof of (b).}
\begin{enumerate}
\item[(1)]
  Let $V \subseteq \mathbf{A}^n$, $W \subseteq \mathbf{A}^m$ be affine varieties.
  Let $\varphi$ be a polynomial map from $V$ to $W$.

\item[(2)]
  Problem 2.7 shows that $\varphi^{-1}(X)$ is closed for every closed subset $X$ of $W$.
  Hence, $\varphi$ is continuous.
\end{enumerate}
$\Box$ \\\\



%%%%%%%%%%%%%%%%%%%%%%%%%%%%%%%%%%%%%%%%%%%%%%%%%%%%%%%%%%%%%%%%%%%%%%%%%%%%%%%%
%%%%%%%%%%%%%%%%%%%%%%%%%%%%%%%%%%%%%%%%%%%%%%%%%%%%%%%%%%%%%%%%%%%%%%%%%%%%%%%%



\subsection*{6.2. Varieties \\}
\addcontentsline{toc}{subsection}{6.2. Varieties}



%%%%%%%%%%%%%%%%%%%%%%%%%%%%%%%%%%%%%%%%%%%%%%%%%%%%%%%%%%%%%%%%%%%%%%%%%%%%%%%%
%%%%%%%%%%%%%%%%%%%%%%%%%%%%%%%%%%%%%%%%%%%%%%%%%%%%%%%%%%%%%%%%%%%%%%%%%%%%%%%%



\subsection*{6.3. Morphisms of Varieties \\}
\addcontentsline{toc}{subsection}{6.3. Morphisms of Varieties}



%%%%%%%%%%%%%%%%%%%%%%%%%%%%%%%%%%%%%%%%%%%%%%%%%%%%%%%%%%%%%%%%%%%%%%%%%%%%%%%%
%%%%%%%%%%%%%%%%%%%%%%%%%%%%%%%%%%%%%%%%%%%%%%%%%%%%%%%%%%%%%%%%%%%%%%%%%%%%%%%%



\subsection*{6.4. Products and Graphs \\}
\addcontentsline{toc}{subsection}{6.4. Products and Graphs}



%%%%%%%%%%%%%%%%%%%%%%%%%%%%%%%%%%%%%%%%%%%%%%%%%%%%%%%%%%%%%%%%%%%%%%%%%%%%%%%%
%%%%%%%%%%%%%%%%%%%%%%%%%%%%%%%%%%%%%%%%%%%%%%%%%%%%%%%%%%%%%%%%%%%%%%%%%%%%%%%%



\subsection*{6.5. Algebraic Function Fields and Dimension of Varieties \\}
\addcontentsline{toc}{subsection}{6.5. Algebraic Function Fields and Dimension of Varieties}



%%%%%%%%%%%%%%%%%%%%%%%%%%%%%%%%%%%%%%%%%%%%%%%%%%%%%%%%%%%%%%%%%%%%%%%%%%%%%%%%
%%%%%%%%%%%%%%%%%%%%%%%%%%%%%%%%%%%%%%%%%%%%%%%%%%%%%%%%%%%%%%%%%%%%%%%%%%%%%%%%



\subsection*{6.6. Rational Maps \\}
\addcontentsline{toc}{subsection}{6.6. Rational Maps}



%%%%%%%%%%%%%%%%%%%%%%%%%%%%%%%%%%%%%%%%%%%%%%%%%%%%%%%%%%%%%%%%%%%%%%%%%%%%%%%%
%%%%%%%%%%%%%%%%%%%%%%%%%%%%%%%%%%%%%%%%%%%%%%%%%%%%%%%%%%%%%%%%%%%%%%%%%%%%%%%%
%%%%%%%%%%%%%%%%%%%%%%%%%%%%%%%%%%%%%%%%%%%%%%%%%%%%%%%%%%%%%%%%%%%%%%%%%%%%%%%%
%%%%%%%%%%%%%%%%%%%%%%%%%%%%%%%%%%%%%%%%%%%%%%%%%%%%%%%%%%%%%%%%%%%%%%%%%%%%%%%%



\newpage
\section*{Chapter 7: Resolution of Singularities \\}
\addcontentsline{toc}{section}{Chapter 7: Resolution of Singularities}



\subsection*{7.1. Rational Maps of Curves \\}
\addcontentsline{toc}{subsection}{7.1. Rational Maps of Curves}



%%%%%%%%%%%%%%%%%%%%%%%%%%%%%%%%%%%%%%%%%%%%%%%%%%%%%%%%%%%%%%%%%%%%%%%%%%%%%%%%
%%%%%%%%%%%%%%%%%%%%%%%%%%%%%%%%%%%%%%%%%%%%%%%%%%%%%%%%%%%%%%%%%%%%%%%%%%%%%%%%



\subsection*{7.2. Blowing up a Point in $\mathbf{A}^{2}$ \\}
\addcontentsline{toc}{subsection}{7.2. Blowing up a Point in $\mathbf{A}^{2}$}



%%%%%%%%%%%%%%%%%%%%%%%%%%%%%%%%%%%%%%%%%%%%%%%%%%%%%%%%%%%%%%%%%%%%%%%%%%%%%%%%
%%%%%%%%%%%%%%%%%%%%%%%%%%%%%%%%%%%%%%%%%%%%%%%%%%%%%%%%%%%%%%%%%%%%%%%%%%%%%%%%



\subsection*{7.3. Blowing up a Point in $\mathbf{P}^{2}$ \\}
\addcontentsline{toc}{subsection}{7.3. Blowing up a Point in $\mathbf{P}^{2}$}



%%%%%%%%%%%%%%%%%%%%%%%%%%%%%%%%%%%%%%%%%%%%%%%%%%%%%%%%%%%%%%%%%%%%%%%%%%%%%%%%
%%%%%%%%%%%%%%%%%%%%%%%%%%%%%%%%%%%%%%%%%%%%%%%%%%%%%%%%%%%%%%%%%%%%%%%%%%%%%%%%



\subsection*{7.4. Quadratic Transformations \\}
\addcontentsline{toc}{subsection}{7.4. Quadratic Transformations}



%%%%%%%%%%%%%%%%%%%%%%%%%%%%%%%%%%%%%%%%%%%%%%%%%%%%%%%%%%%%%%%%%%%%%%%%%%%%%%%%
%%%%%%%%%%%%%%%%%%%%%%%%%%%%%%%%%%%%%%%%%%%%%%%%%%%%%%%%%%%%%%%%%%%%%%%%%%%%%%%%



\subsection*{7.5. Nonsingular Models of Curves \\}
\addcontentsline{toc}{subsection}{7.5. Nonsingular Models of Curves}



%%%%%%%%%%%%%%%%%%%%%%%%%%%%%%%%%%%%%%%%%%%%%%%%%%%%%%%%%%%%%%%%%%%%%%%%%%%%%%%%
%%%%%%%%%%%%%%%%%%%%%%%%%%%%%%%%%%%%%%%%%%%%%%%%%%%%%%%%%%%%%%%%%%%%%%%%%%%%%%%%
%%%%%%%%%%%%%%%%%%%%%%%%%%%%%%%%%%%%%%%%%%%%%%%%%%%%%%%%%%%%%%%%%%%%%%%%%%%%%%%%
%%%%%%%%%%%%%%%%%%%%%%%%%%%%%%%%%%%%%%%%%%%%%%%%%%%%%%%%%%%%%%%%%%%%%%%%%%%%%%%%



\newpage
\section*{Chapter 8: Riemann-Roch Theorem \\}
\addcontentsline{toc}{section}{Chapter 8: Riemann-Roch Theorem}



\subsection*{8.1. Divisors \\}
\addcontentsline{toc}{subsection}{8.1. Divisors}



%%%%%%%%%%%%%%%%%%%%%%%%%%%%%%%%%%%%%%%%%%%%%%%%%%%%%%%%%%%%%%%%%%%%%%%%%%%%%%%%
%%%%%%%%%%%%%%%%%%%%%%%%%%%%%%%%%%%%%%%%%%%%%%%%%%%%%%%%%%%%%%%%%%%%%%%%%%%%%%%%



\subsection*{8.2. The Vector Spaces $L(D)$ \\}
\addcontentsline{toc}{subsection}{8.2. The Vector Spaces $L(D)$}



%%%%%%%%%%%%%%%%%%%%%%%%%%%%%%%%%%%%%%%%%%%%%%%%%%%%%%%%%%%%%%%%%%%%%%%%%%%%%%%%
%%%%%%%%%%%%%%%%%%%%%%%%%%%%%%%%%%%%%%%%%%%%%%%%%%%%%%%%%%%%%%%%%%%%%%%%%%%%%%%%



\subsection*{8.3. Riemann's Theorem \\}
\addcontentsline{toc}{subsection}{8.3. Riemann's Theorem}



%%%%%%%%%%%%%%%%%%%%%%%%%%%%%%%%%%%%%%%%%%%%%%%%%%%%%%%%%%%%%%%%%%%%%%%%%%%%%%%%
%%%%%%%%%%%%%%%%%%%%%%%%%%%%%%%%%%%%%%%%%%%%%%%%%%%%%%%%%%%%%%%%%%%%%%%%%%%%%%%%



\subsection*{8.4. Derivations and Differentials \\}
\addcontentsline{toc}{subsection}{8.4. Derivations and Differentials}



%%%%%%%%%%%%%%%%%%%%%%%%%%%%%%%%%%%%%%%%%%%%%%%%%%%%%%%%%%%%%%%%%%%%%%%%%%%%%%%%
%%%%%%%%%%%%%%%%%%%%%%%%%%%%%%%%%%%%%%%%%%%%%%%%%%%%%%%%%%%%%%%%%%%%%%%%%%%%%%%%



\subsection*{8.5. Canonical Divisors \\}
\addcontentsline{toc}{subsection}{8.5. Canonical Divisors}



%%%%%%%%%%%%%%%%%%%%%%%%%%%%%%%%%%%%%%%%%%%%%%%%%%%%%%%%%%%%%%%%%%%%%%%%%%%%%%%%
%%%%%%%%%%%%%%%%%%%%%%%%%%%%%%%%%%%%%%%%%%%%%%%%%%%%%%%%%%%%%%%%%%%%%%%%%%%%%%%%



\subsection*{8.6. Riemann-Roch Theorem \\}
\addcontentsline{toc}{subsection}{8.6. Riemann-Roch Theorem}



%%%%%%%%%%%%%%%%%%%%%%%%%%%%%%%%%%%%%%%%%%%%%%%%%%%%%%%%%%%%%%%%%%%%%%%%%%%%%%%%
%%%%%%%%%%%%%%%%%%%%%%%%%%%%%%%%%%%%%%%%%%%%%%%%%%%%%%%%%%%%%%%%%%%%%%%%%%%%%%%%



\end{document}