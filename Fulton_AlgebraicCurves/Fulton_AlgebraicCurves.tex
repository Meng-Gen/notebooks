\documentclass{article}
\usepackage{amsfonts}
\usepackage{amsmath}
\usepackage{amssymb}
\usepackage{hyperref}
\usepackage[none]{hyphenat}
\usepackage{mathrsfs}
\parindent=0pt



\title{\textbf{Solutions to the book: \\\emph{Fulton, Algebraic Curves}}}
\author{Meng-Gen Tsai \\ plover@gmail.com}



\begin{document}
\maketitle
\tableofcontents



%%%%%%%%%%%%%%%%%%%%%%%%%%%%%%%%%%%%%%%%%%%%%%%%%%%%%%%%%%%%%%%%%%%%%%%%%%%%%%%%
%%%%%%%%%%%%%%%%%%%%%%%%%%%%%%%%%%%%%%%%%%%%%%%%%%%%%%%%%%%%%%%%%%%%%%%%%%%%%%%%



\newpage
\section*{Chapter 1: Affine Algebraic Sets \\}
\addcontentsline{toc}{section}{Chapter 1: Affine Algebraic Sets}



\subsection*{1.1. Algebraic Preliminaries \\}
\addcontentsline{toc}{subsection}{1.1. Algebraic Preliminaries}



\subsubsection*{Problem 1.1.*}
\addcontentsline{toc}{subsubsection}{Problem 1.1.*}
\emph{Let $R$ be a domain.}
\begin{enumerate}
\item[(a)]
  \emph{If $f$, $g$ are forms of degree $r$, $s$ respectively in $R[x_1,\ldots,x_n]$,
  show that $fg$ is a form of degree $r+s$.}

\item[(b)]
  \emph{Show that any factor of a form in $R[x_1,\ldots,x_n]$ is also a form. } \\
\end{enumerate}

\emph{Proof of (a).}
\begin{enumerate}
\item[(1)]
  Write
  \begin{align*}
    f &= \sum_{(i)} a_{(i)} x^{(i)}, \\
    g &= \sum_{(j)} b_{(j)} x^{(j)},
  \end{align*}
  where $\sum_{(i)}$ is the summation over $(i) = (i_1,\ldots,i_n)$ with $i_1+\cdots+i_n = r$
  and $\sum_{(j)}$ is the summation over $(j) = (j_1,\ldots,j_n)$ with $j_1+\cdots+j_n = s$.

\item[(2)]
  Hence,
  \begin{align*}
    fg
    &= \sum_{(i)} \sum_{(j)} a_{(i)}b_{(j)} x^{(i)}x^{(j)} \\
    &= \sum_{(i),(j)} a_{(i)}b_{(j)} x^{(k)}
  \end{align*}
  where $(k) = (i_1+j_1,\ldots,i_n+j_n)$ with $(i_1+j_1)+\cdots+(i_n+j_n) = r+s$.
  Each $x^{(k)}$ is the form of degree $r+s$ and $a_{(i)}b_{(j)} \in R$.
  Hence $fg$ is a form of degree $r+s$.
\end{enumerate}
$\Box$\\



\emph{Proof of (b).}
\begin{enumerate}
\item[(1)]
  Given any form $f \in R[x_1,\ldots,x_n]$, and write $f = gh$.
  \emph{It suffices to show that $g$ is a form as well.}
  (So does $h$.)

\item[(2)]
  Write
  \[
    g = g_0+\cdots+g_r,
    \qquad
    h = h_0+\cdots+h_s
  \]
  where $g_r \neq 0$ and $h_s \neq 0$.
  So
  \[
    f = gh = g_0h_0 + \cdots + g_r h_s.
  \]
  Since $R$ is a domain, $R[x_1,\ldots,x_n]$ is a domain and thus $g_r h_s \neq 0$.
  The maximality of $r$ and $s$ implies that $\deg f = r+s$.
  Therefore, by the maximality of $r+s$,
  $f = g_r h_s$, or $g = g_r$, or $g$ is a form.
\end{enumerate}
$\Box$\\\\



%%%%%%%%%%%%%%%%%%%%%%%%%%%%%%%%%%%%%%%%%%%%%%%%%%%%%%%%%%%%%%%%%%%%%%%%%%%%%%%%



\subsubsection*{Problem 1.2.*}
\addcontentsline{toc}{subsubsection}{Problem 1.2.*}
\emph{Let $R$ be a UFD,
$K$ the quotient field of $R$.
Show that every element $z$ of $K$ may be written $z = a/b$,
where $a, b \in R$ have no common factors;
this representative is unique up to units of $R$.} \\

\emph{Proof.}
\begin{enumerate}
\item[(1)]
  \emph{Show that every element $z$ of $K$ may be written $z=a/b$,
  where $a$, $b\in R$ have no common factors.}
  Given any $z = a/b \in K$ where $a, b\in R$.
  Write
  \begin{align*}
    a &= p_1 \cdots p_n, \\
    b &= q_1 \cdots q_m
  \end{align*}
  where all $p_1, \ldots, p_n, q_1, \ldots, q_m$ are irreducible in $R$.
  (It is possible since $R$ is a UFD.)
  For each $i$, suppose $p_i \mid q_j$ for some $i, j$.
  Write $q_j = p_i u$ for some $u \in R$.
  By the irreducibility of $p_i$ and $q_j$, $u$ is a unit.
  So
  \[
    z
    = \frac{a}{b}
    = \frac{p_1 \cdots \widehat{p_i} \cdots p_n}{q_1 \cdots \widehat{q_j} \cdots q_m}
    = \frac{p_1 \cdots \widehat{p_i} \cdots p_n}{u q_1 \cdots \widehat{q_j} \cdots q_m}.
  \]
  Continue this method we can write $z = \frac{a'}{b'}$ where $a'$ and $b'$ have no common factors.

\item[(2)]
  Write $z = a/b = a'/b'$ where
  \begin{enumerate}
  \item[(a)]
    $a, b, a', b' \in R$,

  \item[(b)]
    $a$ and $b$ have no common factors,

  \item[(c)]
    $a'$ and $b'$ have no common factors.
  \end{enumerate}
  Write
  \begin{align*}
    a &= p_1 \cdots p_n, \\
    b &= q_1 \cdots q_m, \\
    a' &= p'_1 \cdots p'_{n'}, \\
    b' &= q'_1 \cdots q'_{m'}
  \end{align*}
  where all $p_i, q_j, p'_{i'}, q'_{j'}$ are irreducible in $R$.
  As $z = a/b = a'/b'$, $ab' = a'b$ or
  \[
    p_1 \cdots p_n q'_1 \cdots q'_{m'}
    = p'_1 \cdots p'_{n'} q_1 \cdots q_m.
  \]

\item[(3)]
  For $i = 1$, $p_1 = u_1 p'_{i'}$ for some unit $u_1 \in R$
  since $a$ and $b$ have no common factors and all $p_1, q_j, p'_{i'}$ are irreducible.
  Hence
  \[
    u_1 \widehat{p_1} p_2 \cdots p_n q'_1 \cdots q'_{m'}
    = p'_1 \cdots \widehat{p'_{i'}} \cdots p'_{n'} q_1 \cdots q_m.
  \]
  Continue this method,
  we have $n \leq n'$ and all $p_1, \ldots, p_n$ are canceled.

\item[(4)]
  Conversely, we can apply the argument in (3) to $i' = 1, \ldots n'$ to conclude that $n' \leq n$.
  Therefore, $n = n'$ and
  \[
    \underbrace{u_1 \cdots u_n}_{\text{a unit in $R$}} q'_1 \cdots q'_{m'} = q_1 \cdots q_m.
  \]
  Hence, $b = ub'$ where $u = u_1 \cdots u_n$ is a unit in $R$.
  Similarly, $a = va'$ where $v$ is a unit in $R$.
  So the representative of $z \in K$ is unique up to units of $R$.
\end{enumerate}
$\Box$\\\\



%%%%%%%%%%%%%%%%%%%%%%%%%%%%%%%%%%%%%%%%%%%%%%%%%%%%%%%%%%%%%%%%%%%%%%%%%%%%%%%%



\subsubsection*{Problem 1.3.*}
\addcontentsline{toc}{subsubsection}{Problem 1.3.*}
\emph{Let $R$ be a PID. Let $\mathfrak{p}$ be a nonzero, proper, prime ideal in $R$.}
\begin{enumerate}
\item[(a)]
  \emph{Show that $\mathfrak{p}$ is generated by an irreducible element.}

\item[(b)]
  \emph{Show that $\mathfrak{p}$ is maximal.} \\
\end{enumerate}



\emph{Proof of (a).}
\begin{enumerate}
\item[(1)]
  Let $\mathfrak{p} = (a)$ be a nonzero, proper, prime ideal in $R$.
  It suffices to show that $a$ is irreducible.

\item[(2)]
  Suppose $a = bc$.
  By the primality of $\mathfrak{p}$, $b \in \mathfrak{p}$ or $c \in \mathfrak{p}$.
  Suppose $b \in \mathfrak{p} = (a)$. (The case $c \in \mathfrak{p}$ is similar.)
  Then there is a $d \in R$ such that $b = ad$.
  Hence, $a = bc = adc$ or $(1-dc)a = 0$.

\item[(3)]
  Since $R$ is a domain, $1 = dc$ or $a = 0$.
  $a = 0$ implies that $\mathfrak{p} = (0)$ is a zero ideal, contrary to the assumption.
  Therefore, $1 = dc$, or $c$ is a unit, or $a$ is irreducible.
\end{enumerate}
$\Box$\\



\emph{Proof of (b).}
\begin{enumerate}
\item[(1)]
  Given any ideal $I = (b)$ of $R$ containing $\mathfrak{p} = (a)$.
  As the generator $a$ of $\mathfrak{p}$ is in $\mathfrak{p} \subseteq I$,
  there is some $c \in R$ such that $a = bc$.
  By the irreducibility of $a$ (in (a)), $b$ is a unit or $c$ is a unit.

\item[(2)]
  $b$ is a unit implies that $I = R$.
  $c$ is a unit implies that $I = \mathfrak{p}$.
  In any case, we conclude that $\mathfrak{p}$ is maximal.
\end{enumerate}
$\Box$\\\\



%%%%%%%%%%%%%%%%%%%%%%%%%%%%%%%%%%%%%%%%%%%%%%%%%%%%%%%%%%%%%%%%%%%%%%%%%%%%%%%%



\subsubsection*{Problem 1.4.*}
\addcontentsline{toc}{subsubsection}{Problem 1.4.*}
\emph{Let $k$ be an infinite field,
$f \in k[x_1,\ldots,x_n]$.
Suppose $f(a_1,\ldots,a_n) = 0$ for all $a_1, \ldots, a_n \in k$.
Show that $f = 0$.
(Hint: Write
\[
  f = \sum f_i x_n^{i},
  \qquad
  f_i \in k[x_1,\ldots,x_{n-1}].
\]
Use induction on $n$,
and the fact that $f(a_1, \ldots, a_{n-1}, x_n)$
has only a finite number of roots if any $f_i(a_1, \ldots, a_{n-1}) \neq 0$.)} \\



\emph{Proof.}
\begin{enumerate}
\item[(1)]
  Induction on $n$.
  The case $n = 1$.
  (Reductio ad absurdum)
  If there were a nonzero $f \in k[x_1]$ such that $f(a) = 0$ for all $a \in k$.
  Note that $f$ has at most $\deg f < \infty$ roots,
  contrary to the infinity of $k$.

\item[(2)]
  Assume that the conclusion holds for $n - 1$, then for any $f \in k[x_1,\ldots,x_n]$
  we can write
  \[
    f = \sum f_i x_n^{i},
    \qquad
    f_i \in k[x_1,\ldots,x_{n-1}]
  \]
  as $f \in (k[x_1,\ldots,x_{n-1}])[x_n]$.
  Suppose $f(a_1,\ldots,a_n) = 0$ for all $a_1, \ldots, a_n \in k$.
  For fixed $a_1, \ldots, a_{n-1}$,
  the polynomial $f(a_1, \ldots, a_{n-1}, x_n) \in k[x_n]$ has all distinct roots in
  an infinite field $k$.
  By (1), $f(a_1, \ldots, a_{n-1}, x_n) = 0 \in k[x_n]$,
  or each $f_i(a_1, \ldots, a_{n-1}) = 0$.
  As all $a_1, \ldots, a_{n-1}$ run over $k$,
  we can apply the induction hypothesis
  each $f_i(x_1, \ldots, x_{n-1}) = 0 \in k[x_1,\ldots,x_{n-1}]$.
  Hence, $f = 0 \in k[x_1,\ldots,x_{n}]$.
\end{enumerate}
$\Box$\\



\emph{Note.}
If $k$ is a finite field of order $q = p^k$,
then the polynomial $f(x) = x^q - x$ has $q$ distinct roots in $k$. \\\\



%%%%%%%%%%%%%%%%%%%%%%%%%%%%%%%%%%%%%%%%%%%%%%%%%%%%%%%%%%%%%%%%%%%%%%%%%%%%%%%%



\subsubsection*{Problem 1.5.*}
\addcontentsline{toc}{subsubsection}{Problem 1.5.*}
\emph{Let $k$ be any field.
Show that there are an infinitely number of irreducible monic polynomials in $k[x]$.
(Hint: Suppose $f_1,\ldots,f_n$ were all of them, and factor $f_1\cdots f_n+1$ into irreducible factors.)} \\

\emph{Proof (Due to Euclid).}
\begin{enumerate}
\item[(1)]
  If
  $f_1, \ldots, f_n$ were all irreducible monic polynomials, then
  we consider
  \[
    g = f_1 \cdots f_n + 1 \in k[x].
  \]
  So there is an irreducible monic polynomial $f = f_i$ dividing $g$ for some $i$
  since
  \[
    \deg g = \deg f_1 + \cdots + \deg f_n \geq 1
  \]
  and $k[x]$ is a UFD.

\item[(2)]
  However, $f$ would divide the difference
  \[
    g - f_1 \cdots f_{i-1} f_i f_{i+1} \cdots f_n = 1,
  \]
  contrary to $\deg f_i \geq 1$.
\end{enumerate}
$\Box$\\\\



%%%%%%%%%%%%%%%%%%%%%%%%%%%%%%%%%%%%%%%%%%%%%%%%%%%%%%%%%%%%%%%%%%%%%%%%%%%%%%%%



\subsubsection*{Problem 1.6.*}
\addcontentsline{toc}{subsubsection}{Problem 1.6.*}
\emph{Show that any algebraically closed field is infinite.
(Hint: The irreducible monic polynomials are $x - a$, $a \in k$.)} \\

\emph{Proof (Due to Euclid).}
\begin{enumerate}
\item[(1)]
  Let $k$ be an algebraically closed field.
  If $a_1, \ldots, a_n$ were all elements in $k$, then
  we consider a monic polynomials
  \[
    f(x) = (x - a_1) \cdots (x - a_n) + 1 \in k[x].
  \]

\item[(2)]
  Since $k$ is algebraically closed,
  there is an element $a \in k$ such that $f(a) = 0$.
  By assumption, $a = a_i$ for some $1 \leq i \leq n$,
  and thus $f(a) = f(a_i) = 1$, contrary to the fact that
  a field is a commutative ring where $0 \neq 1$ and all nonzero elements are invertible.
\end{enumerate}
$\Box$\\\\



%%%%%%%%%%%%%%%%%%%%%%%%%%%%%%%%%%%%%%%%%%%%%%%%%%%%%%%%%%%%%%%%%%%%%%%%%%%%%%%%



\subsubsection*{Problem 1.7.*}
\addcontentsline{toc}{subsubsection}{Problem 1.7.*}
\emph{Let $k$ be a field, $f \in k[x_1, \ldots, x_n]$, $a_1, \ldots, a_n \in k$.}
\begin{enumerate}
\item[(a)]
  \emph{Show that}
  \[
    f = \sum \lambda_{(i)} (x_1-a_1)^{i_1} \cdots (x_n-a_n)^{i_n},
    \qquad
    \lambda_{(i)} \in k.
  \]

\item[(b)]
  \emph{If $f(a_1, \ldots, a_n) = 0$,
  show that $f = \sum_{i=1}^n (x_i-a_i) g_i$ for some (not unique) $g_i$ in $k[x_1, \ldots, x_n]$.} \\
\end{enumerate}



\emph{Proof of (a).}
\begin{enumerate}
\item[(1)]
  Regard $k[x_1, \ldots, x_n]$ as $(k[x_1, \ldots, x_{n-1}])[x_n]$.
  Since $(k[x_1, \ldots, x_{n-1}])[x_n]$ is a Euclidean domain with a function
  \[
    f \in (k[x_1, \ldots, x_{n-1}])[x_n] \mapsto \deg_{x_n} f \in \mathbb{Z}_{\geq 0}
  \]
  satisfying the division-with-remainder property.

\item[(2)]
  Apply the division algorithm for $f$ and nonzero $x_n-a_n$
  to produce a quotient $q$ and remainder $r$ with
  $f = (x_n-a_n) q + r$ and either $r = 0$ or $\deg_{x_n}(r) < \deg_{x_n} (x_n-a_n) = 1$.
  That is, $r \in k[x_1, \ldots, x_{n-1}]$ is a constant in $(k[x_1, \ldots, x_{n-1}])[x_n]$.
  Continue this process to get that $f$ is of the form
  \[
    f = \sum_{i_n} f_{i_n} (x_n - a_n)^{i_n}
  \]
  where $f_{i_n} \in k[x_1, \ldots, x_{n-1}]$.

\item[(3)]
  Use the same argument in (2) for each $f_{i_n} \in k[x_1, \ldots, x_{n-1}]$, we have
  \begin{align*}
    f_{i_n}
    &= \sum_{i_{n-1}} \underbrace{f_{i_n,i_{n-1}}}_{\in k[x_1, \ldots, x_{n-2}]}
      (x_{n-1} - a_{n-1})^{i_{n-1}} \\
    f_{i_n,i_{n-1}}
    &=
    \sum_{i_{n-2}} \underbrace{f_{i_n,i_{n-1},i_{n-2}}}_{\in k[x_1, \ldots, x_{n-3}]}
      (x_{n-2} - a_{n-2})^{i_{n-2}}, \\
    & \cdots \\
    f_{i_n,\ldots,i_{2}}
    &= \sum_{i_1} \underbrace{f_{i_n,\ldots,i_1}}_{\in k} (x_1 - a_1)^{i_1}.
  \end{align*}
  Note that $f_{i_n,\ldots,i_1} \in k$, we can write
  \[
    f = \sum \lambda_{(i)} (x_1-a_1)^{i_1} \cdots (x_n-a_n)^{i_n},
    \qquad
    \lambda_{(i)} \in k.
  \]
  by replacing all $f_{i_n,\ldots,i_k}$ by $f_{i_n,\ldots,i_{k-1}}$
  for $k = n, n-1, \ldots, 2$.

\item[(4)]
  Or use the induction on $n$.
\end{enumerate}
$\Box$\\



\emph{Proof of (b).}
\begin{enumerate}
\item[(1)]
  Write
  \[
    f = \sum \lambda_{(i)} (x_1-a_1)^{i_1} \cdots (x_n-a_n)^{i_n},
    \qquad
    \lambda_{(i)} \in k
  \]
  by (a).

\item[(2)]
  As $f(a_1, \cdots, a_n) = 0$,
  $\lambda_{(i)} = 0$ if all $i_1, \ldots, i_n$ are zero, that it,
  there is no nonzero constant term in the representation of $f$.
  Hence, for each term
  \[
    f_{(i)} : = \lambda_{(i)} (x_1-a_1)^{i_1} \cdots (x_n-a_n)^{i_n}
  \]
  with $\lambda_{(i)} \neq 0$,
  there exists one $i_k > 0$ for some $1 \leq k \leq n$.
  So we can write
  \[
    f_{(i)}
    =
    (x_k-a_k)
      \underbrace{
        (\lambda_{(i)} (x_1-a_1)^{i_1} \cdots (x_k-a_k)^{i_k-1} \cdots (x_n-a_n)^{i_n})}_{
          := g_{(i)} \in k[x_1,\ldots,x_n]}.
  \]
  Note that the expression of $f_{(i)}$ is not unique since
  there may exist more than one $i_k > 0$ as $1 \leq k \leq n$.

\item[(3)]
  Now we iterate each nonzero term in $f$, apply the factorization in (2),
  and then group by each $x_k-a_k$.
  Therefore, we can write
  \[
    f = \sum_{i=1}^{n}(x_i-a_i)g_i
  \]
  for some $g_1 \in k[x_1, \ldots, x_n]$.

\item[(4)]
  The expression of $f$ is not unique.
  For example, take $f(x,y) = x^2 + 2xy + y^2 \in k[x,y]$.
  As $f(0,0) = 0$, we can write
  \begin{align*}
    f(x,y)
    &= x \cdot \underbrace{(x+2y)}_{g_1} + y \cdot \underbrace{y}_{g_2}, \text{ or } \\
    &= x \cdot \underbrace{(x+y)}_{g_1} + y \cdot \underbrace{(x+y)}_{g_2}, \text{ or } \\
    &= x \cdot \underbrace{x}_{g_1} + y \cdot \underbrace{(2x+y)}_{g_2}.
  \end{align*}
\end{enumerate}
$\Box$\\\\



%%%%%%%%%%%%%%%%%%%%%%%%%%%%%%%%%%%%%%%%%%%%%%%%%%%%%%%%%%%%%%%%%%%%%%%%%%%%%%%%
%%%%%%%%%%%%%%%%%%%%%%%%%%%%%%%%%%%%%%%%%%%%%%%%%%%%%%%%%%%%%%%%%%%%%%%%%%%%%%%%



\subsection*{1.2. Affine Space and Algebraic Sets \\}
\addcontentsline{toc}{subsection}{1.2. Affine Space and Algebraic Sets}



\subsubsection*{Problem 1.8.*}
\addcontentsline{toc}{subsubsection}{Problem 1.8.*}
\emph{Show that the algebraic subsets of $\mathbf{A}^1(k)$ are just the finite subsets, together
with $\mathbf{A}^1(k)$ itself.} \\

\emph{Proof.}
\begin{enumerate}
\item[(1)]
  \emph{Show that $k[x]$ is a PID if $k$ is a field.}
  \begin{enumerate}
  \item[(a)]
    Let $I$ be an ideal of $k[x]$.

  \item[(b)]
    If $I = \{0\}$ then $I = (0)$ and $I$ is principal.

  \item[(c)]
    If $I \neq \{0\}$, then take $f$ to be a polynomial of minimal degree in $I$.
    It suffices to show that $I = (f)$.
    Clearly, $(f) \subseteq I$ since $I$ is an ideal.
    Conversely, for any $g \in I$,
    \[
      g(x) = f(x)h(x) + r(x)
    \]
    for some $h, r \in k[x]$ with $r = 0$ or $\deg r < \deg f$ (as $k[x]$ is a Euclidean domain).
    Now as
    \[
      r = g - fh \in I,
    \]
    $r = 0$ (otherwise contrary to the minimality of $f$),
    we have $g = fh \in (f)$ for all $g \in I$.
  \end{enumerate}

\item[(2)]
  Let $Y$ be an algebraic subset of $\mathbf{A}^1(k)$,
  say $Y = V(I)$ for some ideal $I$ of $k[x]$.
  Since $k[x]$ is a PID, $I = (f)$ for some $f \in k[x]$.
  \begin{enumerate}
  \item[(a)]
    If $f = 0$, then $I = (0)$ and $Y = V(0) = \mathbf{A}^1(k)$.

  \item[(b)]
    If $f \neq 0$, then $f(x) = 0$ has finitely many roots in $k$,
    say $a_1, \ldots, a_m \in k$.
    Hence,
    \[
      Y = V(I) = V(f) = \{ f(a) = 0 : a \in k \}
      = \{ a_1, \ldots, a_m \}
    \]
    is a finite subsets of $\mathbf{A}^1(k)$.
  \end{enumerate}
  By (a)(b), the result is established.
\end{enumerate}
$\Box$\\



\emph{Notes.}
\begin{enumerate}
\item[(1)]
  By the Hilbert basis theorem, $k[x]$ is Noetherian as $k$ is Noetherian.
  Hence, for any algebraic subset $Y = V(I)$ of $\mathbf{A}^1(k)$,
  we can write $I = (f_1, \cdots, f_m)$.
  Note that
  \[
    Y = V(I) = V(f_1) \cap \cdots \cap V(f_m).
  \]
  Now apply the same argument to get the same conclusion.

\item[(2)]
  Suppose $k = \overline{k}$.
  $\mathbf{A}^1(k)$ is irreducible, because its only proper closed subsets are finite,
  yet it is infinite
  (because $k$ is algebraically closed, hence infinite). \\
\end{enumerate}



%%%%%%%%%%%%%%%%%%%%%%%%%%%%%%%%%%%%%%%%%%%%%%%%%%%%%%%%%%%%%%%%%%%%%%%%%%%%%%%%



\subsubsection*{Problem 1.9.}
\addcontentsline{toc}{subsubsection}{Problem 1.9.}
\emph{If $k$ is a finite field, show that every subset of $\mathbf{A}^{n}(k)$ is algebraic.} \\

\emph{Proof.}
\begin{enumerate}
\item[(1)]
  Every subset of $\mathbf{A}^{n}(k)$ is finite since
  $|\mathbf{A}^{n}(k)| = |k|^n$ is finite.

\item[(2)]
  Note that $V(x_1-a_1,\ldots,x_n-a_n) = \{ (a_1,\ldots,a_n) \} \subseteq \mathbf{A}^{n}(k)$
  (Property (5) in \S 1.2)
  and any finite union of algebraic sets is algebraic (Property (4) in \S 1.2).
  Thus, every subset of $\mathbf{A}^{n}(k)$ is algebraic (by (1)).
\end{enumerate}
$\Box$\\\\



%%%%%%%%%%%%%%%%%%%%%%%%%%%%%%%%%%%%%%%%%%%%%%%%%%%%%%%%%%%%%%%%%%%%%%%%%%%%%%%%



\subsubsection*{Problem 1.10.}
\addcontentsline{toc}{subsubsection}{Problem 1.10.}
\emph{Give an example of a countable collection of algebraic sets whose union is not
algebraic.} \\

\emph{Proof.}
\begin{enumerate}
\item[(1)]
  Let $k = \mathbb{Q}$ be an infinite field.
  $V(x-a) = \{ a \}$ is an algebraic sets for all $a \in \mathbb{Q}$.
  In particular, $V(x-a) = \{ a \}$ is algebraic for all $a \in \mathbb{Z}$.

\item[(2)]
  Note that
  \[
    Y := \bigcup_{a \in \mathbb{Z}} V(x-a) = \mathbb{Z}
  \]
  is a countable union of algebraic sets.
  Since $Y$ is a proper subset of $k = \mathbb{Q}$,
  it cannot be algebraic by Problem 1.8.
\end{enumerate}
$\Box$\\\\



%%%%%%%%%%%%%%%%%%%%%%%%%%%%%%%%%%%%%%%%%%%%%%%%%%%%%%%%%%%%%%%%%%%%%%%%%%%%%%%%



\subsubsection*{Problem 1.11.}
\addcontentsline{toc}{subsubsection}{Problem 1.11.}
\emph{Show that the following are algebraic sets:}
\begin{enumerate}
\item[(a)]
  $\{ (t,t^2,t^3) \in \mathbf{A}^{3}(k) : t \in k \}$;

\item[(b)]
  $\{ (\cos(t),\sin(t)) \in \mathbf{A}^{2}(\mathbb{R}) : t \in \mathbb{R} \}$;

\item[(c)]
  \emph{the set of points in $\mathbf{A}^{2}(\mathbb{R})$
  whose polar coordinates $(r,\theta)$ satisfy the equation $r = \sin(\theta)$.} \\
\end{enumerate}



\emph{Proof of (a).}
\begin{enumerate}
\item[(1)]
  The twisted cubic curve
  \[
    Y = \{ (t,t^2,t^3) \in \mathbf{A}^3(k) : t \in k \}
    =
    V(x^2-y) \cap V(x^3-z)
  \]
  is algebraic.
  We say that $Y$ is given by the parametric representation $x=t$, $y=t^2$, $z=t^3$.

\item[(2)]
  The generators for the ideal $I(Y)$ are $x^2-y$ and $x^3-z$.

\item[(3)]
  $Y$ is an affine variety of dimension $1$.

\item[(4)]
  The affine coordinate ring $A(Y)$ is isomorphic to a polynomial ring in one variable over $k$.
\end{enumerate}
$\Box$\\



\emph{Proof of (b).}
The circle
\[
  \{(\cos(t),\sin(t)) \in \mathbf{A}^2(\mathbb{R}) : t \in \mathbb{R} \} = V(x^2-y^2-1)
\]
is algebraic.
$\Box$\\



\emph{Proof of (c).}
The circle
\[
  \{ (r,\theta) : r = \sin(\theta) \} = V(x^2+y^2-y)
\]
is algebraic again.
$\Box$\\\\



%%%%%%%%%%%%%%%%%%%%%%%%%%%%%%%%%%%%%%%%%%%%%%%%%%%%%%%%%%%%%%%%%%%%%%%%%%%%%%%%



\subsubsection*{Problem 1.12.}
\addcontentsline{toc}{subsubsection}{Problem 1.12.}
\emph{Suppose $C$ is an affine plane curve,
and $L$ is a line in $\mathbf{A}^2(k)$,
$L \not\subseteq C$.
Suppose $C = V(f)$, $f \in k[x,y]$ a polynomial of degree $n$.
Show that $L \cap C$ is a finite set of no more than $n$ points.
(Hint: Suppose $L = V(y - (ax + b))$, and consider
$f(x,ax +b) \in k[x]$.)} \\

\emph{Proof.}
\begin{enumerate}
\item[(1)]
  Say $L = V(y - (ax + b))$ be a line in $\mathbf{A}^2(k)$.
  (The case $L = V(x - (ay + b))$ is similar.)

\item[(2)]
  Note that $L \not\subseteq C$ implies that $(y - (ax + b)) \nmid f$.
  Hence, the polynomial
  \[
    g: x \mapsto f(x,ax + b) \in k[x]
  \]
  is nonzero and $\deg g \leq n$.
  Therefore, the number of roots of $g$ in $k$ is no more than $n$.

\item[(3)]
  Hence,
  \begin{align*}
    L \cap C
    &= V(y - (ax + b)) \cap V(f) \\
    &= \{ (x,y) \in \mathbf{A}^2(k) : y = ax + b \text{ and } f(x,y) = 0 \} \\
    &= \{ (x,y) \in \mathbf{A}^2(k) : f(x,ax + b) = 0 \}
  \end{align*}
  is finite of no more than $n$ points.
\end{enumerate}
$\Box$\\\\



%%%%%%%%%%%%%%%%%%%%%%%%%%%%%%%%%%%%%%%%%%%%%%%%%%%%%%%%%%%%%%%%%%%%%%%%%%%%%%%%



\subsubsection*{Problem 1.13.}
\addcontentsline{toc}{subsubsection}{Problem 1.13.}
\emph{Show that each of the following sets is not algebraic:}
\begin{enumerate}
\item[(a)]
  $\{ (x,y) \in \mathbf{A}^{2}(\mathbb{R}) : y = \sin(x) \}$.

\item[(b)]
  \emph{$\{ (z,w) \in \mathbf{A}^{2}(\mathbb{C}) : |z|^2 + |w|^2 = 1 \}$,
  where $|x+iy|^2 = x^2 + y^2$ for $x, y \in \mathbb{R}$.}

\item[(c)]
  $\{ (\cos(t), \sin(t), t) \in \mathbf{A}^3(\mathbb{R}) : t \in \mathbb{R} \}$. \\
\end{enumerate}



\emph{Proof of (a).}
\begin{enumerate}
\item[(1)]
  (Reductio ad absurdum)
  If
  \[
    Y := \{ (x,y) \in \mathbf{A}^{2}(\mathbb{R}) : y = \sin(x) \}
  \]
  were algebraic,
  then there is a subset $S$ of $\mathbb{R}[x,y]$ such that
  \[
    Y = V(S) = \bigcap_{f \in S} V(f).
  \]

\item[(2)]
  $S \neq \varnothing$ since $Y \neq \mathbf{A}^{2}(\mathbb{R})$.
  ($(89,64) \in \mathbf{A}^{2}(\mathbb{R})-Y$.)

\item[(3)]
  Take a fixed line $L = V(y)$ in $\mathbf{A}^{2}(\mathbb{R})$.
  For each affine curve $f \in S$, we have
  \[
    V(f) \cap L
    \supseteq
    \bigcap_{f \in S} V(f) \cap L
    = Y \cap L
    = \{ (n\pi,0) \in \mathbf{A}^{2}(\mathbb{R}) : n \in \mathbb{Z} \},
  \]
  which is infinite.
  By problem 1.12, $y \mid f$.
  As $f$ runs over $S$, $Y \subseteq V(y) = L$,
  contradicts that $\left(0,\frac{\pi}{2}\right) \in L-Y$.
\end{enumerate}
$\Box$\\\\



\emph{Proof of (b).}
\begin{enumerate}
\item[(1)]
  Similar to (a).
  (Reductio ad absurdum)
  If
  \[
    Y := \{ (x,y) \in \mathbf{A}^{2}(\mathbb{C}) : |x|^2 + |y|^2 = 1 \}
  \]
  were algebraic,
  then there is a subset $S$ of $\mathbb{C}[x,y]$ such that
  \[
    Y = V(S) = \bigcap_{f \in S} V(f).
  \]

\item[(2)]
  $S \neq \varnothing$ since $Y \neq \mathbf{A}^{2}(\mathbb{C})$.
  ($(89,64) \in \mathbf{A}^{2}(\mathbb{C})-Y$.)

\item[(3)]
  Take a fixed line $L = V(x)$ in $\mathbf{A}^{2}(\mathbb{C})$.
  For each affine curve $f \in S$, we have
  \[
    V(f) \cap L
    \supseteq
    \bigcap_{f \in S} V(f) \cap L
    = Y \cap L
    = \{ (0,y) \in \mathbf{A}^{2}(\mathbb{C}) : |y| = 1 \},
  \]
  which is infinite (since $Y$ contains a unit circle in the complex plane).
  By problem 1.12, $x \mid f$.
  As $f$ runs over $S$, $Y \subseteq V(x) = L$,
  contradicts that the origin $(0,0) \in L-Y$.
\end{enumerate}
$\Box$\\



\emph{Proof of (c).}
\begin{enumerate}
\item[(1)]
  Similar to (a) and (b).

\item[(2)]
  \emph{Suppose $C$ is an affine plane curve,
  and $L$ is a line in $\mathbf{A}^3(k)$,
  $L \not\subseteq C$.
  Suppose $C = V(f)$, $f \in k[x,y,z]$ a polynomial of degree $n$.
  Show that $L \cap C$ is a finite set of no more than $n$ points.}
  The proof is similar to Problem 1.12.
  \begin{enumerate}
  \item[(a)]
    Say $L = V(y - (ax + b), z - (cx + d))$ be a line in $\mathbf{A}^3(k)$.

  \item[(b)]
    Note that $L \not\subseteq C$ implies that
    $(y - (ax + b)) \nmid f$ and $(z - (cx + d)) \nmid f$.
    Hence, the polynomial
    \[
      g: x \mapsto f(x,ax + b, cx + d) \in k[x]
    \]
    is nonzero and $\deg g \leq n$.
    Therefore, the number of roots of $g$ in $k$ is no more than $n$.

  \item[(c)]
    Hence,
    \begin{align*}
      L \cap C
      &= V(y - (ax + b), z - (cx + d)) \cap V(f) \\
      &= \{ (x,y) \in \mathbf{A}^2(k) : y = ax + b, z = cx + d \text{ and } f(x,y) = 0 \} \\
      &= \{ (x,y) \in \mathbf{A}^2(k) : f(x, ax + b, cx + d) = 0 \}
    \end{align*}
    is finite of no more than $n$ points.
  \end{enumerate}

\item[(3)]
  (Reductio ad absurdum)
  If
  \[
    Y := \{ (\cos(t), \sin(t), t) \in \mathbf{A}^3(\mathbb{R}) : t \in \mathbb{R} \}
  \]
  were algebraic,
  then there is a subset $S$ of $\mathbb{R}[x,y,z]$ such that
  \[
    Y = V(S) = \bigcap_{f \in S} V(f).
  \]

\item[(4)]
  $S \neq \varnothing$ since $Y \neq \mathbf{A}^{3}(\mathbb{R})$.
  ($(1989,6,4) \in \mathbf{A}^{3}(\mathbb{R})-Y$.)

\item[(5)]
  Take a fixed line $L = V(x-1,y)$ in $\mathbf{A}^{3}(\mathbb{R})$.
  For each affine curve $f \in S$, we have
  \[
    V(f) \cap L
    \supseteq
    \bigcap_{f \in S} V(f) \cap L
    = Y \cap L
    = \{ (1,0,2n\pi) \in \mathbf{A}^{3}(\mathbb{R}) : n \in \mathbb{Z} \},
  \]
  which is infinite.
  By (2), $(x-1) \mid f$ and $y \mid f$.
  As $f$ runs over $S$, $Y \subseteq V(x-1,y) = L$,
  contradicts that $(1,0,\pi) \in L-Y$.
\end{enumerate}
$\Box$\\

\textbf{Supplement.}
\emph{A circular disk of radius $1$ in the plane $xy$ rolls without slipping
along the $x$ axis.
The figure described by a point of the circumference of of the disk is
called a \textbf{cycloid}.
The parametrized curve $\alpha: \mathbb{R} \to \mathbb{R}^2$
is
\begin{equation*}
  \begin{cases}
     x = t - \sin t \\
     y = 1 - \cos t.
  \end{cases}
\end{equation*}
The cycloid is not algebraic (as (a)).} \\\\



%%%%%%%%%%%%%%%%%%%%%%%%%%%%%%%%%%%%%%%%%%%%%%%%%%%%%%%%%%%%%%%%%%%%%%%%%%%%%%%%



\subsubsection*{Problem 1.14.*}
\addcontentsline{toc}{subsubsection}{Problem 1.14.*}
\emph{Let $f$ be a nonconstant polynomial in $k[x_1, \ldots, x_n]$,
$k$ algebraically closed.
Show that $\mathbf{A}^{n}(k) - V(f)$ is infinite if $n \geq 1$,
and $V(f)$ is infinite if $n \geq 2$.
Conclude that the complement of any proper algebraic set is infinite.
(Hint: See Problem 1.4.)} \\

\emph{Proof.}
\begin{enumerate}
\item[(1)]
  \emph{Show that $\mathbf{A}^{n}(k) - V(f)$ is infinite if $n \geq 1$.}
  Since $f$ is a nonconstant polynomial in $k[x_1, \ldots, x_n]$,
  we may assume that $\deg_{x_n}(f) > 0$.
  Hence
  \[
    x_n \mapsto f(1,\ldots,1,x_n)
  \]
  is a nonconstant polynomial of degree $\deg_{x_n}(f) > 0$ in $k[x_n]$.
  So $f$ has finitely many roots in $k$, say $\xi_1, \ldots, \xi_m$ ($m \geq 0$).
  Hence,
  \[
    (1,\ldots,1,x_n) \neq 0
  \]
  whenever $x_n \neq \xi_m$.
  Such subset in $\mathbf{A}^{1}(k)$ is infinite since $k = \overline{k}$ (Problem 1.6).
  Therefore,
  \begin{align*}
    \mathbf{A}^{n}(k) - V(f)
    &=
    \{ (a_1,\ldots,a_n) \in \mathbf{A}^{n}(k) : f(a_1,\ldots,a_n) \neq 0 \} \\
    &\supseteq
    \{ a_n  \in \mathbf{A}^{1}(k) : f(1,\ldots,1,x_n) \neq 0 \}
  \end{align*}
  is infinite.

\item[(2)]
  \emph{Show that $V(f)$ is infinite if $n \geq 2$.}
  \begin{enumerate}
  \item[(a)]
    Similar to (1).
    Since $f$ is a nonconstant polynomial in $k[x_1, \ldots, x_n]$,
    we may assume that $m := \deg_{x_n}(f) > 0$.
    Write
    \[
      f = \sum_{i=0}^{m} f_i(x_1,\ldots,x_{n-1}) x_n^{i}.
    \]
    Note that each $f_i$ is well-defined since $n \geq 2$.

  \item[(b)]
    If $f_n$ is constant in $k[x_1, \ldots, x_{n-1}]$,
    then $f_n$ is nonzero (since $m > 0$) or $V(f_n) = \varnothing$.
    If $f_n$ is nonconstant in $k[x_1, \ldots, x_{n-1}]$,
    then the set $\mathbf{A}^{n-1}(k) - V(f_n)$ is infinite by (1).
    In any case,
    \[
      \mathbf{A}^{n-1}(k) - V(f_n)
    \]
    is infinite.

  \item[(c)]
    For each $P = (a_1,\ldots,a_{n-1}) \in \mathbf{A}^{n-1}(k) - V(f_n)$,
    \[
      g_P: x_n \mapsto f(P,x_n) = f(a_1,\ldots,a_{n-1},x_n)
    \]
    defines a polynomial in $k[x_n]$ of degree $m > 0$.
    Since $k = \overline{k}$, $g_P$ has at least one root $Q \in k$.
    Hence
    \[
      V(f) \supseteq
      \{ (P,Q) \in \mathbf{A}^{n}(k) : P \in \mathbf{A}^{n-1}(k) - V(f_n), g_P(Q) = 0 \}
    \]
    is infinite since the set $\mathbf{A}^{n-1}(k) - V(f_n)$ is infinite.
  \end{enumerate}
  \emph{Note.}
  It is not true if $k \neq \overline{k}$.
  For example, $V(x^2+y^2+1) = \varnothing$ in $\mathbf{A}^{2}(\mathbb{R})$.

\item[(3)]
  Note that
  \[
    \mathbf{A}^{n}(k) - V(S)
    = \mathbf{A}^{n}(k) - \bigcap_{f \in S} V(f)
    = \bigcup_{f \in S}( \mathbf{A}^{n}(k) - V(f) ).
  \]
  Thus the complement of any proper algebraic set is infinite by (1).
\end{enumerate}
$\Box$\\\\



%%%%%%%%%%%%%%%%%%%%%%%%%%%%%%%%%%%%%%%%%%%%%%%%%%%%%%%%%%%%%%%%%%%%%%%%%%%%%%%%



\subsubsection*{Problem 1.15.*}
\addcontentsline{toc}{subsubsection}{Problem 1.15.*}
\emph{Let $V \subseteq \mathbf{A}^n(k)$, $W \subseteq \mathbf{A}^m(k)$ be algebraic sets.
Show that
\[
  V \times W
  = \{(a_1,\ldots,a_n,b_1,\ldots,b_m) : (a_1,\ldots,a_n) \in V, (b_1,\ldots,b_m) \in W \}
\]
is an algebraic set in $\mathbf{A}^{n+m}(k)$.
It is called the \textbf{product} of $V$ and $W$.} \\

\emph{Proof.}
\begin{enumerate}
\item[(1)]
  Write
  \begin{align*}
    V &= V(S_V) = \{ P \in \mathbf{A}^n(k) : f(P) = 0 \: \forall f \in S_V \} \\
    W &= V(S_W) = \{ Q \in \mathbf{A}^m(k) : g(Q) = 0 \: \forall g \in S_W \},
  \end{align*}
  where $S_V \subseteq k[x_1,\ldots,x_n]$ and $S_W \subseteq k[y_1,\ldots,y_m]$.
  It suffices to show that
  \[
    V \times W = V(S),
  \]
  where
  $S \subseteq k[x_1,\ldots,x_n,y_1,\ldots,y_m]$ is the union of $S_V$ and $S_W$.

\item[(2)]
  Here we can identify $S_V$ with the subset of
  $k[x_1,\ldots,x_n,y_1,\ldots,y_m]$
  by noting that
  \[
    k[x_1,\ldots,x_n]
    \hookrightarrow (k[y_1,\ldots,y_m])[x_1,\ldots,x_n]
    = k[x_1,\ldots,x_n,y_1,\ldots,y_m].
  \]
  Here we regard $k$ as a subring of $k[y_1,\ldots,y_m]$.
  Similar treatment to $S_W$.

\item[(3)]
  By construction, $V \times W \subseteq V(S)$.
  Conversely, given any $(P,Q) \in V(S) \subseteq \mathbf{A}^{n+m}(k)$,
  we have $h(P,Q) = 0$ for all $h \in S = S_V \cup S_W$ (by (2)).
  By construction, $f(P) = 0$ for all $f \in S_V$ since $f$ only involve $x_1,\ldots,x_n$.
  Hence, $P \in V$. Similarly, $Q \in W$. Therefore, $(P,Q) \in V \times W$.
\end{enumerate}
$\Box$\\\\



%%%%%%%%%%%%%%%%%%%%%%%%%%%%%%%%%%%%%%%%%%%%%%%%%%%%%%%%%%%%%%%%%%%%%%%%%%%%%%%%
%%%%%%%%%%%%%%%%%%%%%%%%%%%%%%%%%%%%%%%%%%%%%%%%%%%%%%%%%%%%%%%%%%%%%%%%%%%%%%%%



\subsection*{1.3. The Ideal of a Set of Points \\}
\addcontentsline{toc}{subsection}{1.3. The Ideal of a Set of Points}



\subsubsection*{Problem 1.16.*}
\addcontentsline{toc}{subsubsection}{Problem 1.16.*}
\emph{Let $V, W$ be algebraic sets in $\mathbf{A}^n(k)$.
Show that $V = W$ if and only if $I(V) = I(W)$.} \\

\emph{Proof.}
\begin{enumerate}
\item[(1)]
  (Proof of Property (6) in \S 1.3.)
  \emph{Show that if $X \subseteq Y$, then $I(X) \supseteq I(Y)$.}
  If $f \in I(Y)$ then $f(P) = 0$ for all $P \in Y$.
  So $f(P) = 0$ for all $P \in X \subseteq Y$ or $f \in I(X)$.

\item[(2)]
  (Proof of Property (8) in \S 1.3.)
  \emph{$I(V(S)) \supseteq S$ for any set $S$ of polynomials;
  $V(I(X)) \supseteq X$ for any set $X$ of points.}
  \begin{enumerate}
  \item[(a)]
    If $f \in S$ then $f$ vanishes on $V(S)$,
    hence $f \in IV(S)$.

  \item[(b)]
    If $P \in X$ then every polynomial in $I(X)$ vanishes at $P$,
    so $P$ belongs to the zero set of $I(X)$.
  \end{enumerate}

\item[(3)]
  (Proof of Property (9) in \S 1.3.)
  \emph{$V(I(V(S))) = V(S)$ for any set $S$ of polynomials,
  and $I(V(I(X))) = I(X)$ for any set $X$ of points.
  So if $V$ is an algebraic set, $V = V(I(V))$,
  and if $I$ is the ideal of an algebraic set, $I = I(V(I))$.}
  \begin{enumerate}
  \item[(a)]
    In each case, it suffices to show that the left side is a subset of the right side.
    (by Properties (6)(8) in \S 1.3).

  \item[(b)]
    If $P \in V(S)$ then $f(P) = 0$ for all $f \in I(V(S))$, so $P \in V(I(V(S)))$.

  \item[(c)]
    If $f \in I(X)$ then $f(P) = 0$ for all $P \in V(I(X))$.
    Thus $f$ vanishes on $V(I(X))$, so $f \in I(V(I(X)))$.
  \end{enumerate}

\item[(4)]
  \emph{Show that $V = W$ if and only if $I(V) = I(W)$.}
  \begin{enumerate}
  \item[(a)]
    By Property (6) in \S 1.3, $I(V) \supseteq I(W)$ if $V \subseteq W$
    and $I(V) \subseteq I(W)$ if $V \supseteq W$.
    Thus, $I(V) = I(W)$ if $V = W$.

  \item[(b)]
    Conversely,
    $I(V) = I(W)$ implies that $V(I(V)) = V(I(W))$
    by Property (3) in \S 1.2 and similar argument in (a).
    By Property (9) in \S 1.3, $V(I(V)) = V$ and $V(I(W)) = W$.
    Thus, $V = W$.
  \end{enumerate}
\end{enumerate}
$\Box$\\\\



%%%%%%%%%%%%%%%%%%%%%%%%%%%%%%%%%%%%%%%%%%%%%%%%%%%%%%%%%%%%%%%%%%%%%%%%%%%%%%%%



\subsubsection*{Problem 1.17.*}
\addcontentsline{toc}{subsubsection}{Problem 1.17.*}
\begin{enumerate}
\item[(a)]
  \emph{Let $V$ be an algebraic set in $\mathbf{A}^n(k)$,
  $P \in\mathbf{A}^n(k)$ a point not in $V$.
  Show that there is a polynomial $f \in k[x_1,\ldots,x_n]$ such that $f(Q) = 0$
  for all $Q \in V$, but $f(P) = 1$. (Hint: $I(V) \neq I(V \cup \{P\})$.)}

\item[(b)]
  \emph{Let $P_1, \ldots, P_r$ be distinct points in $\mathbf{A}^n(k)$,
  not in an algebraic set $V$.
  Show that there are polynomials $f_1, \ldots, f_r \in I(V)$
  such that $f_i(P_j) = 0$ if $i \neq j,$ and $f_i(P_i) = 1$.
  (Hint: Apply (a) to the union of $V$ and all but one point.)}

\item[(c)]
  \emph{With $P_1, \ldots, P_r$ and $V$ as in (b),
  and $a_{ij} \in k$ for $1 \leq i,j \leq r$,
  show that there are $g_i \in I(V)$ with $g_i(P_j) = a_{ij}$ for all $i$ and $j$.
  (Hint: Consider $\sum_{j} a_{ij} f_j$.)} \\
\end{enumerate}



\emph{Proof of (a).}
\begin{enumerate}
\item[(1)]
  Since $I(V) \supsetneq I(V \cup \{P\})$ (by Problem 1.16),
  there is a polynomial $f \in k[x_1,\ldots,x_n]$ such that $f(Q) = 0$ for all $Q \in V$, but $f(P) \neq 0$.

\item[(2)]
  Since $k$ is a field, $(f(P))^{-1} \in k$.
  Consider the polynomial $(f(P))^{-1} f \in k[x_1,\ldots,x_n]$.
  It is well-defined.
  Also, $((f(P))^{-1} f)(Q) = (f(P))^{-1}f(Q) = 0$ for all $Q \in V$,
  but $(f(P))^{-1} f)(P) = (f(P))^{-1} f(P) = 1$.
\end{enumerate}
$\Box$\\



\emph{Proof of (b).}
\begin{enumerate}
\item[(1)]
  For $1 \leq i \leq$,
  define
  \begin{align*}
    W &= V \cup \{ P_1, \ldots, P_r \} \\
    W_i &= V \cup \{ P_1, \ldots, \widehat{P_i}, \ldots, P_r \}.
  \end{align*}
  Here $W = W_i \cup \{P_i\} \neq W_i$.

\item[(2)]
  By (a), there is a polynomial $f_i \in k[x_1,\ldots,x_n]$ such that $f_i(Q) = 0$
  for all $Q \in W_i$, but $f_i(P_i) = 1$.
  Here $f_i \in I(V)$ and $f_i(P_j) = \delta_{ij}$
  where $\delta_{ij}$ is the Kronecker delta.
\end{enumerate}
$\Box$\\



\emph{Proof of (c).}
\begin{enumerate}
\item[(1)]
  For each $1 \leq i \leq r$,
  define
  \[
    g_i = \sum_{j} a_{ij} f_j \in k[x_1,\ldots,x_n].
  \]

\item[(2)]
  $g_i \in I(V)$ since $g_i$ is a linear combination of $f_j$ and $I(V)$ is an ideal.

\item[(3)]
  Also,
  \[
    g_i(P_j)
    = \sum_{j'} a_{ij'} f_{j'}(P_j)
    = \sum_{j'} a_{ij'} \delta_{j'j}
    = a_{ij}.
  \]
\end{enumerate}
$\Box$\\\\



%%%%%%%%%%%%%%%%%%%%%%%%%%%%%%%%%%%%%%%%%%%%%%%%%%%%%%%%%%%%%%%%%%%%%%%%%%%%%%%%



\subsubsection*{Problem 1.18.*}
\addcontentsline{toc}{subsubsection}{Problem 1.18.*}
\emph{Let $I$ be an ideal in a ring $R$.
If $a^n \in I$, $b^m \in I$, show that $(a+b)^{n+m} \in I$.
Show that $\mathrm{rad}(I)$ is an ideal, in fact a radical ideal.
Show that any prime ideal is radical.} \\

\emph{Proof.}
\begin{enumerate}
\item[(1)]
  \emph{Show that $(a+b)^{n+m} \in I$ if $a^n \in I$, $b^m \in I$.}
  By the binomial theorem,
  \[
    (a+b)^{n+m}=\sum_{i=0}^{n+m} a^i b^{n+m-i}.
  \]
  For each term $a^i b^{n+m-i}$, either $i \geq n$ holds or $n+m-i \geq m$ holds,
  and thus $a^i b^{n+m-i} \in I$ (since $a^n \in I$, $b^m \in I$ and $I$ is an ideal).
  Hence, the result is established.

\item[(2)]
  \emph{Show that $\mathrm{rad}(I)$ is an ideal.}
  \begin{enumerate}
  \item[(a)]
    $0 \in \mathrm{rad}(I)$ since $0 = 0^{1} \in I$ for any ideal in $R$.

  \item[(b)]
    $(a+b)^{n+m} \in I$ if $a^n \in I$, $b^m \in I$ by (1).

  \item[(c)]
    $(-a)^{2n} = (a^n)^2 \in I$ if $a^n \in I$ (since $I$ is an ideal).

  \item[(d)]
    $(ra)^n = r^n a^n \in I$ if $a^n \in I$ and $r \in R$ (since $I$ is an ideal and $R$ is commutative).
  \end{enumerate}

\item[(3)]
  \emph{Show that $\mathrm{rad}(\mathrm{rad}(I)) = \mathrm{rad}(I)$.}
  It suffices to show $\mathrm{rad}(\mathrm{rad}(I)) \subseteq \mathrm{rad}(I)$.
  Given any $a \in \mathrm{rad}(\mathrm{rad}(I))$.
  By definition $a^n \in \mathrm{rad}(I)$ for some positive integer $n$.
  Again by definition $(a^n)^m = a^{nm} \in I$ for some positive integer $m$.
  As $nm$ is a postive integer, $a \in \mathrm{rad}(I)$.

\item[(4)]
  \emph{Show that every prime ideal $\mathfrak{p}$ is radical.}
  Given any $a \in \mathrm{rad}(\mathfrak{p})$, that is,
  $a^n \in \mathfrak{p}$ for some positive integer.
  Write $a^n = a a^{n-1}$ if $n > 1$.
  By the primality of $\mathfrak{p}$, $a \in \mathfrak{p}$ or $a^{n-1} \in \mathfrak{p}$.
  If $a \in \mathfrak{p}$, we are done.
  If $a^{n-1} \in \mathfrak{p}$,
  we continue this descending argument (or the mathematical induction)
  until the power of $a$ is equal to $1$.
  Hence $\mathfrak{p}$ is radical.
\end{enumerate}
$\Box$\\\\



%%%%%%%%%%%%%%%%%%%%%%%%%%%%%%%%%%%%%%%%%%%%%%%%%%%%%%%%%%%%%%%%%%%%%%%%%%%%%%%%



\subsubsection*{Problem 1.19.}
\addcontentsline{toc}{subsubsection}{Problem 1.19.}
\emph{Show that $I = (x^2+1) \subseteq \mathbb{R}[x]$ is a radical (even a prime) ideal,
but $I$ is not the ideal of any set in $\mathbf{A}^1(\mathbb{R})$.} \\

\emph{Proof.}
\begin{enumerate}
\item[(1)]
  \emph{Show that $I = (x^2+1)$ is a prime ideal in $\mathbb{R}[x]$.}
  Given any $fg \in I$.
  It suffices to show that $f \in I$ or $g \in I$.
  By definition of $I$,
  there is a polynomial $h \in \mathbb{R}[x]$ such that $fg = (x^2+1)h$.
  So $(x^2+1) \mid f$ or $(x^2+1) \mid g$
  since $x^2+1$ is irreducible in a unique factorization domain $\mathbb{R}[x]$.
  Therefore, $f \in I$ or $g \in I$.

\item[(2)]
  \emph{Show that $I$ is not the ideal of any set in $\mathbf{A}^1(\mathbb{R})$.}
  Since $x^2+1$ has no roots in $\mathbb{R}$,
  $I$ cannot be the ideal of any nonempty set in $\mathbf{A}^1(\mathbb{R})$.
  Besides,
  $I(\varnothing) = (1) \neq (x^2+1)$.
\end{enumerate}
$\Box$\\\\



%%%%%%%%%%%%%%%%%%%%%%%%%%%%%%%%%%%%%%%%%%%%%%%%%%%%%%%%%%%%%%%%%%%%%%%%%%%%%%%%



\subsubsection*{Problem 1.20.*}
\addcontentsline{toc}{subsubsection}{Problem 1.20.*}
\emph{Show that for any ideal $I$ in $k[x_1,\ldots,x_n]$,
$V(I) = V(\mathrm{rad}(I))$, and $\mathrm{rad}(I) \subseteq I(V(I))$.} \\

\emph{Proof.}
\begin{enumerate}
\item[(1)]
  \emph{Show that $V(I) = V(\mathrm{rad}(I))$.}
  Since $I \subseteq \mathrm{rad}(I)$,
  it suffices to show that $V(I) \subseteq V(\mathrm{rad}(I))$.
  Given any $P \in V(I)$.
  For any $f \in \mathrm{rad}(I)$, $f^n \in I$ for some positive integer $n > 0$.
  Note that
  \[
    0 = (f^n)(P) = f(P)^n
  \]
  since $f^n \in I$ and $P \in V(I)$.
  As $k$ is a domain, $f(P)^n = 0$ implies $f(P) = 0$. So $P \in V(\mathrm{rad}(I))$.

\item[(2)]
  By Properties (6)(8) in \S 1.3,
  \[
    I(V(I)) = I(V(\mathrm{rad}(I))) \supseteq \mathrm{rad}(I).
  \]
\end{enumerate}
$\Box$\\

\emph{Note.}
\begin{enumerate}
\item[(1)]
  By the Hilbert's Nullstellensatz, $I(V(I)) = \mathrm{rad}(I)$ if $k = \overline{k}$.

\item[(2)]
  Take $I = (x^2+1)$ as an ideal in $\mathbb{R}[x]$.
  Note that $I(V(I))= I(\varnothing) = (1)$ and $\mathrm{rad}(I) = I = (x^2+1)$.
  So the equality in $\mathrm{rad}(I) \subsetneq I(V(I))$ might not hold
  if $k \neq \overline{k}$.
  (See Problem 1.19.) \\\\
\end{enumerate}



%%%%%%%%%%%%%%%%%%%%%%%%%%%%%%%%%%%%%%%%%%%%%%%%%%%%%%%%%%%%%%%%%%%%%%%%%%%%%%%%



\subsubsection*{Problem 1.21.*}
\addcontentsline{toc}{subsubsection}{Problem 1.21.*}
\emph{Show that
$I=(x_1-a_1, \ldots, x_n-a_n) \subseteq k[x_1,\ldots,x_n]$ is a maximal ideal,
and that the natural homomorphism from $k$ to $k[x_1,\ldots,x_n]/I$ is an isomorphism.} \\

\emph{Proof.}
\begin{enumerate}
\item[(1)]
  \emph{Show that $I$ is a maximal ideal.}
  Suppose that $J$ is an ideal such that $J \supsetneq I$.
  Take any $f \in J - I$.
  By Problem 1.7(a),
  \[
    f = \sum \lambda_{(i)} (x_1-a_1)^{i_1} \cdots (x_n - a_n)^{i_n}.
  \]
  As $f \not\in I$, there is a nonzero constant term in $f$, say $\lambda \in k - \{0\}$.
  Note that $f - \lambda \in I \subsetneq J$.
  Hence,
  \[
    \lambda = f - (f - \lambda) \in J
  \]
  since $J$ is an ideal.
  As $\lambda \neq 0$, $J = k[x_1,\ldots,x_n]$ is not a proper ideal containing $I$.

\item[(2)]
  Let $\varphi: k \to k[x_1,\ldots,x_n]/I$ be the natural homomorphism.
  (That is, $\varphi: \lambda \to \lambda + I \in k[x_1,\ldots,x_n]/I$.)

\item[(3)]
  \emph{Show that $\varphi$ is surjective.}
  Given any $f + I \in k[x_1,\ldots,x_n]/I$.
  By Problem 1.7(a),
  \[
    f = \sum \lambda_{(i)} (x_1-a_1)^{i_1} \cdots (x_n - a_n)^{i_n}.
  \]
  So
  \begin{align*}
    f + I
    &= \sum \lambda_{(i)} (x_1-a_1)^{i_1} \cdots (x_n - a_n)^{i_n} + I \\
    &= \left(f(a_1,\ldots,a_n)
      + \sum_{\text{nonconstant}} \lambda_{(i)} (x_1-a_1)^{i_1} \cdots (x_n - a_n)^{i_n} \right) + I \\
    &= f(a_1,\ldots,a_n) + I.
  \end{align*}
  (Here the summation over all nonconstant terms is in $I$.)
  Hence
  \[
    \varphi: f(a_1,\ldots,a_n) \in k \mapsto f + I.
  \]

\item[(4)]
  \emph{Show that $\varphi$ is injective.}
  $\ker(\varphi) = \{ \lambda \in k : \lambda \in I \} = k \cap I = \{0\}$
  since $I$ is a proper ideal.

\item[(5)]
  By (2)(3)(4), $\varphi: k \to k[x_1,\ldots,x_n]/(x_1-a_1,\ldots,x_n-a_n)$
  is an isomorphism.
\end{enumerate}
$\Box$\\\\



%%%%%%%%%%%%%%%%%%%%%%%%%%%%%%%%%%%%%%%%%%%%%%%%%%%%%%%%%%%%%%%%%%%%%%%%%%%%%%%%
%%%%%%%%%%%%%%%%%%%%%%%%%%%%%%%%%%%%%%%%%%%%%%%%%%%%%%%%%%%%%%%%%%%%%%%%%%%%%%%%



\subsection*{1.4. The Hilbert Basis Theorem \\}
\addcontentsline{toc}{subsection}{1.4. The Hilbert Basis Theorem}



\subsubsection*{Problem 1.22.*}
\addcontentsline{toc}{subsubsection}{Problem 1.22.*}
\emph{Let $I$ be an ideal in a ring $R$, $\pi: R \to R/I$ the natural homomorphism.}
\begin{enumerate}
\item[(a)]
  \emph{Show that for every ideal $J'$ of $R/I$,
  $\pi^{-1}(J') = J$ is an ideal of $R$ containing $I$,
  and for every ideal $J$ of $R$ containing $I$,
  $\pi(J) = J'$ is an ideal of $R/I$.
  This sets up a natural one-to-one correspondence between
  $\{ \text{ideals of $R/I$} \}$ and $\{ \text{ideals of $R$ that contain $I$} \}$.}

\item[(b)]
  \emph{Show that $J'$ is a radical ideal if and only if $J$ is radical.
  Similarly for prime and maximal ideals.}

\item[(c)]
  \emph{Show that $J'$ is finitely generated if $J$ is.
  Conclude that $R/I$ is Noetherian if $R$ is Noetherian.
  Any ring of the form $k[x_1,\ldots,x_n]/I$ is Noetherian.} \\
\end{enumerate}



\emph{Proof of (a).}
\begin{enumerate}
\item[(1)]
  \emph{Show that for every ideal $J'$ of $R/I$,
  $\pi^{-1}(J') = J$ is an ideal of $R$ containing $I$.}
  \begin{enumerate}
  \item[(a)]
    \emph{Show that $J$ contains $I$.}
    Note that $\pi^{-1}(0) = I \subseteq \pi^{-1}(J') = J$.
    So $J$ contains $I$. In particular, $J \neq \varnothing$ since $I \neq \varnothing$.

  \item[(b)]
    \emph{Show that $J$ is a additive subgroup of $R$.}
    It suffices to show that
    $a - b \in J$
    for any $a \in J$ and $b \in J$.
    Actually,
    \[
      \pi(a - b) = \pi(a) - \pi(b) \in J'
    \]
    implies $a - b \in \pi^{-1}(J') = J$.

  \item[(c)]
    \emph{Show that for every $r \in R$ and every $a \in J$,
    the product $ra \in J$.}
    In fact,
    \[
      \pi(ra) = \pi(r) \pi(a) \in J'
    \]
    implies $ra \in \pi^{-1}(J') = J$.
  \end{enumerate}

\item[(2)]
  \emph{Show that for every ideal $J$ of $R$ containing $I$,
  $\pi(J) = J'$ is an ideal of $R/I$.}
  \begin{enumerate}
  \item[(a)]
    \emph{Show that $J'$ is nonempty.}
    Note that $\pi(a) = 0 \in \pi(I) \subseteq \pi(J) = J'$ for any $a \in I$.
    So $J'$ is nonempty since $J$ is nonempty.

  \item[(b)]
    \emph{Show that $J'$ is a additive subgroup of $R/I$.}
    It suffices to show that
    $\pi(a) - \pi(b) \in J'$ for any $\pi(a) \in J'$, $\pi(b) \in J'$, $a \in J$ and $b \in J$.
    It is trivial since
    \[
      \pi(a) - \pi(b) = \pi(a - b) \in \pi(J) = J',
    \]
    $\pi$ is a ring homomorphism and $J$ is an ideal.

  \item[(c)]
    \emph{Show that for every $\pi(r) \in R/I$ ($r \in R$) and every $\pi(a) \in J'$ ($a \in J$),
    the product $\pi(r)\pi(a) \in J'$.}
    It is trivial since
    \[
      \pi(r)\pi(a) = \pi(ra) \in \pi(J) = J',
    \]
    $\pi$ is a ring homomorphism and $J$ is an ideal.
  \end{enumerate}

\item[(3)]
  By (1)(2), we setup the correspondence between
  \[
    \{ \text{ideals of $R/I$} \}
    \longleftrightarrow
    \{ \text{ideals of $R$ that contain $I$} \}.
  \]
  Note that this correspondence preserves the subset relation,
  and thus this correspondence is one-to-one.
\end{enumerate}
$\Box$\\



\emph{Proof of (b).}
\begin{enumerate}
\item[(1)]
  \emph{Show that $J'$ is radical if $J$ is radical.}
  It suffices to show that $(a + I)^n = a^n + I \in J'$ implies that $a + I \in J'$.
  Note that
  \[
    (a + I)^n = a^n + I \in J'
  \]
  implies that $a^n \in J$ or $a \in J$ since $J$ is radical.
  Hence $a + I \in J/I = J'$.

\item[(2)]
  \emph{Show that $J$ is radical if $J'$ is radical.}
  It suffices to show that $a^n \in J$ implies that $a \in J$.
  Note that
  \[
    \pi(a^n) = \pi(a)^n \in J'
  \]
  implies that $\pi(a) \in J'$ since $J'$ is radical.
  $a \in \pi^{-1}(J') = J$.

\item[(3)]
  \emph{Show that $J'$ is prime if $J$ is prime.}
  It suffices to show that
  $(a + I)(b + I) = ab + I \in J'$ implies that $a + I \in J'$ or $b + I \in J'$.
  Note that
  \[
    (a + I)(b + I) = ab + I \in J'
  \]
  implies that $ab \in J$. So $a \in J$ or $b \in J$ by the primality of $J$.
  Hence $a + I \in J'$ or $b + I \in J'$.

\item[(4)]
  \emph{Show that $J$ is prime if $J'$ is prime.}
  It suffices to show that $ab \in J$ implies that $a \in J$ or $b \in J$.
  Note that
  \[
    \pi(ab) = \pi(a)\pi(b) \in J'
  \]
  implies that $\pi(a) \in J'$ or $\pi(b) \in J'$ by the primality of $J'$.
  So $a \in \pi^{-1}(J') = J$ or $b \in \pi^{-1}(J') = J$.

\item[(5)]
  \emph{Show that $J'$ is maximal if $J$ is maximal.}
  Suppose $\mathfrak{m}$ is an ideal containing $J'$.
  By (a), $\pi^{-1}(\mathfrak{m})$ is an ideal containing $J$.
  So $\pi^{-1}(\mathfrak{m}) = J$ or $\pi^{-1}(\mathfrak{m}) = R$ by the maximality of $J$.
  Hence, $\mathfrak{m} = \pi(J) = J'$ or $\mathfrak{m} = \pi(R) = R/I$.

\item[(6)]
  \emph{Show that $J$ is maximal if $J'$ is maximal.}
  Suppose $\mathfrak{m}$ is an ideal containing $J$.
  By (a), $\pi(\mathfrak{m})$ is an ideal containing $J'$.
  So $\pi(\mathfrak{m}) = J'$ or $\pi(\mathfrak{m}) = R/I$ by the maximality of $J'$.
  Hence, $\mathfrak{m} = \pi^{-1}(J') = J$ or $\mathfrak{m} = \pi^{-1}(R/I) = R$.
\end{enumerate}
$\Box$\\



\emph{Note.}
\begin{enumerate}
\item[(1)]
  Note that
  \[
    R/J \cong (R/I)/(J/I)
  \]
  if $J$ is an ideal of $R$ such that $I \subseteq J$.

\item[(2)]
  Hence, $J$ is prime iff $R/J \cong (R/I)/(J/I)$ is a domain iff $J/I$ is prime.

\item[(3)]
  Also, $J$ is maximal iff $R/J \cong (R/I)/(J/I)$ is a field iff $J/I$ is maximal. \\
\end{enumerate}



\emph{Proof of (c).}
\begin{enumerate}
\item[(1)]
  \emph{Show that $J'$ is finitely generated if $J$ is.}
  Suppose $J$ is generated by $a_1, \ldots, a_m$.
  It suffices to show that $J'$ is generated by
  \[
    a_1 + I, \ldots, a_m + I \in J/I.
  \]
  Given any $a + I \in J'$ where $a \in J$.
  Write $a = \sum_{1 \leq i \leq m} r_i a_i$ for some $r_i \in R$.
  Then
  \[
    a + I = \sum r_i a_i + I = \sum (r_i + I)(a_i + I)
  \]
  is generated by $a_1 + I, \ldots, a_m + I$.

\item[(2)]
  \emph{Show that that $R/I$ is Noetherian if $R$ is Noetherian.}
  Note that $R$ is an ideal of itself.

\item[(3)]
  \emph{Show that any ring of the form $k[x_1,\ldots,x_n]/I$ is Noetherian.}
  By the corollary to the Hilbert basis theorem,
  $k[x_1,\ldots,x_n]$ is Noetherian.
  By (2), the ring $k[x_1,\ldots,x_n]/I$ is Noetherian.
\end{enumerate}
$\Box$\\\\



%%%%%%%%%%%%%%%%%%%%%%%%%%%%%%%%%%%%%%%%%%%%%%%%%%%%%%%%%%%%%%%%%%%%%%%%%%%%%%%%
%%%%%%%%%%%%%%%%%%%%%%%%%%%%%%%%%%%%%%%%%%%%%%%%%%%%%%%%%%%%%%%%%%%%%%%%%%%%%%%%



\subsection*{1.5. Irreducible Components of an Algebraic Set \\}
\addcontentsline{toc}{subsection}{1.5. Irreducible Components of an Algebraic Set}



\subsubsection*{Problem 1.23.}
\addcontentsline{toc}{subsubsection}{Problem 1.23.}
\emph{Give an example of a collection of ideals $\mathscr{S}$ ideals in a Noetherian ring
such that no maximal member of $\mathscr{S}$ is a maximal ideal.} \\

\emph{Proof.}
\begin{enumerate}
\item[(1)]
  Let $R$ be any Noetherian ring.
  Let $\mathscr{S}$ be any collection of ideals containing $R$ itself.
  Then the only maximal member of $\mathscr{S}$ is $R$, which is not a maximal ideal.

\item[(2)]
  Or let $R$ be any Noetherian ring and $R$ is not a field.
  ($R = k[x_1,\ldots,k_n]$ where $k$ is a field for example.)
  Let $\mathscr{S} = \{ (0) \}$.
  Then the only maximal member of $\mathscr{S}$ is $(0)$, which is not maximal
  since $R$ is not a field.
\end{enumerate}
$\Box$\\\\



%%%%%%%%%%%%%%%%%%%%%%%%%%%%%%%%%%%%%%%%%%%%%%%%%%%%%%%%%%%%%%%%%%%%%%%%%%%%%%%%



\subsubsection*{Problem 1.24.}
\addcontentsline{toc}{subsubsection}{Problem 1.24.}
\emph{Show that every proper ideal in a Noetherian ring is contained in a maximal ideal.
(Hint: If $I$ is the ideal, apply the lemma to $\{\text{proper ideals that contain $I$}\}$.)} \\

\emph{Proof.}
\begin{enumerate}
\item[(1)]
  Say $I$ be any proper ideal in a Noetherian ring.
  Let
  \[
    \mathscr{S} = \{\text{proper ideals that contain $I$}\}.
  \]
  Apply the lemma to $\mathscr{S}$ to get that
  $\mathscr{S}$ has a maximal member $\mathfrak{m} \in \mathscr{S}$.

\item[(2)]
  \emph{Show that $\mathfrak{m}$ is maximal.}
  Since $\mathfrak{m} \in \mathscr{S}$, $\mathfrak{m}$ is a proper ideal in $R$.
  Suppose $\mathfrak{m}' \supseteq \mathfrak{m}$
  is a proper ideal containing $\mathfrak{m}$.
  As $\mathfrak{m}$ contains $I$,
  $\mathfrak{m}'$ also contains $I$ or $\mathfrak{m}' \in \mathscr{S}$.
  By the maximality of $\mathfrak{m}$, $\mathfrak{m}' \subseteq \mathfrak{m}$.
  So $\mathfrak{m}' = \mathfrak{m}$.
\end{enumerate}
$\Box$\\\\



%%%%%%%%%%%%%%%%%%%%%%%%%%%%%%%%%%%%%%%%%%%%%%%%%%%%%%%%%%%%%%%%%%%%%%%%%%%%%%%%



\subsubsection*{Problem 1.25.}
\addcontentsline{toc}{subsubsection}{Problem 1.25.}
\begin{enumerate}
\item[(a)]
  \emph{Show that $V(y - x^2) \subseteq \mathbf{A}^2(\mathbb{C})$ is irreducible,
  in fact, $I(V(y - x^2))=(y - x^2)$.}

\item[(b)]
  \emph{Decompose $V(y^4 - x^2, y^4 - x^2 y^2 + xy^2 - x^3) \subseteq \mathbf{A}^2(\mathbb{C})$
  into irreducible components.} \\
\end{enumerate}



\emph{Proof of (a).}
\begin{enumerate}
\item[(1)]
  Let $I = (y - x^2)$ be an ideal of $\mathbb{C}[x,y]$.
  Since $\mathbb{C}$ is algebraically closed,
  \[
    I(V(I)) = \mathrm{rad}(I)
  \]
  by the Hilbert's Nullstellensatz.
  It suffices to show that $I$ is prime,
  or to show that $y - x^2$ is prime.
  Since $\mathbb{C}[x,y]$ is a UFD, it suffices to show that $y - x^2$ is irreducible.

\item[(2)]
  \emph{Show that $y - x^2$ is irreducible in $\mathbb{C}[x,y]$.}
  Write
  \[
    y - x^2 \in (\mathbb{C}[y])[x].
  \]
  Note that $\mathbb{C}[y]$ is a UFD and $y$ is the constant term.
  If we can show that $y$ is prime in $\mathbb{C}[y]$, then by the Eisenstein's criterion
  we can say $y - x^2$ is irreducible in $(\mathbb{C}[y])[x]$.

\item[(3)]
  As $\mathbb{C}[y]/(y) \cong \mathbb{C}$ is a field or a domain,
  $(y)$ is maximal or prime.
  Hence, $y - x^2$ is irreducible.

\item[(4)]
  Or apply Corollary 1 to Proposition 2 in the next section to (2)(3).
\end{enumerate}
$\Box$\\



\emph{Proof of (b).}
\begin{enumerate}
\item[(1)]
  Write
  \begin{align*}
    Y
    :=& \: V(y^4 - x^2, y^4 - x^2 y^2 + xy^2 - x^3) \\
    =& \: V((y^2 - x)(y^2 + x), (y^2 - x^2)(y^2 + x)) \\
    =& \: V(y^2 + x) \cup V(y^2 - x, y^2 - x^2) \\
    =& \: V(y^2 + x) \cup V(y^2 - x, x(x - 1)) \\
    =& \: V(y^2 + x) \cup V(x, y) \cup V(y + 1, x - 1) \cup V(y - 1, x - 1).
  \end{align*}

\item[(2)]
  Here $V(y^2 + x)$ is irreducible as (a).
  Besides,
  $V(x, y)$, $V(y + 1, x - 1)$ and $V(y - 1, x - 1)$ are irreducible
  since all corresponding ideals are maximal
  (by the Hilbert's Nullstellensatz and Problem 1.21).
\end{enumerate}
$\Box$\\\\



%%%%%%%%%%%%%%%%%%%%%%%%%%%%%%%%%%%%%%%%%%%%%%%%%%%%%%%%%%%%%%%%%%%%%%%%%%%%%%%%



\subsubsection*{Problem 1.26.}
\addcontentsline{toc}{subsubsection}{Problem 1.26.}
\emph{Show that $f = y^2 + x^2(x-1)^2 \in \mathbb{R}[x,y]$ is an irreducible polynomial,
but $V(f)$ is reducible.} \\

\emph{Proof.}
\begin{enumerate}
\item[(1)]
  \emph{Show that $f$ is an irreducible polynomial.}
  \begin{enumerate}
  \item[(a)]
    Suppose
    \[
      f = (f_2(x) y^2 + f_1(x) y + f_0(x)) \cdot g(x)
    \]
    for some $f_i(x), g(x) \in \mathbb{R}[x]$.
    So
    \[
      f_2(x)g(x) = 1,
      \qquad
      f_1(x)g(x) = 0,
      \qquad
      f_0(x)g(x) = x^2(x-1)^2.
    \]
    Hence,
    \[
      f_2(x) y^2 + f_1(x) y + f_0(x) = uf,
      \qquad
      g(x) = u^{-1},
    \]
    where $u$ is a unit in $\mathbb{R}$.

  \item[(b)]
    Suppose
    \[
      f = (f_1(x) y + f_0(x)) \cdot (g_1(x) y + g_0(x))
    \]
    for some $f_i(x), g_j(x) \in \mathbb{R}[x]$.
    So
    \begin{align*}
      f_1(x)g_1(x) &= 1, \\
      f_1(x)g_0(x) + f_0(x)g_1(x) &= 0, \\
      f_0(x)g_0(x) &= x^2(x-1)^2.
    \end{align*}
    So $f_1(x) = u$, $g_1(x) = u^{-1}$ for some unit $u \in \mathbb{R}$.
    Hence,
    \[
      u^2 g_0(x)^2 = -x^2(x-1)^2,
    \]
    which is absurd since $\mathbb{R}$ is not algebraically closed.

  \item[(c)]
    By (a)(b), $f$ is irreducible in $\mathbb{R}[x,y]$.
  \end{enumerate}

\item[(2)]
  \emph{Show that $V(f)$ is reducible.}
  $V(f) = \{ (0,0), (1,0) \} = V(x,y) \cup V(x-1,y)$.
  Here $V(x,y)$ and $V(x-1,y)$ are all proper algebraic sets in $V(f)$.
\end{enumerate}
$\Box$\\\\



%%%%%%%%%%%%%%%%%%%%%%%%%%%%%%%%%%%%%%%%%%%%%%%%%%%%%%%%%%%%%%%%%%%%%%%%%%%%%%%%



\subsubsection*{Problem 1.27.}
\addcontentsline{toc}{subsubsection}{Problem 1.27.}
\emph{Let $V, W$ be algebraic sets in $\mathbf{A}^n(k)$ with $V \subseteq W$.
Show that each irreducible component of $V$ is contained in some irreducible component of $W$.} \\

\emph{Proof.}
\begin{enumerate}
\item[(1)]
  Write two decompositions of $V, W$ into irreducible components as
  \begin{align*}
    V &= V_1 \cup \cdots \cup V_r, \\
    W &= W_1 \cup \cdots \cup W_s, \\
  \end{align*}

\item[(2)]
  For each irreducible component $V_i$ of $V$,
  consider $V_i \cap W$:
  \[
    V_i \cap W = (V_i \cap W_1) \cup \cdots \cup (V_i \cap W_s).
  \]
  By the irreducibility of $V_i$,
  there is only one $j$ such that $V_i \cap W_j = V_i$
  and other intersections are empty.
  Therefore, each irreducible component $V_i$ is contained in
  some irreducible component $W_j$ of $W$.
\end{enumerate}
$\Box$\\\\



%%%%%%%%%%%%%%%%%%%%%%%%%%%%%%%%%%%%%%%%%%%%%%%%%%%%%%%%%%%%%%%%%%%%%%%%%%%%%%%%



\subsubsection*{Problem 1.28.}
\addcontentsline{toc}{subsubsection}{Problem 1.28.}
\emph{If $V = V_1 \cup \cdots \cup V_r$ is the decomposition
of an algebraic set into irreducible components,
show that $V_i \not\subseteq \bigcup_{j \neq i} V_j$.} \\

\emph{Proof.}
\begin{enumerate}
\item[(1)]
  (Reductio ad absurdum)
  If
  \[
    V_i \subseteq \bigcup_{j \neq i} V_j
  \]
  for some $i$, then
  \[
    V = V_1 \cup \cdots \cup \widehat{V_i} \cup \cdots \cup V_r
  \]
  is another decomposition of an algebraic set into irreducible components.

\item[(2)]
  By Theorem 2 in \S 1.5, the number of irreducible components is unique determined,
  contrary to the assumption and (1).
\end{enumerate}
$\Box$\\\\



%%%%%%%%%%%%%%%%%%%%%%%%%%%%%%%%%%%%%%%%%%%%%%%%%%%%%%%%%%%%%%%%%%%%%%%%%%%%%%%%



\subsubsection*{Problem 1.29.*}
\addcontentsline{toc}{subsubsection}{Problem 1.29.*}
\emph{Show that $\mathbf{A}^n(k)$ is irreducible if $k$ is infinite.} \\

\emph{Proof.}
\begin{enumerate}
\item[(1)]
  (Reductio ad absurdum)
  If $\mathbf{A}^n(k)$ were reducible,
  then $\mathbf{A}^n(k) = V_1 \cup V_2$ where $V_1, V_2$ are algebraic sets in $\mathbf{A}^n(k)$,
  $V_1$ and $V_2$ are nonempty and proper in $\mathbf{A}^n(k)$.

\item[(2)]
  Take $P_i \in V_i$ for $i = 1, 2$.
  By Problem 1.17,
  there are two polynomials $f_1, f_2 \in k[x_1,\ldots,x_n]$
  such that $f_i(Q) = 0$ for all $Q \in V_i$ and $f_1(P_2) = f_2(P_1) = 1$.

\item[(3)]
  By construction,
  $(f_1 f_2)(a_1,\ldots,a_n) = 0$ for any $a_1,\ldots,a_n \in k$.
  As $k$ is infinite, $f_1 f_2 = 0$ by Problem 1.4.
  Since $k[x_1,\ldots,x_n]$ is a domain, $f_1 = 0$ or $f_2 = 0$,
  contrary to $f_1(P_2) = f_2(P_1) \neq 0$.
\end{enumerate}
$\Box$\\

\emph{Note.}
  $\mathbf{A}^n(k)$ is reducible if $k$ is finite. \\\\



%%%%%%%%%%%%%%%%%%%%%%%%%%%%%%%%%%%%%%%%%%%%%%%%%%%%%%%%%%%%%%%%%%%%%%%%%%%%%%%%
%%%%%%%%%%%%%%%%%%%%%%%%%%%%%%%%%%%%%%%%%%%%%%%%%%%%%%%%%%%%%%%%%%%%%%%%%%%%%%%%



\subsection*{1.6. Algebraic Subsets of the Plane \\}
\addcontentsline{toc}{subsection}{1.6. Algebraic Subsets of the Plane}



\subsubsection*{Problem 1.30.}
\addcontentsline{toc}{subsubsection}{Problem 1.30.}
\emph{Let $k = \mathbb{R}$.}
\begin{enumerate}
\item[(a)]
  \emph{Show that $I(V(x^2+y^2+1))=(1)$.}

\item[(b)]
  \emph{Show that every algebraic subset of $\mathbf{A}^2(\mathbb{R})$ is equal to $V(f)$
  for some $f \in \mathbb{R}[x,y]$.}
\end{enumerate}
\emph{This indicates why we usually require that $k$ be algebraically closed.} \\



\emph{Proof of (a).}
$I(V(x^2+y^2+1)) = I(\varnothing) = (1)$
since $x^2+y^2+1 \geq 1$ is never zero for any $x, y \in \mathbb{R}$.
$\Box$\\



\emph{Proof of (b).}
\begin{enumerate}
\item[(1)]
  Given any algebraic subset $V$ of $\mathbf{A}^2(\mathbb{R})$.
  $V = V(1)$ if $V = \varnothing$.
  $V = V(0)$ if $V = \mathbf{A}^2(\mathbb{R})$.
  Now suppose $V$ is a nonempty proper algebraic subset $V$ of $\mathbf{A}^2(\mathbb{R})$.
  Write $V = V_1 \cup \cdots \cup V_m$,
  where each $V_i$ is irreducible.
  Here $V_i \neq \varnothing$ and $V_i \neq \mathbf{A}^2(\mathbb{R})$ for all $i$.

\item[(2)]
  As $k = \mathbb{R}$ is infinite,
  Corollary 2 to Proposition 2 implies that each $V_i$ is either a point
  or an irreducible plane curves $V(f_i)$,
  where $f_i$ is an irreducible polynomial and $V(f_i)$ is infinite.

\item[(3)]
  If $V_i = \{ (a_i,b_i) \}$ is a point, then define
  \[
    f_i(x,y) = (x-a_i)^2 + (x-b_i)^2.
  \]
  By the property of $\mathbb{R}$, $V_i = V(f_i)$.

\item[(4)]
  Define $f = f_1 \cdots f_m \in \mathbb{R}[x,y]$.
  Hence,
  \begin{align*}
    V
    &= V_1 \cup \cdots \cup V_m \\
    &= V(f_1) \cup \cdots \cup V(f_m) \\
    &= V(f_1 \cdots f_m) \\
    &= V(f).
  \end{align*}
\end{enumerate}
$\Box$\\\\



%%%%%%%%%%%%%%%%%%%%%%%%%%%%%%%%%%%%%%%%%%%%%%%%%%%%%%%%%%%%%%%%%%%%%%%%%%%%%%%%



\subsubsection*{Problem 1.31.}
\addcontentsline{toc}{subsubsection}{Problem 1.31.}
\begin{enumerate}
\item[(a)]
  \emph{Find the irreducible components of $V(y^2-xy-x^2y+x^3)$ in $\mathbf{A}^2(\mathbb{R})$,
  and also in $\mathbf{A}^2(\mathbb{C})$.}

\item[(b)]
  \emph{Do the same for $V(y^2 - x(x^2-1))$, and for $V(x^3+x-x^2y-y)$.} \\
\end{enumerate}



\emph{Proof of (a).}
\begin{enumerate}
\item[(1)]
  Note that
  \begin{align*}
    V(y^2-xy-x^2y+x^3)
    &= V((y-x^2)(y - x)) \\
    &= V(y-x^2) \cup V(y - x).
  \end{align*}

\item[(2)]
  Note that $y-x^2$ and $y-x$ are irreducible in $\mathbb{C}[x,y]$
  and thus also in $\mathbb{R}[x,y]$
  by the similar argument in Problem 1.25(a).
  Also, $V(y-x^2)$ and $V(y - x)$ are infinite in $\mathbf{A}^2(\mathbb{R})$
  and thus also in $\mathbf{A}^2(\mathbb{C})$.

\item[(3)]
  Therefore,
  $V(y-x^2)$ and $V(y - x)$ are the irreducible components of $V(y^2-xy-x^2y+x^3)$
  in $\mathbf{A}^2(\mathbb{R})$ and also in $\mathbf{A}^2(\mathbb{C})$.
\end{enumerate}
$\Box$\\



\emph{Outline of (b).}
\begin{enumerate}
\item[(1)]
  The elliptic curve $V(y^2 - x(x+1)(x-1))$ is irreducible over $\mathbf{A}^2(\mathbb{R})$.

\item[(2)]
  The elliptic curve $V(y^2 - x(x+1)(x-1))$ is irreducible over $\mathbf{A}^2(\mathbb{C})$.

\item[(3)]
  The irreducible component of $V(x^3+x-x^2y-y)$ over $\mathbf{A}^2(\mathbb{R})$ is $V(x - y)$.

\item[(4)]
  The irreducible components of $V(x^3+x-x^2y-y)$ over $\mathbf{A}^2(\mathbb{C})$ are
  $V(x + i)$, $V(x - i)$ and $V(x - y)$. \\
\end{enumerate}



\emph{Proof of (b).}
\begin{enumerate}
\item[(1)]
  Similar to Problem 1.25.
  To show $y^2 - x(x+1)(x-1)$ is irreducible in $\mathbb{C}[x,y]$,
  we write
  \[
    y^2 - x(x+1)(x-1) \in (\mathbb{C}[x])[y].
  \]
  Note that $\mathbb{C}[x]$ is a UFD and $-x(x+1)(x-1)$ is the constant term.
  As $\mathbb{C}[x]/(x) \cong \mathbb{C}$ is a domain, $(x)$ is prime.
  Clearly, $x \mid x(x+1)(x-1)$ but $x^2 \nmid x(x+1)(x-1)$.
  By the Eisenstein's criterion,
  we can say $y^2 - x(x+1)(x-1)$ is irreducible over $(\mathbb{C}[x])[y]$.

\item[(2)]
  Moreover, $V(y^2 - x(x+1)(x-1))$ is infinite over $\mathbf{A}^2(\mathbb{R})$
  and thus also over $\mathbf{A}^2(\mathbb{C})$.
  ($y = f(x) = \sqrt{x(x+1)(x-1)}$ is continuous and strictly increasing on $[1,\infty)$
  in the sense of calculus.
  As the measure of $[1,\infty)$ is $\infty$, the set $V(y^2 - x(x+1)(x-1))$ is infinite
  over $\mathbf{A}^2(\mathbb{R})$.)

\item[(3)]
  By Corollary 1 to Proposition 2,
  $V(y^2 - x(x^2-1))$ itself is irreducible over $\mathbf{A}^2(\mathbb{R})$
  or $\mathbf{A}^2(\mathbb{C})$.

\item[(4)]
  Consider $V(x^3+x-x^2y-y) \subseteq \mathbf{A}^2(\mathbb{R})$.
  \begin{align*}
    V(x^3+x-x^2y-y)
    &= V((x^2 + 1)(x - y)) \\
    &= V(x^2 + 1) \cup V(x - y) \\
    &= \varnothing \cup V(x - y) \\
    &= V(x - y).
  \end{align*}
  Here we use that fact that
  $x^2 + 1 = 0$ has no real solution $x \in \mathbb{R}$.
  Similar to (a), $V(x - y)$ is the only irreducible component of $V(x^3+x-x^2y-y)$
  in $\mathbf{A}^2(\mathbb{R})$.

\item[(5)]
  Consider $V(x^3+x-x^2y-y) \subseteq \mathbf{A}^2(\mathbb{C})$.
  \begin{align*}
    V(x^3+x-x^2y-y)
    &= V((x + i)(x - i)(x - y)) \\
    &= V(x + i) \cup V(x - i) \cup V(x - y).
  \end{align*}
  Similar to (a), $V(x \pm i)$ and $V(x - y)$ are the irreducible components of $V(x^3+x-x^2y-y)$
  in $\mathbf{A}^2(\mathbb{C})$.
\end{enumerate}
$\Box$\\\\



%%%%%%%%%%%%%%%%%%%%%%%%%%%%%%%%%%%%%%%%%%%%%%%%%%%%%%%%%%%%%%%%%%%%%%%%%%%%%%%%
%%%%%%%%%%%%%%%%%%%%%%%%%%%%%%%%%%%%%%%%%%%%%%%%%%%%%%%%%%%%%%%%%%%%%%%%%%%%%%%%



\subsection*{1.7. Hilbert's Nullstellensatz \\}
\addcontentsline{toc}{subsection}{1.7. Hilbert's Nullstellensatz}



\subsubsection*{Problem 1.32.}
\addcontentsline{toc}{subsubsection}{Problem 1.32.}
\emph{Show that both theorems and all of the corollaries are false
if $k$ is not algebraically closed.} \\



\emph{Proof.}
\begin{enumerate}
\item[(1)]
  Weak Nullstellensatz:
  $I = (x^2+1)$ is a proper ideal in $\mathbb{R}[x]$ but $V(I) = \varnothing$.

\item[(2)]
  Hilbert's Nullstellensatz:
  Let $I = (y^2 + x^2(x-1)^2)$ be an ideal in $\mathbb{R}[x,y]$.
  Hence,
  \begin{align*}
    I(V(I))
    &= I(\{ (0,0), (1,0) \})
      &(\text{Problem 1.26.}) \\
    &= (x(x-1),y) \\
    &\neq I \\
    &= \mathrm{rad}(I).
  \end{align*}
  The last equality holds since $f$ is irreducible in a UFD $\mathbb{R}[x,y]$
  and thus $I$ is a prime ideal.

\item[(3)]
  Corollary 1: Same example in the case Hilbert's Nullstellensatz.
  If $I = (y^2 + x^2(x-1)^2)$ is a radical ideal in $\mathbb{R}[x,y]$.
  Then $I(V(I)) \neq I$.

\item[(4)]
  Corollary 2: Same example in the case Hilbert's Nullstellensatz.
  If $I = (y^2 + x^2(x-1)^2)$ is a prime ideal in $\mathbb{R}[x,y]$,
  then
  \[
    V(I) = \{ (0,0), (1,0) \} = V(x,y) \cup V(x-1,y)
  \]
  is reducible.
  Next, consider a prime ideal $J = (x^2+y^2)$ in $\mathbb{R}[x,y]$.
  (Use the same argument in Problem 1.26 to get the irreducibility of $x^2+y^2$.)
  $V(J) = \{ (0,0) \}$ is a point but $J$ is not a maximal ideal
  (since $J \subsetneq (x^2+y^2,x) \subsetneq (1)$).

\item[(5)]
  Corollary 3: Same example in Corollary 2.

\item[(6)]
  Corollary 4:
  Let $I = (x^2 + y^2)$ be an ideal in $\mathbb{R}[x,y]$.
  Then $V(I) = \{ (0,0) \}$ is a finite set.
  But $\mathbb{R}[x,y]/(x^2 + y^2)$ is an infinite dimensional vector space over $\mathbb{R}$.
  In fact, the monomials
  \[
    \{ \overline{x^m}, \overline{x^m y} : m = 0, 1, 2, \ldots \}
  \]
  is a basis for $\mathbb{R}[x,y]/(x^2 + y^2)$.
\end{enumerate}
$\Box$\\\\



%%%%%%%%%%%%%%%%%%%%%%%%%%%%%%%%%%%%%%%%%%%%%%%%%%%%%%%%%%%%%%%%%%%%%%%%%%%%%%%%



\subsubsection*{Problem 1.33.}
\addcontentsline{toc}{subsubsection}{Problem 1.33.}

\begin{enumerate}
\item[(a)]
  \emph{Decompose $V(x^2+y^2-1, x^2-z^2-1) \subseteq \mathbf{A}^3(\mathbb{C})$
  into irreducible components.}

\item[(b)]
  \emph{Let $V = \{(t,t^2,t^3) \in \mathbf{A}^3(\mathbb{C}) : t \in \mathbb{C} \}$.
  Find $I(V)$, and show that $V$ is irreducible.} \\
\end{enumerate}



\emph{Proof of (a).}
\begin{enumerate}
\item[(1)]
  Write
  \begin{align*}
    &\: V(x^2+y^2-1, x^2-z^2-1) \\
    =&\: V(x^2+y^2-1, y^2+z^2) \\
    =&\: V(x^2+y^2-1, (y + iz)(y - iz)) \\
    =&\: V(x^2+y^2-1, y + iz) \cup V(x^2+y^2-1, y - iz).
  \end{align*}
  By the Hilbert's Nullstellensatz,
  it suffices to show that $(x^2+y^2-1, y + iz)$ and $(x^2+y^2-1, y - iz)$ are prime.

\item[(2)]
  \emph{Show that $I = (x^2+y^2-1, y + iz)$ is prime in $\mathbb{C}[x,y,z]$.}
  Note that
  \[
    \mathbb{C}[x,y,z]/I \cong \mathbb{C}[x,y]/(x^2+y^2-1)
  \]
  is a ring isomorphism defined by
  \[
    f(x,y,z) + I \mapsto f(x,y,-iy) + (x^2+y^2-1).
  \]
  (Use the similar argument in (b) to prove it is indeed an isomorphism.)
  So it suffices to show that
  \[
    x^2+y^2-1 \in \mathbb{C}[x,y]
  \]
  is irreducible.
  (Thus, $\mathbb{C}[x,y]/(x^2+y^2-1) \cong \mathbb{C}[x,y,z]/I$ is a domain, or $I$ is prime.)
  We can use the similar argument in Problem 1.31 (b) to show
  $x^2+y^2-1 = y^2 + (x+1)(x-1)$ is irreducible
  as showing the irreducibility of $y^2 - x(x+1)(x-1)$.

\item[(3)]
  Similarly, $I = (x^2+y^2-1, y - iz)$ is prime.
  Therefore,
  the irreducible components of $V(x^2+y^2-1, x^2-z^2-1)$
  are $V(x^2+y^2-1, y + iz)$ and $V(x^2+y^2-1, y - iz)$.
\end{enumerate}
$\Box$\\



\emph{Proof of (b).}
\begin{enumerate}
\item[(1)]
  Write
  \[
    V = \{ (t,t^2,t^3) \in \mathbf{A}^3(\mathbb{C}) : t \in \mathbb{C} \}
    =
    V(x^2-y, x^3-z).
  \]
  Let $I = (x^2-y, x^3-z)$ in $\mathbb{C}[x,y,z]$.
  By the Hilbert's Nullstellensatz,
  $I(V) = \mathrm{rad}(I)$.
  So it suffices to show that $I = (x^2-y, x^3-z)$ is prime
  (and thus $V$ is irreducible).

\item[(2)]
  \emph{Show that
  \[
    \mathbb{C}[x,y,z]/I \cong \mathbb{C}[t]
  \]
  is a domain, and thus $I = (x^2-y, x^3-z)$ is a prime ideal.}
  \begin{enumerate}
  \item[(a)]
    Define a ring homomorphism $\alpha: \mathbb{C}[x,y,z]/I \to \mathbb{C}[t]$
    by
    \[
      \alpha: f(x,y,z) + I \mapsto f(t,t^2,t^3).
    \]
    $\alpha$ is well-defined since $\alpha((x^2-y)+I) = 0$ and $\alpha((x^3-z)+I) = 0$.

  \item[(b)]
    \emph{Show that $\alpha$ is surjective.}
    \[
      \alpha: g(x) + I \in \mathbb{C}[x,y,z]/I \mapsto g(t) \in \mathbb{C}[t]
    \]
    for any $g(t)$.

  \item[(c)]
    \emph{Show that $\alpha$ is injective.}
    Suppose $\alpha(f(x,y,z) + I) = 0$.
    Write
    \begin{align*}
      f(x,y,z) + I
      &= \sum_{(i)} \lambda_{(i)} x^{i_1} (y-x^2)^{i_2} (z-x^3)^{i_3} + I \\
      &= \sum_{i} \lambda_{i} x^{i} + I.
    \end{align*}
    So
    \[
      0
      = \alpha(f(x,y,z)+I)
      = \alpha\left(\sum_{i} \lambda_{i} x^{i}+I\right)
      = \sum_{i} \lambda_{i} t^{i}.
    \]
    Hence, $\ker(\alpha) = I$.
  \end{enumerate}
\end{enumerate}
$\Box$\\\\



%%%%%%%%%%%%%%%%%%%%%%%%%%%%%%%%%%%%%%%%%%%%%%%%%%%%%%%%%%%%%%%%%%%%%%%%%%%%%%%%



\subsubsection*{Problem 1.34.}
\addcontentsline{toc}{subsubsection}{Problem 1.34.}
\emph{Let $R$ be a UFD.}
\begin{enumerate}
\item[(a)]
  \emph{Show that a monic polynomial of degree two or three in $R[x]$ is irreducible
  if and only if it has no root in $R$.}

\item[(b)]
  \emph{$x^2-a \in R[x]$ is irreducible if and only if $a$ is not a square in $R$.} \\
\end{enumerate}



\emph{Proof of (a).}
\begin{enumerate}
\item[(1)]
  It is equivalent to show that
  a monic polynomial of degree two or three in $R[x]$ is reducible
  if and only if it has one root in $R$.

\item[(2)]
  Suppose $f$ is reducible of degree $2$ or $3$.
  Then there exist nonconstant monic polynomials $g, h \in R[x]$ such that $f = gh$.
  By
  \[
    \deg(g) + \deg(h) = \deg(f) = 2 \text{ or } 3,
  \]
  we may assume that $\deg(g) = 1$. (Otherwise $g$ or $h$ will be a constant polynomial.)
  Say $g(x) = x - a$ where $a \in R$.
  Now
  \[
    f(a) = g(a)h(a) = 0
  \]
  implies that $a \in R$ is a root of $f$.

\item[(3)]
  Conversely, if $a \in R$ is a root of $f$, then
  apply the same argument in Problem 1.7 we can write
  \[
    f = (x - a)g
  \]
  for some $g \in R[x]$.
  Here $\deg(g) \geq 1$ since $\deg(f) = 1 + \deg(g) \geq 2$.
  Therefore, $f$ is reducible.
\end{enumerate}
$\Box$\\



\emph{Proof of (b).}
  By (a),
  $x^2-a \in R[x]$ is reducible $\Longleftrightarrow$
  $x^2-a$ has one root $\alpha \in R$ $\Longleftrightarrow$
  $a = \alpha^2$ is a square in $R$ for some $\alpha \in R$.
$\Box$\\\\



%%%%%%%%%%%%%%%%%%%%%%%%%%%%%%%%%%%%%%%%%%%%%%%%%%%%%%%%%%%%%%%%%%%%%%%%%%%%%%%%



\subsubsection*{Problem 1.35.}
\addcontentsline{toc}{subsubsection}{Problem 1.35.}
\emph{Show that $V(y^2-x(x-1)(x-\lambda)) \subseteq \mathbf{A}^2(k)$
is an irreducible curve for any algebraically closed field $k$,
and any $\lambda \in k$.} \\

\emph{Proof.}
\begin{enumerate}
\item[(1)]
  By the Hilbert's Nullstellensatz, it suffices to show that
  \[
    I = (y^2-x(x-1)(x-\lambda))
  \]
  is a prime ideal in $k[x,y]$, or show that
  \[
    y^2-x(x-1)(x-\lambda)
  \]
  is irreducible (since $k[x,y]$ is a UFD).

\item[(2)]
  By Problem 1.34(b),
  $y^2-x(x-1)(x-\lambda) \in (\mathbb{C}[x])[y]$ is irreducible
  if $x(x-1)(x-\lambda)$ is not a square in $\mathbb{C}[x]$.
  Note that every square in $\mathbb{C}[x]$ is of even degree.
  So $x(x-1)(x-\lambda)$ cannot be a square in $\mathbb{C}[x]$
  since $\deg(x(x-1)(x-\lambda)) = 3$ is odd.
\end{enumerate}
$\Box$\\

\emph{Note.}
  $V(y^2-x(x-1)(x-\lambda))$ is the elliptic curve as Problem 1.31. \\\\



%%%%%%%%%%%%%%%%%%%%%%%%%%%%%%%%%%%%%%%%%%%%%%%%%%%%%%%%%%%%%%%%%%%%%%%%%%%%%%%%



\subsubsection*{Problem 1.36.}
\addcontentsline{toc}{subsubsection}{Problem 1.36.}
\emph{Let $I = (y^2-x^2, y^2+x^2) \subseteq \mathbb{C}[x,y]$.
Find $V(I)$ and $\dim_{\mathbb{C}}(\mathbb{C}[x,y]/I)$.} \\

\emph{Proof.}
\begin{enumerate}
\item[(1)]
  Clearly, $V(I) = \{ (0,0) \}$ is a finite set.
  By Corollary 4 to the Hilbert's Nullstellensatz,
  \[
    \dim_{\mathbb{C}}(\mathbb{C}[x,y]/I) < \infty.
  \]
  In fact, $\dim_{\mathbb{C}}(\mathbb{C}[x,y]/I) = 4$.

\item[(2)]
  Given any $f + I \in \mathbb{C}[x,y]/I$ where $f \in \mathbb{C}[x,y]$.
  Write
  \[
    f(x,y) = \sum_{i} f_i(x) y^i
  \]
  where $f_i(x) = \sum_{j} a_{ij} x^{j} \in \mathbb{C}[x]$.
  Note that
  \begin{align*}
    x^2 &= \frac{1}{2}(y^2+x^2) - \frac{1}{2}(y^2-x^2) \in I, \\
    y^2 &= \frac{1}{2}(y^2+x^2) + \frac{1}{2}(y^2-x^2) \in I.
  \end{align*}
  So
  \begin{align*}
    f(x,y) + I
    &= \sum_{i} f_i(x) y^i + I \\
    &= f_0(x) + f_1(x) y + I \\
    &= \sum_{j} a_{0j} x^{j} + \left(\sum_{j} a_{1j} x^{j}\right) y + I \\
    &= a_{00} + a_{01} x + a_{10} y + a_{11} xy + I
  \end{align*}
  is generated by $\mathscr{B} = \{ \overline{1}, \overline{x}, \overline{y}, \overline{xy} \}$.

\item[(3)]
  Note that $\mathscr{B}$ is a basis
  since any linear combination of elements in $\mathscr{B}$ is not in $I$.
  Therefore,
  \[
    \dim_{\mathbb{C}}(\mathbb{C}[x,y]/I) = |\mathscr{B}| = 4.
  \]
\end{enumerate}
$\Box$\\\\



%%%%%%%%%%%%%%%%%%%%%%%%%%%%%%%%%%%%%%%%%%%%%%%%%%%%%%%%%%%%%%%%%%%%%%%%%%%%%%%%



\subsubsection*{Problem 1.37.*}
\addcontentsline{toc}{subsubsection}{Problem 1.37.*}
\emph{Let $K$ be any field, $f \in K[x]$ a polynomial of degree $n > 0$.
Show that the residues $\overline{1}, \overline{x}, \ldots, \overline{x}^{n-1}$
form a basis for $K[x]/(f)$ over $K$.} \\

\emph{Proof.}
\begin{enumerate}
\item[(1)]
  \emph{Show that every element in $K[x]/(f)$ is generated by
  $\mathscr{B} = \{ \overline{1}, \overline{x}, \ldots, \overline{x}^{n-1} \}$.}
  Given any $\overline{g} \in K[x]/(f)$ with $g \in K[x]$.
  By the division-with-remainder property of $K[x]$,
  there are some polynomials $q, r \in K[x]$ such that
  \[
    g = fq + r
  \]
  where $r = 0$ or $\deg(r) < n$ if $r \neq 0$.
  Therefore,
  \[
    g + (f) = fq + r + (f) = r + (f).
  \]
  Note that $r + (f)$ is generated by $\mathscr{B}$.

\item[(2)]
  \emph{Show that $\mathscr{B}$ is a basis for $K[x]/(f)$ over $K$.}
  Suppose
  \[
    a_0 + a_1 x + \cdots + a_{n-1} x^{n-1} \in (f)
  \]
  for $a_1, \ldots, a_{n-1} \in K$.
  We can regard any linear combination of $\{ 1, x, \ldots, x^{n-1} \}$
  as a polynomial $r(x)$ in $K[x]$.
  $r \in (f)$ implies that there exists a polynomial $g \in K[x]$
  such that $r = fg$.
  If $g \neq 0$, then $\deg(r) = \deg(f) + \deg(g) \geq n$, which is impossible.
  So $g = 0$ and thus $r = fg = 0 \in K[x]$.
  Therefore, $a_0 = a_1 = \cdots = a_{n-1} = 0 \in K$ and
  \[
    \dim_{K}(K[x]/(f)) = \deg(f).
  \]
\end{enumerate}
$\Box$\\\\



%%%%%%%%%%%%%%%%%%%%%%%%%%%%%%%%%%%%%%%%%%%%%%%%%%%%%%%%%%%%%%%%%%%%%%%%%%%%%%%%



\subsubsection*{Problem 1.38.*}
\addcontentsline{toc}{subsubsection}{Problem 1.38.*}
\emph{Let $R = k[x_1,\ldots,x_n]$, $k$ algebraically closed, $V = V(I)$.
Show that there is a natural one-to-one correspondence
between algebraic subsets of $V$ and radical ideals in $k[x_1,\ldots,x_n]/I$,
and that irreducible algebraic sets (resp. points) correspond to prime ideals (resp. maximal ideals).
(See Problem 1.22.)} \\



\emph{Proof.}
\begin{enumerate}
\item[(1)]
  Given any algebraic subset $W$ of $V$.
  By the Hilbert's Nullstellensatz,
  \[
    I(W) \supseteq I(V) = \mathrm{rad}(I) \supseteq I.
  \]

\item[(2)]
  By Corollary 1 to the Hilbert's Nullstellensatz and Problem 1.22(b),
  we have a one-to-one correspondence such that
  \begin{align*}
    &\: \{ \text{algebraic subsets of $V$} \} \\
    \longleftrightarrow&\:
    \{ \text{radical ideals containing $I$} \} \\
    \longleftrightarrow&\:
    \{ \text{radical ideals of $k[x_1,\ldots,x_n]/I$} \}.
  \end{align*}

\item[(3)]
  Again
  by Corollary 2 to the Hilbert's Nullstellensatz and Problem 1.22(b),
  we have a one-to-one correspondence such that
  \begin{align*}
    &\: \{ \text{irreducible algebraic subsets (resp. points) of $V$} \} \\
    \longleftrightarrow&\:
    \{ \text{prime (resp. maximal) ideals containing $I$} \} \\
    \longleftrightarrow&\:
    \{ \text{prime (resp. maximal) ideals of $k[x_1,\ldots,x_n]/I$} \}.
  \end{align*}
\end{enumerate}
$\Box$\\\\



%%%%%%%%%%%%%%%%%%%%%%%%%%%%%%%%%%%%%%%%%%%%%%%%%%%%%%%%%%%%%%%%%%%%%%%%%%%%%%%%



\subsubsection*{Problem 1.39.}
\addcontentsline{toc}{subsubsection}{Problem 1.39.}
\begin{enumerate}
\item[(a)]
  \emph{Let $R$ be a UFD, and let $\mathfrak{p} = (t)$ be a principal proper prime ideal.
  Show that there is no prime ideal $\mathfrak{q}$
  such that $0 \subsetneq \mathfrak{q} \subsetneq \mathfrak{p}$.}

\item[(b)]
  \emph{Let $V = V(f)$ be irreducible hypersurface in $\mathbf{A}^n$.
  Show that there is no irreducible algebraic set $W$ such that
  $V \subsetneq W \subsetneq \mathbf{A}^n$.} \\
\end{enumerate}



\emph{Proof of (a).}
\begin{enumerate}
\item[(1)]
  (Reductio ad absurdum)
  Suppose that $\mathfrak{q}$ were a prime ideal in $R$ such that
  $0 \subsetneq \mathfrak{q} \subsetneq \mathfrak{p}$.

\item[(2)]
  \emph{Show that there is an irreducible element in $\mathfrak{q}$.}
  Given any $q \in \mathfrak{q}$. Since $\mathfrak{q}$ is proper,
  we can write
  \[
    q = q_1 \cdots q_n
  \]
  as a product of irreducible elements in a UFD.
  Since $\mathfrak{q}$ is prime, there is one irreducible element $q_i \in \mathfrak{q}$.

\item[(3)]
  Now $q_i \in \mathfrak{q} \subseteq \mathfrak{p} = (t)$.
  So $q_i = ut$ for some $u \in R$.
  By the irreducibility of $q_i$, $u$ is a unit or $t$ is a unit.
  If $u$ is a unit, then
  \[
    (t) = (q_i) \subseteq \mathfrak{q} \subseteq \mathfrak{p} = (t).
  \]
  So $\mathfrak{q} = \mathfrak{p}$, which is absurd.
  If $t$ is a unit, then $\mathfrak{p} = (1)$, contrary to the primality of $\mathfrak{p}$.
\end{enumerate}
$\Box$\\



\emph{Proof of (b).}
\begin{enumerate}
\item[(1)]
  We might assume that $k = \overline{k}$.
  By Corollary 3 to the Hilbert's Nullstellensatz and the irreducibility of $V(f)$,
  there are an irreducible polynomial $g \in k[x_1,\ldots,x_n]$ and an integer $m > 0$
  such that
  \[
    f = g^m,
  \]
  and
  \[
    I(V(f)) = (g).
  \]

\item[(2)]
  (Reductio ad absurdum)
  Suppose that there were an irreducible algebraic set $W$ such that
  $V \subsetneq W \subsetneq \mathbf{A}^n$.
  Then by Corollary 3 to the Hilbert's Nullstellensatz again,
  \[
    (g) = I(V(f)) \supsetneq I(W) \supsetneq (1) \in k[x_1,\ldots,x_n].
  \]
  Here $(g) = I(V(f))$ and $I(W)$ are all prime.

\item[(3)]
  Note that $(g)$ is a principal proper prime ideal in a UFD $k[x_1,\ldots,x_n]$.
  By (a), such ideal $I(W)$ cannot be prime, which is absurd.
\end{enumerate}
$\Box$\\\\



%%%%%%%%%%%%%%%%%%%%%%%%%%%%%%%%%%%%%%%%%%%%%%%%%%%%%%%%%%%%%%%%%%%%%%%%%%%%%%%%



\subsubsection*{Problem 1.40.}
\addcontentsline{toc}{subsubsection}{Problem 1.40.}
\emph{Let $I = (x^2-y^3, y^2-z^3) \subseteq k[x,y,z]$.
Define $\alpha: k[x,y,z] \to k[t]$ by
$\alpha(x) = t^9$, $\alpha(y) = t^6$, $\alpha(z) = t^4$.}

\begin{enumerate}
\item[(a)]
  \emph{Show that every element of $k[x,y,z]/I$ is the residue of an element
  $a + xb + yc + xyd$, for some $a, b, c, d \in k[z]$.}

\item[(b)]
  \emph{If $f = a + xb + yc + xyd$, $a, b, c, d \in k[z]$ and $\alpha(f) = 0$,
  compare like powers of $t$ to conclude that $f = 0$.}

\item[(c)]
  \emph{Show that $\ker(\alpha) = I$, so $I$ is prime,
  $V(I)$ is irreducible, and $I(V(I)) = I$.} \\
\end{enumerate}



\emph{Proof of (a).}
\begin{enumerate}
\item[(1)]
  Take any element $\overline{f} \in k[x,y,z]/I$ where $f \in k[x,y,z]$.
  Regard $f \in (k[y,z])[x]$,
  By the division-with-remainder property of $(k[y,z])[x]$,
  \[
    f = (x^2 - y^3)q + r
  \]
  where $q, r \in (k[y,z])[x]$ and $r = 0$ or $\deg_{x}(r) < 2$.
  In any case,
  $r = x r_1 + r_0$ for some $r_1, r_0 \in k[y,z]$.

\item[(2)]
  Apply the same argument to (1),
  we have
  \begin{align*}
    r_0 &= (y^2 - z^3)q_0 + y c + a \\
    r_1 &= (y^2 - z^3)q_1 + y d + b
  \end{align*}
  where $q_0, q_1 \in k[y,z]$ and $a, b, c, d \in k[z]$.

\item[(3)]
  By $\overline{r_0} = \overline{y c} + \overline{a}$
  and $\overline{r_1} = \overline{y d} + \overline{b}$,
  \begin{align*}
    \overline{f}
    &= \overline{r} \\
    &= \overline{x} \overline{r_1} + \overline{r_0} \\
    &= \overline{x} (\overline{y d} + \overline{b}) + (\overline{y c} + \overline{a}) \\
    &= \overline{a} + \overline{b} \cdot \overline{x}
      + \overline{c} \cdot \overline{y} + \overline{d} \cdot \overline{xy}.
  \end{align*}
\end{enumerate}
$\Box$\\



\emph{Proof of (b).}
  As $0 = \alpha(f) = a + ct^6 + bt^9 + dt^{15} \in k[t]$,
  $a = b = c = d = 0 \in k$.
$\Box$\\



\emph{Proof of (c).}
\begin{enumerate}
\item[(1)]
  $I \subseteq \ker(\alpha)$ is trivial.

\item[(2)]
  \emph{Show that $\ker(\alpha) \subseteq I$.}
  Take any $f \in \ker(\alpha)$, or $\alpha(f) = 0$.
  By (a), $f = r + f_1$ where $f_1 \in I$ and $r = a + bx + cy + dxy \in k[x,y,z]$
  for some $a, b, c, d \in k[z]$.
  Note that $\alpha$ is a ring homomorphism.
  Therefore,
  \[
    0 = \alpha(f) = \alpha(r + f_1) = \alpha(r) + \alpha(g) = \alpha(r).
  \]
  By (b), $r = 0 \in k[x,y,z]$ and thus $f = f_1 \in I$.

\item[(3)]
  Therefore,
  \[
    \alpha: k[x,y,z]/(x^2-y^3, y^2-z^3) \hookrightarrow k[t]
  \]
  is injective.
\end{enumerate}
$\Box$\\\\



%%%%%%%%%%%%%%%%%%%%%%%%%%%%%%%%%%%%%%%%%%%%%%%%%%%%%%%%%%%%%%%%%%%%%%%%%%%%%%%%
%%%%%%%%%%%%%%%%%%%%%%%%%%%%%%%%%%%%%%%%%%%%%%%%%%%%%%%%%%%%%%%%%%%%%%%%%%%%%%%%



\subsection*{1.8. Modules; Finiteness Conditions \\}
\addcontentsline{toc}{subsection}{1.8. Modules; Finiteness Conditions}



\subsubsection*{Problem 1.41.*}
\addcontentsline{toc}{subsubsection}{Problem 1.41.*}
\emph{If $S$ is module-finite over $R$, then $S$ is ring-finite over $R$.} \\

\emph{Proof.}
\begin{enumerate}
\item[(1)]
  Write $S = \sum R s_i$ for some $s_1, \ldots, s_n \in S$
  since $S$ is module-finite over $R$.

\item[(2)]
  \emph{Show that $\sum R s_i = R[s_1,\ldots,s_n]$.}
  $\sum R s_i \subseteq R[s_1,\ldots,s_n]$ is trivial.
  Conversely, take any $v \in R[s_1,\ldots,s_n]$.
  Write
  \[
    v = \sum_{(j)} \overbrace{
      \underbrace{a_{(j)}}_{\in R}
      \underbrace{s_1^{j_1} \cdots s_n^{j_n}}_{\in S = \sum R s_i}}^{\in \sum R s_i}
  \]
  Here each term $a_{(i)} s_1^{i_1} \cdots s_n^{i_n}$ is in $\sum R s_i$.
  As $\sum R s_i$ is an $R$-module,
  \[
    v = \sum_{(i)} a_{(i)} s_1^{i_1} \cdots s_n^{i_n} \in \sum R s_i.
  \]
\end{enumerate}
$\Box$\\



\emph{Note.}
  The converse is not true (by Problem 1.42). \\\\



%%%%%%%%%%%%%%%%%%%%%%%%%%%%%%%%%%%%%%%%%%%%%%%%%%%%%%%%%%%%%%%%%%%%%%%%%%%%%%%%



\subsubsection*{Problem 1.42.}
\addcontentsline{toc}{subsubsection}{Problem 1.42.}
\emph{Show that $S = R[x]$ (the ring of polynomials in one variable)
is ring-finite over $R$, but not module-finite.} \\

\emph{Proof.}
\begin{enumerate}
\item[(1)]
  $S = R[x]$ is ring-finite over $R$ by definition (as $x \in S$).

\item[(2)]
  (Reductio ad absurdum)
  If $S = \sum R s_i$ for some $s_1, \ldots, s_n \in S$ were module-finite over $R$.
  Any element $s \in \sum R s_i$ is of degree
  \[
    \deg s \leq \max_{1 \leq i \leq n} \deg s_i := m.
  \]
  So that $x^{m+1} \in S = R[x]$ but not in $\sum R s_i$,
  which is absurd.
\end{enumerate}
$\Box$\\\\



%%%%%%%%%%%%%%%%%%%%%%%%%%%%%%%%%%%%%%%%%%%%%%%%%%%%%%%%%%%%%%%%%%%%%%%%%%%%%%%%



\subsubsection*{Problem 1.43.*}
\addcontentsline{toc}{subsubsection}{Problem 1.43.*}
\emph{If $L$ is ring-finite over $K$ ($K$, $L$ fields)
then $L$ is a finitely generated field extension of $K$.} \\

\emph{Proof.}
\begin{enumerate}
\item[(1)]
  $L = K[v_1, \cdots, v_n]$ for some $v_i \in L$ since $L$ is ring-finite over $K$.

\item[(2)]
  Apply Proposition 4 in \S 1.10,
  $L$ is module-finite (and hence algebraic) over $K$,
  that is, $L = K[v_1, \cdots, v_n] = K(v_1, \cdots, v_n)$
  is a finitely generated field extension of $K$.
\end{enumerate}
$\Box$\\\\



%%%%%%%%%%%%%%%%%%%%%%%%%%%%%%%%%%%%%%%%%%%%%%%%%%%%%%%%%%%%%%%%%%%%%%%%%%%%%%%%



\subsubsection*{Problem 1.44.*}
\addcontentsline{toc}{subsubsection}{Problem 1.44.*}
\emph{Show that $L = K(x)$ (the field of rational functions in one variable)
is a finitely generated field extension of $K$, but $L$ is not ring-finite over $K$.
(Hint: If $L$ were ring-finite over $K$,
a common denominator of ring generators would be an element $b \in K[x]$ such that
for all $z \in L$, $b^n z \in K[x]$ for some $n$;
but let $z = 1/c$, where $c$ doesn't divide $b$ (Problem 1.5).)} \\

\emph{Proof.}
\begin{enumerate}
\item[(1)]
  (Reductio ad absurdum)
  Suppose that $L$ were ring-finite over $K$.
  Write $L = K[v_1,\ldots,v_m]$
  where $v_1,\ldots,v_m \in L = K(x)$.
  Let $b \in K[x]$ be a common denominator of ring generators $v_1, \ldots, v_m$.
  (So that all $bv_i \in K[x]$.)
  Therefore, for any $z \in L = K[v_1,\ldots,v_m]$, there is an integer $n > 0$
  such that $b^n z \in K[x]$.

\item[(2)]
  Consider $z = 1/c \in K(x)$, where $c \in K[x]$ doesn't divide $b$.
  The existence of $c$ is guaranteed by Problem 1.5.
  Hence, for any integer $n > 0$
  \[
    b^n z = b^n/c
  \]
  is never in $K[x]$ by the construction of $c$, which is absurd.
\end{enumerate}
$\Box$\\\\



%%%%%%%%%%%%%%%%%%%%%%%%%%%%%%%%%%%%%%%%%%%%%%%%%%%%%%%%%%%%%%%%%%%%%%%%%%%%%%%%



\subsubsection*{Problem 1.45.*}
\addcontentsline{toc}{subsubsection}{Problem 1.45.*}
\emph{Let $R$ be a subring of $S$, $S$ a subring of $T$.}
\begin{enumerate}
\item[(a)]
  \emph{If $S = \sum R v_i$, $T = \sum S w_j$, show that $T = \sum R v_i w_j$.}

\item[(b)]
  \emph{If $S = R[v_1,\ldots,v_n]$, $T = S[w_1,\ldots,w_m]$,
  show that $T = R[v_1,\ldots,v_n,w_1,\ldots,w_m]$.}

\item[(c)]
  \emph{If $R$, $S$, $T$ are fields, and $S = R(v_1,\ldots,v_n)$, $T = S(w_1,\ldots,w_m)$,
  show that $T = R(v_1,\ldots,v_n,w_1,\ldots,w_m)$.}
\end{enumerate}
\emph{So each of the three finiteness conditions is a transitive relation.} \\



\emph{Proof of (a).}
\begin{enumerate}
\item[(1)]
  \emph{Show that $T \subseteq \sum R v_i w_j$.}
  Given any $t \in T = \sum S w_j$.
  There are some $s_j \in S$ such that $t = \sum_{j} s_j w_j$.
  As $s_j \in S = \sum R v_i$, there are some $r_{ij} \in R$ such that $s_j = \sum_{i} r_{ij} v_i$.
  Hence,
  \[
    t
    = \sum_{j} s_j w_j
    = \sum_{j} \left(\sum_{i} r_{ij} v_i\right) w_j
    = \sum_{i,j} r_{ij} v_i w_j
    \in \sum R v_i w_j.
  \]

\item[(2)]
  \emph{Show that $T \supseteq \sum R v_i w_j$.}
  Take any $\sum r_{ij} v_i w_j \in \sum R v_i w_j$.
  \[
    \sum r_{ij} v_i w_j
    = \sum_{j} \left(\sum_{i} r_{ij} v_i\right) w_j
    \in \sum_{j} S w_j = T.
  \]

\end{enumerate}
$\Box$\\



\emph{Proof of (b).}
\begin{enumerate}
\item[(1)]
  Note that $R[x_1,\cdots,x_m]$ is canonically isomorphic to $R[x_1,\cdots,x_{m-1}][x_m]$.
  Hence $R[x_1,\cdots,x_m]$ is isomorphic to $R[x_1][x_2] \cdots [x_m]$.

\item[(2)]
  Hence,
  \begin{align*}
    T
    &= S[w_1,\ldots,w_m] \\
    &= R[v_1,\ldots,v_n][w_1,\ldots,w_m] \\
    &= R[v_1,\ldots,v_n][w_1] \cdots [w_m] \\
    &= R[v_1] \cdots [v_n][w_1] \cdots [w_m] \\
    &= R[v_1,\ldots,v_n,w_1,\ldots,w_m].
  \end{align*}
\end{enumerate}
$\Box$\\



\emph{Proof of (c).}
\begin{enumerate}
\item[(1)]
  By (b),
  $R(v_1,\ldots,v_n)$ is canonically isomorphic to $R(v_1,\ldots,v_{n-1})(v_n)$.
  Hence $R(v_1,\ldots,v_n)$ is isomorphic to $R(v_1) \cdots (v_n)$.
  To see this, note that $R[x_1,\cdots,x_m] \cong R[x_1,\cdots,x_{m-1}][x_m]$
  implies that
  \[
    R(x_1,\cdots,x_m)
    \cong R[x_1,\cdots,x_{m-1}](x_m)
    \hookrightarrow R(x_1,\cdots,x_{m-1})(x_m).
  \]
  Conversely,
  for any $a/b \in R(x_1,\cdots,x_{m-1})(x_m)$
  where
  \begin{align*}
    a &= \sum_{i} a_i x_m^{i} \in R(x_1,\cdots,x_{m-1})[x_m], \\
    b &= \sum_{j} b_j x_m^{j} \in R(x_1,\cdots,x_{m-1})[x_m]
  \end{align*}
  and $b \neq 0$, there is a nonzero polynomial $c \in R[x_1,\cdots,x_{m-1}]$
  such that all $c a_i$ and $c b_j$ are in $R[x_1,\cdots,x_{m-1}]$.
  Hence,
  \begin{align*}
    \frac{a}{b}
    &= \frac{\sum_{i} a_i x_m^{i}}{\sum_{j} b_j x_m^{j}} \\
    &= \frac{c\sum_{i} a_i x_m^{i}}{c\sum_{j} b_j x_m^{j}} \\
    &= \frac{\sum_{i} ca_i x_m^{i}}{\sum_{j} cb_j x_m^{j}} \\
    &\in R[x_1,\cdots,x_{m-1}](x_m).
  \end{align*}

\item[(2)]
  Hence,
  \begin{align*}
    T
    &= S(w_1,\ldots,w_m) \\
    &= R(v_1,\ldots,v_n)(w_1,\ldots,w_m) \\
    &= R(v_1,\ldots,v_n)(w_1) \cdots (w_m) \\
    &= R(v_1) \cdots (v_n)(w_1) \cdots (w_m) \\
    &= R(v_1,\ldots,v_n,w_1,\ldots,w_m).
  \end{align*}
\end{enumerate}
$\Box$\\\\



%%%%%%%%%%%%%%%%%%%%%%%%%%%%%%%%%%%%%%%%%%%%%%%%%%%%%%%%%%%%%%%%%%%%%%%%%%%%%%%%
%%%%%%%%%%%%%%%%%%%%%%%%%%%%%%%%%%%%%%%%%%%%%%%%%%%%%%%%%%%%%%%%%%%%%%%%%%%%%%%%



\subsection*{1.9. Integral Elements \\}
\addcontentsline{toc}{subsection}{1.9. Integral Elements}



\subsubsection*{Problem 1.46.* (Transitivity of integral extensions)}
\addcontentsline{toc}{subsubsection}{Problem 1.46.* (Transitivity of integral extensions)}
\emph{Let $R$ be a subring of $S$, $S$ a subring of (a domain) $T$.
If $S$ is integral over $R$, and $T$ is integral over $S$,
show that $T$ is integral over $R$.
(Hint: Let $z \in T$, so we have
$z^n + a_1 z^{n-1} + \cdots + a_n = 0$, $a_i \in S$.
Then $R[a_1, \ldots, a_n, z]$ is module-finite over $R$.)} \\

\emph{Proof (Hint).}
\begin{enumerate}
\item[(1)]
  Let $z \in T$, so we have
  $z^n + a_1 z^{n-1} + \cdots + a_n = 0$, $a_i \in S$.
  Therefore, $z$ is integral over $R[a_1, \ldots, a_n]$,
  or $R[a_1, \ldots, a_n, z]$ is module-finite over $R[a_1, \ldots, a_n]$.

\item[(2)]
  \emph{Show that $R[a_1, \ldots, a_n]$ is module-finite over $R$ if all $a_i \in S$.}
  Note that
  \begin{align*}
    & \text{$a_1$ is integral over $R$}, \\
    & \text{$a_2$ is integral over $R[a_1] \supseteq R$}, \\
    & \ldots \\
    & \text{$a_n$ is integral over $R[a_1,\ldots,a_{n-1}]$}.
  \end{align*}
  By Proposition 3,
  \begin{align*}
    & \text{$R[a_1]$ is module-finite over $R$}, \\
    & \text{$R[a_1][a_2]$ is module-finite over $R[a_1]$}, \\
    & \ldots \\
    & \text{$R[a_1,\ldots,a_{n-1}][a_n]$ is module-finite over $R[a_1,\ldots,a_{n-1}]$}.
  \end{align*}
  Also note that $R[a_1,\ldots,a_i] = R[a_1,\ldots,a_{i-1}][a_i]$ if $i > 1$.
  By the transitive relation of the module-finiteness (Problem 1.45),
  $R[a_1, \ldots, a_n]$ is module-finite over $R$.

\item[(3)]
  Again by the transitive relation of the module-finiteness (Problem 1.45),
  $R[a_1, \ldots, a_n, z]$ is module-finite over $R$.
  Hence, $R[a_1, \ldots, a_n, z]$ is a subring of $T$ containing $R[z]$
  which is module-finite over $R$.
  By Proposition 3, $z$ is integral over $R$.
\end{enumerate}
$\Box$\\\\



%%%%%%%%%%%%%%%%%%%%%%%%%%%%%%%%%%%%%%%%%%%%%%%%%%%%%%%%%%%%%%%%%%%%%%%%%%%%%%%%



\subsubsection*{Problem 1.47.*}
\addcontentsline{toc}{subsubsection}{Problem 1.47.*}
\emph{Suppose (a domain) $S$ is ring-finite over $R$.
Show that $S$ is module-finite over $R$ if and only if $S$ is integral over $R$.} \\

\emph{Proof.}
\begin{enumerate}
\item[(1)]
  Write $S = R[v_1,\ldots,v_m]$ for some $v_i \in S$.

\item[(2)]
  Suppose that $S$ is integral over $R$.
  Then all $v_i$ are integral over $R$.
  Use the same argument in Problem 1.46, we have
  \[
    S = R[v_1, \ldots, v_n]
  \]
  is module-finite over $R$.

\item[(3)]
  Conversely, suppose that $S$ is module-finite over $R$.
  Take any $v \in S$.
  Write $v = \sum_{i} r_i v_i \in S$ since $S$ is module-finite over $R$.
  Note that $S = R[v_1,\ldots,v_m]$ is a subring of $S$ itself containing $R[v]$
  which is module-finite over $R$.
  By Proposition 3, $v$ is integral over $R$.
\end{enumerate}
$\Box$\\\\



%%%%%%%%%%%%%%%%%%%%%%%%%%%%%%%%%%%%%%%%%%%%%%%%%%%%%%%%%%%%%%%%%%%%%%%%%%%%%%%%



\subsubsection*{Problem PLACEHOLDER}
\addcontentsline{toc}{subsubsection}{Problem PLACEHOLDER}
\emph{PLACEHOLDER} \\

\emph{Proof.}
\begin{enumerate}
\item[(1)]
  PLACEHOLDER
\end{enumerate}
$\Box$\\\\



%%%%%%%%%%%%%%%%%%%%%%%%%%%%%%%%%%%%%%%%%%%%%%%%%%%%%%%%%%%%%%%%%%%%%%%%%%%%%%%%



\subsubsection*{Problem PLACEHOLDER}
\addcontentsline{toc}{subsubsection}{Problem PLACEHOLDER}
\emph{PLACEHOLDER} \\

\emph{Proof.}
\begin{enumerate}
\item[(1)]
  PLACEHOLDER
\end{enumerate}
$\Box$\\\\



%%%%%%%%%%%%%%%%%%%%%%%%%%%%%%%%%%%%%%%%%%%%%%%%%%%%%%%%%%%%%%%%%%%%%%%%%%%%%%%%



\subsubsection*{Problem PLACEHOLDER}
\addcontentsline{toc}{subsubsection}{Problem PLACEHOLDER}
\emph{PLACEHOLDER} \\

\emph{Proof.}
\begin{enumerate}
\item[(1)]
  PLACEHOLDER
\end{enumerate}
$\Box$\\\\



%%%%%%%%%%%%%%%%%%%%%%%%%%%%%%%%%%%%%%%%%%%%%%%%%%%%%%%%%%%%%%%%%%%%%%%%%%%%%%%%
%%%%%%%%%%%%%%%%%%%%%%%%%%%%%%%%%%%%%%%%%%%%%%%%%%%%%%%%%%%%%%%%%%%%%%%%%%%%%%%%



\subsection*{1.10. Field Extensions \\}
\addcontentsline{toc}{subsection}{1.10. Field Extensions}



\subsubsection*{Problem PLACEHOLDER}
\addcontentsline{toc}{subsubsection}{Problem PLACEHOLDER}
\emph{PLACEHOLDER} \\

\emph{Proof.}
\begin{enumerate}
\item[(1)]
  PLACEHOLDER
\end{enumerate}
$\Box$\\\\



%%%%%%%%%%%%%%%%%%%%%%%%%%%%%%%%%%%%%%%%%%%%%%%%%%%%%%%%%%%%%%%%%%%%%%%%%%%%%%%%



\subsubsection*{Problem PLACEHOLDER}
\addcontentsline{toc}{subsubsection}{Problem PLACEHOLDER}
\emph{PLACEHOLDER} \\

\emph{Proof.}
\begin{enumerate}
\item[(1)]
  PLACEHOLDER
\end{enumerate}
$\Box$\\\\



%%%%%%%%%%%%%%%%%%%%%%%%%%%%%%%%%%%%%%%%%%%%%%%%%%%%%%%%%%%%%%%%%%%%%%%%%%%%%%%%



\subsubsection*{Problem PLACEHOLDER}
\addcontentsline{toc}{subsubsection}{Problem PLACEHOLDER}
\emph{PLACEHOLDER} \\

\emph{Proof.}
\begin{enumerate}
\item[(1)]
  PLACEHOLDER
\end{enumerate}
$\Box$\\\\



%%%%%%%%%%%%%%%%%%%%%%%%%%%%%%%%%%%%%%%%%%%%%%%%%%%%%%%%%%%%%%%%%%%%%%%%%%%%%%%%



\subsubsection*{Problem PLACEHOLDER}
\addcontentsline{toc}{subsubsection}{Problem PLACEHOLDER}
\emph{PLACEHOLDER} \\

\emph{Proof.}
\begin{enumerate}
\item[(1)]
  PLACEHOLDER
\end{enumerate}
$\Box$\\\\



%%%%%%%%%%%%%%%%%%%%%%%%%%%%%%%%%%%%%%%%%%%%%%%%%%%%%%%%%%%%%%%%%%%%%%%%%%%%%%%%
%%%%%%%%%%%%%%%%%%%%%%%%%%%%%%%%%%%%%%%%%%%%%%%%%%%%%%%%%%%%%%%%%%%%%%%%%%%%%%%%
%%%%%%%%%%%%%%%%%%%%%%%%%%%%%%%%%%%%%%%%%%%%%%%%%%%%%%%%%%%%%%%%%%%%%%%%%%%%%%%%
%%%%%%%%%%%%%%%%%%%%%%%%%%%%%%%%%%%%%%%%%%%%%%%%%%%%%%%%%%%%%%%%%%%%%%%%%%%%%%%%



\newpage
\section*{Chapter 2: Affine Varieties \\}
\addcontentsline{toc}{section}{Chapter 2: Affine Varieties}



\subsection*{2.1. Coordinate Rings \\}
\addcontentsline{toc}{subsection}{2.1. Coordinate Rings}



\subsubsection*{Problem 2.1.*}
\addcontentsline{toc}{subsubsection}{Problem 2.1.*}
\emph{Show that the map which associates to each
$f \in k[x_1,\ldots,x_n]$ a polynomial function in $\mathscr{F}(V,k)$
is a ring homomorphism whose kernel is $I(V)$.} \\

\emph{Proof.}
\begin{enumerate}
\item[(1)]
  Define a map $\alpha: k[x_1,\ldots,x_n] \to \mathscr{F}(V,k)$.
  Every polynomial $f \in k[x_1,\ldots,x_n]$ defines a function
  from $V$ to $k$ by
  \[
    \alpha(f)(a_1,\ldots,a_n) = f(a_1,\ldots,a_n)
  \]
  for all $(a_1,\ldots,a_n) \in V$.

\item[(2)]
  $\alpha$ is a ring homomorphism by construction in (1).

\item[(3)]
  \emph{Show that $\mathrm{ker}(\alpha) = I(V)$.}
  In fact,
  given any $f \in k[x_1,\ldots,x_n]$, we have
  $\alpha(f) = 0$ (sending all $a \in V$ to $0 \in k$)
  if and only if $f(a) = 0$ for all $a \in V$
  if and only if $f \in I(V)$.

\item[(4)]
  Hence $k[x_1,\ldots,x_n]/I(V) = \Gamma(V) \hookrightarrow \mathscr{F}(V,k)$
  is an injective homomorphism.
\end{enumerate}
$\Box$\\\\



%%%%%%%%%%%%%%%%%%%%%%%%%%%%%%%%%%%%%%%%%%%%%%%%%%%%%%%%%%%%%%%%%%%%%%%%%%%%%%%%



\subsubsection*{Problem PLACEHOLDER}
\addcontentsline{toc}{subsubsection}{Problem PLACEHOLDER}
\emph{PLACEHOLDER} \\

\emph{Proof.}
\begin{enumerate}
\item[(1)]
  PLACEHOLDER
\end{enumerate}



%%%%%%%%%%%%%%%%%%%%%%%%%%%%%%%%%%%%%%%%%%%%%%%%%%%%%%%%%%%%%%%%%%%%%%%%%%%%%%%%



\subsection*{2.2. Polynomial Maps \\}
\addcontentsline{toc}{subsection}{2.2. Polynomial Maps}



%%%%%%%%%%%%%%%%%%%%%%%%%%%%%%%%%%%%%%%%%%%%%%%%%%%%%%%%%%%%%%%%%%%%%%%%%%%%%%%%



\subsection*{2.3. Coordinate Changes \\}
\addcontentsline{toc}{subsection}{2.3. Coordinate Changes}



%%%%%%%%%%%%%%%%%%%%%%%%%%%%%%%%%%%%%%%%%%%%%%%%%%%%%%%%%%%%%%%%%%%%%%%%%%%%%%%%



\subsection*{2.4. Rational Functions and Local Rings \\}
\addcontentsline{toc}{subsection}{2.4. Rational Functions and Local Rings}



%%%%%%%%%%%%%%%%%%%%%%%%%%%%%%%%%%%%%%%%%%%%%%%%%%%%%%%%%%%%%%%%%%%%%%%%%%%%%%%%



\subsection*{2.5. Discrete Valuation Rings \\}
\addcontentsline{toc}{subsection}{2.5. Discrete Valuation Rings}



%%%%%%%%%%%%%%%%%%%%%%%%%%%%%%%%%%%%%%%%%%%%%%%%%%%%%%%%%%%%%%%%%%%%%%%%%%%%%%%%



\subsection*{2.6. Forms \\}
\addcontentsline{toc}{subsection}{2.6. Forms}



%%%%%%%%%%%%%%%%%%%%%%%%%%%%%%%%%%%%%%%%%%%%%%%%%%%%%%%%%%%%%%%%%%%%%%%%%%%%%%%%



\subsection*{2.7. Direct Products of Rings \\}
\addcontentsline{toc}{subsection}{2.7. Direct Products of Rings}



%%%%%%%%%%%%%%%%%%%%%%%%%%%%%%%%%%%%%%%%%%%%%%%%%%%%%%%%%%%%%%%%%%%%%%%%%%%%%%%%



\subsection*{2.8. Operations with Ideals \\}
\addcontentsline{toc}{subsection}{2.8. Operations with Ideals}



\subsubsection*{Problem 2.39.*}
\addcontentsline{toc}{subsubsection}{Problem 2.39.*}
\emph{Prove the following relations among ideals $I_i$, $J$ in a ring $R$:} \\
\begin{enumerate}
\item[(a)]
  $(I_1 + I_2) J = I_1 J + I_2 J$.

\item[(b)]
  $(I_1 \cdots I_N)^n = I_1^n \cdots I_N^n$. \\
\end{enumerate}



\emph{Proof of (a).}
\begin{enumerate}
\item[(1)]
  Note that $(I_1 + I_2) J$ and $I_1 J + I_2 J$ are ideals.

\item[(2)]
  \emph{Show that $(I_1 + I_2) J \subseteq I_1 J + I_2 J$.}
  Given any
  \[
    (x_{1} + x_{2}) y \in (I_1 + I_2) J
  \]
  where $x_{i} \in I_i$ and $y \in J$.
  It suffices to show that $(x_{1} + x_{2}) y \in I_1 J + I_2 J$ (by (1)).
  In fact,
  \[
    (x_{1} + x_{2}) y = x_{1} y + x_{2} y \in I_1 J + I_2 J.
  \]

\item[(3)]
  \emph{Show that $(I_1 + I_2) J \supseteq I_1 J + I_2 J$.}
  Given any
  \[
    x_{1} y_{1} + x_{2} y_{2} \in I_1 J + I_2 J
  \]
  where $x_{i} \in I_i$ and $y_{i} \in J$.
  It suffices to show that $x_{1} y_{1} + x_{2} y_{2} \in (I_1 + I_2) J$ (by (1)).
  In fact,
  \[
    x_{1} y_{1} + x_{2} y_{2}
    = (x_{1}+\underbrace{0}_{\in I_2}) y_{1} + (\underbrace{0}_{\in I_1}+x_{2}) y_{2}
    \in (I_1 + I_2) J
  \]
  since $(I_1 + I_2) J$ is an ideal.
\end{enumerate}
$\Box$\\



\emph{Proof of (b).}
\begin{enumerate}
\item[(1)]
  Note that $(I_1 \cdots I_N)^n$ and $I_1^n \cdots I_N^n$ are ideals.

\item[(2)]
  \emph{Show that $(I_1 \cdots I_N)^n \subseteq I_1^n \cdots I_N^n$.}
  Given any
  \[
    x = x_1 \cdots x_n
  \]
  where $x_i \in I_1 \cdots I_N$.
  It suffices to show that $x \in I_1^n \cdots I_N^n$ (by (1)).
  For each $x_i \in I_1 \cdots I_N$, write
  \[
    x_i = \sum_{j(i)} x_{j(i),1} \cdots x_{j(i),N}
  \]
  where $x_{j(i),k} \in I_k$ for $1 \leq k \leq N$.
  Hence
  \begin{align*}
    x
    &= x_1 \cdots x_n \\
    &= \left(\sum_{j(1)} x_{j(1),1} \cdots x_{j(1),N}\right)
      \cdots
      \left(\sum_{j(n)} x_{j(n),1} \cdots x_{j(n),N}\right) \\
    &= \sum_{j(1),\ldots,j(n)} (x_{j(1),1} \cdots x_{j(1),N})
      \cdots (x_{j(n),1} \cdots x_{j(n),N}) \\
    &= \sum_{j(1),\ldots,j(n)}
      (\underbrace{x_{j(1),1} \cdots x_{j(n),1}}_{\in I_1^n})
      \cdots
      (\underbrace{x_{j(1),N} \cdots x_{j(n),N}}_{\in I_N^n}) \\
    &\in I_1^n \cdots I_N^n.
  \end{align*}

\item[(3)]
  \emph{Show that $(I_1 \cdots I_N)^n \supseteq I_1^n \cdots I_N^n$.}
  Given any
  \[
    x = x_1 \cdots x_N \in I_1^n \cdots I_N^n
  \]
  where $x_i \in I_i^n$ ($1 \leq i \leq N$).
  It suffices to show that
  $x \in (I_1 \cdots I_N)^n$ (by (1)).
  For each $x_i \in I_i^n$, write
  \[
    x_i = \sum_{j(i)} x_{j(i),1} \cdots x_{j(i),n}
  \]
  where $x_{j(i),k} \in I_i$ for $1 \leq k \leq n$.
  Hence
  \begin{align*}
    x
    &= x_1 \cdots x_N \\
    &= \left(\sum_{j(1)} x_{j(1),1} \cdots x_{j(1),n}\right)
      \cdots
      \left(\sum_{j(N)} x_{j(N),1} \cdots x_{j(N),n}\right) \\
    &= \sum_{j(1),\ldots,j(N)} (x_{j(1),1} \cdots x_{j(1),n})
      \cdots (x_{j(N),1} \cdots x_{j(N),n}) \\
    &= \sum_{j(1),\ldots,j(N)}
      (\underbrace{x_{j(1),1} \cdots x_{j(N),1}}_{\in I_1 \cdots I_N})
      \cdots
      (\underbrace{x_{j(1),n} \cdots x_{j(N),n}}_{\in I_1 \cdots I_N}) \\
    &\in (I_1 \cdots I_N)^n.
  \end{align*}
\end{enumerate}
$\Box$\\\\



%%%%%%%%%%%%%%%%%%%%%%%%%%%%%%%%%%%%%%%%%%%%%%%%%%%%%%%%%%%%%%%%%%%%%%%%%%%%%%%%



\subsubsection*{Problem 2.41.*}
\addcontentsline{toc}{subsubsection}{Problem 2.41.*}
\emph{Let $I$, $J$ be ideals in $R$.
Suppose $I$ is finitely generated and $I \subseteq \mathrm{rad}(J)$.
Show that $I^n \subseteq J$ for some $n$.} \\

\emph{Proof.}
\begin{enumerate}
\item[(1)]
  Let $I$ be generated by $x_1,\ldots,x_m \in I$.
  As $I \subseteq \mathrm{rad}(J)$, there are integers $n_i > 0$
  such that $x_i^{n_i} \in J$.

\item[(2)]
  Let $N = n_1 + \cdots + n_m$.
  Given any $x = \sum_{i=1}^{m} r_i x_i \in I$,
  so
  \begin{align*}
    x^N
    &= \left( \sum_{i=1}^{m} r_i x_i \right)^{N} \\
    &= \sum_{k_1 + \cdots + k_m = N} {N \choose k_1,\ldots,k_m}
      r_1^{k_1} x_1^{k_1} \cdots r_m^{k_m} x_m^{k_m}.
  \end{align*}

\item[(3)]
  Note that for each term there is some $j$ such that $k_j \geq n_j$.
  Hence,
  \begin{align*}
    & \: x_j^{k_j} = x_j^{k_j-n_j} x_j^{n_j} \in J
      &\text{($J$ is an ideal)} \\
    \Longrightarrow& \:
    r_1^{k_1} x_1^{k_1} \cdots r_m^{k_m} x_m^{k_m} \in J \text{ for each term }
      &\text{($J$ is an ideal)} \\
    \Longrightarrow& \:
    x^N \in J.
      &\text{($J$ is an ideal)} \\
    \Longrightarrow& \:
      I^N \subseteq J.
  \end{align*}
\end{enumerate}
$\Box$\\\\



\textbf{Supplement.}
\emph{(Exercise 1.13 in the textbook:
Eisenbud, Commutative Algebra with a View Toward Algebraic Geometry.)}
\emph{Suppose that $I$ is an ideal in a commutative ring.
Show that if $\mathrm{rad}(I)$ is finitely generated,
then for some integer $N$ we have $(\mathrm{rad}(I))^N \subseteq I$.
Conclude that in a Noetherian ring the ideals $I$ and $J$ have the same radical
iff there is some integer $N$ such that $I^N \subseteq J$ and $J^N \subseteq I$.
Use the Nullstellensatz to deduce that if $I, J \subseteq S = k[x_1,\ldots,x_n]$
are ideals and $k$ is algebraically closed,
then $Z(I) = Z(J)$ iff $I^N \subseteq J$ and $J^N \subseteq I$ for some $N$.} \\

\emph{Proof.}
\begin{enumerate}
  \item[(1)]
  \emph{Show that if $\mathrm{rad}(I)$ is finitely generated,
  then for some integer $N$ we have $(\mathrm{rad}(I))^N \subseteq I$.}
  Say $x_1, \ldots, x_m \in \mathrm{rad}(I)$ generate $\mathrm{rad}(I)$.
  \begin{enumerate}
    \item[(a)]
    For each $i$, there exists an integer $n_i > 0$ such that $x_i^{n_i} \in I$
    (since $\mathrm{rad}(I)$ is radical).
    \item[(b)]
    Let $N = n_1 + \cdots + n_m$.
    Given any $x = \sum_{i=1}^{m} r_i x_i \in \mathrm{rad}(I)$,
    so
    \begin{align*}
      x^N
      &= \left( \sum_{i=1}^{m} r_i x_i \right)^{N} \\
      &= \sum_{k_1 + \cdots + k_m = N} {N \choose k_1,\ldots,k_m}
        r_1^{k_1} x_1^{k_1} \cdots r_m^{k_m} x_m^{k_m}.
    \end{align*}
    \item[(c)]
    Note that for each term there is some $j$ such that $k_j \geq n_j$.
    Hence,
    \begin{align*}
      & \: x_j^{k_j} = x_j^{k_j-n_j} x_j^{n_j} \in I
        &\text{($I$ is an ideal)} \\
      \Longrightarrow& \:
      r_1^{k_1} x_1^{k_1} \cdots r_m^{k_m} x_m^{k_m} \in I \text{ for each term }
        &\text{($I$ is an ideal)} \\
      \Longrightarrow& \:
      x^N \in I.
        &\text{($I$ is an ideal)} \\
      \Longrightarrow& \:
        (\mathrm{rad}(I))^N \subseteq I.
    \end{align*}
  \end{enumerate}
  \item[(2)]
  \emph{Show that in a Noetherian ring the ideals $I$ and $J$ have the same radical
  iff there is some integer $N$ such that $I^N \subseteq J$ and $J^N \subseteq I$.}
  \begin{enumerate}
    \item[(a)]
    $(\Longrightarrow)$
    Since in a Noetherian ring every ideal is finitely generated,
    $\mathrm{rad}(I)$ and $\mathrm{rad}(J)$ are finitely generated.
    By (1), there is a common integer $N$ such that
    \[
      (\mathrm{rad}(I))^N \subseteq I \:\: \text{ and } \:\:
      (\mathrm{rad}(J))^N \subseteq J.
    \]
    Note that $I^N \subseteq (\mathrm{rad}(I))^N$ and $J^N \subseteq (\mathrm{rad}(J))^N$.
    Since $\mathrm{rad}(I)$ = $\mathrm{rad}(J)$ by assumption,
    \begin{align*}
      I^N &\subseteq (\mathrm{rad}(I))^N = (\mathrm{rad}(J))^N \subseteq J, \\
      J^N &\subseteq (\mathrm{rad}(J))^N = (\mathrm{rad}(I))^N \subseteq I.
    \end{align*}
    \item[(b)]
    $(\Longleftarrow)$
    It suffices to show that $\mathrm{rad}(I) \subseteq \mathrm{rad}(J)$.
    $\mathrm{rad}(J) \subseteq \mathrm{rad}(I)$ is similar.
    Given any $x \in \mathrm{rad}(I)$, there is an integer $M > 0$ such that
    $x^M \in I$.
    Hence $x^{MN} \in I^N \subseteq J$, or $x \in \mathrm{rad}(J)$.
  \end{enumerate}
  \item[(3)]
  \emph{Show that if $I, J \subseteq S = k[x_1,\ldots,x_n]$
  are ideals and $k$ is algebraically closed,
  then $Z(I) = Z(J)$ iff $I^N \subseteq J$ and $J^N \subseteq I$ for some $N$.}
  Note that $S$ is Noetherian and we can apply part (2).
  By the Nullstellensatz, $Z(I) = Z(J)$ iff $\mathrm{rad}(I) = \mathrm{rad}(J)$
  iff $I^N \subseteq J$ and $J^N \subseteq I$ for some $N$.
\end{enumerate}
$\Box$ \\\\



%%%%%%%%%%%%%%%%%%%%%%%%%%%%%%%%%%%%%%%%%%%%%%%%%%%%%%%%%%%%%%%%%%%%%%%%%%%%%%%%



\subsection*{2.9. Ideals with a Finite Number of Zeros \\}
\addcontentsline{toc}{subsection}{2.9. Ideals with a Finite Number of Zeros}



%%%%%%%%%%%%%%%%%%%%%%%%%%%%%%%%%%%%%%%%%%%%%%%%%%%%%%%%%%%%%%%%%%%%%%%%%%%%%%%%



\subsection*{2.10. Quotient Modules and Exact Sequences \\}
\addcontentsline{toc}{subsection}{2.10. Quotient Modules and Exact Sequences}



\subsubsection*{Problem 2.51.}
\addcontentsline{toc}{subsubsection}{Problem 2.51.}
\emph{Let
\[
  0 \longrightarrow V_1 \longrightarrow \cdots \longrightarrow V_n \longrightarrow 0
\]
be an exact sequence of finite-dimensional vector spaces.
Show that $\sum (-1)^i \dim(V_i) = 0$.} \\

\emph{Proof (Proposition 7 in \S 2.10).}
\begin{enumerate}
\item[(1)]
  For $i = 0,\ldots,n$,
  by the rank-nullity theorem for a linear transformation
  $\varphi_{i}: V_{i} \to V_{i+1}$, we have
  \[
    \dim V_{i} = \dim \mathrm{im}(\varphi_{i}) + \dim \mathrm{ker}(\varphi_{i}).
  \]
  (Here $V_0 = V_{n+1} := 0$ by convention.)

\item[(2)]
  By the exactness of the sequence, we have
  \begin{enumerate}
  \item[(a)]
    $\mathrm{im}(\varphi_{i}) = \mathrm{ker}(\varphi_{i+1})$ for $i = 0,\ldots, n-1$.
    In particular, $\mathrm{ker}(\varphi_{1}) = \mathrm{im}(\varphi_{0}) = 0$.

  \item[(b)]
    $\mathrm{ker}(\varphi_{n}) = V_n$.
  \end{enumerate}
  Hence,
  \begin{align*}
    \sum_{i=1}^{n-1} (-1)^i \dim(V_i)
    &= \sum_{i=1}^{n-1} (-1)^i \dim \mathrm{im}(\varphi_{i})
      + \sum_{i=1}^{n-1} (-1)^i \dim \mathrm{ker}(\varphi_{i}) \\
    &= \sum_{i=1}^{n-1} (-1)^i \dim \mathrm{ker}(\varphi_{i+1})
      + \sum_{i=1}^{n-1} (-1)^i \dim \mathrm{ker}(\varphi_{i}) \\
    &= (-1)^{n-1} \dim \underbrace{\mathrm{ker}(\varphi_{n})}_{=V_n}
      + (-1)^1 \dim \underbrace{\mathrm{ker}(\varphi_{1})}_{= 0} \\
    &= -(-1)^n \dim V_n,
  \end{align*}
  or $\sum (-1)^i \dim(V_i) = 0$.
\end{enumerate}
$\Box$\\\\



%%%%%%%%%%%%%%%%%%%%%%%%%%%%%%%%%%%%%%%%%%%%%%%%%%%%%%%%%%%%%%%%%%%%%%%%%%%%%%%%



\subsection*{2.11. Free Modules \\}
\addcontentsline{toc}{subsection}{2.11. Free Modules}



%%%%%%%%%%%%%%%%%%%%%%%%%%%%%%%%%%%%%%%%%%%%%%%%%%%%%%%%%%%%%%%%%%%%%%%%%%%%%%%%
%%%%%%%%%%%%%%%%%%%%%%%%%%%%%%%%%%%%%%%%%%%%%%%%%%%%%%%%%%%%%%%%%%%%%%%%%%%%%%%%
%%%%%%%%%%%%%%%%%%%%%%%%%%%%%%%%%%%%%%%%%%%%%%%%%%%%%%%%%%%%%%%%%%%%%%%%%%%%%%%%
%%%%%%%%%%%%%%%%%%%%%%%%%%%%%%%%%%%%%%%%%%%%%%%%%%%%%%%%%%%%%%%%%%%%%%%%%%%%%%%%



\newpage
\section*{Chapter 3: Local Properties of Plane Curves \\}
\addcontentsline{toc}{section}{Chapter 3: Local Properties of Plane Curves}



\subsection*{3.1. Multiple Points and Tangent Lines \\}
\addcontentsline{toc}{subsection}{3.1. Multiple Points and Tangent Lines}



\subsubsection*{Problem PLACEHOLDER}
\addcontentsline{toc}{subsubsection}{Problem PLACEHOLDER}
\emph{PLACEHOLDER} \\

\emph{Proof.}
\begin{enumerate}
\item[(1)]
  PLACEHOLDER
\end{enumerate}
$\Box$\\\\



%%%%%%%%%%%%%%%%%%%%%%%%%%%%%%%%%%%%%%%%%%%%%%%%%%%%%%%%%%%%%%%%%%%%%%%%%%%%%%%%
%%%%%%%%%%%%%%%%%%%%%%%%%%%%%%%%%%%%%%%%%%%%%%%%%%%%%%%%%%%%%%%%%%%%%%%%%%%%%%%%



\subsection*{3.2. Multiplicities and Local Rings \\}
\addcontentsline{toc}{subsection}{3.2. Multiplicities and Local Rings}



%%%%%%%%%%%%%%%%%%%%%%%%%%%%%%%%%%%%%%%%%%%%%%%%%%%%%%%%%%%%%%%%%%%%%%%%%%%%%%%%
%%%%%%%%%%%%%%%%%%%%%%%%%%%%%%%%%%%%%%%%%%%%%%%%%%%%%%%%%%%%%%%%%%%%%%%%%%%%%%%%



\subsection*{3.3. Intersection Numbers \\}
\addcontentsline{toc}{subsection}{3.3. Intersection Numbers}



%%%%%%%%%%%%%%%%%%%%%%%%%%%%%%%%%%%%%%%%%%%%%%%%%%%%%%%%%%%%%%%%%%%%%%%%%%%%%%%%
%%%%%%%%%%%%%%%%%%%%%%%%%%%%%%%%%%%%%%%%%%%%%%%%%%%%%%%%%%%%%%%%%%%%%%%%%%%%%%%%
%%%%%%%%%%%%%%%%%%%%%%%%%%%%%%%%%%%%%%%%%%%%%%%%%%%%%%%%%%%%%%%%%%%%%%%%%%%%%%%%
%%%%%%%%%%%%%%%%%%%%%%%%%%%%%%%%%%%%%%%%%%%%%%%%%%%%%%%%%%%%%%%%%%%%%%%%%%%%%%%%



\newpage
\section*{Chapter 4: Projective Varieties \\}
\addcontentsline{toc}{section}{Chapter 4: Projective Varieties}



\subsection*{4.1. Projective Space \\}
\addcontentsline{toc}{subsection}{4.1. Projective Space}



\subsubsection*{Problem PLACEHOLDER}
\addcontentsline{toc}{subsubsection}{Problem PLACEHOLDER}
\emph{PLACEHOLDER} \\

\emph{Proof.}
\begin{enumerate}
\item[(1)]
  PLACEHOLDER
\end{enumerate}
$\Box$\\\\



%%%%%%%%%%%%%%%%%%%%%%%%%%%%%%%%%%%%%%%%%%%%%%%%%%%%%%%%%%%%%%%%%%%%%%%%%%%%%%%%



\subsection*{4.2. Projective Algebraic Sets \\}
\addcontentsline{toc}{subsection}{4.2. Projective Algebraic Sets}



%%%%%%%%%%%%%%%%%%%%%%%%%%%%%%%%%%%%%%%%%%%%%%%%%%%%%%%%%%%%%%%%%%%%%%%%%%%%%%%%



\subsection*{4.3. Affine and Projective Varieties \\}
\addcontentsline{toc}{subsection}{4.3. Affine and Projective Varieties}



%%%%%%%%%%%%%%%%%%%%%%%%%%%%%%%%%%%%%%%%%%%%%%%%%%%%%%%%%%%%%%%%%%%%%%%%%%%%%%%%



\subsection*{4.4. Multiprojective Space \\}
\addcontentsline{toc}{subsection}{4.4. Multiprojective Space}



%%%%%%%%%%%%%%%%%%%%%%%%%%%%%%%%%%%%%%%%%%%%%%%%%%%%%%%%%%%%%%%%%%%%%%%%%%%%%%%%
%%%%%%%%%%%%%%%%%%%%%%%%%%%%%%%%%%%%%%%%%%%%%%%%%%%%%%%%%%%%%%%%%%%%%%%%%%%%%%%%
%%%%%%%%%%%%%%%%%%%%%%%%%%%%%%%%%%%%%%%%%%%%%%%%%%%%%%%%%%%%%%%%%%%%%%%%%%%%%%%%
%%%%%%%%%%%%%%%%%%%%%%%%%%%%%%%%%%%%%%%%%%%%%%%%%%%%%%%%%%%%%%%%%%%%%%%%%%%%%%%%



\newpage
\section*{Chapter 5: Projective Plane Curves\\}
\addcontentsline{toc}{section}{Chapter 5: Projective Plane Curves}



\subsection*{5.1. Definitions \\}
\addcontentsline{toc}{subsection}{5.1. Definitions}



\subsubsection*{Problem PLACEHOLDER}
\addcontentsline{toc}{subsubsection}{Problem PLACEHOLDER}
\emph{PLACEHOLDER} \\

\emph{Proof.}
\begin{enumerate}
\item[(1)]
  PLACEHOLDER
\end{enumerate}
$\Box$\\\\



%%%%%%%%%%%%%%%%%%%%%%%%%%%%%%%%%%%%%%%%%%%%%%%%%%%%%%%%%%%%%%%%%%%%%%%%%%%%%%%%



\subsection*{5.2. Linear Systems of Curves \\}
\addcontentsline{toc}{subsection}{5.2. Linear Systems of Curves}



%%%%%%%%%%%%%%%%%%%%%%%%%%%%%%%%%%%%%%%%%%%%%%%%%%%%%%%%%%%%%%%%%%%%%%%%%%%%%%%%



\subsection*{5.3. B\'ezout's Theorem  \\}
\addcontentsline{toc}{subsection}{5.3. B\'ezout's Theorem}



%%%%%%%%%%%%%%%%%%%%%%%%%%%%%%%%%%%%%%%%%%%%%%%%%%%%%%%%%%%%%%%%%%%%%%%%%%%%%%%%



\subsection*{5.4. Multiple Points \\}
\addcontentsline{toc}{subsection}{5.4. Multiple Points}



%%%%%%%%%%%%%%%%%%%%%%%%%%%%%%%%%%%%%%%%%%%%%%%%%%%%%%%%%%%%%%%%%%%%%%%%%%%%%%%%



\subsection*{5.5. Max Noether's Fundamental Theorem \\}
\addcontentsline{toc}{subsection}{5.5. Max Noether's Fundamental Theorem}



%%%%%%%%%%%%%%%%%%%%%%%%%%%%%%%%%%%%%%%%%%%%%%%%%%%%%%%%%%%%%%%%%%%%%%%%%%%%%%%%



\subsection*{5.6. Applications of Noether's Theorem \\}
\addcontentsline{toc}{subsection}{5.6. Applications of Noether's Theorem}



%%%%%%%%%%%%%%%%%%%%%%%%%%%%%%%%%%%%%%%%%%%%%%%%%%%%%%%%%%%%%%%%%%%%%%%%%%%%%%%%
%%%%%%%%%%%%%%%%%%%%%%%%%%%%%%%%%%%%%%%%%%%%%%%%%%%%%%%%%%%%%%%%%%%%%%%%%%%%%%%%
%%%%%%%%%%%%%%%%%%%%%%%%%%%%%%%%%%%%%%%%%%%%%%%%%%%%%%%%%%%%%%%%%%%%%%%%%%%%%%%%
%%%%%%%%%%%%%%%%%%%%%%%%%%%%%%%%%%%%%%%%%%%%%%%%%%%%%%%%%%%%%%%%%%%%%%%%%%%%%%%%



\newpage
\section*{Chapter 6: Varieties, Morphisms, and Rational Maps \\}
\addcontentsline{toc}{section}{Chapter 6: Varieties, Morphisms, and Rational Maps}

\subsection*{6.1. The Zariski Topology \\}
\addcontentsline{toc}{subsection}{6.1. The Zariski Topology}

\subsection*{6.2. Varieties \\}
\addcontentsline{toc}{subsection}{6.2. Varieties}

\subsection*{6.3. Morphisms of Varieties \\}
\addcontentsline{toc}{subsection}{6.3. Morphisms of Varieties}

\subsection*{6.4. Products and Graphs \\}
\addcontentsline{toc}{subsection}{6.4. Products and Graphs}

\subsection*{6.5. Algebraic Function Fields and Dimension of Varieties \\}
\addcontentsline{toc}{subsection}{6.5. Algebraic Function Fields and Dimension of Varieties}

\subsection*{6.6. Rational Maps \\}
\addcontentsline{toc}{subsection}{6.6. Rational Maps}



%%%%%%%%%%%%%%%%%%%%%%%%%%%%%%%%%%%%%%%%%%%%%%%%%%%%%%%%%%%%%%%%%%%%%%%%%%%%%%%%
%%%%%%%%%%%%%%%%%%%%%%%%%%%%%%%%%%%%%%%%%%%%%%%%%%%%%%%%%%%%%%%%%%%%%%%%%%%%%%%%
%%%%%%%%%%%%%%%%%%%%%%%%%%%%%%%%%%%%%%%%%%%%%%%%%%%%%%%%%%%%%%%%%%%%%%%%%%%%%%%%
%%%%%%%%%%%%%%%%%%%%%%%%%%%%%%%%%%%%%%%%%%%%%%%%%%%%%%%%%%%%%%%%%%%%%%%%%%%%%%%%



\newpage
\section*{Chapter 7: Resolution of Singularities \\}
\addcontentsline{toc}{section}{Chapter 7: Resolution of Singularities}



\subsection*{7.1. Rational Maps of Curves \\}
\addcontentsline{toc}{subsection}{7.1. Rational Maps of Curves}



\subsubsection*{Problem PLACEHOLDER}
\addcontentsline{toc}{subsubsection}{Problem PLACEHOLDER}
\emph{PLACEHOLDER} \\

\emph{Proof.}
\begin{enumerate}
\item[(1)]
  PLACEHOLDER
\end{enumerate}
$\Box$\\\\



%%%%%%%%%%%%%%%%%%%%%%%%%%%%%%%%%%%%%%%%%%%%%%%%%%%%%%%%%%%%%%%%%%%%%%%%%%%%%%%%



\subsection*{7.2. Blowing up a Point in $\mathbf{A}^{2}$ \\}
\addcontentsline{toc}{subsection}{7.2. Blowing up a Point in $\mathbf{A}^{2}$}



%%%%%%%%%%%%%%%%%%%%%%%%%%%%%%%%%%%%%%%%%%%%%%%%%%%%%%%%%%%%%%%%%%%%%%%%%%%%%%%%



\subsection*{7.3. Blowing up a Point in $\mathbf{P}^{2}$ \\}
\addcontentsline{toc}{subsection}{7.3. Blowing up a Point in $\mathbf{P}^{2}$}



%%%%%%%%%%%%%%%%%%%%%%%%%%%%%%%%%%%%%%%%%%%%%%%%%%%%%%%%%%%%%%%%%%%%%%%%%%%%%%%%



\subsection*{7.4. Quadratic Transformations \\}
\addcontentsline{toc}{subsection}{7.4. Quadratic Transformations}



%%%%%%%%%%%%%%%%%%%%%%%%%%%%%%%%%%%%%%%%%%%%%%%%%%%%%%%%%%%%%%%%%%%%%%%%%%%%%%%%



\subsection*{7.5. Nonsingular Models of Curves \\}
\addcontentsline{toc}{subsection}{7.5. Nonsingular Models of Curves}



%%%%%%%%%%%%%%%%%%%%%%%%%%%%%%%%%%%%%%%%%%%%%%%%%%%%%%%%%%%%%%%%%%%%%%%%%%%%%%%%
%%%%%%%%%%%%%%%%%%%%%%%%%%%%%%%%%%%%%%%%%%%%%%%%%%%%%%%%%%%%%%%%%%%%%%%%%%%%%%%%
%%%%%%%%%%%%%%%%%%%%%%%%%%%%%%%%%%%%%%%%%%%%%%%%%%%%%%%%%%%%%%%%%%%%%%%%%%%%%%%%
%%%%%%%%%%%%%%%%%%%%%%%%%%%%%%%%%%%%%%%%%%%%%%%%%%%%%%%%%%%%%%%%%%%%%%%%%%%%%%%%



\newpage
\section*{Chapter 8: Riemann-Roch Theorem \\}
\addcontentsline{toc}{section}{Chapter 8: Riemann-Roch Theorem}




\subsection*{8.1. Divisors \\}
\addcontentsline{toc}{subsection}{8.1. Divisors}



\subsubsection*{Problem PLACEHOLDER}
\addcontentsline{toc}{subsubsection}{Problem PLACEHOLDER}
\emph{PLACEHOLDER} \\

\emph{Proof.}
\begin{enumerate}
\item[(1)]
  PLACEHOLDER
\end{enumerate}
$\Box$\\\\



%%%%%%%%%%%%%%%%%%%%%%%%%%%%%%%%%%%%%%%%%%%%%%%%%%%%%%%%%%%%%%%%%%%%%%%%%%%%%%%%



\subsection*{8.2. The Vector Spaces $L(D)$ \\}
\addcontentsline{toc}{subsection}{8.2. The Vector Spaces $L(D)$}



%%%%%%%%%%%%%%%%%%%%%%%%%%%%%%%%%%%%%%%%%%%%%%%%%%%%%%%%%%%%%%%%%%%%%%%%%%%%%%%%



\subsection*{8.3. Riemann's Theorem \\}
\addcontentsline{toc}{subsection}{8.3. Riemann's Theorem}



%%%%%%%%%%%%%%%%%%%%%%%%%%%%%%%%%%%%%%%%%%%%%%%%%%%%%%%%%%%%%%%%%%%%%%%%%%%%%%%%



\subsection*{8.4. Derivations and Differentials \\}
\addcontentsline{toc}{subsection}{8.4. Derivations and Differentials}



%%%%%%%%%%%%%%%%%%%%%%%%%%%%%%%%%%%%%%%%%%%%%%%%%%%%%%%%%%%%%%%%%%%%%%%%%%%%%%%%



\subsection*{8.5. Canonical Divisors \\}
\addcontentsline{toc}{subsection}{8.5. Canonical Divisors}



%%%%%%%%%%%%%%%%%%%%%%%%%%%%%%%%%%%%%%%%%%%%%%%%%%%%%%%%%%%%%%%%%%%%%%%%%%%%%%%%



\subsection*{8.6. Riemann-Roch Theorem \\}
\addcontentsline{toc}{subsection}{8.6. Riemann-Roch Theorem}



%%%%%%%%%%%%%%%%%%%%%%%%%%%%%%%%%%%%%%%%%%%%%%%%%%%%%%%%%%%%%%%%%%%%%%%%%%%%%%%%
%%%%%%%%%%%%%%%%%%%%%%%%%%%%%%%%%%%%%%%%%%%%%%%%%%%%%%%%%%%%%%%%%%%%%%%%%%%%%%%%



\end{document}