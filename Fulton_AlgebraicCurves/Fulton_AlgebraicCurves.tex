\documentclass{article}
\usepackage{amsfonts}
\usepackage{amsmath}
\usepackage{amssymb}
\usepackage{hyperref}
\usepackage[none]{hyphenat}
\usepackage{mathrsfs}
\parindent=0pt



\title{\textbf{Solutions to the book: \\\emph{Fulton, Algebraic Curves}}}
\author{Meng-Gen Tsai \\ plover@gmail.com}



\begin{document}
\maketitle
\tableofcontents



%%%%%%%%%%%%%%%%%%%%%%%%%%%%%%%%%%%%%%%%%%%%%%%%%%%%%%%%%%%%%%%%%%%%%%%%%%%%%%%%
%%%%%%%%%%%%%%%%%%%%%%%%%%%%%%%%%%%%%%%%%%%%%%%%%%%%%%%%%%%%%%%%%%%%%%%%%%%%%%%%



\newpage
\section*{Chapter 1: Affine Algebraic Sets \\}
\addcontentsline{toc}{section}{Chapter 1: Affine Algebraic Sets}



%%%%%%%%%%%%%%%%%%%%%%%%%%%%%%%%%%%%%%%%%%%%%%%%%%%%%%%%%%%%%%%%%%%%%%%%%%%%%%%%



\subsection*{1.1. Algebraic Preliminaries \\}
\addcontentsline{toc}{subsection}{1.1. Algebraic Preliminaries}



\subsubsection*{Problem 1.1.*}
\addcontentsline{toc}{subsubsection}{Problem 1.1.*}
\emph{Let $R$ be a domain.}
\begin{enumerate}
\item[(a)]
  \emph{If $f$, $g$ are forms of degree $r$, $s$ respectively in $R[x_1,\ldots,x_n]$,
  show that $fg$ is a form of degree $r+s$.}

\item[(b)]
  \emph{Show that any factor of a form in $R[x_1,\ldots,x_n]$ is also a form. } \\
\end{enumerate}

\emph{Proof of (a).}
\begin{enumerate}
\item[(1)]
  Write
  \begin{align*}
    f &= \sum_{(i)} a_{(i)} x^{(i)}, \\
    g &= \sum_{(j)} b_{(j)} x^{(j)},
  \end{align*}
  where $\sum_{(i)}$ is the summation over $(i) = (i_1,\ldots,i_n)$ with $i_1+\cdots+i_n = r$
  and $\sum_{(j)}$ is the summation over $(j) = (j_1,\ldots,j_n)$ with $j_1+\cdots+j_n = s$.

\item[(2)]
  Hence,
  \begin{align*}
    fg
    &= \sum_{(i)} \sum_{(j)} a_{(i)}b_{(j)} x^{(i)}x^{(j)} \\
    &= \sum_{(i),(j)} a_{(i)}b_{(j)} x^{(k)}
  \end{align*}
  where $(k) = (i_1+j_1,\ldots,i_n+j_n)$ with $(i_1+j_1)+\cdots+(i_n+j_n) = r+s$.
  Each $x^{(k)}$ is the form of degree $r+s$ and $a_{(i)}b_{(j)} \in R$.
  Hence $fg$ is a form of degree $r+s$.
\end{enumerate}
$\Box$\\



\emph{Proof of (b).}
\begin{enumerate}
\item[(1)]
  Given any form $f \in R[x_1,\ldots,x_n]$, and write $f = gh$.
  \emph{It suffices to show that $g$ is a form as well.}
  (So does $h$.)

\item[(2)]
  Write
  \[
    g = g_0+\cdots+g_r,
    \qquad
    h = h_0+\cdots+h_s
  \]
  where $g_r \neq 0$ and $h_s \neq 0$.
  So
  \[
    f = gh = g_0h_0 + \cdots + g_r h_s.
  \]
  Since $R$ is a domain, $R[x_1,\ldots,x_n]$ is a domain and thus $g_r h_s \neq 0$.
  The maximality of $r$ and $s$ implies that $\deg f = r+s$.
  Therefore, by the maximality of $r+s$,
  $f = g_r h_s$, or $g = g_r$, or $g$ is a form.
\end{enumerate}
$\Box$\\\\



%%%%%%%%%%%%%%%%%%%%%%%%%%%%%%%%%%%%%%%%%%%%%%%%%%%%%%%%%%%%%%%%%%%%%%%%%%%%%%%%



\subsubsection*{Problem 1.5.*}
\addcontentsline{toc}{subsubsection}{Problem 1.5.*}
\emph{Let $k$ be any field.
Show that there are an infinitely number of irreducible monic polynomials in $k[x]$.
(Hint: Suppose $f_1,\ldots,f_n$ were all of them, and factor $f_1\cdots f_n+1$ into irreducible factors.)} \\

\emph{Proof (Due to Euclid).}
\begin{enumerate}
\item[(1)]
  If
  $f_1, \ldots, f_n$ were all irreducible monic polynomials, then
  we consider
  \[
    g = f_1 \cdots f_n + 1 \in k[x].
  \]
  So there is an irreducible monic polynomial $f = f_i$ dividing $g$ for some $i$
  since
  \[
    \deg g = \deg f_1 + \cdots + \deg f_n \geq 1.
  \]

\item[(2)]
  However, $f$ would divide the difference
  \[
    g - f_1 \cdots f_{i-1} f_i f_{i+1} \cdots f_n = 1,
  \]
  contrary to $\deg f_i \geq 1$.
\end{enumerate}
$\Box$\\\\



%%%%%%%%%%%%%%%%%%%%%%%%%%%%%%%%%%%%%%%%%%%%%%%%%%%%%%%%%%%%%%%%%%%%%%%%%%%%%%%%



\subsubsection*{Problem 1.6.*}
\addcontentsline{toc}{subsubsection}{Problem 1.6.*}
\emph{Show that any algebraically closed field is infinite.
(Hint: The irreducible monic polynomials are $x - a$, $a \in k$.)} \\

\emph{Proof (Due to Euclid).}
\begin{enumerate}
\item[(1)]
  Let $k$ be an algebraically closed field.
  If $a_1, \ldots, a_n$ were all elements in $k$, then
  we consider a monic polynomials
  \[
    f(x) = (x - a_1) \cdots (x - a_n) + 1 \in k[x].
  \]

\item[(2)]
  Since $k$ is algebraically closed,
  there is an element $a \in k$ such that $f(a) = 0$.
  By assumption, $a = a_i$ for some $1 \leq i \leq n$,
  and thus $f(a) = f(a_i) = 1$, contrary to the fact that
  a field is a commutative ring where $0 \neq 1$ and all nonzero elements are invertible.
\end{enumerate}
$\Box$\\\\



%%%%%%%%%%%%%%%%%%%%%%%%%%%%%%%%%%%%%%%%%%%%%%%%%%%%%%%%%%%%%%%%%%%%%%%%%%%%%%%%
%%%%%%%%%%%%%%%%%%%%%%%%%%%%%%%%%%%%%%%%%%%%%%%%%%%%%%%%%%%%%%%%%%%%%%%%%%%%%%%%



\subsection*{1.2. Affine Space and Algebraic Sets \\}
\addcontentsline{toc}{subsection}{1.2. Affine Space and Algebraic Sets}



%%%%%%%%%%%%%%%%%%%%%%%%%%%%%%%%%%%%%%%%%%%%%%%%%%%%%%%%%%%%%%%%%%%%%%%%%%%%%%%%



\subsubsection*{Problem 1.8.*}
\addcontentsline{toc}{subsubsection}{Problem 1.8.*}
\emph{Show that the algebraic subsets of $\mathbf{A}^1(k)$ are just the finite subsets, together
with $\mathbf{A}^1(k)$ itself.} \\

\emph{Proof.}
\begin{enumerate}
\item[(1)]
  \emph{Show that $k[x]$ is a PID if $k$ is a field.}
  \begin{enumerate}
  \item[(a)]
    Let $I$ be an ideal of $k[x]$.

  \item[(b)]
    If $I = \{0\}$ then $I = (0)$ and $I$ is principal.

  \item[(c)]
    If $I \neq \{0\}$, then take $f$ to be a polynomial of minimal degree in $I$.
    It suffices to show that $I = (f)$.
    Clearly, $(f) \subseteq I$ since $I$ is an ideal.
    Conversely, for any $g \in I$,
    \[
      g(x) = f(x)h(x) + r(x)
    \]
    for some $h, r \in k[x]$ with $r = 0$ or $\deg r < \deg f$.
    Now as
    \[
      r = g - fh \in I,
    \]
    $r = 0$ (otherwise contrary to the minimality of $f$),
    we have $g = fh \in (f)$ for all $g \in I$.
  \end{enumerate}

\item[(2)]
  Let $Y$ be an algebraic subset of $\mathbf{A}^1(k)$,
  say $Y = V(I)$ for some ideal $I$ of $k[x]$.
  Since $k[x]$ is a PID, $I = (f)$ for some $f \in k[x]$.
  \begin{enumerate}
  \item[(a)]
    If $f = 0$, then $I = (0)$ and $Y = V(0) = \mathbf{A}^1(k)$.

  \item[(b)]
    If $f \neq 0$, then $f(x) = 0$ has finitely many roots in $k$,
    say $a_1, \ldots, a_m \in k$.
    Hence,
    \[
      Y = V(I) = V(f) = \{ f(a) = 0 : a \in k \}
      = \{ a_1, \ldots, a_m \}
    \]
    is a finite subsets of $\mathbf{A}^1(k)$.
  \end{enumerate}
  By (a)(b), the result is established.
\end{enumerate}
$\Box$\\



\emph{Notes.}
\begin{enumerate}
\item[(1)]
  By the Hilbert basis theorem, $k[x]$ is Noetherian as $k$ is Noetherian.
  Hence, for any algebraic subset $Y = V(I)$ of $\mathbf{A}^1(k)$,
  we can write $I = (f_1, \cdots, f_m)$.
  Note that
  \[
    Y = V(I) = V(f_1) \cap \cdots \cap V(f_m).
  \]
  Now apply the same argument to get the same conclusion.

\item[(2)]
  Suppose $k = \overline{k}$.
  $\mathbf{A}^1(k)$ is irreducible, because its only proper closed subsets are finite,
  yet it is infinite
  (because $k$ is algebraically closed, hence infinite). \\
\end{enumerate}



%%%%%%%%%%%%%%%%%%%%%%%%%%%%%%%%%%%%%%%%%%%%%%%%%%%%%%%%%%%%%%%%%%%%%%%%%%%%%%%%



\subsubsection*{Problem 1.9.}
\addcontentsline{toc}{subsubsection}{Problem 1.9.}
\emph{If $k$ is a finite field, show that every subset of $\mathbf{A}^{n}(k)$ is algebraic.} \\

\emph{Proof.}
\begin{enumerate}
\item[(1)]
  Every subset of $\mathbf{A}^{n}(k)$ is finite since
  $|\mathbf{A}^{n}(k)| = |k|^n$ is finite.

\item[(2)]
  Note that $V(x_1-a_1,\ldots,x_n-a_n) = \{ (a_1,\ldots,a_n) \} \subseteq \mathbf{A}^{n}(k)$
  (property (5) in this section)
  and any finite union of algebraic sets is algebraic (property (4) in this section).
  Thus, every subset of $\mathbf{A}^{n}(k)$ is algebraic (by (1)).
\end{enumerate}
$\Box$\\\\



%%%%%%%%%%%%%%%%%%%%%%%%%%%%%%%%%%%%%%%%%%%%%%%%%%%%%%%%%%%%%%%%%%%%%%%%%%%%%%%%



\subsubsection*{Problem 1.11.}
\addcontentsline{toc}{subsubsection}{Problem 1.11.}
\emph{Show that the following are algebraic sets:}
\begin{enumerate}
\item[(a)]
  $\{ (t,t^2,t^3) \in \mathbf{A}^{3}(k) : t \in k \}$;

\item[(b)]
  $\{ (\cos(t),\sin(t)) \in \mathbf{A}^{2}(\mathbb{R}) : t \in \mathbb{R} \}$;

\item[(c)]
  \emph{the set of points in $\mathbf{A}^{2}(\mathbb{R})$
  whose polar coordinates $(r,\theta)$ satisfy the equation $r = \sin(\theta)$.} \\
\end{enumerate}



\emph{Proof of (a).}
\begin{enumerate}
\item[(1)]
  The twisted cubic curve
  \[
    Y = \{ (t,t^2,t^3) \in \mathbf{A}^3(k) : t \in k \}
    =
    V(x^2-y) \cap V(x^3-z)
  \]
  is algebraic.
  We say that $Y$ is given by the parametric representation $x=t$, $y=t^2$, $z=t^3$.

\item[(2)]
  The generators for the ideal $I(Y)$ are $x^2-y$ and $x^3-z$.

\item[(3)]
  $Y$ is an affine variety of dimension $1$.

\item[(4)]
  The affine coordinate ring $A(Y)$ is isomorphic to a polynomial ring in one variable over $k$.
\end{enumerate}
$\Box$\\



\emph{Proof of (b).}
The circle
\[
  \{(\cos(t),\sin(t)) \in \mathbf{A}^2(\mathbb{R}) : t \in \mathbb{R} \} = V(x^2-y^2-1)
\]
is algebraic.
$\Box$\\



\emph{Proof of (c).}
The circle
\[
  \{ (r,\theta) : r = \sin(\theta) \} = V(x^2+y^2-y)
\]
is algebraic again.
$\Box$\\\\



%%%%%%%%%%%%%%%%%%%%%%%%%%%%%%%%%%%%%%%%%%%%%%%%%%%%%%%%%%%%%%%%%%%%%%%%%%%%%%%%



\subsubsection*{Problem 1.15.*}
\addcontentsline{toc}{subsubsection}{Problem 1.15.*}
\emph{Let $V \subseteq \mathbf{A}^n(k)$, $W \subseteq \mathbf{A}^m(k)$ be algebraic sets.
Show that
\[
  V \times W
  = \{(a_1,\ldots,a_n,b_1,\ldots,b_m) : (a_1,\ldots,a_n) \in V, (b_1,\ldots,b_m) \in W \}
\]
is an algebraic set in $\mathbf{A}^{n+m}(k)$.
It is called the \textbf{product} of $V$ and $W$.} \\

\emph{Proof.}
\begin{enumerate}
\item[(1)]
  Write
  \begin{align*}
    V &= V(S_V) = \{ a \in \mathbf{A}^n(k) : f(a) = 0 \: \forall f \in S_V \} \\
    W &= V(S_W) = \{ b \in \mathbf{A}^m(k) : g(b) = 0 \: \forall g \in S_W \},
  \end{align*}
  where $S_V \subseteq k[x_1,\ldots,x_n]$ and $S_W \subseteq k[y_1,\ldots,y_m]$.
  It suffices to show that
  \[
    V \times W = V(S),
  \]
  where
  $S \subseteq k[x_1,\ldots,x_n,y_1,\ldots,y_m]$ is the union of $S_V$ and $S_W$.

\item[(2)]
  Here we can identify $S_V$ with the subset of
  $k[x_1,\ldots,x_n,y_1,\ldots,y_m]$
  by noting that
  \[
    k[x_1,\ldots,x_n]
    \hookrightarrow (k[y_1,\ldots,y_m])[x_1,\ldots,x_n]
    = k[x_1,\ldots,x_n,y_1,\ldots,y_m].
  \]
  Here we regard $k$ as a subring of $k[y_1,\ldots,y_m]$.
  Similar treatment to $S_W$.

\item[(3)]
  By construction, $V \times W \subseteq V(S)$.
  Conversely, given any $(a,b) \in V(S)$, we have $h(a,b) = 0$ for all $h \in S = S_V \cup S_W$ (by (2)).
  By construction, $f(a) = 0$ for all $f \in S_V$ since $f$ only involve $x_1,\ldots,x_n$.
  Hence, $a \in V$. Similarly, $b \in W$. Therefore, $(a,b) \in V \times W$.
\end{enumerate}
$\Box$\\\\



%%%%%%%%%%%%%%%%%%%%%%%%%%%%%%%%%%%%%%%%%%%%%%%%%%%%%%%%%%%%%%%%%%%%%%%%%%%%%%%%
%%%%%%%%%%%%%%%%%%%%%%%%%%%%%%%%%%%%%%%%%%%%%%%%%%%%%%%%%%%%%%%%%%%%%%%%%%%%%%%%



\subsection*{1.3. The Ideal of a Set of Points \\}
\addcontentsline{toc}{subsection}{1.3. The Ideal of a Set of Points}



\subsubsection*{Problem 1.18.*}
\addcontentsline{toc}{subsubsection}{Problem 1.18.*}
\emph{Let $I$ be an ideal in a ring $R$.
If $a^n \in I$, $b^m \in I$, show that $(a+b)^{n+m} \in I$.
Show that $\mathrm{Rad}(I)$ is an ideal, in fact a radical ideal.
Show that any prime ideal is radical.} \\

\emph{Proof.}
\begin{enumerate}
\item[(1)]
  \emph{Show that $(a+b)^{n+m} \in I$ if $a^n \in I$, $b^m \in I$.}
  By the binomial theorem,
  \[
    (a+b)^{n+m}=\sum_{i=0}^{n+m} a^i b^{n+m-i}.
  \]
  For each term $a^i b^{n+m-i}$, either $i \geq n$ holds or $n+m-i \geq m$ holds,
  and thus $a^i b^{n+m-i} \in I$ (since $a^n \in I$, $b^m \in I$ and $I$ is an ideal).
  Hence, the result is established.

\item[(2)]
  \emph{Show that $\mathrm{Rad}(I)$ is an ideal.}
  \begin{enumerate}
  \item[(a)]
    $0 \in \mathrm{Rad}(I)$ since $0 = 0^{1} \in I$ for any ideal in $R$.

  \item[(b)]
    $(a+b)^{n+m} \in I$ if $a^n \in I$, $b^m \in I$ by (1).

  \item[(c)]
    $(-a)^{2n} = (a^n)^2 \in I$ if $a^n \in I$ (since $I$ is an ideal).

  \item[(d)]
    $(ra)^n = r^n a^n \in I$ if $a^n \in I$ and $r \in R$ (since $I$ is an ideal and $R$ is commutative).
  \end{enumerate}

\item[(3)]
  \emph{Show that $\mathrm{Rad}(\mathrm{Rad}(I)) = \mathrm{Rad}(I)$.}
  It suffices to show $\mathrm{Rad}(\mathrm{Rad}(I)) \subseteq \mathrm{Rad}(I)$.
  Given any $a \in \mathrm{Rad}(\mathrm{Rad}(I))$.
  By definition $a^n \in \mathrm{Rad}(I)$ for some positive integer $n$.
  Again by definition $(a^n)^m = a^{nm} \in I$ for some positive integer $m$.
  As $nm$ is a postive integer, $a \in \mathrm{Rad}(I)$.

\item[(4)]
  \emph{Show that every prime ideal $\mathfrak{p}$ is radical.}
  Given any $a \in \mathrm{Rad}(\mathfrak{p})$, that is,
  $a^n \in \mathfrak{p}$ for some positive integer.
  Write $a^n = a a^{n-1}$ if $n > 1$.
  By the primality of $\mathfrak{p}$, $a \in \mathfrak{p}$ or $a^{n-1} \in \mathfrak{p}$.
  If $a \in \mathfrak{p}$, we are done.
  If $a^{n-1} \in \mathfrak{p}$,
  we continue this descending argument until the power of $a$ is equal to $1$.
  Hence $\mathfrak{p}$ is radical.
\end{enumerate}
$\Box$\\\\



%%%%%%%%%%%%%%%%%%%%%%%%%%%%%%%%%%%%%%%%%%%%%%%%%%%%%%%%%%%%%%%%%%%%%%%%%%%%%%%%



\subsubsection*{Problem PLACEHOLDER}
\addcontentsline{toc}{subsubsection}{Problem PLACEHOLDER}
\emph{PLACEHOLDER} \\

\emph{Proof.}
\begin{enumerate}
\item[(1)]
  PLACEHOLDER
\end{enumerate}



%%%%%%%%%%%%%%%%%%%%%%%%%%%%%%%%%%%%%%%%%%%%%%%%%%%%%%%%%%%%%%%%%%%%%%%%%%%%%%%%



\subsection*{1.4. The Hilbert Basis Theorem \\}
\addcontentsline{toc}{subsection}{1.4. The Hilbert Basis Theorem}



%%%%%%%%%%%%%%%%%%%%%%%%%%%%%%%%%%%%%%%%%%%%%%%%%%%%%%%%%%%%%%%%%%%%%%%%%%%%%%%%



\subsection*{1.5. Irreducible Components of an Algebraic Set \\}
\addcontentsline{toc}{subsection}{1.5. Irreducible Components of an Algebraic Set}



%%%%%%%%%%%%%%%%%%%%%%%%%%%%%%%%%%%%%%%%%%%%%%%%%%%%%%%%%%%%%%%%%%%%%%%%%%%%%%%%



\subsection*{1.6. Algebraic Subsets of the Plane \\}
\addcontentsline{toc}{subsection}{1.6. Algebraic Subsets of the Plane}



%%%%%%%%%%%%%%%%%%%%%%%%%%%%%%%%%%%%%%%%%%%%%%%%%%%%%%%%%%%%%%%%%%%%%%%%%%%%%%%%



\subsection*{1.7. Hilbert's Nullstellensatz \\}
\addcontentsline{toc}{subsection}{1.7. Hilbert's Nullstellensatz}



%%%%%%%%%%%%%%%%%%%%%%%%%%%%%%%%%%%%%%%%%%%%%%%%%%%%%%%%%%%%%%%%%%%%%%%%%%%%%%%%



\subsection*{1.8. Modules; Finiteness Conditions \\}
\addcontentsline{toc}{subsection}{1.8. Modules; Finiteness Conditions}



%%%%%%%%%%%%%%%%%%%%%%%%%%%%%%%%%%%%%%%%%%%%%%%%%%%%%%%%%%%%%%%%%%%%%%%%%%%%%%%%



\subsection*{1.9. Integral Elements \\}
\addcontentsline{toc}{subsection}{1.9. Integral Elements}



%%%%%%%%%%%%%%%%%%%%%%%%%%%%%%%%%%%%%%%%%%%%%%%%%%%%%%%%%%%%%%%%%%%%%%%%%%%%%%%%



\subsection*{1.10. Field Extensions \\}
\addcontentsline{toc}{subsection}{1.10. Field Extensions}



%%%%%%%%%%%%%%%%%%%%%%%%%%%%%%%%%%%%%%%%%%%%%%%%%%%%%%%%%%%%%%%%%%%%%%%%%%%%%%%%
%%%%%%%%%%%%%%%%%%%%%%%%%%%%%%%%%%%%%%%%%%%%%%%%%%%%%%%%%%%%%%%%%%%%%%%%%%%%%%%%



\newpage
\section*{Chapter 2: Affine Varieties \\}
\addcontentsline{toc}{section}{Chapter 2: Affine Varieties}



\subsection*{2.1. Coordinate Rings \\}
\addcontentsline{toc}{subsection}{2.1. Coordinate Rings}



\subsubsection*{Problem 2.1.*}
\addcontentsline{toc}{subsubsection}{Problem 2.1.*}
\emph{Show that the map which associates to each
$f \in k[x_1,\ldots,x_n]$ a polynomial function in $\mathscr{F}(V,k)$
is a ring homomorphism whose kernel is $I(V)$.} \\

\emph{Proof.}
\begin{enumerate}
\item[(1)]
  Define a map $\alpha: k[x_1,\ldots,x_n] \to \mathscr{F}(V,k)$
  by $\alpha: f \mapsto f|_V$.

\item[(2)]
  Clearly $\alpha$ is a ring homomorphism.

\item[(3)]
  \emph{Show that $\mathrm{ker}(\alpha) = I(V)$.}
  In fact,
  given any $f \in k[x_1,\ldots,x_n]$, we have
  $\alpha(f) = 0$
  if and only if $f(a) = 0$ for all $a \in V$
  if and only if $f \in I(V)$.
  Hence, $\mathscr{F}(V,k) \cong k[x_1,\ldots,x_n]/I(V) = \Gamma(V)$
  as a ring isomorphism.
\end{enumerate}
$\Box$\\\\



%%%%%%%%%%%%%%%%%%%%%%%%%%%%%%%%%%%%%%%%%%%%%%%%%%%%%%%%%%%%%%%%%%%%%%%%%%%%%%%%



\subsubsection*{Problem PLACEHOLDER}
\addcontentsline{toc}{subsubsection}{Problem PLACEHOLDER}
\emph{PLACEHOLDER} \\

\emph{Proof.}
\begin{enumerate}
\item[(1)]
  PLACEHOLDER
\end{enumerate}



%%%%%%%%%%%%%%%%%%%%%%%%%%%%%%%%%%%%%%%%%%%%%%%%%%%%%%%%%%%%%%%%%%%%%%%%%%%%%%%%



\subsection*{2.2. Polynomial Maps \\}
\addcontentsline{toc}{subsection}{2.2. Polynomial Maps}



%%%%%%%%%%%%%%%%%%%%%%%%%%%%%%%%%%%%%%%%%%%%%%%%%%%%%%%%%%%%%%%%%%%%%%%%%%%%%%%%



\subsection*{2.3. Coordinate Changes \\}
\addcontentsline{toc}{subsection}{2.3. Coordinate Changes}



%%%%%%%%%%%%%%%%%%%%%%%%%%%%%%%%%%%%%%%%%%%%%%%%%%%%%%%%%%%%%%%%%%%%%%%%%%%%%%%%



\subsection*{2.4. Rational Functions and Local Rings \\}
\addcontentsline{toc}{subsection}{2.4. Rational Functions and Local Rings}



%%%%%%%%%%%%%%%%%%%%%%%%%%%%%%%%%%%%%%%%%%%%%%%%%%%%%%%%%%%%%%%%%%%%%%%%%%%%%%%%



\subsection*{2.5. Discrete Valuation Rings \\}
\addcontentsline{toc}{subsection}{2.5. Discrete Valuation Rings}



%%%%%%%%%%%%%%%%%%%%%%%%%%%%%%%%%%%%%%%%%%%%%%%%%%%%%%%%%%%%%%%%%%%%%%%%%%%%%%%%



\subsection*{2.6. Forms \\}
\addcontentsline{toc}{subsection}{2.6. Forms}



%%%%%%%%%%%%%%%%%%%%%%%%%%%%%%%%%%%%%%%%%%%%%%%%%%%%%%%%%%%%%%%%%%%%%%%%%%%%%%%%



\subsection*{2.7. Direct Products of Rings \\}
\addcontentsline{toc}{subsection}{2.7. Direct Products of Rings}



%%%%%%%%%%%%%%%%%%%%%%%%%%%%%%%%%%%%%%%%%%%%%%%%%%%%%%%%%%%%%%%%%%%%%%%%%%%%%%%%



\subsection*{2.8. Operations with Ideals \\}
\addcontentsline{toc}{subsection}{2.8. Operations with Ideals}



%%%%%%%%%%%%%%%%%%%%%%%%%%%%%%%%%%%%%%%%%%%%%%%%%%%%%%%%%%%%%%%%%%%%%%%%%%%%%%%%



\subsection*{2.9. Ideals with a Finite Number of Zeros \\}
\addcontentsline{toc}{subsection}{2.9. Ideals with a Finite Number of Zeros}



%%%%%%%%%%%%%%%%%%%%%%%%%%%%%%%%%%%%%%%%%%%%%%%%%%%%%%%%%%%%%%%%%%%%%%%%%%%%%%%%



\subsection*{2.10. Quotient Modules and Exact Sequences \\}
\addcontentsline{toc}{subsection}{2.10. Quotient Modules and Exact Sequences}



%%%%%%%%%%%%%%%%%%%%%%%%%%%%%%%%%%%%%%%%%%%%%%%%%%%%%%%%%%%%%%%%%%%%%%%%%%%%%%%%



\subsection*{2.11. Free Modules \\}
\addcontentsline{toc}{subsection}{2.11. Free Modules}



%%%%%%%%%%%%%%%%%%%%%%%%%%%%%%%%%%%%%%%%%%%%%%%%%%%%%%%%%%%%%%%%%%%%%%%%%%%%%%%%
%%%%%%%%%%%%%%%%%%%%%%%%%%%%%%%%%%%%%%%%%%%%%%%%%%%%%%%%%%%%%%%%%%%%%%%%%%%%%%%%



\newpage
\section*{Chapter 3: Local Properties of Plane Curves \\}
\addcontentsline{toc}{section}{Chapter 3: Local Properties of Plane Curves}



\subsection*{3.1. Multiple Points and Tangent Lines \\}
\addcontentsline{toc}{subsection}{3.1. Multiple Points and Tangent Lines}



\subsubsection*{Problem PLACEHOLDER}
\addcontentsline{toc}{subsubsection}{Problem PLACEHOLDER}
\emph{PLACEHOLDER} \\

\emph{Proof.}
\begin{enumerate}
\item[(1)]
  PLACEHOLDER
\end{enumerate}



%%%%%%%%%%%%%%%%%%%%%%%%%%%%%%%%%%%%%%%%%%%%%%%%%%%%%%%%%%%%%%%%%%%%%%%%%%%%%%%%



\subsection*{3.2. Multiplicities and Local Rings \\}
\addcontentsline{toc}{subsection}{3.2. Multiplicities and Local Rings}



%%%%%%%%%%%%%%%%%%%%%%%%%%%%%%%%%%%%%%%%%%%%%%%%%%%%%%%%%%%%%%%%%%%%%%%%%%%%%%%%



\subsection*{3.3. Intersection Numbers \\}
\addcontentsline{toc}{subsection}{3.3. Intersection Numbers}



%%%%%%%%%%%%%%%%%%%%%%%%%%%%%%%%%%%%%%%%%%%%%%%%%%%%%%%%%%%%%%%%%%%%%%%%%%%%%%%%
%%%%%%%%%%%%%%%%%%%%%%%%%%%%%%%%%%%%%%%%%%%%%%%%%%%%%%%%%%%%%%%%%%%%%%%%%%%%%%%%



\newpage
\section*{Chapter 4: Projective Varieties \\}
\addcontentsline{toc}{section}{Chapter 4: Projective Varieties}



\subsection*{4.1. Projective Space \\}
\addcontentsline{toc}{subsection}{4.1. Projective Space}



\subsubsection*{Problem PLACEHOLDER}
\addcontentsline{toc}{subsubsection}{Problem PLACEHOLDER}
\emph{PLACEHOLDER} \\

\emph{Proof.}
\begin{enumerate}
\item[(1)]
  PLACEHOLDER
\end{enumerate}



%%%%%%%%%%%%%%%%%%%%%%%%%%%%%%%%%%%%%%%%%%%%%%%%%%%%%%%%%%%%%%%%%%%%%%%%%%%%%%%%



\subsection*{4.2. Projective Algebraic Sets \\}
\addcontentsline{toc}{subsection}{4.2. Projective Algebraic Sets}



%%%%%%%%%%%%%%%%%%%%%%%%%%%%%%%%%%%%%%%%%%%%%%%%%%%%%%%%%%%%%%%%%%%%%%%%%%%%%%%%



\subsection*{4.3. Affine and Projective Varieties \\}
\addcontentsline{toc}{subsection}{4.3. Affine and Projective Varieties}



%%%%%%%%%%%%%%%%%%%%%%%%%%%%%%%%%%%%%%%%%%%%%%%%%%%%%%%%%%%%%%%%%%%%%%%%%%%%%%%%



\subsection*{4.4. Multiprojective Space \\}
\addcontentsline{toc}{subsection}{4.4. Multiprojective Space}



%%%%%%%%%%%%%%%%%%%%%%%%%%%%%%%%%%%%%%%%%%%%%%%%%%%%%%%%%%%%%%%%%%%%%%%%%%%%%%%%
%%%%%%%%%%%%%%%%%%%%%%%%%%%%%%%%%%%%%%%%%%%%%%%%%%%%%%%%%%%%%%%%%%%%%%%%%%%%%%%%



\newpage
\section*{Chapter 5: Projective Plane Curves\\}
\addcontentsline{toc}{section}{Chapter 5: Projective Plane Curves}



\subsection*{5.1. Definitions \\}
\addcontentsline{toc}{subsection}{5.1. Definitions}



\subsubsection*{Problem PLACEHOLDER}
\addcontentsline{toc}{subsubsection}{Problem PLACEHOLDER}
\emph{PLACEHOLDER} \\

\emph{Proof.}
\begin{enumerate}
\item[(1)]
  PLACEHOLDER
\end{enumerate}



%%%%%%%%%%%%%%%%%%%%%%%%%%%%%%%%%%%%%%%%%%%%%%%%%%%%%%%%%%%%%%%%%%%%%%%%%%%%%%%%



\subsection*{5.2. Linear Systems of Curves \\}
\addcontentsline{toc}{subsection}{5.2. Linear Systems of Curves}



%%%%%%%%%%%%%%%%%%%%%%%%%%%%%%%%%%%%%%%%%%%%%%%%%%%%%%%%%%%%%%%%%%%%%%%%%%%%%%%%



\subsection*{5.3. B\'ezout's Theorem  \\}
\addcontentsline{toc}{subsection}{5.3. B\'ezout's Theorem}



%%%%%%%%%%%%%%%%%%%%%%%%%%%%%%%%%%%%%%%%%%%%%%%%%%%%%%%%%%%%%%%%%%%%%%%%%%%%%%%%



\subsection*{5.4. Multiple Points \\}
\addcontentsline{toc}{subsection}{5.4. Multiple Points}



%%%%%%%%%%%%%%%%%%%%%%%%%%%%%%%%%%%%%%%%%%%%%%%%%%%%%%%%%%%%%%%%%%%%%%%%%%%%%%%%



\subsection*{5.5. Max Noether's Fundamental Theorem \\}
\addcontentsline{toc}{subsection}{5.5. Max Noether's Fundamental Theorem}



%%%%%%%%%%%%%%%%%%%%%%%%%%%%%%%%%%%%%%%%%%%%%%%%%%%%%%%%%%%%%%%%%%%%%%%%%%%%%%%%



\subsection*{5.6. Applications of Noether's Theorem \\}
\addcontentsline{toc}{subsection}{5.6. Applications of Noether's Theorem}



%%%%%%%%%%%%%%%%%%%%%%%%%%%%%%%%%%%%%%%%%%%%%%%%%%%%%%%%%%%%%%%%%%%%%%%%%%%%%%%%
%%%%%%%%%%%%%%%%%%%%%%%%%%%%%%%%%%%%%%%%%%%%%%%%%%%%%%%%%%%%%%%%%%%%%%%%%%%%%%%%



\newpage
\section*{Chapter 6: Varieties, Morphisms, and Rational Maps \\}
\addcontentsline{toc}{section}{Chapter 6: Varieties, Morphisms, and Rational Maps}

\subsection*{6.1. The Zariski Topology \\}
\addcontentsline{toc}{subsection}{6.1. The Zariski Topology}

\subsection*{6.2. Varieties \\}
\addcontentsline{toc}{subsection}{6.2. Varieties}

\subsection*{6.3. Morphisms of Varieties \\}
\addcontentsline{toc}{subsection}{6.3. Morphisms of Varieties}

\subsection*{6.4. Products and Graphs \\}
\addcontentsline{toc}{subsection}{6.4. Products and Graphs}

\subsection*{6.5. Algebraic Function Fields and Dimension of Varieties \\}
\addcontentsline{toc}{subsection}{6.5. Algebraic Function Fields and Dimension of Varieties}

\subsection*{6.6. Rational Maps \\}
\addcontentsline{toc}{subsection}{6.6. Rational Maps}



%%%%%%%%%%%%%%%%%%%%%%%%%%%%%%%%%%%%%%%%%%%%%%%%%%%%%%%%%%%%%%%%%%%%%%%%%%%%%%%%
%%%%%%%%%%%%%%%%%%%%%%%%%%%%%%%%%%%%%%%%%%%%%%%%%%%%%%%%%%%%%%%%%%%%%%%%%%%%%%%%



\newpage
\section*{Chapter 7: Resolution of Singularities \\}
\addcontentsline{toc}{section}{Chapter 7: Resolution of Singularities}



\subsection*{7.1. Rational Maps of Curves \\}
\addcontentsline{toc}{subsection}{7.1. Rational Maps of Curves}



\subsubsection*{Problem PLACEHOLDER}
\addcontentsline{toc}{subsubsection}{Problem PLACEHOLDER}
\emph{PLACEHOLDER} \\

\emph{Proof.}
\begin{enumerate}
\item[(1)]
  PLACEHOLDER
\end{enumerate}



%%%%%%%%%%%%%%%%%%%%%%%%%%%%%%%%%%%%%%%%%%%%%%%%%%%%%%%%%%%%%%%%%%%%%%%%%%%%%%%%



\subsection*{7.2. Blowing up a Point in $\mathbf{A}^{2}$ \\}
\addcontentsline{toc}{subsection}{7.2. Blowing up a Point in $\mathbf{A}^{2}$}



%%%%%%%%%%%%%%%%%%%%%%%%%%%%%%%%%%%%%%%%%%%%%%%%%%%%%%%%%%%%%%%%%%%%%%%%%%%%%%%%



\subsection*{7.3. Blowing up a Point in $\mathbf{P}^{2}$ \\}
\addcontentsline{toc}{subsection}{7.3. Blowing up a Point in $\mathbf{P}^{2}$}



%%%%%%%%%%%%%%%%%%%%%%%%%%%%%%%%%%%%%%%%%%%%%%%%%%%%%%%%%%%%%%%%%%%%%%%%%%%%%%%%



\subsection*{7.4. Quadratic Transformations \\}
\addcontentsline{toc}{subsection}{7.4. Quadratic Transformations}



%%%%%%%%%%%%%%%%%%%%%%%%%%%%%%%%%%%%%%%%%%%%%%%%%%%%%%%%%%%%%%%%%%%%%%%%%%%%%%%%



\subsection*{7.5. Nonsingular Models of Curves \\}
\addcontentsline{toc}{subsection}{7.5. Nonsingular Models of Curves}



%%%%%%%%%%%%%%%%%%%%%%%%%%%%%%%%%%%%%%%%%%%%%%%%%%%%%%%%%%%%%%%%%%%%%%%%%%%%%%%%
%%%%%%%%%%%%%%%%%%%%%%%%%%%%%%%%%%%%%%%%%%%%%%%%%%%%%%%%%%%%%%%%%%%%%%%%%%%%%%%%



\newpage
\section*{Chapter 8: Riemann-Roch Theorem \\}
\addcontentsline{toc}{section}{Chapter 8: Riemann-Roch Theorem}




\subsection*{8.1. Divisors \\}
\addcontentsline{toc}{subsection}{8.1. Divisors}



\subsubsection*{Problem PLACEHOLDER}
\addcontentsline{toc}{subsubsection}{Problem PLACEHOLDER}
\emph{PLACEHOLDER} \\

\emph{Proof.}
\begin{enumerate}
\item[(1)]
  PLACEHOLDER
\end{enumerate}



%%%%%%%%%%%%%%%%%%%%%%%%%%%%%%%%%%%%%%%%%%%%%%%%%%%%%%%%%%%%%%%%%%%%%%%%%%%%%%%%



\subsection*{8.2. The Vector Spaces $L(D)$ \\}
\addcontentsline{toc}{subsection}{8.1. The Vector Spaces $L(D)$}



%%%%%%%%%%%%%%%%%%%%%%%%%%%%%%%%%%%%%%%%%%%%%%%%%%%%%%%%%%%%%%%%%%%%%%%%%%%%%%%%



\subsection*{8.3. Riemann's Theorem \\}
\addcontentsline{toc}{subsection}{8.1. Riemann's Theorem}



%%%%%%%%%%%%%%%%%%%%%%%%%%%%%%%%%%%%%%%%%%%%%%%%%%%%%%%%%%%%%%%%%%%%%%%%%%%%%%%%



\subsection*{8.4. Derivations and Differentials \\}
\addcontentsline{toc}{subsection}{8.1. Derivations and Differentials}



%%%%%%%%%%%%%%%%%%%%%%%%%%%%%%%%%%%%%%%%%%%%%%%%%%%%%%%%%%%%%%%%%%%%%%%%%%%%%%%%



\subsection*{8.5. Canonical Divisors \\}
\addcontentsline{toc}{subsection}{8.1. Canonical Divisors}



%%%%%%%%%%%%%%%%%%%%%%%%%%%%%%%%%%%%%%%%%%%%%%%%%%%%%%%%%%%%%%%%%%%%%%%%%%%%%%%%



\subsection*{8.6. Riemann-Roch Theorem \\}
\addcontentsline{toc}{subsection}{8.6. Riemann-Roch Theorem}



%%%%%%%%%%%%%%%%%%%%%%%%%%%%%%%%%%%%%%%%%%%%%%%%%%%%%%%%%%%%%%%%%%%%%%%%%%%%%%%%
%%%%%%%%%%%%%%%%%%%%%%%%%%%%%%%%%%%%%%%%%%%%%%%%%%%%%%%%%%%%%%%%%%%%%%%%%%%%%%%%



\end{document}