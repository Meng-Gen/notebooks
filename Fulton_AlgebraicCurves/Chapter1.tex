\documentclass{article}
\usepackage{amsfonts}
\usepackage{amsmath}
\usepackage{amssymb}
\usepackage{hyperref}
\usepackage[none]{hyphenat}
\usepackage{mathrsfs}
\parindent=0pt

\def\upint{\mathchoice%
    {\mkern13mu\overline{\vphantom{\intop}\mkern7mu}\mkern-20mu}%
    {\mkern7mu\overline{\vphantom{\intop}\mkern7mu}\mkern-14mu}%
    {\mkern7mu\overline{\vphantom{\intop}\mkern7mu}\mkern-14mu}%
    {\mkern7mu\overline{\vphantom{\intop}\mkern7mu}\mkern-14mu}%
  \int}
\def\lowint{\mkern3mu\underline{\vphantom{\intop}\mkern7mu}\mkern-10mu\int}

\begin{document}

\textbf{\Large Chapter 1: Affine Algebraic Sets} \\\\



\emph{Author: Meng-Gen Tsai} \\
\emph{Email: plover@gmail.com} \\\\



%%%%%%%%%%%%%%%%%%%%%%%%%%%%%%%%%%%%%%%%%%%%%%%%%%%%%%%%%%%%%%%%%%%%%%%%%%%%%%%%



\textbf{Problem 1.1.}
\emph{Let $R$ be a domain.}
\begin{enumerate}
\item[(a)]
\emph{If $F$, $G$ are forms of degree $r$, $s$ respectively in $R[X_1,\ldots,X_n]$,
show that $FG$ is a form of degree $r+s$.}
\item[(b)]
\emph{Show that any factor of a form in $R[X_1,\ldots,X_n]$ is also a form. } \\
\end{enumerate}

\emph{Proof of (a).}
Write
\begin{align*}
  F &= \sum_{(i)} a_{(i)} X^{(i)}, \\
  G &= \sum_{(j)} b_{(j)} X^{(j)},
\end{align*}
where $\sum_{(i)}$ is the summation over $(i) = (i_1,\ldots,i_n)$ with $i_1+\cdots+i_n = r$
and $\sum_{(j)}$ is the summation over $(j) = (j_1,\ldots,j_n)$ with $j_1+\cdots+j_n = s$.
Hence,
\begin{align*}
  FG
  &= \sum_{(i)} \sum_{(j)} a_{(i)}b_{(j)} X^{(i)}X^{(j)} \\
  &= \sum_{(i),(j)} a_{(i)}b_{(j)} X^{(k)}
\end{align*}
where $(k) = (i_1+j_1,\ldots,i_n+j_n)$ with $(i_1+j_1)+\cdots+(i_n+j_n) = r+s$.
Each $X^{(k)}$ is the form of degree $r+s$ and $a_{(i)}b_{(j)} \in R$.
Hence $FG$ is a form of degree $r+s$.
$\Box$\\

\emph{Proof of (b).}
Given any form $F \in R[X_1,\ldots,X_n]$, and write $F = GH$.
\emph{It suffices to show that $G$ (or $H$) is a form as well.}
Write
\begin{align*}
  G &= G_0+\cdots+G_r, \\
  H &= H_0+\cdots+H_s
\end{align*}
where $G_r \neq 0$ and $H_s \neq 0$.
So
\[
  F = GH = G_0H_0 + \cdots + G_r H_s.
\]
Since $R$ is a domain, $R[X_1,\ldots,X_n]$ is a domain and thus $G_r H_s \neq 0$.
The maximality of $r$ and $s$ implies that $\deg(F) = r+s$.
Therefore, by the maximality of $r+s$,
$F = G_r H_s$, or $G = G_r$, or $G$ is a form.
$\Box$\\\\



%%%%%%%%%%%%%%%%%%%%%%%%%%%%%%%%%%%%%%%%%%%%%%%%%%%%%%%%%%%%%%%%%%%%%%%%%%%%%%%%
%%%%%%%%%%%%%%%%%%%%%%%%%%%%%%%%%%%%%%%%%%%%%%%%%%%%%%%%%%%%%%%%%%%%%%%%%%%%%%%%



\end{document}