\documentclass{article}
\usepackage{amsfonts}
\usepackage{amsmath}
\usepackage{amssymb}
\usepackage{centernot}
\usepackage{hyperref}
\usepackage[none]{hyphenat}
\usepackage{mathrsfs}
\usepackage{mathtools}
\usepackage{physics}
\usepackage{tikz-cd}
\parindent=0pt



\title{\textbf{Notes on the book: \\
\emph{Ash, Probability and Measure Theory, 2nd edition}}}
\author{Meng-Gen Tsai \\ plover@gmail.com}



\begin{document}
\maketitle
\tableofcontents



%%%%%%%%%%%%%%%%%%%%%%%%%%%%%%%%%%%%%%%%%%%%%%%%%%%%%%%%%%%%%%%%%%%%%%%%%%%%%%%%
%%%%%%%%%%%%%%%%%%%%%%%%%%%%%%%%%%%%%%%%%%%%%%%%%%%%%%%%%%%%%%%%%%%%%%%%%%%%%%%%



% Reference:



%%%%%%%%%%%%%%%%%%%%%%%%%%%%%%%%%%%%%%%%%%%%%%%%%%%%%%%%%%%%%%%%%%%%%%%%%%%%%%%%
%%%%%%%%%%%%%%%%%%%%%%%%%%%%%%%%%%%%%%%%%%%%%%%%%%%%%%%%%%%%%%%%%%%%%%%%%%%%%%%%



\newpage
\section*{Chapter 1: Fundamentals of Measure and Integration Theory \\}
\addcontentsline{toc}{section}{Chapter 1: Fundamentals of Measure and Integration Theory}



%%%%%%%%%%%%%%%%%%%%%%%%%%%%%%%%%%%%%%%%%%%%%%%%%%%%%%%%%%%%%%%%%%%%%%%%%%%%%%%%



\subsection*{1.1. Introduction \\}
\addcontentsline{toc}{subsection}{1.1. Introduction}



%%%%%%%%%%%%%%%%%%%%%%%%%%%%%%%%%%%%%%%%%%%%%%%%%%%%%%%%%%%%%%%%%%%%%%%%%%%%%%%%



\subsubsection*{Problem 1.1.1.}
\addcontentsline{toc}{subsubsection}{Problem 1.1.1.}
\emph{Establish formulas (1)-(5).} \\



\emph{Formulas.}
\begin{enumerate}
\item[(1)]
  If $A_n \uparrow A$, then $A_n^{c} \downarrow A^{c}$;
  If $A_n \downarrow A$, then $A_n^{c} \uparrow A^{c}$.

\item[(2)]
  \begin{align*}
    \bigcup_{i=1}^{n} A_i =
    & \:
      A_1
      \cup (A_1^{c} \cap A_2)
      \cup (A_1^{c} \cap A_2^{c} \cap A_3) \\
    & \:
      \cup \cdots \cup (A_1^{c} \cap \cdots \cap A_{n-1}^{c} \cap A_n).
  \end{align*}

\item[(3)]
  Furthermore,
  \[
    \bigcup_{n=1}^{\infty} A_n
    = \bigcup_{n=1}^{\infty} \bigg( A_1^{c} \cap \cdots \cap A_{n-1}^{c} \cap A_n \bigg).
  \]

\item[(4)]
  If the $A_n$ form an increasing sequence, then
  \[
    \bigcup_{i=1}^{n} A_i
    = A_1 \cup (A_2 - A_1) \cup \cdots \cup (A_n - A_{n-1}).
  \]

\item[(5)]
  If the $A_n$ form an increasing sequence, then
  \[
    \bigcup_{n=1}^{\infty} A_n
    = \bigcup_{n=1}^{\infty} (A_n - A_{n-1})
  \]
  (take $A_0$ as the empty set). \\
\end{enumerate}



\emph{Proof of Formula (1).}
\begin{enumerate}
\item[(1)]
  Suppose that $A_n \uparrow A$ is an increasing sequence of sets with limit $A$.
  Then $A_1 \subset A_2 \subset \cdots$ and $\bigcup_{n=1}^{\infty} A_n = A$.
  So $A_1^{c} \supset A_2^{c} \supset \cdots$ and
  \[
    \bigcap_{n} A_n^{c}
    = \bigg( \bigcup_{n} A_n \bigg)^{c} = A^{c}
  \]
  by the De Morgan laws.
  Hence $A_n \uparrow A$ implies that $A_n^{c} \downarrow A^{c}$.

\item[(2)]
  Conversely, suppose that $A_n \downarrow A$ is an decreasing sequence of sets with limit $A$.
  Then $A_1 \supset A_2 \supset \cdots$ and $\bigcap_{n=1}^{\infty} A_n = A$.
  So $A_1^{c} \subset A_2^{c} \subset \cdots$ and
  \[
    \bigcup_{n} A_n^{c}
    = \bigg( \bigcap_{n} A_n \bigg)^{c} = A^{c}
  \]
  by the De Morgan laws.
  Hence $A_n \downarrow A$ implies that $A_n^{c} \uparrow A^{c}$.
\end{enumerate}
$\Box$ \\\\



\emph{Proof of Formula (2).}
\begin{enumerate}
\item[(1)]
  Set
  \[
    B_i = A_1^{c} \cap \cdots \cap A_{i-1}^{c} \cap A_i
  \]
  for $i = 1, \cdots, n$.
  Observe that $B_1 = A_1$.
  So it is equivalent to show that
  \[
    \bigcup_{i=1}^{n} A_i = \bigcup_{i=1}^{n} B_i.
  \]

\item[(2)]
  Since each $B_i$ is a subset of $A_i$,
  $\bigcup_{i=1}^{n} A_i \supset \bigcup_{i=1}^{n} B_i$.

\item[(3)]
  Conversely, given any $x \in \bigcup_{i=1}^{n} A_i$.
  $x \in A_j$ for some $j$.
  Now take the minimal value of $j$ such that $x \in A_j$.
  The minimality of $j$ implies that $x \notin A_1, A_2, \cdots, A_{j-1}$.
  Hence
  \[
    x
    \in A_1^{c} \cap \cdots \cap A_{j-1}^{c} \cap A_j
    = B_j \subset \bigcup_{i=1}^{n} B_i.
  \]
  Therefore, $\bigcup_{i=1}^{n} A_i \subset \bigcup_{i=1}^{n} B_i$.

\item[(4)]
  By (2)(3), $\bigcup_{i=1}^{n} A_i$ and $\bigcup_{i=1}^{n} B_i$ are equal.
\end{enumerate}
$\Box$ \\\\



\emph{Proof of Formula (3).}
  Same as the proof of formula (2)
  since the minimality of $j$ described in part (3) exists.
$\Box$ \\\\



\emph{Proof of Formula (4).}
\begin{enumerate}
\item[(1)]
  As $A_n$ form an increasing sequence,
  $A_1 \subset A_2 \subset \cdots$ or
  $A_1^{c} \supset A_2^{c} \supset \cdots$.
  Hence
  \[
    A_1^{c} \cap \cdots \cap A_{i-1}^{c} = A_{i-1}^c.
  \]
  Therefore, $B_i$ is reduced to
  \[
    B_i
    = A_1^{c} \cap \cdots \cap A_{i-1}^{c} \cap A_i
    = A_{i-1}^{c} \cap A_i
    = A_{i} - A_{i-1}.
  \]

\item[(2)]
  Now formula (2) becomes
  \[
    \bigcup_{i=1}^{n} A_i
    = \bigcup_{i=1}^{n} B_i
    = \bigcup_{i=1}^{n} (A_{i} - A_{i-1}).
  \]
\end{enumerate}
$\Box$ \\\\



\emph{Proof of Formula (5).}
  Note that $B_n = A_{n} - A_{n-1}$ in the proof of formula (4).
  Formula (3) becomes
  $\bigcup_{n=1}^{\infty} A_n = \bigcup_{n=1}^{\infty} (A_n - A_{n-1})$.
$\Box$ \\\\



%%%%%%%%%%%%%%%%%%%%%%%%%%%%%%%%%%%%%%%%%%%%%%%%%%%%%%%%%%%%%%%%%%%%%%%%%%%%%%%%



\subsubsection*{Problem 1.1.2.}
\addcontentsline{toc}{subsubsection}{Problem 1.1.2.}
\emph{Define sets of real numbers as follows.
Let $A_n = (-\frac{1}{n}, 1]$ if $n$ is odd, and
$A_n = (-1, \frac{1}{n}]$ if $n$ is even.
Find $\limsup_n A_n$ and $\liminf_n A_n$.} \\



\emph{Proof.}
\begin{enumerate}
\item[(1)]
  Write
  \begin{align*}
    \bigcup_{k=n}^{\infty} A_k
    =
    & \:
      \Bigg( \bigcup_{k=\lfloor \frac{n}{2} \rfloor}^{\infty} A_{2k+1} \Bigg) \cup
      \Bigg( \bigcup_{k=\lfloor \frac{n+1}{2} \rfloor}^{\infty} A_{2k} \Bigg) \\
    =
    & \:
      \Bigg( \bigcup_{k=\lfloor \frac{n}{2} \rfloor}^{\infty} \bigg( -\frac{1}{2k+1},1 \bigg] \Bigg)
      \cup
      \Bigg( \bigcup_{k=\lfloor \frac{n+1}{2} \rfloor}^{\infty} \bigg( -1, \frac{1}{2k} \bigg] \Bigg) \\
    =
    & \:
      \bigg( -\frac{1}{2\lfloor \frac{n}{2} \rfloor+1},1 \bigg]
      \cup
      \bigg( -1, \frac{1}{2\lfloor \frac{n+1}{2} \rfloor} \bigg] \\
    =
    & \:
      (-1, 1]
  \end{align*}
  for each $k$.
  Hence
  \[
    \limsup_n A_n
    = \bigcap_{n=1}^{\infty} \bigcup_{k=n}^{\infty} A_k
    = \bigcap_{n=1}^{\infty} (-1, 1]
    = (-1, 1].
  \]

\item[(2)]
  Similarly, for each $k$ we have
  \begin{align*}
    \bigcap_{k=n}^{\infty} A_k
    =
    & \:
      \Bigg( \bigcap_{k=\lfloor \frac{n}{2} \rfloor}^{\infty} A_{2k+1} \Bigg) \cap
      \Bigg( \bigcap_{k=\lfloor \frac{n+1}{2} \rfloor}^{\infty} A_{2k} \Bigg) \\
    =
    & \:
      \Bigg( \bigcap_{k=\lfloor \frac{n}{2} \rfloor}^{\infty} \bigg( -\frac{1}{2k+1},1 \bigg] \Bigg)
      \cap
      \Bigg( \bigcap_{k=\lfloor \frac{n+1}{2} \rfloor}^{\infty} \bigg( -1, \frac{1}{2k} \bigg] \Bigg) \\
    =
    & \:
      [0,1] \cup (-1,0] \\
    =
    & \:
      \{ 0 \}.
  \end{align*}
  Hence
  \[
    \liminf_n A_n
    = \bigcup_{n=1}^{\infty} \bigcap_{k=n}^{\infty} A_k
    = \bigcup_{n=1}^{\infty} \{ 0 \}
    = \{ 0 \}.
  \]
\end{enumerate}
$\Box$ \\\\



%%%%%%%%%%%%%%%%%%%%%%%%%%%%%%%%%%%%%%%%%%%%%%%%%%%%%%%%%%%%%%%%%%%%%%%%%%%%%%%%



\subsubsection*{Problem 1.1.5.}
\addcontentsline{toc}{subsubsection}{Problem 1.1.5.}
\emph{Establish formulas (10)-(13).} \\



\emph{Formulas.}
\begin{enumerate}
\item[(10)]
  \[
    \Big( \limsup_{n} A_n \Big)^c
    = \liminf_{n} A_n^c.
  \]

\item[(11)]
  \[
    \Big( \liminf_{n} A_n \Big)^c
    = \limsup_{n} A_n^c.
  \]

\item[(12)]
  \[
    \liminf_{n} A_n \subset \limsup_{n} A_n.
  \]

\item[(13)]
  If $A_n \uparrow A$ or $A_n \downarrow A$,
  then $\liminf_n A_n = \limsup_n A_n= A$. \\
\end{enumerate}



\emph{Proof of Formula (10).}
  The De Morgan laws shows that
  \begin{align*}
    \Big( \limsup_{n} A_n \Big)^c
    =
    & \:
      \Bigg( \bigcap_{n=1}^{\infty} \bigcup_{k=n}^{\infty} A_k \Bigg)^c \\
    =
    & \:
      \bigcup_{n=1}^{\infty} \Bigg( \bigcup_{k=n}^{\infty} A_k \Bigg)^c \\
    =
    & \:
      \bigcup_{n=1}^{\infty} \bigcap_{k=n}^{\infty} A_k^{c} \\
    =
    & \:
      \limsup_{n} A_n^c.
  \end{align*}
$\Box$ \\\\



\emph{Proof of Formula (11).}
  Similar to the proof of formula (10).
  \begin{align*}
    \Big( \liminf_{n} A_n \Big)^c
    =
    & \:
      \Bigg( \bigcup_{n=1}^{\infty} \bigcap_{k=n}^{\infty} A_k \Bigg)^c \\
    =
    & \:
      \bigcap_{n=1}^{\infty} \Bigg( \bigcap_{k=n}^{\infty} A_k \Bigg)^c \\
    =
    & \:
      \bigcap_{n=1}^{\infty} \bigcup_{k=n}^{\infty} A_k^{c} \\
    =
    & \:
      \limsup_{n} A_n^c.
  \end{align*}
$\Box$ \\\\



\emph{Proof of Formula (12).}
  Formulas (7) and (9) give all.
$\Box$ \\\\



\emph{Proof of Formula (13).}
\begin{enumerate}
\item[(1)]
  If $A_n \uparrow A$, then
  \[
    \limsup_{n} A_n
    = \bigcap_{n=1}^{\infty} \bigcup_{k=n}^{\infty} A_k
    = \bigcap_{n=1}^{\infty} A
    = A
  \]
  and
  \[
    \liminf_{n} A_n
    = \bigcup_{n=1}^{\infty} \bigcap_{k=n}^{\infty} A_k
    = \bigcup_{n=1}^{\infty} A_n
    = A.
  \]

\item[(2)]
  If $A_n \downarrow A$, then
  \[
    \limsup_{n} A_n
    = \bigcap_{n=1}^{\infty} \bigcup_{k=n}^{\infty} A_k
    = \bigcap_{n=1}^{\infty} A_n
    = A
  \]
  and
  \[
    \liminf_{n} A_n
    = \bigcup_{n=1}^{\infty} \bigcap_{k=n}^{\infty} A_k
    = \bigcup_{n=1}^{\infty} A
    = A.
  \]
\end{enumerate}
$\Box$ \\\\



%%%%%%%%%%%%%%%%%%%%%%%%%%%%%%%%%%%%%%%%%%%%%%%%%%%%%%%%%%%%%%%%%%%%%%%%%%%%%%%%



\subsubsection*{Problem 1.1.6.}
\addcontentsline{toc}{subsubsection}{Problem 1.1.6.}
\emph{Let $A = (a,b)$ and $B = (c,d)$ be disjoint open intervals of $\mathbb{R}$,
and let $C_n = A$ if $n$ is odd, $C_n = B$ if $n$ is even.
Find $\limsup_n C_n$ and $\liminf_n C_n$.} \\



\emph{Proof.}
\begin{enumerate}
\item[(1)]
  \[
    \limsup_{n} C_n
    = \bigcap_{n=1}^{\infty} \bigcup_{k=n}^{\infty} C_k
    = \bigcap_{n=1}^{\infty} ( A \cup B )
    = A \cup B.
  \]

\item[(2)]
  \[
    \liminf_{n} C_n
    = \bigcup_{n=1}^{\infty} \bigcap_{k=n}^{\infty} C_k
    = \bigcup_{n=1}^{\infty} \varnothing
    = \varnothing.
  \]
\end{enumerate}
$\Box$ \\\\



%%%%%%%%%%%%%%%%%%%%%%%%%%%%%%%%%%%%%%%%%%%%%%%%%%%%%%%%%%%%%%%%%%%%%%%%%%%%%%%%



\subsection*{1.2. Fields, $\sigma$-Fields, and Measures \\}
\addcontentsline{toc}{subsection}{1.2. Fields, $\sigma$-Fields, and Measures}



%%%%%%%%%%%%%%%%%%%%%%%%%%%%%%%%%%%%%%%%%%%%%%%%%%%%%%%%%%%%%%%%%%%%%%%%%%%%%%%%



\subsubsection*{Problem 1.2.5.}
\addcontentsline{toc}{subsubsection}{Problem 1.2.5.}
\emph{Let $\mu$ be a nonnegative,
finitely additive set function on the field $\mathscr{F}$.
If $A_1, A_2, \ldots$ are disjoint sets in $\mathscr{F}$
and $\bigcup_{n=1}^{\infty} A_n \in \mathscr{F}$, show that}
\[
  \mu \Bigg( \bigcup_{n=1}^{\infty} A_n \Bigg)
  \geq
  \sum_{n=1}^{\infty} \mu(A_n).
\] \\



\emph{Proof.}
\begin{enumerate}
\item[(1)]
  Note that $\mu$ is a nonnegative, finitely additive set function on $\mathscr{F}$.
  Hence,
  \begin{align*}
    \mu \Bigg( \bigcup_{n=1}^{\infty} A_n \Bigg)
    \geq
    & \:
      \mu \Bigg( \bigcup_{n=1}^{m} A_n \Bigg)
      & (\text{Theorem 1.2.5}) \\
    =
    & \:
      \sum_{n=1}^{m} \mu(A_n)
  \end{align*}
  for every $m$.

\item[(2)]
  Since $\sum_{n=1}^{m} \mu(A_n)$ is bounded by $\mu \big( \bigcup_{n=1}^{\infty} A_n \big)$
  and $\mu$ is nonnegative,
  the result is established as letting $m \to \infty$.
\end{enumerate}
$\Box$ \\\\


%%%%%%%%%%%%%%%%%%%%%%%%%%%%%%%%%%%%%%%%%%%%%%%%%%%%%%%%%%%%%%%%%%%%%%%%%%%%%%%%
%%%%%%%%%%%%%%%%%%%%%%%%%%%%%%%%%%%%%%%%%%%%%%%%%%%%%%%%%%%%%%%%%%%%%%%%%%%%%%%%



\end{document}