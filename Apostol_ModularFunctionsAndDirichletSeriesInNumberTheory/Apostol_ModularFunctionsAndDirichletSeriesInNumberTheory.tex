\documentclass{article}
\usepackage{amsfonts}
\usepackage{amsmath}
\usepackage{amssymb}
\usepackage{centernot}
\usepackage{hyperref}
\usepackage[none]{hyphenat}
\usepackage{mathrsfs}
\usepackage{mathtools}
\usepackage{physics}
\usepackage{tikz-cd}
\parindent=0pt



\title{\textbf{Notes on the book: \\
\emph{Apostol, Modular Functions and Dirichlet Series in Number Theory, 2nd edition}}}
\author{Meng-Gen Tsai \\ plover@gmail.com}



\begin{document}
\maketitle
\tableofcontents



%%%%%%%%%%%%%%%%%%%%%%%%%%%%%%%%%%%%%%%%%%%%%%%%%%%%%%%%%%%%%%%%%%%%%%%%%%%%%%%%
%%%%%%%%%%%%%%%%%%%%%%%%%%%%%%%%%%%%%%%%%%%%%%%%%%%%%%%%%%%%%%%%%%%%%%%%%%%%%%%%



% Reference:



%%%%%%%%%%%%%%%%%%%%%%%%%%%%%%%%%%%%%%%%%%%%%%%%%%%%%%%%%%%%%%%%%%%%%%%%%%%%%%%%
%%%%%%%%%%%%%%%%%%%%%%%%%%%%%%%%%%%%%%%%%%%%%%%%%%%%%%%%%%%%%%%%%%%%%%%%%%%%%%%%



\newpage
\section*{Chapter 1: Elliptic functions \\}
\addcontentsline{toc}{section}{Chapter 1: Elliptic functions}



\subsubsection*{Exercise 1.11.}
\addcontentsline{toc}{subsubsection}{Exercise 1.11.}
\emph{If $k \geq 2$ and $\tau \in H$ prove that the Eisenstein series
\[
  G_{2k}(\tau)
  = \sum_{(m,n) \neq (0,0)} (m+n\tau)^{-2k}
\]
has the Fourier expansion
}
\[
  G_{2k}(\tau)
  =
  2 \zeta(2k) + \frac{2(2\pi i)^{2k}}{(2k-1)!} \sum_{n=1}^{\infty} \sigma_{2k-1}(n) e^{2\pi i n \tau}.
\] \\



\emph{Proof.}
\begin{enumerate}
\item[(1)]
  Similar to Lemma 1.3 on page 19, we have
  \[
    (2k-1)! \sum_{m=-\infty}^{+\infty} \frac{1}{(\tau+m)^{2k}}
    = (2 \pi i)^{2k} \sum_{r=1}^{\infty} r^{2k-1} e^{2 \pi i r \tau}.
  \]

\item[(2)]
  Similar to Theorem 1.18, we have
  \begin{align*}
    G_{2k}(\tau)
    &= \sum_{(m,n) \neq (0,0)} (m+n\tau)^{-2k} \\
    &= \sum_{\substack{m=-\infty \\ m \neq 0 (n = 0)}}^{+\infty} m^{-2k}
      + \sum_{n=1}^{\infty} \sum_{m=-\infty}^{+\infty} ((m+n\tau)^{-2k} + (m-n\tau)^{-2k}) \\
    &= 2 \zeta(2k)
      + 2 \sum_{n=1}^{\infty} \sum_{m=-\infty}^{+\infty} (m+n\tau)^{-2k} \\
    &= 2 \zeta(2k)
      + 2 \sum_{n=1}^{\infty} \frac{(2\pi i)^{2k}}{(2k-1)!}
        \sum_{r=1}^{\infty} r^{2k-1} e^{2\pi i nr \tau} \\
    &= 2 \zeta(2k)
      + \frac{2(2\pi i)^{2k}}{(2k-1)!} \sum_{n=1}^{\infty}
        \underbrace{\sum_{d|n} d^{2k-1}}_{= \sigma_{2k-1}(n)} e^{2\pi i n \tau}.
  \end{align*}
  In the last double sum we collect together those terms for which $nr$ is constant.
\end{enumerate}
$\Box$ \\\\



%%%%%%%%%%%%%%%%%%%%%%%%%%%%%%%%%%%%%%%%%%%%%%%%%%%%%%%%%%%%%%%%%%%%%%%%%%%%%%%%



\subsubsection*{Exercise 1.12.}
\addcontentsline{toc}{subsubsection}{Exercise 1.12.}
\emph{Refer to Exercise 1.11. If $\tau \in H$ prove that
\[
  G_{2k}\left(-\frac{1}{\tau}\right) = \tau^{2k} G_{2k}(\tau)
\]
and deduce that
\begin{align*}
  G_{2k}\left(\frac{i}{2}\right) &= (-4)^k G_{2k}(2i)
    &\text{for all $k \geq 2$}, \\
  G_{2k}(i) &= 0
    &\text{if $k$ is odd}, \\
  G_{2k}(e^{\frac{2\pi i}{3}}) &= 0
    &\text{if $k \not\equiv 0 \pmod{3}$}.
\end{align*}} \\



\emph{Proof.}
\begin{enumerate}
\item[(1)]
  \begin{align*}
    G_{2k}\left(-\frac{1}{\tau}\right)
    &= \sum_{(m,n) \neq (0,0)} \left(m - \frac{n}{\tau}\right)^{-2k} \\
    &= \tau^{2k} \sum_{(m,n) \neq (0,0)} (\tau m - n)^{-2k} \\
    &= \tau^{2k} G_{2k}(\tau).
  \end{align*}

\item[(2)]
  Let $\tau = 2i$.
  We have $G_{2k}\left(\frac{i}{2}\right) = (-4)^k G_{2k}(2i)$.

\item[(3)]
  Let $\tau = i$.
  We have $G_{2k}(i) = (-1)^k G_{2k}(i)$.
  Hence $G_{2k}(i) = 0$ if $k$ is odd.

\item[(4)]
  Let $\tau = e^{\frac{\pi i}{3}}$.
  We have $G_{2k}(e^{\frac{2\pi i}{3}}) = e^{\frac{2k \pi i}{3}} G_{2k}(e^{\frac{\pi i}{3}})$.
  Since
  \[
    e^{\frac{2\pi i}{3}} = -1 + e^{\frac{\pi i}{3}}
  \]
  and each Eisenstein series is a periodic function of $\tau$ of period $1$,
  we have $G_{2k}(e^{\frac{2\pi i}{3}}) = G_{2k}(e^{\frac{\pi i}{3}})$.
  So $G_{2k}(e^{\frac{2\pi i}{3}}) = e^{\frac{2k \pi i}{3}} G_{2k}(e^{\frac{2\pi i}{3}})$.
  Therefore $G_{2k}(e^{\frac{2\pi i}{3}}) = 0$ if $k \not\equiv 0 \pmod{3}$.
\end{enumerate}
$\Box$ \\\\



%%%%%%%%%%%%%%%%%%%%%%%%%%%%%%%%%%%%%%%%%%%%%%%%%%%%%%%%%%%%%%%%%%%%%%%%%%%%%%%%
%%%%%%%%%%%%%%%%%%%%%%%%%%%%%%%%%%%%%%%%%%%%%%%%%%%%%%%%%%%%%%%%%%%%%%%%%%%%%%%%



\end{document}