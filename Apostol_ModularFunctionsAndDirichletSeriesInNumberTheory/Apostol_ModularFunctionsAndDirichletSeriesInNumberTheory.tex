\documentclass{article}
\usepackage{amsfonts}
\usepackage{amsmath}
\usepackage{amssymb}
\usepackage{centernot}
\usepackage{hyperref}
\usepackage[none]{hyphenat}
\usepackage{mathrsfs}
\usepackage{mathtools}
\usepackage{physics}
\usepackage{tikz-cd}
\parindent=0pt



\title{\textbf{Notes on the book: \\
\emph{Apostol, Modular Functions and Dirichlet Series in Number Theory, 2nd edition}}}
\author{Meng-Gen Tsai \\ plover@gmail.com}



\begin{document}
\maketitle
\tableofcontents



%%%%%%%%%%%%%%%%%%%%%%%%%%%%%%%%%%%%%%%%%%%%%%%%%%%%%%%%%%%%%%%%%%%%%%%%%%%%%%%%
%%%%%%%%%%%%%%%%%%%%%%%%%%%%%%%%%%%%%%%%%%%%%%%%%%%%%%%%%%%%%%%%%%%%%%%%%%%%%%%%



% Reference:



%%%%%%%%%%%%%%%%%%%%%%%%%%%%%%%%%%%%%%%%%%%%%%%%%%%%%%%%%%%%%%%%%%%%%%%%%%%%%%%%
%%%%%%%%%%%%%%%%%%%%%%%%%%%%%%%%%%%%%%%%%%%%%%%%%%%%%%%%%%%%%%%%%%%%%%%%%%%%%%%%



\newpage
\section*{Chapter 1: Elliptic functions \\}
\addcontentsline{toc}{section}{Chapter 1: Elliptic functions}



\subsubsection*{Exercise 1.11.}
\addcontentsline{toc}{subsubsection}{Exercise 1.11.}
\emph{If $k \geq 2$ and $\tau \in H$ prove that the Eisenstein series
\[
  G_{2k}(\tau)
  = \sum_{(m,n) \neq (0,0)} (m+n\tau)^{-2k}
\]
has the Fourier expansion
}
\[
  G_{2k}(\tau)
  =
  2 \zeta(2k) + \frac{2(2\pi i)^{2k}}{(2k-1)!} \sum_{n=1}^{\infty} \sigma_{2k-1}(n) e^{2\pi i n \tau}.
\]



\emph{Proof.}
\begin{enumerate}
\item[(1)]
  Let $q = e^{2 \pi i \tau}$.
  Similar to Lemma 1.3 on page 19, we have
  \[
    (2k-1)! \sum_{m=-\infty}^{+\infty} \frac{1}{(\tau+m)^{2k}}
    = (2 \pi i)^{2k} \sum_{r=1}^{\infty} r^{2k-1} q^r.
  \]

\item[(2)]
  Similar to Theorem 1.18, we have
  \begin{align*}
    G_{2k}(\tau)
    &= \sum_{(m,n) \neq (0,0)} (m+n\tau)^{-2k} \\
    &= \sum_{\substack{m=-\infty \\ m \neq 0 (n = 0)}}^{+\infty} m^{-2k}
      + \sum_{n=1}^{\infty} \sum_{m=-\infty}^{+\infty} ((m+n\tau)^{-2k} + (m-n\tau)^{-2k}) \\
    &= 2 \zeta(2k)
      + 2 \sum_{n=1}^{\infty} \sum_{m=-\infty}^{+\infty} (m+n\tau)^{-2k} \\
    &= 2 \zeta(2k)
      + 2 \sum_{n=1}^{\infty} \frac{(2\pi i)^{2k}}{(2k-1)!}
        \sum_{r=1}^{\infty} r^{2k-1} q^{nr} \\
    &= 2 \zeta(2k)
      + \frac{2(2\pi i)^{2k}}{(2k-1)!} \sum_{n=1}^{\infty}
        \underbrace{\sum_{d|n} d^{2k-1}}_{= \sigma_{2k-1}(n)} q^{n}.
  \end{align*}
  In the last double sum we collect together those terms for which $nr$ is constant.
\end{enumerate}
$\Box$ \\\\



%%%%%%%%%%%%%%%%%%%%%%%%%%%%%%%%%%%%%%%%%%%%%%%%%%%%%%%%%%%%%%%%%%%%%%%%%%%%%%%%



\subsubsection*{Exercise 1.12.}
\addcontentsline{toc}{subsubsection}{Exercise 1.12.}
\emph{Refer to Exercise 1.11. If $\tau \in H$ prove that
\[
  G_{2k}\left(-\frac{1}{\tau}\right) = \tau^{2k} G_{2k}(\tau)
\]
and deduce that
\begin{align*}
  G_{2k}\left(\frac{i}{2}\right) &= (-4)^k G_{2k}(2i)
    &\text{for all $k \geq 2$}, \\
  G_{2k}(i) &= 0
    &\text{if $k$ is odd}, \\
  G_{2k}(e^{\frac{2\pi i}{3}}) &= 0
    &\text{if $k \not\equiv 0 \pmod{3}$}.
\end{align*}}



\emph{Proof.}
\begin{enumerate}
\item[(1)]
  \begin{align*}
    G_{2k}\left(-\frac{1}{\tau}\right)
    &= \sum_{(m,n) \neq (0,0)} \left(m - \frac{n}{\tau}\right)^{-2k} \\
    &= \tau^{2k} \sum_{(m,n) \neq (0,0)} (\tau m - n)^{-2k} \\
    &= \tau^{2k} G_{2k}(\tau).
  \end{align*}

\item[(2)]
  Let $\tau = 2i$.
  We have $G_{2k}\left(\frac{i}{2}\right) = (-4)^k G_{2k}(2i)$.

\item[(3)]
  Let $\tau = i$.
  We have $G_{2k}(i) = (-1)^k G_{2k}(i)$.
  Hence $G_{2k}(i) = 0$ if $k$ is odd.

\item[(4)]
  Let $\tau = e^{\frac{\pi i}{3}}$.
  We have $G_{2k}(e^{\frac{2\pi i}{3}}) = e^{\frac{2k \pi i}{3}} G_{2k}(e^{\frac{\pi i}{3}})$.
  Since
  \[
    e^{\frac{2\pi i}{3}} = -1 + e^{\frac{\pi i}{3}}
  \]
  and each Eisenstein series is a periodic function of $\tau$ of period $1$,
  we have $G_{2k}(e^{\frac{2\pi i}{3}}) = G_{2k}(e^{\frac{\pi i}{3}})$.
  So $G_{2k}(e^{\frac{2\pi i}{3}}) = e^{\frac{2k \pi i}{3}} G_{2k}(e^{\frac{2\pi i}{3}})$.
  Therefore $G_{2k}(e^{\frac{2\pi i}{3}}) = 0$ if $k \not\equiv 0 \pmod{3}$.
\end{enumerate}
$\Box$ \\\\



%%%%%%%%%%%%%%%%%%%%%%%%%%%%%%%%%%%%%%%%%%%%%%%%%%%%%%%%%%%%%%%%%%%%%%%%%%%%%%%%



\subsubsection*{Exercise 1.13.}
\addcontentsline{toc}{subsubsection}{Exercise 1.13.}
\emph{Ramanujan's tau function $\tau(n)$ is defined by the Fourier expansion
\[
  \Delta(\tau)
  =
  (2\pi)^{12} \sum_{n=1}^{\infty} \tau(n) e^{2\pi i n \tau},
\]
derived in Theorem 1.19.
Prove that
\[
  \tau(n)
  =
  8000 \{ (\sigma_3 \circ \sigma_3) \circ \sigma_3 \}(n)
  - 147( \sigma_5 \circ \sigma_5 )(n),
\]
where $f \circ g$ denotes the Cauchy product of two sequences,
\[
  (f \circ g)(n) = \sum_{k=0}^{n} f(k)g(n-k),
\]
and $\sigma_\alpha(n) = \sum_{d|n} d^{\alpha}$ for $n \geq 1$,
with $\sigma_3(0) = \frac{1}{240}$, $\sigma_5(0) = -\frac{1}{504}$.
(Hint: Theorem 1.18.)} \\



\emph{Proof.}
\begin{enumerate}
\item[(1)]
  Let $q = e^{2 \pi i \tau}$.
  Write
  \begin{align*}
    g_2(\tau)
    &=
    \frac{4\pi^4}{3} \left\{ 1 + 240 \sum_{k=1}^{\infty} \sigma_3(k) q^k \right\}
    = \frac{4\pi^4}{3} \cdot \left( 240 \sum_{k=0}^{\infty} \sigma_3(k) q^k \right), \\
    g_3(\tau)
    &=
    \frac{8\pi^6}{27} \left\{ 1 - 504 \sum_{k=1}^{\infty} \sigma_5(k) q^k \right\}
    = \frac{8\pi^6}{27} \cdot \left( -504 \sum_{k=0}^{\infty} \sigma_5(k) q^k \right)
  \end{align*}
  (Theorem 1.18).

\item[(2)]
  Similar to the proof of Theorem 1.19,
  \begin{align*}
    \Delta(\tau)
    =&\:
    g_2(\tau)^3 - 27 g_3(\tau)^2 \\
    =&\:
    \frac{64 \pi^{12}}{27}
      \left\{ \left( 240 \sum_{k=0}^{\infty} \sigma_3(k) q^k \right)^3
        - \left( -504 \sum_{k=0}^{\infty} \sigma_5(k) q^k \right)^2 \right\} \\
    =&\:
    (2\pi)^{12} \left\{ 8000 \left( \sum_{k=0}^{\infty} \sigma_3(k) q^k \right)^3
      - 147 \left( \sum_{k=0}^{\infty} \sigma_5(k) q^k \right)^2 \right\} \\
    =&\:
      (2\pi)^{12} \sum_{n=0}^{\infty}
        \left\{ 8000 \{ (\sigma_3 \circ \sigma_3) \circ \sigma_3 \}(n)
        - 147 ( \sigma_5 \circ \sigma_5 )(n) \right\} q^n \\
    =&\:
      (2\pi)^{12} \sum_{n=1}^{\infty}
        \left\{ 8000 \{ (\sigma_3 \circ \sigma_3) \circ \sigma_3 \}(n)
        - 147 ( \sigma_5 \circ \sigma_5 )(n) \right\} q^n.
  \end{align*}
  (Here $8000 \{ (\sigma_3 \circ \sigma_3) \circ \sigma_3 \}(0)
    - 147 ( \sigma_5 \circ \sigma_5 )(0) = 0$.)

\item[(3)]
  Therefore
  \[
    \tau(n)
    =
    8000 \{ (\sigma_3 \circ \sigma_3) \circ \sigma_3 \}(n)
      - 147 ( \sigma_5 \circ \sigma_5 )(n)
  \]
  for $n \geq 1$.
\end{enumerate}
$\Box$ \\\\



%%%%%%%%%%%%%%%%%%%%%%%%%%%%%%%%%%%%%%%%%%%%%%%%%%%%%%%%%%%%%%%%%%%%%%%%%%%%%%%%



\subsubsection*{Exercise 1.14. (Lambert series)}
\addcontentsline{toc}{subsubsection}{Exercise 1.14. (Lambert series)}
\emph{A series of the form $\sum_{n=1}^{\infty} f(n)\frac{x^n}{1-x^n}$
is called a \textbf{Lambert series}.
Assuming absolute convergence, prove that
\[
  \sum_{n=1}^{\infty} f(n)\frac{x^n}{1-x^n}
  = \sum_{n=1}^{\infty} F(n)x^n,
\]
where
\[
  F(n) = \sum_{d \mid n} f(d).
\]
Apply this result to obtain the following formulas, valid for $|x| < 1$.}
\begin{enumerate}
\item[(a)]
  \[
    \sum_{n=1}^{\infty} \frac{\mu(n)x^n}{1-x^n} = x.
  \]

\item[(b)]
  \[
    \sum_{n=1}^{\infty} \frac{\varphi(n)x^n}{1-x^n} = \frac{x}{(1-x)^2}.
  \]

\item[(c)]
  \[
    \sum_{n=1}^{\infty} \frac{n^{\alpha}x^n}{1-x^n}
    = \sum_{n=1}^{\infty} \sigma_{\alpha}(n)x^n.
  \]

\item[(d)]
  \[
    \sum_{n=1}^{\infty} \frac{\lambda(n)x^n}{1-x^n} = \sum_{n=1}^{\infty} x^{n^2}.
  \]

\item[(e)]
  \emph{Use the result in (c) to express $g_2(\tau)$ and $g_3(\tau)$
  in terms of Lambert series in $x = e^{2\pi i \tau}$.} \\
\end{enumerate}
\emph{Note. In (a), $\mu(n)$ is the M\"obius function;
In (b), $\varphi(n)$ is Euler's totient;
and in (d), $\lambda(n)$ is Liouville's function.} \\



\emph{Proof.}
  Similar to the proof of Exercise 1.11.
  \begin{align*}
    \sum_{n=1}^{\infty} f(n)\frac{x^n}{1-x^n}
    &= \sum_{n=1}^{\infty} f(n) \sum_{r=1}^{\infty} x^{rn} \\
    &= \sum_{n=1}^{\infty} \sum_{r=1}^{\infty} f(n) x^{rn} \\
    &= \sum_{n=1}^{\infty} \underbrace{\left( \sum_{d|n} f(d) \right)}_{= F(n)} x^n.
  \end{align*}
$\Box$ \\



\emph{Proof of (a).}
  Theorem 2.1 in the textbook: \emph{T. M. Apostol, Introduction to Analytic Number Theory}
  shows that
  \begin{equation*}
    F(n) := \sum_{d \mid n} \mu(d) =
    \begin{cases}
      1 & \text{if $n = 1$}, \\
      0 & \text{if $n > 1$}.
    \end{cases}
  \end{equation*}
  Hence
  \[
    \sum_{n=1}^{\infty} \mu(n)\frac{x^n}{1-x^n}
    = \sum_{n=1}^{\infty} F(n)x^n
    = x.
  \]
$\Box$ \\



\emph{Proof of (b).}
  Theorem 2.2 in the textbook: \emph{T. M. Apostol, Introduction to Analytic Number Theory}
  shows that $F(n) := \sum_{d \mid n} \varphi(d) = n$.
  Hence
  \[
    \sum_{n=1}^{\infty} \varphi(n)\frac{x^n}{1-x^n}
    = \sum_{n=1}^{\infty} F(n)x^n
    = \sum_{n=1}^{\infty} n x^n
    = \frac{x}{(1-x)^2}.
  \]
$\Box$ \\\\



\emph{Proof of (c).}
  Since
  \[
    F(n) := \sum_{d \mid n} d^{\alpha} = \sigma_\alpha(n),
  \]
  we have
  \[
    \sum_{n=1}^{\infty} n^{\alpha}\frac{x^n}{1-x^n}
    = \sum_{n=1}^{\infty} F(n)x^n
    = \sum_{n=1}^{\infty} \sigma_\alpha(n) x^n.
  \]
$\Box$ \\



\emph{Proof of (d).}
  Theorem 2.19 in the textbook: \emph{T. M. Apostol, Introduction to Analytic Number Theory}
  shows that
  \begin{equation*}
    F(n) := \sum_{d \mid n} \lambda(d) =
    \begin{cases}
      1 & \text{if $n$ is a square}, \\
      0 & \text{otherwise}.
    \end{cases}
  \end{equation*}
  Hence
  \[
    \sum_{n=1}^{\infty} \lambda(n)\frac{x^n}{1-x^n}
    = \sum_{n=1}^{\infty} F(n)x^n
    = \sum_{n=1}^{\infty} x^{n^2}.
  \]
$\Box$ \\



\emph{Proof of (e).}
\begin{enumerate}
\item[(1)]
  Let $q = x = e^{2\pi i \tau}$.
  \begin{align*}
    g_2(\tau)
    &= \frac{4\pi^4}{3} \left\{ 1 + 240 \sum_{k=1}^{\infty} \sigma_3(k) q^k \right\}
      &(\text{Theorem 1.18}) \\
    &= \frac{4\pi^4}{3} \left\{ 1 + 240 \sum_{k=1}^{\infty} \frac{k^3 q^k}{1-q^k} \right\}
      &(\text{(c)}).
  \end{align*}

\item[(2)]
  Similarly,
  \begin{align*}
    g_3(\tau)
    &= \frac{8\pi^6}{27} \left\{ 1 - 504 \sum_{k=1}^{\infty} \sigma_5(k) q^k \right\}
      &(\text{Theorem 1.18}) \\
    &= \frac{8\pi^6}{27} \left\{ 1 - 504 \sum_{k=1}^{\infty} \frac{k^5 q^k}{1-q^k} \right\}
      &(\text{(c)}).
  \end{align*}
\end{enumerate}
$\Box$ \\\\



%%%%%%%%%%%%%%%%%%%%%%%%%%%%%%%%%%%%%%%%%%%%%%%%%%%%%%%%%%%%%%%%%%%%%%%%%%%%%%%%



\subsubsection*{Exercise 1.15.}
\addcontentsline{toc}{subsubsection}{Exercise 1.15.}
\emph{Let
\[
  G(x) = \sum_{n=1}^{\infty} \frac{n^5 x^n}{1-x^n},
\]
and let
\[
  F(x) = \sum_{\substack{n=1 \\ (\text{$n$ odd})}}^{\infty} \frac{n^5 x^n}{1+x^n}.
\]}
\begin{enumerate}
\item[(a)]
  \emph{Prove that $F(x) = G(x) - 34G(x^2) + 64(x^4)$.}

\item[(b)]
  \emph{Prove that
  \[
    \sum_{\substack{n=1 \\ (\text{$n$ odd})}}^{\infty} \frac{n^5}{1+e^{n\pi}}
    = \frac{31}{504}.
  \]}

\item[(c)]
  \emph{Use Theorem 12.17 in the textbook: T. M. Apostol, Introduction to Analytic Number Theory
  to prove the more general result}
  \[
    \sum_{\substack{n=1 \\ (\text{$n$ odd})}}^{\infty} \frac{n^{4k+1}}{1+e^{n\pi}}
    = \frac{2^{4k+1}-1}{8k+4}B_{4k+2}.
  \]
\end{enumerate}



\emph{Proof of (a).}
\begin{enumerate}
\item[(1)]
  Consider the general case.
  \emph{Let
  \[
    G(x) = \sum_{n=1}^{\infty} \frac{n^{4k+1} x^n}{1-x^n},
  \]
  and let
  \[
    F(x) = \sum_{\substack{n=1 \\ (\text{$n$ odd})}}^{\infty} \frac{n^{4k+1} x^n}{1+x^n}.
  \]
  Show that $F(x) = G(x) - (2^{4k+1} + 2)G(x^2) + 2^{4k+2} G(x^4)$.}

\item[(2)]
  The identity
  \[
    \sum_{n=1}^{\infty} \frac{x^n}{1+x^n}
    = \sum_{n=1}^{\infty} \frac{x^n}{1-x^n}
      - 2 \sum_{n=1}^{\infty} \frac{x^{2n}}{1-x^{2n}}
  \]
  is always true.
  Hence $H(x) := \sum_{n=1}^{\infty} \frac{n^{4k+1} x^n}{1+x^n} = G(x) - 2G(x^2)$.

\item[(3)]
  Note that
  \begin{align*}
    H(x)
    &=
    \sum_{\substack{n=1 \\ (\text{$n$ odd})}}^{\infty} \frac{n^{4k+1} x^n}{1+x^n}
    + \sum_{\substack{n=1 \\ (\text{$n$ even})}}^{\infty} \frac{n^{4k+1} x^n}{1+x^n} \\
    &=
    F(x) + \sum_{n=1}^{\infty} \frac{(2n)^{4k+1} x^{2n}}{1+x^{2n}} \\
    &=
    F(x) + 2^{4k+1} \sum_{n=1}^{\infty} \frac{n^{4k+1} x^{2n}}{1+x^{2n}} \\
    &=
    F(x) + 2^{4k+1}H(x^2).
  \end{align*}
  Hence
  \begin{align*}
    F(x)
    &= H(x) - 2^{4k+1}H(x^2) \\
    &= [G(x) - 2G(x^2)] - 2^{4k+1}[G(x^2) - 2G(x^4)] \\
    &= G(x) - (2^{4k+1} + 2)G(x^2) + 2^{4k+2}G(x^4).
  \end{align*}
\end{enumerate}
$\Box$ \\



\emph{Proof of (b).}
  Take $k = 1$ in part (c), we have
  \[
    \sum_{\substack{n=1 \\ (\text{$n$ odd})}}^{\infty} \frac{n^5}{1+e^{n\pi}}
    = \frac{31}{12} \cdot \frac{1}{42}
    = \frac{31}{504}.
  \]
$\Box$ \\



\emph{Proof of (c).}
\begin{enumerate}
\item[(1)]
  Let $q = e^{2 \pi i \tau}$.
  So
  \begin{align*}
    G_{4k+2}(\tau)
    &=
    2 \zeta(4k+2)
      + \frac{2(2\pi i)^{4k+2}}{(4k+1)!} \sum_{n=1}^{\infty} \sigma_{4k+1}(n) q^n
      &(\text{Exercise 1.11}) \\
    &=
    2 \zeta(4k+2)
      + \frac{2(2\pi i)^{4k+2}}{(4k+1)!} G(q)
      &(\text{Exercise 1.14(c)})
  \end{align*}
  Hence
  \begin{align*}
    & \:
    G_{4k+2}(\tau) - (2^{4k+1} + 2) G_{4k+2}(2\tau) + 2^{4k+2} G_{4k+2}(4\tau) \\
    = &\:
      \left[ 2 \zeta(4k+2)
        + \frac{2(2\pi i)^{4k+2}}{(4k+1)!} G(q) \right] \\
    &\:
      - (2^{4k+1} + 2) \left[ 2 \zeta(4k+2)
        + \frac{2(2\pi i)^{4k+2}}{(4k+1)!} G(q^2) \right] \\
    &\:
      + 2^{4k+2} \left[ 2 \zeta(4k+2)
        + \frac{2(2\pi i)^{4k+2}}{(4k+1)!} G(q^4) \right] \\
    = &\:
    (1 - (2^{4k+1} + 2) + 2^{4k+2}) \cdot 2\zeta(4k+2) \\
    &\:
      + \frac{2(2\pi i)^{4k+2}}{(4k+1)!}
        [G(q) - (2^{4k+1} + 2)G(q^2) + 2^{4k+2}G(q^4)] \\
    = &\:
    (2^{4k+2} - 2) \zeta(4k+2)
      + \frac{2(2\pi i)^{4k+2}}{(4k+1)!} F(q).
  \end{align*}

\item[(2)]
  By taking $\tau = \frac{i}{2}$, we have
  \[
    F(q)
    = F(e^{-\pi})
    = \sum_{\substack{n=1 \\ (\text{$n$ odd})}}^{\infty} \frac{n^{4k+1}}{1+e^{n\pi}}
  \]
  and
  \begin{align*}
    & \:
    G_{4k+2}(\tau) - (2^{4k+1} + 2) G_{4k+2}(2\tau) + 2^{4k+2} G_{4k+2}(4\tau) \\
    = &\:
    G_{4k+2}\left( \frac{i}{2} \right)
      - (2^{4k+1} + 2) G_{4k+2}(i)
      + 2^{4k+2} G_{4k+2}(2i) \\
    = &\:
    (-4)^{2k+1} G_{4k+2}(2i) - (2^{4k+1} + 2) \cdot 0 + 2^{4k+2} G_{4k+2}(2i) \\
    = &\:
    0.
  \end{align*}
  (Exercise 1.12).
  Hence
  \[
    0 = (2^{4k+2} - 2) \zeta(4k+2)
    + \frac{2(2\pi i)^{4k+2}}{(4k+1)!}
    \sum_{\substack{n=1 \\ (\text{$n$ odd})}}^{\infty} \frac{n^{4k+1}}{1+e^{n\pi}}.
  \]

\item[(3)]
  Theorem 12.17 in the textbook: \emph{T. M. Apostol, Introduction to Analytic Number Theory}
  shows that
  \[
    \zeta(4k+2)
    = (-1)^{2k+1+1} \frac{(2\pi)^{4k+2} B_{4k+2}}{2(4k+2)!}
    = \frac{(2\pi)^{4k+2} B_{4k+2}}{2(4k+2)!}.
  \]
  Hence
  \[
    \sum_{\substack{n=1 \\ (\text{$n$ odd})}}^{\infty} \frac{n^{4k+1}}{1+e^{n\pi}}
    = \frac{2^{4k+1}-1}{8k+4}B_{4k+2}.
  \]
\end{enumerate}
$\Box$ \\\\



%%%%%%%%%%%%%%%%%%%%%%%%%%%%%%%%%%%%%%%%%%%%%%%%%%%%%%%%%%%%%%%%%%%%%%%%%%%%%%%%
%%%%%%%%%%%%%%%%%%%%%%%%%%%%%%%%%%%%%%%%%%%%%%%%%%%%%%%%%%%%%%%%%%%%%%%%%%%%%%%%



\end{document}