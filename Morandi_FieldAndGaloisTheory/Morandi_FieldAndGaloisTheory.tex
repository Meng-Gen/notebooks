\documentclass{article}
\usepackage{amsfonts}
\usepackage{amsmath}
\usepackage{amssymb}
\usepackage{centernot}
\usepackage{hyperref}
\usepackage[none]{hyphenat}
\usepackage{mathrsfs}
\usepackage{mathtools}
\usepackage{physics}
\usepackage{tikz-cd}
\parindent=0pt



\title{\textbf{Notes on the book: \\ \emph{Patrick Morandi, Field and Galois Theory}}}
\author{Meng-Gen Tsai \\ plover@gmail.com}



\begin{document}
\maketitle
\tableofcontents



%%%%%%%%%%%%%%%%%%%%%%%%%%%%%%%%%%%%%%%%%%%%%%%%%%%%%%%%%%%%%%%%%%%%%%%%%%%%%%%%
%%%%%%%%%%%%%%%%%%%%%%%%%%%%%%%%%%%%%%%%%%%%%%%%%%%%%%%%%%%%%%%%%%%%%%%%%%%%%%%%
%%%%%%%%%%%%%%%%%%%%%%%%%%%%%%%%%%%%%%%%%%%%%%%%%%%%%%%%%%%%%%%%%%%%%%%%%%%%%%%%
%%%%%%%%%%%%%%%%%%%%%%%%%%%%%%%%%%%%%%%%%%%%%%%%%%%%%%%%%%%%%%%%%%%%%%%%%%%%%%%%



% Reference:



%%%%%%%%%%%%%%%%%%%%%%%%%%%%%%%%%%%%%%%%%%%%%%%%%%%%%%%%%%%%%%%%%%%%%%%%%%%%%%%%
%%%%%%%%%%%%%%%%%%%%%%%%%%%%%%%%%%%%%%%%%%%%%%%%%%%%%%%%%%%%%%%%%%%%%%%%%%%%%%%%
%%%%%%%%%%%%%%%%%%%%%%%%%%%%%%%%%%%%%%%%%%%%%%%%%%%%%%%%%%%%%%%%%%%%%%%%%%%%%%%%
%%%%%%%%%%%%%%%%%%%%%%%%%%%%%%%%%%%%%%%%%%%%%%%%%%%%%%%%%%%%%%%%%%%%%%%%%%%%%%%%



\newpage
\section*{I. Galois Theory \\}
\addcontentsline{toc}{section}{I. Galois Theory}



\subsection*{\S 1. Field Extensions \\}
\addcontentsline{toc}{subsection}{\S 1. Field Extensions}



\subsubsection*{Problem 1.1.}
\addcontentsline{toc}{subsubsection}{Problem 1.1.}
\emph{Let $K$ be a field extension of $F$.
By defining scalar multiplication for $\alpha \in F$ and $a \in K$
by $\alpha \cdot a = \alpha a$, the multiplication in $K$,
show that $K$ is an $F$-vector space.} \\



\emph{Proof.}
\begin{enumerate}
\item[(1)]
$K$ is an additive group.
\item[(2)]
\emph{Show that $(\alpha \beta) \cdot a = \alpha \cdot (\beta \cdot a)$
for $\alpha, \beta \in F$ and $a \in K$.}
In fact,
\begin{align*}
(\alpha \beta) \cdot a
&= \alpha \beta a \in K, \\
\alpha \cdot (\beta \cdot a)
&= \alpha \beta a \in K.
\end{align*}
\item[(3)]
\emph{Show that $(\alpha + \beta) \cdot a = \alpha \cdot a + \beta \cdot a$
for $\alpha, \beta \in F$ and $a \in K$.}
\begin{align*}
(\alpha + \beta) \cdot a
&= (\alpha + \beta) a \\
&= \alpha a + \beta a \in K, \\
\alpha \cdot a + \beta \cdot a
&= \alpha a + \beta a \in K.
\end{align*}
\item[(4)]
\emph{Show that $\alpha \cdot (a + b) = \alpha \cdot a + \alpha \cdot b$
for $\alpha \in F$ and $a, b \in K$.}
\begin{align*}
\alpha \cdot (a + b)
&= \alpha (a + b) \\
&= \alpha a + \alpha b \in K, \\
\alpha \cdot a + \alpha \cdot b
&= \alpha a + \alpha b \in K.
\end{align*}
\item[(5)]
\emph{Show that $1 \cdot a = a$
for $a \in K$.}
$1 \cdot a = 1 a = a \in K$.
\end{enumerate}
By (1) to (5), $K$ is an $F$-vector space.
$\Box$ \\\\



%%%%%%%%%%%%%%%%%%%%%%%%%%%%%%%%%%%%%%%%%%%%%%%%%%%%%%%%%%%%%%%%%%%%%%%%%%%%%%%%



\subsubsection*{Problem 1.2.}
\addcontentsline{toc}{subsubsection}{Problem 1.2.}
\emph{If $K$ is a field extension of $F$, prove that $[K:F] = 1$
if and only if $K = F$.} \\

\emph{Proof.}
\begin{enumerate}
\item[(1)]
\emph{$[K:F] = 1 \Longleftarrow K = F$.}
Take a basis $\{1\}$ for $K$ as an $F$-vector space.
\item[(2)]
\emph{$[K:F] = 1 \Longrightarrow K = F$.}
Take a basis $\{a\}$ for $K$ as an $F$-vector space where $a \in K$.
Since $1 \in K$ as an $F$-vector space,
there exists $\alpha \in F$ such that $1 = \alpha a$.
$a = \alpha^{-1} \in F$, or $K \subseteq F$, or $K = F$.
\end{enumerate}
$\Box$ \\\\



%%%%%%%%%%%%%%%%%%%%%%%%%%%%%%%%%%%%%%%%%%%%%%%%%%%%%%%%%%%%%%%%%%%%%%%%%%%%%%%%



\subsubsection*{Problem 1.3.}
\addcontentsline{toc}{subsubsection}{Problem 1.3.}
\emph{Let $K$ be a field extension of $F$, and let $a \in K$.
Show that the evaluation map $\text{ev}_a: F[x] \to K$ given by
$\text{ev}_a(f(x)) = f(a)$ is a ring and and $F$-vector space homomorphism.
(Such a map is called an $F$-algebra homomorphism.) } \\

\emph{Proof.}
\begin{enumerate}
\item[(1)]
\emph{$\text{ev}_a$ is a ring homomorphism.}
  \begin{enumerate}
  \item[(a)]
  $\text{ev}_a(f(x)+g(x)) = f(a) + g(a) = \text{ev}_a(f(x)) + \text{ev}_a(g(x))$.
  \item[(b)]
  $\text{ev}_a(f(x)g(x)) = g(a)g(b) = \text{ev}_a(f(x)) \text{ev}_a(g(x))$.
  \item[(c)]
  $\text{ev}_a(1) = 1$.
  \end{enumerate}
\item[(2)]
\emph{$\text{ev}_a$ is an $F$-vector space homomorphism.}
  \begin{enumerate}
  \item[(a)]
  $\text{ev}_a(f(x)+g(x)) = f(a) + g(a) = \text{ev}_a(f(x)) + \text{ev}_a(g(x))$.
  \item[(b)]
  Given $c \in F$, $\text{ev}_a(cf(x)) = cf(a) = c\text{ev}_a(f(x))$.
  \end{enumerate}
\end{enumerate}
$\Box$ \\\\


%%%%%%%%%%%%%%%%%%%%%%%%%%%%%%%%%%%%%%%%%%%%%%%%%%%%%%%%%%%%%%%%%%%%%%%%%%%%%%%%



\subsubsection*{Problem 1.4.}
\addcontentsline{toc}{subsubsection}{Problem 1.4.}
\emph{Prove Proposition 1.9:
Let $K$ be a field extension of $F$ and let $a_1, \ldots, a_n \in K$.
Then
$$F[a_1, \ldots, a_n] = \{ f(a_1, \ldots, a_n) : f \in F[x_1, \ldots, x_n] \}$$
and
$$F(a_1, \ldots, a_n) = \left\{
\frac{f(a_1, \ldots, a_n)}{g(a_1, \ldots, a_n)} : f, g \in F[x_1, \ldots, x_n],
g(a_1, \ldots, a_n) \neq 0
\right\},$$
so $F(a_1, \ldots, a_n)$ is the quotient field of $F[x_1, \ldots, x_n]$.
} \\

\emph{Proof (Proposition 1.8).}
\begin{enumerate}
\item[(1)]
The evaluation map
$\text{ev}_{(a_1, \ldots, a_n)}: F[x_1, \ldots, x_n] \to K$ has image
$$\{ f(a_1, \ldots, a_n) : f \in F[x_1, \ldots, x_n] \},$$
so this set is a subring of $K$.
\item[(2)]
If $R$ is a subring of $K$ that contains $F$ and $a_1, \ldots, a_n$,
then $$f(a_1, \ldots, a_n) \in R$$ for any $f(x_1, \ldots, x_n) \in F[x_1, \ldots, x_n]$
by closure of addition and multiplication.
\item[(3)]
So
$\{ f(a_1, \ldots, a_n) : f \in F[x_1, \ldots, x_n] \}$ is contained in all subrings
of $K$ that contains $F$ and $a_1, \ldots, a_n$.
Hence
$$F[a_1, \ldots, a_n] = \{ f(a_1, \ldots, a_n) : f \in F[x_1, \ldots, x_n] \}.$$
\item[(4)]
The quotient field of $F[a_1, \ldots, a_n]$ is then the set
$$\left\{
\frac{f(a_1, \ldots, a_n)}{g(a_1, \ldots, a_n)} : f, g \in F[x_1, \ldots, x_n],
g(a_1, \ldots, a_n) \neq 0
\right\}.$$
It is clearly is contained in any subfield of $K$ that contains $F[a_1, \ldots, a_n]$;
hence, it is equal to $F(a_1, \ldots, a_n)$.
\end{enumerate}
$\Box$ \\\\



%%%%%%%%%%%%%%%%%%%%%%%%%%%%%%%%%%%%%%%%%%%%%%%%%%%%%%%%%%%%%%%%%%%%%%%%%%%%%%%%



\subsubsection*{Problem 1.5.}
\addcontentsline{toc}{subsubsection}{Problem 1.5.}
\emph{Show that 
$\mathbb{Q}(\sqrt{5}, \sqrt{7}) = \mathbb{Q}(\sqrt{5} + \sqrt{7})$.} \\

\emph{Proof.}
\begin{enumerate}
\item[(1)]
$\mathbb{Q}(\sqrt{5}, \sqrt{7}) \supseteq \mathbb{Q}(\sqrt{5} + \sqrt{7})$
since $\sqrt{5} + \sqrt{7} \in \mathbb{Q}(\sqrt{5}, \sqrt{7})$.
\item[(2)]
\begin{align*}
(\sqrt{7} + \sqrt{5})^{-1}
&= \frac{1}{\sqrt{7} + \sqrt{5}} \\
&= \frac{\sqrt{7} - \sqrt{5}}{(\sqrt{7} + \sqrt{5})(\sqrt{7} - \sqrt{5})} \\
&= \frac{\sqrt{7} - \sqrt{5}}{2} \in \mathbb{Q}(\sqrt{5} + \sqrt{7}),
\end{align*}
Or $\sqrt{7} - \sqrt{5} \in \mathbb{Q}(\sqrt{5} + \sqrt{7})$. Thus
\begin{align*}
\sqrt{7}
&= \frac{1}{2} \cdot ((\sqrt{7} + \sqrt{5}) + (\sqrt{7} - \sqrt{5}))
\in \mathbb{Q}(\sqrt{5} + \sqrt{7}), \\
\sqrt{5}
&= \frac{1}{2} \cdot ((\sqrt{7} + \sqrt{5}) - (\sqrt{7} - \sqrt{5}))
\in \mathbb{Q}(\sqrt{5} + \sqrt{7}).
\end{align*}
Thus, $\mathbb{Q}(\sqrt{5}, \sqrt{7}) \subseteq \mathbb{Q}(\sqrt{5} + \sqrt{7})$.
\end{enumerate}
By (1)(2), $\mathbb{Q}(\sqrt{5}, \sqrt{7}) = \mathbb{Q}(\sqrt{5} + \sqrt{7})$.
$\Box$ \\\\



%%%%%%%%%%%%%%%%%%%%%%%%%%%%%%%%%%%%%%%%%%%%%%%%%%%%%%%%%%%%%%%%%%%%%%%%%%%%%%%%



\subsubsection*{Problem 1.9.}
\addcontentsline{toc}{subsubsection}{Problem 1.9.}
\emph{If $K$ is an extension of $F$ such that $[K:F]$ is prime,
show that there are no intermediate fields between $K$ and $F$.} \\

\emph{Proof.}
Let $L$ be any field such that $F \subseteq L \subseteq K$.
By Proposition 1.20,
$$[K:F] = [K:L][L:F].$$
Since $[K:F]$ is prime, $[K:L] = 1$ or $[L:F] = 1$.
By Problem 1.2, $L=K$ or $L=F$,
or there are no intermediate fields between $K$ and $F$.
$\Box$ \\\\



%%%%%%%%%%%%%%%%%%%%%%%%%%%%%%%%%%%%%%%%%%%%%%%%%%%%%%%%%%%%%%%%%%%%%%%%%%%%%%%%



\subsubsection*{Problem 1.11.}
\addcontentsline{toc}{subsubsection}{Problem 1.11.}
\emph{If $K$ is an algebraic extension of $F$ and if $R$ is a subring of $K$ with
$F \subseteq R \subseteq K$, show that $R$ is a field.} \\



\emph{Proof.}
\begin{enumerate}
\item[(1)]
  $R$ is a domain since $R$ is contained in a field $K$.
  To show $R$ is a field, it suffices to show that
  every nonzero element $\alpha \in R$ has an inverse in $R$.

\item[(2)]
  Since $\alpha \in R \subseteq K$ is algebraic over $F$,
  there is a minimal polynomial
  \[
    f(x) = b_n x^n + b_{n-1} x^{n-1} + \cdots + b_0
  \]
  such that $f(\alpha) = 0$,
  where each $b_i \in F$ and $b_0 \neq 0$ by the minimality of $f$.

\item[(3)]
  Note that
  \begin{align*}
    & \:
    f(\alpha) = 0 \\
    \Longleftrightarrow &\:
    b_n \alpha^n + b_{n-1} \alpha^{n-1} + \cdots + b_0 = 0 \\
    \Longleftrightarrow &\:
    b_n \alpha^n + b_{n-1} \alpha^{n-1} + \cdots + b_1 \alpha = -b_0 \\
    \Longleftrightarrow &\:
    \alpha(b_n \alpha^{n-1} + b_{n-1} \alpha^{n-2} + \cdots + b_1) = -b_0 \\
    \Longleftrightarrow &\:
    \alpha(\underbrace{(-b_0)^{-1} b_n \alpha^{n-1} + (-b_0)^{-1}b_{n-1} \alpha^{n-2} + \cdots
      + (-b_0)^{-1}b_1}_{:= \alpha'}) = 1.
  \end{align*}
  Hence $\alpha' \in F[\alpha] \subseteq R$.
  Therefore $\alpha'$ is the inverse of $\alpha$ in $R$.
\end{enumerate}
$\Box$ \\\\



%%%%%%%%%%%%%%%%%%%%%%%%%%%%%%%%%%%%%%%%%%%%%%%%%%%%%%%%%%%%%%%%%%%%%%%%%%%%%%%%



\subsubsection*{Problem 1.12.}
\addcontentsline{toc}{subsubsection}{Problem 1.12.}
\emph{Show that $\mathbb{Q}(\sqrt{2})$ and $\mathbb{Q}(\sqrt{3})$
are not isomorphic as fields but are isomorphic as vector spaces over $\mathbb{Q}$.} \\

\emph{Proof.}
\begin{enumerate}
\item[(1)]
\emph{Show that $\mathbb{Q}(\sqrt{2})$ and $\mathbb{Q}(\sqrt{3})$
are not isomorphic as fields.}
(Reductio ad absurdum)
If $\varphi: \mathbb{Q}(\sqrt{2}) \to \mathbb{Q}(\sqrt{3})$ were an isomorphism
as fields, then $\varphi$ is an identity map on $\mathbb{Q}$, and
\begin{align*}
&\varphi(\sqrt{2}) = a + b\sqrt{3} \text{ for some } a, b \in \mathbb{Q} \\
\Longrightarrow&
\varphi(\sqrt{2})\varphi(\sqrt{2}) = (a + b\sqrt{3})^2 \\
\Longrightarrow&
\varphi(\sqrt{2} \sqrt{2}) = (a + b\sqrt{3})^2 \\
\Longrightarrow&
\varphi(2) = a^2 + 3b^2 + 2ab\sqrt{3} \\
\Longrightarrow&
2 = a^2 + 3b^2 + 2ab\sqrt{3}.
\end{align*}
If $2ab \neq 0$, then $\sqrt{3} = \frac{2-a^2-3b^2}{2ab} \in \mathbb{Q}$,
which is absurd.
Hence $2ab = 0$.
  \begin{enumerate}
  \item[(a)]
  $a = 0$.
  Write $b = \frac{m}{n} \in \mathbb{Q}$ where $m, n \in \mathbb{Z}$ and $(m, n) = 1$.
  Hence
  $$2n^2 = 3m^2.$$
  So $2 \mid 3m^2$, $2 \mid m^2$, $2 \mid m$. So $4 \mid 2n^2$, $2 \mid n^2$, $2 \mid n$.
  Hence $2 \mid (m,n)$, contrary to the assumption that $(m,n) = 1$.
  \item[(b)]
  $b = 0$.
  $2 = a^2$.
  Write $a = \frac{m}{n} \in \mathbb{Q}$ where $m, n \in \mathbb{Z}$ and $(m, n) = 1$.
  Similar to the argument in (a), we will reach a contradiction.
  \end{enumerate}
By (a)(b), no such isomorphism $\varphi$, that is,
$\mathbb{Q}(\sqrt{2})$ and $\mathbb{Q}(\sqrt{3})$
are not isomorphic as fields.
\item[(2)]
\emph{Show that $\mathbb{Q}(\sqrt{2})$ and $\mathbb{Q}(\sqrt{3})$
are isomorphic as $\mathbb{Q}$-vector spaces.}
$[\mathbb{Q}(\sqrt{2}):\mathbb{Q}] = [\mathbb{Q}(\sqrt{3}):\mathbb{Q}] = 2$.
There is a natural map $\varphi: \mathbb{Q}(\sqrt{2}) \to \mathbb{Q}(\sqrt{3})$
defined by $\varphi(a + b\sqrt{2}) = a + b\sqrt{3}$.
Clearly $\varphi$ is well-defined, linear, injective and surjective.
\end{enumerate}
$\Box$ \\\\



%%%%%%%%%%%%%%%%%%%%%%%%%%%%%%%%%%%%%%%%%%%%%%%%%%%%%%%%%%%%%%%%%%%%%%%%%%%%%%%%



\subsubsection*{Problem 1.16.}
\addcontentsline{toc}{subsubsection}{Problem 1.16.}
\emph{Let $\mathbb{A}$ be the algebraic closure of $\mathbb{Q}$ in $\mathbb{C}$.
Prove that $[\mathbb{A}:\mathbb{Q}] = \infty$. } \\

\emph{Proof (Example 1.16).}
By Example 1.16, $[\mathbb{Q}(\sqrt[n]{2}):\mathbb{Q}] = n.$
Therefore,
$$[\mathbb{A}:\mathbb{Q}]
= [\mathbb{A}:\mathbb{Q}(\sqrt[n]{2})][\mathbb{Q}(\sqrt[n]{2}):\mathbb{Q}]
= [\mathbb{A}:\mathbb{Q}(\sqrt[n]{2})]n$$
for arbitrary $n \in \mathbb{Z}^+$.
Hence $[\mathbb{A}:\mathbb{Q}] = \infty$.
$\Box$ \\

\emph{Proof (Example 1.16).}
Given a prime number $p$.
By Example 1.16, $[\mathbb{Q}(\rho):\mathbb{Q}] = p-1$
where $\rho = \exp(2\pi i/p)$.
Therefore,
$$[\mathbb{A}:\mathbb{Q}]
= [\mathbb{A}:\mathbb{Q}(\rho)][\mathbb{Q}(\rho):\mathbb{Q}]
= [\mathbb{A}:\mathbb{Q}(\rho)](p-1)$$
for arbitrary prime $p$.
Hence $[\mathbb{A}:\mathbb{Q}] = \infty$.
$\Box$ \\\\



%%%%%%%%%%%%%%%%%%%%%%%%%%%%%%%%%%%%%%%%%%%%%%%%%%%%%%%%%%%%%%%%%%%%%%%%%%%%%%%%



\subsubsection*{Problem 1.23.}
\addcontentsline{toc}{subsubsection}{Problem 1.23.}
\emph{Recall that the characteristic of a ring $R$ with identity
is the smallest positive integer $n$ for which $n \cdot 1 = 0$,
if such an $n$ exists, or else the characteristic is $0$.
Let $R$ be a ring with identity.
Define $\varphi: \mathbb{Z} \rightarrow R$ by $\varphi(n) = n \cdot 1$,
where $1$ is the identity of $R$.
Show that $\varphi$ is a ring homomorphism
and that $\ker(\varphi) = m\mathbb{Z}$ for a unique nonnegative integer $m$,
and show that $m$ is the characteristic of $R$.} \\

\emph{Proof.}
\begin{enumerate}
\item[(1)]
  \emph{$\varphi$ is a ring homomorphism.}
  \begin{enumerate}
  \item[(a)]
    \emph{$\varphi(a+b) = \varphi(a) + \varphi(b)$.}
    $\varphi(a+b)
    = (a+b) \cdot 1
    = a \cdot 1 + b \cdot 1
    = \varphi(a) + \varphi(b)$.
  \item[(b)]
    \emph{$\varphi(ab) = \varphi(a) \varphi(b)$.}
    $\varphi(ab)
    = (ab) \cdot 1
    = (a \cdot 1)(b \cdot 1)
    = \varphi(a) \varphi(b)$
    since $1 \times 1 = 1$. (Here $\times$ is the multiplication operator of $R$.)
  \end{enumerate}
\item[(2)]
  \emph{$\ker(\varphi) = m\mathbb{Z}$ for a unique nonnegative integer $m$.}
  Since $\ker(\varphi)$ is an ideal of a PID $\mathbb{Z}$,
  there is a unique nonnegative integer $m$
  such that $\ker(\varphi) = m\mathbb{Z}$.
\item[(3)]
  \emph{$m$ is the characteristic of $R$.}
  There are only two possible cases,
  $\text{char}(R) = 0$ or else $\text{char}(R) > 0$.
  \begin{enumerate}
  \item[(a)]
    \emph{$\text{char}(R) = 0$.}
    $\ker(\varphi) = 0$. Thus $m = 0 = \text{char}(R)$.
  \item[(b)]
    \emph{$\text{char}(R) = n > 0$.} $n \in \ker(\varphi)$,
    so $m > 0$ and $m \mid n$.
    By the minimality of $n$, $m = n = \text{char}(R)$.
  \end{enumerate}
\end{enumerate}
$\Box$ \\\\



%%%%%%%%%%%%%%%%%%%%%%%%%%%%%%%%%%%%%%%%%%%%%%%%%%%%%%%%%%%%%%%%%%%%%%%%%%%%%%%%



\subsubsection*{Problem 1.24.}
\addcontentsline{toc}{subsubsection}{Problem 1.24.}
\emph{For any positive integer $n$,
give an example of a ring of characteristic $n$.} \\

\emph{Proof.}
The ring $\mathbb{Z}/n\mathbb{Z}$.
$\Box$ \\\\



%%%%%%%%%%%%%%%%%%%%%%%%%%%%%%%%%%%%%%%%%%%%%%%%%%%%%%%%%%%%%%%%%%%%%%%%%%%%%%%%



\subsubsection*{Problem 1.25.}
\addcontentsline{toc}{subsubsection}{Problem 1.25.}
\emph{If $R$ is an integral domain, show that either
$\text{char}(R) = 0$ or $\text{char}(R)$ is prime.} \\

\emph{Proof.}
\begin{enumerate}
\item[(1)]
  $1$ has infinite order. $\text{char}(R) = 0$. (Nothing to do.)
\item[(2)]
  $1$ has finite order $n$.
  Want to show $n$ is prime.
  If $n = ab$ where $a, b \in \mathbb{Z}^+$,
  then $$0 = n \cdot 1 = (a \cdot 1)(b \cdot 1).$$
  Since $R$ is an integral domain, $a \cdot 1 = $ or $b \cdot 1 = 0$.
  By the minimality of $n$, $a \geq n$ or $b \geq n$.
  $a = n$ or $b = n$. That is, $n$ is prime.
\end{enumerate}
$\Box$ \\\\



%%%%%%%%%%%%%%%%%%%%%%%%%%%%%%%%%%%%%%%%%%%%%%%%%%%%%%%%%%%%%%%%%%%%%%%%%%%%%%%%
%%%%%%%%%%%%%%%%%%%%%%%%%%%%%%%%%%%%%%%%%%%%%%%%%%%%%%%%%%%%%%%%%%%%%%%%%%%%%%%%



\subsection*{\S 2. Automorphisms \\}
\addcontentsline{toc}{subsection}{\S 2. Automorphisms}



\subsubsection*{Problem 2.1.}
\addcontentsline{toc}{subsubsection}{Problem 2.1.}
\emph{Show that the only automorphism of $\mathbb{Q}$ is the identity.} \\

\emph{Proof.}
Given any $\sigma \in \text{Aut}(\mathbb{Q})$.
\begin{enumerate}
\item[(1)]
\emph{Show that $\sigma(1) = 1$.}
Since $1^2 = 1$, $\sigma(1)\sigma(1) = \sigma(1)$. $\sigma(1) = 0$ or $1$.
There are only two possible cases.
  \begin{enumerate}
  \item[(a)]
  Assume that $\sigma(1) = 0$. So
  $$\sigma(a) = \sigma(a \cdot 1) = \sigma(a)\cdot \sigma(1) = \sigma(a) \cdot 0 = 0$$
  for any $a \in \mathbb{Q}$.
  That is, $\sigma = 0 \in \text{Aut}(\mathbb{Q})$, which is absurd.
  \item[(b)]
  Therefore, $\sigma(1) = 1$.
  \end{enumerate}
\item[(2)]
\emph{Show that $\sigma(n) = n$ for all $n \in \mathbb{Z}^+$.}
Write $n = 1 + 1 + \cdots + 1$ ($n$ times $1$).
Applying the additivity of $\sigma$, we have
$$\sigma(n) = \sigma(1) + \sigma(1) + \cdots + \sigma(1) = 1 + 1 + \cdots + 1 = n.$$
(Might use induction on $n$ to eliminate $\cdots$ symbols.)
\item[(3)]
\emph{Show that $\sigma(n) = n$ for all $n \in \mathbb{Z}$.}
By the additivity of $\sigma$, $\sigma(-n) = -\sigma(n) = -n$ for $n \geq 0$.
The result is established.
\end{enumerate}
For any $a = \frac{n}{m} \in \mathbb{Q}$ ($m, n \in \mathbb{Z}$, $n \neq 0$),
applying the multiplication of $\sigma$ on $am = n$,
that is,
$\sigma(a) \sigma(m) = \sigma(n)$. By (3), we have $\sigma(a)m = n$,
or $$\sigma(a) = \frac{m}{n} = a$$
provided $n \neq 0$,
or $\sigma$ is the identity.
$\Box$ \\\\



%%%%%%%%%%%%%%%%%%%%%%%%%%%%%%%%%%%%%%%%%%%%%%%%%%%%%%%%%%%%%%%%%%%%%%%%%%%%%%%%



\subsubsection*{Problem 2.2.}
\addcontentsline{toc}{subsubsection}{Problem 2.2.}
\emph{Show that the only automorphism of $\mathbb{R}$ is the identity.
(Hint: If $\sigma$ is an automorphism, show that $\sigma|_{\mathbb{Q}} = \text{id}$,
and if $a > 0$, then $\sigma(a) > 0$.
It is an interesting fact that there are infinitely many automorphisms of $\mathbb{C}$,
even thought $[\mathbb{C}:\mathbb{R}] = 2$.
Why is this fact not a contradiction to this problem?)} \\

\emph{Proof (Hint).}
Given any $\sigma \in \text{Aut}(\mathbb{R})$.
\begin{enumerate}
\item[(1)]
Apply the same argument in Problem 2.1, we have $\sigma|_{\mathbb{Q}} = \text{id}$.
Notice that $\sigma(a) \neq 0$ for any $a \neq 0$.
\item[(2)]
\emph{Show that $\sigma(a) > 0$ if $a > 0$.}
Given any $a > 0$.
Write $a = \sqrt{a}\sqrt{a}$ (well-defined) and then apply
$\sigma$ on the both sides,
$$\sigma(a) = \sigma(\sqrt{a})\sigma(\sqrt{a}) = \sigma(\sqrt{a})^2 > 0$$
(since $\sqrt{a} \neq 0$ and thus $\sigma(\sqrt{a})$ cannot be zero).
\item[(3)]
\emph{Show that $\sigma(a) > \sigma(b)$ if $a > b$.}
It is a corollary to (2) by applying $\sigma$ on $a - b > 0$.
($\sigma(a - b) > 0$, or $\sigma(a) - \sigma(b) > 0$, or $\sigma(a) > \sigma(b)$.)
\item[(4)]
For any real number $x \in \mathbb{R}$,
choose two sequences $\{p_n\}, \{q_n\}$ of rational numbers
such that $p_n < x < q_n$ and $p_n, q_n \to x$ as $n \to \infty$.
Take $\sigma$ on the inequality, $\sigma(p_n) < \sigma(x) < \sigma(q_n)$.
So $p_n < \sigma(x) < q_n$ since $\sigma|_{\mathbb{Q}} = \text{id}$.
Let $n \to \infty$, we get $x \leq \sigma(x) \leq x$, or $\sigma(x) = x$.
\end{enumerate}
$\Box$ \\

\textbf{Supplement.}
Automorphisms of the Complex Numbers. by Paul B. Yale (Pomona College)
[\href{https://www.maa.org/sites/default/files/pdf/upload_library/22/Ford/PaulBYale.pdf}
{Link}]. \\\\



%%%%%%%%%%%%%%%%%%%%%%%%%%%%%%%%%%%%%%%%%%%%%%%%%%%%%%%%%%%%%%%%%%%%%%%%%%%%%%%%



\subsubsection*{Problem 2.4.}
\addcontentsline{toc}{subsubsection}{Problem 2.4.}
\emph{Let $B$ be an integral domain with quotient field $F$.
If $\sigma: B \to B$ is a ring automorphism, show that $\sigma$ induces
a ring automorphism $\sigma': F \to F$ defined by
$\sigma'(a/b) = \sigma(a)/\sigma(b)$ if $a, b \in B$ with $b \neq 0$.} \\

\emph{Proof.}
\begin{enumerate}
\item[(1)]
\emph{Show that $\sigma'$ is well-defined.}
  \begin{enumerate}
  \item[(a)]
  \emph{$\sigma': F \to F$ is defined.}
  $\sigma(a), \sigma(b) \in B$ since $\sigma$ is a homomorphism.
  $\sigma(b) \neq 0$ since $b \neq 0$ and $\sigma$ is a one-on-one homomorphism.
  \item[(b)]
  \emph{$\sigma'$ is independent of the representation of $a/b \in F$.}
  Suppose $a/b = c/d$ where $a, b, c, d \in B$ and $b, d \neq 0$.
  Hence,
  \begin{align*}
  a/b = c/d
  \Longleftrightarrow&
  ad = bc \\
  \Longleftrightarrow&
  \sigma(ad) = \sigma(bc) \\
  \Longleftrightarrow&
  \sigma(a)\sigma(d) = \sigma(b)\sigma(c)
    &\text{($\sigma$: homomorphism)} \\
  \Longleftrightarrow&
  \sigma(a)/\sigma(d) = \sigma(c)/\sigma(d)
    &(\sigma(b), \sigma(d) \neq 0) \\
  \Longleftrightarrow&
  \sigma'(a/b) = \sigma'(c/d).
  \end{align*}
  \end{enumerate}
\item[(2)]
\emph{Show that $\sigma'$ is a ring homomorphism.}
  \begin{enumerate}
  \item[(a)]
  \emph{Show that $\sigma'(a/b + c/d) = \sigma'(a/b) + \sigma'(c/d)$.}
  \begin{align*}
  \sigma'(a/b + c/d)
  &= \sigma'((ad+bc)/(bd)) \\
  &= \sigma(ad+bc) / \sigma(bd) \\
  &= (\sigma(a)\sigma(d)+\sigma(b)\sigma(c)) / (\sigma(b)\sigma(d))
    &\text{($\sigma$: homomorphism)} \\
  &= \sigma(a)/\sigma(b) + \sigma(c)/\sigma(d) \\
  &= \sigma'(a/b) + \sigma'(c/d).
  \end{align*}
  \item[(b)]
  \emph{Show that $\sigma'(a/b \cdot c/d) = \sigma'(a/b) \cdot \sigma'(c/d)$.}
  \begin{align*}
  \sigma'(a/b \cdot c/d)
  &= \sigma'((ac)/(bd)) \\
  &= \sigma(ac) / \sigma(bd) \\
  &= (\sigma(a)\sigma(c)) / (\sigma(b)\sigma(d))
    &\text{($\sigma$: homomorphism)} \\
  &= \sigma(a)/\sigma(b) \cdot \sigma(c)/\sigma(d) \\
  &= \sigma'(a/b) \cdot \sigma'(c/d).
  \end{align*}
  \end{enumerate}
\item[(3)]
\emph{Show that $\sigma'$ is injective.}
  \begin{align*}
  \sigma'(a/b) = 0
  \Longleftrightarrow&
  \sigma(a)/\sigma(b) = 0 \\
  \Longleftrightarrow&
  \sigma(a) = 0 \\
  \Longleftrightarrow&
   a = 0
    &\text{($\sigma$: injective)} \\
  \Longleftrightarrow&
  a/b = 0/b = 0 \in F
  \end{align*}
\item[(4)]
\emph{Show that $\sigma'$ is a surjective.}
Given any $c/d \in F$, want to show there is $a/b \in F$ such that $\sigma'(a/b) = c/d$.
  \begin{align*}
  c/d \in F
  \Longrightarrow&
  c, d \in B \\
  \Longrightarrow&
  \exists \: a, b \in B \text{ such that } \sigma(a)=c, \sigma(b)=d
    &\text{($\sigma$: surjective)} \\
  \Longrightarrow&
  \exists \: a, b \in B \text{ such that } \sigma(a) / \sigma(b) = c / d \\
  \Longrightarrow&
  \exists \: a, b \in B \text{ such that } \sigma'(a/b) = c/d.
  \end{align*}
\end{enumerate}



%%%%%%%%%%%%%%%%%%%%%%%%%%%%%%%%%%%%%%%%%%%%%%%%%%%%%%%%%%%%%%%%%%%%%%%%%%%%%%%%
%%%%%%%%%%%%%%%%%%%%%%%%%%%%%%%%%%%%%%%%%%%%%%%%%%%%%%%%%%%%%%%%%%%%%%%%%%%%%%%%



\newpage
\section*{II. Some Galois Extensions \\}
\addcontentsline{toc}{section}{II. Some Galois Extensions}



\subsection*{\S 10. Hilbert Theorem 90 and Group Cohomology \\}
\addcontentsline{toc}{subsection}{\S 10. Hilbert Theorem 90 and Group Cohomology}



\subsubsection*{Supplement.}
\addcontentsline{toc}{subsubsection}{Supplement.}
\begin{enumerate}
\item[(1)]
  Corollary 10.4 (Cohomological Hilbert Theorem 90).
  Let $K$ be a cyclic Galois extension of $F$.
  Then $H^{1}(\mathrm{Gal}(K/F), K^{\times}) = 0$.

\item[(2)]
  \emph{(Exercise 10.24 in the textbook:
  Rudin, Principles of Mathematical Analysis, 3rd edition.)
  Let $\omega = \sum a_i(\mathbf{x}) dx_i$ be a $1$-form of class $\mathcal{C}''$
  in a convex open set $E \subseteq \mathbb{R}^n$.
  Assume $d\omega = 0$ and prove that $\omega$ is exact in $E$.}
  Hence the first de Rham cohomology $H^{1}_{\mathrm{dR}}(E) = 0$.

\item[(3)]
  $H^{1}_{\mathrm{dR}}(E) = 0$ if $E$ is simply connected.
  (The converse is not true.)

\item[(4)]
  \emph{(Exercise 10.21 in the textbook:
  Rudin, Principles of Mathematical Analysis, 3rd edition.)
  Consider the $1$-form
  \[
    \eta = \frac{x dy - y dx}{x^2+y^2}
  \]
  in $\mathbb{R}^2-\{\mathbf{0}\}$.}
  \begin{enumerate}
  \item[(a)]
    \emph{Carry out the computation that leads to
    \[
      \int_{\gamma} \eta = 2\pi \neq 0,
    \]
    and prove that $d\eta = 0$.}

  \item[(b)]
    \emph{Let $\gamma(t) = (r \cos t, r \sin t)$, for some $r > 0$,
    and let $\Gamma$ be a $\mathcal{C}''$-curve in $\mathbb{R}^2 - \{\mathbf{0}\}$,
    with parameter interval $[0,2\pi]$,
    with $\Gamma(0) = \Gamma(2\pi)$,
    such that the intervals $[\gamma(t),\Gamma(t)]$ do not contain $\mathbf{0}$
    for any $t \in [0,2\pi]$.
    Prove that
    \[
      \int_{\Gamma} \eta = 2\pi.
    \]}

  \item[(c)]
    \emph{Take $\Gamma(t) = (a\cos t, b\sin t)$ where $a > 0$, $b > 0$ are fixed.
    Show that}
    \[
      \int_{0}^{2\pi} \frac{ab}{a^2\cos^2 t + b^2 \sin^2 t}dt = 2\pi.
    \]

  \item[(d)]
    \emph{Show that
    \[
      \eta = d\left( \arctan\frac{y}{x} \right)
    \]
    in any convex open set in which $x \neq 0$, and that
    \[
      \eta = d\left( -\arctan\frac{x}{y} \right)
    \]
    in any convex open set in which $y \neq 0$.
    Explain why this justifies the notation $\eta = d\theta$,
    in spite of the fact that $\eta$ is not exact in $\mathbb{R}^2 - \{0\}$.}
  \end{enumerate}

\item[(5)]
  \emph{(Exercise 10.22 in the textbook:
  Rudin, Principles of Mathematical Analysis, 3rd edition.)
  Define $\zeta$ in $\mathbb{R}^3-\{\mathbf{0}\}$ by
  \[
    \zeta = \frac{x dy \wedge dz + y dz \wedge dx + z dx \wedge dy}{r^3}
  \]
  where $r = (x^2+y^2+z^2)^{\frac{1}{2}}$,
  let $D$ be the rectangle given by $0 \leq u \leq \pi$, $0 \leq v \leq 2\pi$,
  and let $\Sigma$ be the $2$-surface in $\mathbb{R}^3$,
  with parameter domain $D$, given by
  \[
    x = \sin u \cos v,
    \qquad
    y = \sin u \sin v,
    \qquad
    z = \cos u.
  \]}
  \begin{enumerate}
  \item[(a)]
    \emph{Prove that $d\zeta = 0$ in $\mathbb{R}^3 - \{ \mathbf{0} \}$.}

  \item[(b)]
    \emph{Let $S$ denote the restriction of $\Sigma$ to a parameter domain $E \subseteq D$.
    Prove that
    \[
      \int_{S} \zeta
      = \int_{E} \sin u \: du \: dv
      = A(S),
    \]
    where $A$ denotes area, as in Section 10.46.
    Note that this contains
    \[
      \int_{\Sigma} \zeta
      = \int_{D} \sin u \: du \: dv
      = 4\pi \neq 0
    \]
    as a special case.}

  \item[(c)]
    \emph{Suppose $g, h_1, h_2, h_3$, are $\mathcal{C}''$-functions on $[0,1]$, $g > 0$.
    Let $(x,y,z) = \Phi(s,t)$ define a $2$-surface $\Phi$,
    with parameter domain $I^2$, by
    \[
      x = g(t)h_1(s),
      \qquad
      y = g(t)h_2(s),
      \qquad
      z = g(t)h_3(s).
    \]
    Prove that
    \[
      \int_{\Phi} \zeta = 0.
    \]
    Note the shape of the range of $\Phi$:
    For fixed $s$, $\Phi(s,t)$ runs over an interval on a line through $\mathbf{0}$.
    The range of $\Phi$ thus lies in a ``cone'' with vertex at the origin.}

  \item[(d)]
    \emph{Let $E$ be a closed rectangle in $D$, with edges parallel to those of $D$.
    Suppose $f \in \mathcal{C}''(D)$, $f > 0$.
    Let $\Omega$ be the $2$-surface with parameter domain $E$,
    defined by
    \[
      \Omega(u,v) = f(u,v)\Sigma(u,v).
    \]
    Define $S$ as in (b) and prove that}
    \[
      \int_{\Omega} \zeta = \int_{S} \zeta = A(S).
    \]

  \item[(e)]
    \emph{Put $\lambda = -\frac{z}{r}\eta$, where
    \[
      \eta = \frac{xdy-ydx}{x^2+y^2}.
    \]
    Then $\lambda$ is a $1$-form in the open set $V \subseteq \mathbb{R}^3$ in which $x^2+y^2 > 0$.
    Show that $\zeta$ is exact in $V$ by showing that}
    \[
      \zeta = d\lambda.
    \]

  \item[(f)]
    \emph{Is $\zeta$ exact in the complement of every line through the origin?}
  \end{enumerate}

\item[(6)]
  \emph{(Exercise 10.23 in the textbook:
  Rudin, Principles of Mathematical Analysis, 3rd edition.)
Fix $n$.
Define $r_k = (x_1^2+\cdots+x_k^2)^{\frac{1}{2}}$ for $1 \leq k \leq n$,
let $E_k$ be the set of all $\mathbf{x} \in \mathbb{R}^n$ at which $r_k > 0$,
and let $\omega_k$ be the $(k-1)$-form defined in $E_k$ by
\[
  \omega_k
  = (r_k)^{-k}
    \sum_{i=1}^{k} (-1)^{i-1} x_i
    dx_1 \wedge \cdots \wedge \widehat{dx_i} \wedge \cdots \wedge dx_k
\]
Note that $\omega_2 = \eta$, $\omega_3 = \zeta$ in the terminology of
Exercise 10.21 and Exercise 10.22.
Note also that}
\[
  E_1 \subseteq E_2 \subseteq \cdots \subseteq E_n = \mathbb{R}^n.
\]
\begin{enumerate}
\item[(a)]
  \emph{Prove that $d\omega_k = 0$ in $E_k$.}

\item[(b)]
  \emph{For $k=2,\ldots,n$, prove that $\omega_k$ is exact in $E_{k-1}$,
  by showing that
  \[
    \omega_k = d(f_k\omega_{k-1}) = df_k \wedge \omega_{k-1}
  \]
  where $f_k(\mathbf{x}) = (-1)^k g_k\left( \frac{x_k}{r_k} \right)$
  where}
  \[
    g_k(t) = \int_{-1}^{t} (1-s^2)^{\frac{k-3}{2}} ds
    \qquad
    (-1 < t < 1).
  \]

\item[(c)]
  \emph{Is $\omega_n$ exact in $E_n$?}
\end{enumerate}

\item[(7)]
  $H^{n-1}_{\mathrm{dR}}(\mathbb{R}^n-\{\mathbf{0}\}) = \mathbb{R}^1$.
  (Compare to (5)(6)(7).) \\\\
\end{enumerate}



%%%%%%%%%%%%%%%%%%%%%%%%%%%%%%%%%%%%%%%%%%%%%%%%%%%%%%%%%%%%%%%%%%%%%%%%%%%%%%%%



\subsubsection*{Problem 10.1.}
\addcontentsline{toc}{subsubsection}{Problem 10.1.}
\emph{Let $M$ be a $G$-module.
Show that the boundary map $\delta_n: C^{n}(G,M) \to C^{n+1}(G,M)$
defined in this section is a homomorphism.} \\



\emph{Proof.}
\begin{enumerate}
\item[(1)]
  $\delta_n$ is defined by
  \begin{align*}
    \delta_n(f)(\sigma_1, \ldots, \sigma_{n+1})
    = & \:
    \sigma_1 f(\sigma_2, \ldots, \sigma_{n+1}) \\
    & + \sum_{i=1}^{n} (-1)^{i}
        f(\sigma_1, \ldots, \sigma_i \sigma_{i+1}, \ldots, \sigma_{n+1}) \\
    & + (-1)^{n+1} f(\sigma_1, \ldots, \sigma_{n})
  \end{align*}
  if $n > 0$.
  If $n = 0$, then the map $\delta_0: M = C^{0}(G,M) \to C^{1}(G,M)$ is defined by
  $\delta_0(m)(\sigma) = \sigma m - m$.

\item[(2)]
  It suffices to show that $\delta_n(f + g) = \delta_n(f) + \delta_n(g)$
  for all $n$ and all $n$-cochains $f$ and $g$.

\item[(3)]
  If $n = 0$, then
  \begin{align*}
    \delta_0(f + g)(\sigma)
    &= \sigma (f + g) - (f + g) \\
    &= \sigma f + \sigma g - f - g
      &(\text{$M$: $G$-module}) \\
    &= (\sigma f - f) +  (\sigma g - g)
      &(\text{$M$: abelian group}) \\
    &= \delta_0(f) + \delta_0(g).
  \end{align*}

\item[(4)]
  If $n \geq 1$, then
  \begin{align*}
    & \:
    \delta_n(f + g)(\sigma) \\
    = & \:
    \sigma_1 (f + g)(\sigma_2, \ldots, \sigma_{n+1})
    + \sum_{i=1}^{n} (-1)^{i}
        (f + g)(\sigma_1, \ldots, \sigma_i \sigma_{i+1}, \ldots, \sigma_{n+1}) \\
    & + (-1)^{n+1} (f + g)(\sigma_1, \ldots, \sigma_{n}) \\
    = & \:
    \sigma_1 f(\sigma_2, \ldots, \sigma_{n+1}) + \sigma_1 g(\sigma_2, \ldots, \sigma_{n+1}) \\
    & + \sum_{i=1}^{n} (-1)^{i}
        f(\sigma_1, \ldots, \sigma_i \sigma_{i+1}, \ldots, \sigma_{n+1}) \\
    & + \sum_{i=1}^{n} (-1)^{i}
        g(\sigma_1, \ldots, \sigma_i \sigma_{i+1}, \ldots, \sigma_{n+1}) \\
    & + (-1)^{n+1} f(\sigma_1, \ldots, \sigma_{n}) + (-1)^{n+1} g(\sigma_1, \ldots, \sigma_{n}) \\
    = & \:
    \Bigg\{ \sigma_1 f(\sigma_2, \ldots, \sigma_{n+1}) + \sum_{i=1}^{n} (-1)^{i}
        f(\sigma_1, \ldots, \sigma_i \sigma_{i+1}, \ldots, \sigma_{n+1}) \\
    & + (-1)^{n+1} f(\sigma_1, \ldots, \sigma_{n}) \Bigg\}
        + \Bigg\{ \sigma_1 g(\sigma_2, \ldots, \sigma_{n+1}) \\
    & + \sum_{i=1}^{n} (-1)^{i}
        g(\sigma_1, \ldots, \sigma_i \sigma_{i+1}, \ldots, \sigma_{n+1})
        + (-1)^{n+1} g(\sigma_1, \ldots, \sigma_{n}) \Bigg\} \\
    = & \:
    \delta_n(f)(\sigma) + \delta_n(g)(\sigma).
  \end{align*}
  (Here note that $C^{n}(G,M)$ is an abelian group).
\end{enumerate}
$\Box$ \\\\



%%%%%%%%%%%%%%%%%%%%%%%%%%%%%%%%%%%%%%%%%%%%%%%%%%%%%%%%%%%%%%%%%%%%%%%%%%%%%%%%



\subsubsection*{Problem 10.2.}
\addcontentsline{toc}{subsubsection}{Problem 10.2.}
\emph{With notation as in the previous problem,
show that $\delta_{n+1} \circ \delta_n$ is the zero map.} \\



\emph{Proof.}
\begin{enumerate}
\item[(1)]
  If $n = 0$, then
  \begin{align*}
    (\delta_1 \circ \delta_0)(f)(\sigma_1,\sigma_2)
    = & \:
    \delta_1 (\delta_0(f))(\sigma_1,\sigma_2) \\
    = & \:
    \sigma_1 \delta_0(f)(\sigma_2)
        - \delta_0(f)(\sigma_1 \sigma_2)
        + \delta_0(f)(\sigma_1) \\
    = & \:
    \sigma_1 (\sigma_2 f - f)
        - (\sigma_1 \sigma_2 f - f)
        + (\sigma_1 f - f) \\
    = & \:
    0.
  \end{align*}

\item[(2)]
  If $n \geq 1$, then we write
  \begin{align*}
    & \:
    (\delta_{n+1} \circ \delta_n)(f)(\sigma_1,\ldots,\sigma_{n+2}) \\
    = & \:
    \delta_{n+1} (\delta_n(f))(\sigma_1,\ldots,\sigma_{n+2}) \\
    = & \:
    \underbrace{\sigma_1 \delta_n(f)(\sigma_2,\ldots,\sigma_{n+2})}_{\text{Part (3)}} \\
    & + \sum_{j=1}^{n+1}
        \underbrace{(-1)^{j}
        \delta_n(f)(\sigma_1, \ldots, \sigma_j \sigma_{j+1}, \ldots, \sigma_{n+2})}_{
            \text{Parts (4)(5)(6)}} \\
    & + \underbrace{(-1)^{n+2} \delta_n(f)(\sigma_1, \ldots, \sigma_{n+1})}_{\text{Part (7)}}.
  \end{align*}

\item[(3)]
  The first term is
  \begin{align*}
    & \:
    \sigma_1 \delta_n(f)(\sigma_2,\ldots,\sigma_{n+2}) \\
    = & \:
    \sigma_1 \sigma_2 f(\sigma_3, \ldots, \sigma_{n+2}) \\
    & + \sum_{i=1}^{n} (-1)^{i}
        \sigma_1 f(\sigma_2, \ldots, \sigma_{i+1} \sigma_{i+2}, \ldots, \sigma_{n+2}) \\
    & + (-1)^{n+1} \sigma_1 f(\sigma_2, \ldots, \sigma_{n+1}).
  \end{align*}

\item[(4)]
  The first term ($j = 1$) in the summation is
  \begin{align*}
    & \:
    (-1)^{1} \delta_n(f)(\sigma_1 \sigma_2, \ldots, \sigma_{n+2}) \\
    = & \:
    - \sigma_1\sigma_2 f(\sigma_3, \ldots, \sigma_{n+2}) \\
    & + f(\sigma_1\sigma_2\sigma_3, \ldots, \sigma_{n+2})
        - \sum_{i=2}^{n} (-1)^{i}
        f(\sigma_1\sigma_2, \ldots, \sigma_{i+1} \sigma_{i+2}, \ldots, \sigma_{n+2}) \\
    & - (-1)^{n+1} f(\sigma_1\sigma_2, \ldots, \sigma_{n+1})
  \end{align*}

\item[(5)]
  The $j$th term for $2 \leq j \leq n$ in the summation is
  \begin{align*}
    & \:
    (-1)^j \delta_n(f)(\sigma_1, \ldots, \sigma_j \sigma_{j+1}, \ldots, \sigma_{n+2}) \\
    = &\:
    (-1)^j \sigma_1 f(\sigma_2, \ldots, \sigma_j \sigma_{j+1}, \ldots, \sigma_{n+2}) \\
    & + (-1)^j \sum_{i=1}^{j-2} (-1)^{i}
        f(\sigma_1, \ldots, \sigma_i \sigma_{i+1}, \ldots,
            \sigma_j \sigma_{j+1}, \ldots, \sigma_{n+2}) \\
    & + (-1)^j (-1)^{j-1}
        f(\sigma_1, \ldots, \sigma_{j-1} \sigma_j \sigma_{j+1}, \ldots, \sigma_{n+2}) \\
    & + (-1)^j (-1)^{j}
        f(\sigma_1, \ldots, \sigma_j \sigma_{j+1} \sigma_{j+2}, \ldots, \sigma_{n+2}) \\
    & + (-1)^j \sum_{i=j+1}^{n} (-1)^{i}
        f(\sigma_1, \ldots, \sigma_j \sigma_{j+1}, \ldots,
            \sigma_{i+1} \sigma_{i+2}, \ldots, \sigma_{n+2}) \\
    & + (-1)^j (-1)^{n+1} f(\sigma_1, \ldots, \sigma_j \sigma_{j+1}, \ldots, \sigma_{n+1}).
  \end{align*}

\item[(6)]
  The last term ($j = n+1$) in the summation is
  \begin{align*}
    & \:
    (-1)^{n+1} \delta_n(f)(\sigma_1, \ldots, \sigma_n, \sigma_{n+1}\sigma_{n+2}) \\
    = & \:
    (-1)^{n+1} \sigma_1 f(\sigma_2, \ldots, \sigma_{n+1}\sigma_{n+2}) \\
    & + (-1)^{n+1} \sum_{i=1}^{n-1} (-1)^{i}
        f(\sigma_1, \ldots, \sigma_i \sigma_{i+1}, \ldots, \sigma_{n+1}\sigma_{n+2}) \\
    & + (-1)^{n+1} (-1)^n f(\sigma_1, \ldots, \sigma_{n}\sigma_{n+1}\sigma_{n+2}) \\
    & + (-1)^{n+1} (-1)^{n+1} f(\sigma_1, \ldots, \sigma_{n}).
  \end{align*}

\item[(7)]
  The last term is
  \begin{align*}
    & \:
    (-1)^{n+2} \delta_n(f)(\sigma_1, \ldots, \sigma_{n+1}) \\
    = & \:
    (-1)^{n+2} \sigma_1 f(\sigma_2, \ldots, \sigma_{n+1}) \\
    & + (-1)^{n+2} \sum_{i=1}^{n} (-1)^{i}
        f(\sigma_1, \ldots, \sigma_i \sigma_{i+1}, \ldots, \sigma_{n+1}) \\
    & + (-1)^{n+2} (-1)^{n+1} f(\sigma_1, \ldots, \sigma_{n}).
  \end{align*}

\item[(8)]
  Hence we have
  $(\delta_{n+1} \circ \delta_n)(f)(\sigma_1,\ldots,\sigma_{n+2}) = 0$.
\end{enumerate}
$\Box$ \\



\subsubsection*{Supplement.}
\addcontentsline{toc}{subsubsection}{Supplement.}
\begin{enumerate}
\item[(1)]
  \emph{(Theorem 10.20 in the textbook:
  Rudin, Principles of Mathematical Analysis, 3rd edition.)
  If $\omega$ is a $k$-form of class $\mathscr{C}''$
  in some open set $E \subseteq \mathbb{R}^n$, then $d^2 \omega = 0$.}

\item[(2)]
  \emph{(Exercise 10.16 in the textbook:
  Rudin, Principles of Mathematical Analysis, 3rd edition.)
  If $k \geq 2$ and $\sigma = [\mathbf{p}_0,\mathbf{p}_1,\ldots,\mathbf{p}_k]$
  is an oriented affine $k$-simplex, prove that $\partial^2 \sigma = 0$,
  directly from the definition of the boundary operator $\partial$.
  Deduce from this that $\partial^2 \Psi = 0$ for every chain $\Psi$.} \\\\
\end{enumerate}



%%%%%%%%%%%%%%%%%%%%%%%%%%%%%%%%%%%%%%%%%%%%%%%%%%%%%%%%%%%%%%%%%%%%%%%%%%%%%%%%



\subsubsection*{Problem 10.3.}
\addcontentsline{toc}{subsubsection}{Problem 10.3.}
\emph{Let $M$ be a $G$-module, and let $f \in Z^2(G,M)$.
Show that $f(1,1) = f(1,\sigma) = \sigma^{-1} f(\sigma, 1)$ for all $\sigma \in G$.} \\



\emph{Proof.}
\begin{enumerate}
\item[(1)]
  $f \in Z^2(G,M)$ if and only if $\delta_2(f) = 0$.
  So
  \begin{align*}
    \delta_2(f)(\sigma_1, \sigma_2, \sigma_3)
    &= \sigma_1 f(\sigma_2, \sigma_3)
        - f(\sigma_1\sigma_2,\sigma_3)
        + f(\sigma_1, \sigma_2\sigma_3)
        - f(\sigma_1, \sigma_2) \\
    &= 0.
  \end{align*}
  for any $\sigma_1 \sigma_2, \sigma_3 \in G$.

\item[(2)]
  Take $\sigma_1 = \sigma_2 = 1$ and $\sigma_3 = \sigma$ to get
  \[
    f(1,\sigma) - f(1,\sigma) + f(1,\sigma) - f(1,1) = 0.
  \]
  So $f(1,1) = f(1,\sigma)$.

\item[(3)]
  Take $\sigma_1 = \sigma$ and $\sigma_2 = \sigma_3 = 1$ to get
  \[
    \sigma f(1,1) - f(\sigma,1) + f(\sigma,1) - f(\sigma,1) = 0.
  \]
  So $\sigma f(1,1) = f(\sigma,1)$ or $f(1,1) = \sigma^{-1}f(\sigma,1)$.
\end{enumerate}
$\Box$ \\\\



%%%%%%%%%%%%%%%%%%%%%%%%%%%%%%%%%%%%%%%%%%%%%%%%%%%%%%%%%%%%%%%%%%%%%%%%%%%%%%%%



\subsubsection*{Problem 10.4.}
\addcontentsline{toc}{subsubsection}{Problem 10.4.}
\emph{If $E$ is a group with an abelian normal subgroup $M$,
and if $G = E/M$,
show that the action of $G$ on $M$ given by $\sigma m = eme^{-1}$ if
$eM = \sigma$ is well-defined and makes $M$ into a $G$-module.} \\



\emph{Proof.}
\begin{enumerate}
\item[(1)]
  \emph{Show that $G \times M \to M$ defined by $\sigma m = eme^{-1}$
  is independent of the choice of the coset representation of $\sigma = eM$.}
  Suppose $\sigma = e_1 M = e_2 M$.
  $e_2 = e_1 m_1$ for some $m_1 \in M$.

\item[(2)]
  Therefore
  \[
    e_2 m e_2^{-1}
    = (e_1 m_1)m(e_1 m_1)^{-1}
    = e_1 m_1 m m_1^{-1} e_1^{-1}
    = e_1 m e_1^{-1}.
  \]
  Here $(e_1 m_1)^{-1} = m_1^{-1} e_1^{-1}$ holds in a group $E$
  and $m_1 m m_1^{-1} = m$ since $M$ is an abelian group.

\item[(3)]
  \emph{Show that $M$ is a $G$-module where
  $G \times M \to M$ is defined by $\sigma m = eme^{-1}$.}
  \begin{enumerate}
  \item[(a)]
    \emph{Show that $1m = m$.}
    $1m = 1m1^{-1} = m$ where $1 = 1M \in G = E/M$.
    
  \item[(b)]
    \emph{Show that $\sigma(\tau m) = (\sigma\tau) m$.}
    Write $\sigma = e_{\sigma} M$ and $\tau = e_{\tau} M$.
    Hence $\sigma \tau = e_{\sigma} e_{\tau} M$
    and
    \begin{align*}
      \sigma(\tau m)
      &= \sigma(e_{\tau} m e_{\tau}^{-1}) \\
      &= e_{\sigma}(e_{\tau} m e_{\tau}^{-1})e_{\sigma}^{-1} \\
      &= (e_{\sigma}e_{\tau}) m (e_{\sigma}e_{\tau})^{-1} \\
      &= (\sigma \tau)m.
    \end{align*}

  \item[(c)]
    \emph{Show that $\sigma(m_1 + m_2) = \sigma m_1 + \sigma m_2$.}
    \begin{align*}
      \sigma (m_1 + m_2)
      &= e(m_1 + m_2)e^{-1} \\
      &= em_1e^{-1} + em_2e^{-1} \\
      &= \sigma m_1 + \sigma m_2
    \end{align*}
    where $\sigma = eM$ for some $e \in E$.
  \end{enumerate}
\end{enumerate}
$\Box$ \\\\



%%%%%%%%%%%%%%%%%%%%%%%%%%%%%%%%%%%%%%%%%%%%%%%%%%%%%%%%%%%%%%%%%%%%%%%%%%%%%%%%



\subsubsection*{Problem 10.5.}
\addcontentsline{toc}{subsubsection}{Problem 10.5.}
\emph{With $E$, $M$, $G$ as in the previous problem,
if $e_{\sigma}$ is a coset representative of $\sigma$,
show that the function defined by
$f(\sigma,\tau) = e_{\sigma}e_{\tau}e_{\sigma\tau}^{-1}$ is a $2$-cocycle.} \\



\emph{Proof.}
  It suffices to show that $\delta_2(f)(\sigma, \tau, \upsilon) = 0$
  for any $\sigma, \tau, \upsilon \in G$.
  That is,
  \begin{align*}
    & \:
    \delta_2(f)(\sigma, \tau, \upsilon) \\
    = & \: \sigma f(\tau, \upsilon)
        f(\sigma \tau, \upsilon)^{-1}
        f(\sigma, \tau \upsilon)
        f(\sigma, \tau)^{-1} \\
    = & \: \sigma f(\tau, \upsilon)
        f(\sigma, \tau \upsilon)
        f(\sigma \tau, \upsilon)^{-1}
        f(\sigma, \tau)^{-1}
      &(\text{$M$: abelian}) \\
    = & \: \sigma (e_{\tau}e_{\upsilon}e_{\tau\upsilon}^{-1})
        (e_{\sigma}e_{\tau\upsilon}e_{\sigma\tau\upsilon}^{-1})
        (e_{\sigma\tau}e_{\upsilon}e_{\sigma\tau\upsilon}^{-1})^{-1}
        (e_{\sigma}e_{\tau}e_{\sigma\tau}^{-1})^{-1} \\
    = & \: (e_{\sigma} e_{\tau}e_{\upsilon}e_{\tau\upsilon}^{-1} e_{\sigma}^{-1})
        (e_{\sigma}e_{\tau\upsilon}e_{\sigma\tau\upsilon}^{-1})
        (e_{\sigma\tau\upsilon}e_{\upsilon}^{-1}e_{\sigma\tau}^{-1})
        (e_{\sigma\tau}e_{\tau}^{-1}e_{\sigma}^{-1}) \\
    = & \: 1.
  \end{align*}
$\Box$ \\\\



%%%%%%%%%%%%%%%%%%%%%%%%%%%%%%%%%%%%%%%%%%%%%%%%%%%%%%%%%%%%%%%%%%%%%%%%%%%%%%%%



\subsubsection*{Problem 10.6.}
\addcontentsline{toc}{subsubsection}{Problem 10.6.}
\emph{Suppose that $M$ is a $G$-module.
For each $\sigma \in G$, let $m_{\sigma} \in M$.
Show that the cochain $f$ defined by
$f(\sigma,\tau) = m_{\sigma} + \sigma m_{\tau} - m_{\sigma\tau}$ is a coboundary.} \\



\emph{Proof.}
\begin{enumerate}
\item[(1)]
  To show $f$ is a $2$-coboundary,
  it suffices to show that there is a $g \in C^{1}(G,M)$ such that $f = \delta_1(g)$.

\item[(2)]
  Actually, we can define $g: G \to M$ by $\sigma \mapsto m_{\sigma}$.
  So
  \[
    \delta_1(g)(\sigma,\tau)
    = \sigma g(\tau) - g(\sigma\tau) + g(\sigma)
    = \sigma m_{\tau} - m_{\sigma\tau} + m_{\sigma}
    = f(\sigma,\tau)
  \]
  for all $\sigma, \tau \in G$.
  Hence $f \in B^{2}(G,M)$.
\end{enumerate}
$\Box$ \\\\



%%%%%%%%%%%%%%%%%%%%%%%%%%%%%%%%%%%%%%%%%%%%%%%%%%%%%%%%%%%%%%%%%%%%%%%%%%%%%%%%



\subsubsection*{Problem 10.7.}
\addcontentsline{toc}{subsubsection}{Problem 10.7.}
\emph{If $M$ is a $G$-module and $f \in Z^2(G,M)$,
show that $E_f = M \times G$ with multiplication defined by
\[
  (m,\sigma)(n,\tau) = (m \cdot \sigma n \cdot f(\sigma,\tau), \sigma\tau)
\]
makes $E_f$ into a group.} \\



\emph{Proof.}
\begin{enumerate}
\item[(1)]
  The multiplication is a binary operation on $E_f$.

\item[(2)]
  (Associativity)
  \emph{Show that
  \[
    ((m,\sigma)(n,\tau))(k,\upsilon) = (m,\sigma)((n,\tau)(k,\upsilon)).
  \]
  for all $(m,\sigma), (n,\tau), (k,\upsilon)$.}
  Note that
  \begin{align*}
    ((m,\sigma)(n,\tau))(k,\upsilon)
    = & \:
    (m \cdot \sigma n \cdot f(\sigma,\tau), \sigma\tau)(k,\upsilon) \\
    = & \:
    (m \cdot \sigma n \cdot f(\sigma,\tau)
        \cdot \sigma\tau k \cdot f(\sigma\tau, \upsilon), \sigma\tau\upsilon) \\
    = & \:
    (m \cdot \sigma n \cdot \sigma\tau k 
        \cdot f(\sigma,\tau) \cdot f(\sigma\tau, \upsilon), \sigma\tau\upsilon)
  \end{align*}
  and
  \begin{align*}
    (m,\sigma)((n,\tau)(k,\upsilon))
    = & \:
    (m,\sigma)(n \cdot \tau k \cdot f(\tau,\upsilon), \tau\upsilon) \\
    = & \:
    (m \cdot \sigma(n \cdot \tau k \cdot f(\tau,\upsilon))
        \cdot f(\sigma, \tau\upsilon), \sigma\tau\upsilon) \\
    = & \:
    (m \cdot \sigma n \cdot \sigma \tau k
        \cdot \underbrace{\sigma f(\tau,\upsilon) \cdot f(\sigma, \tau\upsilon)}_{=
            f(\sigma,\tau) \cdot f(\sigma\tau, \upsilon)},
        \sigma\tau\upsilon)
  \end{align*}
  (since $f \in Z^2(G,M)$).

\item[(3)]
  (Identity element)
  \emph{Show that there exists an element
  \[
    1 := (f(1,1)^{-1},1) \in E_f
  \]
  such that
  $1 (m,\sigma) = (m,\sigma) 1 = (m,\sigma)$ for every $(m,\sigma) \in E_f$.}
  Same as Problem 10.3.
  Note that
  \begin{align*}
    (m,\sigma)(f(1,1)^{-1},1)
    = & \:
    (m \cdot \sigma \underbrace{f(1,1)^{-1}}_{= \sigma^{-1} f(\sigma,1)^{-1}}
        \cdot f(\sigma,1), \sigma) \\
    = & \:
    (m \cdot \sigma (\sigma^{-1} f(\sigma,1)^{-1}) \cdot f(\sigma,1), \sigma) \\
    = & \:
    (m \cdot (\sigma \sigma^{-1}) f(\sigma,1)^{-1} \cdot f(\sigma,1), \sigma) \\
    = & \:
    (m, \sigma)
  \end{align*}
  and
  \begin{align*}
    (f(1,1)^{-1},1)(m,\sigma)
    = & \:
    (f(1,1)^{-1} \cdot m \cdot f(1, \sigma), \sigma) \\
    = & \:
    (f(1, \sigma)^{-1} \cdot m \cdot f(1, \sigma), \sigma) \\
    = & \:
    (m, \sigma).
  \end{align*}

\item[(4)]
  \emph{Note.}
  To find the identity element, we need to find $(n,\tau)$
  such that $(m,\sigma)(n,\tau) = (m,\sigma)$.
  So
  \[
    (m,\sigma)(n,\tau) = (m \cdot \sigma n \cdot f(\sigma,\tau), \sigma\tau) = (m,\sigma)
  \]
  implies that $\tau = 1 \in G$ and thus $m \cdot \sigma n \cdot f(\sigma,1) = m$.
  Hence
  \[
    n
    = \sigma^{-1} f(\sigma,1)^{-1}
    = (\sigma^{-1} f(\sigma,1))^{-1}
    = f(1,1)^{-1}
  \]
  (in the multiplicative notation).

\item[(5)]
  (Inverse element)
  \emph{Show that for each $(m,\sigma) \in E_f$,
  there exists an element
  \[
    (n,\tau)
    :=
    \left(\sigma^{-1}\left\{f(\sigma,\sigma^{-1})^{-1} \cdot m^{-1} \cdot f(1,1)^{-1}\right\}, \sigma^{-1} \right)
    \in E_f
  \]
  such that
  $(m,\sigma)(n,\tau) = (n,\tau)(m,\sigma) = 1$, where $1$ is the identity element in $E_f$.}
  (To find the inverse element, we might apply the same argument in part (4).)
  A direct calculation with the fact that $f \in Z^2(G,M)$ gives
  \begin{align*}
    & \:
    (m,\sigma)
    \left(\sigma^{-1}\left\{f(\sigma,\sigma^{-1})^{-1} \cdot m^{-1} \cdot f(1,1)^{-1}\right\}, \sigma^{-1} \right) \\
    = & \:
    (m \cdot \sigma\left(\sigma^{-1}\left\{f(\sigma,\sigma^{-1})^{-1} \cdot m^{-1} \cdot f(1,1)^{-1}\right\}\right)
        \cdot f(\sigma, \sigma^{-1}), 1) \\
    = & \:
    (m \cdot f(\sigma,\sigma^{-1})^{-1} \cdot m^{-1} \cdot f(1,1)^{-1}
        \cdot f(\sigma, \sigma^{-1}), 1) \\
    = & \:
    (f(1,1)^{-1}, 1)
  \end{align*}
  and
  \begin{align*}
    & \:
    \left(\sigma^{-1}\left\{f(\sigma,\sigma^{-1})^{-1} \cdot m^{-1} \cdot f(1,1)^{-1}\right\}, \sigma^{-1} \right)
    (m,\sigma) \\
    = & \:
    (\sigma^{-1}\left\{f(\sigma,\sigma^{-1})^{-1} \cdot m^{-1} \cdot f(1,1)^{-1}\right\}
        \cdot \sigma^{-1} m \cdot f(\sigma^{-1}, \sigma), 1) \\
    = & \:
    (\sigma^{-1} f(\sigma,\sigma^{-1})^{-1}
        \cdot f(\sigma^{-1}, \sigma) \cdot \sigma^{-1} f(1,1)^{-1}, 1) \\
    = & \:
    (f(1,1)^{-1} \cdot \underbrace{\sigma^{-1} f(1,1) \cdot \sigma^{-1} f(1,1)^{-1}}_{= 1}, 1) \\
    = & \:
    (f(1,1)^{-1}, 1).
  \end{align*}
  Here we take $(\sigma_1, \sigma_2, \sigma_3) \mapsto (\sigma^{-1}, \sigma, \sigma^{-1})$
  in part (1) of the proof of Problem 10.3 to get
  \begin{align*}
    \sigma^{-1} f(\sigma,\sigma^{-1})^{-1} \cdot f(\sigma^{-1}, \sigma)
    = & \:
    f(1,\sigma^{-1})^{-1} \cdot f(\sigma^{-1},1) \\
    = & \:
    f(1,1)^{-1} \cdot \sigma^{-1} f(1,1).
  \end{align*}
\end{enumerate}
$\Box$ \\\\



%%%%%%%%%%%%%%%%%%%%%%%%%%%%%%%%%%%%%%%%%%%%%%%%%%%%%%%%%%%%%%%%%%%%%%%%%%%%%%%%



\subsubsection*{Problem 10.8.}
\addcontentsline{toc}{subsubsection}{Problem 10.8.}
\emph{If $M$ is a $G$-module,
show that the group extensions constructed from $2$-cocycles
$f, g \in Z^2(G,M)$ are isomorphic if $f$ and $g$ are cohomologous.} \\



\emph{Proof.}
\begin{enumerate}
\item[(1)]
  Say $f \cdot g^{-1} = \delta_1(h)$ for some $h \in B^{1}(G,B)$, i.e.,
  \[
    f(\sigma,\tau) \cdot g^{-1}(\sigma,\tau)
    = \delta_1(h)(\sigma,\tau)
    = \sigma h(\tau) \cdot h(\sigma\tau)^{-1} \cdot h(\sigma).
  \]

\item[(2)]
  By the help of $h$, define a map $\alpha: E_f \to E_g$ by
  \[
    \alpha((m,\sigma)) = (m \cdot h(\sigma), \sigma).
  \]
  Now it suffices to show that $\alpha$ is a group isomorphism.

\item[(3)]
  \emph{Show that $\alpha$ is a group homomorphism.}
  Note that
  \begin{align*}
    \alpha((m,\sigma)(n,\tau))
    &= \alpha((m \cdot \sigma n \cdot f(\sigma,\tau), \sigma\tau)) \\
    &= (m \cdot \sigma n \cdot f(\sigma,\tau) \cdot h(\sigma\tau), \sigma\tau)
  \end{align*}
  and
  \begin{align*}
    \alpha((m,\sigma))\alpha((n,\tau))
    &= (m \cdot h(\sigma), \sigma)(n \cdot h(\tau), \tau) \\
    &= (m \cdot h(\sigma) \cdot \sigma(n \cdot h(\tau)) \cdot f(\sigma,\tau), \sigma\tau) \\
    &= (m \cdot \sigma n \cdot f(\sigma,\tau) \cdot
        \underbrace{\sigma h(\tau) \cdot h(\sigma)}_{= h(\sigma\tau)}, \sigma\tau).
      &(\text{(1)})
  \end{align*}
  Hence $\alpha((m,\sigma)(n,\tau)) = \alpha((m,\sigma))\alpha((n,\tau))$.

\item[(3)]
  \emph{Show that $\alpha$ is injective.}
  $\alpha((m,\sigma)) = \alpha((n,\tau))$ implies that
  $(m \cdot h(\sigma), \sigma) = (n \cdot h(\tau),\tau)$.
  So $\sigma = \tau$, $h(\sigma) = h(\tau)$, and thus $m = n$.
  
\item[(4)]
  \emph{Show that $\alpha$ is surjective.}
  Given any $(m,\sigma) \in E_g$, we have
  \[
    \alpha((m \cdot h(\sigma)^{-1}, \sigma)) = (m, \sigma).
  \]
\end{enumerate}
$\Box$ \\\\



%%%%%%%%%%%%%%%%%%%%%%%%%%%%%%%%%%%%%%%%%%%%%%%%%%%%%%%%%%%%%%%%%%%%%%%%%%%%%%%%



\subsubsection*{Problem 10.9.}
\addcontentsline{toc}{subsubsection}{Problem 10.9.}
\emph{In the crossed product construction given in this section,
show that the multiplicative identity is $f(1,1)^{-1}x_{\mathrm{id}}$.} \\



\emph{Proof.}
\begin{enumerate}
\item[(1)]
  \begin{align*}
    (f(1,1)^{-1}x_{\mathrm{id}}) \sum_{\sigma \in G} a_{\sigma} x_{\sigma}
    &= \sum_{\sigma \in G} f(1,1)^{-1} \mathrm{id}(a_{\sigma}) f(\mathrm{id}, \sigma)
        x_{\mathrm{id} \cdot \sigma} \\
    &= \sum_{\sigma \in G} f(1,1)^{-1} a_{\sigma} f(1, \sigma) x_{\sigma} \\
    &= \sum_{\sigma \in G} a_{\sigma} x_{\sigma}
  \end{align*}
  for all $\sum_{\sigma \in G} a_{\sigma} x_{\sigma} \in A = (K/F,G,f)$.

\item[(2)]
  \begin{align*}
    \left( \sum_{\sigma \in G} a_{\sigma} x_{\sigma} \right)(f(1,1)^{-1}x_{\mathrm{id}})
    &= \sum_{\sigma \in G} a_{\sigma} \sigma(f(1,1)^{-1}) f(\sigma, \mathrm{id})
        x_{\sigma \cdot \mathrm{id}} \\
    &= \sum_{\sigma \in G} a_{\sigma} \sigma f(1,1)^{-1} f(\sigma, 1)
        x_{\sigma} \\
    &= \sum_{\sigma \in G} a_{\sigma} x_{\sigma}
  \end{align*}
  for all $\sum_{\sigma \in G} a_{\sigma} x_{\sigma} \in A = (K/F,G,f)$.
\end{enumerate}
$\Box$ \\\\



%%%%%%%%%%%%%%%%%%%%%%%%%%%%%%%%%%%%%%%%%%%%%%%%%%%%%%%%%%%%%%%%%%%%%%%%%%%%%%%%



%%%%%%%%%%%%%%%%%%%%%%%%%%%%%%%%%%%%%%%%%%%%%%%%%%%%%%%%%%%%%%%%%%%%%%%%%%%%%%%%



%%%%%%%%%%%%%%%%%%%%%%%%%%%%%%%%%%%%%%%%%%%%%%%%%%%%%%%%%%%%%%%%%%%%%%%%%%%%%%%%
%%%%%%%%%%%%%%%%%%%%%%%%%%%%%%%%%%%%%%%%%%%%%%%%%%%%%%%%%%%%%%%%%%%%%%%%%%%%%%%%
%%%%%%%%%%%%%%%%%%%%%%%%%%%%%%%%%%%%%%%%%%%%%%%%%%%%%%%%%%%%%%%%%%%%%%%%%%%%%%%%
%%%%%%%%%%%%%%%%%%%%%%%%%%%%%%%%%%%%%%%%%%%%%%%%%%%%%%%%%%%%%%%%%%%%%%%%%%%%%%%%



\end{document}