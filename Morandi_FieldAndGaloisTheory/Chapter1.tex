\documentclass{article}
\usepackage{amsfonts}
\usepackage{amsmath}
\usepackage{amssymb}
\usepackage{hyperref}
\usepackage{mathrsfs}
\parindent=0pt

\def\upint{\mathchoice%
    {\mkern13mu\overline{\vphantom{\intop}\mkern7mu}\mkern-20mu}%
    {\mkern7mu\overline{\vphantom{\intop}\mkern7mu}\mkern-14mu}%
    {\mkern7mu\overline{\vphantom{\intop}\mkern7mu}\mkern-14mu}%
    {\mkern7mu\overline{\vphantom{\intop}\mkern7mu}\mkern-14mu}%
  \int}
\def\lowint{\mkern3mu\underline{\vphantom{\intop}\mkern7mu}\mkern-10mu\int}

\begin{document}

\textbf{\Large Chapter 1: Galois Theory} \\\\



\emph{Author: Meng-Gen Tsai} \\
\emph{Email: plover@gmail.com} \\\\



\textbf{\large Section 1.1: Field Extensions} \\\\



\textbf{Exercise 1.1.1.}
\emph{Let $K$ be a field extension of $F$.
By defining scalar multiplication for $\alpha \in F$ and $a \in K$
by $\alpha \cdot a = \alpha a$, the multiplication in $K$,
show that $K$ is an $F$-vector space.} \\

\emph{Proof.}
\begin{enumerate}
\item[(1)]
$K$ is an additive group.
\item[(2)]
\emph{Show that $(\alpha \beta) \cdot a = \alpha \cdot (\beta \cdot a)$
for $\alpha, \beta \in F$ and $a \in K$.}
In fact,
\begin{align*}
(\alpha \beta) \cdot a
&= \alpha \beta a \in K, \\
\alpha \cdot (\beta \cdot a)
&= \alpha \beta a \in K.
\end{align*}
\item[(3)]
\emph{Show that $(\alpha + \beta) \cdot a = \alpha \cdot a + \beta \cdot a$
for $\alpha, \beta \in F$ and $a \in K$.}
\begin{align*}
(\alpha + \beta) \cdot a
&= (\alpha + \beta) a \\
&= \alpha a + \beta a \in K, \\
\alpha \cdot a + \beta \cdot a
&= \alpha a + \beta a \in K.
\end{align*}
\item[(4)]
\emph{Show that $\alpha \cdot (a + b) = \alpha \cdot a + \alpha \cdot b$
for $\alpha \in F$ and $a, b \in K$.}
\begin{align*}
\alpha \cdot (a + b)
&= \alpha (a + b) \\
&= \alpha a + \alpha b \in K, \\
\alpha \cdot a + \alpha \cdot b
&= \alpha a + \alpha b \in K.
\end{align*}
\item[(5)]
\emph{Show that $1 \cdot a = a$
for $a \in K$.}
$1 \cdot a = 1 a = a \in K$.
\end{enumerate}
By (1) to (5), $K$ is an $F$-vector space.
$\Box$ \\\\



\textbf{Exercise 1.1.2.}
\emph{If $K$ is a field extension of $F$, prove that $[K:F] = 1$
if and only if $K = F$.} \\

\emph{Proof.}
\begin{enumerate}
\item[(1)]
\emph{$[K:F] = 1 \Longleftarrow K = F$.}
Take a basis $\{1\}$ for $K$ as an $F$-vector space.
\item[(2)]
\emph{$[K:F] = 1 \Longrightarrow K = F$.}
Take a basis $\{a\}$ for $K$ as an $F$-vector space where $a \in K$.
Since $1 \in K$ as an $F$-vector space,
there exists $\alpha \in F$ such that $1 = \alpha a$.
$a = \alpha^{-1} \in F$, or $K \subseteq F$, or $K = F$.
\end{enumerate}
$\Box$ \\\\



\textbf{Exercise 1.1.5.}
\emph{Show that 
$\mathbb{Q}(\sqrt{5}, \sqrt{7}) = \mathbb{Q}(\sqrt{5} + \sqrt{7})$.} \\

\emph{Proof.}
\begin{enumerate}
\item[(1)]
$\mathbb{Q}(\sqrt{5}, \sqrt{7}) \supseteq \mathbb{Q}(\sqrt{5} + \sqrt{7})$
since $\sqrt{5} + \sqrt{7} \in \mathbb{Q}(\sqrt{5}, \sqrt{7})$.
\item[(2)]
\begin{align*}
(\sqrt{7} + \sqrt{5})^{-1}
&= \frac{1}{\sqrt{7} + \sqrt{5}} \\
&= \frac{\sqrt{7} - \sqrt{5}}{(\sqrt{7} + \sqrt{5})(\sqrt{7} - \sqrt{5})} \\
&= \frac{\sqrt{7} - \sqrt{5}}{2} \in \mathbb{Q}(\sqrt{5} + \sqrt{7}),
\end{align*}
Or $\sqrt{7} - \sqrt{5} \in \mathbb{Q}(\sqrt{5} + \sqrt{7})$. Thus
\begin{align*}
\sqrt{7}
&= \frac{1}{2} \cdot ((\sqrt{7} + \sqrt{5}) + (\sqrt{7} - \sqrt{5}))
\in \mathbb{Q}(\sqrt{5} + \sqrt{7}), \\
\sqrt{5}
&= \frac{1}{2} \cdot ((\sqrt{7} + \sqrt{5}) - (\sqrt{7} - \sqrt{5}))
\in \mathbb{Q}(\sqrt{5} + \sqrt{7}).
\end{align*}
Thus, $\mathbb{Q}(\sqrt{5}, \sqrt{7}) \subseteq \mathbb{Q}(\sqrt{5} + \sqrt{7})$.
\end{enumerate}
By (1)(2), $\mathbb{Q}(\sqrt{5}, \sqrt{7}) = \mathbb{Q}(\sqrt{5} + \sqrt{7})$.
$\Box$ \\\\



\textbf{Exercise 1.1.9.}
\emph{If $K$ is an extension of $F$ such that $[K:F]$ is prime,
show that there are no intermediate fields between $K$ and $F$.} \\

\emph{Proof.}
Let $L$ be any field such that $F \subseteq L \subseteq K$.
By Proposition 1.20,
$$[K:F] = [K:L][L:F].$$
Since $[K:F]$ is prime, $[K:L] = 1$ or $[L:F] = 1$.
By Exercise 1.1.2, $L=K$ or $L=F$,
or there are no intermediate fields between $K$ and $F$.
$\Box$ \\\\



\end{document}