\documentclass{article}
\usepackage{amsfonts}
\usepackage{amsmath}
\usepackage{amssymb}
\usepackage{hyperref}
\usepackage{mathrsfs}
\parindent=0pt

\def\upint{\mathchoice%
    {\mkern13mu\overline{\vphantom{\intop}\mkern7mu}\mkern-20mu}%
    {\mkern7mu\overline{\vphantom{\intop}\mkern7mu}\mkern-14mu}%
    {\mkern7mu\overline{\vphantom{\intop}\mkern7mu}\mkern-14mu}%
    {\mkern7mu\overline{\vphantom{\intop}\mkern7mu}\mkern-14mu}%
  \int}
\def\lowint{\mkern3mu\underline{\vphantom{\intop}\mkern7mu}\mkern-10mu\int}

\begin{document}

\textbf{\Large Chapter 1: Galois Theory} \\\\



\emph{Author: Meng-Gen Tsai} \\
\emph{Email: plover@gmail.com} \\\\



\textbf{\large Section 1.1: Field Extensions} \\\\



\textbf{Problem 1.1.1.}
\emph{Let $K$ be a field extension of $F$.
By defining scalar multiplication for $\alpha \in F$ and $a \in K$
by $\alpha \cdot a = \alpha a$, the multiplication in $K$,
show that $K$ is an $F$-vector space.} \\

\emph{Proof.}
\begin{enumerate}
\item[(1)]
$K$ is an additive group.
\item[(2)]
\emph{Show that $(\alpha \beta) \cdot a = \alpha \cdot (\beta \cdot a)$
for $\alpha, \beta \in F$ and $a \in K$.}
In fact,
\begin{align*}
(\alpha \beta) \cdot a
&= \alpha \beta a \in K, \\
\alpha \cdot (\beta \cdot a)
&= \alpha \beta a \in K.
\end{align*}
\item[(3)]
\emph{Show that $(\alpha + \beta) \cdot a = \alpha \cdot a + \beta \cdot a$
for $\alpha, \beta \in F$ and $a \in K$.}
\begin{align*}
(\alpha + \beta) \cdot a
&= (\alpha + \beta) a \\
&= \alpha a + \beta a \in K, \\
\alpha \cdot a + \beta \cdot a
&= \alpha a + \beta a \in K.
\end{align*}
\item[(4)]
\emph{Show that $\alpha \cdot (a + b) = \alpha \cdot a + \alpha \cdot b$
for $\alpha \in F$ and $a, b \in K$.}
\begin{align*}
\alpha \cdot (a + b)
&= \alpha (a + b) \\
&= \alpha a + \alpha b \in K, \\
\alpha \cdot a + \alpha \cdot b
&= \alpha a + \alpha b \in K.
\end{align*}
\item[(5)]
\emph{Show that $1 \cdot a = a$
for $a \in K$.}
$1 \cdot a = 1 a = a \in K$.
\end{enumerate}
By (1) to (5), $K$ is an $F$-vector space.
$\Box$ \\\\



\textbf{Problem 1.1.2.}
\emph{If $K$ is a field extension of $F$, prove that $[K:F] = 1$
if and only if $K = F$.} \\

\emph{Proof.}
\begin{enumerate}
\item[(1)]
\emph{$[K:F] = 1 \Longleftarrow K = F$.}
Take a basis $\{1\}$ for $K$ as an $F$-vector space.
\item[(2)]
\emph{$[K:F] = 1 \Longrightarrow K = F$.}
Take a basis $\{a\}$ for $K$ as an $F$-vector space where $a \in K$.
Since $1 \in K$ as an $F$-vector space,
there exists $\alpha \in F$ such that $1 = \alpha a$.
$a = \alpha^{-1} \in F$, or $K \subseteq F$, or $K = F$.
\end{enumerate}
$\Box$ \\\\



\textbf{Problem 1.1.5.}
\emph{Show that 
$\mathbb{Q}(\sqrt{5}, \sqrt{7}) = \mathbb{Q}(\sqrt{5} + \sqrt{7})$.} \\

\emph{Proof.}
\begin{enumerate}
\item[(1)]
$\mathbb{Q}(\sqrt{5}, \sqrt{7}) \supseteq \mathbb{Q}(\sqrt{5} + \sqrt{7})$
since $\sqrt{5} + \sqrt{7} \in \mathbb{Q}(\sqrt{5}, \sqrt{7})$.
\item[(2)]
\begin{align*}
(\sqrt{7} + \sqrt{5})^{-1}
&= \frac{1}{\sqrt{7} + \sqrt{5}} \\
&= \frac{\sqrt{7} - \sqrt{5}}{(\sqrt{7} + \sqrt{5})(\sqrt{7} - \sqrt{5})} \\
&= \frac{\sqrt{7} - \sqrt{5}}{2} \in \mathbb{Q}(\sqrt{5} + \sqrt{7}),
\end{align*}
Or $\sqrt{7} - \sqrt{5} \in \mathbb{Q}(\sqrt{5} + \sqrt{7})$. Thus
\begin{align*}
\sqrt{7}
&= \frac{1}{2} \cdot ((\sqrt{7} + \sqrt{5}) + (\sqrt{7} - \sqrt{5}))
\in \mathbb{Q}(\sqrt{5} + \sqrt{7}), \\
\sqrt{5}
&= \frac{1}{2} \cdot ((\sqrt{7} + \sqrt{5}) - (\sqrt{7} - \sqrt{5}))
\in \mathbb{Q}(\sqrt{5} + \sqrt{7}).
\end{align*}
Thus, $\mathbb{Q}(\sqrt{5}, \sqrt{7}) \subseteq \mathbb{Q}(\sqrt{5} + \sqrt{7})$.
\end{enumerate}
By (1)(2), $\mathbb{Q}(\sqrt{5}, \sqrt{7}) = \mathbb{Q}(\sqrt{5} + \sqrt{7})$.
$\Box$ \\\\



\textbf{Problem 1.1.9.}
\emph{If $K$ is an extension of $F$ such that $[K:F]$ is prime,
show that there are no intermediate fields between $K$ and $F$.} \\

\emph{Proof.}
Let $L$ be any field such that $F \subseteq L \subseteq K$.
By Proposition 1.20,
$$[K:F] = [K:L][L:F].$$
Since $[K:F]$ is prime, $[K:L] = 1$ or $[L:F] = 1$.
By Problem 1.1.2, $L=K$ or $L=F$,
or there are no intermediate fields between $K$ and $F$.
$\Box$ \\\\



\textbf{Problem 1.1.23.}
\emph{Recall that the characteristic of a ring $R$ with identity
is the smallest positive integer $n$ for which $n \cdot 1 = 0$,
if such an $n$ exists, or else the characteristic is $0$.
Let $R$ be a ring with identity.
Define $\varphi: \mathbb{Z} \rightarrow R$ by $\varphi(n) = n \cdot 1$,
where $1$ is the identity of $R$.
Show that $\varphi$ is a ring homomorphism
and that $\ker(\varphi) = m\mathbb{Z}$ for a unique nonnegative integer $m$,
and show that $m$ is the characteristic of $R$.} \\

\emph{Proof.}
\begin{enumerate}
\item[(1)]
  \emph{$\varphi$ is a ring homomorphism.}
  \begin{enumerate}
  \item[(a)]
    \emph{$\varphi(a+b) = \varphi(a) + \varphi(b)$.}
    $\varphi(a+b)
    = (a+b) \cdot 1
    = a \cdot 1 + b \cdot 1
    = \varphi(a) + \varphi(b)$.
  \item[(b)]
    \emph{$\varphi(ab) = \varphi(a) \varphi(b)$.}
    $\varphi(ab)
    = (ab) \cdot 1
    = (a \cdot 1)(b \cdot 1)
    = \varphi(a) \varphi(b)$
    since $1 \times 1 = 1$. (Here $\times$ is the multiplication operator of $R$.)
  \end{enumerate}
\item[(2)]
  \emph{$\ker(\varphi) = m\mathbb{Z}$ for a unique nonnegative integer $m$.}
  Since $\ker(\varphi)$ is an ideal of a PID $\mathbb{Z}$,
  there is a unique nonnegative integer $m$
  such that $\ker(\varphi) = m\mathbb{Z}$.
\item[(3)]
  \emph{$m$ is the characteristic of $R$.}
  There are only two possible cases,
  $\text{char}(R) = 0$ or else $\text{char}(R) > 0$.
  \begin{enumerate}
  \item[(a)]
    \emph{$\text{char}(R) = 0$.}
    $\ker(\varphi) = 0$. Thus $m = 0 = \text{char}(R)$.
  \item[(b)]
    \emph{$\text{char}(R) = n > 0$.} $n \in \ker(\varphi)$,
    so $m > 0$ and $m \mid n$.
    By the minimality of $n$, $m = n = \text{char}(R)$.
  \end{enumerate}
\end{enumerate}
$\Box$ \\\\



\textbf{Problem 1.1.24.}
\emph{For any positive integer $n$,
give an example of a ring of characteristic $n$.} \\

\emph{Proof.}
The ring $\mathbb{Z}/n\mathbb{Z}$.
$\Box$ \\\\



\textbf{Problem 1.1.25.}
\emph{If $R$ is an integral domain, show that either
$\text{char}(R) = 0$ or $\text{char}(R)$ is prime.} \\

\emph{Proof.}
\begin{enumerate}
\item[(1)]
  $1$ has infinite order. $\text{char}(R) = 0$. (Nothing to do.)
\item[(2)]
  $1$ has finite order $n$.
  Want to show $n$ is prime.
  If $n = ab$ where $a, b \in \mathbb{Z}^+$,
  then $$0 = n \cdot 1 = (a \cdot 1)(b \cdot 1).$$
  Since $R$ is an integral domain, $a \cdot 1 = $ or $b \cdot 1 = 0$.
  By the minimality of $n$, $a \geq n$ or $b \geq n$.
  $a = n$ or $b = n$. That is, $n$ is prime.
\end{enumerate}
$\Box$ \\\\



%%%%%%%%%%%%%%%%%%%%%%%%%%%%%%%%%%%%%%%%%%%%%%%%%%%%%%%%%%%%%%%%%%%%%%%%%%%%%%%%



\textbf{\large Section 1.2: Automorphisms} \\\\



%%%%%%%%%%%%%%%%%%%%%%%%%%%%%%%%%%%%%%%%%%%%%%%%%%%%%%%%%%%%%%%%%%%%%%%%%%%%%%%%



\textbf{Problem 1.2.1.}
\emph{Show that the only automorphism of $\mathbb{Q}$ is the identity.} \\

\emph{Proof.}
Given any $\sigma \in \text{Aut}(\mathbb{Q})$.
\begin{enumerate}
\item[(1)]
\emph{Show that $\sigma(1) = 1$.}
Since $1^2 = 1$, $\sigma(1)\sigma(1) = \sigma(1)$. $\sigma(1) = 0$ or $1$.
There are only two possible cases.
  \begin{enumerate}
  \item[(a)]
  Assume that $\sigma(1) = 0$. So
  $$\sigma(a) = \sigma(a \cdot 1) = \sigma(a)\cdot \sigma(1) = \sigma(a) \cdot 0 = 0$$
  for any $a \in \mathbb{Q}$.
  That is, $\sigma = 0 \in \text{Aut}(\mathbb{Q})$, which is absurd.
  \item[(b)]
  Therefore, $\sigma(1) = 1$.
  \end{enumerate}
\item[(2)]
\emph{Show that $\sigma(n) = n$ for all $n \in \mathbb{Z}^+$.}
Write $n = 1 + 1 + \cdots + 1$ ($n$ times $1$).
Applying the additivity of $\sigma$, we have
$$\sigma(n) = \sigma(1) + \sigma(1) + \cdots + \sigma(1) = 1 + 1 + \cdots + 1 = n.$$
(Might use induction on $n$ to eliminate $\cdots$ symbols.)
\item[(3)]
\emph{Show that $\sigma(n) = n$ for all $n \in \mathbb{Z}$.}
By the additivity of $\sigma$, $\sigma(-n) = -\sigma(n) = -n$ for $n \geq 0$.
The result is established.
\end{enumerate}
For any $a = \frac{n}{m} \in \mathbb{Q}$ ($m, n \in \mathbb{Z}$, $n \neq 0$),
applying the multiplication of $\sigma$ on $am = n$,
that is,
$\sigma(a) \sigma(m) = \sigma(n)$. By (3), we have $\sigma(a)m = n$,
or $$\sigma(a) = \frac{m}{n} = a$$
provided $n \neq 0$,
or $\sigma$ is the identity.
$\Box$ \\\\



%%%%%%%%%%%%%%%%%%%%%%%%%%%%%%%%%%%%%%%%%%%%%%%%%%%%%%%%%%%%%%%%%%%%%%%%%%%%%%%%



\textbf{Problem 1.2.2.}
\emph{Show that the only automorphism of $\mathbb{R}$ is the identity.
(Hint: If $\sigma$ is an automorphism, show that $\sigma|_{\mathbb{Q}} = \text{id}$,
and if $a > 0$, then $\sigma(a) > 0$.
It is an interesting fact that there are infinitely many automorphisms of $\mathbb{C}$,
even thought $[\mathbb{C}:\mathbb{R}] = 2$.
Why is this fact not a contradiction to this problem?)} \\

\emph{Proof (Hint).}
Given any $\sigma \in \text{Aut}(\mathbb{R})$.
\begin{enumerate}
\item[(1)]
Apply the same argument in Problem 1.2.1, we have $\sigma|_{\mathbb{Q}} = \text{id}$.
Notice that $\sigma(a) \neq 0$ for any $a \neq 0$.
\item[(2)]
\emph{Show that $\sigma(a) > 0$ if $a > 0$.}
Given any $a > 0$.
Write $a = \sqrt{a}\sqrt{a}$ (well-defined) and then apply
$\sigma$ on the both sides,
$$\sigma(a) = \sigma(\sqrt{a})\sigma(\sqrt{a}) = \sigma(\sqrt{a})^2 > 0$$
(since $\sqrt{a} \neq 0$ and thus $\sigma(\sqrt{a})$ cannot be zero).
\item[(3)]
\emph{Show that $\sigma(a) > \sigma(b)$ if $a > b$.}
It is a corollary to (2) by applying $\sigma$ on $a - b > 0$.
($\sigma(a - b) > 0$, or $\sigma(a) - \sigma(b) > 0$, or $\sigma(a) > \sigma(b)$.)
\item[(4)]
For any real number $x \in \mathbb{R}$,
choose two sequences $\{p_n\}, \{q_n\}$ of rational numbers
such that $p_n < x < q_n$ and $p_n, q_n \to x$ as $n \to \infty$.
Take $\sigma$ on the inequality, $\sigma(p_n) < \sigma(x) < \sigma(q_n)$.
So $p_n < \sigma(x) < q_n$ since $\sigma|_{\mathbb{Q}} = \text{id}$.
Let $n \to \infty$, we get $x \leq \sigma(x) \leq x$, or $\sigma(x) = x$.
\end{enumerate}
$\Box$ \\

\textbf{Supplement.}
Automorphisms of the Complex Numbers. by Paul B. Yale (Pomona College)
[\href{https://www.maa.org/sites/default/files/pdf/upload_library/22/Ford/PaulBYale.pdf}
{Link}]. \\\\


\end{document}