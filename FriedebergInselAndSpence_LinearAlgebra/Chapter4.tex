\documentclass{article}
\usepackage{amsfonts}
\usepackage{amsmath}
\usepackage{amssymb}
\usepackage{hyperref}
\usepackage{mathrsfs}
\usepackage{physics}
\parindent=0pt

\def\upint{\mathchoice%
    {\mkern13mu\overline{\vphantom{\intop}\mkern7mu}\mkern-20mu}%
    {\mkern7mu\overline{\vphantom{\intop}\mkern7mu}\mkern-14mu}%
    {\mkern7mu\overline{\vphantom{\intop}\mkern7mu}\mkern-14mu}%
    {\mkern7mu\overline{\vphantom{\intop}\mkern7mu}\mkern-14mu}%
  \int}
\def\lowint{\mkern3mu\underline{\vphantom{\intop}\mkern7mu}\mkern-10mu\int}

\begin{document}

\textbf{\Large Chapter 4: Determinants} \\\\



\emph{Author: Meng-Gen Tsai} \\
\emph{Email: plover@gmail.com} \\\\



%%%%%%%%%%%%%%%%%%%%%%%%%%%%%%%%%%%%%%%%%%%%%%%%%%%%%%%%%%%%%%%%%%%%%%%%%%%%%%%%



\textbf{\large Section 4.1: Determinants of Order $2$} \\\\



\textbf{Exercise 4.1.1.}
\emph{Label the following statements as being true or false.}
\begin{enumerate}
\item[(a)]
\emph{The function $\det: \mathsf{M}_{2 \times 2}(F) \to F$
is a linear transformation.}
\item[(b)]
\emph{The determinant of a $2 \times 2$ matrix is a linear function of each row
of the matrix when the other row is held fixed.}
\item[(c)]
\emph{If $A \in \mathsf{M}_{2 \times 2}(F)$ and $\det(A) = 0$, then $A$ is invertible.}
\item[(d)]
\emph{If $u$ and $v$ are vectors in $\mathbb{R}^2$ emanating from the origin,
then the area of the parallelogram having $u$ and $v$ as adjacent side is
$$\det\begin{pmatrix} u \\ v \end{pmatrix}.$$}
\item[(e)]
\emph{A coordinate system is right-handed if and only if its orientation equals $1$.} \\
\end{enumerate}

\emph{Proof of (a).}
False. Example 4.1.1, or
take
$$A =
\begin{pmatrix}
1 & 0 \\
0 & 0
\end{pmatrix} \in \mathsf{M}_{2 \times 2}(F) \text{ and }
B =
\begin{pmatrix}
0 & 0 \\
0 & 1
\end{pmatrix} \in \mathsf{M}_{2 \times 2}(F).
$$
Then $\det(A+B) = \det(I_2) = 1 \neq 0 = 0 + 0 = \det(A) + \det(B)$.
$\Box$ \\

\emph{Proof of (b).}
True. Proposition 4.1.
$\Box$ \\

\emph{Proof of (c).}
False. Proposition 4.2.
$\Box$ \\

\emph{Proof of (d).}
False. The area should be
$$O\begin{pmatrix} u \\ v \end{pmatrix}
\cdot
\det\begin{pmatrix} u \\ v \end{pmatrix}
= \abs{\det\begin{pmatrix} u \\ v \end{pmatrix}}.$$
$\Box$ \\

\emph{Proof of (e).}
True. See Exercise 4.1.12.
$\Box$ \\\\



%%%%%%%%%%%%%%%%%%%%%%%%%%%%%%%%%%%%%%%%%%%%%%%%%%%%%%%%%%%%%%%%%%%%%%%%%%%%%%%%



\textbf{Exercise 4.1.2.}
\emph{Compute the determinants of the following elements of
$\mathsf{M}_{2 \times 2}(\mathbb{R})$.
\begin{enumerate}
\item[(a)]
$\begin{pmatrix}
6 & -3 \\
2 & 4
\end{pmatrix}$ \\
\item[(b)]
$\begin{pmatrix}
-5 & 2 \\
6 & 1
\end{pmatrix}$ \\
\item[(c)]
$\begin{pmatrix}
8 & 0 \\
3 & -1
\end{pmatrix}$ \\
\end{enumerate}}

\emph{Proof of (a).}
$$\det\begin{pmatrix}
6 & -3 \\
2 & 4
\end{pmatrix}
= 6 \cdot 4 - (-3) \cdot 2 = 24 + 6 = 30.$$
$\Box$ \\

\emph{Proof of (b).}
$$\det\begin{pmatrix}
-5 & 2 \\
6 & 1
\end{pmatrix}
= (-5) \cdot 1 - 2 \cdot 6 = -5 - 12 = -17.$$
$\Box$ \\

\emph{Proof of (c).}
$$\det\begin{pmatrix}
8 & 0 \\
3 & -1
\end{pmatrix}
= 8 \cdot (-1) - 0 \cdot 3 = -8.$$
$\Box$ \\\\



%%%%%%%%%%%%%%%%%%%%%%%%%%%%%%%%%%%%%%%%%%%%%%%%%%%%%%%%%%%%%%%%%%%%%%%%%%%%%%%%



\textbf{Exercise 4.1.3.}
\emph{Compute the determinants of the following elements of
$\mathsf{M}_{2 \times 2}(\mathbb{C})$.
\begin{enumerate}
\item[(a)]
$\begin{pmatrix}
-1+i & 1-4i \\
3+2i & 2-3i
\end{pmatrix}$
\item[(b)]
$\begin{pmatrix}
5-2i & 6+4i \\
-3+i & 7i
\end{pmatrix}$
\item[(c)]
$\begin{pmatrix}
2i & 3 \\
4 & 6i
\end{pmatrix}$ \\
\end{enumerate}}

\emph{Proof of (a).}
\begin{align*}
\det\begin{pmatrix}
-1+i & 1-4i \\
3+2i & 2-3i
\end{pmatrix}
&= (-1+i) \cdot (2-3i) - (1-4i) \cdot (3+2i) \\
&= (1+5i) - (11-10i) \\
&= -10+15i.
\end{align*}
$\Box$ \\

\emph{Proof of (b).}
\begin{align*}
\det\begin{pmatrix}
5-2i & 6+4i \\
-3+i & 7i
\end{pmatrix}
&= (5-2i) \cdot (7i) - (6+4i) \cdot (-3+i) \\
&= (14+35i) - (-22-6i) \\
&= 36+41i.
\end{align*}
$\Box$ \\

\emph{Proof of (c).}
$$\det\begin{pmatrix}
2i & 3 \\
4 & 6i
\end{pmatrix}
= (2i) \cdot (6i) - 3 \cdot 4 = -12 - 12 = -24.$$
$\Box$ \\\\



%%%%%%%%%%%%%%%%%%%%%%%%%%%%%%%%%%%%%%%%%%%%%%%%%%%%%%%%%%%%%%%%%%%%%%%%%%%%%%%%



\textbf{Exercise 4.1.4.}
\emph{For each of the following pairs of vectors $u$ and $v$ in $\mathbb{R}^2$,
compute the area of the parallelogram determined by $u$ and $v$.}
\begin{enumerate}
\item[(a)]
\emph{$u = (3,-2)$ and $v=(2,5)$}
\item[(b)]
\emph{$u = (1,3)$ and $v=(-3,1)$}
\item[(c)]
\emph{$u = (4,-1)$ and $v=(-6,-2)$}
\item[(d)]
\emph{$u = (3,4)$ and $v=(2,-6)$} \\
\end{enumerate}

\emph{Proof of (a).}
$$\abs{\det\begin{pmatrix} 3 & -2 \\ 2 & 5 \end{pmatrix}}
= \abs{19} = 19.$$
$\Box$ \\

\emph{Proof of (b).}
$$\abs{\det\begin{pmatrix} 1 & 3 \\ -3 & 1 \end{pmatrix}}
= \abs{10} = 10.$$
$\Box$ \\

\emph{Proof of (c).}
$$\abs{\det\begin{pmatrix} 4 & -1 \\ -6 & -2 \end{pmatrix}}
= \abs{-14} = 14.$$
$\Box$ \\

\emph{Proof of (d).}
$$\abs{\det\begin{pmatrix} 3 & 4 \\ 2 & -6 \end{pmatrix}}
= \abs{-26} = 26.$$
$\Box$ \\\\



%%%%%%%%%%%%%%%%%%%%%%%%%%%%%%%%%%%%%%%%%%%%%%%%%%%%%%%%%%%%%%%%%%%%%%%%%%%%%%%%



\textbf{Exercise 4.1.5.}
\emph{Prove that if $B$ is the matrix obtained by interchanging the rows
of a $2 \times 2$ matrix $A$,
then $\det(B) = -\det(A)$.} \\

\emph{Proof.}
Write
$$A =
\begin{pmatrix}
a & b \\
c & d
\end{pmatrix} \in \mathsf{M}_{2 \times 2}(F).$$
Then
$$B =
\begin{pmatrix}
c & d \\
a & b
\end{pmatrix} \in \mathsf{M}_{2 \times 2}(F).$$

Then $\det(B) = cb - ad = -(ad - bc) = -\det(A)$.
$\Box$ \\\\



%%%%%%%%%%%%%%%%%%%%%%%%%%%%%%%%%%%%%%%%%%%%%%%%%%%%%%%%%%%%%%%%%%%%%%%%%%%%%%%%



\textbf{Exercise 4.1.6.}
\emph{Prove that if the two columns of $A \in \mathsf{M}_{2 \times 2}(F)$
are identical, then $\det(A) = 0$.} \\

\emph{Proof.}
By assumption, write
$$A =
\begin{pmatrix}
a & a \\
c & c
\end{pmatrix} \in \mathsf{M}_{2 \times 2}(F).$$
Then $\det(A) = ac - ac = 0$.
$\Box$ \\\\



%%%%%%%%%%%%%%%%%%%%%%%%%%%%%%%%%%%%%%%%%%%%%%%%%%%%%%%%%%%%%%%%%%%%%%%%%%%%%%%%



\textbf{Exercise 4.1.7.}
\emph{Prove that for any $A \in \mathsf{M}_{2 \times 2}(F)$,
$\det(A^t) = \det(A)$.} \\

\emph{Proof.}
Write
$$A =
\begin{pmatrix}
a & b \\
c & d
\end{pmatrix} \in \mathsf{M}_{2 \times 2}(F),$$
then
$$A^t =
\begin{pmatrix}
a & c \\
b & d
\end{pmatrix} \in \mathsf{M}_{2 \times 2}(F).$$
So $\det(A) = ad - bc = ad - cb = \det(A^t)$.
$\Box$ \\\\



%%%%%%%%%%%%%%%%%%%%%%%%%%%%%%%%%%%%%%%%%%%%%%%%%%%%%%%%%%%%%%%%%%%%%%%%%%%%%%%%



\textbf{Exercise 4.1.8.}
\emph{Prove that if $A \in \mathsf{M}_{2 \times 2}(F)$ is upper triangular,
then $\det(A)$ equals the product of the diagonal entries of $A$.} \\

\emph{Proof.}
Write
$$A =
\begin{pmatrix}
a & b \\
0 & d
\end{pmatrix} \in \mathsf{M}_{2 \times 2}(F)$$
since $A$ is upper triangular.
Then $\det(A) = ad$, which is equal to the product of the diagonal entries,
$a$ and $d$, of $A$.
$\Box$ \\\\



%%%%%%%%%%%%%%%%%%%%%%%%%%%%%%%%%%%%%%%%%%%%%%%%%%%%%%%%%%%%%%%%%%%%%%%%%%%%%%%%



\textbf{Exercise 4.1.9.}
\emph{Prove that for any $A, B \in \mathsf{M}_{2 \times 2}(F)$
we have $\det(AB) = \det(A) \cdot \det(B)$.} \\

\emph{Proof.}
Write
\begin{align*}
  A =
  \begin{pmatrix}
  a & b \\
  c & d
  \end{pmatrix} \in \mathsf{M}_{2 \times 2}(F), \\
  B =
  \begin{pmatrix}
  e & f \\
  g & h
  \end{pmatrix} \in \mathsf{M}_{2 \times 2}(F).
\end{align*}
Then
$$AB =
\begin{pmatrix}
ae + bg & af + bh \\
ce + dg & cf + dh
\end{pmatrix} \in \mathsf{M}_{2 \times 2}(F).$$
A direct calculation shows
\begin{align*}
\det(AB)
&= (ae + bg)(cf + dh) - (af + bh)(ce + dg) \\
&= (acef + adeh + bcfg + bdgh) - (acef + adfg + bceh + bdgh) \\
&= adeh + bcfg - adfg - bceh \\
&= (ad - bc)(eh - fg) \\
&= \det(A)\det(B).
\end{align*}
$\Box$ \\\\



%%%%%%%%%%%%%%%%%%%%%%%%%%%%%%%%%%%%%%%%%%%%%%%%%%%%%%%%%%%%%%%%%%%%%%%%%%%%%%%%



\textbf{Exercise 4.1.10.}
\emph{The \textbf{classical adjoint} of a $2 \times 2$ matrix
$A \in \mathsf{M}_{2 \times 2}(F)$ is the matrix
$$C =
\begin{pmatrix}
A_{22} & -A_{12} \\
-A_{21} & A_{11}
\end{pmatrix}.$$
Prove
\begin{enumerate}
\item[(a)]
$CA = AC = [\det(A)]I$.
\item[(b)]
$\det(C) = \det(A)$.
\item[(c)]
The classical adjoint of $A^t$ is $C^t$.
\item[(d)]
If $A$ is invertible, then $A^{-1} = [\det(A)]^{-1}C$. \\
\end{enumerate}}

Note that
$$A =
\begin{pmatrix}
A_{11} & A_{12} \\
A_{21} & A_{22}
\end{pmatrix}.$$ \\

\emph{Proof of (a).}
\begin{align*}
  CA
  &=
    \begin{pmatrix}
    A_{22} & -A_{12} \\
    -A_{21} & A_{11}
    \end{pmatrix}
    \begin{pmatrix}
    A_{11} & A_{12} \\
    A_{21} & A_{22}
    \end{pmatrix} \\
  &=
    \begin{pmatrix}
    A_{22}A_{11}-A_{12}A_{21}& A_{22}A_{12}-A_{12}A_{22} \\
    -A_{21}A_{11}+A_{11}A_{21} & -A_{21}A_{12}+A_{11}A_{22}
    \end{pmatrix} \\
  &=
    \begin{pmatrix}
    \det(A) & 0 \\
    0 & \det(A)
    \end{pmatrix} \\
  &= [\det(A)]I.
\end{align*}
\begin{align*}
  AC
  &=
    \begin{pmatrix}
    A_{11} & A_{12} \\
    A_{21} & A_{22}
    \end{pmatrix}
    \begin{pmatrix}
    A_{22} & -A_{12} \\
    -A_{21} & A_{11}
    \end{pmatrix} \\
  &=
    \begin{pmatrix}
    A_{11}A_{22}-A_{12}A_{21}& -A_{11}A_{12}+A_{12}A_{11} \\
    A_{21}A_{22}-A_{22}A_{21} & -A_{21}A_{12}+A_{22}A_{11}
    \end{pmatrix} \\
  &=
    \begin{pmatrix}
    \det(A) & 0 \\
    0 & \det(A)
    \end{pmatrix} \\
  &= [\det(A)]I.
\end{align*}
$\Box$ \\

\emph{Proof of (b).}
\begin{align*}
\det(C)
&= A_{22}A_{11} - (-A_{12})(-A_{21}) \\
&= A_{11}A_{22} - A_{12}A_{21} \\
&= \det(A).
\end{align*}
$\Box$ \\

\emph{Proof of (c).}
$$A^t =
\begin{pmatrix}
A_{11} & A_{21} \\
A_{12} & A_{22}
\end{pmatrix}.$$
The classical adjoint of $A^t$ is
$$
\begin{pmatrix}
A_{22} & -A_{21} \\
-A_{12} & A_{11}
\end{pmatrix} = C^t.$$
$\Box$ \\

\emph{Proof of (d).}
Proposition 4.2.
$\Box$ \\\\



%%%%%%%%%%%%%%%%%%%%%%%%%%%%%%%%%%%%%%%%%%%%%%%%%%%%%%%%%%%%%%%%%%%%%%%%%%%%%%%%



\textbf{Exercise 4.1.11.}
\emph{Let $\delta: \mathsf{M}_{2 \times 2}(F) \to F$
be a function with the following three properties.}
\begin{enumerate}
\item[(i)]
\emph{$\delta$ is a linear function of each row of the matrix when the other row
is held fixed.}
\item[(ii)]
\emph{If the two rows of $A \in \mathsf{M}_{2 \times 2}(F)$ are identical,
then $\delta(A) = 0$.}
\item[(iii)]
\emph{If $I$ is the $2 \times 2$ identity matrix, then $\delta(I) = 1$.}
\end{enumerate}
\emph{Prove that $\delta(A) = \det(A)$ for all $A \in \mathsf{M}_{2 \times 2}(F)$.
(This result is generalized in Section 4.5.)} \\

\emph{Proof.}
Write
$$A =
\begin{pmatrix}
A_{11} & A_{12} \\
A_{21} & A_{22}
\end{pmatrix}.$$

\begin{enumerate}
\item[(1)]
\emph{If $u, v$ are elements of $F^2$ and $k$ is a scalar, then
$$\delta\begin{pmatrix} u \\ v + ku \end{pmatrix}
= \delta\begin{pmatrix} u + kv \\ v \end{pmatrix}
= \delta\begin{pmatrix} u \\ v \end{pmatrix}.$$}
In fact,
\begin{align*}
\delta\begin{pmatrix} u \\ v + ku \end{pmatrix}
&= \delta\begin{pmatrix} u \\ v \end{pmatrix}
   + \delta\begin{pmatrix} u \\ ku \end{pmatrix}
  &\text{(Property (i))} \\
&= \delta\begin{pmatrix} u \\ v \end{pmatrix}
   + k \delta\begin{pmatrix} u \\ u \end{pmatrix}
  &\text{(Property (i))} \\
&= \delta\begin{pmatrix} u \\ v \end{pmatrix}.
  &\text{(Property (ii))}
\end{align*}
Similarly,
$\delta\begin{pmatrix} u + kv \\ v \end{pmatrix}
= \delta\begin{pmatrix} u \\ v \end{pmatrix}$.
\item[(2)]
\emph{If $u, v$ are elements of $F^2$, then
$$\delta\begin{pmatrix} u \\ v  \end{pmatrix}
= -\delta\begin{pmatrix} v \\ u \end{pmatrix}.$$}
In fact,
\begin{align*}
0
&= \delta\begin{pmatrix} u+v \\ u+v \end{pmatrix}
  &\text{(Property (ii))} \\
&= \delta\begin{pmatrix} u+v \\ u \end{pmatrix}
   + \delta\begin{pmatrix} u+v \\ v \end{pmatrix}
  &\text{(Property (i))} \\
&= \delta\begin{pmatrix} v \\ u \end{pmatrix}
   + \delta\begin{pmatrix} u \\ v \end{pmatrix}.
  &\text{((1))}
\end{align*}
\item[(3)]
\emph{If $v$ is an element of $F^2$, then
$$\delta\begin{pmatrix} 0 \\ v \end{pmatrix} = 0.$$}
In fact,
\begin{align*}
\delta\begin{pmatrix} 0 \\ v \end{pmatrix}
&= \delta\begin{pmatrix} 0+0 \\ v \end{pmatrix} \\
&= \delta\begin{pmatrix} 0 \\ v \end{pmatrix}
   + \delta\begin{pmatrix} 0 \\ v \end{pmatrix}.
  &\text{(Property (i))}
\end{align*}
In particular,
$\delta\begin{pmatrix} 0 \\ v \end{pmatrix} = 0
= \det\begin{pmatrix} 0 \\ v \end{pmatrix}$.
\item[(4)]
To show $\delta(A) = \det(A)$,
we consider three possible cases about the first row:
$A_{11} \neq 0$, $A_{12} \neq 0$, or $A_{11} = A_{12} = 0$.
The case $A_{11} = A_{12} = 0$ is proved in (3).
We prove the rest two cases in (5) and (6).
Write
$$u = (A_{11}, A_{12}) \text{ and } v = (A_{21}, A_{22}).$$
\item[(5)]
\emph{Show that $\delta(A) = \det(A)$ if $A_{11} \neq 0$.}
So
\begin{align*}
\delta(A)
&= \delta\begin{pmatrix} u \\ v \end{pmatrix} \\
&= \delta\begin{pmatrix} u \\ v - \frac{A_{21}}{A_{11}} u \end{pmatrix}
  &\text{((1))} \\
&= \delta\begin{pmatrix}
    A_{11} & A_{12} \\
    0 & A_{22}-\frac{A_{12}A_{21}}{A_{11}}
  \end{pmatrix} \\
&= \left( A_{22}-\frac{A_{12}A_{21}}{A_{11}} \right)
  \delta\begin{pmatrix}
    A_{11} & A_{12} \\
    0 & 1
  \end{pmatrix}
  &\text{(Property (i))} \\
&= \left( A_{22}-\frac{A_{12}A_{21}}{A_{11}} \right)
  \delta\begin{pmatrix}
    A_{11} & 0 \\
    0 & 1
  \end{pmatrix}
  &\text{((1))} \\
&= A_{11}\left( A_{22}-\frac{A_{12}A_{21}}{A_{11}} \right)
  \delta\begin{pmatrix}
    1 & 0 \\
    0 & 1
  \end{pmatrix}
  &\text{(Property (i))} \\
&= \det(A)\delta(I) \\
&= \det(A).
  &\text{(Property (iii))}
\end{align*}
\item[(6)]
\emph{Show that $\delta(A) = \det(A)$ if $A_{12} \neq 0$.}
So
\begin{align*}
\delta(A)
&= \delta\begin{pmatrix} u \\ v \end{pmatrix} \\
&= \delta\begin{pmatrix} u \\ v - \frac{A_{22}}{A_{12}} u \end{pmatrix}
  &\text{((1))} \\
&= \delta\begin{pmatrix}
    A_{11} & A_{12} \\
    A_{21}-\frac{A_{22}A_{11}}{A_{12}} & 0
  \end{pmatrix} \\
&= \left( A_{21}-\frac{A_{22}A_{11}}{A_{12}} \right)
  \delta\begin{pmatrix}
    A_{11} & A_{12} \\
    1 & 0
  \end{pmatrix}
  &\text{(Property (i))} \\
&= \left( A_{21}-\frac{A_{22}A_{11}}{A_{12}} \right)
  \delta\begin{pmatrix}
    0 & A_{12} \\
    1 & 0
  \end{pmatrix}
  &\text{((1))} \\
&= A_{12}\left( A_{21}-\frac{A_{22}A_{11}}{A_{12}} \right)
  \delta\begin{pmatrix}
    0 & 1 \\
    1 & 0
  \end{pmatrix}
  &\text{(Property (i))} \\
&= -A_{12}\left( A_{21}-\frac{A_{22}A_{11}}{A_{12}} \right)
  \delta\begin{pmatrix}
    1 & 0 \\
    0 & 1
  \end{pmatrix}
  &\text{((2))} \\
&= \det(A)\delta(I) \\
&= \det(A).
  &\text{(Property (iii))}
\end{align*}


\end{enumerate}
$\Box$ \\\\



%%%%%%%%%%%%%%%%%%%%%%%%%%%%%%%%%%%%%%%%%%%%%%%%%%%%%%%%%%%%%%%%%%%%%%%%%%%%%%%%



\textbf{Exercise 4.1.12.}
\emph{Let $\{ u, v \}$ be an ordered basis for $\mathbb{R}^2$.
Prove that
$$O\begin{pmatrix} u \\ v \end{pmatrix} = 1$$
if and only if $\{ u, v \}$ forms a right-handed coordinate system.
(Hint: Recall the definition of a rotation given in Example 2.1.2.)} \\

If $\beta = \{ u, v \}$ is an ordered basis for $\mathbb{R}^2$,
define the orientation of $\beta$ as
$$O\begin{pmatrix} u \\ v \end{pmatrix}
= \frac
{\det\begin{pmatrix} u \\ v \end{pmatrix}}
{\abs{\det\begin{pmatrix} u \\ v \end{pmatrix}}}.$$ \\

A coordinate system $\{ u, v \}$ is called right-handed if
$u$ can be rotated in a counterclockwise direction through an angle $\theta$
$(0 < \theta < \pi)$ to coincide with $v$. \\

\textbf{Example 2.1.2.}
For any angle $\theta$, define
$\mathsf{T}_{\theta}: \mathbb{R}^2 \to \mathbb{R}^2$ by
$$\mathsf{T}_{\theta}(a_1, a_2)
= (a_1 \cos\theta - a_2\sin\theta, a_1\sin\theta + a_2\cos\theta).$$
$\mathsf{T}_{\theta}$ is called the rotation by $\theta$.\\

\emph{Proof.}
\begin{enumerate}
\item[(1)]
By Example 2.1.2, for any coordinate system $\{ u, v \}$,
there is $0 < \theta < 2\pi$ and $\alpha > 0$ such that
$v = \alpha \mathsf{T}_{\theta}(u)$.
Write $u = (u_1, u_2) \in \mathbb{R}^2, v = (v_1, v_2) \in \mathbb{R}^2$.
\item[(2)]
Calculate $\det\begin{pmatrix} u \\ v \end{pmatrix}$.
\begin{align*}
\det\begin{pmatrix} u \\ v \end{pmatrix}
&= \det\begin{pmatrix} u \\ \alpha \mathsf{T}_{\theta}(u) \end{pmatrix} \\
&= \alpha \det\begin{pmatrix} u \\ \mathsf{T}_{\theta}(u) \end{pmatrix} \\
&= \alpha \det\begin{pmatrix}
  u_1 & u_2 \\
  u_1\cos\theta-u_2\sin\theta & u_1\sin\theta+u_2\cos\theta \end{pmatrix} \\
&= \alpha(u_1^2 + u_2^2) \sin\theta.
\end{align*}
\item[(3)]
\begin{align*}
O\begin{pmatrix} u \\ v \end{pmatrix} = 1
&\Longleftrightarrow
\det\begin{pmatrix} u \\ v \end{pmatrix} = \alpha(u_1^2 + u_2^2) \sin\theta > 0 \\
&\Longleftrightarrow
\sin\theta > 0 \\
&\Longleftrightarrow
0 < \theta < \pi \\
&\Longleftrightarrow
\{ u, v \} \text{ is a right-handed coordinate system}.
\end{align*}
\end{enumerate}
$\Box$ \\\\



% No exercises left.



%%%%%%%%%%%%%%%%%%%%%%%%%%%%%%%%%%%%%%%%%%%%%%%%%%%%%%%%%%%%%%%%%%%%%%%%%%%%%%%%
%%%%%%%%%%%%%%%%%%%%%%%%%%%%%%%%%%%%%%%%%%%%%%%%%%%%%%%%%%%%%%%%%%%%%%%%%%%%%%%%



\textbf{\large Section 4.2: Determinants of Order $n$} \\\\



\textbf{Exercise 4.2.2.}
\emph{Find the value of $k$ that satisfies the following equation.
$$\det
  \begin{pmatrix}
    3a_1 & 3a_2 & 3a_3 \\
    3b_1 & 3b_2 & 3b_3 \\
    3c_1 & 3c_2 & 3c_3
  \end{pmatrix}
= k\det
  \begin{pmatrix}
    a_1 & a_2 & a_3 \\
    b_1 & b_2 & b_3 \\
    c_1 & c_2 & c_3
  \end{pmatrix}$$} \\

\emph{Proof (Exercise 4.2.25).}
By Exercise 4.2.25,
$\det(3A) = 3^3 \det(A)$ for any $A \in \mathsf{M}_{3 \times 3}(F)$,
or $k = 3^3 = 27$.
$\Box$ \\\\



%%%%%%%%%%%%%%%%%%%%%%%%%%%%%%%%%%%%%%%%%%%%%%%%%%%%%%%%%%%%%%%%%%%%%%%%%%%%%%%%



\textbf{Exercise 4.2.26.}
\emph{Let $A \in \mathsf{M}_{n \times n}(F)$.
Under what conditions is $\det(-A) = \det(A)$?} \\

\emph{Proof (Exercise 4.2.25).}
By Exercise 4.2.25,
$\det(-A) = (-1)^n \det(A)$ for any $A \in \mathsf{M}_{n \times n}(F)$.
That is, $n$ is even if and only if $\det(-A) = \det(A)$.
$\Box$ \\\\



%%%%%%%%%%%%%%%%%%%%%%%%%%%%%%%%%%%%%%%%%%%%%%%%%%%%%%%%%%%%%%%%%%%%%%%%%%%%%%%%
%%%%%%%%%%%%%%%%%%%%%%%%%%%%%%%%%%%%%%%%%%%%%%%%%%%%%%%%%%%%%%%%%%%%%%%%%%%%%%%%



\textbf{\large Section 4.3: Properties of Determinants} \\\\



\textbf{Exercise 4.3.9.}
\emph{A matrix $M \in \mathsf{M}_{n \times n}(\mathbb{C})$
is called nilpotent if,
for some positive integer $k$, $M^k = O$, where $O$ is the $n \times n$ zero matrix.
Prove that if $M$ is nilpotent, then $\det(M) = 0$.} \\

\emph{Proof.}
Given any nilpotent matrix $M \in \mathsf{M}_{n \times n}(\mathbb{C})$
such that $M^k = O$ for some $k \in \mathbb{Z}^+$.
\begin{align*}
M^k = O
&\Longrightarrow
\det(M^k) = \det(O) \\
&\Longleftrightarrow
\det(M)^k = 0
  &\text{(Theorem 4.7)} \\
&\Longleftrightarrow
\det(M) = 0.
\end{align*}
$\Box$ \\\\



%%%%%%%%%%%%%%%%%%%%%%%%%%%%%%%%%%%%%%%%%%%%%%%%%%%%%%%%%%%%%%%%%%%%%%%%%%%%%%%%



\textbf{Exercise 4.3.11.}
\emph{A matrix $Q \in \mathsf{M}_{n \times n}(\mathbb{R})$
is called orthogonal if $QQ^t = I$.
Prove that if $Q$ is orthogonal, then $\det(Q) = \pm 1$.} \\

\emph{Proof.}
By the orthogonality of $Q$, $QQ^t = I$. So
\begin{align*}
QQ^t = I
&\Longrightarrow
\det(QQ^t) = \det(I) \\
&\Longleftrightarrow
\det(Q)\det(Q^t) = \det(I)
  &\text{(Theorem 4.7)} \\
&\Longleftrightarrow
\det(Q)\det(Q) = \det(I)
  &\text{(Theorem 4.8)} \\
&\Longleftrightarrow
\det(Q)^2 = 1
  &\text{(Example 4.2.4)} \\
&\Longleftrightarrow
\det(Q) = \pm 1.
\end{align*}
$\Box$ \\\\



%%%%%%%%%%%%%%%%%%%%%%%%%%%%%%%%%%%%%%%%%%%%%%%%%%%%%%%%%%%%%%%%%%%%%%%%%%%%%%%%



\textbf{Exercise 4.3.14.}
\emph{Prove that if $A, B \in \mathsf{M}_{n \times n}(F)$
are similar, then $\det(A) = \det(B)$.} \\

\emph{Proof.}
Since $A, B$ are similar, there exists an invertible matrix $Q$
such that $B = Q^{-1}AQ$.
So
\begin{align*}
\det(B)
&= \det(Q^{-1}AQ) \\
&= \det(Q^{-1})\det(A)\det(Q)
  &\text{(Theorem 4.7)} \\
&= \det(Q)\det(Q^{-1})\det(A)
  &\text{($F$ is field)} \\
&= \det(Q Q^{-1})\det(A)
  &\text{(Theorem 4.7)} \\
&= \det(I)\det(A) \\
&= 1 \cdot \det(A)
  &\text{(Example 4.2.4)} \\
&= \det(A).
\end{align*}
$\Box$ \\\\



\end{document}