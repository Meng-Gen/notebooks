\documentclass{article}
\usepackage{amsfonts}
\usepackage{amsmath}
\usepackage{amssymb}
\usepackage{hyperref}
\usepackage{mathrsfs}
\parindent=0pt

\def\upint{\mathchoice%
    {\mkern13mu\overline{\vphantom{\intop}\mkern7mu}\mkern-20mu}%
    {\mkern7mu\overline{\vphantom{\intop}\mkern7mu}\mkern-14mu}%
    {\mkern7mu\overline{\vphantom{\intop}\mkern7mu}\mkern-14mu}%
    {\mkern7mu\overline{\vphantom{\intop}\mkern7mu}\mkern-14mu}%
  \int}
\def\lowint{\mkern3mu\underline{\vphantom{\intop}\mkern7mu}\mkern-10mu\int}

\begin{document}

\textbf{\Large Chapter 4: Determinants} \\\\



\emph{Author: Meng-Gen Tsai} \\
\emph{Email: plover@gmail.com} \\\\



\textbf{\large Section 4.3: Properties of Determinants} \\\\



\textbf{Exercise 4.3.11.}
\emph{A matrix $Q \in \mathsf{M}_{n \times n}(\mathbb{R})$
is called orthogonal if $QQ^t = I$.
Prove that if $Q$ is orthogonal, then $\det(Q) = \pm 1$.} \\

\emph{Proof.}
By the orthogonality of $Q$, $QQ^t = I$. So
\begin{align*}
QQ^t = I
&\Longrightarrow
\det(QQ^t) = \det(I) \\
&\Longleftrightarrow
\det(Q)\det(Q^t) = \det(I)
  &\text{(Theorem 4.7)} \\
&\Longleftrightarrow
\det(Q)\det(Q) = \det(I)
  &\text{(Theorem 4.8)} \\
&\Longleftrightarrow
\det(Q)^2 = 1
  &\text{(Example 4.2.4)} \\
&\Longleftrightarrow
\det(Q) = \pm 1.
\end{align*}
$\Box$ \\\\



\textbf{Exercise 4.3.14.}
\emph{Prove that if $A, B \in \mathsf{M}_{n \times n}(F)$
are similar, then $\det(A) = \det(B)$.} \\

\emph{Proof.}
Since $A, B$ are similar, there exists an invertible matrix $Q$
such that $B = Q^{-1}AQ$.
So
\begin{align*}
\det(B)
&= \det(Q^{-1}AQ) \\
&= \det(Q^{-1})\det(A)\det(Q)
  &\text{(Theorem 4.7)} \\
&= \det(Q)\det(Q^{-1})\det(A)
  &\text{($F$ is field)} \\
&= \det(Q Q^{-1})\det(A)
  &\text{(Theorem 4.7)} \\
&= \det(I)\det(A) \\
&= 1 \cdot \det(A)
  &\text{(Example 4.2.4)} \\
&= \det(A).
\end{align*}
$\Box$ \\\\



\end{document}