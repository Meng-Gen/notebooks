\documentclass{article}
\usepackage{amsfonts}
\usepackage{amsmath}
\usepackage{amssymb}
\usepackage{hyperref}
\usepackage{mathrsfs}
\parindent=0pt

\def\upint{\mathchoice%
    {\mkern13mu\overline{\vphantom{\intop}\mkern7mu}\mkern-20mu}%
    {\mkern7mu\overline{\vphantom{\intop}\mkern7mu}\mkern-14mu}%
    {\mkern7mu\overline{\vphantom{\intop}\mkern7mu}\mkern-14mu}%
    {\mkern7mu\overline{\vphantom{\intop}\mkern7mu}\mkern-14mu}%
  \int}
\def\lowint{\mkern3mu\underline{\vphantom{\intop}\mkern7mu}\mkern-10mu\int}

\begin{document}

\textbf{\Large Chapter 6: Inner Product Spaces} \\\\



\emph{Author: Meng-Gen Tsai} \\
\emph{Email: plover@gmail.com} \\\\



\textbf{\large Section 6.1: Inner Products and Norms} \\\\



\textbf{Exercise 6.1.6.}
\emph{Complete the proof of Theorem 6.1.} \\

\textbf{Theorem 6.1}
\emph{Let $\mathsf{V}$ be an inner product space.
Then for $x, y, z, \in \mathsf{V}$ and $c \in F$
\begin{enumerate}
\item[(a)]
$\langle x, y+z \rangle = \langle x, y \rangle + \langle x, z \rangle$,
\item[(b)]
$\langle x, cy \rangle = \overline{c} \langle x, y \rangle$,
\item[(c)]
$\langle x, x \rangle = 0$ if and only if $x = 0$,
\item[(d)]
if $\langle x, y \rangle = \langle x, z \rangle$ for all $x \in \mathsf{V}$,
then $y = z$.
\end{enumerate}}

\emph{Proof of (a).}
\begin{align*}
\langle x, y+z \rangle
&= \overline{\langle y+z, x \rangle} \\
&= \overline{\langle y, x \rangle + \langle z, x \rangle} \\
&= \overline{\langle y, x \rangle} + \overline{\langle z, x \rangle} \\
&= \langle x, y \rangle + \langle x, z \rangle.
\end{align*}
$\Box$ \\

\emph{Proof of (b).}
\begin{align*}
\langle x, cy \rangle
&= \overline{\langle cy, x \rangle} \\
&= \overline{c \langle y, x \rangle} \\
&= \overline{c} \overline{\langle y, x \rangle} \\
&= \overline{c} \langle x, y \rangle.
\end{align*}
$\Box$ \\

\emph{Proof of (c).}
\begin{enumerate}
\item[(1)]
$(\Longrightarrow)$
If $x$ were nonzero, by the definition of the inner product,
$\langle x, x \rangle > 0$, contrary to the assumption.
Hence $x = 0$.
\item[(2)]
$(\Longleftarrow)$
Since $0 = 0+0$,
$\langle 0,0 \rangle = \langle 0+0,0 \rangle = \langle 0,0 \rangle + \langle 0,0 \rangle.$
Thus $\langle 0,0 \rangle = 0$.
\end{enumerate}
$\Box$ \\

\emph{Proof of (d).}
\begin{align*}
  \langle x,y \rangle = \langle x,z \rangle \:\: \forall x \in \mathsf{V}
  &\Longleftrightarrow
  0 = \langle x,y \rangle - \langle x,z \rangle \:\: \forall x \in \mathsf{V} \\
  &\Longleftrightarrow
  0 = \langle x,y-z \rangle \:\: \forall x \in \mathsf{V}
    &\text{((a))} \\
  &\Longrightarrow
  0 = \langle y-z,y-z \rangle
    &\text{(Take $x=y-z \in \mathsf{V}$)} \\
  &\Longleftrightarrow
  y-z = 0
    &\text{((c))} \\
  &\Longleftrightarrow
  y = z.
\end{align*}
$\Box$ \\\\



\end{document}