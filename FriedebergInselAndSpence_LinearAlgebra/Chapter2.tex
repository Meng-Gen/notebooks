\documentclass{article}
\usepackage{amsfonts}
\usepackage{amsmath}
\usepackage{amssymb}
\usepackage{hyperref}
\usepackage{mathrsfs}
\parindent=0pt

\def\upint{\mathchoice%
    {\mkern13mu\overline{\vphantom{\intop}\mkern7mu}\mkern-20mu}%
    {\mkern7mu\overline{\vphantom{\intop}\mkern7mu}\mkern-14mu}%
    {\mkern7mu\overline{\vphantom{\intop}\mkern7mu}\mkern-14mu}%
    {\mkern7mu\overline{\vphantom{\intop}\mkern7mu}\mkern-14mu}%
  \int}
\def\lowint{\mkern3mu\underline{\vphantom{\intop}\mkern7mu}\mkern-10mu\int}

\begin{document}

\textbf{\Large Chapter 2: Linear Transformations and Matrices} \\\\



\emph{Author: Meng-Gen Tsai} \\
\emph{Email: plover@gmail.com} \\\\



% https://math.stackexchange.com/questions/3852/if-ab-i-then-ba-i



\textbf{\large Section 2.4: Invertibility and Isomorphisms} \\\\



\textbf{Exercise 2.4.8.}
\emph{Let $A$ and $B$ be $n \times n$ matrices such that $AB = I_n$.
Prove
\begin{enumerate}
\item[(a)]
$A$ and $B$ are invertible.
\item[(b)]
$A = B^{-1}$ (and hence $B = A^{-1}$).
(We are in effect saying that for square matrices,
a ``one-sided'' inverse is a ``two-sided'' inverse.)
\item[(c)]
State and prove analogous results for linear transformations
defined on finite-dimensional vector spaces. \\
\end{enumerate}}

\emph{Proof of (a).}
Regard $\mathsf{V} = \mathsf{M}_{n \times n}(F)$ as a finite-dimensional vector space over $F$.
Given $X \in \mathsf{M}_{n \times n}(F)$,
consider the subset $\mathsf{V}_X$ of $\mathsf{V}$ defined by
$$\mathsf{V}_X = \{ XY : Y \in \mathsf{M}_{n \times n}(F) \}.$$
\begin{enumerate}
\item[(1)]
$\mathsf{V}_0 = 0$.
\item[(2)]
$\mathsf{V}_{I_n} = \mathsf{V}$.
In general,
$\mathsf{V}_X = \mathsf{V}$ for any invertible matrix $X \in \mathsf{M}_{n \times n}(F)$.
\item[(3)]
$\mathsf{V}_X$ is a subspace of $\mathsf{V}$ for any $X \in \mathsf{M}_{n \times n}(F)$.
\item[(4)]
There is a descending sequence of subspaces
$$\mathsf{V}
  \supseteq \mathsf{V}_X
  \supseteq \cdots
  \supseteq \mathsf{V}_{X^k}
  \supseteq \cdots
$$
This sequence must be stationary since $\mathsf{V}$ is finite-dimensional,
that is,
$$\mathsf{V}_{X^k} = \mathsf{V}_{X^{k+1}} = \cdots$$ for some $k$.
(Descending chain condition.)
In particular, $B^k = B^{k+1}C$ for some $C \in \mathsf{V}$.
Multiply with $A^k$ on the left to get $I_n = BC$.
($A^k B^k = A^{k-1}(AB)B^{k-1} = A^{k-1}B^{k-1} = \cdots = I_n$.)
\item[(4)]
Since $AB = I_n$ and $BC  = I_n$,
$A = AI_n = A(BC) = (AB)C = I_nC = C$,
or $AB = BA = I_n$.
By definition of invertibility, $A$ and $B$ are invertible.
\end{enumerate}
$\Box$ \\

\emph{Proof of (b).}
By (a), $A = B^{-1}$ and $B = A^{-1}$.
$\Box$ \\

\emph{Proof of (c).}
\emph{Let $\mathsf{V}$ be a finite-dimensional vector space,
and let $\mathsf{S}, \mathsf{T}: \mathsf{V} \to \mathsf{V}$ be linear
such that $\mathsf{S}\mathsf{T}$ is invertible.
Show that $\mathsf{S}$ and $\mathsf{T}$ are invertible.}
Let
$$\beta = \{ \beta_1, ..., \beta_n \}$$
be an ordered basis for $\mathsf{V}$ where $n = \dim(\mathsf{V})$.
Let $A = [\mathsf{S}]_\beta$ and $B = [\mathsf{T}]_\beta$.
So
$$AB
= [\mathsf{S}]_\beta [\mathsf{T}]_\beta
= [\mathsf{S} \mathsf{T}]_\beta
= [\mathsf{I}_{\mathsf{V}} ]_\beta = I_n$$ (Theorem 2.11).
By (a), $A = [\mathsf{S}]_\beta$ and $B = [\mathsf{T}]_\beta$ are invertible,
or $\mathsf{S}$ and $\mathsf{T}$ are invertible (Theorem 2.18).
$\Box$ \\\\



\end{document}