\documentclass{article}
\usepackage{amsfonts}
\usepackage{amsmath}
\usepackage{amssymb}
\usepackage{hyperref}
\usepackage[none]{hyphenat}
\usepackage{mathrsfs}
\usepackage{physics}
\parindent=0pt

\def\upint{\mathchoice%
    {\mkern13mu\overline{\vphantom{\intop}\mkern7mu}\mkern-20mu}%
    {\mkern7mu\overline{\vphantom{\intop}\mkern7mu}\mkern-14mu}%
    {\mkern7mu\overline{\vphantom{\intop}\mkern7mu}\mkern-14mu}%
    {\mkern7mu\overline{\vphantom{\intop}\mkern7mu}\mkern-14mu}%
  \int}
\def\lowint{\mkern3mu\underline{\vphantom{\intop}\mkern7mu}\mkern-10mu\int}

\begin{document}

\textbf{\Large Chapter 2: Linear Transformations and Matrices} \\\\



\emph{Author: Meng-Gen Tsai} \\
\emph{Email: plover@gmail.com} \\\\



% https://math.stackexchange.com/questions/3852/if-ab-i-then-ba-i



%%%%%%%%%%%%%%%%%%%%%%%%%%%%%%%%%%%%%%%%%%%%%%%%%%%%%%%%%%%%%%%%%%%%%%%%%%%%%%%%



\textbf{\large Section 2.4: Invertibility and Isomorphisms} \\\\



\textbf{Exercise 2.4.8.}
\emph{Let $A$ and $B$ be $n \times n$ matrices such that $AB = I_n$.
Prove}
\begin{enumerate}
\item[(a)]
\emph{$A$ and $B$ are invertible.}
\item[(b)]
\emph{$A = B^{-1}$ (and hence $B = A^{-1}$).
(We are in effect saying that for square matrices,
a ``one-sided'' inverse is a ``two-sided'' inverse.)}
\item[(c)]
\emph{State and prove analogous results for linear transformations
defined on finite-dimensional vector spaces.} \\
\end{enumerate}

\emph{Proof of (a).}
Regard $\mathsf{V} = \mathsf{M}_{n \times n}(F)$ as a finite-dimensional vector space over $F$.
Given $X \in \mathsf{M}_{n \times n}(F)$,
consider the subset $\mathsf{V}_X$ of $\mathsf{V}$ defined by
$$\mathsf{V}_X = \{ XY : Y \in \mathsf{M}_{n \times n}(F) \}.$$
\begin{enumerate}
\item[(1)]
$\mathsf{V}_0 = 0$.
\item[(2)]
$\mathsf{V}_{I_n} = \mathsf{V}$.
In general,
$\mathsf{V}_X = \mathsf{V}$ for any invertible matrix $X \in \mathsf{M}_{n \times n}(F)$.
\item[(3)]
$\mathsf{V}_X$ is a subspace of $\mathsf{V}$ for any $X \in \mathsf{M}_{n \times n}(F)$.
\item[(4)]
There is a descending sequence of subspaces
$$\mathsf{V}
  \supseteq \mathsf{V}_X
  \supseteq \cdots
  \supseteq \mathsf{V}_{X^k}
  \supseteq \cdots
$$
This sequence must be stationary since $\mathsf{V}$ is finite-dimensional,
that is,
$$\mathsf{V}_{X^k} = \mathsf{V}_{X^{k+1}} = \cdots$$ for some $k$.
(Descending chain condition.)
In particular, $B^k = B^{k+1}C$ for some $C \in \mathsf{V}$.
Multiply with $A^k$ on the left to get $I_n = BC$.
($A^k B^k = A^{k-1}(AB)B^{k-1} = A^{k-1}B^{k-1} = \cdots = I_n$.)
\item[(4)]
Since $AB = I_n$ and $BC  = I_n$,
$A = AI_n = A(BC) = (AB)C = I_nC = C$,
or $AB = BA = I_n$.
By definition of invertibility, $A$ and $B$ are invertible.
\end{enumerate}
$\Box$ \\

\emph{Proof of (b).}
By (a), $A = B^{-1}$ and $B = A^{-1}$.
$\Box$ \\

\emph{Proof of (c).}
\emph{Let $\mathsf{V}$ be a finite-dimensional vector space,
and let $\mathsf{S}, \mathsf{T}: \mathsf{V} \to \mathsf{V}$ be linear
such that $\mathsf{S}\mathsf{T}$ is invertible.
Show that $\mathsf{S}$ and $\mathsf{T}$ are invertible.}
Let
$$\beta = \{ \beta_1, ..., \beta_n \}$$
be an ordered basis for $\mathsf{V}$ where $n = \dim(\mathsf{V})$.
Let $A = [\mathsf{S}]_\beta$ and $B = [\mathsf{T}]_\beta$.
So
$$AB
= [\mathsf{S}]_\beta [\mathsf{T}]_\beta
= [\mathsf{S} \mathsf{T}]_\beta
= [\mathsf{I}_{\mathsf{V}} ]_\beta = I_n$$ (Theorem 2.11).
By (a), $A = [\mathsf{S}]_\beta$ and $B = [\mathsf{T}]_\beta$ are invertible,
or $\mathsf{S}$ and $\mathsf{T}$ are invertible (Theorem 2.18).
$\Box$ \\\\



%%%%%%%%%%%%%%%%%%%%%%%%%%%%%%%%%%%%%%%%%%%%%%%%%%%%%%%%%%%%%%%%%%%%%%%%%%%%%%%%



\textbf{\large Section 2.7: Homogeneous Linear Differential Equations
with Constant Coefficients} \\\\



\textbf{Exercise 2.7.3.}
\emph{Find a basis for the solution space of each of the following differential
equations}
\begin{enumerate}
\item[(a)]
$y''+2y'+y = 0$
\item[(b)]
$y'''=y'$
\item[(c)]
$y^{(4)} - 2y^{(2)} + y = 0$
\item[(d)]
$y''+2y'+y = 0$
\item[(e)]
$y^{(3)} - y^{(2)} + 3y^{(1)} + 5y = 0$. \\
\end{enumerate}

Use Theorem 2.35. \\

\emph{Proof of (a).}
The auxiliary polynomial is $t^2+ty+1 = (t+1)^2$.
$\{ e^{-t}, te^{-t} \}$ is a basis for the solution space.
$\Box$ \\

\emph{Proof of (b).}
The auxiliary polynomial is $t^3-t = t(t-1)(t+1)$.
$\{ 1, e^{t}, e^{-t} \}$ is a basis for the solution space.
$\Box$ \\

\emph{Proof of (c).}
The auxiliary polynomial is $t^4-2t^2+1 = (t-1)^2(t+1)^2$.
$\{ e^{t}, te^{t}, e^{-t}, te^{-t} \}$ is a basis for the solution space.
$\Box$ \\

\emph{Proof of (d).}
Same as (a).
$\Box$ \\

\emph{Proof of (e).}
The auxiliary polynomial is $$t^3-t^2+3t+5 = (t+1)(t-1-2i)(t-1+2i).$$
$\{ e^{-t}, e^{(1+2i)t}, e^{(1-2i)t} \}$,
or $\{ e^{-t}, e^{t}\cos(2t), e^{t}\sin(2t) \}$
is a basis for the solution space.
$\Box$ \\\\



%%%%%%%%%%%%%%%%%%%%%%%%%%%%%%%%%%%%%%%%%%%%%%%%%%%%%%%%%%%%%%%%%%%%%%%%%%%%%%%%



\textbf{Exercise 2.7.4.}
\emph{Find a basis for each of the following subspaces of $\mathsf{C}^{\infty}$.}
\begin{enumerate}
\item[(a)]
$\mathsf{N}(\mathsf{D}^2-\mathsf{D}-\mathsf{I})$
\item[(b)]
$\mathsf{N}(\mathsf{D}^3-3\mathsf{D}^2+3\mathsf{D}-\mathsf{I})$
\item[(c)]
$\mathsf{N}(\mathsf{D}^3-6\mathsf{D}^2-8\mathsf{D})$ \\
\end{enumerate}

Use Theorem 2.35. \\

\emph{Proof of (a).}
The auxiliary polynomial is
$$t^2-t-1 = \left(t-\frac{1+\sqrt{5}}{2}\right)\left(t-\frac{1-\sqrt{5}}{2}\right).$$
$\left\{ e^{\frac{1+\sqrt{5}}{2}t}, e^{\frac{1-\sqrt{5}}{2}t} \right\}$
is a basis for the solution space.
$\Box$ \\

\emph{Proof of (b).}
The auxiliary polynomial is
$t^3-3t^2+3t-1 = (t-1)^3.$
$\{ e^{t}, te^{t}, t^2e^{t} \}$
is a basis for the solution space.
$\Box$ \\

\emph{Proof of (c).}
The auxiliary polynomial is
$t^3+6t^2+8t = t(t+2)(t+4).$
$\{ 1, e^{-2t}, e^{-4t} \}$
is a basis for the solution space.
$\Box$ \\\\



%%%%%%%%%%%%%%%%%%%%%%%%%%%%%%%%%%%%%%%%%%%%%%%%%%%%%%%%%%%%%%%%%%%%%%%%%%%%%%%%



\textbf{Exercise 2.7.5.}
\emph{Show that $\mathsf{C}^{\infty}$ is a subspace of
$\mathcal{F}(\mathbb{R}, \mathbb{C})$. } \\

\emph{Proof.}
\begin{enumerate}
\item[(1)]
$0 \in \mathcal{F}(\mathbb{R}, \mathbb{C})$ clearly.
\item[(2)]
Given any $f, g \in \mathsf{C}^{\infty}$.
For any nonnegative $k$,
$\mathsf{D}^k(f+g) = \mathsf{D}^k(f) + \mathsf{D}^k(g)$ holds.
Thus $f+g \in \mathsf{C}^{\infty}$.
\item[(3)]
Given any $f \in \mathcal{F}(\mathbb{R}, \mathbb{C})$, $r \in \mathbb{C}$.
For any nonnegative $k$,
$\mathsf{D}^k(cf) = c\mathsf{D}^k(f)$ holds.
Thus $cf \in \mathsf{C}^{\infty}$.
\end{enumerate}
By Theorem 1.3, $\mathsf{C}^{\infty}$ is a subspace.
$\Box$ \\\\



%%%%%%%%%%%%%%%%%%%%%%%%%%%%%%%%%%%%%%%%%%%%%%%%%%%%%%%%%%%%%%%%%%%%%%%%%%%%%%%%



\end{document}