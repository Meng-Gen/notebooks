\documentclass{article}
\usepackage{amsfonts}
\usepackage{amsmath}
\usepackage{amssymb}
\usepackage{hyperref}
\usepackage{mathrsfs}
\parindent=0pt

\def\upint{\mathchoice%
    {\mkern13mu\overline{\vphantom{\intop}\mkern7mu}\mkern-20mu}%
    {\mkern7mu\overline{\vphantom{\intop}\mkern7mu}\mkern-14mu}%
    {\mkern7mu\overline{\vphantom{\intop}\mkern7mu}\mkern-14mu}%
    {\mkern7mu\overline{\vphantom{\intop}\mkern7mu}\mkern-14mu}%
  \int}
\def\lowint{\mkern3mu\underline{\vphantom{\intop}\mkern7mu}\mkern-10mu\int}

\begin{document}

\textbf{\Large Chapter 1: Vector Spaces} \\\\



\emph{Author: Meng-Gen Tsai} \\
\emph{Email: plover@gmail.com} \\\\



\textbf{\large Section 1.6: Bases and Dimension} \\\\

\textbf{Exercise 1.6.19.}
\emph{Let $\mathsf{V}$ be a vector space having dimension $n$,
and let $S$ be a subset of $\mathsf{V}$ that generates $\mathsf{V}$.
\begin{enumerate}
\item[(a)]
Prove that there is a subset of $S$ that is a basis for $\mathsf{V}$.
(Be careful not to assume that $S$ is finite.)
\item[(b)]
Prove that $S$ contains at least $n$ elements.
\end{enumerate}}

\emph{Proof of (a).}
Similar to the argument in Theorem 1.9.
\begin{enumerate}
\item[(1)]
If $S = \varnothing$ or $S = \{0\}$, then $\mathsf{V} = \{0\}$
and $\varnothing$ is a subset of $S$ that is a basis for $\mathsf{V}$.
\item[(2)]
Otherwise $S$ contains a nonzero element $u_1$.
$\{u_1\}$ is a linearly independent set.
Continue, if possible, choosing elements $u_2, ..., u_k$ in $S$ such that
$\{u_1, u_2, ..., u_k\}$ is linearly independent.
By the Replacement Theorem (Theorem 1.10),
we must eventually reach a stage at which
$\beta = \{u_1, u_2, ..., u_k\}$ is a linearly independent subset of $S$
with $k \leq n$.
\item[(3)]
$\beta$ generates $S$ by the construction of $\beta$,
and $S$ generates $\mathsf{V}$.
Therefore, $\beta$ generates $\mathsf{V}$
(and thus $k = n$ by the definition of dimension).
\end{enumerate}
Therefore,
there is a subset of $S$ that is a basis for $\mathsf{V}$.
$\Box$ \\

\emph{Proof of (b).}
By (a), there is a subset $\beta \subseteq S$ of size $n$
that is a basis for $\mathsf{V}$.
So $S$ contains at least $n$ elements of $\beta$.
$\Box$ \\\\



\end{document}