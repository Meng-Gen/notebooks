\documentclass{article}
\usepackage{amsfonts}
\usepackage{amsmath}
\usepackage{amssymb}
\usepackage{hyperref}
\usepackage[none]{hyphenat}
\usepackage{mathrsfs}
\usepackage{physics}
\parindent=0pt



\title{\textbf{Solutions to the book: \\
\emph{do Carmo, Differential Geometry of Curves and Surfaces}}}
\author{Meng-Gen Tsai \\ plover@gmail.com}



\begin{document}
\maketitle
\tableofcontents



%%%%%%%%%%%%%%%%%%%%%%%%%%%%%%%%%%%%%%%%%%%%%%%%%%%%%%%%%%%%%%%%%%%%%%%%%%%%%%%%
%%%%%%%%%%%%%%%%%%%%%%%%%%%%%%%%%%%%%%%%%%%%%%%%%%%%%%%%%%%%%%%%%%%%%%%%%%%%%%%%



\newpage
\section*{Chapter 1: Curves \\}
\addcontentsline{toc}{section}{Chapter 1: Curves}



% http://www.math.ualberta.ca/~xinweiyu/348.A1.17f/L07_Curves_I_20170926.pdf
% https://mast.queensu.ca/~offind/math280/solutions05.pdf



%%%%%%%%%%%%%%%%%%%%%%%%%%%%%%%%%%%%%%%%%%%%%%%%%%%%%%%%%%%%%%%%%%%%%%%%%%%%%%%%
%%%%%%%%%%%%%%%%%%%%%%%%%%%%%%%%%%%%%%%%%%%%%%%%%%%%%%%%%%%%%%%%%%%%%%%%%%%%%%%%



\subsection*{1-1. Introduction \\}
\addcontentsline{toc}{subsection}{1-1. Introduction}

Classical differential geometry: the study of local properties of
curves and surfaces. \\

Global differential geometry: the study of the influence of the local properties
on the behavior of the entire curve and surface. \\

\emph{No exercises.} \\\\



%%%%%%%%%%%%%%%%%%%%%%%%%%%%%%%%%%%%%%%%%%%%%%%%%%%%%%%%%%%%%%%%%%%%%%%%%%%%%%%%
%%%%%%%%%%%%%%%%%%%%%%%%%%%%%%%%%%%%%%%%%%%%%%%%%%%%%%%%%%%%%%%%%%%%%%%%%%%%%%%%



\subsection*{1-2. Parametrized Curves \\}
\addcontentsline{toc}{subsection}{1-2. Parametrized Curves}



\subsubsection*{Exercise 1-2.1.}
\addcontentsline{toc}{subsubsection}{Exercise 1-2.1.}
\emph{Find a parametrized curve $\alpha(t)$ whose trace is the circle
$x^2 + y^2 = 1$ such that $\alpha(t)$ runs clockwise around the circle with
$\alpha(0) = (0,1)$.} \\



\emph{Proof.}
$\alpha(t) = (\sin t, \cos t)$, $t \in \mathbb{R}$.
$\Box$ \\\\



%%%%%%%%%%%%%%%%%%%%%%%%%%%%%%%%%%%%%%%%%%%%%%%%%%%%%%%%%%%%%%%%%%%%%%%%%%%%%%%%



\subsubsection*{Exercise 1-2.2.}
\addcontentsline{toc}{subsubsection}{Exercise 1-2.2.}
\emph{Let $\alpha(t)$ be a parametrized curve which does not pass through the origin.
If $\alpha(t_0)$ is the point of the trace of $\alpha$ closest to the origin
and $\alpha'(t_0) \neq 0$, show that the position vector $\alpha(t_0)$ is
orthogonal to $\alpha'(t_0)$.} \\



\emph{Proof.}
Let $f(t) = |\alpha(t)|^2 = \alpha(t) \cdot \alpha(t)$.
$f(t)$ is differentiable and $f(t)$ has a local minimum at a point $t = t_0 \in I$.
So $f'(t_0) = 0$. [Theorem 5.8 in \emph{W. Rudin, Principles of Mathematical Analysis, 3rd edition}.]
Since
$$f'(t) = 2 \alpha(t) \cdot \alpha'(t),$$
$f'(t_0) = 2 \alpha(t_0) \cdot \alpha'(t_0) = 0$,
or $\alpha(t_0) \cdot \alpha'(t_0) = 0$.
Since $\alpha(t_0) \neq 0$ and $\alpha'(t_0) \neq 0$,
$\alpha(t_0)$ is orthogonal to $\alpha'(t_0)$.
$\Box$ \\\\



%%%%%%%%%%%%%%%%%%%%%%%%%%%%%%%%%%%%%%%%%%%%%%%%%%%%%%%%%%%%%%%%%%%%%%%%%%%%%%%%



\subsubsection*{Exercise 1-2.3.}
\addcontentsline{toc}{subsubsection}{Exercise 1-2.3.}
\emph{A parametrized curve $\alpha(t)$ has a property that its second derivative
$\alpha''(t)$ is identically zero.
What can be said about $\alpha$?} \\



\emph{Proof.}
\begin{enumerate}
\item[(1)]
  $\alpha(t)$ is a straight line.

\item[(2)]
  Since $\alpha''(t)$ is identically zero,
  $\alpha'(t) = a$ is a constant.
  [Theorem 5.11 in \emph{W. Rudin, Principles of Mathematical Analysis, 3rd edition.}]
  Define $f(t) = \alpha(t) - at$ (on $I$).
  Since $f'(t) = \alpha'(t) - a = 0$,
  $f(t) = \alpha(t) - at = b$ is a constant again.
  Therefore, $\alpha(t) = at+b$, which is a straight line (on $I$).
\end{enumerate}
$\Box$ \\\\



%%%%%%%%%%%%%%%%%%%%%%%%%%%%%%%%%%%%%%%%%%%%%%%%%%%%%%%%%%%%%%%%%%%%%%%%%%%%%%%%



\subsubsection*{Exercise 1-2.4.}
\addcontentsline{toc}{subsubsection}{Exercise 1-2.4.}
\emph{Let $\alpha: I \to \mathbb{R}^3$ be a parametrized curve and
let $v \in \mathbb{R}^3$ be a fixed vector.
Assume that $\alpha'(t)$ is orthogonal to $v$ for all $t \in I$
and that $\alpha(0)$ is orthogonal to $v$.
Prove that $\alpha(t)$ is orthogonal to $v$ for all $t \in I$.} \\

Need to assume that $\alpha(t) \neq 0$ for all $t \in I$. \\



\emph{Proof.}
Given any $t \neq 0 \in I$. (Nothing to do at $t = 0$.)
Define $f: I \to \mathbb{R}$ by $f(t) = \alpha(t) \cdot v$.
By the mean value theorem, there exists a point $\xi$ between $0$ and $t$
such that
$$f(t) - f(0) = f'(\xi)(t - 0),$$
where $f'(t) = \alpha'(t) \cdot v + \alpha(t) \cdot v' = \alpha'(t) \cdot v$.
Note that $f(0) = 0$ since $\alpha(0)$ is orthogonal to $v$,
and $f'(\xi) = 0$ since $\alpha'(t)$ is orthogonal to $v$.
So the identity is reduced to
$$f(t) = 0,$$
or $\alpha(t) \cdot v = 0$,
or $\alpha(t)$ is orthogonal to $v$.
$\Box$ \\\\



%%%%%%%%%%%%%%%%%%%%%%%%%%%%%%%%%%%%%%%%%%%%%%%%%%%%%%%%%%%%%%%%%%%%%%%%%%%%%%%%



\subsubsection*{Exercise 1-2.5.}
\addcontentsline{toc}{subsubsection}{Exercise 1-2.5.}
\emph{Let $\alpha: I \to \mathbb{R}^3$ be a parametrized curve,
with $\alpha'(t) \neq 0$ for all $t \in I$.
Show that $|\alpha(t)|$ is a nonzero constant
if and only if
$\alpha(t)$ is orthogonal to $\alpha'(t)$ for all $t \in I$.} \\

The same trick in Exercise 1-2.2. \\



\emph{Proof.}
It is equivalent to
\emph{show that $|\alpha(t)|^2$ is a nonzero constant
if and only if
$\alpha(t)$ is orthogonal to $\alpha'(t)$ for all $t \in I$.}
Let $$f(t) = |\alpha(t)|^2 = \alpha(t) \cdot \alpha(t).$$
Notice that $\alpha'(t) \neq 0$, and thus
\begin{align*}
  & \: |\alpha(t)| \text{ is a nonzero constant} \\
  \Longleftrightarrow& \:
  f(t) = |\alpha(t)|^2 \text{ is a nonzero constant} \\
  \Longleftrightarrow& \:
  f'(t) = 0 \text{ and } f(t) \text{ is a nonzero constant} \\
  \Longleftrightarrow& \:
  \alpha(t) \cdot \alpha'(t) = 0 \text{ and } \alpha(t) \text{ is a nonzero constant} \\
  \Longleftrightarrow& \:
  \alpha(t) \text{ is orthogonal to } \alpha'(t) \text{ for all } t \in I.
\end{align*}
$\Box$ \\\\



%%%%%%%%%%%%%%%%%%%%%%%%%%%%%%%%%%%%%%%%%%%%%%%%%%%%%%%%%%%%%%%%%%%%%%%%%%%%%%%%
%%%%%%%%%%%%%%%%%%%%%%%%%%%%%%%%%%%%%%%%%%%%%%%%%%%%%%%%%%%%%%%%%%%%%%%%%%%%%%%%



\subsection*{1-3. Regular Curves; Arc Length \\}
\addcontentsline{toc}{subsection}{1-3. Regular Curves; Arc Length}



\subsubsection*{Exercise 1-3.1.}
\addcontentsline{toc}{subsubsection}{Exercise 1-3.1.}
\emph{Show that the tangent lines to the regular parametrized curve
$\alpha(t) = (3t, 3t^2, 2t^3)$ make a constant angle with the line $y = 0$, $z = x$. } \\



\emph{Proof.}
$\alpha'(t) = (3, 6t, 6t^2)$.
The line $y = 0$, $z = x$ is $\beta(t) = (1,0,1)$.
The cosine of the angle $\theta$ between these to curves is
\begin{align*}
  \cos \theta
  &= \frac{(3, 6t, 6t^2) \cdot (1,0,1)}{|(3, 6t, 6t^2)||(1,0,1)|} \\
  &= \frac{3+6t^2}{\sqrt{3^2 + (6t)^2 + (6t^2)^2}\sqrt{2}} \\
  &= \frac{3+6t^2}{\sqrt{9 + 36 t^2 + 36t^4}\sqrt{2}} \\
  &= \frac{3+6t^2}{\sqrt{(3 + 6t^2)^2}\sqrt{2}} \\
  &= \frac{1}{\sqrt{2}}.
\end{align*}
(Notice $3 + 6t^2 > 0$ for all $t \in \mathbb{R}$.)
That is, the angle between $\alpha'$ and $\beta$ is a constant ($= \pi/4$).
$\Box$ \\\\



%%%%%%%%%%%%%%%%%%%%%%%%%%%%%%%%%%%%%%%%%%%%%%%%%%%%%%%%%%%%%%%%%%%%%%%%%%%%%%%%



\subsubsection*{Exercise 1-3.2. (Cycloid)}
\addcontentsline{toc}{subsubsection}{Exercise 1-3.2. (Cycloid)}
\emph{A circular disk of radius $1$ in the plane $xy$ rolls without slipping
along the $x$ axis.
The figure described by a point of the circumference of of the disk is
called a \textbf{cycloid}
(Figure 1-7 in Mantredo P. do Carmo, Differential Geometry of Curves and Surfaces).}
\begin{enumerate}
\item[(a)]
  \emph{Obtain a parametrized curve $\alpha: \mathbb{R} \to \mathbb{R}^2$
  the trace of which is the cycloid and determine its singular points.}

\item[(b)]
  \emph{Compute the arc length of the cycloid
  corresponding to a complete rotation of the disk.} \\
\end{enumerate}



\emph{Proof of (a).}
\begin{enumerate}
\item[(1)]
Since
\begin{equation*}
  \begin{cases}
     x = t - \sin t \\
     y = 1 - \cos t,
  \end{cases}
\end{equation*}
we define $\alpha(t) = (t - \sin t, 1 - \cos t)$.
\item[(2)]
$\alpha'(t) = (1 - \cos t, \sin t)$.
$\alpha'(t) = 0$ if and only if $t = 2n\pi$ where $n \in \mathbb{Z}$.
That is, all singular points are $\alpha(2n\pi) = (2n\pi, 0)$ where $n \in \mathbb{Z}$.
\end{enumerate}
$\Box$ \\

\emph{Proof of (b).}
The arc length of the cycloid corresponding to a complete rotation of the disk is
\begin{align*}
  \int_{0}^{2\pi} |\alpha'(t)| dt
  &= \int_{0}^{2\pi} \sqrt{(1-\cos t)^2 + (\sin t)^2} dt \\
  &= \int_{0}^{2\pi} \sqrt{2} \sqrt{1 - \cos t} dt \\
  &= \int_{0}^{2\pi} 2 \sin \frac{t}{2} dt \\
  &= \left[ -4 \cos\frac{t}{2} \right]_{t=0}^{t=2\pi} \\
  &= 8.
\end{align*}
$\Box$ \\

\textbf{Supplement.}
The cycloid is not an algebraic curve. \\\\



%%%%%%%%%%%%%%%%%%%%%%%%%%%%%%%%%%%%%%%%%%%%%%%%%%%%%%%%%%%%%%%%%%%%%%%%%%%%%%%%



\subsubsection*{Exercise 1-3.3. (Cissoid of Diocles)}
\addcontentsline{toc}{subsubsection}{Exercise 1-3.3. (Cissoid of Diocles)}
\emph{Let $0A = 2a$ be the diameter of a circle $\mathbb{S}^1$ and
$0Y$ and $AV$ be the tangents to $\mathbb{S}^1$ at $0$ and $A$, respectively.
A half-line $r$ is drawn from $0$ which
meets the circle $\mathbb{S}^1$ at $C$ and the line $AV$ at $B$.
On $0B$ mark off the segment $0p = CB$.
If we rotate $r$ about $0$, the point $p$ will describe a curve called
the \textbf{cissoid of Diocles}.
By taking $0A$ as the $x$ axis and $0Y$ as the $y$ axis, prove that}
\begin{enumerate}
\item[(a)]
  \emph{The tract of
  \[
    \alpha(t) = \left( \frac{2at^2}{1+t^2}, \frac{2at^3}{1+t^2} \right),
    \qquad
    t \in \mathbb{R},
  \]
  is the cissoid of Diocles ($t = \tan\theta$;
  see Figure 1-8 in Mantredo P. do Carmo, Differential Geometry of Curves and Surfaces).}

\item[(b)]
  \emph{The origin $(0,0)$ is a singular point of the cissoid.}

\item[(c)]
  \emph{As $t \to \infty$, $\alpha(t)$ approaches the line $x = 2a$,
  and $\alpha'(t) \to (0,2a)$.
  Thus, as $t \to \infty$, the curve and its tangent approach the line $x = 2a$;
  we say that $x = 2a$ is an \textbf{asymptote} to the cissoid.} \\
\end{enumerate}



\emph{Proof of (a).}
\begin{enumerate}
\item[(1)]
  The polar equations of the circle $\mathbb{S}^1$ and the half-line $r$ is
  \begin{align*}
    r &= 2a\cos\theta, \\
    r &= 2a\sec\theta,
  \end{align*}
  respectively.

\item[(2)]
  By construction, the polar equation of the cissoid is
  \[
    r
    = 2a\sec\theta - 2a\cos\theta
    = 2a\frac{\sin^2\theta}{\cos\theta}
    = 2a\sin\theta \tan\theta.
  \]

\item[(3)]
  Put $t = \tan\theta$, we have
  \begin{align*}
    x &= r \cos\theta = 2a \sin^2 \theta = \frac{2at^2}{1+t^2}, \\
    y &= r \sin\theta = tx = \frac{2at^3}{1+t^2}.
  \end{align*}
  So
  \[
    \alpha(t) = (x,y) = \left( \frac{2at^2}{1+t^2}, \frac{2at^3}{1+t^2} \right).
  \]
\end{enumerate}
$\Box$ \\

\textbf{Supplement.}
The cissoid is an algebraic curve $= V((x^2+y^2)x = 2ay^2)$. \\



\emph{Proof of (b).}
Note that $\alpha(0) = (0,0)$ and
\[
  \alpha'(t) = \left( \frac{4at}{(t^2+1)^2}, \frac{2at^2(t^2+3)}{(t^2+1)^2} \right).
\]
Hence $\alpha'(0) = (0,0)$. That is, $(0,0)$ is a singular point of the cissoid.
(In fact, the origin is the unique singular point of the cissoid.)
$\Box$ \\



\emph{Proof of (c).}
\begin{enumerate}
\item[(1)]
  Note that
  \begin{align*}
    \lim_{t \to \pm \infty} x(t)
    &= \lim_{t \to \pm \infty} \frac{2at^2}{1+t^2}
    = 2a, \\
    \lim_{t \to \pm \infty} y(t)
    &= \lim_{t \to \pm \infty} \frac{2at^3}{1+t^2}
    = \pm \infty.
  \end{align*}
  Hence, $\alpha(t)$ approaches the line $x = 2a$ as $t \to \pm \infty$.

\item[(2)]
  Similarly,
  \begin{align*}
    \lim_{t \to \pm \infty} x'(t)
    &= \lim_{t \to \pm \infty} \frac{4at}{(t^2+1)^2}
    = 0, \\
    \lim_{t \to \pm \infty} y'(t)
    &= \lim_{t \to \pm \infty} \frac{2at^2(t^2+3)}{(t^2+1)^2}
    = 2a.
  \end{align*}
  Therefore, $\alpha'(t) \to (0,2a)$ as $t \to \pm \infty$.

\item[(3)]
  By (1)(2), the curve and its tangent approach the line $x = 2a$ as $t \to \pm \infty$,
  or $x = 2a$ is an asymptote to the cissoid.
\end{enumerate}
$\Box$ \\\\



%%%%%%%%%%%%%%%%%%%%%%%%%%%%%%%%%%%%%%%%%%%%%%%%%%%%%%%%%%%%%%%%%%%%%%%%%%%%%%%%



\subsubsection*{Exercise 1-3.4. (Tractrix)}
\addcontentsline{toc}{subsubsection}{Exercise 1-3.4. (Tractrix)}
\emph{Let $\alpha: (0, \pi) \to \mathbb{R}^2$ be given by
$$\alpha(t) = \left(\sin t, \cos t + \log\tan\frac{t}{2}\right),$$
where $t$ is the angle that the $y$ axis makes with the vector $\alpha(t)$.
The trace of $\alpha$ is called the \textbf{tractrix}.
(Figure 1-9 in Mantredo P. do Carmo, Differential Geometry of Curves and Surfaces).
Show that}
\begin{enumerate}
\item[(a)]
  \emph{$\alpha$ is a differentiable parametrized curve,
  regular except at $t = \frac{\pi}{2}$.}

\item[(b)]
  \emph{The length of the segment of the tangent of the tractrix between
  the point of tangency and the $y$ axis is constantly equal to $1$.} \\
\end{enumerate}

\emph{Proof of (a).}
\begin{align*}
  \alpha'(t)
  &= \left(
    \cos t,
    -\sin t + \frac{1}{\tan\frac{t}{2}} \frac{1}{\cos^{2}\frac{t}{2}} \frac{1}{2}
  \right) \\
  &= \left(
    \cos t,
    -\sin t + \frac{1}{2 \sin\frac{t}{2} \cos\frac{t}{2}}
  \right) \\
  &= \left(
    \cos t,
    \frac{\cos^2 t}{\sin t}
  \right)
\end{align*}
exists.
And $\alpha'(t) = 0$ if and only if $t = \frac{\pi}{2}$.
That is, there is an unique singular point at $t = \frac{\pi}{2}$.
$\Box$ \\



\emph{Proof of (b).}
The the tangent line of the tractrix through the regular point $t$
is parametrized by $\beta: \mathbb{R} \to \mathbb{R}^2$ which is defined by
\begin{align*}
  \beta(u)
  &= u\alpha'(t) + \alpha(t) \\
  &= \left( u \cos t + \sin t, u \frac{\cos^2 t}{\sin t} + \cos t + \log\tan\frac{t}{2} \right).
\end{align*}
By construction, this tangent line $\beta(u)$ meets the tractrix at $u = 0$,
and meets the $y$-axis when $u \cos t + \sin t = 0$ or $u = -\tan t$.
So the length of the segment is
\begin{align*}
  |\beta(0) - \beta(-\tan t)|
  &= \sqrt{(-\tan t \cos t)^2+ \left( -\tan t \frac{\cos^2 t}{\sin t} \right)^2} \\
  &= \sqrt{(\sin t)^2+ (\cos t)^2} \\
  &= 1.
\end{align*}
$\Box$ \\\\



%%%%%%%%%%%%%%%%%%%%%%%%%%%%%%%%%%%%%%%%%%%%%%%%%%%%%%%%%%%%%%%%%%%%%%%%%%%%%%%%



\subsubsection*{Exercise 1-3.5. (Folium of Descartes)}
\addcontentsline{toc}{subsubsection}{Exercise 1-3.5. (Folium of Descartes)}
\emph{Let $\alpha: (-1, +\infty) \to \mathbb{R}^2$ be given by
\[
  \alpha(t) = \left( \frac{3at}{1+t^3}, \frac{3at^2}{1+t^3} \right).
\]
Prove that:}
\begin{enumerate}
\item[(a)]
  \emph{For $t = 0$, $\alpha$ is tangent to the $x$ axis.}

\item[(b)]
  \emph{As $t \to +\infty$, $\alpha(t) \to (0,0)$ and $\alpha'(t) = (0,0)$.}

\item[(c)]
  \emph{Take the curve the the opposite orientation.
  Now, as $t \to -1$, the curve and its tangent approach the line $x+y+a=0$.}
\end{enumerate}
\emph{The figure obtained by completing the trace of $\alpha$ in such a way that
it becomes symmetric relative the the line $y = x$ is called the
\textbf{folium of Descartes}
(See Figure 1-10 in Mantredo P. do Carmo, Differential Geometry of Curves and Surfaces).} \\



\emph{Proof of (a).}
  Note that
  \[
    \alpha'(t) = \left( \frac{3a(1-2t^3)}{(1+t^3)^2}, \frac{3at(2-t^3)}{(1+t^3)^2} \right).
  \]
  Hence, $\alpha'(0) = (3a, 0)$, or $\alpha$ is tangent to the $x$ axis when $t = 0$.
$\Box$ \\



\emph{Proof of (b).}
\begin{enumerate}
\item[(1)]
  \begin{align*}
    \lim_{t \to +\infty} \alpha(t)
    &= \lim_{t \to +\infty} \left( \frac{3at}{1+t^3}, \frac{3at^2}{1+t^3} \right) \\
    &= \left( \lim_{t \to +\infty} \frac{3at}{1+t^3},
      \lim_{t \to +\infty} \frac{3at^2}{1+t^3} \right) \\
    &= (0, 0).
  \end{align*}
\item[(2)]
  \begin{align*}
    \lim_{t \to +\infty} \alpha'(t)
    &= \lim_{t \to +\infty} \left( \frac{3a(1-2t^3)}{(1+t^3)^2}, \frac{3at(2-t^3)}{(1+t^3)^2} \right) \\
    &= \left( \lim_{t \to +\infty} \frac{3a(1-2t^3)}{(1+t^3)^2},
      \lim_{t \to +\infty} \frac{3at(2-t^3)}{(1+t^3)^2} \right) \\
    &= (0, 0).
  \end{align*}
\end{enumerate}
$\Box$ \\



\emph{Proof of (c).}
\begin{enumerate}
\item[(1)]
  Note that
  \begin{align*}
    \lim_{t \to -1^{+}} \alpha(t)
    &= \lim_{t \to -1^{+}} \left( \frac{3at}{1+t^3}, \frac{3at^2}{1+t^3} \right) \\
    &= \left( \lim_{t \to -1^{+}} \frac{3at}{1+t^3},
      \lim_{t \to -1^{+}} \frac{3at^2}{1+t^3} \right) \\
    &= (-\infty, +\infty)
  \end{align*}
  and
  \begin{align*}
    \lim_{t \to -1^{+}} (x(t) + y(t))
    &= \lim_{t \to -1^{+}} \left( \frac{3at}{1+t^3} + \frac{3at^2}{1+t^3} \right) \\
    &= \lim_{t \to -1^{+}} \frac{3at}{1-t+t^2} \\
    &= -a.
  \end{align*}
  Therefore, as $t \to -1$, the curve approaches the line $x+y+a=0$.

\item[(2)]
  Note that
  \begin{align*}
    \lim_{t \to -1^{+}} \frac{y'(t)}{x'(t)}
    &= \lim_{t \to -1^{+}} \frac{\frac{3a(1-2t^3)}{(1+t^3)^2}}{\frac{3at(2-t^3)}{(1+t^3)^2}} \\
    &= \lim_{t \to -1^{+}} \frac{1-2t^3}{t(2-t^3)} \\
    &= -1.
  \end{align*}
  Hence, as $t \to -1$, its tangent also approaches the line $x+y+a=0$.
\end{enumerate}
$\Box$ \\\\



%%%%%%%%%%%%%%%%%%%%%%%%%%%%%%%%%%%%%%%%%%%%%%%%%%%%%%%%%%%%%%%%%%%%%%%%%%%%%%%%



\subsubsection*{Exercise 1-3.6. (Logarithmic spiral)}
\addcontentsline{toc}{subsubsection}{Exercise 1-3.6. (Logarithmic spiral)}
\emph{Let $\alpha(t) = (ae^{bt} \cos t, ae^{bt} \sin t)$, $t \in \mathbb{R}$,
$a$ and $b$ constants, $a > 0$, $b < 0$, be a parametrized curve.}
\begin{enumerate}
\item[(a)]
  \emph{Show that as $t \to +\infty$, $\alpha(t)$ approaches the origin $0$,
  spiraling around it (because of this, the trace of $\alpha$ is called the \textbf{logarithmic spiral};
  See Figure 1-11 in Mantredo P. do Carmo, Differential Geometry of Curves and Surfaces).}

\item[(b)]
  \emph{Show that $\alpha'(t) \to (0,0)$ as $t \to +\infty$ and that
  \[
    \lim_{t \to + \infty} \int_{t_0}^{t} |\alpha'(t)| dt
  \]
  is finite; that is, $\alpha$ has finite arc length in $[t_0,\infty)$.} \\
\end{enumerate}



\emph{Proof of (a).}
\begin{enumerate}
\item[(1)]
  Note that
  \[
    \lim_{t \to +\infty} x(t)
    = \lim_{t \to +\infty} \frac{\overbrace{a \cos t}^{\text{bounded}}}
      {\underbrace{e^{-bt}}_{\to +\infty}}
    = 0
  \]
  and $\lim_{t \to +\infty} y(t) = 0$ (by the similar argument).
  Hence $\alpha(t)$ approaches the origin $0$ as $t \to +\infty$.

\item[(2)]
  $\alpha(t) = (ae^{bt} \cos t, ae^{bt} \sin t)$
  is moving in counter-clockwise on a circle path and
  sweeping out a length $ae^{bt}$ as $t$ is moving from $t_0$ to $+\infty$.
  Note that $t \mapsto ae^{bt}$ is decreasing strictly (as $t$ is moving from $t_0$ to $+\infty$).
  Hence $\alpha$ spiraling around the origin.
\end{enumerate}
$\Box$ \\



\emph{Proof of (b).}
\begin{enumerate}
\item[(1)]
  Note that
  \[
    \alpha'(t)
    = ( ae^{bt} (\underbrace{b\cos t - \sin t}_{\text{bounded}}),
      ae^{bt} (\underbrace{b\sin t + \cos t}_{\text{bounded}}) ).
  \]
  As $t \to +\infty$, $\alpha'(t) \to (0,0)$.

\item[(2)]
  As
  \begin{align*}
    \int_{t_0}^{+\infty} |\alpha'(t)| dt
    &= \int_{t_0}^{+\infty} ae^{bt} \sqrt{b^2 + 1} dt \\
    &= \left[ \frac{a}{b} e^{bt} \sqrt{b^2 + 1} \right]_{t = t_0}^{t = +\infty} \\
    &= - \frac{a}{b} e^{bt_0} \sqrt{b^2 + 1} \\
    &< +\infty,
  \end{align*}
  $\alpha$ has finite arc length in $[t_0,\infty)$.
\end{enumerate}
$\Box$ \\\\



%%%%%%%%%%%%%%%%%%%%%%%%%%%%%%%%%%%%%%%%%%%%%%%%%%%%%%%%%%%%%%%%%%%%%%%%%%%%%%%%



\subsubsection*{Exercise 1-3.7.}
\addcontentsline{toc}{subsubsection}{Exercise 1-3.7.}
\emph{A map $\alpha: I \to \mathbb{R}^3$ is called \textbf{a curve of class $\mathcal{C}^{k}$} if
each of the coordinate functions in the expression $\alpha(t) = (x(t), y(t), z(t))$
has continuous derivatives up to order $k$.
If $\alpha$ is merely continuous, we say that $\alpha$ is of class $\mathcal{C}^{0}$.
A curve $\alpha$ is called \textbf{simple} is the map $\alpha$ is one-to-one.
Thus, the curve $\alpha(t) = (t^3-4t, t^2-4)$ ($t \in \mathbb{R}$) is not simple.} \\

\emph{Let $\alpha: I \to \mathbb{R}^3$ be a simple curve of class $\mathcal{C}^{0}$.
We say that $\alpha$ has a \textbf{weak tangent} at $t = t_0 \in I$ if
the line determined by $\alpha(t_0 + h)$ and $\alpha(t_0)$ has a limit position when $h \to 0$.
We say that $\alpha$ has a \textbf{strong tangent} at $t = t_0 \in I$ if
the line determined by $\alpha(t_0 + h)$ and $\alpha(t_0 + k)$ has a limit position when $h, k \to 0$.
Show that}
\begin{enumerate}
\item[(a)]
  \emph{$\alpha(t) = (t^3,t^2)$, $t \in \mathbb{R}$, has a weak tangent
  but not a strong tangent at $t = 0$.}

\item[(b)]
  \emph{If $\alpha: I \to \mathbb{R}^3$ is of class $\mathcal{C}^{1}$ and
  regular at $t = t_0$, then it has a strong tangent at $t = t_0$.}

\item[(c)]
  \emph{The curve given by
  \begin{equation*}
    \alpha(t) =
    \begin{cases}
       (t^2,t^2), & t \geq 0, \\
       (t^2,-t^2), & t \leq 0,
    \end{cases}
  \end{equation*}
  is of class $\mathcal{C}^{1}$ but not of class $\mathcal{C}^{2}$.
  Draw a sketch of the curve and its tangent vectors.} \\
\end{enumerate}



\emph{Proof of (a).}
\begin{enumerate}
\item[(1)]
  Note that $\alpha(0) = (0,0)$ and $\alpha(h) = (h^3,h^2)$.
  The line passing $\alpha(0)$ and $\alpha(h)$ is
  \begin{align*}
    & \: (x - 0)(h^2 - 0) - (y - 0)(h^3 - 0) = 0 \\
    \Longleftrightarrow & \:
    x - hy = 0.
  \end{align*}
  As $h \to 0$, the line has a limit position $x = 0$.
  Therefore, $\alpha(t)$ has a weak tangent.

\item[(2)]
  The line passing $\alpha(h)$ and $\alpha(k)$ is
  \begin{align*}
    & \: (x - k^2)(h^2 - k^2) - (y - k^3)(h^3 - k^3) = 0 \\
    \Longleftrightarrow & \:
    (x - k^2)(h + k) - (y - k^3)(h^2 + hk + k^2) = 0.
  \end{align*}
  As $h \to 0$, the line has a limit position
  \begin{align*}
    & \: (x - k^2) - (y - k^3)k = 0 \\
    \Longleftrightarrow & \:
    x - ky + k^4 - k^2 = 0.
  \end{align*}
  As $k \to 0$, the line has a limit position $x = 0$.

\item[(3)]
  On the other hand, as $h = -k$ we have $y - k^3 = 0$.
  As $k \to 0$, the line has a limit position $y = 0$,
  contrary to (2).
  Therefore, $\alpha(t)$ has a strong tangent.
\end{enumerate}
$\Box$ \\



\emph{Proof of (b).}
\begin{enumerate}
\item[(1)]
  The line $L$ passing $\alpha(t_0+h)$ and $\alpha(t_0+k)$ is
  \begin{align*}
    x(s) &= x(t_0) + \frac{x(t_0+h)-x(t_0+k)}{h-k} s, \\
    y(s) &= y(t_0) + \frac{y(t_0+h)-y(t_0+k)}{h-k} s, \\
    z(s) &= z(t_0) + \frac{z(t_0+h)-z(t_0+k)}{h-k} s.
  \end{align*}

\item[(2)]
  By the mean value theorem,
  \[
    \frac{x(t_0+h)-x(t_0+k)}{h-k} = x'(t_0 + \xi)
  \]
  for some $\xi$ between $h$ and $k$.
  Since $\alpha \in \mathcal{C}^{1}$, $x(t) \in \mathcal{C}^{1}$.
  Hence
  \begin{align*}
    \lim_{h,k \to 0} \frac{x(t_0+h)-x(t_0+k)}{h-k}
    &=
    \lim_{h,k \to 0} x'(t_0 + \xi) \\
    &=
    \lim_{\xi \to 0} x'(t_0 + \xi) \\
    &= x'(t_0).
  \end{align*}
  Similarly, we have $\lim_{h,k \to 0} \frac{y(t_0+h)-y(t_0+k)}{h-k} = y'(t_0)$ and
  $\lim_{h,k \to 0} \frac{z(t_0+h)-z(t_0+k)}{h-k} = z'(t_0)$.
  Since $\alpha$ is regular,
  $\lim_{h,k \to 0} L$ is a non degenerate line
  \begin{align*}
    x(s) &= x(t_0) + x'(t_0) s, \\
    y(s) &= y(t_0) + y'(t_0) s, \\
    z(s) &= z(t_0) + z'(t_0) s
  \end{align*}
  and thus $\lim_{h,k \to 0} L$ is a strong tangent at $t = t_0$.
\end{enumerate}
$\Box$ \\



\emph{Proof of (c).}
\begin{enumerate}
\item[(1)]
  Since
  \begin{equation*}
    \alpha'(t) =
    \begin{cases}
       (2t,2t), & t \geq 0, \\
       (2t,-2t), & t \leq 0,
    \end{cases}
  \end{equation*}
  $\alpha$ is of class $\mathcal{C}^{1}$.

\item[(2)]
  Since
  \begin{equation*}
    \alpha''(t) =
    \begin{cases}
       (2,2), & t > 0, \\
       \text{undefined}, & t = 0 \\
       (2,-2), & t < 0,
    \end{cases}
  \end{equation*}
  $\alpha$ is not of class $\mathcal{C}^{2}$.
\end{enumerate}
(Skip drawing a sketch of the curve and its tangent vectors.)
$\Box$ \\\\



%%%%%%%%%%%%%%%%%%%%%%%%%%%%%%%%%%%%%%%%%%%%%%%%%%%%%%%%%%%%%%%%%%%%%%%%%%%%%%%%



\subsubsection*{Exercise 1-3.8.}
\addcontentsline{toc}{subsubsection}{Exercise 1-3.8.}
\emph{Let $\alpha: I \to \mathbb{R}^3$ be a differentiable curve and
let $[a,b] \subseteq I$ be a closed interval.
For every partition
\[
  a = t_0 < t_1 < \cdots < t_n = b
\]
of $[a,b]$, consider the sum
\[
  \sum_{i=1}^{n} |\alpha(t_i) - \alpha(t_{i-1})| = l(\alpha,P),
\]
where $P$ stands for the given partition.
The norm $|P|$ of a partition $P$ is defined as
\[
  |P| = \max(t_i - t_{i-1}), i = 1, \cdots, n.
\]
Geometrically, $l(\alpha,P)$ is the length of a polygon inscribed in $\alpha([a,b])$
with vertices in $\alpha(t_i)$
(see Figure 1-3 in Mantredo P. do Carmo, Differential Geometry of Curves and Surfaces).
The point of the exercise is to show that the arc length of $\alpha([a,b])$ is,
in some sense, a limit of lengths of inscribed polygons.
Prove that given $\varepsilon > 0$ there exists $\delta > 0$ such that
if $|P| < \delta$ then
\[
  \abs{ \int_{a}^{b} |\alpha'(t)|dt - l(\alpha,P) } < \varepsilon.
\]} \\

Assume that $\alpha'(t)$ is continuous. \\



\emph{Proof.}
Given $\varepsilon > 0$.
\begin{enumerate}
\item[(1)]
  Since $\alpha'(t)$ is continuous on a compact set $[a,b]$,
  $\alpha'(t)$ is uniformly continuous, that is,
  there there exists $\delta > 0$ such that
  \[
    |\alpha'(s) - \alpha'(t)| < \frac{\varepsilon}{2(b-a)}
    \text{ whenever } |s-t| < \delta.
  \]
\item[(2)]
  Let $P = \{a = t_0, t_1, \ldots, t_n = b \}$
  be a partition of $[a,b]$, with $\Delta t_i = t_i - t_{i-1} < \delta$
  for all $i = 1, \ldots, n$.
  If $t_{i-1} \leq t \leq t_i$, it follows that
  \[
    |\alpha'(t_i)| - \frac{\varepsilon}{2(b-a)}
    \leq |\alpha'(t)|
    \leq |\alpha'(t_i)| + \frac{\varepsilon}{2(b-a)}.
  \]
  Hence,
  \begin{align*}
    & \: \int_{t_{i-1}}^{t_i} |\alpha'(t)| dt \\
    \geq& \: |\alpha'(t_i)| \Delta t_i
      - \frac{\varepsilon}{2(b-a)} \Delta t_i \\
    =& \: \abs{ \int_{t_{i-1}}^{t_i} [\alpha'(t)+\alpha'(t_i)-\alpha'(t)] dt }
      - \frac{\varepsilon}{2(b-a)} \Delta t_i \\
    \geq& \: \abs{ \int_{t_{i-1}}^{t_i} \alpha'(t) dt }
      - \abs{ \int_{t_{i-1}}^{t_i} [\alpha'(t_i)-\alpha'(t)] dt }
      - \frac{\varepsilon}{2(b-a)} \Delta t_i \\
    \geq& \: \abs{ \alpha(t_i) - \alpha(t_{i-1}) }
      - \frac{\varepsilon}{b-a} \Delta t_i
  \end{align*}
  and
  \begin{align*}
    & \: \int_{t_{i-1}}^{t_i} |\alpha'(t)| dt \\
    \leq& \: |\alpha'(t_i)| \Delta t_i
      + \frac{\varepsilon}{2(b-a)} \Delta t_i \\
    =& \: \abs{ \int_{t_{i-1}}^{t_i} [\alpha'(t)+\alpha'(t_i)-\alpha'(t)] dt }
      + \frac{\varepsilon}{2(b-a)} \Delta t_i \\
    \leq& \: \abs{ \int_{t_{i-1}}^{t_i} \alpha'(t) dt }
      + \abs{ \int_{t_{i-1}}^{t_i} [\alpha'(t_i)-\alpha'(t)] dt }
      + \frac{\varepsilon}{2(b-a)} \Delta t_i \\
    \leq& \: \abs{ \alpha(t_i) - \alpha(t_{i-1}) }
      + \frac{\varepsilon}{b-a} \Delta t_i.
  \end{align*}

\item[(3)]
  If we add these inequalities, we obtain
  \[
    l(\alpha,P) - \varepsilon
    \leq \int_{a}^{b} |\alpha'(t)| dt
    \leq l(\alpha,P) + \varepsilon.
  \]
\end{enumerate}
$\Box$ \\\\



%%%%%%%%%%%%%%%%%%%%%%%%%%%%%%%%%%%%%%%%%%%%%%%%%%%%%%%%%%%%%%%%%%%%%%%%%%%%%%%%



\subsubsection*{Exercise 1-3.9.}
\addcontentsline{toc}{subsubsection}{Exercise 1-3.9.}
\begin{enumerate}
\item[(a)]
  \emph{Let $\alpha: I \to \mathbb{R}^3$ be a curve of class $\mathcal{C}^{0}$
  (compare Exercise 1-3.7).
  Use the approximation by polygons described in Exercise 1-3.8 to give
  a reasonable defintion of arc length of $\alpha$.}

\item[(b)]
  \emph{(A Nonrectifiable Curve.)
  The following example shows that, with any reasonable definition,
  the arc length of a $\mathcal{C}^{0}$ curve in a closed interval may be unbounded.
  Let $\alpha:[0,1] \to \mathbb{R}^2$ be given as
  $\alpha(t) = (t, t \sin(\frac{\pi}{t}))$ if $t \neq 0$,
  and $\alpha(0) = (0,0)$.
  Show, geometrically, that the arc length of the portion of the curve corresponding to
  $\frac{1}{n+1} \leq t \leq \frac{1}{n}$ is at least $\frac{2}{n + \frac{1}{2}}$.
  Use this to show that the length of curve in the interval $\frac{1}{N} \leq t \leq 1$
  is greater than $2 \sum_{n=1}^{N-1} \frac{1}{n+1}$,
  and thus it tends to infinity as $N \to \infty$.} \\
\end{enumerate}



\emph{Proof of (a).}
Define
\[
  l(\alpha) = \sup \{ l(\alpha,P) : P \text{ is a partition of } [a,b] \}.
\]
$\Box$ \\

\emph{Note.}
(Theorem 6.17 in \emph{Tom. M. Apostol, Mathematical Analysis, 2nd edition.}).
$\alpha$ is rectifiable if and only $\alpha$ is of bounded variation on $[a,b]$. \\



\emph{Proof of (b).}
\begin{enumerate}
  \item[(1)]
  Consider a partition
  $P = \left\{ \frac{1}{n+1}, \frac{1}{n+\frac{1}{2}}, \frac{1}{n} \right\}$
  of $\left[ \frac{1}{n+1},\frac{1}{n} \right]$.
  So that $\alpha(\frac{1}{n+1}) = \alpha(\frac{1}{n}) = 0$
  and $\alpha(\frac{1}{n+\frac{1}{2}}) = \pm 1$.

  \item[(2)]
  Thus,
  \begin{align*}
    & \: \text{The arc length of the portion of $\alpha$ over
      $\left[ \frac{1}{n+1},\frac{1}{n} \right]$} \\
    \geq& \:
    \text{The sum of each length of the individual chords} \\
    =& \: \sqrt{
      \left( \frac{1}{n+\frac{1}{2}} - \frac{1}{n+1} \right)^2
        + \left( \frac{1}{n + \frac{1}{2}} \right)^2
    } \\
    & + \sqrt{
      \left( \frac{1}{n} - \frac{1}{n+\frac{1}{2}} \right)^2
        + \left( \frac{1}{n + \frac{1}{2}} \right)^2
    } \\
    \geq& \:
    \frac{2}{n + \frac{1}{2}}.
  \end{align*}

  \item[(3)]
  So
  \begin{align*}
    & \: \text{The arc length of $\alpha$ over $\left[\frac{1}{N},1\right]$} \\
    =& \: \sum_{n=1}^{N-1}
      \left\{
        \text{The arc length of $\alpha$ over $\left[\frac{1}{n+1},\frac{1}{n}\right]$}
      \right\} \\
    \geq& \: \sum_{n=1}^{N-1} \frac{2}{n + \frac{1}{2}} \\
    >& \: 2 \sum_{n=1}^{N-1} \frac{1}{n + 1}.
  \end{align*}
  It tends to infinity as $N \to \infty$, or
  $\alpha$ is nonrectifiable.
\end{enumerate}
$\Box$ \\\\



%%%%%%%%%%%%%%%%%%%%%%%%%%%%%%%%%%%%%%%%%%%%%%%%%%%%%%%%%%%%%%%%%%%%%%%%%%%%%%%%



\subsubsection*{Exercise 1-3.10. (Straight Lines as Shortest)}
\addcontentsline{toc}{subsubsection}{Exercise 1-3.10. (Straight Lines as Shortest)}
\emph{Let $\alpha: I \to \mathbb{R}^3$ be a parametrized curve.
Let $[a,b] \subseteq I$ and set $\alpha(a) = p$, $\alpha(b) = q$.}
\begin{enumerate}
\item[(a)]
  \emph{Show that, for any constant vector $v$, $|v| = 1$,
  $$(q-p) \cdot v
  = \int_{a}^{b} \alpha'(t) \cdot v dt
  \leq \int_{a}^{b} |\alpha'(t)| dt.$$}

\item[(b)]
  \emph{Set
  $$v = \frac{q-p}{|q-p|}$$
  and show that
  $$|\alpha(b) - \alpha(a)| \leq \int_{a}^{b} |\alpha'(t)| dt;$$
  that is, the curve of shortest length from
  $\alpha(a)$ to $\alpha(b)$ is the straight line joining these points.} \\
\end{enumerate}

Assume $p \neq q$ (otherwise $v = \frac{q-p}{|q-p|}$ is meaningless). \\



\emph{Proof of (a).}
Let $f(t) = \alpha(t) \cdot v$ defined on $I$.
By the fundamental theorem of calculus,
$$\int_{a}^{b} f'(t) dt = f(b) - f(a).$$
Since $f'(t) = \alpha'(t) \cdot v$,
$$(\alpha(b) - \alpha(a)) \cdot v = \int_{a}^{b} \alpha'(t) \cdot v dt.$$
Therefore,
\begin{align*}
  (q - p) \cdot v
  &= \int_{a}^{b} \alpha'(t) \cdot v dt \\
  &\leq \int_{a}^{b} |\alpha'(t) \cdot v| dt \\
  &\leq \int_{a}^{b} |\alpha'(t)||v| dt \\
  &= \int_{a}^{b} |\alpha'(t)| dt.
\end{align*}
$\Box$ \\



\emph{Proof of (b).}
$|v| = \frac{|q-p|}{|q-p|} = 1$.
So,
\begin{align*}
  (q-p) \cdot \frac{q-p}{|q-p|}
  &\leq \int_{a}^{b} |\alpha'(t)| dt, \\
  |q-p|
  &\leq \int_{a}^{b} |\alpha'(t)| dt.
\end{align*}
$\Box$ \\\\



%%%%%%%%%%%%%%%%%%%%%%%%%%%%%%%%%%%%%%%%%%%%%%%%%%%%%%%%%%%%%%%%%%%%%%%%%%%%%%%%
%%%%%%%%%%%%%%%%%%%%%%%%%%%%%%%%%%%%%%%%%%%%%%%%%%%%%%%%%%%%%%%%%%%%%%%%%%%%%%%%



\subsection*{1-4. The Vector Product in $\mathbb{R}^{3}$ \\}
\addcontentsline{toc}{subsection}{1-4. The Vector Product in $\mathbb{R}^{3}$}



\subsubsection*{Exercise 1-4.1.}
\addcontentsline{toc}{subsubsection}{Exercise 1-4.1.}
\emph{Check whether the following bases are positive:}
\begin{enumerate}
\item[(a)]
  \emph{The basis $\{(1,3), (4,2)\}$ in $\mathbb{R}^2$.}

\item[(b)]
  \emph{The basis $\{(1,3,5), (2,3,7), (4,8,3)\}$ in $\mathbb{R}^3$.} \\
\end{enumerate}



\emph{Proof of (a).}
Write $u = (1,3)$ and $v = (4,2)$.
Then
\[
  \det(u,v)
  = \begin{vmatrix}
    1 & 3 \\
    4 & 2
    \end{vmatrix}
  = -10 < 0.
\]
Thus $\{u,v\}$ is negative w.r.t. the natural order basis $\{ e_1 = (1,0), e_2 = (0,1) \}$.
$\Box$ \\

\emph{Proof of (b).}
Write $u = (1,3,5)$, $v = (2,3,7)$, $w = (4,8,3)$.
Then
\[
  \det(u,v,w)
  = \begin{vmatrix}
    1 & 3 & 5 \\
    2 & 3 & 7 \\
    4 & 8 & 3
    \end{vmatrix}
  = 39 > 0.
\]
Thus $\{u,v,w\}$ is positive w.r.t. the natural order basis $\{ e_1, e_2, e_3 \}$.
$\Box$ \\\\



%%%%%%%%%%%%%%%%%%%%%%%%%%%%%%%%%%%%%%%%%%%%%%%%%%%%%%%%%%%%%%%%%%%%%%%%%%%%%%%%



\subsubsection*{Exercise 1-4.2.}
\addcontentsline{toc}{subsubsection}{Exercise 1-4.2.}
\emph{A plane $P$ contained in $\mathbb{R}^3$ is given by the equation $ax+by+cz+d=0$.
Show that the vector $v=(a,b,c)$ is perpendicular to the plane and that
$|d|/\sqrt{a^2+b^2+c^2}$ measures the distance from the plane to the origin $(0,0,0)$.} \\

Say $v$ is a normal vector of $E$. \\

In general, the distance from the plane $E$ to
any point $(x_0, y_0, z_0) \in \mathbb{R}^3$ is
$$\frac{|ax_0+by_0+cz_0+d|}{\sqrt{a^2+b^2+c^2}}.$$



\emph{Proof.}
\begin{enumerate}
\item[(1)]
  To show $v=(a,b,c)$ is perpendicular to the plane,
  it suffices to show that $v \cdot u = 0$ for any vector $u$ lying on the plane $E$.
  Write $u = \overrightarrow{PQ}$ where
  $P = (x_1, y_1, z_1) \in E$ and $Q = (x_2, y_2, z_2) \in E$.
  Hence $u = (x_2-x_1, y_2-y_1, z_2-z_1)$.
  \begin{align*}
    v \cdot u
    &= (a,b,c) \cdot (x_2-x_1, y_2-y_1, z_2-z_1) \\
    &= a(x_2-x_1) + b(y_2-y_1) + c(z_2-z_1) \\
    &= (ax_2 + by_2 + cz_2) - (ax_1 + by_1 + cz_1) \\
    &= (-d) - (-d) \\
    &= 0.
  \end{align*}

\item[(2)]
  Pick any point $(x_1,y_1,z_1) \in E$.
  The distance from the plane $E$ to the point $(x_0, y_0, z_0)$ is
  \begin{align*}
    & \: \abs{(x_1-x_0,y_1-y_0,z_1-z_0) \cdot \frac{v}{|v|}} \\
    =& \: \abs{(x_1-x_0,y_1-y_0,z_1-z_0) \cdot \frac{(a,b,c)}{\sqrt{a^2+b^2+c^2}}} \\
    =& \: \frac{\abs{a(x_1-x_0)+b(y_1-y_0)+c(z_1-z_0)}}{\sqrt{a^2+b^2+c^2}} \\
    =& \: \frac{\abs{(ax_1+by_1+cz_1) - (ax_0+by_0+cz_0)}}{\sqrt{a^2+b^2+c^2}} \\
    =& \: \frac{\abs{-d - (ax_0+by_0+cz_0)}}{\sqrt{a^2+b^2+c^2}} \\
    =& \: \frac{\abs{ax_0+by_0+cz_0+d}}{\sqrt{a^2+b^2+c^2}}.
  \end{align*}
\end{enumerate}
$\Box$ \\\\



%%%%%%%%%%%%%%%%%%%%%%%%%%%%%%%%%%%%%%%%%%%%%%%%%%%%%%%%%%%%%%%%%%%%%%%%%%%%%%%%



\subsubsection*{Exercise 1-4.3.}
\addcontentsline{toc}{subsubsection}{Exercise 1-4.3.}
\emph{Determine the angle of intersection of the two planes
$5x+3y+2z-4=0$ and $3x+4y-7z=0$.} \\



\emph{Proof.}
\begin{enumerate}
\item[(1)]
The angle of intersection of the two planes
is equal to a angle between two normal vectors of planes.
\item[(2)]
Let
  \begin{enumerate}
  \item[(a)]
  the angle of intersection of the two planes be $\theta$.
  \item[(b)]
  the normal vector of $5x+3y+2z-4=0$ be $n_1 = (5,3,2)$.
  \item[(c)]
  the normal vector of $3x+4y-7z=0$ be $n_2 = (3,4,-7)$.
  \end{enumerate}
\item[(3)]
Hence,
$$\cos\theta = \frac{n_1 \cdot n_2}{|n_1||n_2|} = \frac{13}{2\sqrt{703}}.$$
$\theta = \cos^{-1}\left( \frac{13}{2\sqrt{703}} \right)$.
\end{enumerate}
$\Box$ \\\\



%%%%%%%%%%%%%%%%%%%%%%%%%%%%%%%%%%%%%%%%%%%%%%%%%%%%%%%%%%%%%%%%%%%%%%%%%%%%%%%%



\subsubsection*{Exercise 1-4.4.}
\addcontentsline{toc}{subsubsection}{Exercise 1-4.4.}
\emph{Given two planes $a_i x + b_i y + c_i z + d_i = 0$, $i = 1, 2$,
prove that a necessary and sufficient condition for them to be parallel is
\[
  \frac{a_1}{a_2} = \frac{b_1}{b_2} = \frac{c_1}{c_2},
\]
where the convention is made that if a denominator is zero,
the corresponding numerator is also zero
(we say that two planes are parallel if they either coincide or do not intersect).} \\



\emph{Proof.}
\begin{enumerate}
\item[(1)]
  Write
  \[
    E_i: a_i x + b_i y + c_i z + d_i = 0.
  \]
  By Exercise 1-4.2,
  the vector $(a_i,b_i,c_i)$ is perpendicular to the plane $E_i$.

\item[(2)]
  Hence,
  \begin{align*}
    \text{$E_1$ is parallel to $E_2$}
    & \Longleftrightarrow
    \text{$(a_1,b_1,c_1)$ is parallel to $(a_2,b_2,c_2)$} \\
    & \Longleftrightarrow
    \frac{a_1}{a_2} = \frac{b_1}{b_2} = \frac{c_1}{c_2}.
  \end{align*}
  (where the convention is made that if a denominator is zero,
  the corresponding numerator is also zero).
\end{enumerate}
$\Box$ \\\\



%%%%%%%%%%%%%%%%%%%%%%%%%%%%%%%%%%%%%%%%%%%%%%%%%%%%%%%%%%%%%%%%%%%%%%%%%%%%%%%%



\subsubsection*{Exercise 1-4.5.}
\addcontentsline{toc}{subsubsection}{Exercise 1-4.5.}
\emph{Show that the equation of a plane passing through three noncolinear points
$p_i = (x_i, y_i, z_i)$, $i = 1, 2, 3$ is given by
\[
  (p - p_1) \wedge (p - p_2) \cdot (p - p_3) = 0.
\]
where $p = (x, y, z)$ is an arbitrary point of the plane and $p - p_1$,
for instance,
means the vector $(x - x_1, y - y_1, z - z_1)$.} \\



\emph{Proof.}
\begin{enumerate}
\item[(1)]
  By Exercise 1-4.11(a),
  the volume $V$ of a parallelepiped generated by
  $p - p_1, p - p_2, p - p_3 \in \mathbb{R}^3$ is given by
  \[
    V = (p - p_1) \wedge (p - p_2) \cdot (p - p_3).
  \]

\item[(2)]
  Since all vectors $p - p_1, p - p_2, p - p_3$ are lying on the same plane,
  $V = 0$.
  Therefore, the equation of a plane is $(p - p_1) \wedge (p - p_2) \cdot (p - p_3) = 0$.
\end{enumerate}
$\Box$ \\\\



%%%%%%%%%%%%%%%%%%%%%%%%%%%%%%%%%%%%%%%%%%%%%%%%%%%%%%%%%%%%%%%%%%%%%%%%%%%%%%%%



\subsubsection*{Exercise 1-4.6.}
\addcontentsline{toc}{subsubsection}{Exercise 1-4.6.}
\emph{Given two nonparallel planes $a_i x + b_i y + c_i z + d_i = 0$, $i = 1, 2$,
show that their line of intersection may be parametrized as
\[
  x-x_0 = u_1 t, \qquad y-y_0 = u_2 t, \qquad z-z_0 = u_3 t,
\]
where $(x_0, y_0, z_0)$ belongs to the intersection and
$u = (u_1, u_2, u_3)$ is the vector product $u = v_1 \wedge v_2$,
$v_i = (a_i, b_i, c_i)$, $i = 1, 2$.} \\



\emph{Proof.}
\begin{enumerate}
\item[(1)]
  Suppose that the line of intersection is
  \[
    L: x-x_0 = u_1' t, \qquad y-y_0 = u_2' t, \qquad z-z_0 = u_3' t
  \]
  where $(x_0, y_0, z_0)$ belongs to the intersection.

\item[(2)]
  By Exercise 1-4.2,
  the vector $v_i = (a_i,b_i,c_i)$ is perpendicular to
  the plane $E_i: a_i x + b_i y + c_i z + d_i = 0$.
  Hence $v_i$ is perpendicular to $u' = (u_1', u_2', u_3')$.
  Since $v_i \cdot (v_1 \wedge v_2) = 0$, we may choose $u' = v_1 \wedge v_2 = u$.
\end{enumerate}
$\Box$ \\\\



%%%%%%%%%%%%%%%%%%%%%%%%%%%%%%%%%%%%%%%%%%%%%%%%%%%%%%%%%%%%%%%%%%%%%%%%%%%%%%%%



\subsubsection*{Exercise 1-4.7.}
\addcontentsline{toc}{subsubsection}{Exercise 1-4.7.}
\emph{Prove that a necessary and sufficient condition for the plane
\[
  ax+by+cz+d = 0
\]
and the line
\[
  x-x_0 = u_1 t, \qquad y-y_0 = u_2 t, \qquad z-z_0 = u_3 t
\]
to be parallel is}
\[
  au_1+bu_2+cu_3 = 0.
\]



\emph{Proof.}
  Write
  \begin{align*}
    E &: ax+by+cz+d = 0 \\
    L &: x-x_0 = u_1 t, \qquad y-y_0 = u_2 t, \qquad z-z_0 = u_3 t.
  \end{align*}
  By Exercise 1-4.2,
  the vector $(a,b,c)$ is perpendicular to the plane $E$.
  Hence,
  \begin{align*}
    \text{$E$ is parallel to $L$}
    & \Longleftrightarrow
    \text{$(a,b,c)$ is perpendicular to $L$} \\
    & \Longleftrightarrow
    \text{$(a,b,c)$ is perpendicular to $u = (u_1,u_2,u_3)$} \\
    & \Longleftrightarrow
    0 = (a,b,c) \cdot (u_1,u_2,u_3) = au_1+bu_2+cu_3.
  \end{align*}
$\Box$ \\\\



%%%%%%%%%%%%%%%%%%%%%%%%%%%%%%%%%%%%%%%%%%%%%%%%%%%%%%%%%%%%%%%%%%%%%%%%%%%%%%%%



\subsubsection*{Exercise 1-4.8.}
\addcontentsline{toc}{subsubsection}{Exercise 1-4.8.}
\emph{Prove that the distance $\rho$ between the nonparallel lines
\begin{align*}
  & x-x_0 = u_1 t, \qquad y-y_0 = u_2 t, \qquad z-z_0 = u_3 t, \\
  & x-x_0 = v_1 t, \qquad y-y_0 = v_2 t, \qquad z-z_0 = v_3 t
\end{align*}
is given by
\[
  \rho = \frac{\abs{(u \wedge v) \cdot r}}{\abs{u \wedge v}}
\]
where $u = (u_1,u_2,u_3)$, $v = (v_1,v_2,v_3)$,
$r = (x_0-x_1,y_0-y_1,z_0-z_1)$.} \\



\emph{Proof.}
\[
  \rho
  = |r| \abs{\cos\angle(u \wedge v, r)}
  = |r| \frac{\abs{(u \wedge v) \cdot r}}{\abs{u \wedge v}|r|}
  = \frac{\abs{(u \wedge v) \cdot r}}{\abs{u \wedge v}}.
\]
It is well-defined ($\abs{u \wedge v} > 0$) since two lines are nonparallel.
$\Box$ \\\\



%%%%%%%%%%%%%%%%%%%%%%%%%%%%%%%%%%%%%%%%%%%%%%%%%%%%%%%%%%%%%%%%%%%%%%%%%%%%%%%%



\subsubsection*{Exercise 1-4.9.}
\addcontentsline{toc}{subsubsection}{Exercise 1-4.9.}
\emph{Determine the angle of intersection of the plane $3x+4y+7z+8=0$ and
the line $x-2=3t$, $y-3=5t$, $z-5=9t$.} \\



\emph{Proof.}
\begin{enumerate}
\item[(1)]
  The angle of intersection is equal to $\pi/2$ minus an acute angle $\theta$ between
  the normal vector of the plane and the direction vector of a line.

\item[(2)]
  \[
    \cos\theta
    = \frac{(3,4,7) \cdot (3,5,9)}{|(3,4,7)||(3,5,9)|}
    = \frac{92}{\sqrt{74}\sqrt{115}}.
  \]
  Hence, the angle of intersection is
  \[
    \pi/2 - \arccos\left( \frac{92}{\sqrt{74}\sqrt{115}} \right).
  \]
\end{enumerate}
$\Box$ \\\\



%%%%%%%%%%%%%%%%%%%%%%%%%%%%%%%%%%%%%%%%%%%%%%%%%%%%%%%%%%%%%%%%%%%%%%%%%%%%%%%%



\subsubsection*{Exercise 1-4.10. (Oriented area)}
\addcontentsline{toc}{subsubsection}{Exercise 1-4.10. (Oriented area)}
\emph{The natural orientation of $\mathbb{R}^2$
makes it possible to associate a sign to the area $A$
of a parallelogram generated by two linearly independent vectors $u, v \in \mathbb{R}^2$,
and write $u = u_1 e_1 + u_2 e_2$, $v = v_1 e_1 + v_2 e_2$.
Observe the matrix relation
\[
  \begin{bmatrix}
    u \cdot u & u \cdot v \\
    v \cdot u & v \cdot v
  \end{bmatrix}
  =
  \begin{bmatrix}
    u_1 & u_2 \\
    v_1 & v_2
  \end{bmatrix}
  \begin{bmatrix}
    u_1 & v_1 \\
    u_2 & v_2
  \end{bmatrix}
\]
and conclude that
\[
  A^2
  =
  \begin{vmatrix}
    u_1 & u_2 \\
    v_1 & v_2
  \end{vmatrix}^2.
\]
Since the last determinant has the same sign as the basis $\{u,v\}$,
we can say that $A$ is positive or $negative$ according to whether
the orientation of $\{u,v\}$ is positive or negative.
This is called the \textbf{oriented area} in $\mathbb{R}^2$.} \\



\emph{Proof.}
  \begin{align*}
    A^2
    &=
    \begin{vmatrix}
      u \cdot u & u \cdot v \\
      v \cdot u & v \cdot v
    \end{vmatrix} \\
    &=
    \begin{vmatrix}
      u_1 & u_2 \\
      v_1 & v_2
    \end{vmatrix}
    \begin{vmatrix}
      u_1 & v_1 \\
      u_2 & v_2
    \end{vmatrix} \\
    &=
    \begin{vmatrix}
      u_1 & u_2 \\
      v_1 & v_2
    \end{vmatrix}^2
  \end{align*}
  since $\det A^T = \det A$.
$\Box$ \\\\



%%%%%%%%%%%%%%%%%%%%%%%%%%%%%%%%%%%%%%%%%%%%%%%%%%%%%%%%%%%%%%%%%%%%%%%%%%%%%%%%



\subsubsection*{Exercise 1-4.11. (Oriented volume)}
\addcontentsline{toc}{subsubsection}{Exercise 1-4.11. (Oriented volume)}
\begin{enumerate}
\item[(a)]
  \emph{Show that the volume $V$ of a parallelepiped generated by three linearly independent vectors
  $u, v, w \in \mathbb{R}^3$ is given by $V = \abs{(u \wedge v) \cdot w}$,
  and introduce an \textbf{oriented volume} in $\mathbb{R}^3$.}

\item[(b)]
  \emph{Prove that}
  \[
    V^2
    = \begin{vmatrix}
      u \cdot u & u \cdot v & u \cdot w \\
      v \cdot u & v \cdot v & v \cdot w \\
      w \cdot u & w \cdot v & w \cdot w
    \end{vmatrix}.
  \]
\end{enumerate}



\emph{Proof of (a).}
\begin{enumerate}
\item[(1)]
  We can calculate the volume $V$ by
  multiplying the area of the base $\abs{u \wedge v}$ and the height $h$.

\item[(2)]
  Note that
  \[
    h
    = |w| \abs{\cos\angle(u \wedge v, w)}
    = |w| \frac{\abs{(u \wedge v) \cdot w}}{\abs{u \wedge v}|w|}
    = \frac{\abs{(u \wedge v) \cdot w}}{\abs{u \wedge v}}.
  \]
  It is well-defined since $u$ and $v$ are linearly independent.
  Therefore,
  \[
    V = \abs{u \wedge v} h = \abs{(u \wedge v) \cdot w}.
  \]

\item[(3)]
  The oriented volume is defined by
  \[
    (u \wedge v) \cdot w.
  \]
\end{enumerate}
$\Box$ \\



\emph{Proof of (b).}
  Recall $(u \wedge v) \cdot w = \det(u,v,w)$.
  Also note that $\det((u,v,w)^{T}) = \det(u,v,w)$.
  So
  \begin{align*}
    \begin{vmatrix}
      u \cdot u & u \cdot v & u \cdot w \\
      v \cdot u & v \cdot v & v \cdot w \\
      w \cdot u & w \cdot v & w \cdot w
    \end{vmatrix}
    &= \det((u,v,w) \cdot (u,v,w)^{T}) \\
    &= \det(u,v,w) \det((u,v,w)^{T}) \\
    &= \det(u,v,w)^2 \\
    &= ((u \wedge v) \cdot w)^2 \\
    &= V^2.
  \end{align*}
$\Box$ \\\\



%%%%%%%%%%%%%%%%%%%%%%%%%%%%%%%%%%%%%%%%%%%%%%%%%%%%%%%%%%%%%%%%%%%%%%%%%%%%%%%%



\subsubsection*{Exercise 1-4.12.}
\addcontentsline{toc}{subsubsection}{Exercise 1-4.12.}
\emph{Given the vectors $v \neq 0$ and $w$,
show that there exists a vector $u$ such that $u \wedge v = w$
if and only if $v$ is perpendicular to $w$.
Is this vector $u$ uniquely determined?
If not, what is the most general solution?} \\



\emph{Proof.}
\begin{enumerate}
\item[(1)]
  Suppose that there exists a vector $u$ such that $u \wedge v = w$.
  So
  \[
    v \cdot w = w \cdot v = (u \wedge v) \cdot v = 0
  \]
  or $v$ is perpendicular to $w$.

\item[(2)]
  Suppose that $v$ is perpendicular to $w$.
  Linear algebra says that there exists a vector $u$
  such that $\{ \widetilde{u}, v, w \}$ is a basis of $\mathbb{R}^3$.
  Note that $\widetilde{u} \wedge v$ is parallel to $w$, say
  $\widetilde{u} \wedge v = c w$ for some nonzero constant $c \in \mathbb{R}$.
  Take $u = c^{-1} \widetilde{u}$ and thus
  \[
    u \wedge v = c^{-1} \widetilde{u} \wedge v = c^{-1} cw = w.
  \]
  (Note that $\{ u, v, w \}$ is also a basis of $\mathbb{R}^3$.)

\item[(3)]
  Such vector $u$ is not uniquely determined.
  Let $L$ be a line passing the point $u = (u_1, u_2, u_3)$ and parallel to the vector $v$.
  By the definition of vector product,
  for any point $p = (p_1, p_2, p_3) \in L$ we have $p \wedge v = w$ as a vector.
\end{enumerate}
$\Box$ \\\\



%%%%%%%%%%%%%%%%%%%%%%%%%%%%%%%%%%%%%%%%%%%%%%%%%%%%%%%%%%%%%%%%%%%%%%%%%%%%%%%%



\subsubsection*{Exercise 1-4.13.}
\addcontentsline{toc}{subsubsection}{Exercise 1-4.13.}
\emph{Let $u(t) = (u_1(t), u_2(t), u_3(t))$ and $v(t) = (v_1(t), v_2(t), v_3(t))$
be differentiable maps from the interval $(a,b)$ into $\mathbb{R}^3$.
If the derivatives $u'(t)$ and $v'(t)$ satisfy the conditions
\[
  u'(t) = au(t) + bv(t),
  \qquad
  v'(t) = cu(t) - av(t),
\]
where $a$, $b$, and $c$ are constants, show that
$u(t) \wedge v(t)$ is a constant vector.} \\



\emph{Proof.}
Since
\begin{align*}
  \frac{d}{dt}(u(t) \wedge v(t))
  &= u'(t) \wedge v(t) + u(t) \wedge v'(t) \\
  &= (au(t) + bv(t))\wedge v(t) + u(t) \wedge (cu(t) - av(t)) \\
  &= au(t) \wedge v(t) + u(t) \wedge (-av(t)) \\
  &= a(u(t) \wedge v(t)) + (-a)(u(t) \wedge v(t)) \\
  &= (0, 0, 0),
\end{align*}
$u(t) \wedge v(t)$ is a constant vector.
$\Box$ \\\\



%%%%%%%%%%%%%%%%%%%%%%%%%%%%%%%%%%%%%%%%%%%%%%%%%%%%%%%%%%%%%%%%%%%%%%%%%%%%%%%%



\subsubsection*{Exercise 1-4.14.}
\addcontentsline{toc}{subsubsection}{Exercise 1-4.14.}
\emph{Find all unit vectors which are perpendicular to the vector $(2,2,1)$ and
parallel to the plane determined by the points
$(0,0,0)$, $(1,-2,1)$, $(-1,1,1)$.} \\



\emph{Proof.}
\begin{enumerate}
\item[(1)]
  Let $E$ be the plane determined by the points
  $p_0 = (0,0,0)$, $p_1 = (1,-2,1)$, $p_2 (-1,1,1)$.
  The normal vector of $E$ is
  \[
    (p_1 - p_0) \wedge (p_2 - p_0)
    = (1,-2,1) \wedge (-1,1,1)
    = (-3,-2,-1).
  \]

\item[(2)]
  All unit vectors which are perpendicular to $(2,2,1)$ and parallel to $E$
  are all unit vectors which are perpendicular to $(2,2,1)$ and $(-3,-2,-1)$.
  Note that such vector is in the direction of
  \[
    (2,2,1) \wedge (-3,-2,-1) = (0,-1,2).
  \]
  Hence, the desired unit vectors are
  \[
    \pm \left( 0,-\frac{1}{\sqrt{5}},\frac{2}{\sqrt{5}} \right).
  \]
\end{enumerate}
$\Box$ \\\\



%%%%%%%%%%%%%%%%%%%%%%%%%%%%%%%%%%%%%%%%%%%%%%%%%%%%%%%%%%%%%%%%%%%%%%%%%%%%%%%%
%%%%%%%%%%%%%%%%%%%%%%%%%%%%%%%%%%%%%%%%%%%%%%%%%%%%%%%%%%%%%%%%%%%%%%%%%%%%%%%%



\subsection*{1-5. The Local Theory of Curves Parametrized by Arc Length \\}
\addcontentsline{toc}{subsection}{1-5. The Local Theory of Curves Parametrized by Arc Length}



\emph{Unless explicity stated, $\alpha: I \to \mathbb{R}^3$ is a curve parametrized by arc length $s$,
with curvature $\kappa(s) \neq 0$, for all $s \in I$.} \\


\subsubsection*{Exercise 1-5.1.}
\addcontentsline{toc}{subsubsection}{Exercise 1-5.1.}
\emph{Given the parametrized curve (helix)
\[
  \alpha(s) = \left( a \cos\frac{s}{c}, a \sin\frac{s}{c}, b \frac{s}{c} \right),
  \qquad s \in \mathbb{R},
\]
where $c^2 = a^2 + b^2$.}
\begin{enumerate}
\item[(a)]
  \emph{Show that the parameter $s$ is the arc length.}

\item[(b)]
  \emph{Determine the curvature and the torsion of $\alpha$.}

\item[(c)]
  \emph{Determine the osculating plane of $\alpha$.}

\item[(d)]
  \emph{Show that the lines containing $n(s)$ and passing through $\alpha(s)$
  meet the $z$ axis under a constant angle equal to $\frac{\pi}{2}$.}

\item[(e)]
  \emph{Show that the tangent lines to $\alpha$ make a constant angle with the $z$ axis.} \\
\end{enumerate}



\emph{Proof of (a).}
  Since
  \[
    \alpha'(s)
    = \left( -\frac{a}{c} \sin\frac{s}{c}, \frac{a}{c} \cos\frac{s}{c}, \frac{b}{c} \right)
  \]
  or $\abs{ \alpha'(s) } = 1$,
  \[
    s(t)
    = \int_{0}^{t} \abs{ \alpha'(u) } du
    = \int_{0}^{t} du
    = t
  \]
  is indeed the arc length.
$\Box$ \\



\emph{Proof of (b).}
\begin{enumerate}
\item[(1)]
  Note that
  \[
    \alpha''(s)
    = \left( -\frac{a}{c^2} \cos\frac{s}{c}, -\frac{a}{c^2} \sin\frac{s}{c}, 0 \right).
  \]
  So the curvature of $\alpha$ is
  \[
    \kappa(s)
    = \abs{ \alpha''(s) }
    = \frac{\abs{a}}{c^2}.
  \]

\item[(2)]
  Note that
  \[
    n(s)
    = \frac{1}{\kappa(s)} \alpha''(s)
    = -\mathrm{sgn}(a) \left( \cos\frac{s}{c}, \sin\frac{s}{c}, 0 \right).
  \]
  Hence,
  \begin{align*}
    & \: b(s)
    = t(s) \wedge n(s)
    = \mathrm{sgn}(a)
      \left( \frac{b}{c} \sin\frac{s}{c}, \frac{b}{c} \cos\frac{s}{c}, \frac{a}{c} \right) \\
    \Longrightarrow& \:
    b'(s)
    = \mathrm{sgn}(a)
      \left( \frac{b}{c^2} \cos\frac{s}{c}, -\frac{b}{c^2} \sin\frac{s}{c}, 0 \right) \\
    \Longrightarrow& \:
    \tau(s) = \abs{ b'(s) } = \frac{\abs{b}}{c^2}.
  \end{align*}
\end{enumerate}
$\Box$ \\



\emph{Proof of (c).}
  Since the binormal vector $b(s)$ is normal to the osculating plane,
  the osculating plane is
  \[
    \left( \frac{b}{c} \sin\frac{s}{c} \right) x
      + \left( \frac{b}{c} \cos\frac{s}{c} \right) y
      + \left( \frac{a}{c} \right) z
    = \frac{ab}{c} \sin\frac{2s}{c} + \frac{ab}{c^2} s
  \]
  for $s \in I$.
$\Box$ \\



\emph{Proof of (d).}
\begin{enumerate}
\item[(1)]
  The line $L$ containing $n(s)$ and passing through $\alpha(s)$ is
  \begin{align*}
    x &= a \cos\frac{s}{c} + \cos\frac{s}{c} t, \\
    y &= a \sin\frac{s}{c} + \sin\frac{s}{c} t, \\
    z &= b \frac{s}{c},
  \end{align*}
  for all $t \in \mathbb{R}$.
  Hence $L$ meets the $z$ axis at $t = -a$.

\item[(2)]
  The directional vector of $L$ (resp. the $z$ axis) is
  $\left( \cos\frac{s}{c}, \sin\frac{s}{c}, 0 \right)$ (resp. $(0,0,1)$).
  Since the inner product
  \[
    \left( \cos\frac{s}{c}, \sin\frac{s}{c}, 0 \right)
    \cdot (0,0,1) = 0,
  \]
  $L$ meets the $z$ axis under a constant angle $= \frac{\pi}{2}$.
\end{enumerate}
$\Box$ \\



\emph{Proof of (e).}
  Note that the directional vector of the tangent line (resp. the $z$ axis) is
  $t(s)$ (resp. $(0,0,1)$).
  Since the inner product of $t(s)$ and $(0,0,1)$ is a constant $\frac{b}{c}$,
  the conclusion holds.
$\Box$ \\\\



%%%%%%%%%%%%%%%%%%%%%%%%%%%%%%%%%%%%%%%%%%%%%%%%%%%%%%%%%%%%%%%%%%%%%%%%%%%%%%%%



\subsubsection*{Exercise 1-5.2.}
\addcontentsline{toc}{subsubsection}{Exercise 1-5.2.}
\emph{Show that the torsion $\tau$ of $\alpha$ is given by
$$\tau(s) = -\frac{\alpha'(s) \wedge \alpha''(s) \cdot \alpha'''(s)}{|\kappa(s)|^2}.$$} \\



\emph{Proof.}
\begin{enumerate}
\item[(1)]
  Take inner product $n(s)$
  to the definition of torsion $\tau(s) n(s) = b'(s)$,
  we have $$\tau(s) = b'(s) \cdot n(s).$$
  Since $b'(s) = t(s) \wedge n'(s)$, we have to compute $n'(s)$ first.
\item[(2)]
  Compute $n'(s)$.
  $$n'(s)
  = \frac{d}{ds} \left( \frac{\alpha''(s)}{\kappa(s)} \right) \\
  = \frac{\alpha'''(s)}{\kappa(s)} - \frac{\alpha''(s)\kappa'(s)}{\kappa(s)^2}.$$
\item[(3)]
  By (1)(2),
  \begin{align*}
  \tau(s)
  &= b'(s) \cdot n(s) \\
  &= (t(s) \wedge n'(s)) \cdot n(s) \\
  &= \left(
    \alpha'(s)
      \wedge
    \left(
      \frac{\alpha'''(s)}{\kappa(s)}
      - \frac{\alpha''(s)\kappa'(s)}{\kappa(s)^2}
    \right)
  \right)
  \cdot \frac{\alpha''(s)}{\kappa(s)} \\
  &= \left(
    \alpha'(s) \wedge \frac{\alpha'''(s)}{\kappa(s)}
  \right)
  \cdot \frac{\alpha''(s)}{\kappa(s)} \\
  &= \frac{\alpha'(s) \wedge \alpha'''(s) \cdot \alpha''(s)}{|\kappa(s)|^2},
  \end{align*}
  or
  $$\tau(s) = \frac{\alpha'(s) \wedge \alpha'''(s) \cdot \alpha''(s)}{\alpha''(s)^2}.$$
\end{enumerate}
$\Box$ \\\\



%%%%%%%%%%%%%%%%%%%%%%%%%%%%%%%%%%%%%%%%%%%%%%%%%%%%%%%%%%%%%%%%%%%%%%%%%%%%%%%%
%%%%%%%%%%%%%%%%%%%%%%%%%%%%%%%%%%%%%%%%%%%%%%%%%%%%%%%%%%%%%%%%%%%%%%%%%%%%%%%%



\subsection*{1-6. The Local Canonical Form \\}
\addcontentsline{toc}{subsection}{1-6. The Local Canonical Form}



%%%%%%%%%%%%%%%%%%%%%%%%%%%%%%%%%%%%%%%%%%%%%%%%%%%%%%%%%%%%%%%%%%%%%%%%%%%%%%%%
%%%%%%%%%%%%%%%%%%%%%%%%%%%%%%%%%%%%%%%%%%%%%%%%%%%%%%%%%%%%%%%%%%%%%%%%%%%%%%%%



\subsection*{1-7. Global Properties of Plane Curves \\}
\addcontentsline{toc}{subsection}{1-7. Global Properties of Plane Curves}



%%%%%%%%%%%%%%%%%%%%%%%%%%%%%%%%%%%%%%%%%%%%%%%%%%%%%%%%%%%%%%%%%%%%%%%%%%%%%%%%
%%%%%%%%%%%%%%%%%%%%%%%%%%%%%%%%%%%%%%%%%%%%%%%%%%%%%%%%%%%%%%%%%%%%%%%%%%%%%%%%



\end{document}