\documentclass{article}
\usepackage{amsfonts}
\usepackage{amsmath}
\usepackage{amssymb}
\usepackage{hyperref}
\usepackage[none]{hyphenat}
\usepackage{mathrsfs}
\usepackage{physics}
\parindent=0pt

\def\upint{\mathchoice%
    {\mkern13mu\overline{\vphantom{\intop}\mkern7mu}\mkern-20mu}%
    {\mkern7mu\overline{\vphantom{\intop}\mkern7mu}\mkern-14mu}%
    {\mkern7mu\overline{\vphantom{\intop}\mkern7mu}\mkern-14mu}%
    {\mkern7mu\overline{\vphantom{\intop}\mkern7mu}\mkern-14mu}%
  \int}
\def\lowint{\mkern3mu\underline{\vphantom{\intop}\mkern7mu}\mkern-10mu\int}

\begin{document}



\textbf{\Large Chapter 1: Curves} \\\\



\emph{Author: Meng-Gen Tsai} \\
\emph{Email: plover@gmail.com} \\\\



% http://www.math.ualberta.ca/~xinweiyu/348.A1.17f/L07_Curves_I_20170926.pdf
% https://mast.queensu.ca/~offind/math280/solutions05.pdf



%%%%%%%%%%%%%%%%%%%%%%%%%%%%%%%%%%%%%%%%%%%%%%%%%%%%%%%%%%%%%%%%%%%%%%%%%%%%%%%%



\textbf{\large Section 1-1: Introduction} \\

Classical differential geometry: the study of local properties of
curves and surfaces. \\

Global differential geometry: the study of the influence of the local properties
on the behavior of the entire curve and surface. \\

\emph{No exercises.} \\\\



%%%%%%%%%%%%%%%%%%%%%%%%%%%%%%%%%%%%%%%%%%%%%%%%%%%%%%%%%%%%%%%%%%%%%%%%%%%%%%%%
%%%%%%%%%%%%%%%%%%%%%%%%%%%%%%%%%%%%%%%%%%%%%%%%%%%%%%%%%%%%%%%%%%%%%%%%%%%%%%%%



\textbf{\large Section 1-2: Parametrized Curves} \\

\textbf{Exercise 1-2.1.}
\emph{Find a parametrized curve $\alpha(t)$ whose trace is the circle
$x^2 + y^2 = 1$ such that $\alpha(t)$ runs clockwise around the circle
with $\alpha(0) = (0,1)$.} \\

\emph{Proof.}
$\alpha(t) = (\sin t, \cos t)$, $t \in \mathbb{R}$.
$\Box$ \\\\



\textbf{Exercise 1-2.2.}
\emph{Let $\alpha(t)$ be a parametrized curve which does not pass through the origin.
If $\alpha(t_0)$ is the point of the trace of $\alpha$ closest to the origin
and $\alpha'(t_0) \neq 0$, show that the position vector $\alpha(t_0)$ is
orthogonal to $\alpha'(t_0)$.} \\

\emph{Proof.}
Let $f(t) = |\alpha(t)|^2 = \alpha(t) \cdot \alpha(t)$.
$f(t)$ is differentiable and $f(t)$ has a local minimum at a point $t = t_0 \in I$.
So $f'(t_0) = 0$. [Theorem 5.8 in W. Rudin, Principles of Mathematical Analysis, 3rd edition.]
Since
$$f'(t) = 2 \alpha(t) \cdot \alpha'(t),$$
$f'(t_0) = 2 \alpha(t_0) \cdot \alpha'(t_0) = 0$,
or $\alpha(t_0) \cdot \alpha'(t_0) = 0$.
Since $\alpha(t_0) \neq 0$ and $\alpha'(t_0) \neq 0$,
$\alpha(t_0)$ is orthogonal to $\alpha'(t_0)$.
$\Box$ \\\\



\textbf{Exercise 1-2.3.}
\emph{A parametrized curve $\alpha(t)$ has a property that its second derivative
$\alpha''(t)$ is identically zero.
What can be said about $\alpha$?} \\

$\alpha(t)$ is a straight line. \\

\emph{Proof.}
Since $\alpha''(t)$ is identically zero,
$\alpha'(t) = a$ is a constant.
[Theorem 5.11 in W. Rudin, Principles of Mathematical Analysis, 3rd edition.]
Define $f(t) = \alpha(t) - at$ (on $I$).
Since $f'(t) = \alpha'(t) - a = 0$,
$f(t) = \alpha(t) - at = b$ is a constant again.
Therefore, $\alpha(t) = at+b$, which is a straight line (on $I$).
$\Box$ \\\\



\textbf{Exercise 1-2.4.}
\emph{Let $\alpha: I \to \mathbb{R}^3$ be a parametrized curve and
let $v \in \mathbb{R}^3$ be a fixed vector.
Assume that $\alpha'(t)$ is orthogonal to $v$ for all $t \in I$
and that $\alpha(0)$ is orthogonal to $v$.
Prove that $\alpha(t)$ is orthogonal to $v$ for all $t \in I$.} \\

Need to assume that $\alpha(t) \neq 0$ for all $t \in I$. \\

\emph{Proof.}
Given any $t \neq 0 \in I$. (Nothing to do at $t = 0$.)
Define $f: I \to \mathbb{R}$ by $f(t) = \alpha(t) \cdot v$.
By the mean value theorem, there exists a point $\xi$ between $0$ and $t$
such that
$$f(t) - f(0) = f'(\xi)(t - 0),$$
where $f'(t) = \alpha'(t) \cdot v + \alpha(t) \cdot v' = \alpha'(t) \cdot v$.
Note that $f(0) = 0$ since $\alpha(0)$ is orthogonal to $v$,
and $f'(\xi) = 0$ since $\alpha'(t)$ is orthogonal to $v$.
So the identity is reduced to
$$f(t) = 0,$$
or $\alpha(t) \cdot v = 0$,
or $\alpha(t)$ is orthogonal to $v$.
$\Box$ \\\\



\textbf{Exercise 1-2.5.}
\emph{Let $\alpha: I \to \mathbb{R}^3$ be a parametrized curve,
with $\alpha'(t) \neq 0$ for all $t \in I$.
Show that $|\alpha(t)|$ is a nonzero constant
if and only if
$\alpha(t)$ is orthogonal to $\alpha'(t)$ for all $t \in I$.} \\

The same trick in Exercise 1-2.2. \\

\emph{Proof.}
It is equivalent to
\emph{show that $|\alpha(t)|^2$ is a nonzero constant
if and only if
$\alpha(t)$ is orthogonal to $\alpha'(t)$ for all $t \in I$.}
Let $$f(t) = |\alpha(t)|^2 = \alpha(t) \cdot \alpha(t).$$
Notice that $\alpha'(t) \neq 0$, and thus
\begin{align*}
&|\alpha(t)| \text{ is a nonzero constant} \\
\Longleftrightarrow&
f(t) = |\alpha(t)|^2 \text{ is a nonzero constant} \\
\Longleftrightarrow&
f'(t) = 0 \text{ and } f(t) \text{ is a nonzero constant} \\
\Longleftrightarrow&
\alpha(t) \cdot \alpha'(t) = 0 \text{ and } \alpha(t) \text{ is a nonzero constant} \\
\Longleftrightarrow&
\alpha(t) \text{ is orthogonal to } \alpha'(t) \text{ for all } t \in I.
\end{align*}
$\Box$ \\\\



% No exercises left.



%%%%%%%%%%%%%%%%%%%%%%%%%%%%%%%%%%%%%%%%%%%%%%%%%%%%%%%%%%%%%%%%%%%%%%%%%%%%%%%%
%%%%%%%%%%%%%%%%%%%%%%%%%%%%%%%%%%%%%%%%%%%%%%%%%%%%%%%%%%%%%%%%%%%%%%%%%%%%%%%%



\textbf{\large Section 1-3: Regular Curves; Arc Length} \\



\textbf{Exercise 1-3.1.}
\emph{Show that the tangent lines to the regular parametrized curve
$\alpha(t) = (3t, 3t^2, 2t^3)$ make a constant angle with the line $y = 0$, $z = x$. } \\

\emph{Proof.}
$\alpha'(t) = (3, 6t, 6t^2)$.
The line $y = 0$, $z = x$ is $\beta(t) = (1,0,1)$.
The cosine of the angle $\theta$ between these to curves is
\begin{align*}
\cos \theta
&= \frac{(3, 6t, 6t^2) \cdot (1,0,1)}{|(3, 6t, 6t^2)||(1,0,1)|} \\
&= \frac{3+6t^2}{\sqrt{3^2 + (6t)^2 + (6t^2)^2}\sqrt{2}} \\
&= \frac{3+6t^2}{\sqrt{9 + 36 t^2 + 36t^4}\sqrt{2}} \\
&= \frac{3+6t^2}{\sqrt{(3 + 6t^2)^2}\sqrt{2}} \\
&= \frac{1}{\sqrt{2}}.
\end{align*}
(Notice $3 + 6t^2 > 0$ for all $t \in \mathbb{R}$.)
That is, the angle between $\alpha'$ and $\beta$ is a constant ($= \pi/4$).
$\Box$ \\\\



%%%%%%%%%%%%%%%%%%%%%%%%%%%%%%%%%%%%%%%%%%%%%%%%%%%%%%%%%%%%%%%%%%%%%%%%%%%%%%%%



\textbf{Exercise 1-3.2.}
\emph{A circular disk of radius $1$ in the plane $xy$ rolls without slipping
along the $x$ axis.
The figure described by a point of the circumference of of the disk is
called a \textbf{cycloid}
(Figure 1-7 in Mantredo P. do Carmo, Differential Geometry of Curves and Surfaces).}
\begin{enumerate}
  \item[(a)]
  \emph{Obtain a parametrized curve $\alpha: \mathbb{R} \to \mathbb{R}^2$
  the trace of which is the cycloid and determine its singular points.}
  \item[(b)]
  \emph{Compute the arc length of the cycloid
  corresponding to a complete rotation of the disk.} \\
\end{enumerate}

\emph{Proof of (a).}
\begin{enumerate}
\item[(1)]
Since
\begin{equation*}
  \begin{cases}
     x = t - \sin t \\
     y = 1 - \cos t,
  \end{cases}
\end{equation*}
we define $\alpha(t) = (t - \sin t, 1 - \cos t)$.
\item[(2)]
$\alpha'(t) = (1 - \cos t, \sin t)$.
$\alpha'(t) = 0$ if and only if $t = 2n\pi$ where $n \in \mathbb{Z}$.
That is, all singular points are $\alpha(2n\pi) = (2n\pi, 0)$ where $n \in \mathbb{Z}$.
\end{enumerate}
$\Box$ \\

\emph{Proof of (b).}
The arc length of the cycloid corresponding to a complete rotation of the disk is
\begin{align*}
\int_{0}^{2\pi} |\alpha'(t)| dt
&= \int_{0}^{2\pi} \sqrt{(1-\cos t)^2 + (\sin t)^2} dt \\
&= \int_{0}^{2\pi} \sqrt{2} \sqrt{1 - \cos t} dt \\
&= \int_{0}^{2\pi} 2 \sin \frac{t}{2} dt \\
&= \left[ -4 \cos\frac{t}{2} \right]_{t=0}^{t=2\pi} \\
&= 8.
\end{align*}
$\Box$ \\

\textbf{Supplement.}
The cycloid is not an algebraic curve. \\\\



%%%%%%%%%%%%%%%%%%%%%%%%%%%%%%%%%%%%%%%%%%%%%%%%%%%%%%%%%%%%%%%%%%%%%%%%%%%%%%%%



\textbf{Exercise 1-3.4.}
\emph{Let $\alpha: (0, \pi) \to \mathbb{R}^2$ be given by
$$\alpha(t) = \left(\sin t, \cos t + \log\tan\frac{t}{2}\right),$$
where $t$ is the angle that the $y$ axis makes with the vector $\alpha(t)$.
The trace of $\alpha$ is called the \textbf{tractrix}.
(Figure 1-9 in Mantredo P. do Carmo, Differential Geometry of Curves and Surfaces).
Show that}
\begin{enumerate}
  \item[(a)]
  \emph{$\alpha$ is a differentiable parametrized curve,
  regular except at $t = \frac{\pi}{2}$.}
  \item[(b)]
  \emph{The length of the segment of the tangent of the tractrix between
  the point of tangency and the $y$ axis is constantly equal to $1$.} \\
\end{enumerate}

\emph{Proof of (a).}
\begin{align*}
\alpha'(t)
&= \left(
\cos t,
-\sin t + \frac{1}{\tan\frac{t}{2}} \frac{1}{\cos^{2}\frac{t}{2}} \frac{1}{2}
\right) \\
&= \left(
\cos t,
-\sin t + \frac{1}{2 \sin\frac{t}{2} \cos\frac{t}{2}}
\right) \\
&= \left(
\cos t,
\frac{\cos^2 t}{\sin t}
\right)
\end{align*}
exists.
And $\alpha'(t) = 0$ if and only if $t = \frac{\pi}{2}$.
That is, there is an unique singular point at $t = \frac{\pi}{2}$.
$\Box$ \\

\emph{Proof of (b).}
The the tangent line of the tractrix through the regular point $t$
is parametrized by $\beta: \mathbb{R} \to \mathbb{R}^2$ which is defined by
\begin{align*}
\beta(u)
&= u\alpha'(t) + \alpha(t) \\
&= \left( u \cos t + \sin t, u \frac{\cos^2 t}{\sin t} + \cos t + \log\tan\frac{t}{2} \right).
\end{align*}
By construction, this tangent line $\beta(u)$ meets the tractrix at $u = 0$,
and meets the $y$-axis when $u \cos t + \sin t = 0$ or $u = -\tan t$.
So the length of the segment is
\begin{align*}
|\beta(0) - \beta(-\tan t)|
&= \sqrt{(-\tan t \cos t)^2+ \left( -\tan t \frac{\cos^2 t}{\sin t} \right)^2} \\
&= \sqrt{(\sin t)^2+ (\cos t)^2} \\
&= 1.
\end{align*}
$\Box$ \\\\



%%%%%%%%%%%%%%%%%%%%%%%%%%%%%%%%%%%%%%%%%%%%%%%%%%%%%%%%%%%%%%%%%%%%%%%%%%%%%%%%



\textbf{Exercise 1-3.10.}
\emph{(Straight Lines as Shortest.)
Let $\alpha: I \to \mathbb{R}^3$ be a parametrized curve.
Let $[a,b] \subseteq I$ and set $\alpha(a) = p$, $\alpha(b) = q$.}
\begin{enumerate}
  \item[(a)]
  \emph{Show that, for any constant vector $v$, $|v| = 1$,
  $$(q-p) \cdot v
  = \int_{a}^{b} \alpha'(t) \cdot v dt
  \leq \int_{a}^{b} |\alpha'(t)| dt.$$}
  \item[(b)]
  \emph{Set
  $$v = \frac{q-p}{|q-p|}$$
  and show that
  $$|\alpha(b) - \alpha(a)| \leq \int_{a}^{b} |\alpha'(t)| dt;$$
  that is, the curve of shortest length from
  $\alpha(a)$ to $\alpha(b)$ is the straight line
  joining these points.} \\
\end{enumerate}

Assume $p \neq q$ (otherwise $v = \frac{q-p}{|q-p|}$ is meaningless). \\

\emph{Proof of (a).}
Let $f(t) = \alpha(t) \cdot v$ defined on $I$.
By the fundamental theorem of calculus,
$$\int_{a}^{b} f'(t) dt = f(b) - f(a).$$
Since $f'(t) = \alpha'(t) \cdot v$,
$$(\alpha(b) - \alpha(a)) \cdot v = \int_{a}^{b} \alpha'(t) \cdot v dt.$$
Therefore,
\begin{align*}
(q - p) \cdot v
&= \int_{a}^{b} \alpha'(t) \cdot v dt \\
&\leq \int_{a}^{b} |\alpha'(t) \cdot v| dt \\
&\leq \int_{a}^{b} |\alpha'(t)||v| dt \\
&= \int_{a}^{b} |\alpha'(t)| dt.
\end{align*}
$\Box$ \\

\emph{Proof of (b).}
$|v| = \frac{|q-p|}{|q-p|} = 1$.
So,
\begin{align*}
(q-p) \cdot \frac{q-p}{|q-p|}
&\leq \int_{a}^{b} |\alpha'(t)| dt, \\
|q-p|
&\leq \int_{a}^{b} |\alpha'(t)| dt.
\end{align*}
$\Box$ \\\\



%%%%%%%%%%%%%%%%%%%%%%%%%%%%%%%%%%%%%%%%%%%%%%%%%%%%%%%%%%%%%%%%%%%%%%%%%%%%%%%%



\textbf{\large Section 1-4: The Vector Product in $\mathbb{R}^{3}$} \\

\textbf{Exercise 1-4.13.}
\emph{Let $u(t) = (u_1(t), u_2(t), u_3(t))$ and $v(t) = (v_1(t), v_2(t), v_3(t))$
be differentiable maps from the interval $(a,b)$ into $\mathbb{R}^3$.
If the derivatives $u'(t)$ and $v'(t)$ satisfy the conditions
$$u'(t) = au(t) + bv(t), v'(t) = cu(t) - av(t),$$
where $a$, $b$, and $c$ are constants, show that
$u(t) \wedge v(t)$ is a constant vector.} \\

\emph{Proof.}
Since
\begin{align*}
  \frac{d}{dt}(u(t) \wedge v(t))
  &= u'(t) \wedge v(t) + u(t) \wedge v'(t) \\
  &= (au(t) + bv(t))\wedge v(t) + u(t) \wedge (cu(t) - av(t)) \\
  &= au(t) \wedge v(t) + u(t) \wedge (-av(t)) \\
  &= a(u(t) \wedge v(t)) + (-a)(u(t) \wedge v(t)) \\
  &= (0, 0, 0),
\end{align*}
$u(t) \wedge v(t)$ is a constant vector.
$\Box$ \\\\



%%%%%%%%%%%%%%%%%%%%%%%%%%%%%%%%%%%%%%%%%%%%%%%%%%%%%%%%%%%%%%%%%%%%%%%%%%%%%%%%



\textbf{\large Section 1-5: The Local Theory of Curves Parametrized by Arc Length} \\

\textbf{Exercise 1-5.2.}
\emph{Show that the torsion $\tau$ of $\alpha$ is given by
$$\tau(s) = -\frac{\alpha'(s) \wedge \alpha''(s) \cdot \alpha'''(s)}{|\kappa(s)|^2}.$$} \\

\emph{Proof.}
\begin{enumerate}
\item[(1)]
Take inner product $n(s)$
to the definition of torsion $\tau(s) n(s) = b'(s)$,
we have $$\tau(s) = b'(s) \cdot n(s).$$
Since $b'(s) = t(s) \wedge n'(s)$, we have to compute $n'(s)$ first.
\item[(2)]
Compute $n'(s)$.
$$n'(s)
= \frac{d}{ds} \left( \frac{\alpha''(s)}{\kappa(s)} \right) \\
= \frac{\alpha'''(s)}{\kappa(s)} - \frac{\alpha''(s)\kappa'(s)}{\kappa(s)^2}.$$
\item[(3)]
By (1)(2),
\begin{align*}
\tau(s)
&= b'(s) \cdot n(s) \\
&= (t(s) \wedge n'(s)) \cdot n(s) \\
&= \left(
  \alpha'(s)
    \wedge
  \left(
    \frac{\alpha'''(s)}{\kappa(s)}
    - \frac{\alpha''(s)\kappa'(s)}{\kappa(s)^2}
  \right)
\right)
\cdot \frac{\alpha''(s)}{\kappa(s)} \\
&= \left(
  \alpha'(s) \wedge \frac{\alpha'''(s)}{\kappa(s)}
\right)
\cdot \frac{\alpha''(s)}{\kappa(s)} \\
&= \frac{\alpha'(s) \wedge \alpha'''(s) \cdot \alpha''(s)}{|\kappa(s)|^2}, \\
\end{align*}
or
$$\tau(s) = \frac{\alpha'(s) \wedge \alpha'''(s) \cdot \alpha''(s)}{\alpha''(s)^2}.$$

\end{enumerate}
$\Box$ \\\\



%%%%%%%%%%%%%%%%%%%%%%%%%%%%%%%%%%%%%%%%%%%%%%%%%%%%%%%%%%%%%%%%%%%%%%%%%%%%%%%%



\textbf{\large Section 1-6: The Local Canonical Form} \\\\



%%%%%%%%%%%%%%%%%%%%%%%%%%%%%%%%%%%%%%%%%%%%%%%%%%%%%%%%%%%%%%%%%%%%%%%%%%%%%%%%



\textbf{\large Section 1-7: Global Properties of Plane Curves} \\\\



%%%%%%%%%%%%%%%%%%%%%%%%%%%%%%%%%%%%%%%%%%%%%%%%%%%%%%%%%%%%%%%%%%%%%%%%%%%%%%%%



\end{document}