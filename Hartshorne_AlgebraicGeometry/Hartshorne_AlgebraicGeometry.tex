\documentclass{article}
\usepackage{amsfonts}
\usepackage{amsmath}
\usepackage{amssymb}
\usepackage{centernot}
\usepackage{hyperref}
\usepackage[none]{hyphenat}
\usepackage{mathrsfs}
\usepackage{mathtools}
\usepackage{physics}
\usepackage{tikz-cd}
\parindent=0pt



\title{\textbf{Notes on the book: \\ \emph{Robin Hartshorne, Algebraic Geometry}}}
\author{Meng-Gen Tsai \\ plover@gmail.com}



\begin{document}
\maketitle
\tableofcontents



%%%%%%%%%%%%%%%%%%%%%%%%%%%%%%%%%%%%%%%%%%%%%%%%%%%%%%%%%%%%%%%%%%%%%%%%%%%%%%%%
%%%%%%%%%%%%%%%%%%%%%%%%%%%%%%%%%%%%%%%%%%%%%%%%%%%%%%%%%%%%%%%%%%%%%%%%%%%%%%%%
%%%%%%%%%%%%%%%%%%%%%%%%%%%%%%%%%%%%%%%%%%%%%%%%%%%%%%%%%%%%%%%%%%%%%%%%%%%%%%%%
%%%%%%%%%%%%%%%%%%%%%%%%%%%%%%%%%%%%%%%%%%%%%%%%%%%%%%%%%%%%%%%%%%%%%%%%%%%%%%%%



% Reference:
% https://sv.20file.org/up1/1431_0.pdf
% https://www.math.arizona.edu/~cais/CourseNotes/AlgGeom04/Hartshorne_Solutions.pdf



%%%%%%%%%%%%%%%%%%%%%%%%%%%%%%%%%%%%%%%%%%%%%%%%%%%%%%%%%%%%%%%%%%%%%%%%%%%%%%%%
%%%%%%%%%%%%%%%%%%%%%%%%%%%%%%%%%%%%%%%%%%%%%%%%%%%%%%%%%%%%%%%%%%%%%%%%%%%%%%%%
%%%%%%%%%%%%%%%%%%%%%%%%%%%%%%%%%%%%%%%%%%%%%%%%%%%%%%%%%%%%%%%%%%%%%%%%%%%%%%%%
%%%%%%%%%%%%%%%%%%%%%%%%%%%%%%%%%%%%%%%%%%%%%%%%%%%%%%%%%%%%%%%%%%%%%%%%%%%%%%%%



\newpage
\section*{Chapter I: Varieties \\}
\addcontentsline{toc}{section}{Chapter I: Varieties}



\subsection*{\S I.1 Affine Varieties \\}
\addcontentsline{toc}{subsection}{\S I.1 Affine Varieties}



\subsubsection*{Exercise I.1.6.}
\addcontentsline{toc}{subsubsection}{Exercise I.1.6.}
\emph{Any nonempty open subset of an irreducible topological space is dense and irreducible.
If $Y$ is a subset of a topological space $X$, which is irreducible in its induced topology,
then the closure $\overline{Y}$ is also irreducible.} \\



\emph{Proof.}
\begin{enumerate}
\item[(1)]
\emph{Show that any nonempty open subset of an irreducible topological space
is dense.}
It suffices to show that $U_1 \cap U_2 \neq \varnothing$ for any
nonempty open subsets of an irreducible topological space.
  \begin{align*}
  & \: \forall \text{ nonempty open sets } U_1 \text{ and } U_2,
  U_1 \cap U_2 \neq \varnothing \\
  \Longleftrightarrow& \:
  \forall \text{ nonempty open sets } U_1 \text{ and } U_2,
  X - (U_1 \cap U_2) \neq X \\
  \Longleftrightarrow& \:
  \forall \text{ nonempty open sets } U_1 \text{ and } U_2,
  (X - U_1) \cup (X - U_2) \neq X \\
  \Longleftrightarrow& \:
  \forall \text{ proper closed sets } Y_1 \text{ and } Y_2,
  Y_1 \cup Y_2 \neq X \\
  \Longleftrightarrow& \:
  \nexists \text{ proper closed sets } Y_1 \text{ and } Y_2,
  Y_1 \cup Y_2 = X.
  \end{align*}
\item[(2)]
\emph{Show that any nonempty open subset of an irreducible topological space
is irreducible.}
Given any open subset $U$ of an irreducible topological space $X$.
Write $U \subseteq Y_1 \cup Y_2$ where $Y_1$ and $Y_2$ are closed in $X$.
  \begin{align*}
  & \: U \subseteq Y_1 \cup Y_2 \\
  \Longrightarrow& \:
  \overline{U} \subseteq \overline{Y_1 \cup Y_2} \\
  \Longrightarrow& \:
  X \subseteq Y_1 \cup Y_2
    &\text{($U$ is dense, $Y_1 \cup Y_2$ is closed)} \\
  \Longrightarrow& \:
  Y_1 = X \supseteq U \text{ or } Y_2 = X \supseteq U
    &\text{($X$ is irreducible)} \\
  \Longrightarrow& \:
  \text{$U$ is irreducible.}
  \end{align*}
\item[(3)]
\emph{Show that if $Y$ is a subset of a topological space $X$,
which is irreducible (in its induced topology),
then the closure $\overline{Y}$ is also irreducible.}
(Reductio ad absurdum)
If $\overline{Y}$ were reducible, there are two closed sets $Y_1$ and $Y_2$
such that
$$\overline{Y} \subseteq Y_1 \cup Y_2,
\overline{Y} \not\subseteq Y_i (i = 1, 2).$$
  \begin{enumerate}
  \item[(a)]
  $Y \subseteq \overline{Y} \subseteq Y_1 \cup Y_2$.
  \item[(b)]
  $Y\not\subseteq Y_i (i = 1, 2)$. If not, $Y \subseteq Y_i$ for some $i$.
  Take closure to get $\overline{Y} \subseteq \overline{Y_i} = Y_i$ (since $Y_i$ is closed),
  contrary to the assumption.
  \end{enumerate}
  By (a)(b), $Y$ is reducible, which is absurd.
\end{enumerate}
$\Box$ \\



%%%%%%%%%%%%%%%%%%%%%%%%%%%%%%%%%%%%%%%%%%%%%%%%%%%%%%%%%%%%%%%%%%%%%%%%%%%%%%%%
%%%%%%%%%%%%%%%%%%%%%%%%%%%%%%%%%%%%%%%%%%%%%%%%%%%%%%%%%%%%%%%%%%%%%%%%%%%%%%%%



\newpage
\section*{Chapter II: Schemes \\}
\addcontentsline{toc}{section}{Chapter II: Schemes}



\subsection*{\S II.1 Sheaves \\}
\addcontentsline{toc}{subsection}{\S II.1 Sheaves}



\subsubsection*{Exercise II.1.1. (Constant presheaf)}
\addcontentsline{toc}{subsubsection}{Exercise II.1.1. (Constant presheaf)}
\emph{Let $A$ be an abelian group, and define the \textbf{constant presheaf}
associated to $A$ on the topological space $X$ to be
the presheaf $U \mapsto A$ for all $U \neq \varnothing$,
with restriction maps the identity.
Show that the constant sheaf $\mathscr{A}$ defined in the text is
the sheaf associated to this presheaf.} \\



\emph{Proof.}
\begin{enumerate}
\item[(1)]
  Let $\mathscr{F}$ be the constant presheaf.

\item[(2)]
  Let $\theta: \mathscr{F} \to \mathscr{A}$ be a morphism
  consists of a morphism of abelian groups $\theta(U): \mathscr{F}(U) = A \to \mathscr{A}(U)$
  for each open set $U \subseteq X$
  such that $\theta(U)(a) = f_a: x \mapsto a$ for each element $x \in U$. (It is well-defined.)

\item[(3)]
  Given any sheaf $\mathscr{G}$ and any morphism $\varphi: \mathscr{F} \to \mathscr{G}$,
  it suffices to find a morphism $\psi: \mathscr{A} \to \mathscr{G}$
  such that $\varphi = \psi \circ \theta$.

\item[(4)]
  Given an open set $U \subseteq X$.
  Suppose $f \in \mathscr{A}(U)$ is a continuous maps of $U$ into $A$.
  Since $A$ is equipped with the discrete topology, $f$ is locally constant, that is,
  \[
    f(V_i) = a_i
  \]
  where each $V_i$ is a connected component of $U$.
  (In particular, $\{V_i\}$ is an open covering of $U$.)

\item[(5)]
  Now
  \[
    s_i := \varphi(V_i)(a_i) \in \mathscr{G}(V_i)
  \]
  is defined.
  Since $\mathscr{G}$ is a sheaf and all $V_i$ are disjoint,
  there is a $s \in \mathscr{G}(U)$ such that $s|_{V_i} = s_i$ for each $i$.
  Now we define $\psi(U)$ by
  \[
    \psi(U)(f) = s.
  \]
  Thus $\psi$ is a morphism and $\varphi = \psi \circ \theta$ by construction.
\end{enumerate}
$\Box$ \\\\



%%%%%%%%%%%%%%%%%%%%%%%%%%%%%%%%%%%%%%%%%%%%%%%%%%%%%%%%%%%%%%%%%%%%%%%%%%%%%%%%
%%%%%%%%%%%%%%%%%%%%%%%%%%%%%%%%%%%%%%%%%%%%%%%%%%%%%%%%%%%%%%%%%%%%%%%%%%%%%%%%
%%%%%%%%%%%%%%%%%%%%%%%%%%%%%%%%%%%%%%%%%%%%%%%%%%%%%%%%%%%%%%%%%%%%%%%%%%%%%%%%
%%%%%%%%%%%%%%%%%%%%%%%%%%%%%%%%%%%%%%%%%%%%%%%%%%%%%%%%%%%%%%%%%%%%%%%%%%%%%%%%



\end{document}