\documentclass{article}
\usepackage{amsfonts}
\usepackage{amsmath}
\usepackage{amssymb}
\usepackage{centernot}
\usepackage{hyperref}
\usepackage[none]{hyphenat}
\usepackage{mathrsfs}
\usepackage{mathtools}
\usepackage{physics}
\usepackage{tikz-cd}
\parindent=0pt



\title{\textbf{Solutions to the book: \\
\emph{Ireland and Rosen, A Classical Introduction to Modern Number Theory, 2nd edition}}}
\author{Meng-Gen Tsai \\ plover@gmail.com}



\begin{document}
\maketitle
\tableofcontents



%%%%%%%%%%%%%%%%%%%%%%%%%%%%%%%%%%%%%%%%%%%%%%%%%%%%%%%%%%%%%%%%%%%%%%%%%%%%%%%%
%%%%%%%%%%%%%%%%%%%%%%%%%%%%%%%%%%%%%%%%%%%%%%%%%%%%%%%%%%%%%%%%%%%%%%%%%%%%%%%%



% Reference:



%%%%%%%%%%%%%%%%%%%%%%%%%%%%%%%%%%%%%%%%%%%%%%%%%%%%%%%%%%%%%%%%%%%%%%%%%%%%%%%%
%%%%%%%%%%%%%%%%%%%%%%%%%%%%%%%%%%%%%%%%%%%%%%%%%%%%%%%%%%%%%%%%%%%%%%%%%%%%%%%%



\newpage
\section*{Chapter 1: Unique Factorization \\}
\addcontentsline{toc}{section}{Chapter 1: Unique Factorization}



\subsubsection*{Exercise 1.10.}
\addcontentsline{toc}{subsubsection}{Exercise 1.10.}
\emph{Suppose that $(u,v)=1$.
Show that $(u+v,u-v)$ is either $1$ or $2$.} \\

Each case is possible:
\begin{enumerate}
  \item[(1)]
  $u=3, v=2$. $(u,v)=1$ and $(u+v,u-v)=1$.
  \item[(2)]
  $u=3, v=1$. $(u,v)=1$ and $(u+v,u-v)=2$. \\
\end{enumerate}

\emph{Proof (Exercise 1.6).}
Since $(u,v)=1$, there is $m, n \in \mathbb{Z}$ such that $mu+nv=1$ (Exercise 1.4).
So
\begin{align*}
mu+nv=1
\Longleftrightarrow& 2mu+2nv=2 \\
\Longleftrightarrow& ((u+v) + (u-v))m + ((u+v)-(u-v))n = 2 \\
\Longleftrightarrow& (m+n)(u+v) + (m-n)(u-v) = 2,
\end{align*}
or $(x,y)=(m+n,m-n)$ is an integer solution to $(u+v)x + (u-v)y = 2$.
So $2 \mid (u+v, u-v)$ (Exercise 1.6).
Hence $(u+v, u-v) = 1$ or $2$.
$\Box$ \\\\



%%%%%%%%%%%%%%%%%%%%%%%%%%%%%%%%%%%%%%%%%%%%%%%%%%%%%%%%%%%%%%%%%%%%%%%%%%%%%%%%



\subsubsection*{Exercise 1.11.}
\addcontentsline{toc}{subsubsection}{Exercise 1.11.}
\emph{Show that $(a,a+k) \mid k$.} \\

\emph{Proof (Exercise 1.6).}
The equation $ax + (a+k)y = k$ has solution $(x,y) = (-1,1) \in \mathbb{Z}^2$.
Hence $(a,a+k) \mid k$ (Exercise 1.6).
$\Box$ \\\\



%%%%%%%%%%%%%%%%%%%%%%%%%%%%%%%%%%%%%%%%%%%%%%%%%%%%%%%%%%%%%%%%%%%%%%%%%%%%%%%%



\subsubsection*{Exercise 1.31.}
\addcontentsline{toc}{subsubsection}{Exercise 1.31.}
\emph{Show that $2$ is divided by $(1+i)^2 \in \mathbb{Z}[i]$.} \\

$1+i$ is irreducible in $\mathbb{Z}[i]$. \\

The ring morphism $\mathbb{Z} \rightarrow \mathbb{Z}[i]$
corresponds to a map of schemes
$f: \text{Spec}(\mathbb{Z}[i]) \rightarrow \text{Spec}(\mathbb{Z})$.
Suppose $(p)$ is a prime ideal of $\mathbb{Z}$.
Might find the points of $f^{-1}(p) \in \text{Spec}(\mathbb{Z}[i])$. \\

\emph{Proof.}
$(1+i)^2 = 2i \in \mathbb{Z}[i]$.
Thus $2 \mid (1+i)^2 \in \mathbb{Z}[i]$.
$\Box$ \\\\



%%%%%%%%%%%%%%%%%%%%%%%%%%%%%%%%%%%%%%%%%%%%%%%%%%%%%%%%%%%%%%%%%%%%%%%%%%%%%%%%



\subsubsection*{Exercise 1.34.}
\addcontentsline{toc}{subsubsection}{Exercise 1.34.}
\emph{Show that $3$ is divided by $(1-\omega)^2 \in \mathbb{Z}[\omega]$.} \\

\emph{Proof.}
$(1 - \omega)^2
= 1 - 2\omega + \omega^2
= (1 + \omega + \omega^2) - 3 \omega
= - 3 \omega \in \mathbb{Z}[\omega]$.
Thus $3 \mid (1 - \omega)^2 \in \mathbb{Z}[\omega]$.
$\Box$ \\\\



%%%%%%%%%%%%%%%%%%%%%%%%%%%%%%%%%%%%%%%%%%%%%%%%%%%%%%%%%%%%%%%%%%%%%%%%%%%%%%%%
%%%%%%%%%%%%%%%%%%%%%%%%%%%%%%%%%%%%%%%%%%%%%%%%%%%%%%%%%%%%%%%%%%%%%%%%%%%%%%%%



\newpage
\section*{Chapter 2: Applications of Unique Factorization \\}
\addcontentsline{toc}{section}{Chapter 2: Applications of Unique Factorization}



\subsubsection*{Exercise.}
\addcontentsline{toc}{subsubsection}{Exercise.}
\emph{If $\frac{a}{b} \in \mathbb{Z}_{p}$ is not a unit, prove that
$\frac{a}{b} + 1$ is a unit.} \\

\emph{Proof.}
$\frac{a}{b} \in \mathbb{Z}_{p}$ is not a unit iff $p \mid a$ and $p \nmid b$.
Thus $p \nmid (a +b)$. That is, $\frac{a}{b} + 1 = \frac{a + b}{b} \in \mathbb{Z}_{p}$ is a unit.
$\Box$ \\\\



%%%%%%%%%%%%%%%%%%%%%%%%%%%%%%%%%%%%%%%%%%%%%%%%%%%%%%%%%%%%%%%%%%%%%%%%%%%%%%%%



\subsubsection*{Exercise 2.6. ($p$-adic valuation)}
\addcontentsline{toc}{subsubsection}{Exercise 2.6. ($p$-adic valuation)}
\emph{For a rational number $r$ let $[r]$ be the largest integer less than or equal to $r$,
e.g., $[\frac{1}{2}] = 0$, $[2] = 2$, $[3 \frac{1}{3}] = 3$. Prove
$$\text{ord}_p n!
= \left[\frac{n}{p}\right] + \left[\frac{n}{p^2}\right] + \left[\frac{n}{p^3}\right] + \cdots.$$
}

Notice that $[\frac{n}{p}] + [\frac{n}{p^2}] + [\frac{n}{p^3}] + \cdots$ is a finite sum. \\

\emph{Proof.}
For any $k = 1, 2, ..., n$, we can express $k$ as $k = p^s t$
where $s = \text{ord}_p k$ is a non-negative integer and $(t, p) = 1$.
There are $[\frac{n}{p^a}]$ numbers such that $p^a \mid k$ for $a = 1, 2, ...$.
Therefore, there are $$\left[\frac{n}{p^a}\right] - \left[\frac{n}{p^{a+1}}\right]$$
numbers such that $\text{ord}_p k = a$ for $a = 1, 2, ...$. Hence,
\begin{align*}
\text{ord}_p n!
&= \left( \left[\frac{n}{p}\right] - \left[\frac{n}{p^2}\right] \right)
 + 2 \left( \left[\frac{n}{p^2}\right] - \left[\frac{n}{p^3}\right] \right)
 + 3 \left( \left[\frac{n}{p^3}\right] - \left[\frac{n}{p^4}\right] \right) + \cdots \\
&= \left[\frac{n}{p}\right] + \left[\frac{n}{p^2}\right] + \left[\frac{n}{p^3}\right] + \cdots.
\end{align*}
$\Box$ \\



\subsubsection*{Supplement 2.6.1.}
\addcontentsline{toc}{subsubsection}{Supplement 2.6.1.}
Related problems.
\begin{enumerate}
\item[(1)]
\emph{Prove that
$$\frac{(m + n)!}{m!n!}$$
is an integer for all non-negative integers $m$ and $n$.} \\

\emph{Proof.}
It is sufficient to show that
$$\text{ord}_p (m + n)! \geq \text{ord}_p m! + \text{ord}_p n!$$
for any prime $p$, or show that
$$\left[\frac{m + n}{p^k}\right]
\geq \left[\frac{m}{p^k}\right] + \left[\frac{n}{p^k}\right]$$
for any prime $p$ and $k \in \mathbb{Z}^+$ by Exercise 4.6, or show that
$$[x + y] \geq [x] + [y]$$
for any rational (or real) numbers $x$ and $y$.
It is trivial by considering that the sum of two fractional parts $\{x\} = x - [x]$
might be greater than or equal to $1$, so $[x + y] = [x] + [y]$ or $[x] + [y] + 1$.
$\Box$ \\

\emph{Note.}
$\frac{(m + n)!}{m!n!}$ is a binomial coefficient.
Similarly, a multinomial coefficient is
$$\frac{(n_1 + n_2 + \cdots + n_k)!}{n_1!n_2! \cdots n_k!}.$$
We can show that the multinomial coefficient is an integer
by using the above argument. \\

\item[(2)]
\emph{Prove that
$$\frac{(2m)!(2n)!}{m!n!(m + n)!}$$
is an integer for all non-negative integers $m$ and $n$.} \\

\emph{Proof.}
Similar to (1), it is sufficient to show that
$$[2x] + [2y] \geq [x] + [y] + [x + y]$$
for any rational (or real) numbers $x$ and $y$.
Notice that $[2x] = [x] + [x + \frac{1}{2}]$, and thus we might show that
$[x + \frac{1}{2}] + [y + \frac{1}{2}] \geq [x + y]$.
Again it is trivial and we omit the tedious calculation.
$\Box$ \\

\item[(3)]
\emph{Hermite's identity:
$[nx] = \sum_{k=0}^{n-1} [x + \frac{k}{n}]$ for $n \in \mathbb{Z}^+$.} \\

Let $n = 2$ and we can get $[2x] = [x] + [x + \frac{1}{2}]$ too. \\

\emph{Proof.}
Consider the function $f(x) = \sum_{k=0}^{n-1} [x + \frac{k}{n}] - [nx]$.
Notice that $f(x + \frac{1}{n}) = f(x)$. $f$ has period $\frac{1}{n}$.
It then suffices to prove that $f(x) = 0$ on $[0, \frac{1}{n})$.
But in this case, the integral part of each summand in $f$ is equal to $0$.
Therefore $f = 0$ on $\mathbb{R}$.
$\Box$ \\

\item[(4)]
\emph{Show
$$\frac{(5m)!(5n)!}{m! n! (3m + n)! (3n + m)!}$$
is an integer for all non-negative integers $m$ and $n$.} \\

Try to deduce the inequality $[5x] + [5y] \geq [x] + [y] + [3x + y] + [3y + x]$. \\\\
\end{enumerate}



%%%%%%%%%%%%%%%%%%%%%%%%%%%%%%%%%%%%%%%%%%%%%%%%%%%%%%%%%%%%%%%%%%%%%%%%%%%%%%%%



\subsubsection*{Exercise 2.7.}
\addcontentsline{toc}{subsubsection}{Exercise 2.7.}
\emph{Deduce from Exercise 2.6 that $\text{ord}_p n! \leq \frac{n}{p - 1}$ and that
$n!^{\frac{1}{n}} \leq \prod_{p|n!}p^{\frac{1}{p - 1}}$.} \\

\emph{Proof.}
\begin{align*}
\text{ord}_p n!
&= \left[\frac{n}{p}\right] + \left[\frac{n}{p^2}\right] + \left[\frac{n}{p^3}\right] + \cdots \\
&\leq \frac{n}{p} + \frac{n}{p^2} + \frac{n}{p^3} + \cdots \\
&= \frac{\frac{n}{p}}{1 - \frac{1}{p}} \\
&= \frac{n}{p - 1}.
\end{align*}
Thus,
$$n!
= \prod_{p|n!} p^{\text{ord}_p n!}
\leq \prod_{p|n!} p^{\frac{n}{p - 1}}
= \left( \prod_{p|n!} p^{\frac{1}{p - 1}} \right)^n, $$
or
$$n!^{\frac{1}{n}} \leq \prod_{p|n!}p^{\frac{1}{p - 1}}.$$
$\Box$ \\\\



%%%%%%%%%%%%%%%%%%%%%%%%%%%%%%%%%%%%%%%%%%%%%%%%%%%%%%%%%%%%%%%%%%%%%%%%%%%%%%%%



\subsubsection*{Exercise 2.8.}
\addcontentsline{toc}{subsubsection}{Exercise 2.8.}
\emph{Use Exercise 2.7 to show that there are infinitely many primes.
[Hint: $(n!)^2 \geq n^n$.]
(This proof is due to Eckford Cohen.)} \\


\emph{Proof.}
\begin{enumerate}
\item[(1)]
  \emph{Show that $(n!)^2 \geq n^n$.}
  Write
  $(n!)^2 = \prod_{k=1}^n k \prod_{k=1}^n (n + 1 - k) = \prod_{k=1}^n k(n + 1 - k)$,
  and $n^n = \prod_{k=1}^n n$.
  It suffices to show that $k(n + 1 - k) \geq n$ for each $1 \leq k \leq n$.
  Notice that $k(n + 1 - k) - n = (n - k)(k - 1) \geq 0$ for $1 \leq k \leq n$.
  The inequality holds.

\item[2]
  By Exercise 2.7 and (1),
  \[
    \prod_{p|n!} p^{\frac{1}{p - 1}} \geq (n!)^{\frac{1}{n}} \geq \sqrt{n}.
  \]
  Assume that there are finitely many primes,
  the value $\prod_{p|n!} p^{\frac{1}{p - 1}}$ is a finite number
  whenever the value of $n$.
  However, $\sqrt{n} \rightarrow \infty$ as $n \rightarrow \infty$,
  which leads to a contradiction.
  Hence there are infinitely many primes.
\end{enumerate}
$\Box$ \\\\



%%%%%%%%%%%%%%%%%%%%%%%%%%%%%%%%%%%%%%%%%%%%%%%%%%%%%%%%%%%%%%%%%%%%%%%%%%%%%%%%



\subsubsection*{Exercise 2.27.}
\addcontentsline{toc}{subsubsection}{Exercise 2.27.}
\emph{Show that ${\sum}' \frac{1}{n}$, the sum being over square free integers, diverges.
Conclude that $\prod_{p \leq N} ( 1 + \frac{1}{p} ) \rightarrow \infty$ as $N \rightarrow \infty$.
Since $e^x > 1 + x$, conclude that $\sum_{p \leq N} \frac{1}{p} \rightarrow \infty$.
(This proof is due to I. Niven.)} \\

There are many proofs of $\sum_{p} \frac{1}{p}$ diverges. \\



\emph{Proof.}
\begin{enumerate}
\item[(1)]
For any positive integers $n$, we can write $n = a^2 b$ where $a \in \mathbb{Z}^+$ and
$b$ is a square free integer.
Given $N$,
$$\sum_{n \leq N} \frac{1}{n}
\leq \left(\sum_{a = 1}^{\infty} \frac{1}{a^2} \right)
\left( {\sum_{b \leq N}}' \frac{1}{b} \right).$$
Notices that $\sum_{a = 1}^{\infty} \frac{1}{a^2}$ converges.
Since $\sum_{n \leq N} \frac{1}{n} \rightarrow \infty$ as $N \rightarrow \infty$,
$\sum'_{b \leq N}\frac{1}{b} \rightarrow \infty$ as $N \rightarrow \infty$.
\item[(2)]
By the unique factorization theorem on $n \leq N$,
$$\prod_{p \leq N} \left( 1 + \frac{1}{p} \right)
\geq {\sum_{n \leq N}}' \frac{1}{n}.$$
Since ${\sum_{n \leq N}}' \frac{1}{n} \rightarrow \infty$ as $N \rightarrow \infty$,
$\prod_{p \leq N} ( 1 + \frac{1}{p} ) \rightarrow \infty$ as $N \rightarrow \infty$.
\item[(3)]
By applying the inequality $e^x > 1 + x$ on any prime $p$,
$$\exp\left(\frac{1}{p}\right) > 1 + \frac{1}{p}.$$
Now multiplying the inequality over all primes $p \leq N$ and noticing that
$\exp(x) \cdot \exp(y) = \exp(x + y)$, we have
$$\exp\left(\sum_{p \leq N} \frac{1}{p} \right)
> \prod_{p \leq N} \left( 1 + \frac{1}{p} \right).$$
So
$\exp\left(\sum_{p \leq N} \frac{1}{p} \right) \rightarrow \infty$ as $N \rightarrow \infty$, or
$\sum_{p \leq N} \frac{1}{p} \rightarrow \infty$ as $N \rightarrow \infty$.
$\Box$ \\\\
\end{enumerate}



%%%%%%%%%%%%%%%%%%%%%%%%%%%%%%%%%%%%%%%%%%%%%%%%%%%%%%%%%%%%%%%%%%%%%%%%%%%%%%%%
%%%%%%%%%%%%%%%%%%%%%%%%%%%%%%%%%%%%%%%%%%%%%%%%%%%%%%%%%%%%%%%%%%%%%%%%%%%%%%%%



\newpage
\section*{Chapter 3: Congruence \\}
\addcontentsline{toc}{section}{Chapter 3: Congruence}



\subsubsection*{Exercise 3.12.}
\addcontentsline{toc}{subsubsection}{Exercise 3.12.}
\emph{Let $${p \choose k} = \frac{p!}{k!(p-k)!}$$
be a binomial coefficient, and suppose that $p$ is a prime.
If $1 \leq k \leq p-1$, show that $p$ divides ${p \choose k}$.
Deduce $(a+1)^p \equiv a^p + 1 \pmod{p}$.} \\

\emph{Proof.}
\begin{enumerate}
  \item[(1)]
  If $1 \leq k \leq p-1$, then $p \nmid k!$ and $p \nmid (p-k)!$
  since $p$ is a prime number.
  \item[(2)]
  Write $a = \frac{p!}{k!(p-k)!} \in \mathbb{Z}$.
  \begin{align*}
  a = \frac{p!}{k!(p-k)!}
  &\Longleftrightarrow
  p! = ak!(p-k)! \\
  &\Longrightarrow
  p \mid p! \text{ or } p \mid ak!(p-k)! \\
  &\Longrightarrow
  p \mid a
    &((1))
  \end{align*}
  Hence $p$ divides ${p \choose k}$ if $1 \leq k \leq p-1$.
  \item[(3)]
  \begin{align*}
  (a+1)^p
  &\equiv \sum_{k=0}^{p} {p \choose k} a^k \\
  &\equiv 1 + \left( \sum_{k=1}^{p-1} {p \choose k} a^k \right) + a^p \\
  &\equiv 1 + a^p \\
  &\equiv a^p + 1 \pmod{p}.
  \end{align*}
\end{enumerate}
$\Box$\\\\



%%%%%%%%%%%%%%%%%%%%%%%%%%%%%%%%%%%%%%%%%%%%%%%%%%%%%%%%%%%%%%%%%%%%%%%%%%%%%%%%
%%%%%%%%%%%%%%%%%%%%%%%%%%%%%%%%%%%%%%%%%%%%%%%%%%%%%%%%%%%%%%%%%%%%%%%%%%%%%%%%



\newpage
\section*{Chapter 4: The Structure of $U(\mathbb{Z}/n\mathbb{Z})$ \\}
\addcontentsline{toc}{section}{Chapter 4: The Structure of $U(\mathbb{Z}/n\mathbb{Z})$}



\subsubsection*{Theorem 1.}
\addcontentsline{toc}{subsubsection}{Theorem 1.}
\emph{$U(\mathbb{Z}/p\mathbb{Z})$ is a cyclic group.} \\



\emph{Proof.}
Let $p - 1 = q_1^{e_1} q_2^{e_2} \cdots q_t^{e^t} = \prod_{q} q^e$ be the prime
decomposition of $p - 1$. Consider the congruences

\begin{enumerate}
\item[(1)]
$x^{q^{e-1}} \equiv 1 (p)$
\item[(2)]
$x^{q^{e}} \equiv 1 (p)$
\end{enumerate}

Therefore,

\begin{enumerate}
\item[(1)]
Every solution to $x^{q^{e-1}} \equiv 1 \: (p)$ is a solution of $x^{q^{e}} \equiv 1 \: (p)$.
\item[(2)]
$x^{q^{e}} \equiv 1 \: (p)$ has more solutions than $x^{q^{e-1}} \equiv 1 \: (p)$.
In fact, $x^{q^{e-1}} \equiv 1 \: (p)$ has $q^{e-1}$ solutions and $x^{q^{e}} \equiv 1 \: (p)$
has $q^{e}$ solutions by Proposition 4.1.2.
\end{enumerate}

Therefore, there exists $g_i \in \mathbb{Z}/p\mathbb{Z}$ generating a subgroup of
$U(\mathbb{Z}/p\mathbb{Z})$ of order $q_i^{e_i}$ for all $i = 1, ..., t$.
Pick $g = g_1 g_2 \cdots g_t \in \mathbb{Z}/p\mathbb{Z}$ generating a subgroup of
$U(\mathbb{Z}/p\mathbb{Z})$ of order $q_1^{e_1} q_2^{e_2} \cdots q_t^{e^t} = p - 1$.
That is, $\left \langle g \right \rangle = U(\mathbb{Z}/p\mathbb{Z})$.
$\Box$ \\\\



%%%%%%%%%%%%%%%%%%%%%%%%%%%%%%%%%%%%%%%%%%%%%%%%%%%%%%%%%%%%%%%%%%%%%%%%%%%%%%%%



\subsubsection*{Exercise 4.1.}
\addcontentsline{toc}{subsubsection}{Exercise 4.1.}
\emph{Show that $2$ is a primitive root module $29$.}\\

\emph{Proof.}
$2^1 \equiv 2 \: (29)$,
$2^2 \equiv 4 \: (29)$,
$2^3 \equiv 8 \: (29)$,
$2^4 \equiv 16 \: (29)$,
$2^5 \equiv 3 \: (29)$,
$2^6 \equiv 6 \: (29)$,
$2^7 \equiv 12 \: (29)$,
$2^8 \equiv 24 \: (29)$,
$2^9 \equiv 19 \: (29)$,
$2^{10} \equiv 9 \: (29)$,
$2^{11} \equiv 18 \: (29)$,
$2^{12} \equiv 7 \: (29)$,
$2^{13} \equiv 14 \: (29)$,
$2^{14} \equiv 28 \: (29)$,
$2^{15} \equiv 27 \: (29)$,
$2^{16} \equiv 25 \: (29)$,
$2^{17} \equiv 21 \: (29)$,
$2^{18} \equiv 13 \: (29)$,
$2^{19} \equiv 26 \: (29)$,
$2^{20} \equiv 23 \: (29)$,
$2^{21} \equiv 17 \: (29)$,
$2^{22} \equiv 5 \: (29)$,
$2^{23} \equiv 10 \: (29)$,
$2^{24} \equiv 20 \: (29)$,
$2^{25} \equiv 11 \: (29)$,
$2^{26} \equiv 22 \: (29)$,
$2^{27} \equiv 15 \: (29)$,
$2^{28} \equiv 1 \: (29)$. Thus
$U(\mathbb{Z}/29\mathbb{Z}) = \left \langle 2 \right \rangle$.
$\Box$ \\

\emph{Proof (A shorter version).}
$2^{28} \equiv 1 \: (29)$.
It suffices to show that
$2^{14} \not\equiv 1 \: (29)$ and $2^{4} \not\equiv 1 \: (29)$.
Actually, $2^{14} \equiv 28 \: (29)$ and $2^{4} \equiv 16 \: (29)$.
$\Box$ \\\\



%%%%%%%%%%%%%%%%%%%%%%%%%%%%%%%%%%%%%%%%%%%%%%%%%%%%%%%%%%%%%%%%%%%%%%%%%%%%%%%%



\subsubsection*{Exercise 4.11.}
\addcontentsline{toc}{subsubsection}{Exercise 4.11.}
\emph{Prove that $1^k + 2^k + \cdots + (p-1)^k \equiv 0 \: (p)$
if $p - 1 \nmid k$ and $-1 (p)$ if $p - 1 \mid k$.} \\

\emph{Proof.}
Write $\left \langle g \right \rangle = U(\mathbb{Z}/p\mathbb{Z})$, and
$S = 1^k + 2^k + \cdots + (p-1)^k \equiv g^k + (g^k)^2 + \cdots + (g^k)^{p - 1} \: (p)$. \\

If $p - 1 \mid k$, $g^k \equiv 1 \: (p)$. Thus
$S \equiv 1 + 1 + \cdots + 1 = p - 1 \equiv -1 \: (p)$. \\

If $p - 1 \nmid k$, $g^k$ is also a generator of $U(\mathbb{Z}/p\mathbb{Z})$ by Exercise 13.
There are three proofs of this case.
\begin{enumerate}
\item[(1)]
$S$ is the sum of a geometric series.
So $(1 - g^k) S = g^k (1 - (g^k)^{p - 1}) = g^k (1 - (g^{p - 1})^k) \equiv 0 \: (p)$.
Since $g^k \not\equiv 1 \: (p)$, $S \equiv 0 \: (p)$.
\item[(2)]
$\left \langle g^k \right \rangle = U(\mathbb{Z}/p\mathbb{Z})$. So
$S \equiv g^k + (g^k)^2 + \cdots + (g^k)^{p - 1} \equiv 1 + 2 + \cdots + (p - 1)
\equiv \frac{p(p - 1)}{2} \equiv 0 \: (p)$ since $p$ is odd and
thus $\frac{p - 1}{2}$ is an integer.
(If $p = 2$ is even, then there does not exist any $k$ such that $p - 1 \nmid k$.)
\item[(3)]
Similar to (2), write $S \equiv 1 + 2 + \cdots + (p - 1) \: (p)$. Notice that the equation
$x^{p - 1} - 1 \equiv (x - 1)(x - 2) \cdots (x - (p - 1)) \: (p)$ holds by Proposition 4.1.1.
So $S \equiv 0 \: (p)$ by comparing the coefficient of $x^{p - 2}$ on the both sides if $p > 2$.
(Again $p = 2$ is impossible in this case.)
\end{enumerate}
$\Box$ \\\\



%%%%%%%%%%%%%%%%%%%%%%%%%%%%%%%%%%%%%%%%%%%%%%%%%%%%%%%%%%%%%%%%%%%%%%%%%%%%%%%%



\subsubsection*{Exercise 4.12.}
\addcontentsline{toc}{subsubsection}{Exercise 4.12.}
\emph{Use the existence of a primitive root to give another proof
of Wilson's theorem $(p - 1)! \equiv -1 \: (p)$.} \\

\emph{Proof.}
Say $p > 2$. ($p = 2$ is trivial.)
Let $g$ be a primitive root of $U(\mathbb{Z}/p\mathbb{Z})$.
So $(p - 1)! \equiv g \cdot g^2 \cdots g^{p - 1} \equiv g^{\frac{p(p - 1)}{2}} \: (p)$. \\

The equation $x^2 \equiv 1 \: (p)$ has exactly 2 solutions $x \equiv 1, -1 \: (p)$
by Proposition 4.1.2.
Notice that $x \equiv g^{\frac{p - 1}{2}} \: (p)$ is a solution of the equation
$x^2 \equiv 1 \: (p)$ and $g^{\frac{p - 1}{2}} \not\equiv 1 \: (p)$
since $g$ is a primitive root of $U(\mathbb{Z}/p\mathbb{Z})$.
Therefore, $$g^{\frac{p - 1}{2}} \equiv -1 \: (p).$$
So $(p - 1)! \equiv g^{\frac{p(p - 1)}{2}} \equiv (-1)^p \equiv -1 \: (p)$
since $p$ is an odd prime.
$\Box$ \\



\subsubsection*{Supplement 4.12.1.}
\addcontentsline{toc}{subsubsection}{Supplement 4.12.1.}
There are many proofs of Wilson's theorem.
\begin{enumerate}
\item[(1)]
Exercise 3.9. Use a reduced residue system modulo $p$.
\item[(2)]
Corollary of Proposition 4.1.1. $x^{p - 1} - 1 \equiv (x - 1)(x - 2) \cdots (x - p + 1) \: (p)$.
\item[(3)]
Exercise 4.12. Use the existence of a primitive root.
\item[(4)]
Inclusion-exclusion principle
(\href{http://campus.lakeforest.edu/trevino/WilsonCapsule.pdf}
{Enrique Treviño, An Inclusion-Exclusion Proof of Wilson's Theorem}). \\
\textbf{Lemma.}
$$n! = \sum_{k = 0}^{n}(-1)^k {n \choose k} (n - k)^{n}.$$

\emph{Proof of lemma.}
Consider the number of permutations on $S = \{1, 2, ..., n\}$.
On the one hand, the number is $n!$.
On the other hand, we can think of a permutation on $S$ as a function
$f: S \rightarrow S$ that is onto.
The number of functions $g: S \rightarrow S$ is $n^n$.
To find the onto functions, we have to remove whichever ones are not onto.
Therefore, we must remove those that miss at least $1$ value.
There are ${n \choose 1}$ ways of choosing the missed value and ${(n - 1)}^n$
functions missing that particular value.
But when we remove all of these functions, we took out some too many times, indeed,
any function that misses at least $2$ values was over counted. So we have to add it back in.
We get ${n \choose 2} {(n - 2)}^n$ such functions. Continue this process.
$\Box$ \\

\emph{Proof.}
Now we use the equation $n! = \sum_{k = 0}^{n}(-1)^k {n \choose k} (n - k)^{n}$
by substituting $n = p - 1$ and then get
$$(p - 1)! = \sum_{k = 0}^{p - 1}(-1)^k {p - 1 \choose k} (p - 1 - k)^{p - 1}.$$
Now look at the $k$-term in the summation. \\

$k!(p - 1 - k)! \equiv (-1)^k (p - k)(p - (k - 1)) \cdots (p - 1) \cdot (p - 1 - k)!
\equiv (-1)^k (p - 1)! \: (p)$.
So ${p - 1 \choose k} = \frac{(p - 1)!}{k!(p - 1 - k)!} \equiv (-1)^k \: (p)$.
Also, ${(p - 1 - k)}^{p - 1} \equiv {(-1 - k)}^{p - 1} \equiv {(1 + k)}^{p - 1} \: (p)$
since $(-1)^{p - 1} = 1$ if $p > 2$. ($p = 2$ is trivial.) Therefore,
$$(p - 1)!
\equiv \sum_{k = 0}^{p - 1}(-1)^k \cdot (-1)^k \cdot {(1 + k)}^{p - 1}
\equiv \sum_{k = 1}^{p - 1} k^{p - 1} \: (p).$$
(We adjust the index of the summation and notice that $p^{p - 1} \equiv 0 \: (p)$).
By Fermat’s Little Theorem, $k^{p - 1} \equiv 1 \: (p)$.
Therefore, the right-hand sum consists of $(p - 1)$ ones and the proof is completed.
$\Box$ \\

The original proof in the paper is not very beautiful.
We don't need to use the inclusion-exclusion expression of $p!$
and then cancel out $p$ on the both sides. Please use $(p - 1)!$ directly.
\item[(5)]
One combinatorial proof
(\href{https://www.youtube.com/watch?v=4qbh7mC6YCY}
{Cheenta, Wilson's Theorem and It's Geometric proof}). \\
\emph{Proof.}
Consider a circumference with $p$ points that correspond to the vertices of a regular $p$-gon.
There are $\frac{(p - 1)!}{2}$ (non-regular or regular) polygons
that we form by joining these vertices. \\

Now among $\frac{(p - 1)!}{2}$ of them, we have $\frac{p - 1}{2}$ unaltered
when rotated by $\frac{2 \pi}{p}$ radian.
That is, there are $\frac{p - 1}{2}$ regular polygons due to the rotational symmetry. \\

Therefore, there are $\frac{(p - 1)!}{2} - \frac{p - 1}{2}$ non-regular polygons.
Notices that the number of non-regular polygons is divided by $p$ since $p$ is a prime. \\

So $\frac{(p - 1)!}{2} - \frac{p - 1}{2} \equiv 0 \: (p)$.
Hence, $(p - 1)! \equiv p - 1 \equiv -1 \: (p)$ if $p > 2$. ($p = 2$ is trivial.)
$\Box$ \\

\end{enumerate}



\subsubsection*{Supplement 4.12.2.}
\addcontentsline{toc}{subsubsection}{Supplement 4.12.2.}
Related problems.
\begin{enumerate}
\item[(1)]
(\href{https://projecteuler.net/problem=381}
{ProjectEuler 381: (prime-k) factorial}).
\emph{Let $S(p) = \sum_{1 \leq k \leq 5}(p-k)! \: (p)$ for a prime $p$.
Find $\sum_{1 \leq p \leq {10}^8} S(p)$} (by using computer programs).
\item[(2)]
\emph{Let $g$ be a primitive root modulo the odd prime $p$.
Prove that $g^{\frac{p - 1}{2}} \equiv -1 \: (p)$.
Deduce that if $g, h$ are primitive roots modulo the odd prime $p$
then $g \cdot h$ cannot be a primitive root.} \\\\
\end{enumerate}



%%%%%%%%%%%%%%%%%%%%%%%%%%%%%%%%%%%%%%%%%%%%%%%%%%%%%%%%%%%%%%%%%%%%%%%%%%%%%%%%



\subsubsection*{Exercise 4.13. (Generators of a cyclic group)}
\addcontentsline{toc}{subsubsection}{Exercise 4.13. (Generators of a cyclic group)}
\emph{Let $G$ be a finite cyclic group and $g \in G$ is a generator.
Show that all the other generators are of the form $g^k$,
where $(k, n) = 1$, $n$ being the order of $G$.} \\

\emph{Proof.}
Suppose that $h = g^k$ with $(k, n) = 1$.
Then clearly $\left \langle h \right \rangle \subseteq \left \langle g \right \rangle$
as a subset. For the reverse containment ($\supseteq$),
write $rk + sn = 1$ where $r, s \in \mathbb{Z}$. Then
$h^r = g^{kr} = g^{1 - sn} = g \cdot (g^n)^{-s} = g \cdot 1 = g$. Then again
$\left \langle g \right \rangle \subseteq \left \langle h \right \rangle$ as a subset. \\

Now suppose that $\left \langle g \right \rangle = \left \langle h \right \rangle$.
Then $h = g^k$ for some $k \in \mathbb{Z}$. Also, $g = h^r$ for some $r \in \mathbb{Z}$.
So $g = h^r = g^{kr}$ or $g^{kr - 1} = 1$. So $n | (kr - 1)$, or
$ar + ns = 1$ for some $s \in \mathbb{Z}$, that is, $(a, n) = 1$.
$\Box$ \\

Reference:
\href{http://ramanujan.math.trinity.edu/rdaileda/teach/s18/m3341/ZnZ.pdf}
{R. C. Daileda, The Structure of $U(\mathbb{Z}/n\mathbb{Z})$.} \\

\textbf{Corollary.}
\emph{Let $G$ be a finite cyclic group of order $n$.
Then $G$ has exactly $\phi(n)$ generators.} \\

\textbf{Corollary.}
\emph{$U(\mathbb{Z}/p\mathbb{Z})$ has exactly $\phi(p - 1)$ generators.
$U(\mathbb{Z}/p^l\mathbb{Z})$ has exactly $\phi(p^{l-1}(p - 1))$ generators if $p$ is odd.} \\\\



%%%%%%%%%%%%%%%%%%%%%%%%%%%%%%%%%%%%%%%%%%%%%%%%%%%%%%%%%%%%%%%%%%%%%%%%%%%%%%%%



\subsubsection*{Exercise 4.22.}
\addcontentsline{toc}{subsubsection}{Exercise 4.22.}
\emph{If $a$ has order $3$ modulo $p$, show that $1 + a$ has order $6$.} \\

\emph{Proof.}
Since $a$ has order $3$, $0 \equiv a^3 - 1 \equiv (a - 1)(a^2 + a + 1) \: (p)$.
Since $p$ is a prime, $a - 1 \equiv 0 \: (p)$ or $a^2 + a + 1 \equiv 0 \: (p)$.
$a$ cannot be $1$ since $a$ has order $3 \neq 1$.
Therefore,
$$a^2 + a + 1 \equiv 0 \: (p),$$
or
$1 + a \equiv -a^2 \equiv -a^{-1} \: (p)$.
So
\begin{align*}
(1 + a)^6 &\equiv (-a^{-1})^6 \equiv 1 \: (p), \\
1 + a     &\not\equiv 1 \: (p), \\
(1 + a)^2 &\equiv a \not\equiv 1 \: (p), \\
(1 + a)^3 &\equiv -1 \not\equiv 1 \: (p).
\end{align*}
Hence $1 + a$ has order $6$.
$\Box$ \\\\



%%%%%%%%%%%%%%%%%%%%%%%%%%%%%%%%%%%%%%%%%%%%%%%%%%%%%%%%%%%%%%%%%%%%%%%%%%%%%%%%
%%%%%%%%%%%%%%%%%%%%%%%%%%%%%%%%%%%%%%%%%%%%%%%%%%%%%%%%%%%%%%%%%%%%%%%%%%%%%%%%



\newpage
\section*{Chapter 5: Quadratic Reciprocity \\}
\addcontentsline{toc}{section}{Chapter 5: Quadratic Reciprocity}



% Reference:
% https://github.com/xyzzyz/ireland-rosen/blob/master/rosen.pdf



\subsubsection*{Exercise 5.2.}
\addcontentsline{toc}{subsubsection}{Exercise 5.2.}
\emph{Show that the number of solutions to $x^2 \equiv a \: (p)$
is given by $1 + (a/p)$.} \\

$p$ is an odd prime. \\

\emph{Proof.}
\begin{enumerate}
\item[(1)]
If $x \equiv t \: (p)$ is a solution of the equation $x^2 \equiv a \: (p)$,
then $x \equiv -t \: (p)$ is also a solution.
Notice that $t \not\equiv -t \: (p)$ if $t \not\equiv 0 \: (p)$
by using the fact that $p$ is odd.
\item[(2)]
(Lemma 4.1.) Let $f(x) \in k[x]$, $k$ a field. Suppose that $\deg f(x) = n$.
Then $f$ has at most $n$ distinct roots.
\item[(3)]
If $a = 0$, then $x^2 \equiv 0 \: (p)$ has only one solution $x \equiv 0 \: (p)$,
or $1 + (a/p)$ solution (where $(a/p) = 0$ in this case).
\item[(4)]
If $a \neq 0$ is a quadratic residue mod $p$, then by (1)(2)
the equation $x^2 \equiv a \: (p)$ has exactly 2 solutions, or $1 + (a/p)$ solutions
(where $(a/p) = 1$ in this case).
\item[(5)]
If $a$ is not a quadratic residue mod $p$,
then there is no solutions of the equation $x^2 \equiv a \: (p)$,
or $1 + (a/p)$ solutions (where $(a/p) = -1$ in this case).
\end{enumerate}
By (3)(4)(5), in any case the number of solutions to $x^2 \equiv a \: (p)$
is given by $1 + (a/p)$.
$\Box$ \\\\



%%%%%%%%%%%%%%%%%%%%%%%%%%%%%%%%%%%%%%%%%%%%%%%%%%%%%%%%%%%%%%%%%%%%%%%%%%%%%%%%



\subsubsection*{Exercise 5.4.}
\addcontentsline{toc}{subsubsection}{Exercise 5.4.}
\emph{Prove that $\sum_{a=1}^{p-1} (a/p) = 0$.} \\

\emph{Note.}
$\sum_{a=0}^{p-1} (a/p) = 0$ since $(0/p) = 0$. \\

\emph{Proof.}
There are as many residues as nonresidues mod $p$ (Corollary to Proposition 5.1.2).
$\Box$ \\\\



%%%%%%%%%%%%%%%%%%%%%%%%%%%%%%%%%%%%%%%%%%%%%%%%%%%%%%%%%%%%%%%%%%%%%%%%%%%%%%%%



\subsubsection*{Exercise 5.5.}
\addcontentsline{toc}{subsubsection}{Exercise 5.5.}
\emph{Prove that $\sum_{x=0}^{p-1} \left( \frac{ax+b}{p} \right) = 0$
provided that $p \nmid a$.} \\

\emph{Proof.}
Since $x$ $(x = 1, \ldots, p-1)$ is a reduced residue system modulo $p$,
$ax$ $(x = 1, \ldots, p-1)$ is again a reduced residue system modulo $p$ if $p \nmid a$
(Exercise 3.6).
Hence
$$\sum_{x=1}^{p-1} \left( \frac{ax}{p} \right) = 0.$$
Note that $\left( \frac{0}{p} \right) = 0$,
and thus
$0
= \sum_{x=0}^{p-1} \left( \frac{ax}{p} \right)
= \sum_{x=0}^{p-1} \left( \frac{a(x+a^{-1}b)}{p} \right)
= \sum_{x=0}^{p-1} \left( \frac{ax+b}{p} \right)$.
$\Box$ \\\\



%%%%%%%%%%%%%%%%%%%%%%%%%%%%%%%%%%%%%%%%%%%%%%%%%%%%%%%%%%%%%%%%%%%%%%%%%%%%%%%%



\subsubsection*{Exercise 5.6.}
\addcontentsline{toc}{subsubsection}{Exercise 5.6.}
\emph{Show that the number of solutions to $x^2 - y^2 \equiv a \: (p)$
is given by
$$\sum_{y=0}^{p-1} \left( 1 + \left( \frac{y^2 + a}{p} \right) \right).$$ }

\emph{Proof.}
Write $x^2 \equiv y^2 + a \: (p)$.
For every fixed $y = 0, \ldots, p-1$,
the number of solutions $x$ to $x^2 \equiv y^2 + a \: (p)$
is given by $1 + \left( \frac{y^2 + a}{p} \right)$ (Exercise 5.2).
Hence, the number of solutions $(x, y)$ to $x^2 - y^2 \equiv a \: (p)$
is
$$\sum_{y=0}^{p-1} \left( 1 + \left( \frac{y^2 + a}{p} \right) \right).$$
$\Box$ \\\\



%%%%%%%%%%%%%%%%%%%%%%%%%%%%%%%%%%%%%%%%%%%%%%%%%%%%%%%%%%%%%%%%%%%%%%%%%%%%%%%%



\subsubsection*{Exercise 5.7.}
\addcontentsline{toc}{subsubsection}{Exercise 5.7.}
\emph{By calculating directly show that
the number of solutions to $x^2 - y^2 \equiv a \: (p)$
is $p-1$ if $p \nmid a$ and $2p-1$ if $p \mid a$.
(Hint: Use the change of variables $u=x+y, v=x-y$.) } \\

\emph{Proof (Hint).}
Write $(x+y)(x-y) \equiv a \: (p)$ or $uv \equiv a \: (p)$ where $u=x+y, v=x-y$.
For any $a$, either $a \equiv 0 \: (p)$ or $a \not\equiv 0 \: (p)$.
\begin{enumerate}
\item[(1)]
$a \equiv 0 \: (p)$. Then $u = 0$ or $v = 0$.
Consider three possible cases (may be overlapped).
  \begin{enumerate}
  \item[(a)]
  $u = 0$, or $x+y = 0$. In this case, the number of solutions is $p$.
  ($x = k, y = -k$ for $k = 0, \ldots, p-1$.)
  \item[(b)]
  $v = 0$. Similar to (a), the number of solutions is $p$.
  ($x = k, y = k$ for $k = 0, \ldots, p-1$.)
  \item[(c)]
  $u = v = 0$. $x = y = 0$.
  \end{enumerate}
  By (a)(b)(c), there are $2p-1$ solutions to $x^2 - y^2 \equiv 0 \: (p)$.
\item[(2)]
$a \not\equiv 0 \: (p)$. $u \neq 0$ and $v \neq 0$.
For each $u = k$ for $k = 1, \ldots, p-1$, there is one unique $v = a k^{-1}$
such that $uv \equiv a \: (p)$. Solve $u$ and $v$ to get
$(x, y) = (2^{-1} (k + a k^{-1}), 2^{-1} (k - a k^{-1})) \in \mathbb{Z}/p\mathbb{Z}$
for $k = 1, \ldots, p-1$.
So there are $p-1$ solutions to $x^2 - y^2 \equiv a \: (p)$ where $a \not\equiv 0 \: (p)$.
\end{enumerate}
By (1)(2), the result holds.
$\Box$ \\\\



%%%%%%%%%%%%%%%%%%%%%%%%%%%%%%%%%%%%%%%%%%%%%%%%%%%%%%%%%%%%%%%%%%%%%%%%%%%%%%%%



\subsubsection*{Exercise 5.8.}
\addcontentsline{toc}{subsubsection}{Exercise 5.8.}
\emph{Combining the results of Exercise 5.6 and 5.7 show that
\begin{equation*}
  \sum_{y=0}^{p-1} \left( \frac{y^2 + a}{p} \right) =
    \begin{cases}
      -1,  & \text{ if $p \nmid a$}, \\
      p-1, & \text{ if $p \mid a$}.
    \end{cases}
\end{equation*}} \\

\emph{Proof.}
By Exercise 5.6 and 5.7,
\begin{equation*}
  \sum_{y=0}^{p-1} \left( 1 + \left( \frac{y^2 + a}{p} \right) \right) =
    \begin{cases}
      p-1,  & \text{ if $p \nmid a$}, \\
      2p-1, & \text{ if $p \mid a$}.
    \end{cases}
\end{equation*}
Hence the result holds.
$\Box$ \\\\



%%%%%%%%%%%%%%%%%%%%%%%%%%%%%%%%%%%%%%%%%%%%%%%%%%%%%%%%%%%%%%%%%%%%%%%%%%%%%%%%
%%%%%%%%%%%%%%%%%%%%%%%%%%%%%%%%%%%%%%%%%%%%%%%%%%%%%%%%%%%%%%%%%%%%%%%%%%%%%%%%



\newpage
\section*{Chapter 6: Quadratic Gauss Sums \\}
\addcontentsline{toc}{section}{Chapter 6: Quadratic Gauss Sums}



\subsubsection*{Exercise 6.1.}
\addcontentsline{toc}{subsubsection}{Exercise 6.1.}
\emph{Show that $\sqrt{2} + \sqrt{3}$ is an algebraic integer.} \\

\emph{Proof.}
Let $\alpha = \sqrt{2} + \sqrt{3}$. So $\alpha - \sqrt{2} = \sqrt{3}$.
Eliminating $\sqrt{3}$ by squaring:
$(\alpha - \sqrt{2})^2 = (\sqrt{3})^2$, or
$\alpha^2 - 2\sqrt{2}\alpha + 2 = 3$, or
$\alpha^2 - 1 = 2\sqrt{2}\alpha$.
Eliminating $\sqrt{2}$ by squaring again:
$(\alpha^2 - 1)^2 = (2\sqrt{2}\alpha)^2$, or
$\alpha^4 - 2 \alpha^2 + 1 = 8 \alpha^2$, or
$\alpha^4 - 10 \alpha^2 + 1 = 0$.
That is, $\alpha$ is a root of $x^4 - 10x^2 + 1 = 0$, i.e.,
$\alpha$ is an algebraic integer.
$\Box$ \\

Actually,
$x^4 - 10x^2 + 1 =
(x - \sqrt{2} - \sqrt{3})(x + \sqrt{2} - \sqrt{3})
(x - \sqrt{2} + \sqrt{3})(x + \sqrt{2} + \sqrt{3})$. \\

\emph{Proof (Proposition 6.1.5).}
Since $\sqrt{2}$ and $\sqrt{3}$ are algebraic integers,
then $\sqrt{2} + \sqrt{3}$ is an algebraic integer by Proposition 6.1.5.
(The set of algebraic integers forms a ring.)
$\Box$ \\\\



%%%%%%%%%%%%%%%%%%%%%%%%%%%%%%%%%%%%%%%%%%%%%%%%%%%%%%%%%%%%%%%%%%%%%%%%%%%%%%%%



\subsubsection*{Exercise 6.2.}
\addcontentsline{toc}{subsubsection}{Exercise 6.2.}
\emph{Let $\alpha$ be an algebraic number.
Show that there is an integer $n$ such that $n\alpha$ is an algebraic integer.} \\

It is trivial if taking $n = 0$. So we assume that $n \neq 0$. \\

\emph{Proof.}
There exists a polynomial
$f(x) = a_0 x^m + a_1 x^{m-1} + \cdots + a_m \in \mathbb{Q}[x]$ with $a_0 \neq 0$,
such that $f(\alpha) = 0$.
There exists an integer $d \neq 0$ such that
$b_i = d \cdot a_i \in \mathbb{Z}$ for all $i = 1, 2, ..., m$.
Therefore,
$$b_0 \alpha^m + b_1 \alpha^{m-1} + \cdots + b_m = 0,$$
which is not necessary a monic polynomial in $\mathbb{Z}[x]$.
So we need to do a trick to absort $b_0$ into $\alpha$,
and that is why we come out multipling $\alpha$ by an non-zero integer
$b_0 = d \cdot a_0$. \\

Multiply $b_0^{m-1}$ on the both sides.
\begin{align*}
b_0^m \alpha^m + b_0^{m-1} b_1 \alpha^{m-1} + b_0^{m-1} b_2 \alpha^{m-2}
+ \cdots + b_0^{m-1} b_m &= 0. \\
(b_0\alpha)^m + b_1 (b_0\alpha)^{m-1} + b_0 b_2 (n\alpha)^{m-1}
+ \cdots + b_0^{m-1} b_m &= 0.
\end{align*}
That is, the monic polynomial
$g(x) = x^m + c_1 x^{m-1} + c_2 x^{m-2} + \cdots + c_m \in \mathbb{Z}[x]$
(with $c_i = b_0^{i - 1} b_i$ for $i = 1, 2, ..., m$)
has a root $x = b_0\alpha$, i.e.,
$b_0\alpha$ is an algebraic integer for some integer $b_0$.
$\Box$ \\\\



%%%%%%%%%%%%%%%%%%%%%%%%%%%%%%%%%%%%%%%%%%%%%%%%%%%%%%%%%%%%%%%%%%%%%%%%%%%%%%%%



\subsubsection*{Exercise 6.4.}
\addcontentsline{toc}{subsubsection}{Exercise 6.4.}
\emph{A polynomial $f(x) \in \mathbb{Z}[x]$ is said to be primitive
if the greatest common divisor of its coefficients is $1$.
Prove that the product of primitive polynomials is again primitive.
This is one of the many results knows as Gauss' lemma.} \\

\emph{Proof.}
Let
\begin{align*}
f(x) &= a_0 x^n + a_1 x^{n-1} + \cdots + a_n, \\
g(x) &= b_0 x^m + b_1 x^{n-1} + \cdots + b_m
\end{align*}
be primitive.
\begin{enumerate}
\item[(1)]
Given prime $p$.
Let $a_i$ and $b_j$ be the coefficients with the smallest index such that
$p \nmid a_i$ and $p \nmid b_j$ respectively.
Consider the coefficient of $x^{i+j}$ in $f(x)g(x)$,
$$(\cdots + a_{i-1} b_{j+1}) + a_i b_j + (a_{i+1} b_{j-1} + \cdots).$$
$p \nmid a_i b_j$ since $p$ is a prime.
$p \mid (\cdots + a_{i-1} b_{j+1})$ by the definition of index $i$.
$p \mid (a_{i+1} b_{j-1} + \cdots)$ by the definition of index $j$.
That is, the coefficient of $x^{i+j}$ in $f(x)g(x)$ is not divided by $p$.
\item[(2)]
If $h(x) = f(x)g(x)$ is not primitive,
there exists a prime $p$ such that $p$ divides all coefficients of $h(x)$.
By (1), such $i$ or $j$ does not exist.
That is, $p$ is a factor of the greatest common divisor of $f(x)$'s or $g(x)$'s coefficients.
So $f(x)$ or $g(x)$ is not primitive, which is absurd.
\end{enumerate}
$\Box$ \\\\



%%%%%%%%%%%%%%%%%%%%%%%%%%%%%%%%%%%%%%%%%%%%%%%%%%%%%%%%%%%%%%%%%%%%%%%%%%%%%%%%



\subsubsection*{Exercise 6.16.}
\addcontentsline{toc}{subsubsection}{Exercise 6.16.}
\emph{Let $\alpha$ be an algebraic number with minimal polynomial $f(x)$.
Show that $f(x)$ does not have repeated roots $\alpha$ in $\mathbb{C}$.} \\

\emph{Proof.}
Assume not true, write $f(x) = (x - \alpha)^2 g(x)$,
where $g(x) \in \mathbb{C}[x]$.
Differentiating $f(x)$ to get new polynomial $f'(x) \in \mathbb{Q}[x]$ and
\begin{align*}
f'(x)
&= 2(x - \alpha) g(x) + (x - \alpha)^2 g'(x) \\
&= (x - \alpha)(2 g(x) + (x - \alpha) g'(x)).
\end{align*}
So $f'(\alpha) = 0$.
Notices that $\deg f(x) \geq 2$ and thus $\deg f'(x) = \deg f(x) - 1 \geq 1$.
$f'(x)$ is not zero. Thus $f(x) \mid f'(x)$ by Proposition 6.1.7, which
contradicts the fact $0 < \deg f'(x) < \deg f(x)$.
$\Box$ \\\\



%%%%%%%%%%%%%%%%%%%%%%%%%%%%%%%%%%%%%%%%%%%%%%%%%%%%%%%%%%%%%%%%%%%%%%%%%%%%%%%%



\subsubsection*{Exercise 6.17.}
\addcontentsline{toc}{subsubsection}{Exercise 6.17.}
\emph{Show that the minimal polynomial for $\sqrt[3]{2}$ is $x^3 - 2$.} \\

\emph{Proof.}
Let $f(x) = x^3 - 2$. $f(\sqrt[3]{2}) = 0$.
By Eisenstein's irreducibility criterion, $f(x)$ is irreducible over $\mathbb{Q}$.
By Proposition 6.1.7, $f(x) = x^3 - 2$ is the minimal polynomial of $\sqrt[3]{2}$.
$\Box$ \\\\



%%%%%%%%%%%%%%%%%%%%%%%%%%%%%%%%%%%%%%%%%%%%%%%%%%%%%%%%%%%%%%%%%%%%%%%%%%%%%%%%



\subsubsection*{Exercise 6.18.}
\addcontentsline{toc}{subsubsection}{Exercise 6.18.}
\emph{Show that there exist algebraic numbers of arbitrarily high degree.} \\

A generalization to Exercise 6.17. \\

If $p$ is a prime, then $x^n - p$ is irreducible over $\mathbb{Q}$,
by Eisenstein's irreducibility criterion, so
$[\mathbb{Q}(\sqrt[n]{p}):\mathbb{Q}] = n$.
(Example 1.16 in Patrick Morandi, Field and Galois Theory.) \\


\emph{Proof.}
Let $\alpha = \sqrt[n]{p}$ for any positive integer $n$ with $n \geq 2$
and prime $p$.
Apply the similar argument in Exercise 6.17 to show that
$f(x) = x^n - p$ is the minimal polynomial of $\sqrt[n]{p}$.
$\Box$ \\\\



%%%%%%%%%%%%%%%%%%%%%%%%%%%%%%%%%%%%%%%%%%%%%%%%%%%%%%%%%%%%%%%%%%%%%%%%%%%%%%%%



\subsubsection*{Exercise 6.23.}
\addcontentsline{toc}{subsubsection}{Exercise 6.23.}
\emph{If $f(x) = x^n + a_1 x^{n-1} + \cdots + a_n$, $a_i \in \mathbb{Z}$
and $p$ is a prime such that $p \mid a_i$, $i = 1, ..., n$, $p^2 \nmid a_n$.
Show that $f(x)$ is irreducible over $\mathbb{Q}$
(Eisenstein's irreducibility criterion).} \\

\emph{Proof.}
\begin{enumerate}
\item[(1)]
\emph{If $f(x) = x^n + a_1 x^{n-1} + \cdots + a_n$, $a_i \in \mathbb{Z}$
and $p$ is a prime such that $p \mid a_i$, $i = 1, ..., n$, $p^2 \nmid a_n$.
Then $f(x)$ is irreducible over $\mathbb{Z}$.}
Assume not true.
Write $f(x) = g(x)h(x)$ as a product of two non-trivial polynomials in $\mathbb{Z}[x]$,
\begin{align*}
g(x) &= b_0 x^s + b_1 x^{s-1} + \cdots + b_s, \\
h(x) &= c_0 x^t + c_1 x^{t-1} + \cdots + c_t,
\end{align*}
where $b_0 = c_0 = 1$, $0 < s < n$, and $0 < t < n$. \\

Since $p \nmid b_0$, there exists largest index $i$ such that $p \nmid b_i$.
(Therefore $p \mid b_{i+1}$, $p \mid b_{i+2}$, and so on.)
Similarly, there exists largest index $j$ such that $p \nmid c_j$.
($p \mid c_{j+1}$, $p \mid c_{j+2}$, and so on.)
Now we consider the coefficient $a_{i+j}$.
$$a_{i+j} = (\cdots + b_{i-1} c_{j+1}) + b_i c_j + (b_{i+1} c_{j-1} + \cdots).$$
$p \nmid b_i c_j$ since $p$ is a prime.
$p \mid (b_{i+1} c_{j-1} + \cdots)$ by the definition of index $i$.
$p \mid (\cdots + b_{i-1} c_{j+1})$ By the definition of index $j$.
Thus, $p \nmid a_{i+j}$.
Hence $i = 0$ and $j = 0$. Especially, $p \mid b_s$ and $p \mid c_t$.
$p^2 \mid b_s c_t$, or $p^2 \mid a_n$ which contradicts.
$\Box$
\item[(2)]
\emph{$f(x)$ is irreducible over $\mathbb{Q}$
if $f(x)$ is primitive and irreducible over $\mathbb{Z}$.}
Assume $f(x) = g(x)h(x) \in \mathbb{Q}[x]$ is reducible.
Let $a$ and $b$ be the least common multiple of the denominators of
$g(x)$ and $h(x)$ respectively.
Then
$$ab \cdot f(x) = (a \cdot g(x))(b \cdot h(x)) = c g_0(x) d h_0(x),$$
where $g_0(x)$, $h_0(x)$ are primitive polynomials in $\mathbb{Z}[x]$, and
$c$ and $d$ are the greatest common divisor of
$(a \cdot g(x))$'s and $(b \cdot h(x))$'s coefficients respectively
Since $g_0(x) h_0(x)$ is again primitive (Exercise 4),
$$ab = \pm cd, f(x) = g_0(x) h_0(x).$$
Notice that $\deg(g_0(x)) = \deg(g(x))$ and $\deg(h_0(x)) = \deg(h(x))$.
So $f(x)$ is reducible over $\mathbb{Z}$, which is absurd.
\end{enumerate}
$\Box$ \\\\



%%%%%%%%%%%%%%%%%%%%%%%%%%%%%%%%%%%%%%%%%%%%%%%%%%%%%%%%%%%%%%%%%%%%%%%%%%%%%%%%
%%%%%%%%%%%%%%%%%%%%%%%%%%%%%%%%%%%%%%%%%%%%%%%%%%%%%%%%%%%%%%%%%%%%%%%%%%%%%%%%



\newpage
\section*{Chapter 15: Bernoulli Numbers \\}
\addcontentsline{toc}{section}{Chapter 15: Bernoulli Numbers}



% Reference:
% https://dms.umontreal.ca/~mlalin/bernoulli.pdf
% http://cds.iisc.ac.in/faculty/amohanty/SE288/bn.pdf
% http://people.math.sfu.ca/~cbm/aands/abramowitz_and_stegun.pdf



\subsubsection*{Supplement.}
\addcontentsline{toc}{subsubsection}{Supplement.}
Equation (4) on page 231.
\emph{Prove that $$x \cot x = 1 - 2 \sum_{n=1}^{\infty} \frac{x^2}{n^2 \pi^2 - x^2}.$$} \\

\emph{Proof (Exercise 6.73 in the book Graham, Knuth and Patashnik,
Concrete Mathematics, Second Edition).}

\begin{enumerate}
\item[(1)]
\emph{Show that $$\cot x
= \frac{1}{2^n} \sum_{k=0}^{2^{n} - 1} \cot \frac{x + k\pi}{2^n}$$
for all integers $n \geq 1$.}
Notice that
\begin{align*}
\cot(x + \pi) &= \cot x, \\
\cot\left( x + \frac{\pi}{2} \right) &= -\tan x, \\
\cot x &= \frac{1}{2} \left( \cot\frac{x}{2} - \tan\frac{x}{2} \right).
\end{align*}
Use mathematical induction.
The case $n = 1$ is the same as the note.
Assume the case $n = m$ holds.
For $n = m+1$,
\begin{align*}
\sum_{k=0}^{2^{m+1} - 1} \cot \frac{x + k\pi}{2^{m+1}}
&= \sum_{k=0}^{2^{m} - 1} \cot \frac{x + k\pi}{2^{m+1}}
+ \sum_{k=2^{m}}^{2^{m+1} - 1} \cot \frac{x + k\pi}{2^{m+1}} \\
&= \sum_{k=0}^{2^{m} - 1} \cot \frac{x + k\pi}{2^{m+1}}
+ \sum_{k=0}^{2^{m} - 1} \cot \frac{x + (2^{m} + k)\pi}{2^{m+1}} \\
&= \sum_{k=0}^{2^{m} - 1} \cot \frac{x + k\pi}{2^{m+1}}
+ \sum_{k=0}^{2^{m} - 1} \cot \left( \frac{x + k\pi}{2^{m+1}} + \frac{\pi}{2} \right) \\
&= \sum_{k=0}^{2^{m} - 1}
\left( \cot \frac{x + k\pi}{2^{m+1}} - \tan \frac{x + k\pi}{2^{m+1}} \right) \\
&= \sum_{k=0}^{2^{m} - 1}
\left( \cot \frac{x + k\pi}{2^{m+1}} - \tan \frac{x + k\pi}{2^{m+1}} \right) \\
&= 2 \sum_{k=0}^{2^{m} - 1} \cot \frac{x + k\pi}{2^{m}}.
\end{align*}
Therefore,
\begin{align*}
\frac{1}{2^{m+1}} \sum_{k=0}^{2^{m+1} - 1} \cot \frac{x + k\pi}{2^{m+1}}
&= \frac{1}{2^{m+1}} \cdot 2 \sum_{k=0}^{2^{m} - 1} \cot \frac{x + k\pi}{2^{m}} \\
&= \frac{1}{2^{m}} \sum_{k=0}^{2^{m} - 1} \cot \frac{x + k\pi}{2^{m}} \\
&= \cot x.
\end{align*}
\item[(2)]
By rearranging the index of summation of the identity in (1), we have
$$x \cot x
= \frac{x}{2^n} \cot \frac{x}{2^n} - \frac{x}{2^n} \tan \frac{x}{2^n}
+ \sum_{k=1}^{2^{n-1} - 1} \frac{x}{2^n}
\left( \cot \frac{x + k\pi}{2^n} + \cot \frac{x - k\pi}{2^n} \right)$$
for all integers $n \geq 1$.
\item[(3)]
Notice that $\lim_{x \rightarrow 0} x \cot x = 1$.
Let $n \rightarrow \infty$, the result is established.
\end{enumerate}
$\Box$ \\\\



%%%%%%%%%%%%%%%%%%%%%%%%%%%%%%%%%%%%%%%%%%%%%%%%%%%%%%%%%%%%%%%%%%%%%%%%%%%%%%%%



\subsubsection*{Exercise 15.1.}
\addcontentsline{toc}{subsubsection}{Exercise 15.1.}
\emph{Using the definition of the Bernoulli number show
$B_{10} = \frac{5}{66}$ and $B_{12} = -\frac{691}{2730}$.} \\

\emph{Proof.}
\begin{enumerate}
\item[(1)]
It is known that
$B_1 = -\frac{1}{2}$,
$B_2 = \frac{1}{6}$,
$B_4 = -\frac{1}{30}$,
$B_6 = \frac{1}{42}$,
and $B_m = 0$ for odd $m > 1$.
\item[(2)]
Recall the implicit recurrence relation,
$$\sum_{k = 0}^{m} {m+1 \choose k} B_k = [m = 0],$$
where $[m = 0]$ is the Iverson brackets which is equal to
the Kronecker delta $\delta_{m0}$.
\item[(3)]
So
\begin{align*}
0 &= 1 + 9 B_1 + 36 B_2 + 84 B_3 + 126 B_4 + 126 B_5 + 84 B_6 + 36 B_7 + 9 B_8, \\
0 &= 1 + 9 B_1 + 36 B_2 + 126 B_4 + 84 B_6 + 9 B_8, \\
0 &= 1 + 9 \left( -\frac{1}{2} \right)
+ 36 \left( \frac{1}{6} \right)
+ 126 \left( -\frac{1}{30} \right)
+ 84 \left( \frac{1}{42} \right)
+ 9 B_8, \\
0 &= \frac{3}{10} + 9B_8,
\end{align*}
Thus $B_8 = -\frac{1}{30}$. \\
\item[(4)]
Again,
\begin{align*}
0 =& 1 + 11 B_1 + 55 B_2 + 165 B_3 + 330 B_4 + 462 B_5 + 462 B_6 + \\
   & 330 B_7 + 165 B_8 + 55 B_9 + 11 B_{10}, \\
0 =& 1 + 11 B_1 + 55 B_2 + 330 B_4 + 462 B_6 + 165 B_8 + 11 B_{10}, \\
0 =& 1 + 11 \left( -\frac{1}{2} \right) +
     55 \left( \frac{1}{6} \right) +
     330 \left( -\frac{1}{30} \right) +
     462 \left( \frac{1}{42} \right) + \\
   & 165 \left( -\frac{1}{30} \right) +
     11 B_{10}, \\
0 =& -\frac{5}{6} + 11 B_{10},
\end{align*}
Thus $B_{10} = \frac{5}{66}$. \\
\item[(4)]
Finally,
\begin{align*}
0 =& 1 + 13 B_1 + 78 B_2 + 286 B_3 + 715 B_4 + 1287 B_5 + 1716 B_6 + \\
   & 1716 B_7 + 1287 B_8 + 715 B_9 + 286 B_{10} + 78 B_{11} + 13 B_{12}, \\
0 =& 1 + 13 B_1 + 78 B_2 + 715 B_4 + 1716 B_6 + 1287 B_8 + 286 B_{10} + 13 B_{12}, \\
0 =& 1 + 13 \left( -\frac{1}{2} \right) +
     78 \left( \frac{1}{6} \right) +
     715 \left( -\frac{1}{30} \right) +
     1716 \left( \frac{1}{42} \right) + \\
   & 1287 \left( -\frac{1}{30} \right) +
     286 \left( \frac{5}{66} \right) +
     13 B_{12}, \\
0 =& \frac{691}{210} + 13 B_{12},
\end{align*}
Thus $B_{12} = -\frac{691}{2730}$.
\end{enumerate}
$\Box$\\\\



%%%%%%%%%%%%%%%%%%%%%%%%%%%%%%%%%%%%%%%%%%%%%%%%%%%%%%%%%%%%%%%%%%%%%%%%%%%%%%%%



\subsubsection*{Exercise 15.2.}
\addcontentsline{toc}{subsubsection}{Exercise 15.2.}
\emph{If $a \in \mathbb{Z}$,
show $a(a^m - 1)B_m \in \mathbb{Z}$ for all $m > 0$.} \\

\emph{Proof.}
\begin{enumerate}
\item[(1)] \emph{Trivial cases.}
If $m = 1$, $a(a - 1) B_1 = -\frac{1}{2} a(a - 1) \in \mathbb{Z}$ for any $a \in \mathbb{Z}$.
For odd $m > 1$, $B_m = 0$ or $a(a^m - 1)B_m = 0 \in \mathbb{Z}$ (Proposition 15.1.1). \\
\item[(2)] \emph{Consider that $m > 1$ and even.}
By Theorem 3,
$$B_{2m} + \sum_{p-1 \mid 2m} \frac{1}{p} \in \mathbb{Z}$$
where the sum is over all primes $p$ such that $p-1 \mid 2m$.
So it suffices to show
$$a(a^{2m} - 1) \sum_{p-1 \mid 2m} \frac{1}{p} \in \mathbb{Z}, $$
or
$$a(a^{2m} - 1) \frac{1}{p} \in \mathbb{Z}$$
for any $a \in \mathbb{Z}$ and any prime $p$ such that $p-1 \mid 2m$.
\item[(3)] \emph{Consider all possible $a$.}
If $p \mid a$, it is trivial.
If $p \nmid a$, $a^{p - 1} \equiv 1 \: (p)$ by Fermat's Little Theorem,
or $a^{2m} \equiv 1 \: (p)$ by $p-1 \mid 2m$.
In any cases, $a(a^{2m} - 1)\frac{1}{p} \in \mathbb{Z}$.
\end{enumerate}
$\Box$ \\\\



%%%%%%%%%%%%%%%%%%%%%%%%%%%%%%%%%%%%%%%%%%%%%%%%%%%%%%%%%%%%%%%%%%%%%%%%%%%%%%%%



\subsubsection*{Exercise 15.6.}
\addcontentsline{toc}{subsubsection}{Exercise 15.6.}
\emph{For $m \geq 3$, show $|B_{2m+2}| > |B_{2m}|$. (Hint: Use Theorem 2.)} \\

\emph{Proof.}
By Theorem 2,
$$2 \zeta(2m) = (-1)^{m+1} \frac{(2\pi)^{2m}}{(2m)!} B_{2m}.$$
Thus,
$$\frac{|B_{2m+2}|}{|B_{2m}|}
= \frac{\zeta(2m+2)(2m+2)(2m+1)}{\zeta(2m)(2\pi)^2}
> \frac{1 \cdot 8 \cdot 7}{\zeta(6) \cdot (2\pi)^2}
= \frac{13230}{\pi^8}
> 1,$$
or $|B_{2m+2}| > |B_{2m}|$.
$\Box$ \\\\



%%%%%%%%%%%%%%%%%%%%%%%%%%%%%%%%%%%%%%%%%%%%%%%%%%%%%%%%%%%%%%%%%%%%%%%%%%%%%%%%



\subsubsection*{Exercise 15.8.}
\addcontentsline{toc}{subsubsection}{Exercise 15.8.}
\emph{Consider the power series expansion of $\tan x$ about the origin;
$$\sum_{k=1}^{\infty} T_k \frac{x^{2k - 1}}{(2k - 1)!}.$$
Show $$T_k = (-1)^{k-1} \frac{B_{2k}}{2k} (2^{2k} - 1) 2^{2k}.$$
Note that $T_k \in \mathbb{Z}$ for all $k$ by Exercise 3.} \\

\emph{Proof.}
\begin{enumerate}
\item[(1)]
By the equation (6) on page 232,
$$x \cot x = 1 + \sum_{k=2}^{\infty} B_k \frac{(2ix)^k}{k!}.$$
Since $B_k = 0$ for odd $k > 1$,
$$x \cot x
= 1 + \sum_{k=1}^{\infty} B_{2k} \frac{(2ix)^{2k}}{(2k)!}
= 1 + \sum_{k=1}^{\infty} \frac{(-1)^k 2^{2k}B_{2k}}{(2k)!} x^{2k},$$
or
$$\cot x
= \frac{1}{x} + \sum_{k=1}^{\infty} \frac{(-1)^k 2^{2k}B_{2k}}{(2k)!} x^{2k-1}.$$
Combine the first term $\frac{1}{x}$ into the summation,
$$\cot x = \sum_{k=0}^{\infty} \frac{(-1)^k 2^{2k}B_{2k}}{(2k)!} x^{2k-1}.$$
\item[(2)]
Note that $\tan x = \cot x - 2 \cot(2x)$.
By (1),
\begin{align*}
\tan x
&= \sum_{k=0}^{\infty} \frac{(-1)^k 2^{2k}B_{2k}}{(2k)!} x^{2k-1}
- 2 \sum_{k=0}^{\infty} \frac{(-1)^k 2^{2k}B_{2k}}{(2k)!} (2x)^{2k-1} \\
&= \sum_{k=0}^{\infty} \frac{(-1)^k (1 - 2^{2k}) 2^{2k} B_{2k}}{(2k)!} x^{2k-1} \\
&= \sum_{k=1}^{\infty} \frac{(-1)^k (1 - 2^{2k}) 2^{2k} B_{2k}}{(2k)!} x^{2k-1}.
\end{align*}
Write $T_k = (-1)^{k-1} (2^{2k} - 1) 2^{2k} \frac{B_{2k}}{2k}$.
Therefore, $\tan x = \sum_{k=1}^{\infty} T_k \frac{x^{2k - 1}}{(2k - 1)!}$.
\end{enumerate}
By Exercise 15.3, $(2^{2k} - 1) 2^{2k} \frac{B_{2k}}{2k} \in \mathbb{Z}$,
or $T_k \in \mathbb{Z}$ for all $k$.
$\Box$ \\\\



%%%%%%%%%%%%%%%%%%%%%%%%%%%%%%%%%%%%%%%%%%%%%%%%%%%%%%%%%%%%%%%%%%%%%%%%%%%%%%%%



\subsubsection*{Exercise 15.12.}
\addcontentsline{toc}{subsubsection}{Exercise 15.12.}
\emph{Recall the definition of the Bernoulli polynomials;
$$B_m(x) = \sum_{k=0}^{m} {m \choose k} B_k x^{m-k}.$$
Show that
$$\frac{te^{tx}}{e^t - 1} = \sum_{m=0}^{\infty} B_m(x) \frac{t^m}{m!}.$$} \\

\emph{Proof.}
By Lemma 1,
$$\frac{t}{e^t - 1} = \sum_{m=0}^{\infty} B_m \frac{t^m}{m!}.$$
So
$$
\frac{te^{tx}}{e^t - 1}
=
\left( \sum_{m=0}^{\infty} B_m \frac{t^m}{m!} \right)
\left( \sum_{m=0}^{\infty} x^m \frac{t^m}{m!} \right).$$
Write $\frac{te^{tx}}{e^t - 1} = \sum_{m=0}^{\infty} b_m(x) \frac{t^m}{m!}$
and we want to check if $b_m(x) = B_m(x)$ or not.
The result is established if $b_m(x) = B_m(x)$ holds.
Equating coefficients of $t^m$ gives
\begin{align*}
\frac{b_m(x)}{m!}
&= \sum_{k=0}^{m} \frac{B_k x^{m-k}}{k!(m-k)!}, \\
b_m(x)
&= \sum_{k=0}^{m} \frac{m!}{k!(m-k)!} B_k x^{m-k} \\
&= \sum_{k=0}^{m} {m \choose k} B_k x^{m-k} \\
&= B_m(x).
\end{align*}
$\Box$ \\\\



%%%%%%%%%%%%%%%%%%%%%%%%%%%%%%%%%%%%%%%%%%%%%%%%%%%%%%%%%%%%%%%%%%%%%%%%%%%%%%%%



\subsubsection*{Exercise 15.13.}
\addcontentsline{toc}{subsubsection}{Exercise 15.13.}
\emph{Show $B_m(x+1) - B_m(x) = mx^{m-1}$.} \\

\emph{Proof.}
Let $f(t, x) = \frac{te^{tx}}{e^t - 1}$.
\begin{enumerate}
\item[(1)]
$$f(t, x+1) - f(t, x)
= \frac{te^{t(x+1)}}{e^t - 1} - \frac{te^{tx}}{e^t - 1}
= te^{tx}.$$
Expand $te^{tx}$ in a power series about the origin as follows
\begin{align*}
te^{tx}
&= t \sum_{m=0}^{\infty} x^m \frac{t^m}{m!} \\
&= \sum_{m=0}^{\infty} x^m \frac{t^{m+1}}{m!} \\
&= \sum_{m=1}^{\infty} x^{m-1} \frac{t^m}{(m-1)!} \\
&= \sum_{m=1}^{\infty} mx^{m-1} \frac{t^m}{m!} \\
&= \sum_{m=0}^{\infty} mx^{m-1} \frac{t^m}{m!}.
\end{align*}
So,
$$f(t, x+1) - f(t, x) = \sum_{m=0}^{\infty} mx^{m-1} \frac{t^m}{m!}.$$
\item[(2)]
By Exercise 15.12,
\begin{align*}
f(t, x+1) - f(t, x)
&= \sum_{m=0}^{\infty} B_m(x+1) \frac{t^m}{m!}
- \sum_{m=0}^{\infty} B_m(x) \frac{t^m}{m!} \\
&= \sum_{m=0}^{\infty} (B_m(x+1) - B_m(x)) \frac{t^m}{m!}.
\end{align*}
\end{enumerate}
By (1)(2), comparing coefficients of $t^m$ yields
$$mx^{m-1} = B_m(x+1) - B_m(x).$$
$\Box$ \\\\



%%%%%%%%%%%%%%%%%%%%%%%%%%%%%%%%%%%%%%%%%%%%%%%%%%%%%%%%%%%%%%%%%%%%%%%%%%%%%%%%



\subsubsection*{Exercise 15.14.}
\addcontentsline{toc}{subsubsection}{Exercise 15.14.}
\emph{Use Exercise 13 to give a new proof of Theorem 1:
$$S_m(n) = \frac{1}{m+1}(B_{m+1}(n) - B_{m+1}).$$}

\emph{Proof.}
By Exercise 13,
$$B_{m+1}(k) - B_{m+1}(k-1) = (m+1)(k-1)^m$$
for any $k$.
So,
\begin{align*}
\sum_{k=1}^{n} (B_{m+1}(k) - B_{m+1}(k-1))
&= \sum_{k=1}^{n} (m+1)(k-1)^m, \\
B_{m+1}(n) - B_{m+1}(0)
&= (m+1)S_m(n).
\end{align*}
Note that $B_{m+1}(0) = B_{m+1}$ for any $m$.
So Theorem 1 is established by a new proof.
$\Box$ \\\\



%%%%%%%%%%%%%%%%%%%%%%%%%%%%%%%%%%%%%%%%%%%%%%%%%%%%%%%%%%%%%%%%%%%%%%%%%%%%%%%%



\subsubsection*{Exercise 15.15.}
\addcontentsline{toc}{subsubsection}{Exercise 15.15.}
\emph{Suppose $f(x) = \sum_{k=0}^{n} a_k x^k$
be a polynomial with complex coefficients.
Use Exercise 13 to find a polynomial $F(x)$ such that
$F(x+1) - F(x) = f(x)$.} \\

\emph{Proof.}
By Exercise 15.13,
$$x^k = \frac{1}{k+1}(B_{k+1}(x+1) - B_{k+1}(x))$$
for $k \geq 0$.
Thus,
\begin{align*}
f(x)
&= \sum_{k=0}^{n} a_k x^k \\
&= \sum_{k=0}^{n} a_k \cdot \frac{1}{k+1} (B_{k+1}(x+1) - B_{k+1}(x)) \\
&= \sum_{k=0}^{n} \frac{a_k}{k+1} B_{k+1}(x+1) - \sum_{k=0}^{n} \frac{a_k}{k+1} B_{k+1}(x).
\end{align*}
Let
$$F(x) = \sum_{k=0}^{n} \frac{a_k}{k+1} B_{k+1}(x),$$
and we get $f(x) = F(x+1) - F(x)$.
$\Box$ \\\\



%%%%%%%%%%%%%%%%%%%%%%%%%%%%%%%%%%%%%%%%%%%%%%%%%%%%%%%%%%%%%%%%%%%%%%%%%%%%%%%%



\subsubsection*{Exercise 15.16.}
\addcontentsline{toc}{subsubsection}{Exercise 15.16.}
\emph{For $n \geq 1$, show $\frac{d}{dx}B_n(x) = nB_{n-1}(x)$.} \\

\emph{Proof.}
For $n \geq 1$,
$$\frac{d}{dx}B_n(x)
= \sum_{k=0}^{n}(n - k) {n \choose k} B_k x^{n - k - 1}
= \sum_{k=0}^{n-1}(n - k) {n \choose k} B_k x^{n - k - 1}.$$
Note that
$$(n-k) {n \choose k} = n {n-1 \choose k}.$$
So
\begin{align*}
\frac{d}{dx}B_n(x)
&= \sum_{k=0}^{n-1} n {n-1 \choose k} B_k x^{n - k - 1} \\
&= n \sum_{k=0}^{n-1} {n-1 \choose k} B_k x^{n - k - 1} \\
&= n B_{n-1}(x).
\end{align*}
$\Box$ \\\\



%%%%%%%%%%%%%%%%%%%%%%%%%%%%%%%%%%%%%%%%%%%%%%%%%%%%%%%%%%%%%%%%%%%%%%%%%%%%%%%%



\subsubsection*{Exercise 15.17.}
\addcontentsline{toc}{subsubsection}{Exercise 15.17.}
\emph{Show $B_n(1-x) = (-1)^n B_n(x)$.} \\

\emph{Proof.}
Let $f(t, x) = \frac{te^{tx}}{e^t - 1}$.
\begin{enumerate}
\item[(1)]
\emph{$f(t, 1-x) = f(-t, x)$.}
$$f(t, 1-x)
= \frac{te^{t(1-x)}}{e^t - 1}
= e^t \cdot \frac{te^{-tx}}{e^t - 1}
= \frac{-te^{-tx}}{e^{-t} - 1} = f(-t, x).$$
\item[(2)]
By Exercise 15.12,
\begin{align*}
f(t, 1-x)
&= \sum_{n=0}^{\infty} B_n(1-x) \frac{t^n}{n!} \\
f(-t, x)
&= \sum_{n=0}^{\infty} (-1)^n B_n(x) \frac{t^n}{n!}.
\end{align*}
By (1), comparing coefficients of $t^n$ yields
$B_n(1-x) = (-1)^n B_n(x)$.
\end{enumerate}
$\Box$ \\\\



%%%%%%%%%%%%%%%%%%%%%%%%%%%%%%%%%%%%%%%%%%%%%%%%%%%%%%%%%%%%%%%%%%%%%%%%%%%%%%%%



\subsubsection*{Exercise 15.18.}
\addcontentsline{toc}{subsubsection}{Exercise 15.18.}
\emph{Use Exercise 13 and 17 to give a new proof that
$B_n = 0$ for $n$ odd and $n > 1$.} \\

\emph{Proof.}
\begin{enumerate}
\item[(1)]
\emph{$B_m(1) - B_m(0) = 0$ for any $m > 1$.}
Taking $x = 0$ in Exercise 15.13 and keeping $m - 1 > 0$ or $m > 1$.
\item[(2)]
\emph{$B_m(1) = -B_m(0)$ for any odd $m$.}
Taking $x = 0$ in Exercise 15.17 and keeping $m$ is odd.
\begin{align*}
f(t, 1-x)
&= \sum_{n=0}^{\infty} B_n(1-x) \frac{t^n}{n!} \\
f(-t, x)
&= \sum_{n=0}^{\infty} (-1)^n B_n(x) \frac{t^n}{n!}.
\end{align*}
\end{enumerate}
By (1)(2), for $m$ odd and $m > 1$, $B_m(0) = 0$ or $B_m = 0$.
$\Box$ \\\\



%%%%%%%%%%%%%%%%%%%%%%%%%%%%%%%%%%%%%%%%%%%%%%%%%%%%%%%%%%%%%%%%%%%%%%%%%%%%%%%%



\subsubsection*{Exercise 15.19. (Multiplication theorem for Bernoulli polynomial)}
\addcontentsline{toc}{subsubsection}{Exercise 15.19. (Multiplication theorem for Bernoulli polynomial)}
\emph{Suppose $n$ and $F$ are integers and $n, F > 0$. Show that
$$B_n(Fx) = F^{n-1} \sum_{a=0}^{F-1} B_n \left(x + \frac{a}{F} \right).$$
(Hint: Use Exercise 12.)} \\

\emph{Proof.}
By $x^n - y^n = (x - y)(x^{n-1} + x^{n-2}y + \cdots + y^{n-1})$ (Exercise 1.24),
$$e^{Ft} - 1
= (e^t - 1)(1 + e^t + e^{2t} + \cdots + e^{(F-1)t})
= (e^t - 1) \sum_{a=0}^{F-1} e^{at}.$$
So,
\begin{align*}
\frac{1}{e^t - 1}
&= \frac{1}{e^{Ft} - 1} \sum_{a=0}^{F-1} e^{at}, \\
\frac{te^{tFx}}{e^t - 1}
&= \frac{te^{tFx}}{e^{Ft} - 1} \sum_{a=0}^{F-1} e^{at} \\
&= \sum_{a=0}^{F-1} \frac{te^{(Fx+a)t}}{e^{Ft} - 1} \\
&= \sum_{a=0}^{F-1} \frac{te^{(Fx+a)t}}{e^{Ft} - 1} \\
&= \sum_{a=0}^{F-1} F^{-1} \frac{(Ft) e^{(x + \frac{a}{F})(Ft)}}{e^{Ft} - 1}. \\
\end{align*}
By Exercise 15.12,
\begin{align*}
\sum_{n=0}^{\infty} B_n(Fx) \frac{t^n}{n!}
&= \sum_{a=0}^{F-1} F^{-1}
\sum_{n=0}^{\infty} B_n \left( x + \frac{a}{F} \right) \frac{(Ft)^n}{n!} \\
&= \sum_{n=0}^{\infty} \sum_{a=0}^{F-1} F^{-1}
B_n \left( x + \frac{a}{F} \right) \frac{(Ft)^n}{n!} \\
&= \sum_{n=0}^{\infty} \sum_{a=0}^{F-1} F^{n-1}
B_n \left( x + \frac{a}{F} \right) \frac{t^n}{n!}.
\end{align*}
Comparing coefficients of $t^n$ on the both sides of the above equation
and yields
$B_n(Fx) = F^{n-1} \sum_{a=0}^{F-1} B_n \left(x + \frac{a}{F} \right)$.
$\Box$ \\



\subsubsection*{Supplement 15.19.1. (Multiplication Theorem for $\frac{1}{\exp(z) - 1}$)}
\addcontentsline{toc}{subsubsection}{Supplement 15.19.1. (Multiplication Theorem for $\frac{1}{\exp(z) - 1}$)}
\emph{$$\frac{1}{\exp(nz) - 1}
= \frac{1}{n} \sum_{k=0}^{n-1} \frac{1}{\exp(z + \frac{2 k \pi i}{n}) - 1}.$$} \\
\emph{Proof.}
Let $\zeta$ be one $n$-th root of unity.
Write $f(x) = x^n - 1 = \prod_{k=0}^{n-1}(x - \zeta^k)$.
By Lagrange interpolation,
\begin{align*}
\frac{1}{f(x)}
&=
\sum_{k=0}^{n-1} \frac{1}{f'(\zeta^k)} \cdot \frac{1}{x - \zeta^k} \\
\frac{1}{x^n - 1}
&= \sum_{k=0}^{n-1} \frac{1}{n \zeta^{-k}} \cdot \frac{1}{x - \zeta^k} \\
&= \frac{1}{n} \sum_{k=0}^{n-1} \frac{\zeta^k}{x - \zeta^k}.
\end{align*}
Let $x = \exp(z)$, $\zeta = \exp(-\frac{2 \pi i}{n})$.
$\Box$ \\



\subsubsection*{Supplement 15.19.2. (Multiplication theorem for $\cot z$)}
\addcontentsline{toc}{subsubsection}{Supplement 15.19.2. (Multiplication theorem for $\cot z$)}
\emph{$$\cot z = \frac{1}{n} \sum_{k=0}^{n-1} \cot\frac{z + k\pi}{n}.$$} \\

This equation yields
$x \cot x = 1 - 2 \sum_{n=1}^{\infty} \frac{x^2}{n^2 \pi^2 - x^2}$ again. \\

\emph{Proof.}
By Supplement 15.12.1,
\begin{align*}
\frac{1}{\exp(z) - 1}
&=
\frac{1}{n} \sum_{k=0}^{n-1} \frac{1}{\exp(\frac{z + 2 k \pi i}{n}) - 1}. \\
\frac{1}{\exp(2iz) - 1}
&=
\frac{1}{n} \sum_{k=0}^{n-1} \frac{1}{\exp(\frac{2i (z + k \pi)}{n}) - 1}.
\end{align*}
Notice that $\cot z = i + \frac{2i}{e^{2ix} - 1}$,
$\cot z = \frac{1}{n} \sum_{k=0}^{n-1} \cot\frac{z + k\pi}{n}$.
$\Box$ \\



\subsubsection*{Supplement 15.19.3. (Multiplication theorem for Gamma function)(Gauss's multiplication formula)}
\addcontentsline{toc}{subsubsection}{Supplement 15.19.3. (Multiplication theorem for Gamma function)(Gauss's multiplication formula)}
\emph{
$$\Gamma(z)
\Gamma\left( z+\frac{1}{k} \right)
\Gamma\left( z+\frac{2}{k} \right) \cdots
\Gamma\left( z+\frac{k-1}{k} \right)
= (2\pi)^{\frac{k-1}{2}} k^{\frac{1-2kz}{2}}
\Gamma\left( kz \right).$$} \\\\



%%%%%%%%%%%%%%%%%%%%%%%%%%%%%%%%%%%%%%%%%%%%%%%%%%%%%%%%%%%%%%%%%%%%%%%%%%%%%%%%



\subsubsection*{Exercise 15.20.}
\addcontentsline{toc}{subsubsection}{Exercise 15.20.}
\emph{Suppose $H(x)$ is a polynomial of degree $n$ with complex coefficients.
Suppose that for all integers $n$, $F > 0$ we have
$H(Fx) = F^{n-1}\sum_{a=0}^{F-1}H(x+\frac{a}{F})$.
Show that $H(x) = CB_n(x)$ for some constant $C$.
(Hint: Use Exercise 16 and induction on $n$.)} \\

Use induction on $n$ to show that $H(x) = CB_n(x)$
where $C$ is the leading coefficient of $H(x)$
(since the leading coefficient of every Bernoulli polynomial is $1$).

\begin{enumerate}
\item[(1)]
As $n = 1$, write $H(x) = C_1 x + C_0 \in \mathbb{C}$.
Then
\begin{align*}
H(Fx)
&= \sum_{a=0}^{F-1} H(x+\frac{a}{F}), \\
C_1 F x + C_0
&= \sum_{a=0}^{F-1} \left( C_1\left(x+\frac{a}{F}\right) + C_0\right) \\
&= C_1Fx + C_1 \cdot \frac{F-1}{2} + C_0F, \\
C_0 &= \frac{-1}{2} C_1
\end{align*}
if $F > 1$. That is, $H(x) = C_1 B_1(x)$ where $C = C_1$ is a constant.
In fact, $C$ is the leading coefficient of $H(x)$.
\item[(2)]
Assume that the conclusion holds for $n = k$.
As $n = k+1$, it suffices to show $f(x) = H(x) - CB_{k+1}(x) = 0$,
where $C$ is the leading coefficient of $H(x)$.
\item[(3)]
Differentiate $f(x) = H(x) - CB_{k+1}(x)$ and use Exercise 15.16,
$$f'(x) = H'(x) - C \cdot (k+1) \cdot B_k(x).$$
Might show $f'(x) = 0$ and then get that $H(x) - CB_{k+1}(x)$ is a constant.
\item[(4)]
Notice that the leading coefficient of $H'(x)$ is $C \cdot (k+1)$.
Besides, by differentiating $H(Fx) = F^{k}\sum_{a=0}^{F-1}H(x+\frac{a}{F})$,
\begin{align*}
H'(Fx) \cdot F
&= F^{k} \sum_{a=0}^{F-1} H'(x+\frac{a}{F}), \\
H'(Fx)
&= F^{k-1} \sum_{a=0}^{F-1} H'(x+\frac{a}{F}).
\end{align*}
By the induction hypothesis,
$H'(x) = (C \cdot (k+1)) B_k(x)$ since $H'(x)$ has degree $(k+1)-1 = k$.
Therefore, $f'(x) = 0$ or $f(x) = H(x) - CB_{k+1}(x) = A$ is a constant.
\item[(5)]
By $f(Fx) = H(Fx) - CB_{k+1}(Fx) = A$,
\begin{align*}
A
&= F^{k} \sum_{a=0}^{F-1} H \left( x+\frac{a}{F} \right)
- C F^{k} \sum_{a=0}^{F-1} B_{k+1} \left( x+\frac{a}{F} \right) \\
&= F^{k} \sum_{a=0}^{F-1} \left( H \left( x+\frac{a}{F} \right)
- C B_{k+1} \left( x+\frac{a}{F} \right) \right) \\
&= F^{k} \sum_{a=0}^{F-1} A \\
&= F^{k+1} A,
\end{align*}
or $(F^{k+1} - 1) A = 0$. For $F > 1$, $A = 0$.
That is, $H(x) = CB_{k+1}(x)$.
\end{enumerate}
By mathematical induction the result is established.
$\Box$ \\\\



%%%%%%%%%%%%%%%%%%%%%%%%%%%%%%%%%%%%%%%%%%%%%%%%%%%%%%%%%%%%%%%%%%%%%%%%%%%%%%%%



\subsubsection*{Exercise 15.21.}
\addcontentsline{toc}{subsubsection}{Exercise 15.21.}
\emph{Show $2^{n-1} B_n(\frac{1}{2}) = (1 - 2^{n-1})B_n$.} \\

The original identity $B_n(\frac{1}{2}) = (1 - 2^{n-1})B_n$ is wrong.
For $n = 2$, $B_2(x) = x^2 - x + \frac{1}{6}$ and thus
$-\frac{1}{12} = B_2(\frac{1}{2}) \neq (1 - 2^{2-1})B_2 = -\frac{1}{6}$. \\

\emph{Proof.}
Taking $F = 2$ in Exercise 15.19,
\begin{align*}
B_n(2x)
&= 2^{n-1} \sum_{a = 0}^{1} B_n\left( x + \frac{a}{2} \right) \\
&= 2^{n-1} B_n(x) + 2^{n-1} B_n\left( x + \frac{1}{2} \right).
\end{align*}
Let $x = 0$,
$$B_n(0) = 2^{n-1} B_n(0) + 2^{n-1} B_n\left( \frac{1}{2} \right), $$
So
$$2^{n-1} B_n\left( \frac{1}{2} \right)
= (1 - 2^{n-1}) B_n(0)
= (1 - 2^{n-1}) B_n.$$
$\Box$ \\\\



%%%%%%%%%%%%%%%%%%%%%%%%%%%%%%%%%%%%%%%%%%%%%%%%%%%%%%%%%%%%%%%%%%%%%%%%%%%%%%%%



\subsubsection*{Exercise 15.22.}
\addcontentsline{toc}{subsubsection}{Exercise 15.22.}
\emph{More generally, show that
$(1 - F^{n-1})B_n = F^{n-1} \sum_{a=1}^{F-1} B_n(\frac{a}{F})$.} \\

The original identity $(1 - F^{n-1})B_n = \sum_{a=1}^{F-1} B_n(\frac{a}{F})$
is wrong again. \\

\emph{Proof.}
Let $x = 0$ in Exercise 15.19,
$$B_n(0)
= F^{n-1} \sum_{a=0}^{F-1} B_n\left( \frac{a}{F} \right)
= F^{n-1} B_n(0) + F^{n-1} \sum_{a=1}^{F-1} B_n\left( \frac{a}{F} \right), $$
So
$$F^{n-1} \sum_{a=1}^{F-1} B_n\left( \frac{a}{F} \right)
= (1 - F^{n-1}) B_n(0)
= (1 - F^{n-1}) B_n.$$
$\Box$ \\\\



%%%%%%%%%%%%%%%%%%%%%%%%%%%%%%%%%%%%%%%%%%%%%%%%%%%%%%%%%%%%%%%%%%%%%%%%%%%%%%%%
%%%%%%%%%%%%%%%%%%%%%%%%%%%%%%%%%%%%%%%%%%%%%%%%%%%%%%%%%%%%%%%%%%%%%%%%%%%%%%%%



\end{document}