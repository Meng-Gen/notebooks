\documentclass{article}
\usepackage{amsfonts}
\usepackage{amssymb}
\parindent=0pt

\begin{document}

\textbf{\Large Chapter 2: Applications of Unique Factorization} \\\\

\textbf{Exercise.} If $\frac{a}{b} \in \mathbb{Z}_{p}$ is not a unit, prove that
$\frac{a}{b} + 1$ is a unit. \\

\emph{Proof:}

$\frac{a}{b} \in \mathbb{Z}_{p}$ is not a unit iff $p \mid a$ and $p \nmid b$.
Thus $p \nmid (a +b)$. That is, $\frac{a}{b} + 1 = \frac{a + b}{b} \in \mathbb{Z}_{p}$ is a unit.
$\Box$

\end{document}