\documentclass{article}
\usepackage{amsfonts}
\usepackage{amsmath}
\usepackage{amssymb}
\usepackage{hyperref}
\parindent=0pt

\begin{document}

\textbf{\Large Chapter 2: Applications of Unique Factorization} \\\\



\textbf{Exercise.}
\emph{If $\frac{a}{b} \in \mathbb{Z}_{p}$ is not a unit, prove that
$\frac{a}{b} + 1$ is a unit.} \\

\emph{Proof.}
$\frac{a}{b} \in \mathbb{Z}_{p}$ is not a unit iff $p \mid a$ and $p \nmid b$.
Thus $p \nmid (a +b)$. That is, $\frac{a}{b} + 1 = \frac{a + b}{b} \in \mathbb{Z}_{p}$ is a unit.
$\Box$ \\



\textbf{Exercise 2.6. ($p$-adic valuation.)}
\emph{For a rational number $r$ let $[r]$ be the largest integer less than or equal to $r$,
e.g., $[\frac{1}{2}] = 0$, $[2] = 2$, $[3 \frac{1}{3}] = 3$. Prove
$$\text{ord}_p n!
= \left[\frac{n}{p}\right] + \left[\frac{n}{p^2}\right] + \left[\frac{n}{p^3}\right] + \cdots.$$
}

Notice that $[\frac{n}{p}] + [\frac{n}{p^2}] + [\frac{n}{p^3}] + \cdots$ is a finite sum. \\

\emph{Proof.}
For any $k = 1, 2, ..., n$, we can express $k$ as $k = p^s t$
where $s = \text{ord}_p k$ is a non-negative integer and $(t, p) = 1$.
There are $[\frac{n}{p^a}]$ numbers such that $p^a \mid k$ for $a = 1, 2, ...$.
Therefore, there are $$\left[\frac{n}{p^a}\right] - \left[\frac{n}{p^{a+1}}\right]$$
numbers such that $\text{ord}_p k = a$ for $a = 1, 2, ...$. Hence,
\begin{align*}
\text{ord}_p n!
&= \left( \left[\frac{n}{p}\right] - \left[\frac{n}{p^2}\right] \right)
 + 2 \left( \left[\frac{n}{p^2}\right] - \left[\frac{n}{p^3}\right] \right)
 + 3 \left( \left[\frac{n}{p^3}\right] - \left[\frac{n}{p^4}\right] \right) + \cdots \\
&= \left[\frac{n}{p}\right] + \left[\frac{n}{p^2}\right] + \left[\frac{n}{p^3}\right] + \cdots.
\end{align*}
$\Box$ \\

\textbf{Supplement.} Related problems.
\begin{enumerate}
\item[(1)]
\emph{Prove that
$$\frac{(m + n)!}{m!n!}$$
is an integer for all non-negative integers $m$ and $n$.} \\

\emph{Proof.}
It is sufficient to show that
$$\text{ord}_p (m + n)! \geq \text{ord}_p m! + \text{ord}_p n!$$
for any prime $p$, or show that
$$\left[\frac{m + n}{p^k}\right]
\geq \left[\frac{m}{p^k}\right] + \left[\frac{n}{p^k}\right]$$
for any prime $p$ and $k \in \mathbb{Z}^+$ by Exercise 4.6, or show that
$$[x + y] \geq [x] + [y]$$
for any rational (or real) numbers $x$ and $y$.
It is trivial by considering that the sum of two fractional parts $\{x\} = x - [x]$
might be greater than or equal to $1$, so $[x + y] = [x] + [y]$ or $[x] + [y] + 1$.
$\Box$ \\

\emph{Note.}
$\frac{(m + n)!}{m!n!}$ is a binomial coefficient.
Similarly, a multinomial coefficient is
$$\frac{(n_1 + n_2 + \cdots + n_k)!}{n_1!n_2! \cdots n_k!}.$$
We can show that the multinomial coefficient is an integer
by using the above argument. \\

\item[(2)]
\emph{Prove that
$$\frac{(2m)!(2n)!}{m!n!(m + n)!}$$
is an integer for all non-negative integers $m$ and $n$.} \\

\emph{Proof.}
Similar to (1), it is sufficient to show that
$$[2x] + [2y] \geq [x] + [y] + [x + y]$$
for any rational (or real) numbers $x$ and $y$.
Notice that $[2x] = [x] + [x + \frac{1}{2}]$, and thus we might show that
$[x + \frac{1}{2}] + [y + \frac{1}{2}] \geq [x + y]$.
Again it is trivial and we omit the tedious calculation.
$\Box$ \\

\item[(3)]
\emph{Hermite's identity:
$[nx] = \sum_{k=0}^{n-1} [x + \frac{k}{n}]$ for $n \in \mathbb{Z}^+$.} \\

Let $n = 2$ and we can get $[2x] = [x] + [x + \frac{1}{2}]$ too. \\

\emph{Proof.}
Consider the function $f(x) = \sum_{k=0}^{n-1} [x + \frac{k}{n}] - [nx]$.
Notice that $f(x + \frac{1}{n}) = f(x)$. $f$ has period $\frac{1}{n}$.
It then suffices to prove that $f(x) = 0$ on $[0, \frac{1}{n})$.
But in this case, the integral part of each summand in $f$ is equal to $0$.
Therefore $f = 0$ on $\mathbb{R}$.
$\Box$ \\

\item[(4)]
\emph{Show
$$\frac{(5m)!(5n)!}{m! n! (3m + n)! (3n + m)!}$$
is an integer for all non-negative integers $m$ and $n$.} \\

Try to deduce the inequality $[5x] + [5y] \geq [x] + [y] + [3x + y] + [3y + x]$.

\end{enumerate}

\end{document}