\documentclass{article}
\usepackage{amsfonts}
\usepackage{amsmath}
\usepackage{amssymb}
\usepackage{hyperref}
\usepackage{mathrsfs}
\parindent=0pt

\def\upint{\mathchoice%
    {\mkern13mu\overline{\vphantom{\intop}\mkern7mu}\mkern-20mu}%
    {\mkern7mu\overline{\vphantom{\intop}\mkern7mu}\mkern-14mu}%
    {\mkern7mu\overline{\vphantom{\intop}\mkern7mu}\mkern-14mu}%
    {\mkern7mu\overline{\vphantom{\intop}\mkern7mu}\mkern-14mu}%
  \int}
\def\lowint{\mkern3mu\underline{\vphantom{\intop}\mkern7mu}\mkern-10mu\int}

\begin{document}

\textbf{\Large Chapter 5: Quadratic Reciprocity} \\\\



\emph{Author: Meng-Gen Tsai} \\
\emph{Email: plover@gmail.com} \\\\



%%%%%%%%%%%%%%%%%%%%%%%%%%%%%%%%%%%%%%%%%%%%%%%%%%%%%%%%%%%%%%%%%%%%%%%%%%%%%%%%



\textbf{Exercise 5.2.}
\emph{Show that the number of solutions to $x^2 \equiv a \: (p)$
is given by $1 + (a/p)$.} \\

$p$ is an odd prime. \\

\emph{Proof.}
\begin{enumerate}
\item[(1)]
If $x \equiv t \: (p)$ is a solution of the equation $x^2 \equiv a \: (p)$,
then $x \equiv -t \: (p)$ is also a solution.
Notice that $t \not\equiv -t \: (p)$ if $t \not\equiv 0 \: (p)$
by using the fact that $p$ is odd.
\item[(2)]
(Lemma 4.1.) Let $f(x) \in k[x]$, $k$ a field. Suppose that $\deg f(x) = n$.
Then $f$ has at most $n$ distinct roots.
\item[(3)]
If $a = 0$, then $x^2 \equiv 0 \: (p)$ has only one solution $x \equiv 0 \: (p)$,
or $1 + (a/p)$ solution (where $(a/p) = 0$ in this case).
\item[(4)]
If $a \neq 0$ is a quadratic residue mod $p$, then by (1)(2)
the equation $x^2 \equiv a \: (p)$ has exactly 2 solutions, or $1 + (a/p)$ solutions
(where $(a/p) = 1$ in this case).
\item[(5)]
If $a$ is not a quadratic residue mod $p$,
then there is no solutions of the equation $x^2 \equiv a \: (p)$,
or $1 + (a/p)$ solutions (where $(a/p) = -1$ in this case).
\end{enumerate}
By (3)(4)(5), in any case the number of solutions to $x^2 \equiv a \: (p)$
is given by $1 + (a/p)$.
$\Box$ \\\\



%%%%%%%%%%%%%%%%%%%%%%%%%%%%%%%%%%%%%%%%%%%%%%%%%%%%%%%%%%%%%%%%%%%%%%%%%%%%%%%%



\end{document}