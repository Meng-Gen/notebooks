\documentclass{article}
\usepackage{amsfonts}
\usepackage{amsmath}
\usepackage{amssymb}
\usepackage{hyperref}
\usepackage[none]{hyphenat}
\usepackage{mathrsfs}
\usepackage{physics}
\parindent=0pt

\def\upint{\mathchoice%
    {\mkern13mu\overline{\vphantom{\intop}\mkern7mu}\mkern-20mu}%
    {\mkern7mu\overline{\vphantom{\intop}\mkern7mu}\mkern-14mu}%
    {\mkern7mu\overline{\vphantom{\intop}\mkern7mu}\mkern-14mu}%
    {\mkern7mu\overline{\vphantom{\intop}\mkern7mu}\mkern-14mu}%
  \int}
\def\lowint{\mkern3mu\underline{\vphantom{\intop}\mkern7mu}\mkern-10mu\int}

\begin{document}



\textbf{\Large Chapter 3: Congruence} \\\\



\emph{Author: Meng-Gen Tsai} \\
\emph{Email: plover@gmail.com} \\\\



%%%%%%%%%%%%%%%%%%%%%%%%%%%%%%%%%%%%%%%%%%%%%%%%%%%%%%%%%%%%%%%%%%%%%%%%%%%%%%%%



\textbf{Exercise 3.12.}
\emph{Let $${p \choose k} = \frac{p!}{k!(p-k)!}$$
be a binomial coefficient, and suppose that $p$ is a prime.
If $1 \leq k \leq p-1$, show that $p$ divides ${p \choose k}$.
Deduce $(a+1)^p \equiv a^p + 1 \pmod{p}$.} \\

\emph{Proof.}
\begin{enumerate}
  \item[(1)]
  If $1 \leq k \leq p-1$, then $p \nmid k!$ and $p \nmid (p-k)!$
  since $p$ is a prime number.
  \item[(2)]
  Write $a = \frac{p!}{k!(p-k)!} \in \mathbb{Z}$.
  \begin{align*}
  a = \frac{p!}{k!(p-k)!}
  &\Longleftrightarrow
  p! = ak!(p-k)! \\
  &\Longrightarrow
  p \mid p! \text{ or } p \mid ak!(p-k)! \\
  &\Longrightarrow
  p \mid a
    &((1))
  \end{align*}
  Hence $p$ divides ${p \choose k}$ if $1 \leq k \leq p-1$.
  \item[(3)]
  \begin{align*}
  (a+1)^p
  &\equiv \sum_{k=0}^{p} {p \choose k} a^k \\
  &\equiv 1 + \left( \sum_{k=1}^{p-1} {p \choose k} a^k \right) + a^p \\
  &\equiv 1 + a^p \\
  &\equiv a^p + 1 \pmod{p}.
  \end{align*}
\end{enumerate}
$\Box$\\\\



%%%%%%%%%%%%%%%%%%%%%%%%%%%%%%%%%%%%%%%%%%%%%%%%%%%%%%%%%%%%%%%%%%%%%%%%%%%%%%%%
%%%%%%%%%%%%%%%%%%%%%%%%%%%%%%%%%%%%%%%%%%%%%%%%%%%%%%%%%%%%%%%%%%%%%%%%%%%%%%%%



\end{document}