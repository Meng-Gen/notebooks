\documentclass{article}
\usepackage{amsfonts}
\usepackage{amsmath}
\usepackage{amssymb}
\usepackage{hyperref}
\usepackage{mathrsfs}
\parindent=0pt

\def\upint{\mathchoice%
    {\mkern13mu\overline{\vphantom{\intop}\mkern7mu}\mkern-20mu}%
    {\mkern7mu\overline{\vphantom{\intop}\mkern7mu}\mkern-14mu}%
    {\mkern7mu\overline{\vphantom{\intop}\mkern7mu}\mkern-14mu}%
    {\mkern7mu\overline{\vphantom{\intop}\mkern7mu}\mkern-14mu}%
  \int}
\def\lowint{\mkern3mu\underline{\vphantom{\intop}\mkern7mu}\mkern-10mu\int}

\begin{document}

\textbf{\Large Chapter 15: Bernoulli Numbers} \\\\



\emph{Author: Meng-Gen Tsai} \\
\emph{Email: plover@gmail.com} \\\\



\textbf{Supplement.}
\emph{Prove that $$x \cot x = 1 - 2 \sum_{n=1}^{\infty} \frac{x^2}{n^2 \pi^2 - x^2}.$$} \\

\emph{Proof.}
\begin{enumerate}
\item[(1)]
\emph{Show that $$\cot x
= \frac{1}{2^n} \sum_{k=0}^{2^{n} - 1} \cot \frac{x + k\pi}{2^n}$$
for all integers $n \geq 1$.}
Notice that
\begin{align*}
\cot(x + \pi) &= \cot x, \\
\cot\left( x + \frac{\pi}{2} \right) &= -\tan x, \\
\cot x &= \frac{1}{2} \left( \cot\frac{x}{2} - \tan\frac{x}{2} \right).
\end{align*}
Use mathematical induction.
The case $n = 1$ is the same as the note.
Assume the case $n = m$ holds.
For $n = m+1$,
\begin{align*}
\sum_{k=0}^{2^{m+1} - 1} \cot \frac{x + k\pi}{2^{m+1}}
&= \sum_{k=0}^{2^{m} - 1} \cot \frac{x + k\pi}{2^{m+1}}
+ \sum_{k=2^{m}}^{2^{m+1} - 1} \cot \frac{x + k\pi}{2^{m+1}} \\
&= \sum_{k=0}^{2^{m} - 1} \cot \frac{x + k\pi}{2^{m+1}}
+ \sum_{k=0}^{2^{m} - 1} \cot \frac{x + (2^{m} + k)\pi}{2^{m+1}} \\
&= \sum_{k=0}^{2^{m} - 1} \cot \frac{x + k\pi}{2^{m+1}}
+ \sum_{k=0}^{2^{m} - 1} \cot \left( \frac{x + k\pi}{2^{m+1}} + \frac{\pi}{2} \right) \\
&= \sum_{k=0}^{2^{m} - 1}
\left( \cot \frac{x + k\pi}{2^{m+1}} - \tan \frac{x + k\pi}{2^{m+1}} \right) \\
&= \sum_{k=0}^{2^{m} - 1}
\left( \cot \frac{x + k\pi}{2^{m+1}} - \tan \frac{x + k\pi}{2^{m+1}} \right) \\
&= 2 \sum_{k=0}^{2^{m} - 1} \cot \frac{x + k\pi}{2^{m}}.
\end{align*}
Therefore,
\begin{align*}
\frac{1}{2^{m+1}} \sum_{k=0}^{2^{m+1} - 1} \cot \frac{x + k\pi}{2^{m+1}}
&= \frac{1}{2^{m+1}} \cdot 2 \sum_{k=0}^{2^{m} - 1} \cot \frac{x + k\pi}{2^{m}} \\
&= \frac{1}{2^{m}} \sum_{k=0}^{2^{m} - 1} \cot \frac{x + k\pi}{2^{m}} \\
&= \cot x.
\end{align*}
\item[(2)]
By rearranging the index of summation of the identity in (1), we have
$$x \cot x
= \frac{x}{2^n} \cot \frac{x}{2^n} - \frac{x}{2^n} \tan \frac{x}{2^n}
+ \sum_{k=1}^{2^{n-1} - 1} \frac{x}{2^n}
\left( \cot \frac{x + k\pi}{2^n} + \cot \frac{x - k\pi}{2^n} \right)$$
for all integers $n \geq 1$.
\item[(3)]
Notice that $\lim_{x \rightarrow 0} x \cot x = 1$.
Let $n \rightarrow \infty$, the result is established.
\end{enumerate}
$\Box$ \\\\



\textbf{Exercise 15.6.}
\emph{For $m \geq 3$, show $|B_{2m+2}| > |B_{2m}|$. (Hint: Use Theorem 2.)} \\

\emph{Proof.}
By Theorem 2,
$$2 \zeta(2m) = (-1)^{m+1} \frac{(2\pi)^{2m}}{(2m)!} B_{2m}.$$
Thus,
$$\frac{|B_{2m+2}|}{|B_{2m}|} = \frac{\zeta(2m+2)}{\zeta(2m)} \cdot \frac{(2m+2)(2m+1)}{(2\pi)^2}.$$
\begin{enumerate}
\item[(1)]
The first term is
$$\frac{\zeta(2m+2)}{\zeta(2m)} > \frac{1}{\zeta(6)} = \frac{945}{\pi^6}.$$
\item[(2)]
The second term is
$$\frac{(2m+2)(2m+1)}{(2\pi)^2} \geq \frac{8 \cdot 7}{(2\pi)^2} = \frac{14}{\pi^2}.$$
\end{enumerate}
By (1)(2),
$$\frac{|B_{2m+2}|}{|B_{2m}|}
> \frac{945}{\pi^6} \cdot \frac{14}{\pi^2} > 1,$$
or $|B_{2m+2}| > |B_{2m}|$.
$\Box$ \\\\



\end{document}