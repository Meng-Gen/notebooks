\documentclass{article}
\usepackage{amsfonts}
\usepackage{amsmath}
\usepackage{amssymb}
\usepackage{hyperref}
\usepackage{mathrsfs}
\parindent=0pt

\def\upint{\mathchoice%
    {\mkern13mu\overline{\vphantom{\intop}\mkern7mu}\mkern-20mu}%
    {\mkern7mu\overline{\vphantom{\intop}\mkern7mu}\mkern-14mu}%
    {\mkern7mu\overline{\vphantom{\intop}\mkern7mu}\mkern-14mu}%
    {\mkern7mu\overline{\vphantom{\intop}\mkern7mu}\mkern-14mu}%
  \int}
\def\lowint{\mkern3mu\underline{\vphantom{\intop}\mkern7mu}\mkern-10mu\int}

\begin{document}

\textbf{\Large Chapter 15: Bernoulli Numbers} \\\\



\emph{Author: Meng-Gen Tsai} \\
\emph{Email: plover@gmail.com} \\\\



% https://dms.umontreal.ca/~mlalin/bernoulli.pdf
% http://cds.iisc.ac.in/faculty/amohanty/SE288/bn.pdf
% http://people.math.sfu.ca/~cbm/aands/abramowitz_and_stegun.pdf



\textbf{Supplement.} Equation (4) on page 231.
\emph{Prove that $$x \cot x = 1 - 2 \sum_{n=1}^{\infty} \frac{x^2}{n^2 \pi^2 - x^2}.$$} \\

\emph{Proof (Exercise 6.73 in the book Graham, Knuth and Patashnik,
Concrete Mathematics, Second Edition).}

\begin{enumerate}
\item[(1)]
\emph{Show that $$\cot x
= \frac{1}{2^n} \sum_{k=0}^{2^{n} - 1} \cot \frac{x + k\pi}{2^n}$$
for all integers $n \geq 1$.}
Notice that
\begin{align*}
\cot(x + \pi) &= \cot x, \\
\cot\left( x + \frac{\pi}{2} \right) &= -\tan x, \\
\cot x &= \frac{1}{2} \left( \cot\frac{x}{2} - \tan\frac{x}{2} \right).
\end{align*}
Use mathematical induction.
The case $n = 1$ is the same as the note.
Assume the case $n = m$ holds.
For $n = m+1$,
\begin{align*}
\sum_{k=0}^{2^{m+1} - 1} \cot \frac{x + k\pi}{2^{m+1}}
&= \sum_{k=0}^{2^{m} - 1} \cot \frac{x + k\pi}{2^{m+1}}
+ \sum_{k=2^{m}}^{2^{m+1} - 1} \cot \frac{x + k\pi}{2^{m+1}} \\
&= \sum_{k=0}^{2^{m} - 1} \cot \frac{x + k\pi}{2^{m+1}}
+ \sum_{k=0}^{2^{m} - 1} \cot \frac{x + (2^{m} + k)\pi}{2^{m+1}} \\
&= \sum_{k=0}^{2^{m} - 1} \cot \frac{x + k\pi}{2^{m+1}}
+ \sum_{k=0}^{2^{m} - 1} \cot \left( \frac{x + k\pi}{2^{m+1}} + \frac{\pi}{2} \right) \\
&= \sum_{k=0}^{2^{m} - 1}
\left( \cot \frac{x + k\pi}{2^{m+1}} - \tan \frac{x + k\pi}{2^{m+1}} \right) \\
&= \sum_{k=0}^{2^{m} - 1}
\left( \cot \frac{x + k\pi}{2^{m+1}} - \tan \frac{x + k\pi}{2^{m+1}} \right) \\
&= 2 \sum_{k=0}^{2^{m} - 1} \cot \frac{x + k\pi}{2^{m}}.
\end{align*}
Therefore,
\begin{align*}
\frac{1}{2^{m+1}} \sum_{k=0}^{2^{m+1} - 1} \cot \frac{x + k\pi}{2^{m+1}}
&= \frac{1}{2^{m+1}} \cdot 2 \sum_{k=0}^{2^{m} - 1} \cot \frac{x + k\pi}{2^{m}} \\
&= \frac{1}{2^{m}} \sum_{k=0}^{2^{m} - 1} \cot \frac{x + k\pi}{2^{m}} \\
&= \cot x.
\end{align*}
\item[(2)]
By rearranging the index of summation of the identity in (1), we have
$$x \cot x
= \frac{x}{2^n} \cot \frac{x}{2^n} - \frac{x}{2^n} \tan \frac{x}{2^n}
+ \sum_{k=1}^{2^{n-1} - 1} \frac{x}{2^n}
\left( \cot \frac{x + k\pi}{2^n} + \cot \frac{x - k\pi}{2^n} \right)$$
for all integers $n \geq 1$.
\item[(3)]
Notice that $\lim_{x \rightarrow 0} x \cot x = 1$.
Let $n \rightarrow \infty$, the result is established.
\end{enumerate}
$\Box$ \\\\



\textbf{Exercise 15.1.}
\emph{Using the definition of the Bernoulli number show
$B_{10} = \frac{5}{66}$ and $B_{12} = -\frac{691}{2730}$.} \\

\emph{Proof.}
\begin{enumerate}
\item[(1)]
It is known that
$B_1 = -\frac{1}{2}$,
$B_2 = \frac{1}{6}$,
$B_4 = -\frac{1}{30}$,
$B_6 = \frac{1}{42}$,
and $B_m = 0$ for odd $m > 1$.
\item[(2)]
Recall the implicit recurrence relation,
$$\sum_{k = 0}^{m} {m+1 \choose k} B_k = [m = 0],$$
where $[m = 0]$ is the Iverson brackets which is equal to
the Kronecker delta $\delta_{m0}$.
\item[(3)]
So
\begin{align*}
0 &= 1 + 9 B_1 + 36 B_2 + 84 B_3 + 126 B_4 + 126 B_5 + 84 B_6 + 36 B_7 + 9 B_8, \\
0 &= 1 + 9 B_1 + 36 B_2 + 126 B_4 + 84 B_6 + 9 B_8, \\
0 &= 1 + 9 \left( -\frac{1}{2} \right)
+ 36 \left( \frac{1}{6} \right)
+ 126 \left( -\frac{1}{30} \right)
+ 84 \left( \frac{1}{42} \right)
+ 9 B_8, \\
0 &= \frac{3}{10} + 9B_8,
\end{align*}
Thus $B_8 = -\frac{1}{30}$. \\
\item[(4)]
Again,
\begin{align*}
0 =& 1 + 11 B_1 + 55 B_2 + 165 B_3 + 330 B_4 + 462 B_5 + 462 B_6 + \\
   & 330 B_7 + 165 B_8 + 55 B_9 + 11 B_{10}, \\
0 =& 1 + 11 B_1 + 55 B_2 + 330 B_4 + 462 B_6 + 165 B_8 + 11 B_{10}, \\
0 =& 1 + 11 \left( -\frac{1}{2} \right) +
     55 \left( \frac{1}{6} \right) +
     330 \left( -\frac{1}{30} \right) +
     462 \left( \frac{1}{42} \right) + \\
   & 165 \left( -\frac{1}{30} \right) +
     11 B_{10}, \\
0 =& -\frac{5}{6} + 11 B_{10},
\end{align*}
Thus $B_{10} = \frac{5}{66}$. \\
\item[(4)]
Finally,
\begin{align*}
0 =& 1 + 13 B_1 + 78 B_2 + 286 B_3 + 715 B_4 + 1287 B_5 + 1716 B_6 + \\
   & 1716 B_7 + 1287 B_8 + 715 B_9 + 286 B_{10} + 78 B_{11} + 13 B_{12}, \\
0 =& 1 + 13 B_1 + 78 B_2 + 715 B_4 + 1716 B_6 + 1287 B_8 + 286 B_{10} + 13 B_{12}, \\
0 =& 1 + 13 \left( -\frac{1}{2} \right) +
     78 \left( \frac{1}{6} \right) +
     715 \left( -\frac{1}{30} \right) +
     1716 \left( \frac{1}{42} \right) + \\
   & 1287 \left( -\frac{1}{30} \right) +
     286 \left( \frac{5}{66} \right) +
     13 B_{12}, \\
0 =& \frac{691}{210} + 13 B_{12},
\end{align*}
Thus $B_{12} = -\frac{691}{2730}$.
\end{enumerate}
$\Box$\\\\



\textbf{Exercise 15.2.}
\emph{If $a \in \mathbb{Z}$,
show $a(a^m - 1)B_m \in \mathbb{Z}$ for all $m > 0$.} \\

\emph{Proof.}
\begin{enumerate}
\item[(1)] \emph{Trivial cases.}
If $m = 1$, $a(a - 1) B_1 = -\frac{1}{2} a(a - 1) \in \mathbb{Z}$ for any $a \in \mathbb{Z}$.
For odd $m > 1$, $B_m = 0$ or $a(a^m - 1)B_m = 0 \in \mathbb{Z}$ (Proposition 15.1.1). \\
\item[(2)] \emph{Consider that $m > 1$ and even.}
By Theorem 3,
$$B_{2m} + \sum_{p-1 \mid 2m} \frac{1}{p} \in \mathbb{Z}$$
where the sum is over all primes $p$ such that $p-1 \mid 2m$.
So it suffices to show
$$a(a^{2m} - 1) \sum_{p-1 \mid 2m} \frac{1}{p} \in \mathbb{Z}, $$
or
$$a(a^{2m} - 1) \frac{1}{p} \in \mathbb{Z}$$
for any $a \in \mathbb{Z}$ and any prime $p$ such that $p-1 \mid 2m$.
\item[(3)] \emph{Consider all possible $a$.}
If $p \mid a$, it is trivial.
If $p \nmid a$, $a^{p - 1} \equiv 1 \: (p)$ by Fermat's Little Theorem,
or $a^{2m} \equiv 1 \: (p)$ by $p-1 \mid 2m$.
In any cases, $a(a^{2m} - 1)\frac{1}{p} \in \mathbb{Z}$.
\end{enumerate}
$\Box$ \\\\



\textbf{Exercise 15.6.}
\emph{For $m \geq 3$, show $|B_{2m+2}| > |B_{2m}|$. (Hint: Use Theorem 2.)} \\

\emph{Proof.}
By Theorem 2,
$$2 \zeta(2m) = (-1)^{m+1} \frac{(2\pi)^{2m}}{(2m)!} B_{2m}.$$
Thus,
$$\frac{|B_{2m+2}|}{|B_{2m}|}
= \frac{\zeta(2m+2)(2m+2)(2m+1)}{\zeta(2m)(2\pi)^2}
> \frac{1 \cdot 8 \cdot 7}{\zeta(6) \cdot (2\pi)^2}
= \frac{13230}{\pi^8}
> 1,$$
or $|B_{2m+2}| > |B_{2m}|$.
$\Box$ \\\\



\textbf{Exercise 15.8.}
\emph{Consider the power series expansion of $\tan x$ about the origin;
$$\sum_{k=1}^{\infty} T_k \frac{x^{2k - 1}}{(2k - 1)!}.$$
Show $$T_k = (-1)^{k-1} \frac{B_{2k}}{2k} (2^{2k} - 1) 2^{2k}.$$
Note that $T_k \in \mathbb{Z}$ for all $k$ by Exercise 3.} \\

\emph{Proof.}
\begin{enumerate}
\item[(1)]
By the equation (6) on page 232,
$$x \cot x = 1 + \sum_{k=2}^{\infty} B_k \frac{(2ix)^k}{k!}.$$
Since $B_k = 0$ for odd $k > 1$,
$$x \cot x
= 1 + \sum_{k=1}^{\infty} B_{2k} \frac{(2ix)^{2k}}{(2k)!}
= 1 + \sum_{k=1}^{\infty} \frac{(-1)^k 2^{2k}B_{2k}}{(2k)!} x^{2k},$$
or
$$\cot x
= \frac{1}{x} + \sum_{k=1}^{\infty} \frac{(-1)^k 2^{2k}B_{2k}}{(2k)!} x^{2k-1}.$$
Combine the first term $\frac{1}{x}$ into the summation,
$$\cot x = \sum_{k=0}^{\infty} \frac{(-1)^k 2^{2k}B_{2k}}{(2k)!} x^{2k-1}.$$
\item[(2)]
Note that $\tan x = \cot x - 2 \cot(2x)$.
By (1),
\begin{align*}
\tan x
&= \sum_{k=0}^{\infty} \frac{(-1)^k 2^{2k}B_{2k}}{(2k)!} x^{2k-1}
- 2 \sum_{k=0}^{\infty} \frac{(-1)^k 2^{2k}B_{2k}}{(2k)!} (2x)^{2k-1} \\
&= \sum_{k=0}^{\infty} \frac{(-1)^k (1 - 2^{2k}) 2^{2k} B_{2k}}{(2k)!} x^{2k-1} \\
&= \sum_{k=1}^{\infty} \frac{(-1)^k (1 - 2^{2k}) 2^{2k} B_{2k}}{(2k)!} x^{2k-1}.
\end{align*}
Write $T_k = (-1)^{k-1} (2^{2k} - 1) 2^{2k} \frac{B_{2k}}{2k}$.
Therefore, $\tan x = \sum_{k=1}^{\infty} T_k \frac{x^{2k - 1}}{(2k - 1)!}$.
\end{enumerate}
By Exercise 15.3, $(2^{2k} - 1) 2^{2k} \frac{B_{2k}}{2k} \in \mathbb{Z}$,
or $T_k \in \mathbb{Z}$ for all $k$.
$\Box$ \\\\



\textbf{Exercise 15.12.}
\emph{Recall the definition of the Bernoulli polynomials;
$$B_m(x) = \sum_{k=0}^{m} {m \choose k} B_k x^{m-k}.$$
Show that
$$\frac{te^{tx}}{e^t - 1} = \sum_{m=0}^{\infty} B_m(x) \frac{t^m}{m!}.$$} \\

\emph{Proof.}
By Lemma 1,
$$\frac{t}{e^t - 1} = \sum_{m=0}^{\infty} B_m \frac{t^m}{m!}.$$
So
$$
\frac{te^{tx}}{e^t - 1}
=
\left( \sum_{m=0}^{\infty} B_m \frac{t^m}{m!} \right)
\left( \sum_{m=0}^{\infty} x^m \frac{t^m}{m!} \right).$$
Write $\frac{te^{tx}}{e^t - 1} = \sum_{m=0}^{\infty} b_m(x) \frac{t^m}{m!}$
and we want to check if $b_m(x) = B_m(x)$ or not.
The result is established if $b_m(x) = B_m(x)$ holds.
Equating coefficients of $t^m$ gives
\begin{align*}
\frac{b_m(x)}{m!}
&= \sum_{k=0}^{m} \frac{B_k x^{m-k}}{k!(m-k)!}, \\
b_m(x)
&= \sum_{k=0}^{m} \frac{m!}{k!(m-k)!} B_k x^{m-k} \\
&= \sum_{k=0}^{m} {m \choose k} B_k x^{m-k} \\
&= B_m(x).
\end{align*}
$\Box$ \\\\



\textbf{Exercise 15.13.}
\emph{Show $B_m(x+1) - B_m(x) = mx^{m-1}$.} \\

\emph{Proof.}
Let $f(t, x) = \frac{te^{tx}}{e^t - 1}$.
\begin{enumerate}
\item[(1)]
$$f(t, x+1) - f(t, x)
= \frac{te^{t(x+1)}}{e^t - 1} - \frac{te^{tx}}{e^t - 1}
= te^{tx}.$$
Expand $te^{tx}$ in a power series about the origin as follows
\begin{align*}
te^{tx}
&= t \sum_{m=0}^{\infty} x^m \frac{t^m}{m!} \\
&= \sum_{m=0}^{\infty} x^m \frac{t^{m+1}}{m!} \\
&= \sum_{m=1}^{\infty} x^{m-1} \frac{t^m}{(m-1)!} \\
&= \sum_{m=1}^{\infty} mx^{m-1} \frac{t^m}{m!} \\
&= \sum_{m=0}^{\infty} mx^{m-1} \frac{t^m}{m!}.
\end{align*}
So,
$$f(t, x+1) - f(t, x) = \sum_{m=0}^{\infty} mx^{m-1} \frac{t^m}{m!}.$$
\item[(2)]
By Exercise 15.12,
\begin{align*}
f(t, x+1) - f(t, x)
&= \sum_{m=0}^{\infty} B_m(x+1) \frac{t^m}{m!}
- \sum_{m=0}^{\infty} B_m(x) \frac{t^m}{m!} \\
&= \sum_{m=0}^{\infty} (B_m(x+1) - B_m(x)) \frac{t^m}{m!}.
\end{align*}
\end{enumerate}
By (1)(2), comparing coefficients of $t^m$ yields
$$mx^{m-1} = B_m(x+1) - B_m(x).$$
$\Box$ \\\\



\textbf{Exercise 15.14.}
\emph{Use Exercise 13 to give a new proof of Theorem 1:
$$S_m(n) = \frac{1}{m+1}(B_{m+1}(n) - B_{m+1}).$$}

\emph{Proof.}
By Exercise 13,
$$B_{m+1}(k) - B_{m+1}(k-1) = (m+1)(k-1)^m$$
for any $k$.
So,
\begin{align*}
\sum_{k=1}^{n} (B_{m+1}(k) - B_{m+1}(k-1))
&= \sum_{k=1}^{n} (m+1)(k-1)^m, \\
B_{m+1}(n) - B_{m+1}(0)
&= (m+1)S_m(n).
\end{align*}
Note that $B_{m+1}(0) = B_{m+1}$ for any $m$.
So Theorem 1 is established by a new proof.
$\Box$ \\\\


\textbf{Exercise 15.15.}
\emph{Suppose $f(x) = \sum_{k=0}^{n} a_k x^k$
be a polynomial with complex coefficients.
Use Exercise 13 to find a polynomial $F(x)$ such that
$F(x+1) - F(x) = f(x)$.} \\

\emph{Proof.}
By Exercise 15.13,
$$x^k = \frac{1}{k+1}(B_{k+1}(x+1) - B_{k+1}(x))$$
for $k \geq 0$.
Thus,
\begin{align*}
f(x)
&= \sum_{k=0}^{n} a_k x^k \\
&= \sum_{k=0}^{n} a_k \cdot \frac{1}{k+1} (B_{k+1}(x+1) - B_{k+1}(x)) \\
&= \sum_{k=0}^{n} \frac{a_k}{k+1} B_{k+1}(x+1) - \sum_{k=0}^{n} \frac{a_k}{k+1} B_{k+1}(x).
\end{align*}
Let
$$F(x) = \sum_{k=0}^{n} \frac{a_k}{k+1} B_{k+1}(x),$$
and we get $f(x) = F(x+1) - F(x)$.
$\Box$ \\\\



\textbf{Exercise 15.16.}
\emph{For $n \geq 1$, show $\frac{d}{dx}B_n(x) = nB_{n-1}(x)$.} \\

\emph{Proof.}
For $n \geq 1$,
$$\frac{d}{dx}B_n(x)
= \sum_{k=0}^{n}(n - k) {n \choose k} B_k x^{n - k - 1}
= \sum_{k=0}^{n-1}(n - k) {n \choose k} B_k x^{n - k - 1}.$$
Note that
$$(n-k) {n \choose k} = n {n-1 \choose k}.$$
So
\begin{align*}
\frac{d}{dx}B_n(x)
&= \sum_{k=0}^{n-1} n {n-1 \choose k} B_k x^{n - k - 1} \\
&= n \sum_{k=0}^{n-1} {n-1 \choose k} B_k x^{n - k - 1} \\
&= n B_{n-1}(x).
\end{align*}
$\Box$ \\\\



\textbf{Exercise 15.17.}
\emph{Show $B_n(1-x) = (-1)^n B_n(x)$.} \\

\emph{Proof.}
Let $f(t, x) = \frac{te^{tx}}{e^t - 1}$.
\begin{enumerate}
\item[(1)]
\emph{$f(t, 1-x) = f(-t, x)$.}
$$f(t, 1-x)
= \frac{te^{t(1-x)}}{e^t - 1}
= e^t \cdot \frac{te^{-tx}}{e^t - 1}
= \frac{-te^{-tx}}{e^{-t} - 1} = f(-t, x).$$
\item[(2)]
By Exercise 15.12,
\begin{align*}
f(t, 1-x)
&= \sum_{n=0}^{\infty} B_n(1-x) \frac{t^n}{n!} \\
f(-t, x)
&= \sum_{n=0}^{\infty} (-1)^n B_n(x) \frac{t^n}{n!}.
\end{align*}
By (1), comparing coefficients of $t^n$ yields
$B_n(1-x) = (-1)^n B_n(x)$.
\end{enumerate}
$\Box$ \\\\



\textbf{Exercise 15.18.}
\emph{Use Exercise 13 and 17 to give a new proof that
$B_n = 0$ for $n$ odd and $n > 1$.} \\

\emph{Proof.}
\begin{enumerate}
\item[(1)]
\emph{$B_m(1) - B_m(0) = 0$ for any $m > 1$.}
Taking $x = 0$ in Exercise 15.13 and keeping $m - 1 > 0$ or $m > 1$.
\item[(2)]
\emph{$B_m(1) = -B_m(0)$ for any odd $m$.}
Taking $x = 0$ in Exercise 15.17 and keeping $m$ is odd.
\begin{align*}
f(t, 1-x)
&= \sum_{n=0}^{\infty} B_n(1-x) \frac{t^n}{n!} \\
f(-t, x)
&= \sum_{n=0}^{\infty} (-1)^n B_n(x) \frac{t^n}{n!}.
\end{align*}
\end{enumerate}
By (1)(2), for $m$ odd and $m > 1$, $B_m(0) = 0$ or $B_m = 0$.
$\Box$ \\\\



\textbf{Exercise 15.19 (Multiplication theorem for Bernoulli polynomial).}
\emph{Suppose $n$ and $F$ are integers and $n, F > 0$. Show that
$$B_n(Fx) = F^{n-1} \sum_{a=0}^{F-1} B_n \left(x + \frac{a}{F} \right).$$
(Hint: Use Exercise 12.)} \\

\emph{Proof.}
By $x^n - y^n = (x - y)(x^{n-1} + x^{n-2}y + \cdots + y^{n-1})$ (Exercise 1.24),
$$e^{Ft} - 1
= (e^t - 1)(1 + e^t + e^{2t} + \cdots + e^{(F-1)t})
= (e^t - 1) \sum_{a=0}^{F-1} e^{at}.$$
So,
\begin{align*}
\frac{1}{e^t - 1}
&= \frac{1}{e^{Ft} - 1} \sum_{a=0}^{F-1} e^{at}, \\
\frac{te^{tFx}}{e^t - 1}
&= \frac{te^{tFx}}{e^{Ft} - 1} \sum_{a=0}^{F-1} e^{at} \\
&= \sum_{a=0}^{F-1} \frac{te^{(Fx+a)t}}{e^{Ft} - 1} \\
&= \sum_{a=0}^{F-1} \frac{te^{(Fx+a)t}}{e^{Ft} - 1} \\
&= \sum_{a=0}^{F-1} F^{-1} \frac{(Ft) e^{(x + \frac{a}{F})(Ft)}}{e^{Ft} - 1}. \\
\end{align*}
By Exercise 15.12,
\begin{align*}
\sum_{n=0}^{\infty} B_n(Fx) \frac{t^n}{n!}
&= \sum_{a=0}^{F-1} F^{-1}
\sum_{n=0}^{\infty} B_n \left( x + \frac{a}{F} \right) \frac{(Ft)^n}{n!} \\
&= \sum_{n=0}^{\infty} \sum_{a=0}^{F-1} F^{-1}
B_n \left( x + \frac{a}{F} \right) \frac{(Ft)^n}{n!} \\
&= \sum_{n=0}^{\infty} \sum_{a=0}^{F-1} F^{n-1}
B_n \left( x + \frac{a}{F} \right) \frac{t^n}{n!}.
\end{align*}
Comparing coefficients of $t^n$ on the both sides of the above equation
and yields
$B_n(Fx) = F^{n-1} \sum_{a=0}^{F-1} B_n \left(x + \frac{a}{F} \right)$.
$\Box$ \\

\textbf{Supplement 15.12.1 (Multiplication Theorem for $\frac{1}{\exp(z) - 1}$).}
\emph{$$\frac{1}{\exp(nz) - 1}
= \frac{1}{n} \sum_{k=0}^{n-1} \frac{1}{\exp(z + \frac{2 k \pi i}{n}) - 1}.$$} \\
\emph{Proof.}
Let $\zeta$ be one $n$-th root of unity.
Write $f(x) = x^n - 1 = \prod_{k=0}^{n-1}(x - \zeta^k)$.
By Lagrange interpolation,
\begin{align*}
\frac{1}{f(x)}
&=
\sum_{k=0}^{n-1} \frac{1}{f'(\zeta^k)} \cdot \frac{1}{x - \zeta^k} \\
\frac{1}{x^n - 1}
&= \sum_{k=0}^{n-1} \frac{1}{n \zeta^{-k}} \cdot \frac{1}{x - \zeta^k} \\
&= \frac{1}{n} \sum_{k=0}^{n-1} \frac{\zeta^k}{x - \zeta^k}.
\end{align*}
Let $x = \exp(z)$, $\zeta = \exp(-\frac{2 \pi i}{n})$.
$\Box$ \\

\textbf{Supplement 15.12.2 (Multiplication theorem for $\cot z$.)}
\emph{$$\cot z = \frac{1}{n} \sum_{k=0}^{n-1} \cot\frac{z + k\pi}{n}.$$} \\

This equation yields
$x \cot x = 1 - 2 \sum_{n=1}^{\infty} \frac{x^2}{n^2 \pi^2 - x^2}$ again. \\

\emph{Proof.}
By Supplement 15.12.1,
\begin{align*}
\frac{1}{\exp(z) - 1}
&=
\frac{1}{n} \sum_{k=0}^{n-1} \frac{1}{\exp(\frac{z + 2 k \pi i}{n}) - 1}. \\
\frac{1}{\exp(2iz) - 1}
&=
\frac{1}{n} \sum_{k=0}^{n-1} \frac{1}{\exp(\frac{2i (z + k \pi)}{n}) - 1}.
\end{align*}
Notice that $\cot z = i + \frac{2i}{e^{2ix} - 1}$,
$\cot z = \frac{1}{n} \sum_{k=0}^{n-1} \cot\frac{z + k\pi}{n}$.
$\Box$ \\

\textbf{Supplement 15.12.3
(Multiplication theorem for Gamma function)
(Gauss's multiplication formula).} \\

\emph{
$$\Gamma(z)
\Gamma\left( z+\frac{1}{k} \right)
\Gamma\left( z+\frac{2}{k} \right) \cdots
\Gamma\left( z+\frac{k-1}{k} \right)
= (2\pi)^{\frac{k-1}{2}} k^{\frac{1-2kz}{2}}
\Gamma\left( kz \right).$$} \\\\



\textbf{Exercise 15.20.}
\emph{Suppose $H(x)$ is a polynomial of degree $n$ with complex coefficients.
Suppose that for all integers $n$, $F > 0$ we have
$H(Fx) = F^{n-1}\sum_{a=0}^{F-1}H(x+\frac{a}{F})$.
Show that $H(x) = CB_n(x)$ for some constant $C$.
(Hint: Use Exercise 16 and induction on $n$.)} \\

Use induction on $n$ to show that $H(x) = CB_n(x)$
where $C$ is the leading coefficient of $H(x)$
(since the leading coefficient of every Bernoulli polynomial is $1$).

\begin{enumerate}
\item[(1)]
As $n = 1$, write $H(x) = C_1 x + C_0 \in \mathbb{C}$.
Then
\begin{align*}
H(Fx)
&= \sum_{a=0}^{F-1} H(x+\frac{a}{F}), \\
C_1 F x + C_0
&= \sum_{a=0}^{F-1} \left( C_1\left(x+\frac{a}{F}\right) + C_0\right) \\
&= C_1Fx + C_1 \cdot \frac{F-1}{2} + C_0F, \\
C_0 &= \frac{-1}{2} C_1
\end{align*}
if $F > 1$. That is, $H(x) = C_1 B_1(x)$ where $C = C_1$ is a constant.
In fact, $C$ is the leading coefficient of $H(x)$.
\item[(2)]
Assume that the conclusion holds for $n = k$.
As $n = k+1$, it suffices to show $f(x) = H(x) - CB_{k+1}(x) = 0$,
where $C$ is the leading coefficient of $H(x)$.
\item[(3)]
Differentiate $f(x) = H(x) - CB_{k+1}(x)$ and use Exercise 15.16,
$$f'(x) = H'(x) - C \cdot (k+1) \cdot B_k(x).$$
Might show $f'(x) = 0$ and then get that $H(x) - CB_{k+1}(x)$ is a constant.
\item[(4)]
Notice that the leading coefficient of $H'(x)$ is $C \cdot (k+1)$.
Besides, by differentiating $H(Fx) = F^{k}\sum_{a=0}^{F-1}H(x+\frac{a}{F})$,
\begin{align*}
H'(Fx) \cdot F
&= F^{k} \sum_{a=0}^{F-1} H'(x+\frac{a}{F}), \\
H'(Fx)
&= F^{k-1} \sum_{a=0}^{F-1} H'(x+\frac{a}{F}).
\end{align*}
By the induction hypothesis,
$H'(x) = (C \cdot (k+1)) B_k(x)$ since $H'(x)$ has degree $(k+1)-1 = k$.
Therefore, $f'(x) = 0$ or $f(x) = H(x) - CB_{k+1}(x) = A$ is a constant.
\item[(5)]
By $f(Fx) = H(Fx) - CB_{k+1}(Fx) = A$,
\begin{align*}
A
&= F^{k} \sum_{a=0}^{F-1} H \left( x+\frac{a}{F} \right)
- C F^{k} \sum_{a=0}^{F-1} B_{k+1} \left( x+\frac{a}{F} \right) \\
&= F^{k} \sum_{a=0}^{F-1} \left( H \left( x+\frac{a}{F} \right)
- C B_{k+1} \left( x+\frac{a}{F} \right) \right) \\
&= F^{k} \sum_{a=0}^{F-1} A \\
&= F^{k+1} A,
\end{align*}
or $(F^{k+1} - 1) A = 0$. For $F > 1$, $A = 0$.
That is, $H(x) = CB_{k+1}(x)$.
\end{enumerate}
By mathematical induction the result is established.
$\Box$ \\\\



\textbf{Exercise 15.21.}
\emph{Show $2^{n-1} B_n(\frac{1}{2}) = (1 - 2^{n-1})B_n$.} \\

The original identity $B_n(\frac{1}{2}) = (1 - 2^{n-1})B_n$ is wrong.
For $n = 2$, $B_2(x) = x^2 - x + \frac{1}{6}$ and thus
$-\frac{1}{12} = B_2(\frac{1}{2}) \neq (1 - 2^{2-1})B_2 = -\frac{1}{6}$. \\

\emph{Proof.}
Taking $F = 2$ in Exercise 15.19,
\begin{align*}
B_n(2x)
&= 2^{n-1} \sum_{a = 0}^{1} B_n\left( x + \frac{a}{2} \right) \\
&= 2^{n-1} B_n(x) + 2^{n-1} B_n\left( x + \frac{1}{2} \right).
\end{align*}
Let $x = 0$,
$$B_n(0) = 2^{n-1} B_n(0) + 2^{n-1} B_n\left( \frac{1}{2} \right), $$
So
$$2^{n-1} B_n\left( \frac{1}{2} \right)
= (1 - 2^{n-1}) B_n(0)
= (1 - 2^{n-1}) B_n.$$
$\Box$ \\\\



\textbf{Exercise 15.22.}
\emph{More generally, show that
$(1 - F^{n-1})B_n = F^{n-1} \sum_{a=1}^{F-1} B_n(\frac{a}{F})$.} \\

The original identity $(1 - F^{n-1})B_n = \sum_{a=1}^{F-1} B_n(\frac{a}{F})$
is wrong again. \\

\emph{Proof.}
Let $x = 0$ in Exercise 15.19,
$$B_n(0)
= F^{n-1} \sum_{a=0}^{F-1} B_n\left( \frac{a}{F} \right)
= F^{n-1} B_n(0) + F^{n-1} \sum_{a=1}^{F-1} B_n\left( \frac{a}{F} \right), $$
So
$$F^{n-1} \sum_{a=1}^{F-1} B_n\left( \frac{a}{F} \right)
= (1 - F^{n-1}) B_n(0)
= (1 - F^{n-1}) B_n.$$
$\Box$ \\\\



\end{document}