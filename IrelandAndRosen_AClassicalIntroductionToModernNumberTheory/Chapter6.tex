\documentclass{article}
\usepackage{amsfonts}
\usepackage{amsmath}
\usepackage{amssymb}
\usepackage{hyperref}
\usepackage{mathrsfs}
\parindent=0pt

\def\upint{\mathchoice%
    {\mkern13mu\overline{\vphantom{\intop}\mkern7mu}\mkern-20mu}%
    {\mkern7mu\overline{\vphantom{\intop}\mkern7mu}\mkern-14mu}%
    {\mkern7mu\overline{\vphantom{\intop}\mkern7mu}\mkern-14mu}%
    {\mkern7mu\overline{\vphantom{\intop}\mkern7mu}\mkern-14mu}%
  \int}
\def\lowint{\mkern3mu\underline{\vphantom{\intop}\mkern7mu}\mkern-10mu\int}

\begin{document}

\textbf{\Large Chapter 6: Quadratic Gauss Sums} \\\\



\emph{Author: Meng-Gen Tsai} \\
\emph{Email: plover@gmail.com} \\\\



\textbf{Exercise 6.1.}
\emph{Show that $\sqrt{2} + \sqrt{3}$ is an algebraic integer.} \\

\emph{Proof.}
Let $\alpha = \sqrt{2} + \sqrt{3}$. So $\alpha - \sqrt{2} = \sqrt{3}$.
Eliminating $\sqrt{3}$ by squaring:
$(\alpha - \sqrt{2})^2 = (\sqrt{3})^2$, or
$\alpha^2 - 2\sqrt{2}\alpha + 2 = 3$, or
$\alpha^2 - 1 = 2\sqrt{2}\alpha$.
Eliminating $\sqrt{2}$ by squaring again:
$(\alpha^2 - 1)^2 = (2\sqrt{2}\alpha)^2$, or
$\alpha^4 - 2 \alpha^2 + 1 = 8 \alpha^2$, or
$\alpha^4 - 10 \alpha^2 + 1 = 0$.
That is, $\alpha$ is a root of $x^4 - 10x^2 + 1 = 0$, i.e.,
$\alpha$ is an algebraic integer.
$\Box$ \\

Actually,
$x^4 - 10x^2 + 1 =
(x - \sqrt{2} - \sqrt{3})(x + \sqrt{2} - \sqrt{3})
(x - \sqrt{2} + \sqrt{3})(x + \sqrt{2} + \sqrt{3})$. \\

\emph{Proof (Proposition 6.1.5).}
Since $\sqrt{2}$ and $\sqrt{3}$ are algebraic integers,
then $\sqrt{2} + \sqrt{3}$ is an algebraic integer by Proposition 6.1.5.
(The set of algebraic integers forms a ring.)
$\Box$ \\\\



\textbf{Exercise 6.2.}
\emph{Let $\alpha$ be an algebraic number.
Show that there is an integer $n$ such that $n\alpha$ is an algebraic integer.} \\

It is trivial if taking $n = 0$. So we assume that $n \neq 0$. \\

\emph{Proof.}
There exists a polynomial
$f(x) = a_0 x^m + a_1 x^{m-1} + \cdots + a_m \in \mathbb{Q}[x]$ with $a_0 \neq 0$,
such that $f(\alpha) = 0$.
There exists an integer $d \neq 0$ such that
$b_i = d \cdot a_i \in \mathbb{Z}$ for all $i = 1, 2, ..., m$.
Therefore,
$$b_0 \alpha^m + b_1 \alpha^{m-1} + \cdots + b_m = 0,$$
which is not necessary a monic polynomial in $\mathbb{Z}[x]$.
So we need to do a trick to absort $b_0$ into $\alpha$,
and that is why we come out multipling $\alpha$ by an non-zero integer
$b_0 = d \cdot a_0$. \\

Multiply $b_0^{m-1}$ on the both sides.
\begin{align*}
b_0^m \alpha^m + b_0^{m-1} b_1 \alpha^{m-1} + b_0^{m-1} b_2 \alpha^{m-2}
+ \cdots + b_0^{m-1} b_m &= 0. \\
(b_0\alpha)^m + b_1 (b_0\alpha)^{m-1} + b_0 b_2 (n\alpha)^{m-1}
+ \cdots + b_0^{m-1} b_m &= 0.
\end{align*}
That is, the monic polynomial
$g(x) = x^m + c_1 x^{m-1} + c_2 x^{m-2} + \cdots + c_m \in \mathbb{Z}[x]$
(with $c_i = b_0^{i - 1} b_i$ for $i = 1, 2, ..., m$)
has a root $x = b_0\alpha$, i.e.,
$b_0\alpha$ is an algebraic integer for some integer $b_0$.
$\Box$ \\\\



\textbf{Exercise 6.4.}
\emph{A polynomial $f(x) \in \mathbb{Z}[x]$ is said to be primitive
if the greatest common divisor of its coefficients is $1$.
Prove that the product of primitive polynomials is again primitive.
This is one of the many results knows as Gauss' lemma.} \\

\emph{Proof.}
Let
\begin{align*}
f(x) &= a_0 x^n + a_1 x^{n-1} + \cdots + a_n, \\
g(x) &= b_0 x^m + b_1 x^{n-1} + \cdots + b_m
\end{align*}
be primitive.
\begin{enumerate}
\item[(1)]
Given prime $p$.
Let $a_i$ and $b_j$ be the coefficients with the smallest index such that
$p \nmid a_i$ and $p \nmid b_j$ respectively.
Consider the coefficient of $x^{i+j}$ in $f(x)g(x)$,
$$(\cdots + a_{i-1} b_{j+1}) + a_i b_j + (a_{i+1} b_{j-1} + \cdots).$$
$p \nmid a_i b_j$ since $p$ is a prime.
$p \mid (\cdots + a_{i-1} b_{j+1})$ by the definition of index $i$.
$p \mid (a_{i+1} b_{j-1} + \cdots)$ by the definition of index $j$.
That is, the coefficient of $x^{i+j}$ in $f(x)g(x)$ is not divided by $p$.
\item[(2)]
If $h(x) = f(x)g(x)$ is not primitive,
there exists a prime $p$ such that $p$ divides all coefficients of $h(x)$.
By (1), such $i$ or $j$ does not exist.
That is, $p$ is a factor of the greatest common divisor of $f(x)$'s or $g(x)$'s coefficients.
So $f(x)$ or $g(x)$ is not primitive, which is absurd.
\end{enumerate}
$\Box$ \\\\



\textbf{Exercise 6.16.}
\emph{Let $\alpha$ be an algebraic number with minimal polynomial $f(x)$.
Show that $f(x)$ does not have repeated roots $\alpha$ in $\mathbb{C}$.} \\

\emph{Proof.}
Assume not true, write $f(x) = (x - \alpha)^2 g(x)$,
where $g(x) \in \mathbb{C}[x]$.
Differentiating $f(x)$ to get new polynomial $f'(x) \in \mathbb{Q}[x]$ and
\begin{align*}
f'(x)
&= 2(x - \alpha) g(x) + (x - \alpha)^2 g'(x) \\
&= (x - \alpha)(2 g(x) + (x - \alpha) g'(x)).
\end{align*}
So $f'(\alpha) = 0$.
Notices that $\deg f(x) \geq 2$ and thus $\deg f'(x) = \deg f(x) - 1 \geq 1$.
$f'(x)$ is not zero. Thus $f(x) \mid f'(x)$ by Proposition 6.1.7, which
contradicts the fact $0 < \deg f'(x) < \deg f(x)$.
$\Box$ \\\\



\textbf{Exercise 6.17.}
\emph{Show that the minimal polynomial for $\sqrt[3]{2}$ is $x^3 - 2$.} \\

\emph{Proof.}
Let $f(x) = x^3 - 2$. $f(\sqrt[3]{2}) = 0$.
By Eisenstein's irreducibility criterion, $f(x)$ is irreducible over $\mathbb{Q}$.
By Proposition 6.1.7, $f(x) = x^3 - 2$ is the minimal polynomial of $\sqrt[3]{2}$.
$\Box$ \\\\



\textbf{Exercise 6.18.}
\emph{Show that there exist algebraic numbers of arbitrarily high degree.} \\

A generalization to Exercise 6.17. \\

If $p$ is a prime, then $x^n - p$ is irreducible over $\mathbb{Q}$,
by Eisenstein's irreducibility criterion, so
$[\mathbb{Q}(\sqrt[n]{p}):\mathbb{Q}] = n$.
(Example 1.16 in Patrick Morandi, Field and Galois Theory.) \\


\emph{Proof.}
Let $\alpha = \sqrt[n]{p}$ for any positive integer $n$ with $n \geq 2$
and prime $p$.
Apply the similar argument in Exercise 6.17 to show that
$f(x) = x^n - p$ is the minimal polynomial of $\sqrt[n]{p}$.
$\Box$ \\\\



\textbf{Exercise 6.23.}
\emph{If $f(x) = x^n + a_1 x^{n-1} + \cdots + a_n$, $a_i \in \mathbb{Z}$
and $p$ is a prime such that $p \mid a_i$, $i = 1, ..., n$, $p^2 \nmid a_n$.
Show that $f(x)$ is irreducible over $\mathbb{Q}$
(Eisenstein's irreducibility criterion).} \\

\emph{Proof.}
\begin{enumerate}
\item[(1)]
\emph{If $f(x) = x^n + a_1 x^{n-1} + \cdots + a_n$, $a_i \in \mathbb{Z}$
and $p$ is a prime such that $p \mid a_i$, $i = 1, ..., n$, $p^2 \nmid a_n$.
Then $f(x)$ is irreducible over $\mathbb{Z}$.}
Assume not true.
Write $f(x) = g(x)h(x)$ as a product of two non-trivial polynomials in $\mathbb{Z}[x]$,
\begin{align*}
g(x) &= b_0 x^s + b_1 x^{s-1} + \cdots + b_s, \\
h(x) &= c_0 x^t + c_1 x^{t-1} + \cdots + c_t,
\end{align*}
where $b_0 = c_0 = 1$, $0 < s < n$, and $0 < t < n$. \\

Since $p \nmid b_0$, there exists largest index $i$ such that $p \nmid b_i$.
(Therefore $p \mid b_{i+1}$, $p \mid b_{i+2}$, and so on.)
Similarly, there exists largest index $j$ such that $p \nmid c_j$.
($p \mid c_{j+1}$, $p \mid c_{j+2}$, and so on.)
Now we consider the coefficient $a_{i+j}$.
$$a_{i+j} = (\cdots + b_{i-1} c_{j+1}) + b_i c_j + (b_{i+1} c_{j-1} + \cdots).$$
$p \nmid b_i c_j$ since $p$ is a prime.
$p \mid (b_{i+1} c_{j-1} + \cdots)$ by the definition of index $i$.
$p \mid (\cdots + b_{i-1} c_{j+1})$ By the definition of index $j$.
Thus, $p \nmid a_{i+j}$.
Hence $i = 0$ and $j = 0$. Especially, $p \mid b_s$ and $p \mid c_t$.
$p^2 \mid b_s c_t$, or $p^2 \mid a_n$ which contradicts.
$\Box$
\item[(2)]
\emph{$f(x)$ is irreducible over $\mathbb{Q}$
if $f(x)$ is primitive and irreducible over $\mathbb{Z}$.}
Assume $f(x) = g(x)h(x) \in \mathbb{Q}[x]$ is reducible.
Let $a$ and $b$ be the least common multiple of the denominators of
$g(x)$ and $h(x)$ respectively.
Then
$$ab \cdot f(x) = (a \cdot g(x))(b \cdot h(x)) = c g_0(x) d h_0(x),$$
where $g_0(x)$, $h_0(x)$ are primitive polynomials in $\mathbb{Z}[x]$, and
$c$ and $d$ are the greatest common divisor of
$(a \cdot g(x))$'s and $(b \cdot h(x))$'s coefficients respectively
Since $g_0(x) h_0(x)$ is again primitive (Exercise 4),
$$ab = \pm cd, f(x) = g_0(x) h_0(x).$$
Notice that $\deg(g_0(x)) = \deg(g(x))$ and $\deg(h_0(x)) = \deg(h(x))$.
So $f(x)$ is reducible over $\mathbb{Z}$, which is absurd.
\end{enumerate}
$\Box$ \\\\



\end{document}