\documentclass{article}
\usepackage{amsfonts}
\usepackage{amssymb}
\parindent=0pt

\begin{document}

\textbf{\Large Chapter 4: The Structure of $U(\mathbb{Z}/n\mathbb{Z})$} \\\\

\textbf{Lemma (Generators of a cyclic group).} \emph{Let $G = \left \langle g \right \rangle$
be a finite cyclic group of order $n$. Then $G = \left \langle h \right \rangle$ iff
$h \in \{ g^a \mid (a, n) = 1 \}$.} \\

\emph{Proof.}
Suppose that $h = g^a$ with $(a, n) = 1$. Then clearly $\left \langle h \right \rangle \subseteq
\left \langle g \right \rangle$ as a subset. For the reverse containment ($\supseteq$),
write $ra + sn = 1$ where $r, s \in \mathbb{Z}$. Then
$h^r = g^{ar} = g^{1 - sn} = g \cdot (g^n)^{-s} = g \cdot 1 = g$. Then again
$\left \langle g \right \rangle \subseteq \left \langle h \right \rangle$ as a subset. \\

Now suppose that $\left \langle g \right \rangle = \left \langle h \right \rangle$.
Then $h = g^a$ for some $a \in \mathbb{Z}$. Also, $g = h^r$ for some $r \in \mathbb{Z}$.
So $g = h^r = g^{ar}$ or $g^{ar - 1} = 1$. So $n | (ar - 1)$, or
$ar + ns = 1$ for some $s \in \mathbb{Z}$, that is, $(a, n) = 1$.
$\Box$ \\

\textbf{Corollary.} \emph{Let $G$ be a finite cyclic group of order $n$.
Then $G$ has exactly $\phi(n)$ generators.} \\

\textbf{Theorem 1.} \emph{$U(\mathbb{Z}/p\mathbb{Z})$ is a cyclic group.} \\

\emph{Proof.}
Let $p - 1 = q_1^{e_1} q_2^{e_2} \cdots q_t^{e^t} = \prod_{q} q^e$ be the prime
decomposition of $p - 1$. Consider the congruences

\begin{enumerate}
\item[(1)]
$x^{q^{e-1}} \equiv 1 (p)$
\item[(2)]
$x^{q^{e}} \equiv 1 (p)$
\end{enumerate}

Therefore,

\begin{enumerate}
\item[(1)]
Every solution to $x^{q^{e-1}} \equiv 1 (p)$ is a solution of $x^{q^{e}} \equiv 1 (p)$.
\item[(2)]
$x^{q^{e}} \equiv 1 (p)$ has more solutions than $x^{q^{e-1}} \equiv 1 (p)$.
In fact, $x^{q^{e-1}} \equiv 1 (p)$ has $q^{e-1}$ solutions and $x^{q^{e}} \equiv 1 (p)$
has $q^{e}$ solutions by Proposition 4.1.2.
\end{enumerate}

Therefore, there exists $g_i \in \mathbb{Z}/p\mathbb{Z}$ generating a subgroup of
$U(\mathbb{Z}/p\mathbb{Z})$ of order $q_i^{e_i}$ for all $i = 1, ..., t$.
Pick $g = g_1 g_2 \cdots g_t \in \mathbb{Z}/p\mathbb{Z}$ generating a subgroup of
$U(\mathbb{Z}/p\mathbb{Z})$ of order $q_1^{e_1} q_2^{e_2} \cdots q_t^{e^t} = p - 1$.
That is, $\left \langle g \right \rangle = U(\mathbb{Z}/p\mathbb{Z})$.
$\Box$ \\

\textbf{Corollary.} \emph{$U(\mathbb{Z}/p\mathbb{Z})$ has exactly $\phi(p - 1)$ generators.} \\

http://ramanujan.math.trinity.edu/rdaileda/teach/s18/m3341/ZnZ.pdf

\textbf{Exercise 4.1.} \emph{Show that $2$ is a primitive root module $29$.}\\

\emph{Proof.}
$2^1 \equiv 2 (29)$,
$2^2 \equiv 4 (29)$,
$2^3 \equiv 8 (29)$,
$2^4 \equiv 16 (29)$,
$2^5 \equiv 3 (29)$,
$2^6 \equiv 6 (29)$,
$2^7 \equiv 12 (29)$,
$2^8 \equiv 24 (29)$,
$2^9 \equiv 19 (29)$,
$2^{10} \equiv 9 (29)$,
$2^{11} \equiv 18 (29)$,
$2^{12} \equiv 7 (29)$,
$2^{13} \equiv 14 (29)$,
$2^{14} \equiv 28 (29)$,
$2^{15} \equiv 27 (29)$,
$2^{16} \equiv 25 (29)$,
$2^{17} \equiv 21 (29)$,
$2^{18} \equiv 13 (29)$,
$2^{19} \equiv 26 (29)$,
$2^{20} \equiv 23 (29)$,
$2^{21} \equiv 17 (29)$,
$2^{22} \equiv 5 (29)$,
$2^{23} \equiv 10 (29)$,
$2^{24} \equiv 20 (29)$,
$2^{25} \equiv 11 (29)$,
$2^{26} \equiv 22 (29)$,
$2^{27} \equiv 15 (29)$,
$2^{28} \equiv 1 (29)$. Thus
$U(\mathbb{Z}/29\mathbb{Z}) = \left \langle 2 \right \rangle$.
$\Box$ \\

\textbf{Exercise 4.11.} \emph{Prove that $1^k + 2^k + \cdots + (p-1)^k \equiv 0 (p)$
if $p - 1 \nmid k$ and $-1 (p)$ if $p - 1 \mid k$.} \\

\emph{Proof.}
Write $\left \langle g \right \rangle = U(\mathbb{Z}/p\mathbb{Z})$, and
$S = 1^k + 2^k + \cdots + (p-1)^k \equiv g^k + (g^k)^2 + \cdots + (g^k)^{p - 1} (p)$. \\

If $p - 1 \mid k$, $g^k \equiv 1 (p)$. Thus
$S \equiv 1 + 1 + \cdots + 1 = p - 1 \equiv -1 (p)$. \\

If $p - 1 \nmid k$, $g^k$ is also a generator of $U(\mathbb{Z}/p\mathbb{Z})$
by Lemma (Generators of a cyclic group).
There are three proofs of this case.
\begin{enumerate}
\item[(1)]
$S$ is the sum of a geometric series.
So $(1 - g^k) S = g^k (1 - (g^k)^{p - 1}) = g^k (1 - (g^{p - 1})^k) \equiv 0 (p)$.
Since $g^k \not\equiv 1 (p)$, $S \equiv 0 (p)$.
\item[(2)]
$\left \langle g^k \right \rangle = U(\mathbb{Z}/p\mathbb{Z})$. So
$S \equiv g^k + (g^k)^2 + \cdots + (g^k)^{p - 1} \equiv 1 + 2 + \cdots + (p - 1)
\equiv \frac{p(p - 1)}{2} \equiv 0 (p)$ since $p$ is odd and thus $\frac{p - 1}{2}$ is an integer.
(If $p = 2$ is even, then there does not exist any $k$ such that $p - 1 \nmid k$.)
\item[(3)]
Similar to (2), write $S \equiv 1 + 2 + \cdots + (p - 1) (p)$. Notice that the equation
$x^{p - 1} - 1 \equiv (x - 1)(x - 2) \cdots (x - (p - 1)) (p)$ holds by Proposition 4.1.1.
So $S \equiv 0 (p)$ by comparing the coefficient of $x^{p - 2}$ on the both sides if $p > 2$.
(Again $p = 2$ is impossible in this case.)
\end{enumerate}

$\Box$ \\

\end{document}