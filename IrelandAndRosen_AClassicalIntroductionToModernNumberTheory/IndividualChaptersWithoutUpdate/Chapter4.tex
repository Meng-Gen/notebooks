\documentclass{article}
\usepackage{amsfonts}
\usepackage{amsmath}
\usepackage{amssymb}
\usepackage{hyperref}
\usepackage{mathrsfs}
\parindent=0pt

\def\upint{\mathchoice%
    {\mkern13mu\overline{\vphantom{\intop}\mkern7mu}\mkern-20mu}%
    {\mkern7mu\overline{\vphantom{\intop}\mkern7mu}\mkern-14mu}%
    {\mkern7mu\overline{\vphantom{\intop}\mkern7mu}\mkern-14mu}%
    {\mkern7mu\overline{\vphantom{\intop}\mkern7mu}\mkern-14mu}%
  \int}
\def\lowint{\mkern3mu\underline{\vphantom{\intop}\mkern7mu}\mkern-10mu\int}

\begin{document}

\textbf{\Large Chapter 4: The Structure of $U(\mathbb{Z}/n\mathbb{Z})$} \\\\



\emph{Author: Meng-Gen Tsai} \\
\emph{Email: plover@gmail.com} \\\\

% https://yutsumura.com/order-of-the-product-of-two-elements-in-an-abelian-group/
% https://yutsumura.com/the-existence-of-an-element-in-an-abelian-group-of-order-the-least-common-multiple-of-two-elements/



\textbf{Theorem 1.} \emph{$U(\mathbb{Z}/p\mathbb{Z})$ is a cyclic group.} \\

\emph{Proof.}
Let $p - 1 = q_1^{e_1} q_2^{e_2} \cdots q_t^{e^t} = \prod_{q} q^e$ be the prime
decomposition of $p - 1$. Consider the congruences

\begin{enumerate}
\item[(1)]
$x^{q^{e-1}} \equiv 1 (p)$
\item[(2)]
$x^{q^{e}} \equiv 1 (p)$
\end{enumerate}

Therefore,

\begin{enumerate}
\item[(1)]
Every solution to $x^{q^{e-1}} \equiv 1 \: (p)$ is a solution of $x^{q^{e}} \equiv 1 \: (p)$.
\item[(2)]
$x^{q^{e}} \equiv 1 \: (p)$ has more solutions than $x^{q^{e-1}} \equiv 1 \: (p)$.
In fact, $x^{q^{e-1}} \equiv 1 \: (p)$ has $q^{e-1}$ solutions and $x^{q^{e}} \equiv 1 \: (p)$
has $q^{e}$ solutions by Proposition 4.1.2.
\end{enumerate}

Therefore, there exists $g_i \in \mathbb{Z}/p\mathbb{Z}$ generating a subgroup of
$U(\mathbb{Z}/p\mathbb{Z})$ of order $q_i^{e_i}$ for all $i = 1, ..., t$.
Pick $g = g_1 g_2 \cdots g_t \in \mathbb{Z}/p\mathbb{Z}$ generating a subgroup of
$U(\mathbb{Z}/p\mathbb{Z})$ of order $q_1^{e_1} q_2^{e_2} \cdots q_t^{e^t} = p - 1$.
That is, $\left \langle g \right \rangle = U(\mathbb{Z}/p\mathbb{Z})$.
$\Box$ \\\\



\textbf{Exercise 4.1.} \emph{Show that $2$ is a primitive root module $29$.}\\

\emph{Proof.}
$2^1 \equiv 2 \: (29)$,
$2^2 \equiv 4 \: (29)$,
$2^3 \equiv 8 \: (29)$,
$2^4 \equiv 16 \: (29)$,
$2^5 \equiv 3 \: (29)$,
$2^6 \equiv 6 \: (29)$,
$2^7 \equiv 12 \: (29)$,
$2^8 \equiv 24 \: (29)$,
$2^9 \equiv 19 \: (29)$,
$2^{10} \equiv 9 \: (29)$,
$2^{11} \equiv 18 \: (29)$,
$2^{12} \equiv 7 \: (29)$,
$2^{13} \equiv 14 \: (29)$,
$2^{14} \equiv 28 \: (29)$,
$2^{15} \equiv 27 \: (29)$,
$2^{16} \equiv 25 \: (29)$,
$2^{17} \equiv 21 \: (29)$,
$2^{18} \equiv 13 \: (29)$,
$2^{19} \equiv 26 \: (29)$,
$2^{20} \equiv 23 \: (29)$,
$2^{21} \equiv 17 \: (29)$,
$2^{22} \equiv 5 \: (29)$,
$2^{23} \equiv 10 \: (29)$,
$2^{24} \equiv 20 \: (29)$,
$2^{25} \equiv 11 \: (29)$,
$2^{26} \equiv 22 \: (29)$,
$2^{27} \equiv 15 \: (29)$,
$2^{28} \equiv 1 \: (29)$. Thus
$U(\mathbb{Z}/29\mathbb{Z}) = \left \langle 2 \right \rangle$.
$\Box$ \\

\emph{Proof (A shorter version).}
$2^{28} \equiv 1 \: (29)$.
It suffices to show that
$2^{14} \not\equiv 1 \: (29)$ and $2^{4} \not\equiv 1 \: (29)$.
Actually, $2^{14} \equiv 28 \: (29)$ and $2^{4} \equiv 16 \: (29)$.
$\Box$ \\\\



\textbf{Exercise 4.11.} \emph{Prove that $1^k + 2^k + \cdots + (p-1)^k \equiv 0 \: (p)$
if $p - 1 \nmid k$ and $-1 (p)$ if $p - 1 \mid k$.} \\

\emph{Proof.}
Write $\left \langle g \right \rangle = U(\mathbb{Z}/p\mathbb{Z})$, and
$S = 1^k + 2^k + \cdots + (p-1)^k \equiv g^k + (g^k)^2 + \cdots + (g^k)^{p - 1} \: (p)$. \\

If $p - 1 \mid k$, $g^k \equiv 1 \: (p)$. Thus
$S \equiv 1 + 1 + \cdots + 1 = p - 1 \equiv -1 \: (p)$. \\

If $p - 1 \nmid k$, $g^k$ is also a generator of $U(\mathbb{Z}/p\mathbb{Z})$ by Exercise 13.
There are three proofs of this case.
\begin{enumerate}
\item[(1)]
$S$ is the sum of a geometric series.
So $(1 - g^k) S = g^k (1 - (g^k)^{p - 1}) = g^k (1 - (g^{p - 1})^k) \equiv 0 \: (p)$.
Since $g^k \not\equiv 1 \: (p)$, $S \equiv 0 \: (p)$.
\item[(2)]
$\left \langle g^k \right \rangle = U(\mathbb{Z}/p\mathbb{Z})$. So
$S \equiv g^k + (g^k)^2 + \cdots + (g^k)^{p - 1} \equiv 1 + 2 + \cdots + (p - 1)
\equiv \frac{p(p - 1)}{2} \equiv 0 \: (p)$ since $p$ is odd and
thus $\frac{p - 1}{2}$ is an integer.
(If $p = 2$ is even, then there does not exist any $k$ such that $p - 1 \nmid k$.)
\item[(3)]
Similar to (2), write $S \equiv 1 + 2 + \cdots + (p - 1) \: (p)$. Notice that the equation
$x^{p - 1} - 1 \equiv (x - 1)(x - 2) \cdots (x - (p - 1)) \: (p)$ holds by Proposition 4.1.1.
So $S \equiv 0 \: (p)$ by comparing the coefficient of $x^{p - 2}$ on the both sides if $p > 2$.
(Again $p = 2$ is impossible in this case.)
\end{enumerate}

$\Box$ \\\\



\textbf{Exercise 4.12.} \emph{Use the existence of a primitive root to give another proof
of Wilson's theorem $(p - 1)! \equiv -1 \: (p)$.} \\

\emph{Proof.}
Say $p > 2$. ($p = 2$ is trivial.)
Let $g$ be a primitive root of $U(\mathbb{Z}/p\mathbb{Z})$.
So $(p - 1)! \equiv g \cdot g^2 \cdots g^{p - 1} \equiv g^{\frac{p(p - 1)}{2}} \: (p)$. \\

The equation $x^2 \equiv 1 \: (p)$ has exactly 2 solutions $x \equiv 1, -1 \: (p)$
by Proposition 4.1.2.
Notice that $x \equiv g^{\frac{p - 1}{2}} \: (p)$ is a solution of the equation
$x^2 \equiv 1 \: (p)$ and $g^{\frac{p - 1}{2}} \not\equiv 1 \: (p)$
since $g$ is a primitive root of $U(\mathbb{Z}/p\mathbb{Z})$.
Therefore, $$g^{\frac{p - 1}{2}} \equiv -1 \: (p).$$
So $(p - 1)! \equiv g^{\frac{p(p - 1)}{2}} \equiv (-1)^p \equiv -1 \: (p)$
since $p$ is an odd prime.
$\Box$ \\

\textbf{Supplement 1.}
There are many proofs of Wilson's theorem.
\begin{enumerate}
\item[(1)]
Exercise 3.9. Use a reduced residue system modulo $p$.
\item[(2)]
Corollary of Proposition 4.1.1. $x^{p - 1} - 1 \equiv (x - 1)(x - 2) \cdots (x - p + 1) \: (p)$.
\item[(3)]
Exercise 4.12. Use the existence of a primitive root.
\item[(4)]
Inclusion-exclusion principle
(\href{http://campus.lakeforest.edu/trevino/WilsonCapsule.pdf}
{Enrique Treviño, An Inclusion-Exclusion Proof of Wilson's Theorem}). \\
\textbf{Lemma.}
$$n! = \sum_{k = 0}^{n}(-1)^k {n \choose k} (n - k)^{n}.$$

\emph{Proof of lemma.}
Consider the number of permutations on $S = \{1, 2, ..., n\}$.
On the one hand, the number is $n!$.
On the other hand, we can think of a permutation on $S$ as a function
$f: S \rightarrow S$ that is onto.
The number of functions $g: S \rightarrow S$ is $n^n$.
To find the onto functions, we have to remove whichever ones are not onto.
Therefore, we must remove those that miss at least $1$ value.
There are ${n \choose 1}$ ways of choosing the missed value and ${(n - 1)}^n$
functions missing that particular value.
But when we remove all of these functions, we took out some too many times, indeed,
any function that misses at least $2$ values was over counted. So we have to add it back in.
We get ${n \choose 2} {(n - 2)}^n$ such functions. Continue this process.
$\Box$ \\

\emph{Proof.}
Now we use the equation $n! = \sum_{k = 0}^{n}(-1)^k {n \choose k} (n - k)^{n}$
by substituting $n = p - 1$ and then get
$$(p - 1)! = \sum_{k = 0}^{p - 1}(-1)^k {p - 1 \choose k} (p - 1 - k)^{p - 1}.$$
Now look at the $k$-term in the summation. \\

$k!(p - 1 - k)! \equiv (-1)^k (p - k)(p - (k - 1)) \cdots (p - 1) \cdot (p - 1 - k)!
\equiv (-1)^k (p - 1)! \: (p)$.
So ${p - 1 \choose k} = \frac{(p - 1)!}{k!(p - 1 - k)!} \equiv (-1)^k \: (p)$.
Also, ${(p - 1 - k)}^{p - 1} \equiv {(-1 - k)}^{p - 1} \equiv {(1 + k)}^{p - 1} \: (p)$
since $(-1)^{p - 1} = 1$ if $p > 2$. ($p = 2$ is trivial.) Therefore,
$$(p - 1)!
\equiv \sum_{k = 0}^{p - 1}(-1)^k \cdot (-1)^k \cdot {(1 + k)}^{p - 1}
\equiv \sum_{k = 1}^{p - 1} k^{p - 1} \: (p).$$
(We adjust the index of the summation and notice that $p^{p - 1} \equiv 0 \: (p)$).
By Fermat’s Little Theorem, $k^{p - 1} \equiv 1 \: (p)$.
Therefore, the right-hand sum consists of $(p - 1)$ ones and the proof is completed.
$\Box$ \\

The original proof in the paper is not very beautiful.
We don't need to use the inclusion-exclusion expression of $p!$
and then cancel out $p$ on the both sides. Please use $(p - 1)!$ directly.
\item[(5)]
One combinatorial proof
(\href{https://www.youtube.com/watch?v=4qbh7mC6YCY}
{Cheenta, Wilson's Theorem and It's Geometric proof}). \\
\emph{Proof.}
Consider a circumference with $p$ points that correspond to the vertices of a regular $p$-gon.
There are $\frac{(p - 1)!}{2}$ (non-regular or regular) polygons
that we form by joining these vertices. \\

Now among $\frac{(p - 1)!}{2}$ of them, we have $\frac{p - 1}{2}$ unaltered
when rotated by $\frac{2 \pi}{p}$ radian.
That is, there are $\frac{p - 1}{2}$ regular polygons due to the rotational symmetry. \\

Therefore, there are $\frac{(p - 1)!}{2} - \frac{p - 1}{2}$ non-regular polygons.
Notices that the number of non-regular polygons is divided by $p$ since $p$ is a prime. \\

So $\frac{(p - 1)!}{2} - \frac{p - 1}{2} \equiv 0 \: (p)$.
Hence, $(p - 1)! \equiv p - 1 \equiv -1 \: (p)$ if $p > 2$. ($p = 2$ is trivial.)
$\Box$ \\

\end{enumerate}

\textbf{Supplement 2.} Related problems.
\begin{enumerate}
\item[(1)]
(\href{https://projecteuler.net/problem=381}
{ProjectEuler 381: (prime-k) factorial}).
\emph{Let $S(p) = \sum_{1 \leq k \leq 5}(p-k)! \: (p)$ for a prime $p$.
Find $\sum_{1 \leq p \leq {10}^8} S(p)$} (by using computer programs).
\item[(2)]
\emph{Let $g$ be a primitive root modulo the odd prime $p$.
Prove that $g^{\frac{p - 1}{2}} \equiv -1 \: (p)$.
Deduce that if $g, h$ are primitive roots modulo the odd prime $p$
then $g \cdot h$ cannot be a primitive root.} \\\\
\end{enumerate}



\textbf{Exercise 4.13 (Generators of a cyclic group).}
\emph{Let $G$ be a finite cyclic group and $g \in G$ is a generator.
Show that all the other generators are of the form $g^k$,
where $(k, n) = 1$, $n$ being the order of $G$.} \\

\emph{Proof.}
Suppose that $h = g^k$ with $(k, n) = 1$.
Then clearly $\left \langle h \right \rangle \subseteq \left \langle g \right \rangle$
as a subset. For the reverse containment ($\supseteq$),
write $rk + sn = 1$ where $r, s \in \mathbb{Z}$. Then
$h^r = g^{kr} = g^{1 - sn} = g \cdot (g^n)^{-s} = g \cdot 1 = g$. Then again
$\left \langle g \right \rangle \subseteq \left \langle h \right \rangle$ as a subset. \\

Now suppose that $\left \langle g \right \rangle = \left \langle h \right \rangle$.
Then $h = g^k$ for some $k \in \mathbb{Z}$. Also, $g = h^r$ for some $r \in \mathbb{Z}$.
So $g = h^r = g^{kr}$ or $g^{kr - 1} = 1$. So $n | (kr - 1)$, or
$ar + ns = 1$ for some $s \in \mathbb{Z}$, that is, $(a, n) = 1$.
$\Box$ \\

Reference:
\href{http://ramanujan.math.trinity.edu/rdaileda/teach/s18/m3341/ZnZ.pdf}
{R. C. Daileda, The Structure of $U(\mathbb{Z}/n\mathbb{Z})$.} \\

\textbf{Corollary.}
\emph{Let $G$ be a finite cyclic group of order $n$.
Then $G$ has exactly $\phi(n)$ generators.} \\

\textbf{Corollary.}
\emph{$U(\mathbb{Z}/p\mathbb{Z})$ has exactly $\phi(p - 1)$ generators.
$U(\mathbb{Z}/p^l\mathbb{Z})$ has exactly $\phi(p^{l-1}(p - 1))$ generators if $p$ is odd.} \\\\



\textbf{Exercise 4.22.}
\emph{If $a$ has order $3$ modulo $p$, show that $1 + a$ has order $6$.} \\

\emph{Proof.}
Since $a$ has order $3$, $0 \equiv a^3 - 1 \equiv (a - 1)(a^2 + a + 1) \: (p)$.
Since $p$ is a prime, $a - 1 \equiv 0 \: (p)$ or $a^2 + a + 1 \equiv 0 \: (p)$.
$a$ cannot be $1$ since $a$ has order $3 \neq 1$.
Therefore,
$$a^2 + a + 1 \equiv 0 \: (p),$$
or
$1 + a \equiv -a^2 \equiv -a^{-1} \: (p)$.
So
\begin{align*}
(1 + a)^6 &\equiv (-a^{-1})^6 \equiv 1 \: (p), \\
1 + a     &\not\equiv 1 \: (p), \\
(1 + a)^2 &\equiv a \not\equiv 1 \: (p), \\
(1 + a)^3 &\equiv -1 \not\equiv 1 \: (p).
\end{align*}
Hence $1 + a$ has order $6$.
$\Box$ \\\\



\end{document}