\documentclass{article}
\usepackage{amsfonts}
\usepackage{amsmath}
\usepackage{amssymb}
\usepackage{hyperref}
\usepackage{mathrsfs}
\parindent=0pt

\def\upint{\mathchoice%
    {\mkern13mu\overline{\vphantom{\intop}\mkern7mu}\mkern-20mu}%
    {\mkern7mu\overline{\vphantom{\intop}\mkern7mu}\mkern-14mu}%
    {\mkern7mu\overline{\vphantom{\intop}\mkern7mu}\mkern-14mu}%
    {\mkern7mu\overline{\vphantom{\intop}\mkern7mu}\mkern-14mu}%
  \int}
\def\lowint{\mkern3mu\underline{\vphantom{\intop}\mkern7mu}\mkern-10mu\int}

\begin{document}

\textbf{\Large Chapter 5: Quadratic Reciprocity} \\\\



\emph{Author: Meng-Gen Tsai} \\
\emph{Email: plover@gmail.com} \\\\




% https://github.com/xyzzyz/ireland-rosen/blob/master/rosen.pdf



%%%%%%%%%%%%%%%%%%%%%%%%%%%%%%%%%%%%%%%%%%%%%%%%%%%%%%%%%%%%%%%%%%%%%%%%%%%%%%%%



\textbf{Exercise 5.2.}
\emph{Show that the number of solutions to $x^2 \equiv a \: (p)$
is given by $1 + (a/p)$.} \\

$p$ is an odd prime. \\

\emph{Proof.}
\begin{enumerate}
\item[(1)]
If $x \equiv t \: (p)$ is a solution of the equation $x^2 \equiv a \: (p)$,
then $x \equiv -t \: (p)$ is also a solution.
Notice that $t \not\equiv -t \: (p)$ if $t \not\equiv 0 \: (p)$
by using the fact that $p$ is odd.
\item[(2)]
(Lemma 4.1.) Let $f(x) \in k[x]$, $k$ a field. Suppose that $\deg f(x) = n$.
Then $f$ has at most $n$ distinct roots.
\item[(3)]
If $a = 0$, then $x^2 \equiv 0 \: (p)$ has only one solution $x \equiv 0 \: (p)$,
or $1 + (a/p)$ solution (where $(a/p) = 0$ in this case).
\item[(4)]
If $a \neq 0$ is a quadratic residue mod $p$, then by (1)(2)
the equation $x^2 \equiv a \: (p)$ has exactly 2 solutions, or $1 + (a/p)$ solutions
(where $(a/p) = 1$ in this case).
\item[(5)]
If $a$ is not a quadratic residue mod $p$,
then there is no solutions of the equation $x^2 \equiv a \: (p)$,
or $1 + (a/p)$ solutions (where $(a/p) = -1$ in this case).
\end{enumerate}
By (3)(4)(5), in any case the number of solutions to $x^2 \equiv a \: (p)$
is given by $1 + (a/p)$.
$\Box$ \\\\



%%%%%%%%%%%%%%%%%%%%%%%%%%%%%%%%%%%%%%%%%%%%%%%%%%%%%%%%%%%%%%%%%%%%%%%%%%%%%%%%



\textbf{Exercise 5.4.}
\emph{Prove that $\sum_{a=1}^{p-1} (a/p) = 0$.} \\

\emph{Note.}
$\sum_{a=0}^{p-1} (a/p) = 0$ since $(0/p) = 0$. \\

\emph{Proof.}
There are as many residues as nonresidues mod $p$ (Corollary to Proposition 5.1.2).
$\Box$ \\\\



%%%%%%%%%%%%%%%%%%%%%%%%%%%%%%%%%%%%%%%%%%%%%%%%%%%%%%%%%%%%%%%%%%%%%%%%%%%%%%%%



\textbf{Exercise 5.5.}
\emph{Prove that $\sum_{x=0}^{p-1} \left( \frac{ax+b}{p} \right) = 0$
provided that $p \nmid a$.} \\

\emph{Proof.}
Since $x$ $(x = 1, \ldots, p-1)$ is a reduced residue system modulo $p$,
$ax$ $(x = 1, \ldots, p-1)$ is again a reduced residue system modulo $p$ if $p \nmid a$
(Exercise 3.6).
Hence
$$\sum_{x=1}^{p-1} \left( \frac{ax}{p} \right) = 0.$$
Note that $\left( \frac{0}{p} \right) = 0$,
and thus
$0
= \sum_{x=0}^{p-1} \left( \frac{ax}{p} \right)
= \sum_{x=0}^{p-1} \left( \frac{a(x+a^{-1}b)}{p} \right)
= \sum_{x=0}^{p-1} \left( \frac{ax+b}{p} \right)$.
$\Box$ \\\\



%%%%%%%%%%%%%%%%%%%%%%%%%%%%%%%%%%%%%%%%%%%%%%%%%%%%%%%%%%%%%%%%%%%%%%%%%%%%%%%%



\textbf{Exercise 5.6.}
\emph{Show that the number of solutions to $x^2 - y^2 \equiv a \: (p)$
is given by
$$\sum_{y=0}^{p-1} \left( 1 + \left( \frac{y^2 + a}{p} \right) \right).$$ }

\emph{Proof.}
Write $x^2 \equiv y^2 + a \: (p)$.
For every fixed $y = 0, \ldots, p-1$,
the number of solutions $x$ to $x^2 \equiv y^2 + a \: (p)$
is given by $1 + \left( \frac{y^2 + a}{p} \right)$ (Exercise 5.2).
Hence, the number of solutions $(x, y)$ to $x^2 - y^2 \equiv a \: (p)$
is
$$\sum_{y=0}^{p-1} \left( 1 + \left( \frac{y^2 + a}{p} \right) \right).$$
$\Box$ \\\\



%%%%%%%%%%%%%%%%%%%%%%%%%%%%%%%%%%%%%%%%%%%%%%%%%%%%%%%%%%%%%%%%%%%%%%%%%%%%%%%%



\textbf{Exercise 5.7.}
\emph{By calculating directly show that
the number of solutions to $x^2 - y^2 \equiv a \: (p)$
is $p-1$ if $p \nmid a$ and $2p-1$ if $p \mid a$.
(Hint: Use the change of variables $u=x+y, v=x-y$.) } \\

\emph{Proof (Hint).}
Write $(x+y)(x-y) \equiv a \: (p)$ or $uv \equiv a \: (p)$ where $u=x+y, v=x-y$.
For any $a$, either $a \equiv 0 \: (p)$ or $a \not\equiv 0 \: (p)$.
\begin{enumerate}
\item[(1)]
$a \equiv 0 \: (p)$. Then $u = 0$ or $v = 0$.
Consider three possible cases (may be overlapped).
  \begin{enumerate}
  \item[(a)]
  $u = 0$, or $x+y = 0$. In this case, the number of solutions is $p$.
  ($x = k, y = -k$ for $k = 0, \ldots, p-1$.)
  \item[(b)]
  $v = 0$. Similar to (a), the number of solutions is $p$.
  ($x = k, y = k$ for $k = 0, \ldots, p-1$.)
  \item[(c)]
  $u = v = 0$. $x = y = 0$.
  \end{enumerate}
  By (a)(b)(c), there are $2p-1$ solutions to $x^2 - y^2 \equiv 0 \: (p)$.
\item[(2)]
$a \not\equiv 0 \: (p)$. $u \neq 0$ and $v \neq 0$.
For each $u = k$ for $k = 1, \ldots, p-1$, there is one unique $v = a k^{-1}$
such that $uv \equiv a \: (p)$. Solve $u$ and $v$ to get
$(x, y) = (2^{-1} (k + a k^{-1}), 2^{-1} (k - a k^{-1})) \in \mathbb{Z}/p\mathbb{Z}$
for $k = 1, \ldots, p-1$.
So there are $p-1$ solutions to $x^2 - y^2 \equiv a \: (p)$ where $a \not\equiv 0 \: (p)$.
\end{enumerate}
By (1)(2), the result holds.
$\Box$ \\\\



%%%%%%%%%%%%%%%%%%%%%%%%%%%%%%%%%%%%%%%%%%%%%%%%%%%%%%%%%%%%%%%%%%%%%%%%%%%%%%%%



\textbf{Exercise 5.8.}
\emph{Combining the results of Exercise 5.6 and 5.7 show that
\begin{equation*}
  \sum_{y=0}^{p-1} \left( \frac{y^2 + a}{p} \right) =
    \begin{cases}
      -1,  & \text{ if $p \nmid a$}, \\
      p-1, & \text{ if $p \mid a$}.
    \end{cases}
\end{equation*}} \\

\emph{Proof.}
By Exercise 5.6 and 5.7,
\begin{equation*}
  \sum_{y=0}^{p-1} \left( 1 + \left( \frac{y^2 + a}{p} \right) \right) =
    \begin{cases}
      p-1,  & \text{ if $p \nmid a$}, \\
      2p-1, & \text{ if $p \mid a$}.
    \end{cases}
\end{equation*}
Hence the result holds.
$\Box$ \\\\



%%%%%%%%%%%%%%%%%%%%%%%%%%%%%%%%%%%%%%%%%%%%%%%%%%%%%%%%%%%%%%%%%%%%%%%%%%%%%%%%



\end{document}