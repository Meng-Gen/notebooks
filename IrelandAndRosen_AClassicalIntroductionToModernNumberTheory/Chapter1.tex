\documentclass{article}
\usepackage{amsfonts}
\usepackage{amsmath}
\usepackage{amssymb}
\usepackage{hyperref}
\usepackage{mathrsfs}
\parindent=0pt

\def\upint{\mathchoice%
    {\mkern13mu\overline{\vphantom{\intop}\mkern7mu}\mkern-20mu}%
    {\mkern7mu\overline{\vphantom{\intop}\mkern7mu}\mkern-14mu}%
    {\mkern7mu\overline{\vphantom{\intop}\mkern7mu}\mkern-14mu}%
    {\mkern7mu\overline{\vphantom{\intop}\mkern7mu}\mkern-14mu}%
  \int}
\def\lowint{\mkern3mu\underline{\vphantom{\intop}\mkern7mu}\mkern-10mu\int}

\begin{document}

\textbf{\Large Chapter 1: Unique Factorization} \\\\



\emph{Author: Meng-Gen Tsai} \\
\emph{Email: plover@gmail.com} \\\\



%%%%%%%%%%%%%%%%%%%%%%%%%%%%%%%%%%%%%%%%%%%%%%%%%%%%%%%%%%%%%%%%%%%%%%%%%%%%%%%%



\textbf{Exercise 1.10.}
\emph{Suppose that $(u,v)=1$.
Show that $(u+v,u-v)$ is either $1$ or $2$.} \\

Each case is possible:
\begin{enumerate}
  \item[(1)]
  $u=3, v=2$. $(u,v)=1$ and $(u+v,u-v)=1$.
  \item[(2)]
  $u=3, v=1$. $(u,v)=1$ and $(u+v,u-v)=2$. \\
\end{enumerate}

\emph{Proof (Exercise 1.6).}
Since $(u,v)=1$, there is $m, n \in \mathbb{Z}$ such that $mu+nv=1$ (Exercise 1.4).
So
\begin{align*}
mu+nv=1
\Longleftrightarrow& 2mu+2nv=2 \\
\Longleftrightarrow& ((u+v) + (u-v))m + ((u+v)-(u-v))n = 2 \\
\Longleftrightarrow& (m+n)(u+v) + (m-n)(u-v) = 2,
\end{align*}
or $(x,y)=(m+n,m-n)$ is an integer solution to $(u+v)x + (u-v)y = 2$.
So $2 \mid (u+v, u-v)$ (Exercise 1.6).
Hence $(u+v, u-v) = 1$ or $2$.
$\Box$ \\\\



%%%%%%%%%%%%%%%%%%%%%%%%%%%%%%%%%%%%%%%%%%%%%%%%%%%%%%%%%%%%%%%%%%%%%%%%%%%%%%%%



\textbf{Exercise 1.11.}
\emph{Show that $(a,a+k) \mid k$.} \\

\emph{Proof (Exercise 1.6).}
The equation $ax + (a+k)y = k$ has solution $(x,y) = (-1,1) \in \mathbb{Z}^2$.
Hence $(a,a+k) \mid k$ (Exercise 1.6).
$\Box$ \\\\



%%%%%%%%%%%%%%%%%%%%%%%%%%%%%%%%%%%%%%%%%%%%%%%%%%%%%%%%%%%%%%%%%%%%%%%%%%%%%%%%



\textbf{Exercise 1.31.}
\emph{Show that $2$ is divided by $(1+i)^2 \in \mathbb{Z}[i]$.} \\

$1+i$ is irreducible in $\mathbb{Z}[i]$. \\

The ring morphism $\mathbb{Z} \rightarrow \mathbb{Z}[i]$
corresponds to a map of schemes
$f: \text{Spec}(\mathbb{Z}[i]) \rightarrow \text{Spec}(\mathbb{Z})$.
Suppose $(p)$ is a prime ideal of $\mathbb{Z}$.
Might find the points of $f^{-1}(p) \in \text{Spec}(\mathbb{Z}[i])$. \\

\emph{Proof.}
$(1+i)^2 = 2i \in \mathbb{Z}[i]$.
Thus $2 \mid (1+i)^2 \in \mathbb{Z}[i]$.
$\Box$ \\\\



%%%%%%%%%%%%%%%%%%%%%%%%%%%%%%%%%%%%%%%%%%%%%%%%%%%%%%%%%%%%%%%%%%%%%%%%%%%%%%%%



\textbf{Exercise 1.34.}
\emph{Show that $3$ is divided by $(1-\omega)^2 \in \mathbb{Z}[\omega]$.} \\

\emph{Proof.}
$(1 - \omega)^2
= 1 - 2\omega + \omega^2
= (1 + \omega + \omega^2) - 3 \omega
= - 3 \omega \in \mathbb{Z}[\omega]$.
Thus $3 \mid (1 - \omega)^2 \in \mathbb{Z}[\omega]$.
$\Box$ \\\\



%%%%%%%%%%%%%%%%%%%%%%%%%%%%%%%%%%%%%%%%%%%%%%%%%%%%%%%%%%%%%%%%%%%%%%%%%%%%%%%%



\end{document}