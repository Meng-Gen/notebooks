\documentclass{article}
\usepackage{amsfonts}
\usepackage{amsmath}
\usepackage{amssymb}
\usepackage{hyperref}
\usepackage{mathrsfs}
\parindent=0pt

\def\upint{\mathchoice%
    {\mkern13mu\overline{\vphantom{\intop}\mkern7mu}\mkern-20mu}%
    {\mkern7mu\overline{\vphantom{\intop}\mkern7mu}\mkern-14mu}%
    {\mkern7mu\overline{\vphantom{\intop}\mkern7mu}\mkern-14mu}%
    {\mkern7mu\overline{\vphantom{\intop}\mkern7mu}\mkern-14mu}%
  \int}
\def\lowint{\mkern3mu\underline{\vphantom{\intop}\mkern7mu}\mkern-10mu\int}

\begin{document}

\textbf{\Large Chapter 1: Unique Factorization} \\\\



\emph{Author: Meng-Gen Tsai} \\
\emph{Email: plover@gmail.com} \\\\




\textbf{Exercise 1.31.}
\emph{Show that $2$ is divided by $(1+i)^2 \in \mathbb{Z}[i]$.} \\

The ring morphism $\mathbb{Z} \rightarrow \mathbb{Z}[i]$
corresponds to a map of schemes
$f: \text{Spec}(\mathbb{Z}[i]) \rightarrow \text{Spec}(\mathbb{Z})$.
Suppose $(p)$ is a prime ideal of $\mathbb{Z}$.
Might find the points of $f^{-1}(p) \in \text{Spec}(\mathbb{Z}[i])$. \\

\emph{Proof.}
$(1+i)^2 = 2i \in \mathbb{Z}[i]$.
Thus $2 \mid (1+i)^2 \in \mathbb{Z}[i]$.
$\Box$ \\\\



\textbf{Exercise 1.34.}
\emph{Show that $3$ is divided by $(1-\omega)^2 \in \mathbb{Z}[\omega]$.} \\

\emph{Proof.}
$(1 - \omega)^2
= 1 - 2\omega + \omega^2
= (1 + \omega + \omega^2) - 3 \omega
= - 3 \omega \in \mathbb{Z}[\omega]$.
Thus $3 \mid (1 - \omega)^2 \in \mathbb{Z}[\omega]$.
$\Box$ \\\\



\end{document}