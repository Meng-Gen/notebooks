\documentclass{article}
\usepackage{amsfonts}
\usepackage{amsmath}
\usepackage{amssymb}
\usepackage{centernot}
\usepackage{hyperref}
\usepackage[none]{hyphenat}
\usepackage{mathrsfs}
\usepackage{mathtools}
\usepackage{physics}
\usepackage{tikz-cd}
\parindent=0pt



\title{\textbf{Solutions to the book: \\\emph{Neukirch, J\"{u}rgen, Algebraic Number Theory}}}
\author{Meng-Gen Tsai \\ plover@gmail.com}



\begin{document}
\maketitle
\tableofcontents



%%%%%%%%%%%%%%%%%%%%%%%%%%%%%%%%%%%%%%%%%%%%%%%%%%%%%%%%%%%%%%%%%%%%%%%%%%%%%%%%
%%%%%%%%%%%%%%%%%%%%%%%%%%%%%%%%%%%%%%%%%%%%%%%%%%%%%%%%%%%%%%%%%%%%%%%%%%%%%%%%



% Reference:



%%%%%%%%%%%%%%%%%%%%%%%%%%%%%%%%%%%%%%%%%%%%%%%%%%%%%%%%%%%%%%%%%%%%%%%%%%%%%%%%
%%%%%%%%%%%%%%%%%%%%%%%%%%%%%%%%%%%%%%%%%%%%%%%%%%%%%%%%%%%%%%%%%%%%%%%%%%%%%%%%



\newpage
\section*{Chapter I: Algebraic Integers \\}
\addcontentsline{toc}{section}{Chapter I: Algebraic Integers}



\subsection*{1.1. The Gaussian Integers \\}
\addcontentsline{toc}{subsection}{1.1. The Gaussian Integers}


\subsubsection*{Exercise 1.}
\addcontentsline{toc}{subsubsection}{Exercise 1.}
\emph{$\alpha \in \mathbb{Z}[i]$ is a unit if and only if $N(\alpha) = 1$.} \\



\emph{Proof.}
\begin{enumerate}
\item[(1)]
  \emph{Show that for all $\alpha, \beta \in \mathbb{Z}[i]$,
  $N(\alpha\beta) = N(\alpha)N(\beta)$,
  either by direct computation or using the fact that
  $N(a+bi) = (a+bi)(a-bi)$.
  Conclude that if $\alpha \mid \gamma$ in $\mathbb{Z}[i]$,
  then $N(\alpha) \mid N(\gamma)$ in $\mathbb{Z}$.}

\item[(2)]
  (Direct computation.)
  Write $\alpha = a+bi, \beta=c+di$ where $a, b, c, d \in \mathbb{Z}$.
  Thus,
  \begin{align*}
    N(\alpha\beta)
    &= N((a+bi)(c+di)) \\
    &= N((ac-bd) + (ad+bc)i) \\
    &= (ac-bd)^2 + (ad+bc)^2 \\
    &= (a^2 c^2 - 2abcd + b^2 d^2) + (a^2 d^2 + 2abcd + b^2 c^2) \\
    &= a^2 c^2 + b^2 d^2 + a^2 d^2 + b^2 c^2, \\
    N(\alpha)N(\beta)
    &= N(a+bi) N(c+di) \\
    &= (a^2 + b^2)(c^2 + d^2) \\
    &= a^2 c^2 + b^2 d^2 + a^2 d^2 + b^2 c^2.
  \end{align*}
  Therefore, $N(\alpha\beta) = N(\alpha)N(\beta)$.
  (Note that we also get the identity
  $(a^2 + b^2)(c^2 + d^2) = (ac-bd)^2 + (ad+bc)^2$.)

\item[(3)]
  (Using the fact that $N(a+bi) = (a+bi)(a-bi)$,
  or $N(\alpha) = \alpha \overline{\alpha}$
  for any $\alpha \in \mathbb{Z}[i]$.)
  \begin{align*}
    N(\alpha\beta)
    &= \alpha\beta\overline{\alpha\beta} \\
    &= \alpha\beta\overline{\alpha}\overline{\beta} \\
    &= \alpha\overline{\alpha}\beta\overline{\beta} \\
    &= N(\alpha)N(\beta).
  \end{align*}

\item[(4)]
  \emph{Show that if $\alpha \mid \gamma$ in $\mathbb{Z}[i]$,
  then $N(\alpha) \mid N(\gamma)$ in $\mathbb{Z}$.}
  Write $\gamma = \alpha \beta$ for some $\beta \in \mathbb{Z}[i]$.
  So $N(\gamma) = N(\alpha) N(\beta) \in \mathbb{Z}$,
  or $N(\alpha) \mid N(\gamma)$ in $\mathbb{Z}$.

\item[(5)]
  \emph{$(\Longrightarrow)$}
  Since $\alpha$ is a unit, there is $\beta \in \mathbb{Z}[i]$ such that
  $\alpha \beta = 1$.
  By (1), $N(\alpha \beta) = N(1)$, or $N(\alpha) N(\beta) = 1$.
  Since the image of $N$ is nonnegative integers, $N(\alpha) = 1$.

\item[(6)]
  \emph{$(\Longleftarrow)$}
  By (1), $N(\alpha) = \alpha \overline{\alpha}$,
  or $1 = \alpha \overline{\alpha}$ since $N(\alpha) = 1$.
  That is, $\overline{\alpha} \in \mathbb{Z}[i]$ is
  the inverse of $\alpha \in \mathbb{Z}[i]$.
  (Or we solve the equation $N(\alpha) = a^2 + b^2 = 1$,
  and show that all four solutions ($\pm 1$ and $\pm i$) are unit.)

\item[(7)]
  Conclusion: a unit $\alpha = a+bi$ of $\mathbb{Z}[i]$
  is satisfying the equation $N(\alpha) = a^2 + b^2 = 1$ by (5)(6).
  That is, the only unit of $\mathbb{Z}[i]$ are $\pm 1$ and $\pm i$.
\end{enumerate}
$\Box$ \\\\



%%%%%%%%%%%%%%%%%%%%%%%%%%%%%%%%%%%%%%%%%%%%%%%%%%%%%%%%%%%%%%%%%%%%%%%%%%%%%%%%



\end{document}