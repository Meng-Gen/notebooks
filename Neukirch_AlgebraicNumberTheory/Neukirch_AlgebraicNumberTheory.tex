\documentclass{article}
\usepackage{amsfonts}
\usepackage{amsmath}
\usepackage{amssymb}
\usepackage{centernot}
\usepackage{hyperref}
\usepackage[none]{hyphenat}
\usepackage{mathrsfs}
\usepackage{mathtools}
\usepackage{physics}
\usepackage{tikz-cd}
\parindent=0pt



\title{\textbf{Notes on the book: \\\emph{J\"{u}rgen Neukirch, Algebraic Number Theory}}}
\author{Meng-Gen Tsai \\ plover@gmail.com}



\begin{document}
\maketitle
\tableofcontents



%%%%%%%%%%%%%%%%%%%%%%%%%%%%%%%%%%%%%%%%%%%%%%%%%%%%%%%%%%%%%%%%%%%%%%%%%%%%%%%%
%%%%%%%%%%%%%%%%%%%%%%%%%%%%%%%%%%%%%%%%%%%%%%%%%%%%%%%%%%%%%%%%%%%%%%%%%%%%%%%%
%%%%%%%%%%%%%%%%%%%%%%%%%%%%%%%%%%%%%%%%%%%%%%%%%%%%%%%%%%%%%%%%%%%%%%%%%%%%%%%%



% Reference:
% https://faculty.math.illinois.edu/~r-ash/ANT.html
% http://pi.math.cornell.edu/~dmehrle/notes/cornell/18sp/6370notes.pdf
% http://math_research.uct.ac.za/marques/US/ANThw6sol.pdf



%%%%%%%%%%%%%%%%%%%%%%%%%%%%%%%%%%%%%%%%%%%%%%%%%%%%%%%%%%%%%%%%%%%%%%%%%%%%%%%%
%%%%%%%%%%%%%%%%%%%%%%%%%%%%%%%%%%%%%%%%%%%%%%%%%%%%%%%%%%%%%%%%%%%%%%%%%%%%%%%%
%%%%%%%%%%%%%%%%%%%%%%%%%%%%%%%%%%%%%%%%%%%%%%%%%%%%%%%%%%%%%%%%%%%%%%%%%%%%%%%%


\newpage
\section*{Chapter I: Algebraic Integers \\}
\addcontentsline{toc}{section}{Chapter I: Algebraic Integers}



\subsection*{I.1. The Gaussian Integers \\}
\addcontentsline{toc}{subsection}{I.1. The Gaussian Integers}



\subsubsection*{Exercise I.1.1.}
\addcontentsline{toc}{subsubsection}{Exercise I.1.1.}
\emph{$\alpha \in \mathbb{Z}[i]$ is a unit if and only if $N(\alpha) = 1$.} \\



\emph{Proof.}
\begin{enumerate}
\item[(1)]
  \emph{$(\Longrightarrow)$}
  Since $\alpha$ is a unit, there is $\beta \in \mathbb{Z}[i]$ such that
  $\alpha \beta = 1$.
  So $N(\alpha \beta) = N(1)$, or $N(\alpha) N(\beta) = 1$.
  Since the image of $N$ is nonnegative integers, $N(\alpha) = 1$.

\item[(2)]
  \emph{$(\Longleftarrow)$}
  $N(\alpha) = \alpha \overline{\alpha}$,
  or $1 = \alpha \overline{\alpha}$ since $N(\alpha) = 1$.
  That is, $\overline{\alpha} \in \mathbb{Z}[i]$ is
  the inverse of $\alpha \in \mathbb{Z}[i]$.
  (Or we solve the equation $N(\alpha) = a^2 + b^2 = 1$,
  and show that all four solutions ($\pm 1$ and $\pm i$) are units.)

\item[(3)]
  Conclusion: a unit $\alpha = a+bi$ of $\mathbb{Z}[i]$
  is satisfying the equation $N(\alpha) = a^2 + b^2 = 1$ by (1)(2).
  That is, the only unit of $\mathbb{Z}[i]$ are $\pm 1$ and $\pm i$.
\end{enumerate}
$\Box$ \\\\



%%%%%%%%%%%%%%%%%%%%%%%%%%%%%%%%%%%%%%%%%%%%%%%%%%%%%%%%%%%%%%%%%%%%%%%%%%%%%%%%



%%%%%%%%%%%%%%%%%%%%%%%%%%%%%%%%%%%%%%%%%%%%%%%%%%%%%%%%%%%%%%%%%%%%%%%%%%%%%%%%



\subsubsection*{Exercise I.1.3. (Pythagorean triples)}
\addcontentsline{toc}{subsubsection}{Exercise I.1.3. (Pythagorean triples)}
\emph{Show that the integer solutions of the equation
\[
  x^2+y^2 = z^2
\]
such that $x, y, z > 0$ and $(x,y,z) = 1$ (``pythagorean triples'') are all given,
up to possible permutation of $x$ and $y$, by the formula
\[
  x = u^2-v^2,
  \qquad
  y = 2uv,
  \qquad
  z = u^2+v^2
\]
where $u, v \in \mathbb{Z}$, $u > v > 0$, $(u,v) = 1$, $u, v$ not both odd.} \\



\emph{Proof.}
\begin{enumerate}
\item[(1)]
  Since $x^2+y^2=z^2$ and $(x,y,z) = 1$,
  we might assume that $(x,y) = 1$, $x$ is odd and $y$ is even.
  Thus $z$ is odd.
  Write $z^2 = x^2+y^2 = (x+iy)(x-iy) \in \mathbb{Z}[i]$.

\item[(2)]
  \emph{Show that $x+iy$ and $x-iy$ are relatively prime over $\mathbb{Z}[i]$.}
  \begin{enumerate}
  \item[(a)]
    Suppose $\delta \mid x \pm iy$ for some $\delta \in \mathbb{Z}[i]$.
    Consider the norm of $\delta$.
    \[
      N(\delta) \mid N(x \pm iy) = x^2 + y^2 = z^2
    \]
    implies that $N(\delta)$ is odd.
  
  \item[(b)]
    $\delta \mid 2x = (x+iy)+(x-iy)$ and $\delta \mid 2iy = (x+iy)-(x-iy)$
    imply that $N(\delta) \mid 4x^2$ and $N(\delta) \mid 4y^2$.
    Since $(x,y) = 1$, $N(\delta) \mid 4$ or $N(\delta) = 1, 2, 4$.

  \item[(c)]
    Hence $N(\delta) = 1$ by (1)(2).
    Exercise I.1.1 shows that $\delta$ is a unit.
    Therefore the conclusion holds.
  \end{enumerate}

\item[(3)]
  Exercise I.1.2 shows that $x+iy = \varepsilon (u+iv)^2$
  for some $u, v \in \mathbb{Z}$ and a unit $\varepsilon$.
  As $x > 0$ is odd and $y > 0$ is even, we might take $\varepsilon = 1$ and $u > v$.
  Hence
  \[
    x = u^2-v^2,
    \qquad
    y = 2uv,
    \qquad
    z = u^2+v^2
  \]
  where $u, v \in \mathbb{Z}$, $u > v > 0$, $(u,v) = 1$.
  Note that $u, v$ cannot be both odd since $x = u^2-v^2$ is not even.
\end{enumerate}
$\Box$ \\\\



%%%%%%%%%%%%%%%%%%%%%%%%%%%%%%%%%%%%%%%%%%%%%%%%%%%%%%%%%%%%%%%%%%%%%%%%%%%%%%%%



\subsubsection*{Exercise I.1.4.}
\addcontentsline{toc}{subsubsection}{Exercise I.1.4.}
\emph{Show that the ring $\mathbb{Z}[i]$ cannot be ordered.} \\



\emph{Proof.}
  Similar to the fact that $i$ cannot be ordered in $\mathbb{C}$.
  Thus $i$ cannot be ordered in $\mathbb{Z}[i]$ either.
$\Box$ \\\\



%%%%%%%%%%%%%%%%%%%%%%%%%%%%%%%%%%%%%%%%%%%%%%%%%%%%%%%%%%%%%%%%%%%%%%%%%%%%%%%%



\subsubsection*{Exercise I.1.5.}
\addcontentsline{toc}{subsubsection}{Exercise I.1.5.}
\emph{Show that the only units of the ring $\mathbb{Z}[\sqrt{-d}] = \mathbb{Z} + \mathbb{Z}\sqrt{-d}$,
for any rational integer $d > 1$, are $\pm 1$.} \\



\emph{Proof.}
\begin{enumerate}
\item[(1)]
  Define the norm $N$ on $\mathbb{Z}[\sqrt{-d}]$ by
  \[
    N(x+y\sqrt{-d}) = (x+y\sqrt{-d})(x-y\sqrt{-d}) = x^2 + y^2 d,
  \]
  i.e., by $N(z) = |z|^2$.
  It is multiplicative.

\item[(2)]
  Similar to Exercise I.1.1,
  \begin{align*}
    \text{$x+y\sqrt{-d} \in \mathbb{Z}[\sqrt{-d}]$ is a unit}
    &\Longleftrightarrow
    N(x+y\sqrt{-d}) = x^2 + y^2 d = 1 \\
    &\Longleftrightarrow
    x^2 = 1 \text{ and } y = 0 \\
    &\Longleftrightarrow
    x = \pm 1 \text{ and } y = 0.
  \end{align*}
  Hence the only units of the ring $\mathbb{Z}[\sqrt{-d}]$ are $\pm 1$ ($d > 1$).
\end{enumerate}
$\Box$ \\\\



%%%%%%%%%%%%%%%%%%%%%%%%%%%%%%%%%%%%%%%%%%%%%%%%%%%%%%%%%%%%%%%%%%%%%%%%%%%%%%%%
%%%%%%%%%%%%%%%%%%%%%%%%%%%%%%%%%%%%%%%%%%%%%%%%%%%%%%%%%%%%%%%%%%%%%%%%%%%%%%%%



\subsection*{I.2. Integrality \\}
\addcontentsline{toc}{subsection}{I.2. Integrality}



\subsubsection*{Exercise I.2.1.}
\addcontentsline{toc}{subsubsection}{Exercise I.2.1.}
\emph{Is $\frac{3+2\sqrt{6}}{1-\sqrt{6}}$ an algebraic integer?} \\



\emph{Proof.}
\begin{enumerate}
\item[(1)]
  $\alpha := \frac{3+2\sqrt{6}}{1-\sqrt{6}} = -3-\sqrt{6}$.
  Since the set of all algebraic integers is a ring,
  $\alpha$ is an algebraic integer.

\item[(2)]
  Or show that $\alpha$ satisfies a monic equation $x^2 + 6x + 3 = 0 \in \mathbb{Z}[x]$.
\end{enumerate}
$\Box$ \\\\



%%%%%%%%%%%%%%%%%%%%%%%%%%%%%%%%%%%%%%%%%%%%%%%%%%%%%%%%%%%%%%%%%%%%%%%%%%%%%%%%



\subsubsection*{Exercise I.2.2.}
\addcontentsline{toc}{subsubsection}{Exercise I.2.2.}
\emph{Show that, if the integral domain $A$ is integrally closed,
then so is the polynomial ring $A[t]$.} \\



\emph{Proof.}
\begin{enumerate}
\item[(1)]
  \emph{Suppose $A$ is integrally closed in $B$.
  Show that $A[t]$ is integrally closed in $B[t]$.}
  Suppose $f \in B[t]$ is integral over $A[t]$.
  Write
  \[
    f^n + g_1 f^{n-1} + \cdots + g_{n-1} f + g_n = 0 
  \]
  where $n > 0$ and $g_i \in A[t]$.
  Hence
  \begin{align*}
    &\:
    f^n + g_1 f^{n-1} + \cdots + g_{n-1} f = -g_n \in A[t] \\
    \Longrightarrow &\:
    f(\underbrace{f^{n-1} + g_1 f^{n-1} + \cdots + g_{n-1}}_{:= g}) \in A[t].
  \end{align*}
  It is possible to show that $fg \in A[t]$ implies that $f \in A[t]$ and $g \in A[t]$
  by using the fact that $A$ is integrally closed in $B$.

\item[(2)]
  \emph{Suppose $f, g$ are monic polynomials in $B[t]$.
  Show that $fg \in A[t]$ implies that $f \in A[t]$ and $g \in A[t]$.}
  Write
  \[
    f = \prod (t - \xi_i),
    \qquad
    g = \prod (t - \eta_j)
  \]
  in some splitting field $F$ of $f$ and $g$ containing the quotient field of $B$.
  Note that each $\xi_i$ and each $\eta_j$ is a root of a monic equation $fg$ in $A[t]$.
  Since $A$ is integrally closed in $B$, $\xi_i, \eta_j \in A$.
  Hence $f, g \in A[t]$.

\item[(3)]
  To apply part (2), we need to remedy leading coefficients of $f$ and $g$.
  Take an integer $m > \max\{\deg(f), \deg(g_1), \ldots, \deg(g_n) \}$.
  Let $f_0 = t^m + f$ be a monic polynomial in $B[t]$.
  Hence
  \begin{align*}
    &\:
    (f_0 - t^m)^n + g_1 (f_0 - t^m)^{n-1} + \cdots + g_n = 0 \\
    \Longrightarrow &\:
    f_0^n + h_1 f_0^{n-1} + \cdots + h_n = 0
  \end{align*}
  where
  \[
    h_n = t^{mn} + (-1)^{n-1} g_1 t^{m(n-1)} + \cdots + g_n \in A[t]
  \]
  is also monic.
  So
  \begin{align*}
    &\:
    f_0^n + h_1 f_0^{n-1} + \cdots + h_{n-1} f = -h_n \text{ is monic in } A[t] \\
    \Longrightarrow &\:
    f_0(\underbrace{f_0^{n-1} + h_1 f^{n-1} + \cdots + h_{n-1}}_{:= h_0}) \in A[t] \text{ where } \\
    &\: \text{$f_0$ and $h_0$ both are monic in } B[t].
  \end{align*}
  Now we can apply part (2) safely.

\item[(4)]
  In part (1), we let $B$ be the quotient field of $A$
  and thus the quotient field of $A[t]$ is $B(t)$.
  Hence
  \begin{align*}
    &\:
    \text{$f \in B(t)$ integral over $A[t]$} \\
    \Longrightarrow &\:
    \text{$f \in B(t)$ integral over $B[t]$}
      &(A[t] \subseteq B[t]) \\
    \Longrightarrow &\:
    f \in B[t]
      &(\text{$B[t]$ is a UFD}) \\
    \Longrightarrow &\:
    \text{$f \in B[t]$ integral over $A[t]$} \\
    \Longrightarrow &\:
    f \in A[t].
      &((1))
  \end{align*}
\end{enumerate}
$\Box$ \\\\



%%%%%%%%%%%%%%%%%%%%%%%%%%%%%%%%%%%%%%%%%%%%%%%%%%%%%%%%%%%%%%%%%%%%%%%%%%%%%%%%



\subsubsection*{Exercise I.2.3.}
\addcontentsline{toc}{subsubsection}{Exercise I.2.3.}
\emph{In the polynomial ring $A = \mathbb{Q}[x,y]$,
consider the principal ideal $\mathfrak{p} = (x^2-y^3)$.
Show that $\mathfrak{p}$ is a prime ideal,
but $A/\mathfrak{p}$ is not integrally closed.} \\



\emph{Proof.}
\begin{enumerate}
\item[(1)]
  It is easy to show that $x^2-y^3$ is irreducible in $A$.
  Hence $\mathfrak{p} = (x^2-y^3)$ is prime since $A$ is a UFD.

\item[(2)]
  By substituting $x = t^3$, $y = t^2$,
  $A/\mathfrak{p} \cong \mathbb{Q}[t^3,t^2]$,
  with quotient field $\mathbb{Q}(t)$ (by noting $t = \frac{x}{y}$).
  Note that $\mathbb{Q}[t]$ is a UFD, thus is already integrally closed.
  So the integral closure will be $\mathbb{Q}[t] \supsetneq \mathbb{Q}[t^3,t^2]$.
  It suggests that $A/\mathfrak{p}$ might not be integrally closed.

\item[(3)]
  (Reductio ad absurdum)
  If not, then the element $\frac{x}{y}$ satisfies a monic equation
  $t^2 - y = 0 \in (A/\mathfrak{p})[t]$.
  So $\frac{x}{y} \in A/\mathfrak{p}$ or $t \in \mathbb{Q}[t^3,t^2]$, which is absurd.
\end{enumerate}
$\Box$ \\



\emph{Note.}
\begin{enumerate}
\item[(1)]
  Serre's criterion for normality.

\item[(2)]
  Hence smoothness is the same as normality for affine curves in $\mathbb{Q}[x,y]$.
  Note that $x^2-y^3$ is an irreducible cubic with a cusp at the origin $(0,0)$.

\item[(3)]
  There is an affine variety $X \in \mathbb{Q}[x,y,z]$ such that
  $X$ is normal but not smooth.
  ($X = V(x^2 + y^2 - z^2)$ for example.) \\\\
\end{enumerate}



%%%%%%%%%%%%%%%%%%%%%%%%%%%%%%%%%%%%%%%%%%%%%%%%%%%%%%%%%%%%%%%%%%%%%%%%%%%%%%%%



\subsubsection*{Exercise I.2.4.}
\addcontentsline{toc}{subsubsection}{Exercise I.2.4.}
\emph{Let $D$ be a squarefree rational integer $\neq 0, 1$ and
$d$ the discriminant of the quadratic number field $K = \mathbb{Q}(\sqrt{D})$.
Show that
\begin{equation*}
  d = \begin{cases}
    D  & \text{if $D \equiv 1 \pmod 4$,}\\
    4D & \text{if $D \equiv 2, 3 \pmod 4$.}
  \end{cases}
\end{equation*}
and that an integral basis of $K$ is given by $\{1, \sqrt{D}\}$ in the second case,
by $\left\{ 1, \frac{1+\sqrt{D}}{2} \right\}$ in the first case,
and by $\left\{ 1, \frac{d+\sqrt{d}}{2} \right\}$ in both case.} \\



\emph{Proof.}
\begin{enumerate}
\item[(1)]
  The Galois group of $K|\mathbb{Q}$ has two elements, the identity
  and an automorphism sending $\sqrt{D}$ to $-\sqrt{D}$.

\item[(2)]
  Note that $\alpha \in \mathcal{O}_K$
  iff $\mathrm{Tr}_{K|\mathbb{Q}}(\alpha), N_{K|\mathbb{Q}}(\alpha) \in \mathbb{Z}$
  (by noting that the equation
  $x^2 - \mathrm{Tr}_{K|\mathbb{Q}}(\alpha) x + N_{K|\mathbb{Q}}(\alpha) = 0$ has
  a root $x = \alpha$).
  So given $\alpha = x + y \sqrt{D} \in \mathcal{O}_K$, we have
  \begin{align*}
    \mathrm{Tr}_{K|\mathbb{Q}}(\alpha) &= 2x \in \mathbb{Z}, \\
    N_{K|\mathbb{Q}}(\alpha) &= x^2 - D y^2 \in \mathbb{Z}.
  \end{align*}

\item[(3)]
  So $4(x^2 - D y^2) = (2x)^2 - D(2y)^2 \in \mathbb{Z}$.
  So $D(2y)^2 \in \mathbb{Z}$ since $2x \in \mathbb{Z}$.
  So $2y \in \mathbb{Z}$ since $D$ is squarefree $\neq 0, 1$.
  Let $r = 2x, s = 2y$.
  Then $r^2 - Ds^2 \equiv 0 \pmod 4$.
  Note that a square $\equiv 0, 1 \pmod 4$.

\item[(4)]
  If $D \equiv 1 \pmod 4$, then
  \begin{align*}
    &\:r^2 - Ds^2 \equiv r^2 - s^2 \pmod 4 \\
    \Longrightarrow &\:
    \text{$r$ and $s$ has the same parity} \\
    \Longrightarrow &\:
    \mathcal{O}_K = \left\{ \frac{r+s\sqrt{D}}{2} : r \equiv s \pmod 2 \right\} \\
    \Longrightarrow &\:
    \mathcal{O}_K = \left\{ \frac{r-s}{2} + s \cdot \frac{1+\sqrt{D}}{2}
      : r \equiv s \pmod 2 \right\} \\
    \Longrightarrow &\:
    \mathcal{O}_K = \mathbb{Z} + \mathbb{Z} \frac{1+\sqrt{D}}{2}.
  \end{align*}
  So $\left\{ 1, \frac{1+\sqrt{D}}{2} \right\}$ is an integral basis of $K$.
  Hence
  \[
    d
    =
    \begin{vmatrix}
      1 & \frac{1+\sqrt{D}}{2} \\
      1 & \frac{1-\sqrt{D}}{2}
    \end{vmatrix}^2
    = D.
  \]

\item[(5)]
  If $D \equiv 2, 3 \pmod 4$, then
  \begin{align*}
    &\:r^2 - Ds^2 \equiv r^2+2s^2 \text{ or } r^2+s^2 \pmod 4 \\
    \Longrightarrow &\:
    \text{both $r$ and $s$ are even} \\
    \Longrightarrow &\:
    \text{both $x$ and $y$ are rational integers} \\
    \Longrightarrow &\:
    \mathcal{O}_K = \mathbb{Z} + \mathbb{Z}\sqrt{D}.
  \end{align*}
  So $\{ 1, \sqrt{D} \}$ is an integral basis of $K$.
  Hence
  \[
    d
    =
    \begin{vmatrix}
      1 & \sqrt{D} \\
      1 & -\sqrt{D}
    \end{vmatrix}^2
    = 4D.
  \]

\item[(6)]
  By (4)(5),
  $\left\{ 1, \frac{d+\sqrt{d}}{2} \right\}$ is an integral basis of $K$
  for any case.
\end{enumerate}
$\Box$ \\\\



%%%%%%%%%%%%%%%%%%%%%%%%%%%%%%%%%%%%%%%%%%%%%%%%%%%%%%%%%%%%%%%%%%%%%%%%%%%%%%%%



\subsubsection*{Exercise I.2.7. (Stickelberger's discriminant relation)}
\addcontentsline{toc}{subsubsection}{Exercise I.2.7. (Stickelberger's discriminant relation)}
\emph{The discriminant $d_K$ of an algebraic number field $K$ is always
$\equiv 0 \pmod 4$ or $\equiv 1 \pmod 4$.
(Hint: The discriminant $\det(\sigma_i \omega_j)$ of an integral basis $\omega_j$
is a sum of terms, each prefixed by a positive or a negative sign.
Writing $P$ (resp. $N$) for the sum of the positive (resp. negative) terms,
one find $d_K = (P-N)^2 = (P+N)^2 - 4PN$.)} \\



\emph{Proof (Hint).}
\begin{enumerate}
\item[(1)]
  Let $S_n$ be the symmetric group of degree $n$, and
  $A_n$ be the alternating group of degree $n$.
  So
  \begin{align*}
    \det(\sigma_i \omega_j)
    &= \sum_{\pi \in S_n}
      \left( \mathrm{sgn}(\pi) \prod_{i=1}^{n} \sigma_i \omega_{\pi(i)} \right) \\
    &= \underbrace{\sum_{\pi \in A_n} \prod_{i=1}^{n} \sigma_i \omega_{\pi(i)}}_{:= P}
      - \underbrace{\sum_{\pi \in S_n - A_n} \prod_{i=1}^{n} \sigma_i \omega_{\pi(i)}}_{:= N}.
  \end{align*}

\item[(2)]
  Note that $\sigma_i(P+N) = P+N$ and $\sigma_i(PN) = PN$ for all $\sigma_i$.
  Hence $P+N, PN \in \mathbb{Q}$.
  Therefore $P+N, PN \in \mathbb{Q} \cap \mathcal{O}_K = \mathbb{Z}$.

\item[(3)]
  By (1)(2),
  \begin{align*}
    d_K
    &= \det(\sigma_i \omega_j)^2 \\
    &= (P-N)^2 \\
    &= (P+N)^2 - 4PN \\
    &\equiv 0, 1 \pmod 4.
  \end{align*}
\end{enumerate}
$\Box$ \\\\



%%%%%%%%%%%%%%%%%%%%%%%%%%%%%%%%%%%%%%%%%%%%%%%%%%%%%%%%%%%%%%%%%%%%%%%%%%%%%%%%
%%%%%%%%%%%%%%%%%%%%%%%%%%%%%%%%%%%%%%%%%%%%%%%%%%%%%%%%%%%%%%%%%%%%%%%%%%%%%%%%



% http://math_research.uct.ac.za/marques/US/ANThw5sol.pdf



\subsection*{I.3. Ideals \\}
\addcontentsline{toc}{subsection}{I.3. Ideals}



\subsubsection*{Exercise I.3.4.}
\addcontentsline{toc}{subsubsection}{Exercise I.3.4.}
\emph{A Dedekind domain with a finite number of prime ideals is a principal ideal domain.
(Hint: If $\mathfrak{a} = \mathfrak{p}_{1}^{\nu_1} \cdots \mathfrak{p}_{r}^{\nu_r} \neq 0$
is an ideal, then choose elements
$\pi_i \in \mathfrak{p}_{i} \smallsetminus \mathfrak{p}_{i}^2$
and apply the Chinese remainder theorem for the cosets
$\pi_i^{\nu_i} \pmod{\mathfrak{p}_{i}^{\nu_i+1}}$.)} \\



\emph{Proof.}
\begin{enumerate}
\item[(1)]
  The hint gives all.

\item[(2)]
  The existence of $\pi_i$ is guaranteed by Theorem I.3.3 (the unique prime factorization).
  The Chinese remainder theorem shows that there is one element $\pi \in \mathcal{O}$
  such that $\pi = \pi_i^{\nu_i} \pmod{\mathfrak{p}_{i}^{\nu_i+1}}$ for each $i$.

\item[(3)]
  Hence $\mathfrak{p} = (\pi)$ since they have the same prime factorization.
\end{enumerate}
$\Box$ \\\\



%%%%%%%%%%%%%%%%%%%%%%%%%%%%%%%%%%%%%%%%%%%%%%%%%%%%%%%%%%%%%%%%%%%%%%%%%%%%%%%%



\subsubsection*{Exercise I.3.5.}
\addcontentsline{toc}{subsubsection}{Exercise I.3.5.}
\emph{The quotient ring $\mathcal{O}/\mathfrak{a}$ of a Dedekind domain by
an ideal $\mathfrak{a} \neq 0$ is a principal ideal domain.
(Hint: For $\mathfrak{a} = \mathfrak{p}^n$ the only proper ideals of $\mathcal{O}/\mathfrak{a}$
are given by $\mathfrak{p}/\mathfrak{p}^n, \ldots, \mathfrak{p}^{n-1}/\mathfrak{p}^n$.
Choose $\pi \in \mathfrak{p} \smallsetminus \mathfrak{p}^2$
and show that $\mathfrak{p}^{\nu} = \mathcal{O}\pi^{\nu} + \mathfrak{p}^n$.)} \\



\emph{Proof.}
\begin{enumerate}
\item[(1)]
  By the Chinese remainder theorem,
  it suffices to show the case $\mathfrak{a} = \mathfrak{p}^n$ where $\mathfrak{p}$ is prime.

\item[(2)]
  There is a natural correspondence between
  \[
    \{ \text{ideals of $\mathcal{O}/\mathfrak{p}^n$} \}
    \longleftrightarrow
    \{ \text{ideals of $\mathcal{O}$ containing $\mathfrak{p}^n$} \}.
  \]
  Hence the proper ideals of $\mathcal{O}/\mathfrak{p}^n$
  are given by $\mathfrak{p}/\mathfrak{p}^n, \ldots, \mathfrak{p}^{n-1}/\mathfrak{p}^n$.

\item[(3)]
  Similar to Exercise I.3.4,
  choose $\pi \in \mathfrak{p} \smallsetminus \mathfrak{p}^2$
  and thus $\mathfrak{p}^{\nu} = \mathcal{O}\pi^{\nu} + \mathfrak{p}^n$
  ($\nu = 1, \ldots, n-1$)
  since they have the same prime factorization.
  Hence
  $\mathfrak{p}^{\nu}/\mathfrak{p}^n = (\pi^{\nu} + \mathfrak{p}^n)$ is principal.
\end{enumerate}
$\Box$ \\\\



%%%%%%%%%%%%%%%%%%%%%%%%%%%%%%%%%%%%%%%%%%%%%%%%%%%%%%%%%%%%%%%%%%%%%%%%%%%%%%%%



\subsubsection*{Exercise I.3.6.}
\addcontentsline{toc}{subsubsection}{Exercise I.3.6.}
\emph{Every ideal of a Dedekind domain can be generated by two elements.
(Hint: Use Exercise I.3.5.)} \\



\emph{Proof.}
\begin{enumerate}
\item[(1)]
  Given an ideal $\mathfrak{a} \neq 0$  of a Dedekind domain $\mathcal{O}$.
  (Nothing to do if $\mathfrak{a} = 0 = (0)$.)
  So $\mathcal{O}/\mathfrak{a}$ is a principal ideal domain (Exercise I.3.5).

\item[(2)]
  Take any $\alpha \in \mathfrak{a} \smallsetminus \{0\}$.
  So
  $(\alpha)/\mathfrak{a} = (\beta \pmod{\mathfrak{a}})$ is a principal ideal
  for some $\beta \in \mathcal{O}$.
  So $\mathfrak{a} = (\alpha, \beta)$ is generated by two elements.
\end{enumerate}
$\Box$ \\\\



%%%%%%%%%%%%%%%%%%%%%%%%%%%%%%%%%%%%%%%%%%%%%%%%%%%%%%%%%%%%%%%%%%%%%%%%%%%%%%%%
%%%%%%%%%%%%%%%%%%%%%%%%%%%%%%%%%%%%%%%%%%%%%%%%%%%%%%%%%%%%%%%%%%%%%%%%%%%%%%%%



\subsection*{I.4. Lattices \\}
\addcontentsline{toc}{subsection}{I.4. Lattices}



% https://math.mit.edu/classes/18.785/2015fa/LectureNotes13.pdf



\subsubsection*{Exercise I.4.1.}
\addcontentsline{toc}{subsubsection}{Exercise I.4.1.}
\emph{Show that a lattice $\Gamma$ in $\mathbb{R}^n$ is complete
if and only if the quotient $\mathbb{R}^n/\Gamma$ is compact.} \\



\emph{Proof.}
\begin{enumerate}
\item[(1)]
  ($\Longrightarrow$)
  Define a natural homeomorphism
  $\varphi: \mathbb{R}^n/\Gamma \to \mathbb{S}^1 \times \cdots \times \mathbb{S}^1$
  by sending $(x_1, \ldots, x_n)$ to $(x_1 \pmod{1}, \ldots, x_n \pmod{1})$
  (where $\mathbb{S}^1 \subseteq \mathbb{R}^2$ is a unit circle).
  Note that $\mathbb{S}^1 \times \cdots \times \mathbb{S}^1$ is compact.

\item[(2)]
  ($\Longleftarrow$)
  Let $V_0$ be the linear subspace of $V$ which is spanned by the set $\Gamma$.
  Since the vector space $V/V_0$ is contained in a compact set $V/\Gamma$,
  \[
    \dim(V/V_0) = 0
  \]
  (otherwise $V/V_0$ is unbounded).
  Hence $V_0 = V$ or $\Gamma$ is complete.
\end{enumerate}
$\Box$ \\\\



%%%%%%%%%%%%%%%%%%%%%%%%%%%%%%%%%%%%%%%%%%%%%%%%%%%%%%%%%%%%%%%%%%%%%%%%%%%%%%%%



\subsubsection*{Exercise I.4.2.}
\addcontentsline{toc}{subsubsection}{Exercise I.4.2.}
\emph{Show that Minkowski's lattice point theorem cannot be improved,
by giving an example of a centrally symmetric convex set $X \subset V$ such that
$\mathrm{vol}(X) = 2^n \mathrm{vol}(\Gamma)$
which does not contain any nonzero point of the lattice $\Gamma$.
If $X$ is compact, however,
then the statement $\mathrm{vol}(X) > 2^n \mathrm{vol}(\Gamma)$ does remain true
in the case of equality.} \\



\emph{Proof.}
\begin{enumerate}
\item[(1)]
  Let $V = \mathbb{R}^n$, $\Gamma = \mathbb{Z}^n$ be a complete lattice in $V$, and
  $X = (-1,1)^n \subseteq \mathbb{R}^n$ be a centrally symmetric convex set in $V$.
  Hence $\mathrm{vol}(X) = 2^n \mathrm{vol}(\Gamma)$
  and $X$ does not contain any nonzero point of $\Gamma$.

\item[(2)]
  Suppose $X$ is compact.
  Consider $X_\nu = (1+\frac{1}{m})X$ for each $\nu \in \mathbb{Z}_{> 0}$.
  Thus $X_\nu$ is again a centrally symmetric convex set in $V$
  and
  \begin{align*}
    \mathrm{vol}(X_\nu)
    &= \left( 1+\frac{1}{\nu} \right) \mathrm{vol}(X) \\
    &\geq \left( 1+\frac{1}{\nu} \right) 2^n \mathrm{vol}(\Gamma) \\
    &> 2^n \mathrm{vol}(\Gamma).
  \end{align*}
  Minkowski's lattice point theorem shows that
  there is one nonzero lattice point $\gamma_\nu \in \Gamma$ for $\nu = 1, 2, 3 \ldots$.

\item[(3)]
  By the compactness of $X_1$,
  there is a subsequence of $\{ \gamma_\nu \}$ converging to $\gamma \in X_1$.
  Since $\Gamma$ is discrete (Proposition I.4.2),
  there are infinitely many $\nu$ such that $\gamma = \gamma_\nu \in X_\nu$.
  (In particular, $\gamma \neq 0$.)
  Hence $\gamma \in X$ by the compactness of $X$.
\end{enumerate}
$\Box$ \\\\



%%%%%%%%%%%%%%%%%%%%%%%%%%%%%%%%%%%%%%%%%%%%%%%%%%%%%%%%%%%%%%%%%%%%%%%%%%%%%%%%
%%%%%%%%%%%%%%%%%%%%%%%%%%%%%%%%%%%%%%%%%%%%%%%%%%%%%%%%%%%%%%%%%%%%%%%%%%%%%%%%



\subsection*{I.5. Minkowski Theory \\}
\addcontentsline{toc}{subsection}{I.5. Minkowski Theory}



\subsubsection*{Exercise I.5.2.}
\addcontentsline{toc}{subsubsection}{Exercise I.5.2.}
\emph{Show that the convex, centrally symmetric set
\[
  X = \left\{ (z_\tau) \in K_{\mathbb{R}} : \sum_{\tau}|z_\tau| < t \right\}
\]
has volume $\mathrm{vol}(X) = 2^r \pi^s \frac{t^n}{n!}$.} \\



\emph{Proof.}
  It is the same as Lemma III.2.15.
$\Box$ \\\\



%%%%%%%%%%%%%%%%%%%%%%%%%%%%%%%%%%%%%%%%%%%%%%%%%%%%%%%%%%%%%%%%%%%%%%%%%%%%%%%%



% http://pi.math.cornell.edu/~dmehrle/notes/cornell/18sp/6370notes.pdf



\subsubsection*{Exercise I.5.3. (Minkowski bound)}
\addcontentsline{toc}{subsubsection}{Exercise I.5.3. (Minkowski bound)}
\emph{Show that in every ideal $\mathfrak{a} \neq 0$ of $\mathcal{O}_K$
there exists an $a \neq 0$ such that
\[
  \abs{ N_{K|\mathbb{Q}}(a) } \leq M (\mathcal{O}_K : \mathfrak{a}),
\]
where $M = \frac{n!}{n^n} \left( \frac{4}{\pi} \right)^{s} \sqrt{|d_K|}$
(the so-called \textbf{Minkowski bound}.)} \\



\emph{Proof.}
\begin{enumerate}
\item[(1)]
  Let
  \[
    X_t = \left\{ (z_\tau) \in K_{\mathbb{R}} : \sum_{\tau}|z_\tau| \leq t \right\}
  \]
  be a convex, centrally symmetric set for any $t > 0$.
  Note that $\mathrm{vol}(X_t) = 2^r \pi^s \frac{t^n}{n!}$ (same as Exercise I.5.2).

\item[(2)]
  In particular, we take $t > 0$ so that
  \[
    \mathrm{vol}(X_t)
    = 2^r \pi^s \frac{t^n}{n!}
    = 2^n \mathrm{vol}(\Gamma).
  \]
  Thus the hypothesis of Minkowski's lattice point theorem in Exercise I.4.2
  is satisfied.
  So there does indeed exist a lattice point $ja \in X_t$, $a \neq 0$, $a \in \mathfrak{a}$;
  in other words, $\sum_\tau |\tau a| \leq t$.

\item[(3)]
  Hence
  \begin{align*}
    \abs{ N_{K|\mathbb{Q}}(a) }
    &= \prod_{\tau} |\tau a| \\
    &\leq \left( \frac{1}{n} \sum_{\tau}|\tau a| \right)^n
      &(\text{AM-GM inequality}) \\
    &\leq \frac{t^n}{n^n}
      &(ja \in X_t) \\
    &= \frac{1}{n^n} \frac{n!}{2^r \pi^s} 2^n \mathrm{vol}(\Gamma)
      &(\text{Definition of $t^n$}) \\
    &= \frac{1}{n^n} \frac{n!}{2^r \pi^s} 2^n \sqrt{|d_K|} (\mathcal{O}_K : \mathfrak{a})
      &(\text{Proposition I.5.2}) \\
    &= \underbrace{\frac{n!}{n^n} \left( \frac{4}{\pi} \right)^{s} \sqrt{|d_K|}}_{:= M}
      (\mathcal{O}_K : \mathfrak{a}).
      &(n = r + 2s)
  \end{align*}
\end{enumerate}
$\Box$ \\\\



%%%%%%%%%%%%%%%%%%%%%%%%%%%%%%%%%%%%%%%%%%%%%%%%%%%%%%%%%%%%%%%%%%%%%%%%%%%%%%%%
%%%%%%%%%%%%%%%%%%%%%%%%%%%%%%%%%%%%%%%%%%%%%%%%%%%%%%%%%%%%%%%%%%%%%%%%%%%%%%%%



\subsection*{I.6. The Class Number \\}
\addcontentsline{toc}{subsection}{I.6. The Class Number}



\subsubsection*{Exercise I.6.3.}
\addcontentsline{toc}{subsubsection}{Exercise I.6.3.}
\emph{Show that in every ideal class of an algebraic number field $K$ of degree $n$,
there exists an integral ideal $\mathfrak{a}_1$ such that
\[
  \mathfrak{N}(\mathfrak{a}_1)
  \leq
  \frac{n!}{n^n}\left( \frac{4}{\pi} \right)^s \sqrt{|d_K|}
\]
(Hint: Use Exercise I.3.5, proceed as in the proof of Theorem I.6.3.)} \\



\emph{Proof.}
\begin{enumerate}
\item[(1)]
  The hint gives all.

\item[(2)]
  Take an arbitrary representative $\mathfrak{a}$ of the class in the ideal class group,
  and a $\gamma \in \mathcal{O}_K$, $\gamma \neq 0$,
  such that $\mathfrak{b} := \gamma \mathfrak{a}^{-1} \subseteq \mathcal{O}_K$.
  By Exercise I.3.5, there exists $\alpha \in \mathfrak{b}$, $\alpha \neq 0$,
  such that
  \[
    \abs{ N_{K|\mathbb{Q}}(\alpha) } \cdot \mathfrak{N}(\mathfrak{b})^{-1}
    = \mathfrak{N}((\alpha) \mathfrak{b}^{-1})
    = \mathfrak{N}(\alpha \mathfrak{b}^{-1})
    \leq \frac{n!}{n^n} \left( \frac{4}{\pi} \right)^{s} \sqrt{|d_K|}.
  \]
  The ideal
  \[
    \mathfrak{a}_1
    := \alpha \mathfrak{b}^{-1}
    = \alpha \gamma^{-1} \mathfrak{a}
    \in [\mathfrak{a}]
  \]
  therefore has the required property.

\item[(3)]
  This exercise also shows that $\mathrm{Cl}_K$ is a finite group.
\end{enumerate}
$\Box$ \\\\



%%%%%%%%%%%%%%%%%%%%%%%%%%%%%%%%%%%%%%%%%%%%%%%%%%%%%%%%%%%%%%%%%%%%%%%%%%%%%%%%
%%%%%%%%%%%%%%%%%%%%%%%%%%%%%%%%%%%%%%%%%%%%%%%%%%%%%%%%%%%%%%%%%%%%%%%%%%%%%%%%



\subsection*{I.7. Dirichlet's Unit Theorem \\}
\addcontentsline{toc}{subsection}{I.7. Dirichlet's Unit Theorem}



% http://rvirk.com/notes/alnum/HW10.pdf
% http://rvirk.com/notes/alnum/alnumhw10sol.pdf



\subsubsection*{Exercise I.7.3. (The Battle of Hastings (October 14, 1066))}
\addcontentsline{toc}{subsubsection}{Exercise I.7.3. (The Battle of Hastings (October 14, 1066))}
\emph{``The men of Harold stood well together, as their wont was, and formed thirteen squares,
with a like number of men in every square thereof,
and woe to the hardy Norman who ventured to enter their redoubts;
for a single blow of a Saxon war hatched would break his lance and cut his coat of mail...
When Harold threw himself into the fray the Saxons were one mighty square of men,
shouting the battle-cries `Ut!', `Olicrosee!', `Godemite!'.''
[Fictitious historical text, following essentially problem no. 129 in:
H.E. Dundeney, Amusements in Mathematics, 1917 (Dover reprints 1958 and 1970).]
Question. How many troops does this suggest Harold II had at the battle of Hastings?} \\



\emph{Proof.}
\begin{enumerate}
\item[(1)]
  Before Harold joins his men, they are in $13$ squares,
  each square consisting of an equal number of men.
  Once Harold joins them, they all together rearrange themselves to form a single square.

\item[(2)]
  Hence the corresponding equation is
  \[
    13x^2 + 1 = y^2
  \]
  and the number of troops is $y^2$.

\item[(3)]
  So $(x,y) = (180,649), (233640, 842401), \ldots$ by WolframAlpha.
  Note that the world population in 1066 was less than it is today.
  The number of troops was $13 \cdot 180^2 + 1 = 649^2 = 421201$.
\end{enumerate}
$\Box$ \\\\



%%%%%%%%%%%%%%%%%%%%%%%%%%%%%%%%%%%%%%%%%%%%%%%%%%%%%%%%%%%%%%%%%%%%%%%%%%%%%%%%
%%%%%%%%%%%%%%%%%%%%%%%%%%%%%%%%%%%%%%%%%%%%%%%%%%%%%%%%%%%%%%%%%%%%%%%%%%%%%%%%



\subsection*{I.11. Localization \\}
\addcontentsline{toc}{subsection}{I.11. Localization}



\subsubsection*{Exercise I.11.7. (Nakayama's lemma)}
\addcontentsline{toc}{subsubsection}{Exercise I.11.7. (Nakayama's lemma)}
\emph{Let $A$ be a local ring with maximal ideal $\mathfrak{m}$,
let $M$ be an $A$-module and
$N \subseteq M$ a submodule such that $M/N$ is finitely generated.
Then one has the implication:}
\[
  M = N + \mathfrak{m}M \Longrightarrow M = N.
\] \\



\emph{Proof.}
\begin{enumerate}
\item[(1)]
  Note that
  \[
    M = N + \mathfrak{m}M
    \Longrightarrow
    M/N = (N + \mathfrak{m}M)/N = \mathfrak{m}(M/N).
  \]
  So it suffices to show that $M' := M/N = 0$.

\item[(2)]
  (Reductio ad absurdum)
  If $M' \neq 0$,
  then there exists a minimal set of generators $\{ x_1, \ldots, x_n \}$ for $M'$.
  Take $x_n \in M' = \mathfrak{m}(M')$.
  We have an equation of the form
  \begin{align*}
    &\: x_n = m_1 x_1 + \cdots + m_n x_n \\
    \Longleftrightarrow &\:
    (1-m_n) x_n = m_1 x_1 + \cdots + m_{n-1} x_{n-1}.
  \end{align*}
  where $m_\nu \in \mathfrak{m}$ for all $\nu$.
  Since $\mathfrak{m}$ is the maximal ideal of a local ring, $1-m_n$ is a unit.
  So $x_n$ is in the submodule of $M'$ generated by $\{ x_1, \ldots, x_{n-1} \}$,
  contrary to the minimality of $n$.
\end{enumerate}
$\Box$ \\\\



%%%%%%%%%%%%%%%%%%%%%%%%%%%%%%%%%%%%%%%%%%%%%%%%%%%%%%%%%%%%%%%%%%%%%%%%%%%%%%%%
%%%%%%%%%%%%%%%%%%%%%%%%%%%%%%%%%%%%%%%%%%%%%%%%%%%%%%%%%%%%%%%%%%%%%%%%%%%%%%%%



\subsection*{I.13. One-dimensional Schemes \\}
\addcontentsline{toc}{subsection}{I.13. One-dimensional Schemes}



\emph{No exercises.} \\\\



%%%%%%%%%%%%%%%%%%%%%%%%%%%%%%%%%%%%%%%%%%%%%%%%%%%%%%%%%%%%%%%%%%%%%%%%%%%%%%%%
%%%%%%%%%%%%%%%%%%%%%%%%%%%%%%%%%%%%%%%%%%%%%%%%%%%%%%%%%%%%%%%%%%%%%%%%%%%%%%%%



\subsection*{I.14. Function Fields \\}
\addcontentsline{toc}{subsection}{I.14. Function Fields}



\emph{No exercises.} \\\\



%%%%%%%%%%%%%%%%%%%%%%%%%%%%%%%%%%%%%%%%%%%%%%%%%%%%%%%%%%%%%%%%%%%%%%%%%%%%%%%%
%%%%%%%%%%%%%%%%%%%%%%%%%%%%%%%%%%%%%%%%%%%%%%%%%%%%%%%%%%%%%%%%%%%%%%%%%%%%%%%%
%%%%%%%%%%%%%%%%%%%%%%%%%%%%%%%%%%%%%%%%%%%%%%%%%%%%%%%%%%%%%%%%%%%%%%%%%%%%%%%%
%%%%%%%%%%%%%%%%%%%%%%%%%%%%%%%%%%%%%%%%%%%%%%%%%%%%%%%%%%%%%%%%%%%%%%%%%%%%%%%%



\newpage
\section*{Chapter II: The Theory of Valuations \\}
\addcontentsline{toc}{section}{Chapter II: The Theory of Valuations}



\subsection*{II.1. The $p$-adic Numbers \\}
\addcontentsline{toc}{subsection}{II.1. The $p$-adic Numbers}



\subsubsection*{Exercise II.1.2.}
\addcontentsline{toc}{subsubsection}{Exercise II.1.2.}
\emph{A $p$-adic integer $a = a_0 + a_1 p + a_2 p^2 + \cdots$
is a unit in the ring $\mathbb{Z}_p$ if and only if $a_0 \neq 0$.} \\



\emph{Proof.}
\begin{enumerate}
\item[(1)]
  ($\Longrightarrow$)
  If $b = b_0 + b_1 p + b_2 p^2 + \cdots$ is an inverse of $a$,
  then $ab = 1$ implies that $a_0 b_0 = 1$ so that $a_0$ is a unit in $\mathbb{Z}/p\mathbb{Z}$
  or $a_0 \neq 0$.

\item[(2)]
  ($\Longleftarrow$)
  Our goal is to find
  \[
    b = b_0 + b_1 p + b_2 p^2 + \cdots \in \mathbb{Z}_p
  \]
  such that the Cauchy product
  \[
    ab = c_0 + c_1 p + c_2 p^2 + \cdots
  \]
  is equal to $1 \in \mathbb{Z}_p$.
  Here $c_n = \sum_{\nu = 0}^{n} a_{\nu} b_{n-\nu}$.
  By the assumption we have that $c_0 = 1$ and $c_1 = c_2 = \cdots = 0$.
  Hence
  \begin{align*}
    b_0 &= a_0^{-1} \\
    b_1 &= -a_0^{-1} a_1 b_0 \\
    & \cdots \\
    b_n &= a_0^{-1} \sum_{\nu = 1}^{n} a_\nu b_{n-\nu} \\
    & \cdots
  \end{align*}
  by induction.

\item[(3)]
  Also see Exercise 1.5 in the textbook:
  \emph{Atiyah \& Macdonald, Introduction to Commutative Algebra.}
  Let $A$ be a commutative ring with $1$ and $A[[x]]$ be
  the ring of formal power series $f = \sum_{n=0}^{\infty} a_n x^n$
  with coefficients in $A$.
  Then $f$ is a unit in $A[[x]]$ if and only if $a_0$ is a unit in $A$.
\end{enumerate}
$\Box$ \\


\subsubsection*{Supplement II.1.2.1.}
\addcontentsline{toc}{subsubsection}{Supplement II.1.2.1.}
\emph{(Exercise 1.5 (i) in the textbook: Atiyah \& Macdonald, Introduction to Commutative Algebra.)
Let $A$ be a ring and let $A[[x]]$ be the ring of formal power series
$f = \sum_{n=0}^{\infty} a_n x^n$ with coefficients in $A$.
Show that $f$ is a unit in $A[[x]]$ if and only if $a_0$ is a unit in $A$.} \\



\emph{Proof.}
\begin{enumerate}
\item[(1)]
  ($\Longrightarrow$)
  If $g = \sum_{n=0}^{\infty} b_n x^n$ is an inverse of $f$,
  then $fg = 1$ implies that $a_0 b_0 = 1$ so that $a_0$ is a unit in $A$.

\item[(2)]
  ($\Longleftarrow$)
  Our goal is to find $g = \sum_{n=0}^{\infty} b_n x^n$
  such that the Cauchy product
  $fg = \sum_{n=0}^{\infty} c_n x^n$
  is equal to $1 \in A[x]$.
  Here $c_n = \sum_{r = 0}^{n} a_{r} b_{n-r}$.
  By the assumption we have that $c_0 = 1$ and $c_1 = c_2 = \cdots = 0$.
  Hence
  \begin{align*}
    b_0 &= a_0^{-1} \\
    b_1 &= -a_0^{-1} a_1 b_0 \\
    & \cdots \\
    b_n &= a_0^{-1} \sum_{r = 1}^{n} a_r b_{n-r} \\
    & \cdots
  \end{align*}
  by induction.
\end{enumerate}
$\Box$ \\\\



%%%%%%%%%%%%%%%%%%%%%%%%%%%%%%%%%%%%%%%%%%%%%%%%%%%%%%%%%%%%%%%%%%%%%%%%%%%%%%%%
%%%%%%%%%%%%%%%%%%%%%%%%%%%%%%%%%%%%%%%%%%%%%%%%%%%%%%%%%%%%%%%%%%%%%%%%%%%%%%%%



\subsection*{II.2. The $p$-adic Absolute Value \\}
\addcontentsline{toc}{subsection}{II.2. The $p$-adic Absolute Value}



\subsubsection*{Exercise II.2.1.}
\addcontentsline{toc}{subsubsection}{Exercise II.2.1.}
\emph{$|x-y|_p \geq \abs{ |x|_p - |y|_p }$.} \\



\emph{Proof.}
Note that $|x+y|_p \leq \max\{ |x|_p, |y|_p \} \leq |x|_p + |y|_p$ and $\abs{\pm 1}_p = 1$.
$\Box$ \\\\



%%%%%%%%%%%%%%%%%%%%%%%%%%%%%%%%%%%%%%%%%%%%%%%%%%%%%%%%%%%%%%%%%%%%%%%%%%%%%%%%
%%%%%%%%%%%%%%%%%%%%%%%%%%%%%%%%%%%%%%%%%%%%%%%%%%%%%%%%%%%%%%%%%%%%%%%%%%%%%%%%
%%%%%%%%%%%%%%%%%%%%%%%%%%%%%%%%%%%%%%%%%%%%%%%%%%%%%%%%%%%%%%%%%%%%%%%%%%%%%%%%
%%%%%%%%%%%%%%%%%%%%%%%%%%%%%%%%%%%%%%%%%%%%%%%%%%%%%%%%%%%%%%%%%%%%%%%%%%%%%%%%



\newpage
\section*{Chapter VII: Zeta Functions and $L$-series \\}
\addcontentsline{toc}{section}{Chapter VII: Zeta Functions and $L$-series}



\subsection*{VII.1. The Riemann Zeta Function \\}
\addcontentsline{toc}{subsection}{VII.1. The Riemann Zeta Function}



\subsubsection*{Exercise VII.1.4.}
\addcontentsline{toc}{subsubsection}{Exercise VII.1.4.}
\emph{For the power sum
\[
  s_k(n) = 1^k + 2^k + 3^k + \cdots + n^k
\]
one has}
\[
  s_k(n) = \frac{1}{k+1} (B_{k+1}(n) - B_{k+1}(0)).
\] \\



\emph{Proof.}
  By Exercise VII.1.3,
  \[
    x^k = \frac{1}{k+1} (B_{k+1}(x) - B_{k+1}(x-1)).
  \]
  Hence the telescoping sum is
  \begin{align*}
     s_k(n)
     &= \sum_{x=1}^{n} x^k \\
     &= \sum_{x=1}^{n} \frac{1}{k+1} (B_{k+1}(x) - B_{k+1}(x-1)) \\
     &= \frac{1}{k+1} (B_{k+1}(n) - B_{k+1}(0)).
  \end{align*}
$\Box$ \\\\



%%%%%%%%%%%%%%%%%%%%%%%%%%%%%%%%%%%%%%%%%%%%%%%%%%%%%%%%%%%%%%%%%%%%%%%%%%%%%%%%
%%%%%%%%%%%%%%%%%%%%%%%%%%%%%%%%%%%%%%%%%%%%%%%%%%%%%%%%%%%%%%%%%%%%%%%%%%%%%%%%
%%%%%%%%%%%%%%%%%%%%%%%%%%%%%%%%%%%%%%%%%%%%%%%%%%%%%%%%%%%%%%%%%%%%%%%%%%%%%%%%



\end{document}