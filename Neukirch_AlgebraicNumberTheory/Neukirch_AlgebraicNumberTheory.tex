\documentclass{article}
\usepackage{amsfonts}
\usepackage{amsmath}
\usepackage{amssymb}
\usepackage{centernot}
\usepackage{hyperref}
\usepackage[none]{hyphenat}
\usepackage{mathrsfs}
\usepackage{mathtools}
\usepackage{physics}
\usepackage{tikz-cd}
\parindent=0pt



\title{\textbf{Solutions to the book: \\\emph{J\"{u}rgen Neukirch, Algebraic Number Theory}}}
\author{Meng-Gen Tsai \\ plover@gmail.com}



\begin{document}
\maketitle
\tableofcontents



%%%%%%%%%%%%%%%%%%%%%%%%%%%%%%%%%%%%%%%%%%%%%%%%%%%%%%%%%%%%%%%%%%%%%%%%%%%%%%%%
%%%%%%%%%%%%%%%%%%%%%%%%%%%%%%%%%%%%%%%%%%%%%%%%%%%%%%%%%%%%%%%%%%%%%%%%%%%%%%%%
%%%%%%%%%%%%%%%%%%%%%%%%%%%%%%%%%%%%%%%%%%%%%%%%%%%%%%%%%%%%%%%%%%%%%%%%%%%%%%%%



% Reference:



%%%%%%%%%%%%%%%%%%%%%%%%%%%%%%%%%%%%%%%%%%%%%%%%%%%%%%%%%%%%%%%%%%%%%%%%%%%%%%%%
%%%%%%%%%%%%%%%%%%%%%%%%%%%%%%%%%%%%%%%%%%%%%%%%%%%%%%%%%%%%%%%%%%%%%%%%%%%%%%%%
%%%%%%%%%%%%%%%%%%%%%%%%%%%%%%%%%%%%%%%%%%%%%%%%%%%%%%%%%%%%%%%%%%%%%%%%%%%%%%%%


\newpage
\section*{Chapter I: Algebraic Integers \\}
\addcontentsline{toc}{section}{Chapter I: Algebraic Integers}



\subsection*{I.1. The Gaussian Integers \\}
\addcontentsline{toc}{subsection}{I.1. The Gaussian Integers}



\subsubsection*{Exercise I.1.1.}
\addcontentsline{toc}{subsubsection}{Exercise I.1.1.}
\emph{$\alpha \in \mathbb{Z}[i]$ is a unit if and only if $N(\alpha) = 1$.} \\



\emph{Proof.}
\begin{enumerate}
\item[(1)]
  \emph{$(\Longrightarrow)$}
  Since $\alpha$ is a unit, there is $\beta \in \mathbb{Z}[i]$ such that
  $\alpha \beta = 1$.
  So $N(\alpha \beta) = N(1)$, or $N(\alpha) N(\beta) = 1$.
  Since the image of $N$ is nonnegative integers, $N(\alpha) = 1$.

\item[(2)]
  \emph{$(\Longleftarrow)$}
  $N(\alpha) = \alpha \overline{\alpha}$,
  or $1 = \alpha \overline{\alpha}$ since $N(\alpha) = 1$.
  That is, $\overline{\alpha} \in \mathbb{Z}[i]$ is
  the inverse of $\alpha \in \mathbb{Z}[i]$.
  (Or we solve the equation $N(\alpha) = a^2 + b^2 = 1$,
  and show that all four solutions ($\pm 1$ and $\pm i$) are units.)

\item[(3)]
  Conclusion: a unit $\alpha = a+bi$ of $\mathbb{Z}[i]$
  is satisfying the equation $N(\alpha) = a^2 + b^2 = 1$ by (1)(2).
  That is, the only unit of $\mathbb{Z}[i]$ are $\pm 1$ and $\pm i$.
\end{enumerate}
$\Box$ \\\\



%%%%%%%%%%%%%%%%%%%%%%%%%%%%%%%%%%%%%%%%%%%%%%%%%%%%%%%%%%%%%%%%%%%%%%%%%%%%%%%%



\subsubsection*{Exercise I.1.4.}
\addcontentsline{toc}{subsubsection}{Exercise I.1.4.}
\emph{Show that the ring $\mathbb{Z}[i]$ cannot be ordered.} \\



\emph{Proof.}
  Similar to the fact that $i$ cannot be ordered in $\mathbb{C}$,
  $i$ cannot be ordered in $\mathbb{Z}[i]$ either.
$\Box$ \\\\



%%%%%%%%%%%%%%%%%%%%%%%%%%%%%%%%%%%%%%%%%%%%%%%%%%%%%%%%%%%%%%%%%%%%%%%%%%%%%%%%



\subsubsection*{Exercise I.1.5.}
\addcontentsline{toc}{subsubsection}{Exercise I.1.5.}
\emph{Show that the only units of the ring $\mathbb{Z}[\sqrt{-d}] = \mathbb{Z} + \mathbb{Z}\sqrt{-d}$,
for any rational integer $d > 1$, are $\pm 1$.} \\



\emph{Proof.}
\begin{enumerate}
\item[(1)]
  Define the norm $N$ on $\mathbb{Z}[\sqrt{-d}]$ by
  \[
    N(x+y\sqrt{-d}) = (x+y\sqrt{-d})(x-y\sqrt{-d}) = x^2 + y^2 d,
  \]
  i.e., by $N(z) = |z|^2$.
  It is multiplicative.

\item[(2)]
  Similar to Exercise I.1.1,
  \begin{align*}
    \text{$x+y\sqrt{-d} \in \mathbb{Z}[\sqrt{-d}]$ is a unit}
    &\Longleftrightarrow
    N(x+y\sqrt{-d}) = x^2 + y^2 d = 1 \\
    &\Longleftrightarrow
    x^2 = 1 \text{ and } y = 0 \\
    &\Longleftrightarrow
    x = \pm 1 \text{ and } y = 0.
  \end{align*}
  Hence the only units of the ring $\mathbb{Z}[\sqrt{-d}]$ are $\pm 1$ ($d > 1$).
\end{enumerate}
$\Box$ \\\\



%%%%%%%%%%%%%%%%%%%%%%%%%%%%%%%%%%%%%%%%%%%%%%%%%%%%%%%%%%%%%%%%%%%%%%%%%%%%%%%%
%%%%%%%%%%%%%%%%%%%%%%%%%%%%%%%%%%%%%%%%%%%%%%%%%%%%%%%%%%%%%%%%%%%%%%%%%%%%%%%%



\subsection*{I.2. Integrality \\}
\addcontentsline{toc}{subsection}{I.2. Integrality}



\subsubsection*{Exercise I.2.1.}
\addcontentsline{toc}{subsubsection}{Exercise I.2.1.}
\emph{Is $\frac{3+2\sqrt{6}}{1-\sqrt{6}}$ an algebraic integer?} \\



\emph{Proof.}
\begin{enumerate}
\item[(1)]
  $\alpha := \frac{3+2\sqrt{6}}{1-\sqrt{6}} = -3-\sqrt{6}$.
  Since the set of all algebraic integers is a ring,
  $\alpha$ is an algebraic integer.

\item[(2)]
  Or show that $\alpha$ satisfies a monic equation $x^2 + 6x + 3 = 0 \in \mathbb{Z}[x]$.
\end{enumerate}
$\Box$ \\\\



%%%%%%%%%%%%%%%%%%%%%%%%%%%%%%%%%%%%%%%%%%%%%%%%%%%%%%%%%%%%%%%%%%%%%%%%%%%%%%%%
%%%%%%%%%%%%%%%%%%%%%%%%%%%%%%%%%%%%%%%%%%%%%%%%%%%%%%%%%%%%%%%%%%%%%%%%%%%%%%%%
%%%%%%%%%%%%%%%%%%%%%%%%%%%%%%%%%%%%%%%%%%%%%%%%%%%%%%%%%%%%%%%%%%%%%%%%%%%%%%%%



\end{document}