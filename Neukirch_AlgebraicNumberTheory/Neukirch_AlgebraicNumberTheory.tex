\documentclass{article}
\usepackage{amsfonts}
\usepackage{amsmath}
\usepackage{amssymb}
\usepackage{centernot}
\usepackage{hyperref}
\usepackage[none]{hyphenat}
\usepackage{mathrsfs}
\usepackage{mathtools}
\usepackage{physics}
\usepackage{tikz-cd}
\parindent=0pt



\title{\textbf{Solutions to the book: \\\emph{J\"{u}rgen Neukirch, Algebraic Number Theory}}}
\author{Meng-Gen Tsai \\ plover@gmail.com}



\begin{document}
\maketitle
\tableofcontents



%%%%%%%%%%%%%%%%%%%%%%%%%%%%%%%%%%%%%%%%%%%%%%%%%%%%%%%%%%%%%%%%%%%%%%%%%%%%%%%%
%%%%%%%%%%%%%%%%%%%%%%%%%%%%%%%%%%%%%%%%%%%%%%%%%%%%%%%%%%%%%%%%%%%%%%%%%%%%%%%%



% Reference:



%%%%%%%%%%%%%%%%%%%%%%%%%%%%%%%%%%%%%%%%%%%%%%%%%%%%%%%%%%%%%%%%%%%%%%%%%%%%%%%%
%%%%%%%%%%%%%%%%%%%%%%%%%%%%%%%%%%%%%%%%%%%%%%%%%%%%%%%%%%%%%%%%%%%%%%%%%%%%%%%%



\newpage
\section*{Chapter I: Algebraic Integers \\}
\addcontentsline{toc}{section}{Chapter I: Algebraic Integers}



\subsection*{I.1. The Gaussian Integers \\}
\addcontentsline{toc}{subsection}{I.1. The Gaussian Integers}


\subsubsection*{Exercise I.1.1.}
\addcontentsline{toc}{subsubsection}{Exercise I.1.1.}
\emph{$\alpha \in \mathbb{Z}[i]$ is a unit if and only if $N(\alpha) = 1$.} \\



\emph{Proof.}
\begin{enumerate}
\item[(1)]
  \emph{$(\Longrightarrow)$}
  Since $\alpha$ is a unit, there is $\beta \in \mathbb{Z}[i]$ such that
  $\alpha \beta = 1$.
  So $N(\alpha \beta) = N(1)$, or $N(\alpha) N(\beta) = 1$.
  Since the image of $N$ is nonnegative integers, $N(\alpha) = 1$.

\item[(2)]
  \emph{$(\Longleftarrow)$}
  $N(\alpha) = \alpha \overline{\alpha}$,
  or $1 = \alpha \overline{\alpha}$ since $N(\alpha) = 1$.
  That is, $\overline{\alpha} \in \mathbb{Z}[i]$ is
  the inverse of $\alpha \in \mathbb{Z}[i]$.
  (Or we solve the equation $N(\alpha) = a^2 + b^2 = 1$,
  and show that all four solutions ($\pm 1$ and $\pm i$) are unit.)

\item[(3)]
  Conclusion: a unit $\alpha = a+bi$ of $\mathbb{Z}[i]$
  is satisfying the equation $N(\alpha) = a^2 + b^2 = 1$ by (1)(2).
  That is, the only unit of $\mathbb{Z}[i]$ are $\pm 1$ and $\pm i$.
\end{enumerate}
$\Box$ \\\\



%%%%%%%%%%%%%%%%%%%%%%%%%%%%%%%%%%%%%%%%%%%%%%%%%%%%%%%%%%%%%%%%%%%%%%%%%%%%%%%%



\end{document}