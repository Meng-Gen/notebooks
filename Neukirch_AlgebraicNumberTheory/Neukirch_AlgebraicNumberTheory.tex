\documentclass{article}
\usepackage{amsfonts}
\usepackage{amsmath}
\usepackage{amssymb}
\usepackage{centernot}
\usepackage{hyperref}
\usepackage[none]{hyphenat}
\usepackage{mathrsfs}
\usepackage{mathtools}
\usepackage{physics}
\usepackage{tikz-cd}
\parindent=0pt



\title{\textbf{Solutions to the book: \\\emph{J\"{u}rgen Neukirch, Algebraic Number Theory}}}
\author{Meng-Gen Tsai \\ plover@gmail.com}



\begin{document}
\maketitle
\tableofcontents



%%%%%%%%%%%%%%%%%%%%%%%%%%%%%%%%%%%%%%%%%%%%%%%%%%%%%%%%%%%%%%%%%%%%%%%%%%%%%%%%
%%%%%%%%%%%%%%%%%%%%%%%%%%%%%%%%%%%%%%%%%%%%%%%%%%%%%%%%%%%%%%%%%%%%%%%%%%%%%%%%
%%%%%%%%%%%%%%%%%%%%%%%%%%%%%%%%%%%%%%%%%%%%%%%%%%%%%%%%%%%%%%%%%%%%%%%%%%%%%%%%



% Reference:



%%%%%%%%%%%%%%%%%%%%%%%%%%%%%%%%%%%%%%%%%%%%%%%%%%%%%%%%%%%%%%%%%%%%%%%%%%%%%%%%
%%%%%%%%%%%%%%%%%%%%%%%%%%%%%%%%%%%%%%%%%%%%%%%%%%%%%%%%%%%%%%%%%%%%%%%%%%%%%%%%
%%%%%%%%%%%%%%%%%%%%%%%%%%%%%%%%%%%%%%%%%%%%%%%%%%%%%%%%%%%%%%%%%%%%%%%%%%%%%%%%


\newpage
\section*{Chapter I: Algebraic Integers \\}
\addcontentsline{toc}{section}{Chapter I: Algebraic Integers}



\subsection*{I.1. The Gaussian Integers \\}
\addcontentsline{toc}{subsection}{I.1. The Gaussian Integers}



\subsubsection*{Exercise I.1.1.}
\addcontentsline{toc}{subsubsection}{Exercise I.1.1.}
\emph{$\alpha \in \mathbb{Z}[i]$ is a unit if and only if $N(\alpha) = 1$.} \\



\emph{Proof.}
\begin{enumerate}
\item[(1)]
  \emph{$(\Longrightarrow)$}
  Since $\alpha$ is a unit, there is $\beta \in \mathbb{Z}[i]$ such that
  $\alpha \beta = 1$.
  So $N(\alpha \beta) = N(1)$, or $N(\alpha) N(\beta) = 1$.
  Since the image of $N$ is nonnegative integers, $N(\alpha) = 1$.

\item[(2)]
  \emph{$(\Longleftarrow)$}
  $N(\alpha) = \alpha \overline{\alpha}$,
  or $1 = \alpha \overline{\alpha}$ since $N(\alpha) = 1$.
  That is, $\overline{\alpha} \in \mathbb{Z}[i]$ is
  the inverse of $\alpha \in \mathbb{Z}[i]$.
  (Or we solve the equation $N(\alpha) = a^2 + b^2 = 1$,
  and show that all four solutions ($\pm 1$ and $\pm i$) are units.)

\item[(3)]
  Conclusion: a unit $\alpha = a+bi$ of $\mathbb{Z}[i]$
  is satisfying the equation $N(\alpha) = a^2 + b^2 = 1$ by (1)(2).
  That is, the only unit of $\mathbb{Z}[i]$ are $\pm 1$ and $\pm i$.
\end{enumerate}
$\Box$ \\\\



%%%%%%%%%%%%%%%%%%%%%%%%%%%%%%%%%%%%%%%%%%%%%%%%%%%%%%%%%%%%%%%%%%%%%%%%%%%%%%%%



\subsubsection*{Exercise I.1.4.}
\addcontentsline{toc}{subsubsection}{Exercise I.1.4.}
\emph{Show that the ring $\mathbb{Z}[i]$ cannot be ordered.} \\



\emph{Proof.}
  Similar to the fact that $i$ cannot be ordered in $\mathbb{C}$.
  Thus $i$ cannot be ordered in $\mathbb{Z}[i]$ either.
$\Box$ \\\\



%%%%%%%%%%%%%%%%%%%%%%%%%%%%%%%%%%%%%%%%%%%%%%%%%%%%%%%%%%%%%%%%%%%%%%%%%%%%%%%%



\subsubsection*{Exercise I.1.5.}
\addcontentsline{toc}{subsubsection}{Exercise I.1.5.}
\emph{Show that the only units of the ring $\mathbb{Z}[\sqrt{-d}] = \mathbb{Z} + \mathbb{Z}\sqrt{-d}$,
for any rational integer $d > 1$, are $\pm 1$.} \\



\emph{Proof.}
\begin{enumerate}
\item[(1)]
  Define the norm $N$ on $\mathbb{Z}[\sqrt{-d}]$ by
  \[
    N(x+y\sqrt{-d}) = (x+y\sqrt{-d})(x-y\sqrt{-d}) = x^2 + y^2 d,
  \]
  i.e., by $N(z) = |z|^2$.
  It is multiplicative.

\item[(2)]
  Similar to Exercise I.1.1,
  \begin{align*}
    \text{$x+y\sqrt{-d} \in \mathbb{Z}[\sqrt{-d}]$ is a unit}
    &\Longleftrightarrow
    N(x+y\sqrt{-d}) = x^2 + y^2 d = 1 \\
    &\Longleftrightarrow
    x^2 = 1 \text{ and } y = 0 \\
    &\Longleftrightarrow
    x = \pm 1 \text{ and } y = 0.
  \end{align*}
  Hence the only units of the ring $\mathbb{Z}[\sqrt{-d}]$ are $\pm 1$ ($d > 1$).
\end{enumerate}
$\Box$ \\\\



%%%%%%%%%%%%%%%%%%%%%%%%%%%%%%%%%%%%%%%%%%%%%%%%%%%%%%%%%%%%%%%%%%%%%%%%%%%%%%%%
%%%%%%%%%%%%%%%%%%%%%%%%%%%%%%%%%%%%%%%%%%%%%%%%%%%%%%%%%%%%%%%%%%%%%%%%%%%%%%%%



\subsection*{I.2. Integrality \\}
\addcontentsline{toc}{subsection}{I.2. Integrality}



\subsubsection*{Exercise I.2.1.}
\addcontentsline{toc}{subsubsection}{Exercise I.2.1.}
\emph{Is $\frac{3+2\sqrt{6}}{1-\sqrt{6}}$ an algebraic integer?} \\



\emph{Proof.}
\begin{enumerate}
\item[(1)]
  $\alpha := \frac{3+2\sqrt{6}}{1-\sqrt{6}} = -3-\sqrt{6}$.
  Since the set of all algebraic integers is a ring,
  $\alpha$ is an algebraic integer.

\item[(2)]
  Or show that $\alpha$ satisfies a monic equation $x^2 + 6x + 3 = 0 \in \mathbb{Z}[x]$.
\end{enumerate}
$\Box$ \\\\



%%%%%%%%%%%%%%%%%%%%%%%%%%%%%%%%%%%%%%%%%%%%%%%%%%%%%%%%%%%%%%%%%%%%%%%%%%%%%%%%



\subsubsection*{Exercise I.2.2.}
\addcontentsline{toc}{subsubsection}{Exercise I.2.2.}
\emph{Show that, if the integral domain $A$ is integrally closed,
then so is the polynomial ring $A[t]$.} \\



\emph{Proof.}
\begin{enumerate}
\item[(1)]
  Let $K$ be the quotient field of $A$.

\end{enumerate}
$\Box$ \\\\



%%%%%%%%%%%%%%%%%%%%%%%%%%%%%%%%%%%%%%%%%%%%%%%%%%%%%%%%%%%%%%%%%%%%%%%%%%%%%%%%



\subsubsection*{Exercise I.2.3.}
\addcontentsline{toc}{subsubsection}{Exercise I.2.3.}
\emph{In the polynomial ring $A = \mathbb{Q}[x,y]$,
consider the principal ideal $\mathfrak{p} = (x^2-y^3)$.
Show that $\mathfrak{p}$ is a prime ideal,
but $A/\mathfrak{p}$ is not integrally closed.} \\



\emph{Proof.}
\begin{enumerate}
\item[(1)]
  It is easy to show that $x^2-y^3$ is irreducible in $A$.
  Hence $\mathfrak{p} = (x^2-y^3)$ is prime since $A$ is a UFD.

\item[(2)]
  By substituting $x = t^3$, $y = t^2$,
  $A/\mathfrak{p} \cong \mathbb{Q}[t^3,t^2]$,
  with quotient field $\mathbb{Q}(t)$ (by noting $t = \frac{x}{y}$).
  Note that $\mathbb{Q}[t]$ is a UFD, thus is already integrally closed.
  So the integral closure will be $\mathbb{Q}[t] \supsetneq \mathbb{Q}[t^3,t^2]$.
  It suggests that $A/\mathfrak{p}$ might not be integrally closed.

\item[(3)]
  (Reductio ad absurdum)
  If not, then the element $\frac{x}{y}$ satisfies a monic equation
  $t^2 - y = 0 \in (A/\mathfrak{p})[t]$.
  $\frac{x}{y} \in A/\mathfrak{p}$ or $t \in \mathbb{Q}[t^3,t^2]$, which is absurd.
\end{enumerate}
$\Box$ \\



\emph{Note.}
\begin{enumerate}
\item[(1)]
  Serre's criterion for normality.

\item[(2)]
  Hence smoothness is the same as normality for affine curves in $\mathbb{Q}[x,y]$.
  Note that $x^2-y^3$ is an irreducible cubic with a cusp at the origin $(0,0)$.

\item[(3)]
  There is an affine variety $X \in \mathbb{Q}[x,y,z]$ such that
  $X$ is normal but not smooth.
  ($X = V(x^2 + y^2 - z^2)$ for example.) \\\\
\end{enumerate}



%%%%%%%%%%%%%%%%%%%%%%%%%%%%%%%%%%%%%%%%%%%%%%%%%%%%%%%%%%%%%%%%%%%%%%%%%%%%%%%%



\subsubsection*{Exercise I.2.4.}
\addcontentsline{toc}{subsubsection}{Exercise I.2.4.}
\emph{Let $D$ be a squarefree rational integer $\neq 0, 1$ and
$d$ the discriminant of the quadratic number field $K = \mathbb{Q}(\sqrt{D})$.
Show that
\begin{equation*}
  d = \begin{cases}
    D  & \text{if $D \equiv 1 \pmod 4$,}\\
    4D & \text{if $D \equiv 2, 3 \pmod 4$.}
  \end{cases}
\end{equation*}
and that an integral basis of $K$ is given by $\{1, \sqrt{D}\}$ in the second case,
by $\left\{ 1, \frac{1+\sqrt{D}}{2} \right\}$ in the first case,
and by $\left\{ 1, \frac{d+\sqrt{d}}{2} \right\}$ in both case.} \\



\emph{Proof.}
\begin{enumerate}
\item[(1)]
  The Galois group of $K|\mathbb{Q}$ has two elements, the identity
  and an automorphism sending $\sqrt{D}$ to $-\sqrt{D}$.

\item[(2)]
  Note that $\alpha \in \mathcal{O}_K$
  iff $\mathrm{Tr}_{K|\mathbb{Q}}(\alpha), N_{K|\mathbb{Q}}(\alpha) \in \mathbb{Z}$
  (by noting that the equation
  $x^2 - \mathrm{Tr}_{K|\mathbb{Q}}(\alpha) x + N_{K|\mathbb{Q}}(\alpha) = 0$ has
  a root $x = \alpha$).
  So given $\alpha = x + y \sqrt{D} \in \mathcal{O}_K$, we have
  \begin{align*}
    \mathrm{Tr}_{K|\mathbb{Q}}(\alpha) &= 2x \in \mathbb{Z}, \\
    N_{K|\mathbb{Q}}(\alpha) &= x^2 - D y^2 \in \mathbb{Z}.
  \end{align*}

\item[(3)]
  So $4(x^2 - D y^2) = (2x)^2 - D(2y)^2 \in \mathbb{Z}$.
  So $D(2y)^2 \in \mathbb{Z}$ since $2x \in \mathbb{Z}$.
  So $2y \in \mathbb{Z}$ since $D$ is squarefree $\neq 0, 1$.
  Let $r = 2x, s = 2y$.
  Then $r^2 - Ds^2 \equiv 0 \pmod 4$.
  Note that a square $\equiv 0, 1 \pmod 4$.

\item[(4)]
  If $D \equiv 1 \pmod 4$, then
  \begin{align*}
    &\:r^2 - Ds^2 \equiv r^2 - s^2 \pmod 4 \\
    \Longrightarrow &\:
    \text{$r$ and $s$ has the same parity} \\
    \Longrightarrow &\:
    \mathcal{O}_K = \left\{ \frac{r+s\sqrt{D}}{2} : r \equiv s \pmod 2 \right\} \\
    \Longrightarrow &\:
    \mathcal{O}_K = \left\{ \frac{r-s}{2} + s \cdot \frac{1+\sqrt{D}}{2}
      : r \equiv s \pmod 2 \right\} \\
    \Longrightarrow &\:
    \mathcal{O}_K = \mathbb{Z} + \mathbb{Z} \frac{1+\sqrt{D}}{2}.
  \end{align*}
  So $\left\{ 1, \frac{1+\sqrt{D}}{2} \right\}$ is an integral basis of $K$.
  Hence
  \[
    d
    =
    \begin{vmatrix}
      1 & \frac{1+\sqrt{D}}{2} \\
      1 & \frac{1-\sqrt{D}}{2}
    \end{vmatrix}^2
    = D.
  \]

\item[(5)]
  If $D \equiv 2, 3 \pmod 4$, then
  \begin{align*}
    &\:r^2 - Ds^2 \equiv r^2+2s^2 \text{ or } r^2+s^2 \pmod 4 \\
    \Longrightarrow &\:
    \text{both $r$ and $s$ are even} \\
    \Longrightarrow &\:
    \text{both $x$ and $y$ are rational integers} \\
    \Longrightarrow &\:
    \mathcal{O}_K = \mathbb{Z} + \mathbb{Z}\sqrt{D}.
  \end{align*}
  So $\{ 1, \sqrt{D} \}$ is an integral basis of $K$.
  Hence
  \[
    d
    =
    \begin{vmatrix}
      1 & \sqrt{D} \\
      1 & -\sqrt{D}
    \end{vmatrix}^2
    = 4D.
  \]

\item[(6)]
  By (4)(5),
  $\left\{ 1, \frac{d+\sqrt{d}}{2} \right\}$ is an integral basis of $K$
  for any case.
\end{enumerate}
$\Box$ \\\\



%%%%%%%%%%%%%%%%%%%%%%%%%%%%%%%%%%%%%%%%%%%%%%%%%%%%%%%%%%%%%%%%%%%%%%%%%%%%%%%%



\subsubsection*{Exercise I.2.7. (Stickelberger's discriminant relation)}
\addcontentsline{toc}{subsubsection}{Exercise I.2.7. (Stickelberger's discriminant relation)}
\emph{The discriminant $d_K$ of an algebraic number field $K$ is always
$\equiv 0 \pmod 4$ or $\equiv 1 \pmod 4$.
(Hint: The discriminant $\det(\sigma_i \omega_j)$ of an integral basis $\omega_j$
is a sum of terms, each prefixed by a positive or a negative sign.
Writing $P$ (resp. $N$) for the sum of the positive (resp. negative) terms,
one find $d_K = (P-N)^2 = (P+N)^2 - 4PN$.)} \\



\emph{Proof (Hint).}
\begin{enumerate}
\item[(1)]
  Let $S_n$ be the symmetric group of degree $n$, and
  $A_n$ be the alternating group of degree $n$.
  So
  \begin{align*}
    \det(\sigma_i \omega_j)
    &= \sum_{\pi \in S_n}
      \left( \mathrm{sgn}(\pi) \prod_{i=1}^{n} \sigma_i \omega_{\pi(i)} \right) \\
    &= \underbrace{\sum_{\pi \in A_n} \prod_{i=1}^{n} \sigma_i \omega_{\pi(i)}}_{:= P}
      - \underbrace{\sum_{\pi \in S_n - A_n} \prod_{i=1}^{n} \sigma_i \omega_{\pi(i)}}_{:= N}.
  \end{align*}

\item[(2)]
  Note that $\sigma_i(P+N) = P+N$ and $\sigma_i(PN) = PN$ for all $\sigma_i$.
  Hence $P+N, PN \in \mathbb{Q}$.
  Therefore $P+N, PN \in \mathbb{Q} \cap \mathcal{O}_K = \mathbb{Z}$.

\item[(3)]
  By (1)(2),
  \begin{align*}
    d_K
    &= \det(\sigma_i \omega_j)^2 \\
    &= (P-N)^2 \\
    &= (P+N)^2 - 4PN \\
    &\equiv 0, 1 \pmod 4.
  \end{align*}
\end{enumerate}
$\Box$ \\\\



%%%%%%%%%%%%%%%%%%%%%%%%%%%%%%%%%%%%%%%%%%%%%%%%%%%%%%%%%%%%%%%%%%%%%%%%%%%%%%%%
%%%%%%%%%%%%%%%%%%%%%%%%%%%%%%%%%%%%%%%%%%%%%%%%%%%%%%%%%%%%%%%%%%%%%%%%%%%%%%%%
%%%%%%%%%%%%%%%%%%%%%%%%%%%%%%%%%%%%%%%%%%%%%%%%%%%%%%%%%%%%%%%%%%%%%%%%%%%%%%%%



\newpage
\section*{Chapter VII: Zeta Functions and $L$-series \\}
\addcontentsline{toc}{section}{Chapter VII: Zeta Functions and $L$-series}



\subsection*{VII.1. The Riemann Zeta Function \\}
\addcontentsline{toc}{subsection}{VII.1. The Riemann Zeta Function}



\subsubsection*{Exercise VII.1.4.}
\addcontentsline{toc}{subsubsection}{Exercise VII.1.4.}
\emph{For the power sum
\[
  s_k(n) = 1^k + 2^k + 3^k + \cdots + n^k
\]
one has}
\[
  s_k(n) = \frac{1}{k+1} (B_{k+1}(n) - B_{k+1}(0)).
\] \\



\emph{Proof.}
  By Exercise VII.1.3,
  \[
    x^k = \frac{1}{k+1} (B_{k+1}(x) - B_{k+1}(x-1)).
  \]
  Hence the telescoping sum is
  \begin{align*}
     s_k(n)
     &= \sum_{x=1}^{n} x^k \\
     &= \sum_{x=1}^{n} \frac{1}{k+1} (B_{k+1}(x) - B_{k+1}(x-1)) \\
     &= \frac{1}{k+1} (B_{k+1}(n) - B_{k+1}(0)).
  \end{align*}
$\Box$ \\\\



%%%%%%%%%%%%%%%%%%%%%%%%%%%%%%%%%%%%%%%%%%%%%%%%%%%%%%%%%%%%%%%%%%%%%%%%%%%%%%%%
%%%%%%%%%%%%%%%%%%%%%%%%%%%%%%%%%%%%%%%%%%%%%%%%%%%%%%%%%%%%%%%%%%%%%%%%%%%%%%%%
%%%%%%%%%%%%%%%%%%%%%%%%%%%%%%%%%%%%%%%%%%%%%%%%%%%%%%%%%%%%%%%%%%%%%%%%%%%%%%%%



\end{document}