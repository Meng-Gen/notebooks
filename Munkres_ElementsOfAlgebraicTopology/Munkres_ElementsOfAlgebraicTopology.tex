\documentclass{article}
\usepackage{amsfonts}
\usepackage{amsmath}
\usepackage{amssymb}
\usepackage{centernot}
\usepackage{hyperref}
\usepackage[none]{hyphenat}
\usepackage{mathrsfs}
\usepackage{mathtools}
\usepackage{physics}
\usepackage{tikz-cd}
\parindent=0pt



\title{\textbf{Notes on the book: \\ \emph{James R. Munkres, Elements of Algebraic Topology}}}
\author{Meng-Gen Tsai \\ plover@gmail.com}



\begin{document}
\maketitle
\tableofcontents



%%%%%%%%%%%%%%%%%%%%%%%%%%%%%%%%%%%%%%%%%%%%%%%%%%%%%%%%%%%%%%%%%%%%%%%%%%%%%%%%
%%%%%%%%%%%%%%%%%%%%%%%%%%%%%%%%%%%%%%%%%%%%%%%%%%%%%%%%%%%%%%%%%%%%%%%%%%%%%%%%
%%%%%%%%%%%%%%%%%%%%%%%%%%%%%%%%%%%%%%%%%%%%%%%%%%%%%%%%%%%%%%%%%%%%%%%%%%%%%%%%
%%%%%%%%%%%%%%%%%%%%%%%%%%%%%%%%%%%%%%%%%%%%%%%%%%%%%%%%%%%%%%%%%%%%%%%%%%%%%%%%



% Reference:



%%%%%%%%%%%%%%%%%%%%%%%%%%%%%%%%%%%%%%%%%%%%%%%%%%%%%%%%%%%%%%%%%%%%%%%%%%%%%%%%
%%%%%%%%%%%%%%%%%%%%%%%%%%%%%%%%%%%%%%%%%%%%%%%%%%%%%%%%%%%%%%%%%%%%%%%%%%%%%%%%
%%%%%%%%%%%%%%%%%%%%%%%%%%%%%%%%%%%%%%%%%%%%%%%%%%%%%%%%%%%%%%%%%%%%%%%%%%%%%%%%
%%%%%%%%%%%%%%%%%%%%%%%%%%%%%%%%%%%%%%%%%%%%%%%%%%%%%%%%%%%%%%%%%%%%%%%%%%%%%%%%



\newpage
\section*{Chapter 1: Homology Groups of a Simplicial Complex \\}
\addcontentsline{toc}{section}{Chapter 1: Homology Groups of a Simplicial Complex}



\subsection*{\S 1. Simplices \\}
\addcontentsline{toc}{subsection}{\S 1. Simplices}



\subsubsection*{Exercise 1.1.}
\addcontentsline{toc}{subsubsection}{Exercise 1.1.}
\emph{Verify properties (1)-(3) of simplices:}
\begin{enumerate}
\item[(1)]
  \emph{The barycentric coordinates $t_i(x)$ of $x$ with respect to $a_0, \ldots, a_n$
  are continuous functions of $x$.}

\item[(2)]
  \emph{}

\item[(3)]
  \emph{$\sigma$ is compact, convex set in $\mathbb{R}^N$,
  which equals the intersection of all convex sets in $\mathbb{R}^N$ containing $a_0, \ldots, a_n$.} \\
\end{enumerate}



\emph{Proof of property (1).}
\begin{enumerate}
\item[(1)]
  Let $\sigma$ be the $n$-simplex spanned by $a_0, \ldots, a_n$.
  It suffices to show that $t_i(x)$ is linear.
  Therefore $t_i(x)$ is automatically continuous
  (Theorem 9.7 in the textbook: \emph{Rudin, Principles of Mathematical Analysis, 3rd edition}).

\item[(2)]
  Let
  \[
    E
    =
    \left\{ x = \sum_{i=1}^{n} \widetilde{t_i}(x) a_i : \widetilde{t_i}(x) \in \mathbb{R} \right\}
    \supseteq \sigma
  \]
  be the plane spanned by $a_0, \ldots, a_n$.
  $\widetilde{t_i}(x)$ is well-defined on $E$ and thus $\widetilde{t_i}|_\sigma = t_i$
  (since $\{a_0, \ldots, a_n\}$ is geometrically independent in $\mathbb{R}^{N}$).
  So it suffices to show that $\widetilde{t_i}$ is linear.

\item[(3)]
  Suppose $x = \sum_{i=1}^{n} \widetilde{t_i}(x) a_i \in E$
  and $y = \sum_{i=1}^{n} \widetilde{t_i}(y) a_i \in E$.
  Then
  \[
    x + y = \sum_{i=1}^{n} (\widetilde{t_i}(x) + \widetilde{t_i}(y)) a_i.
  \]
  Note that the coefficient of $a_i$ is uniquely determined by $x + y$.
  Thus $\widetilde{t_i}(x+y) = \widetilde{t_i}(x) + \widetilde{t_i}(y)$.
  Similarly, $\widetilde{t_i}(rx) = r\widetilde{t_i}(x)$ for $r \in \mathbb{R}$.
  Hence $\widetilde{t_i}$ is linear.
\end{enumerate}
$\Box$ \\



\emph{Proof of property (2).}
\begin{enumerate}
\item[(1)]
\end{enumerate}
$\Box$ \\



\emph{Proof of property (3).}
\begin{enumerate}
\item[(1)]
  \emph{Show that $\sigma$ is compact.}

\item[(2)]
  \emph{Show that $\sigma$ is convex.}
  Given any
  $x = \sum_i t_i a_i \in \sigma$ (with $\sum_i t_i = 1$),
  $y = \sum_i s_i a_i \in \sigma$ (with $\sum_i s_i = 1$) and $0 < \lambda < 1$,
  it suffices to show that
  \[
    \lambda x + (1-\lambda) y \in \sigma.
  \]
  In fact,
  \begin{align*}
    \lambda x + (1-\lambda) y
    &= \lambda \sum_i t_i a_i \in \sigma + (1-\lambda) \sum_i s_i a_i \\
    &= \sum_i (\lambda t_i + (1-\lambda) s_i) a_i,  
  \end{align*}
  where each $\lambda t_i + (1-\lambda) s_i \geq 0$
  and
  \[
    \sum_i (\lambda t_i + (1-\lambda) s_i)
    = \lambda \sum_i t_i + (1-\lambda) \sum_i s_i
    = \lambda + (1-\lambda)
    = 1.
  \]
  So $\lambda x + (1-\lambda) y \in \sigma$.

\item[(3)]
  \emph{Let $\mathscr{C}$ be the collection
  of all convex sets in $\mathbb{R}^N$ containing $a_0, \ldots, a_n$.
  Show that $\sigma = \bigcap_{E \in \mathscr{C}} E$.}
  By (2), $\sigma \in \mathscr{C}$ and thus $\sigma \supseteq \bigcap_{E \in \mathscr{C}} E$.
  Conversely, suppose $E \in \mathscr{C}$.
  The convexity of $E$ implies that
  $\sum_i t_i a_i \in E$ whenever $\sum_i t_i = 1$ and  $t_i \geq 0$.
  Hence $\sigma \subseteq E$ and thus $\sigma \subseteq \bigcap_{E \in \mathscr{C}} E$.
\end{enumerate}
$\Box$ \\\\



%%%%%%%%%%%%%%%%%%%%%%%%%%%%%%%%%%%%%%%%%%%%%%%%%%%%%%%%%%%%%%%%%%%%%%%%%%%%%%%%
%%%%%%%%%%%%%%%%%%%%%%%%%%%%%%%%%%%%%%%%%%%%%%%%%%%%%%%%%%%%%%%%%%%%%%%%%%%%%%%%
%%%%%%%%%%%%%%%%%%%%%%%%%%%%%%%%%%%%%%%%%%%%%%%%%%%%%%%%%%%%%%%%%%%%%%%%%%%%%%%%
%%%%%%%%%%%%%%%%%%%%%%%%%%%%%%%%%%%%%%%%%%%%%%%%%%%%%%%%%%%%%%%%%%%%%%%%%%%%%%%%



\end{document}