\documentclass{article}
\usepackage{amsfonts}
\usepackage{amsmath}
\usepackage{amssymb}
\usepackage{hyperref}
\usepackage[none]{hyphenat}
\usepackage{mathrsfs}
\usepackage{physics}
\parindent=0pt

\def\upint{\mathchoice%
    {\mkern13mu\overline{\vphantom{\intop}\mkern7mu}\mkern-20mu}%
    {\mkern7mu\overline{\vphantom{\intop}\mkern7mu}\mkern-14mu}%
    {\mkern7mu\overline{\vphantom{\intop}\mkern7mu}\mkern-14mu}%
    {\mkern7mu\overline{\vphantom{\intop}\mkern7mu}\mkern-14mu}%
  \int}
\def\lowint{\mkern3mu\underline{\vphantom{\intop}\mkern7mu}\mkern-10mu\int}

\begin{document}



\textbf{\Large Chapter 8: Some Special Functions} \\\\



\emph{Author: Meng-Gen Tsai} \\
\emph{Email: plover@gmail.com} \\\\



% http://math.ucsd.edu/~lni/math140/HW140B_7_solutions.pdf
% http://www.math.ucsd.edu/~ibejenar/teaching/2019/140B/HW1S.pdf
% https://minds.wisconsin.edu/bitstream/handle/1793/67009/rudin%20ch%208.pdf?sequence=4
% https://math.mit.edu/~gs/cse/websections/cse41.pdf



%%%%%%%%%%%%%%%%%%%%%%%%%%%%%%%%%%%%%%%%%%%%%%%%%%%%%%%%%%%%%%%%%%%%%%%%%%%%%%%%
%%%%%%%%%%%%%%%%%%%%%%%%%%%%%%%%%%%%%%%%%%%%%%%%%%%%%%%%%%%%%%%%%%%%%%%%%%%%%%%%



\textbf{Supplement.} Fourier coefficients in Definition 8.9.
\begin{enumerate}
\item[(1)]
Write $$f(x) = a_0 + \sum_{n = 1}^{N}(a_n \cos(nx) + b_n \sin(nx)),
x \in \mathbb{R}$$
(as the textbook Rudin, Principles of Mathematical Analysis, Third Edition).
Then
\begin{align*}
a_0 &= \frac{1}{2 \pi} \int_{-\pi}^\pi f(x) dx. \\
a_n &= \frac{1}{\pi} \int_{-\pi}^\pi f(x) \cos(nx) dx, n \in \mathbb{Z}^+. \\
b_n &= \frac{1}{\pi} \int_{-\pi}^\pi f(x) \sin(nx) dx, n \in \mathbb{Z}^+.
\end{align*}

\item[(2)]
One might write in one different form,
$$f(x) = \frac{a_0}{2} + \sum_{n = 1}^{N}(a_n \cos(nx) + b_n \sin(nx)),
x \in \mathbb{R}.$$
The only difference between the new one and the old one is $a_0$,
so $a_0$ should be
$$a_0 = \frac{1}{\pi} \int_{-\pi}^\pi f(x) dx.$$

\item[(3)]
Again, one might write in one different form,
$$f(x) = \frac{a_0}{\sqrt{2}} + \sum_{n = 1}^{N}(a_n \cos(nx) + b_n \sin(nx)),
x \in \mathbb{R}.$$ Similarly, $a_0$ should be
$$a_0 = \frac{1}{\pi} \int_{-\pi}^\pi \frac{f(x)}{\sqrt{2}} dx.$$

\item[(4)]
Recall $f(x) = \sum_{-N}^{N} c_n e^{inx}$ ($x \in \mathbb{R}$) where
$$c_n = \frac{1}{2 \pi} \int_{-\pi}^\pi f(x) e^{-inx} dx.$$
The relations among $a_n$, $b_n$ of this textbook and $c_n$ are
\begin{align*}
c_0 &= a_0 \\
c_n &= \frac{1}{2} \left( a_n + i b_n \right), n \in \mathbb{Z}^+.
\end{align*}

\item[(5)]
In some textbooks (Henryk Iwaniec, Topics in Classical Automorphic Forms),
it is convenient to consider periodic functions $f$ of period $1$.
Define $$e(n) = e^{2 \pi i x} = \cos(2 \pi x) + i \sin(2 \pi x).$$
Any periodic and piecewise continuous function $f$ has the Fourier series representation
$$f(x) = \sum_{-\infty}^{\infty} a_n e(nx)$$
with coefficients given by
$$a_n = \int_{0}^{1} f(x) e(-nx) dx.$$
Here is one exercise for this representation.
\emph{Show that the fractional part of $x$, $\{x\} = x - [x]$, is given by
$$\{x\} = \frac{1}{2} - \sum_{n = 1}^{\infty} \frac{\sin(2 \pi nx)}{\pi n}.$$}
\end{enumerate}



\textbf{Supplement.} Parseval's theorem 8.16.
\begin{enumerate}
\item[(1)]
Given $$f(x) = a_0 + \sum_{n = 1}^{\infty}(a_n \cos(nx) + b_n \sin(nx)),
x \in \mathbb{R}.$$
Then
$$\frac{1}{\pi} \int_{-\pi}^\pi |f(x)|^2 dx
= 2 a_0^2 + \sum_{n = 1}^{\infty}(a_n^2 + b_n^2).$$
\item[(2)]
Given $$f(x) = \frac{a_0}{2} + \sum_{n = 1}^{\infty}(a_n \cos(nx) + b_n \sin(nx)),
x \in \mathbb{R}.$$
Then
$$\frac{1}{\pi} \int_{-\pi}^\pi |f(x)|^2 dx
= \frac{a_0^2}{2} + \sum_{n = 1}^{\infty}(a_n^2 + b_n^2).$$
\item[(3)]
Given $$f(x) = \frac{a_0}{\sqrt{2}} + \sum_{n = 1}^{\infty}(a_n \cos(nx) + b_n \sin(nx)),
x \in \mathbb{R}.$$
Then
$$\frac{1}{\pi} \int_{-\pi}^\pi |f(x)|^2 dx
= a_0^2 + \sum_{n = 1}^{\infty}(a_n^2 + b_n^2).$$ \\
\end{enumerate}



%%%%%%%%%%%%%%%%%%%%%%%%%%%%%%%%%%%%%%%%%%%%%%%%%%%%%%%%%%%%%%%%%%%%%%%%%%%%%%%%



\textbf{Exercise 8.1.}
\emph{Define
\begin{equation*}
  f(x) =
    \begin{cases}
      e^{-\frac{1}{x^2}} & (x \neq 0), \\
      0                  & (x = 0).
    \end{cases}
\end{equation*}
Prove that $f$ has derivatives of all orders at $x = 0$,
and that $f^{(n)}(0) = 0$ for $n = 1,2,3,\ldots$} \\

$f(x)$ is an example of non-analytic smooth function, that is,
infinitely differentiable functions are not necessarily analytic.
In this exercise, we will show that Taylor series of $f$ at the origin
converges everywhere to the zero function.
So the Taylor series does not equal $f(x)$ for $x \neq 0$.
Consequently, $f$ is not analytic at $x = 0$. \\

\emph{Proof.}
\begin{enumerate}
\item[(1)]
\emph{Show that
\[
  \lim_{x \rightarrow 0} g(x) e^{-\frac{1}{x^2}} = 0
\]
for any rational function $g(x) \in \mathbb{R}(x)$.}
  \begin{enumerate}
  \item[(a)]
  Write $g(x) = \frac{p(x)}{q(x)}$ for some $p(x), q(x) \in \mathbb{R}[x]$,
  $g(x) \neq 0$.

  \item[(b)]
  Write $q(x) = b_m x^m + b_{m - 1} x^{m - 1} + \cdots + b_0$.
  $q(x)$ is not identically zero, that is, there exists the unique coefficient
  of the least power of $x$ in $q(x)$ which is non-zero, say $b_M \neq 0$.

  \item[(c)]
  Thus,
  \[
    g(x) = \frac{p(x)/x^M}{q(x)/x^M}.
  \]
  The denominator of $g(x)$ tends to $b_M \neq 0$ as $x \rightarrow 0$.
  By the similar argument in Theorem 8.6(f), we have
  \[
    \frac{p(x)}{x^M} e^{-\frac{1}{x^2}} \rightarrow 0 \text{ as } x \rightarrow 0.
  \]
  Hence, $\lim_{x \rightarrow 0} g(x) e^{-\frac{1}{x^2}} = 0$
  for any $g(x) \in \mathbb{R}(x)$.
  \end{enumerate}

\item[(2)]
\emph{Given any real $x \neq 0$, show that
\[
  f^{(n)}(x) = g_n(x) e^{-\frac{1}{x^2}}
\]
for some rational function $g(x) \in \mathbb{R}(x)$.}
  \begin{enumerate}
  \item[(a)]
  Say $g_0(x) = 1 \in \mathbb{R}(x)$.

  \item[(b)]
  $\mathbb{R}(x)$ is a field.
  \emph{Show that $g'(x) \in \mathbb{R}(x)$ for any $g(x) \in \mathbb{R}(x)$.}
  Write $g(x) = \frac{p(x)}{q(x)}$ for some $p(x), q(x) \in \mathbb{R}[x]$, $q(x) \neq 0$.
  Thus
  \[
    g'(x) = \frac{p'(x)q(x) - p(x)q'(x)}{q(x)^2}.
  \]
  The numerator of $g'(x)$ is in $\mathbb{R}[x]$ since
  the differentiation operator on $\mathbb{R}[x]$ is closed in $\mathbb{R}[x]$.
  Also, the denominator of $g'(x) = q(x)^2 \neq 0$
  since $\mathbb{R}[x]$ is an integral domain.
  Therefore, $g'(x) \in \mathbb{R}(x)$.

  \item[(c)]
  Induction on $n$.
  For $n = 1$, we have
  \begin{align*}
    f'(x)
    &= g_0'(x) e^{-\frac{1}{x^2}}
      + g_0(x) \cdot \left( -\frac{1}{x^2} \right)' e^{-\frac{1}{x^2}} \\
    &= \left( g_0'(x) + g_0(x) \cdot \left( -\frac{1}{x^2} \right)' \right) e^{-\frac{1}{x^2}} \\
    &= g_1(x) e^{-\frac{1}{x^2}}
  \end{align*}
  where
  \[
    g_1(x) = g_0'(x) + g_0(x) \cdot \left(-\frac{1}{x^2}\right)' \in \mathbb{R}(x).
  \]
  Now assume that the conclusion holds for $n = k$.
  As $n = k + 1$, similar to the case $n = 1$,
  \[
    f^{(k + 1)}(x) = g_{k + 1}(x) e^{-\frac{1}{x^2}}
  \]
  where
  \[
    g_{k + 1}(x)
    = g_k'(x) + g_k(x) \cdot \left( -\frac{1}{x^2} \right)' \in \mathbb{R}(x).
  \]
  By induction, the conclusion is true.
  \end{enumerate}

\item[(3)]
Induction on $n$.
For $n = 1$, by (1) we have
\[
  f'(0) = \lim_{t \rightarrow 0} \frac{e^{- \frac{1}{t^2}} - 0}{t} = 0.
\]
Now assume that the statement holds for $n = k$.
As $n = k + 1$, by (1)(2) we have
\[
  f^{(k + 1)}(0)
  = \lim_{t \rightarrow 0} \frac{f^{(k)}(t) - f^{(k)}(0)}{t}
  = \lim_{t \rightarrow 0} \frac{g_k(t) e^{- \frac{1}{t^2}} - 0}{t} = 0.
\]
Thus, $f^{(n)}(0) = 0$ for $n \in \mathbb{Z}^+$.
\end{enumerate}
$\Box$ \\\\



%%%%%%%%%%%%%%%%%%%%%%%%%%%%%%%%%%%%%%%%%%%%%%%%%%%%%%%%%%%%%%%%%%%%%%%%%%%%%%%%



\textbf{Exercise 8.2.}
\emph{Let $a_{ij}$ be the number in the $i$th row and $j$th column of the array
\[
\begin{array}{rrrrr}
           -1 &           0 &           0 &      0 & \cdots \\
  \frac{1}{2} &          -1 &           0 &      0 & \cdots \\
  \frac{1}{4} & \frac{1}{2} &          -1 &      0 & \cdots \\
  \frac{1}{8} & \frac{1}{4} & \frac{1}{2} &     -1 & \cdots \\
       \vdots &      \vdots &      \vdots & \vdots & \ddots
\end{array}
\]
so that
\begin{equation*}
  a_{ij} =
    \begin{cases}
      0       & (i < j), \\
      -1      & (i = j), \\
      2^{j-i} & (i > j).
    \end{cases}
\end{equation*}
Prove that
\[
  \sum_{i} \sum_{j} a_{ij} = -2, \:\: \sum_{j} \sum_{i} a_{ij} = 0.
\]
} \\

Also see Theorem 8.3. \\

\emph{Proof (Brute-force).}
\begin{align*}
  \sum_{i} \sum_{j} a_{ij}
  &= \sum_{i=1}^{\infty} \left( \sum_{j = i} a_{ij} + \sum_{j < i} a_{ij} \right) \\
  &= \sum_{i=1}^{\infty} \left( -1 + \sum_{j=1}^{i-1} 2^{j-i} \right) \\
  &= \sum_{i=1}^{\infty} ( -1 + (1 - 2^{1-i}) ) \\
  &= \sum_{i=1}^{\infty} -2^{1-i} \\
  &= -2.
\end{align*}
\begin{align*}
  \sum_{j} \sum_{i} a_{ij}
  &= \sum_{j=1}^{\infty} \left( \sum_{i = j} a_{ij} + \sum_{i > j} a_{ij} \right) \\
  &= \sum_{j=1}^{\infty} \left( -1 + \sum_{i=j+1}^{\infty} 2^{j-i} \right) \\
  &= \sum_{j=1}^{\infty} ( -1 + 1 ) \\
  &= \sum_{j=1}^{\infty} 0 \\
  &= 0.
\end{align*}
$\Box$ \\\\



%%%%%%%%%%%%%%%%%%%%%%%%%%%%%%%%%%%%%%%%%%%%%%%%%%%%%%%%%%%%%%%%%%%%%%%%%%%%%%%%



\textbf{Exercise 8.3.}



%%%%%%%%%%%%%%%%%%%%%%%%%%%%%%%%%%%%%%%%%%%%%%%%%%%%%%%%%%%%%%%%%%%%%%%%%%%%%%%%



\textbf{Exercise 8.4.}



%%%%%%%%%%%%%%%%%%%%%%%%%%%%%%%%%%%%%%%%%%%%%%%%%%%%%%%%%%%%%%%%%%%%%%%%%%%%%%%%



\textbf{Exercise 8.5.}



%%%%%%%%%%%%%%%%%%%%%%%%%%%%%%%%%%%%%%%%%%%%%%%%%%%%%%%%%%%%%%%%%%%%%%%%%%%%%%%%



\textbf{Exercise 8.6.}
\emph{Suppose $f(x)f(y) = f(x + y)$ for all real $x$ and $y$.}
\begin{enumerate}
\item[(a)]
\emph{Assuming that $f$ is differentiable and not zero, prove that
$$f(x) = e^{cx}$$
where $c$ is a constant.}
\item[(b)]
\emph{Prove the same thing, assuming only that $f$ is continuous.} \\
\end{enumerate}

Part (b) implies part (a). We prove part (b) directly. \\

\emph{Proof of (b).}
\begin{enumerate}
\item[(1)]
Since $f(x)$ is not zero, there exists $x_0 \in \mathbb{R}$ such that $f(x_0) \neq 0$.
So $f(0)f(x_0) = f(x_0)$, or $f(0) = 1$ by cancelling $f(x_0) \neq 0$.

\item[(2)]
Next, $f(\frac{n}{m}) = f(\frac{1}{m})^n$ for $m \in \mathbb{Z}$, $n \in \mathbb{Z}^{+}$.
Since $f$ is continuous at $x = 0$, $f$ is positive in the neighborhood of $x = 0$.
That is, there exists $N \in \mathbb{Z}^{+}$ such that $f(\frac{1}{m}) > 0$
whenever $|m| \geq N$.
So, $f(\frac{n}{m}) = f(\frac{1}{m})^n > 0$.
(Since $f(\frac{n}{m}) = f(\frac{kn}{km})$ for any $k \in \mathbb{Z}^{+}$,
we can rescale $m$ to $km$ such that $|km| \geq N$.)
That is, $f$ is positive on $\mathbb{Q}$.
Since $\mathbb{Q}$ is dense in $\mathbb{R}$ and $f$ is continuous on $\mathbb{R}$,
$f$ is positive on $\mathbb{R}$.

\item[(3)]
Now let $c = \log f(1)$ (which is well-defined since $f > 0$).
We write $f(1)$ in the two ways.
Firstly, $f(1) = f(\frac{n}{n}) = f(\frac{1}{n})^n$ where $n \in \mathbb{Z}^{+}$.
Secondly, $f(1) = e^c = (e^{\frac{c}{n}})^n$.
Since the positive $n$-th root is unique (Theorem 1.21),
$f(\frac{1}{n}) = e^{\frac{c}{n}}$ for $n \in \mathbb{Z}^{+}$.
By $f(x)f(-x) = f(0) = 1$ or $f(-x) = \frac{1}{f(x)}$,
$f(-\frac{1}{n}) = \frac{1}{e^{\frac{c}{n}}} = e^{-\frac{c}{n}}$ for $n \in \mathbb{Z}^{+}$.
Therefore,
\[
  f\left( \frac{1}{m} \right) = e^{\frac{c}{m}} \text{ where } m \in \mathbb{Z}.
\]

\item[(4)]
By using
$f(\frac{n}{m}) = f(\frac{1}{m})^n$ for $m \in \mathbb{Z}$, $n \in \mathbb{Z}^{+}$ again,
$f(\frac{n}{m}) = e^{c \frac{n}{m}}$ where $m \in \mathbb{Z}, n \in \mathbb{Z}^{+}$, or
$$f(x) = e^{cx} \text{ where } x \in \mathbb{Q}.$$
Since $g(x) = f(x) - e^{cx}$ vanishes on a dense set of $\mathbb{Q}$
and $g$ is continuous on $\mathbb{R}$, $g$ vanishes on $\mathbb{R}$.
Therefore, $f(x) = e^{cx}$ for $x \in \mathbb{R}$.
\end{enumerate}
$\Box$ \\



\textbf{Supplement.}
\emph{Proof of (a).}

\begin{enumerate}
\item[(1)]
Since $f(x)$ is not zero, there exists $x_0 \in \mathbb{R}$ such that $f(x_0) \neq 0$.
So $f(0)f(x_0) = f(x_0)$, or $f(0) = 1$ by cancelling $f(x_0) \neq 0$.
\item[(2)]
Since $f$ is differentiable, for any $x \in \mathbb{R}$,
\begin{align*}
f'(x)
=& \lim_{h \rightarrow 0} \frac{f(x + h) - f(x)}{h} \\
=& \lim_{h \rightarrow 0} \frac{f(x)f(h) - f(x)}{h} \\
=& f(x) \lim_{h \rightarrow 0} \frac{f(h) - 1}{h} \\
=& f(x) \lim_{h \rightarrow 0} \frac{f(h) - f(0)}{h} \\
=& f(x) f'(0).
\end{align*}
Let $c = f'(0)$ be a constant. Then $f'(x) = c f(x)$.
So $f(x) = e^{cx}$ for $x \in \mathbb{R}$.
(To see this, let $g(x) = \frac{f(x)}{e^{cx}}$ be well-defined on $\mathbb{R}$. $g(0) = 1$.
$g'(x) = 0$ since $f'(x) = c f(x)$. So $g(x)$ is a constant, or $g(x) = 1$ since $g(0) = 1$.
Therefore, $f(x) = e^{cx}$ on $\mathbb{R}$.)
\end{enumerate}
$\Box$ \\



\textbf{Supplement.} Cauchy's functional equation.
\begin{enumerate}
\item[(1)]
\emph{(Cauchy's functional equation.) Suppose $f(x) + f(y) = f(x + y)$ for all real $x$ and $y$.
Assuming that $f$ is continuous, prove that $f(x) = cx$ where $c$ is a constant}. \\

Notice that we cannot let $g(x) = \log f(x)$
and apply Cauchy's functional equation on $g(x)$
to prove Exercise 8.6 since $f(x)$ is not necessary positive and thus
$g(x) = \log f(x)$ might be meaningless.
However, this wrong approach gives you some useful ideas such as
you need to prove that $f(x)$ is positive first,
and $f(x)$ should be equal to $e^{cx}$ where $c = g(1) = \log f(1)$.

\item[(2)]
\emph{Suppose $f(xy) = f(x) + f(y)$ for all positive real $x$ and $y$.
Assuming that $f$ is continuous, prove that $f(x) = c \log x$ where $c$ is a constant}.

\item[(3)]
\emph{Suppose $f(xy) = f(x)f(y)$ for all positive real $x$ and $y$.
Assuming that $f$ is continuous and positive,
prove that $f(x) = x^c$ where $c$ is a constant}.

\item[(4)]
\emph{Suppose $f(x + y) = f(x) + f(y) + xy$ for all real $x$ and $y$.
Assuming that $f$ is continuous,
prove that $f(x) = \frac{1}{2}x^2 + cx$ where $c$ is a constant}.

\item[(5)]
\emph{(USA 2002.) Suppose $f(x^2 - y^2) = x f(x) - y f(y)$ for all real $x$ and $y$.
Assuming that $f$ is continuous,
prove that $f(x) = cx$ where $c$ is a constant}. \\
\end{enumerate}



\textbf{Supplement.}
\emph{Show that the only automorphism of $\mathbb{Q}$ is the identity.} \\

\emph{Proof.}
Given any $\sigma \in \text{Aut}(\mathbb{Q})$.
\begin{enumerate}
\item[(1)]
\emph{Show that $\sigma(1) = 1$.}
Since $1^2 = 1$, $\sigma(1)\sigma(1) = \sigma(1)$. $\sigma(1) = 0$ or $1$.
There are only two possible cases.
  \begin{enumerate}
  \item[(a)]
  Assume that $\sigma(1) = 0$. So
  $$\sigma(a) = \sigma(a \cdot 1) = \sigma(a)\cdot \sigma(1) = \sigma(a) \cdot 0 = 0$$
  for any $a \in \mathbb{Q}$.
  That is, $\sigma = 0 \in \text{Aut}(\mathbb{Q})$, which is absurd.
  \item[(b)]
  Therefore, $\sigma(1) = 1$.
  \end{enumerate}
\item[(2)]
\emph{Show that $\sigma(n) = n$ for all $n \in \mathbb{Z}^+$.}
Write $n = 1 + 1 + \cdots + 1$ ($n$ times $1$).
Applying the additivity of $\sigma$, we have
$$\sigma(n) = \sigma(1) + \sigma(1) + \cdots + \sigma(1) = 1 + 1 + \cdots + 1 = n.$$
(Might use induction on $n$ to eliminate $\cdots$ symbols.)
\item[(3)]
\emph{Show that $\sigma(n) = n$ for all $n \in \mathbb{Z}$.}
By the additivity of $\sigma$, $\sigma(-n) = -\sigma(n) = -n$ for $n \geq 0$.
The result is established.
\end{enumerate}
For any $a = \frac{n}{m} \in \mathbb{Q}$ ($m, n \in \mathbb{Z}$, $n \neq 0$),
applying the multiplication of $\sigma$ on $am = n$,
that is,
$\sigma(a) \sigma(m) = \sigma(n)$. By (3), we have $\sigma(a)m = n$,
or $$\sigma(a) = \frac{m}{n} = a$$
provided $n \neq 0$,
or $\sigma$ is the identity.
$\Box$ \\\\



%%%%%%%%%%%%%%%%%%%%%%%%%%%%%%%%%%%%%%%%%%%%%%%%%%%%%%%%%%%%%%%%%%%%%%%%%%%%%%%%



\textbf{Exercise 8.8.}
\emph{For $n=0,1,2,\ldots$, and $x$ real, prove that
\[
  |\sin(nx)| \leq n |\sin x|.
\]
Note that this inequality may be false for other values of $n$.
For instance,
\[
  \abs{ \sin(\frac{1}{2} \pi) } > \frac{1}{2}|\sin \pi|.
\]} \\

\emph{Proof.}
Induction on $n$.
\begin{enumerate}
\item[(1)]
Note that
\[
  \sin(a+b) = \sin a \cos b + \cos a \sin b
\]
for any $a, b \in \mathbb{R}$.
\item[(2)]
$n = 0,1$ are clearly true.
\item[(3)]
Assume the induction hypothesis that for the single case $n = k$ holds,
meaning
\[
  |\sin(kx)| \leq k |\sin x|
\]
is true.
It follows that
\begin{align*}
  |\sin((k+1)x)|
  &= |\sin(kx) \cos x + \cos(kx) \sin x|
    &\text{((1))} \\
  &\leq |\sin(kx)| |\cos x| + |\cos(kx)| |\sin x|
    &\text{(Triangle inequality)} \\
  &\leq |\sin(kx)| + |\sin x|
    &\text{($|\cos(\cdot)| \leq 1$)} \\
  &\leq k |\sin x| + |\sin x|
    &\text{(Induction hypothesis)} \\
  &\leq (k+1)|\sin x|.
\end{align*}
\end{enumerate}
$\Box$ \\\\



%%%%%%%%%%%%%%%%%%%%%%%%%%%%%%%%%%%%%%%%%%%%%%%%%%%%%%%%%%%%%%%%%%%%%%%%%%%%%%%%



\textbf{Exercise 8.10.}
\emph{Prove that $\sum \frac{1}{p}$ diverges; the sum extends over all primes.} \\

There are many proofs of this result. We provide some of them. \\

\emph{Proof (Due to hint).}
Given $N$.
\begin{enumerate}
\item[(1)]
\emph{Show that
\[
  \sum_{n \leq N} \frac{1}{n}
  \leq \prod_{p \leq N} \left( 1 - \frac{1}{p} \right)^{-1}.
\]}

By the unique factorization theorem on $n \leq N$,
\[
  \sum_{n \leq N} \frac{1}{n}
  \leq \prod_{p \leq N} \left( 1 + \frac{1}{p} + \frac{1}{p^2} + \cdots \right)
  = \prod_{p \leq N} \left( 1 - \frac{1}{p} \right)^{-1}.
\]

\item[(2)]
By (1) and the fact that $\sum \frac{1}{n}$ diverges,
there are infinitely many primes.

\item[(3)]
\emph{Show that
\[
  \prod_{p \leq N} \left( 1 - \frac{1}{p} \right)^{-1}
  \leq \exp \left( \sum_{p \leq N} \frac{2}{p} \right).
\]}

By applying the inequality $(1 - x)^{-1} < e^{2x}$ where $x \in (0, \frac{1}{2}]$
on any prime $p$,
\[
  \left( 1 - \frac{1}{p} \right)^{-1} < \exp \left( \frac{2}{p} \right).
\]
Now multiplying the inequality over all primes $p \leq N$ and noticing that
$\exp(x) \cdot \exp(y) = \exp(x + y)$, we have
\[
  \prod_{p \leq N} \left( 1 - \frac{1}{p} \right)^{-1}
  \leq \exp \left( \sum_{p \leq N} \frac{2}{p} \right).
\]

\item[(4)]
By (1)(3),
\[
  \sum_{n \leq N} \frac{1}{n}
  \leq \exp \left( \sum_{p \leq N} \frac{2}{p} \right).
\]
Since $\sum_{n \leq N} \frac{1}{n}$ diverges, the result holds.
\end{enumerate}
$\Box$ \\



\emph{Proof (Due to Kenneth Ireland and Michael Rosen).}
The proof in Kenneth Ireland and Michael Rosen,
A Classical Introduction to Modern Number Theory, Second Edition (Theorem 3 in Chapter 2)
does not use the inequality $(1 - x)^{-1} < e^{2x}$ ($x \in (0, \frac{1}{2}]$) directly.
Instead, the authors take the logarithm on $(1 - p^{-1})^{-1}$ and estimate it.
(So the length of proof is longer than the proof due to hint.)
That is,
\begin{align*}
- \log(1 - p^{-1})
&= \sum_{n = 1}^{\infty} \frac{p^{-n}}{n} \\
&= \frac{1}{p} + \sum_{n = 2}^{\infty} \frac{p^{-n}}{n} \\
&< \frac{1}{p} + \sum_{n = 2}^{\infty} p^{-n} \\
&= \frac{1}{p} + \frac{p^{-2}}{1 - p^{-1}} \\
&< \frac{1}{p} + 2 \cdot \frac{1}{p^2}.
\end{align*}
Now we sum over all primes $p \leq N$,
$$\log \left( \prod_{p \leq N} \left( 1 - \frac{1}{p} \right)^{-1} \right)
< \sum_{p \leq N} \frac{1}{p} + 2 \sum_{p \leq N} \frac{1}{p^2}.$$
So
$$\log \sum_{n \leq N} \frac{1}{n}
< \sum_{p \leq N} \frac{1}{p} + 2 \sum_{p \leq N} \frac{1}{p^2}.$$
Notice that $\sum \frac{1}{n}$ diverges and $\sum \frac{1}{p^2}$ converges
(since $\sum \frac{1}{n^2}$ converges).
Therefore, $\sum \frac{1}{p}$ diverges.
$\Box$ \\



\emph{Proof (Due to I. Niven).}
It is an exercise in Kenneth Ireland and Michael Rosen,
A Classical Introduction to Modern Number Theory, Second Edition. See Exercise 27 in Chapter 2.

\begin{enumerate}
\item[(1)]
\emph{Show that ${\sum}' \frac{1}{n}$, the sum being over square free integers, diverges.}
For any positive integers $n$, we can write $n = a^2 b$ where $a \in \mathbb{Z}^+$ and
$b$ is a square free integer.
Given $N$,
$$\sum_{n \leq N} \frac{1}{n}
\leq \left(\sum_{a = 1}^{\infty} \frac{1}{a^2} \right)
\left( {\sum_{b \leq N}}' \frac{1}{b} \right).$$
Notice that $\sum_{a = 1}^{\infty} \frac{1}{a^2}$ converges.
Since $\sum_{n \leq N} \frac{1}{n} \rightarrow \infty$ as $N \rightarrow \infty$,
$\sum'_{b \leq N}\frac{1}{b} \rightarrow \infty$ as $N \rightarrow \infty$.

\item[(2)]
\emph{Show that
\[
  \prod_{p \leq N} ( 1 + \frac{1}{p} ) \rightarrow \infty \text{ as } N \rightarrow \infty.
\]}

By the unique factorization theorem on $n \leq N$,
$$\prod_{p \leq N} \left( 1 + \frac{1}{p} \right)
\geq {\sum_{n \leq N}}' \frac{1}{n}.$$
Since ${\sum_{n \leq N}}' \frac{1}{n} \rightarrow \infty$ as $N \rightarrow \infty$ by (1),
the conclusion is established.

\item[(3)]
By applying the inequality $e^x > 1 + x$ on any prime $p$,
$$\exp\left(\frac{1}{p}\right) > 1 + \frac{1}{p}.$$
Now multiplying the inequality over all primes $p \leq N$ and noticing that
$\exp(x) \cdot \exp(y) = \exp(x + y)$, we have
$$\exp\left(\sum_{p \leq N} \frac{1}{p} \right)
> \prod_{p \leq N} \left( 1 + \frac{1}{p} \right).$$
By (2),
$\exp\left(\sum_{p \leq N} \frac{1}{p} \right) \rightarrow \infty$ as $N \rightarrow \infty$, or
$\sum_{p \leq N} \frac{1}{p} \rightarrow \infty$ as $N \rightarrow \infty$.
\end{enumerate}
$\Box$ \\\\



%%%%%%%%%%%%%%%%%%%%%%%%%%%%%%%%%%%%%%%%%%%%%%%%%%%%%%%%%%%%%%%%%%%%%%%%%%%%%%%%



\textbf{Exercise 8.12.}
\emph{Suppose $0 < \delta < \pi$,
\begin{equation*}
  f(x) =
    \begin{cases}
      1 & \text{ if } |x| \leq \delta, \\
      0 & \text{ if } \delta < |x| \leq \pi,
    \end{cases}
\end{equation*}
and $f(x + 2\pi) = f(x)$ for all $x$.}

\begin{enumerate}
\item[(a)]
\emph{Compute the Fourier coefficients of $f$.}

\item[(b)]
\emph{Compute that
$$\sum_{n = 1}^{\infty} \frac{\sin(n\delta)}{n} = \frac{\pi - \delta}{2}
\:\:\:\:\:\:\:\:
(0 < \delta < \pi).$$}

\item[(c)]
\emph{Deduce from Parseval's theorem that
$$\sum_{n = 1}^{\infty} \frac{(\sin(n\delta))^2}{n^2 \delta} = \frac{\pi - \delta}{2}.$$}

\item[(d)]
\emph{Let $\delta \rightarrow 0$ and prove that
$$\int_{0}^{\infty} \left( \frac{\sin x}{x} \right)^2 dx
= \frac{\pi}{2}.$$}

\item[(e)]
\emph{Put $\delta = \frac{\pi}{2}$ in (c). What do you get?} \\
\end{enumerate}

It is a centered square pulse around $x = 0$ with shift $\delta$.
Besides, $f(x)$ is an even function. \\

\emph{Proof of (a).}
\begin{align*}
c_0
&= \frac{1}{2 \pi} \int_{-\pi}^\pi f(x) dx \\
&= \frac{1}{2 \pi} \int_{-\delta}^\delta dx \\
&= \frac{\delta}{\pi}.
\end{align*}
For $0 \neq n \in \mathbb{Z}$,
\begin{align*}
c_n
&= \frac{1}{2 \pi} \int_{-\pi}^\pi f(x) e^{-inx} dx \\
&= \frac{1}{2 \pi} \int_{-\delta}^\delta e^{-inx} dx \\
&= \frac{1}{2 \pi} \cdot \frac{2 \sin(n \delta)}{n} \\
&= \frac{\sin(n \delta)}{n \pi}.
\end{align*}
$\Box$ \\

\textbf{Supplement.} Find $a_n$ and $b_n$ of this textbook. \\
By (a), $a_0 = \frac{\delta}{\pi}$,
$a_n = \frac{2 \sin(n \delta)}{n \pi}$, $b_n = 0$ for $n \in \mathbb{Z}^+$.
Surely, we can compute $a_n$ and $b_n$ ($n > 0$) directly.
Since $f(x)$ is an even function, $b_n = 0$.
And
\begin{align*}
a_n
&= \frac{1}{\pi} \int_{-\pi}^\pi f(x) \cos(nx) dx \\
&= \frac{2}{\pi} \int_{0}^\delta \cos(nx) dx \\
&= \frac{2 \sin(n \delta)}{n \pi}.
\end{align*}

\emph{Proof of (b).}
Given $x = 0$, there are constants $\delta' = \delta > 0$ and $M = 1 < \infty$ such that
$$|f(0 + t) - f(0)| \leq M|t|$$ for all $t \in (-\delta', \delta')$.
By Theorem 8.14,
$$\sum_{-\infty}^{\infty} c_n = f(0).$$
Notice that $c_{-n} = c_n$ for $n \in \mathbb{Z}^+$, so
\begin{align*}
\frac{\delta}{\pi} + 2 \sum_{n = 1}^{\infty} \frac{\sin(n \delta)}{n \pi}
&= 1 \\
\sum_{n = 1}^{\infty} \frac{\sin(n \delta)}{n}
&= \frac{\pi - \delta}{2}.
\end{align*}
$\Box$ \\

We can also use the expression $a_n$ and $b_n$ to prove the same thing.
Besides, taking $\delta = 1$ yields
$$\sum_{n = 1}^{\infty} \frac{\sin n}{n} = \frac{\pi - 1}{2}.$$ \\

\emph{Proof of (c).}
Since $f(x)$ is a Riemann-integrable function with period $2 \pi$,
by Parseval's theorem
$$\frac{1}{2 \pi} \int_{-\pi}^\pi |f(x)|^2 dx = \sum_{-\infty}^{\infty} |c_n|^2.$$
So
$$\frac{\delta}{\pi}
= \frac{\delta^2}{\pi^2} + 2 \sum_{n = 1}^{\infty} \frac{(\sin(n\delta))^2}{n^2 \pi^2}, $$
or
$$\sum_{n = 1}^{\infty} \frac{(\sin(n\delta))^2}{n^2 \delta}
= \frac{\pi - \delta}{2}.$$
$\Box$ \\

Notices that
$$\sum_{n = 1}^{\infty} \frac{(\sin n)^2}{n^2} = \frac{\pi - 1}{2}$$
as $\delta = 1$. \\

\emph{Proof of (d).}
Given $\varepsilon > 0$.
By Exercise 6.8,
$$\int_{0}^{\infty} \left( \frac{\sin x}{x} \right)^2 dx$$ exists.
So there exists $b > 0$ such that
$$\left| \int_{0}^{b} \left( \frac{\sin x}{x} \right)^2 dx
- \int_{0}^{\infty} \left( \frac{\sin x}{x} \right)^2 dx
\right| < \frac{\varepsilon}{4}$$

By Supplement in Chapter 6, there exists
$\delta > 0$ such that for any partition
$P_m = \{ 0, \frac{b}{m}, \frac{2b}{m}, \ldots, \frac{(m - 1)b}{m}, b\}$ of $[0, b]$
with $\Vert P \Vert = \frac{b}{m} < \delta$, or $m > \frac{b}{\delta}$,
we have
\begin{align*}
\left|
\sum_{n = 1}^{m} \frac{(\sin(n \frac{b}{m}))^2}{(n \frac{b}{m})^2} \cdot \frac{b}{m}
- \int_{0}^{b} \left( \frac{\sin x}{x} \right)^2 dx
\right|
&< \frac{\varepsilon}{4}, \\
\left|
\sum_{n = 1}^{m} \frac{(\sin(n \frac{b}{m}))^2}{n^2 \frac{b}{m}}
- \int_{0}^{b} \left( \frac{\sin x}{x} \right)^2 dx
\right| &< \frac{\varepsilon}{4}.
\end{align*}
For simplicity we resize $\delta$ to $\delta < \pi$ to make $0 < \frac{b}{m} < \delta < \pi$.
Besides, since $\sum_{n = 1}^{\infty} \frac{1}{n^2}$ converges,
there exists $N > 0$ such that
$$\left|
\sum_{n = 1}^{\infty} \frac{(\sin(n \frac{b}{m}))^2}{n^2 \frac{b}{m}}
- \sum_{n = 1}^{m} \frac{(\sin(n \frac{b}{m}))^2}{n^2 \frac{b}{m}}
\right|
< \frac{\varepsilon}{4}$$
whenever $m \geq N$.
By (c),
$$\left|
\frac{\pi - \frac{b}{m}}{2}
- \sum_{n = 1}^{m} \frac{(\sin(n \frac{b}{m}))^2}{n^2 \frac{b}{m}}
\right|
< \frac{\varepsilon}{4}$$
whenever $m \geq N$.
Last, it is easy to get
$$\left|
\frac{\pi}{2}
- \frac{\pi - \frac{b}{m}}{2}
\right|
< \frac{\varepsilon}{4}$$
whenever $m > \frac{2b}{\varepsilon}$.
Now we have
$$\left|
\frac{\pi}{2}
- \int_{0}^{\infty} \left( \frac{\sin x}{x} \right)^2 dx
\right|
< \varepsilon
$$
whenever $m > \max(\frac{b}{\delta}, N, \frac{2b}{\varepsilon})$.
Since $\varepsilon$ is arbitrary,
$\int_{0}^{\infty} \left( \frac{\sin x}{x} \right)^2 dx
= \frac{\pi}{2}$.
$\Box$ \\


\emph{Proof of (e).}
$$\sum_{n = 1}^{\infty} \frac{1}{(2n - 1)^2} = \frac{\pi^2}{8}.$$
Write
\begin{align*}
\sum_{n = 1}^{\infty} \frac{1}{n^2}
&= \sum_{n = 1}^{\infty} \frac{1}{(2n - 1)^2} + \sum_{n = 1}^{\infty} \frac{1}{(2n)^2} \\
&= \sum_{n = 1}^{\infty} \frac{1}{(2n - 1)^2} + \frac{1}{4} \sum_{n = 1}^{\infty} \frac{1}{n^2},
\end{align*}
so
$$\sum_{n = 1}^{\infty} \frac{1}{n^2}
= \frac{4}{3} \sum_{n = 1}^{\infty} \frac{1}{(2n - 1)^2}
= \frac{\pi^2}{6}.$$
$\Box$ \\\\



%%%%%%%%%%%%%%%%%%%%%%%%%%%%%%%%%%%%%%%%%%%%%%%%%%%%%%%%%%%%%%%%%%%%%%%%%%%%%%%%



\textbf{Exercise 8.13.}
\emph{Put $f(x) = x$ if $0 \leq x < 2 \pi$, and apply Parseval's theorem to conclude that
$$\sum_{n = 1}^{\infty} \frac{1}{n^2} = \frac{\pi}{6}.$$}
\emph{Proof.}
\begin{align*}
c_0
&= \frac{1}{2 \pi} \int_{0}^{2 \pi} x dx \\
&= \pi,
\end{align*}
For $n \neq 0$,
\begin{align*}
c_n
&= \frac{1}{2 \pi} \int_{0}^{2 \pi} x e^{-inx} dx \\
&= \frac{1}{2 \pi} \left(
\left[ - \frac{1}{i n} x e^{-inx} \right]_{x = 0}^{x = 2 \pi}
- \int_{0}^{2 \pi} - \frac{1}{i n} e^{-inx} dx \right) \\
&= \frac{i}{n}.
\end{align*}
Since $f(x)$ is a Riemann-integrable function with period $2 \pi$,
by Parseval's theorem
$$\frac{1}{2 \pi} \int_{-\pi}^\pi |f(x)|^2 dx = \sum_{-\infty}^{\infty} |c_n|^2.$$
So
$$\frac{1}{2 \pi} \cdot \frac{(2 \pi)^3}{3}
= \pi^2 + 2 \sum_{n = 1}^{\infty} \frac{1}{n^2}, $$
or
$$\sum_{n = 1}^{\infty} \frac{1}{n^2}
= \frac{\pi^2}{6}.$$
$\Box$ \\\\

\textbf{Supplement.} \emph{
Put $f(x) = x^k$ if $k \in \mathbb{Z}^+$ and $0 \leq x < 2 \pi$.
Might show that
$$\sum_{n = 1}^{\infty} \frac{1}{n^{2k}} = r_k \pi^{2k}, r_k \in \mathbb{Q}.$$}

\end{document}