\documentclass{article}
\usepackage{amsfonts}
\usepackage{amsmath}
\usepackage{amssymb}
\usepackage{bm}
\usepackage{hyperref}
\usepackage[none]{hyphenat}
\usepackage{mathrsfs}
\usepackage{physics}
\parindent=0pt

\def\upint{\mathchoice%
    {\mkern13mu\overline{\vphantom{\intop}\mkern7mu}\mkern-20mu}%
    {\mkern7mu\overline{\vphantom{\intop}\mkern7mu}\mkern-14mu}%
    {\mkern7mu\overline{\vphantom{\intop}\mkern7mu}\mkern-14mu}%
    {\mkern7mu\overline{\vphantom{\intop}\mkern7mu}\mkern-14mu}%
  \int}
\def\lowint{\mkern3mu\underline{\vphantom{\intop}\mkern7mu}\mkern-10mu\int}

\begin{document}



\textbf{\Large Chapter 9: Functions of Several Variables} \\\\



\emph{Author: Meng-Gen Tsai} \\
\emph{Email: plover@gmail.com} \\\\



%%%%%%%%%%%%%%%%%%%%%%%%%%%%%%%%%%%%%%%%%%%%%%%%%%%%%%%%%%%%%%%%%%%%%%%%%%%%%%%%
%%%%%%%%%%%%%%%%%%%%%%%%%%%%%%%%%%%%%%%%%%%%%%%%%%%%%%%%%%%%%%%%%%%%%%%%%%%%%%%%



\textbf{Exercise 9.1.}
\emph{If $S$ is a nonempty subset of a vector space $X$,
prove (as asserted in Section 9.1) that the span of $S$ is a vector space.} \\

Denote the span of $S$ by $\mathrm{span}(S)$. \\

\emph{Proof.}
\begin{enumerate}
\item[(1)]
  Since $S \neq \varnothing$, there is $\mathbf{z} \in S$.
  So $1\mathbf{z} = \mathbf{z} \in \mathrm{span}(S) \neq \varnothing$.
  (In fact, $\mathrm{span}(S) \supseteq S$.)

\item[(2)]
  If $\mathbf{x}, \mathbf{y} \in \mathrm{span}(S)$,
  then there exist elements
  $\mathbf{x}_1, \ldots, \mathbf{x}_m$, $\mathbf{y}_1, \ldots, \mathbf{y}_n \in S$
  and scalars $a_1, \ldots, a_m$, $b_1, \ldots, b_n$ such that
  \begin{align*}
    \mathbf{x} &= a_1 \mathbf{x}_1 + \cdots + a_m \mathbf{x}_m, \\
    \mathbf{y} &= b_1 \mathbf{y}_1 + \cdots + b_n \mathbf{y}_n.
  \end{align*}
  Then
  \[
    \mathbf{x}+\mathbf{y}
    = a_1 \mathbf{x}_1 + \cdots + a_m \mathbf{x}_m
      + b_1 \mathbf{y}_1 + \cdots + b_n \mathbf{y}_n
  \]
  is a linear combination of the elements of $S$.
  For any scalar $c$,
  \[
    c\mathbf{x} = (ca_1) \mathbf{x}_1 + \cdots + (ca_m) \mathbf{x}_m
  \]
  is again linear combination of the elements of $S$.

\item[(3)]
  By (1)(2), $\mathrm{span}(S)$ is a vector space.
\end{enumerate}
$\Box$ \\

\emph{Note.}
Any subspace of $X$ that contains $S$ must also contain $\mathrm{span}(S)$. \\\\



%%%%%%%%%%%%%%%%%%%%%%%%%%%%%%%%%%%%%%%%%%%%%%%%%%%%%%%%%%%%%%%%%%%%%%%%%%%%%%%%
%%%%%%%%%%%%%%%%%%%%%%%%%%%%%%%%%%%%%%%%%%%%%%%%%%%%%%%%%%%%%%%%%%%%%%%%%%%%%%%%



\end{document}