\documentclass{article}
\usepackage{amsfonts}
\usepackage{amsmath}
\usepackage{amssymb}
\usepackage{bm}
\usepackage{hyperref}
\usepackage[none]{hyphenat}
\usepackage{mathrsfs}
\usepackage{physics}
\parindent=0pt

\def\upint{\mathchoice%
    {\mkern13mu\overline{\vphantom{\intop}\mkern7mu}\mkern-20mu}%
    {\mkern7mu\overline{\vphantom{\intop}\mkern7mu}\mkern-14mu}%
    {\mkern7mu\overline{\vphantom{\intop}\mkern7mu}\mkern-14mu}%
    {\mkern7mu\overline{\vphantom{\intop}\mkern7mu}\mkern-14mu}%
  \int}
\def\lowint{\mkern3mu\underline{\vphantom{\intop}\mkern7mu}\mkern-10mu\int}

\begin{document}



\textbf{\Large Chapter 9: Functions of Several Variables} \\\\



\emph{Author: Meng-Gen Tsai} \\
\emph{Email: plover@gmail.com} \\\\



% http://www2.math.uu.se/~astrombe/reellanalys2019/solutions_pub.pdf
% https://www.math.utah.edu/~toledo/3220hw4.pdf



%%%%%%%%%%%%%%%%%%%%%%%%%%%%%%%%%%%%%%%%%%%%%%%%%%%%%%%%%%%%%%%%%%%%%%%%%%%%%%%%
%%%%%%%%%%%%%%%%%%%%%%%%%%%%%%%%%%%%%%%%%%%%%%%%%%%%%%%%%%%%%%%%%%%%%%%%%%%%%%%%



\textbf{Exercise 9.1.}
\emph{If $S$ is a nonempty subset of a vector space $X$,
prove (as asserted in Section 9.1) that the span of $S$ is a vector space.} \\

Denote the span of $S$ by $\mathrm{span}(S)$. \\

\emph{Proof.}
\begin{enumerate}
\item[(1)]
  Since $S \neq \varnothing$, there is $\mathbf{z} \in S$.
  So $1\mathbf{z} = \mathbf{z} \in \mathrm{span}(S) \neq \varnothing$.
  (In fact, $\mathrm{span}(S) \supseteq S$.)

\item[(2)]
  If $\mathbf{x}, \mathbf{y} \in \mathrm{span}(S)$,
  then there exist elements
  $\mathbf{x}_1, \ldots, \mathbf{x}_m$, $\mathbf{y}_1, \ldots, \mathbf{y}_n \in S$
  and scalars $a_1, \ldots, a_m$, $b_1, \ldots, b_n$ such that
  \begin{align*}
    \mathbf{x} &= a_1 \mathbf{x}_1 + \cdots + a_m \mathbf{x}_m, \\
    \mathbf{y} &= b_1 \mathbf{y}_1 + \cdots + b_n \mathbf{y}_n.
  \end{align*}
  Then
  \[
    \mathbf{x}+\mathbf{y}
    = a_1 \mathbf{x}_1 + \cdots + a_m \mathbf{x}_m
      + b_1 \mathbf{y}_1 + \cdots + b_n \mathbf{y}_n
  \]
  is a linear combination of the elements of $S$.
  For any scalar $c$,
  \[
    c\mathbf{x} = (ca_1) \mathbf{x}_1 + \cdots + (ca_m) \mathbf{x}_m
  \]
  is again linear combination of the elements of $S$.

\item[(3)]
  By (1)(2), $\mathrm{span}(S)$ is a vector space.
\end{enumerate}
$\Box$ \\

\emph{Note.}
Any subspace of $X$ that contains $S$ must also contain $\mathrm{span}(S)$. \\\\



%%%%%%%%%%%%%%%%%%%%%%%%%%%%%%%%%%%%%%%%%%%%%%%%%%%%%%%%%%%%%%%%%%%%%%%%%%%%%%%%



\textbf{Exercise 9.2.}
\emph{Prove (as asserted in Section 9.6) that $BA$ is linear
if $A$ and $B$ are linear transformations.
Prove also that $A^{-1}$ is linear and invertible if $A$ is invertible.} \\

\emph{Proof.}
Use the notation in Definitions 9.6.
\begin{enumerate}
\item[(1)]
  \emph{Show that $BA$ is linear if $A$ and $B$ are linear transformations.}
  Let $X, Y, Z$ be vector spaces, $A \in L(X,Y)$ and $B \in L(Y,Z)$.
  \begin{enumerate}
  \item[(a)]
    Given any $\mathbf{x}_1, \mathbf{x}_2 \in X$.
    \begin{align*}
      (BA)(\mathbf{x}_1+\mathbf{x}_2)
      &= B(A(\mathbf{x}_1+\mathbf{x}_2)) \\
      &= B(A\mathbf{x}_1+A\mathbf{x}_2)
        & (\text{$A$ is a linear transformation}) \\
      &= B(A\mathbf{x}_1) + B(A\mathbf{x}_2)
        & (\text{$B$ is a linear transformation}) \\
      &= (BA)\mathbf{x}_1 + (BA)\mathbf{x}_2.
    \end{align*}

  \item[(b)]
    For any $\mathbf{x} \in X$ and scalar $c$,
    \begin{align*}
    (BA)(c\mathbf{x})
    &= B(A(c\mathbf{x})) \\
    &= B(cA\mathbf{x})
      & (\text{$A$ is a linear transformation}) \\
    &= cB(A\mathbf{x})
      & (\text{$B$ is a linear transformation}) \\
    &= c(BA)\mathbf{x}.
    \end{align*}
  \end{enumerate}
  By (a)(b), $BA \in L(X,Z)$.

\item[(2)]
  \emph{Show that $A^{-1}$ is linear if $A$ is invertible.}
  \begin{enumerate}
  \item[(a)]
    Given any $\mathbf{y}_1, \mathbf{y}_2 \in X$.
    Since $A$ is surjective,
    there exist $\mathbf{x}_1, \mathbf{x}_2 \in X$ such that
    \begin{align*}
      \mathbf{y}_1 &= A\mathbf{x}_1 \\
      \mathbf{y}_2 &= A\mathbf{x}_2.
    \end{align*}
    So
    \begin{align*}
      A^{-1}\mathbf{y}_1 &= A^{-1}(A\mathbf{x}_1) = \mathbf{x}_1 \\
      A^{-1}\mathbf{y}_2 &= A^{-1}(A\mathbf{x}_2) = \mathbf{x}_2
    \end{align*}
    (by Definitions 9.4).
    Hence
    \begin{align*}
      A^{-1}(\mathbf{y}_1+\mathbf{y}_2)
      &= A^{-1}(A\mathbf{x}_1+A\mathbf{x}_2) \\
      &= A^{-1}(A(\mathbf{x}_1+\mathbf{x}_2))
        & (\text{$A$ is a linear transformation}) \\
      &= \mathbf{x}_1+\mathbf{x}_2
        & (\text{Definitions 9.4}) \\
      &= A^{-1}\mathbf{y}_1+A^{-1}\mathbf{y}_2.
    \end{align*}

  \item[(b)]
    For any $\mathbf{y} \in X$ and scalar $c$,
    there is a corresponding $\mathbf{x} \in X$ such that $\mathbf{y} = A\mathbf{x}$
    since $A$ is surjective. So $A^{-1}\mathbf{y} = \mathbf{x}$ by Definition 9.4.
    Hence
    \begin{align*}
      A^{-1}(c\mathbf{y})
      &= A^{-1}(cA\mathbf{x}) \\
      &= A^{-1}(A(c\mathbf{x}))
        & (\text{$A$ is a linear transformation}) \\
      &= c\mathbf{x}
        & (\text{Definitions 9.4}) \\
      &= cA^{-1}\mathbf{y}.
    \end{align*}
  \end{enumerate}
  By (a)(b), $A^{-1} \in L(X)$.

\item[(3)]
  \emph{Show that $A^{-1}$ is invertible if $A$ is invertible.}
  It suffices to show that $A^{-1}$ is injective and surjective.
  \begin{enumerate}
  \item[(a)]
    \emph{Show that $A^{-1}$ is injective.}
    Given any $\mathbf{y}_1, \mathbf{y}_2 \in X$.
    Since $A$ is surjective,
    there exist $\mathbf{x}_1, \mathbf{x}_2 \in X$ such that
    \begin{align*}
      \mathbf{y}_1 &= A\mathbf{x}_1 \\
      \mathbf{y}_2 &= A\mathbf{x}_2.
    \end{align*}
    Suppose $A^{-1}\mathbf{y}_1 = A^{-1}\mathbf{y}_2$.
    So $A^{-1}(A\mathbf{x}_1) = A^{-1}(A\mathbf{x}_2)$,
    or $\mathbf{x}_1 = \mathbf{x}_2$,
    or $\mathbf{y}_1 = A\mathbf{x}_1 = A\mathbf{x}_2 = \mathbf{y}_2$.

  \item[(b)]
    \emph{Show that $A^{-1}$ is surjective.}
    For any $\mathbf{x} \in X$, there exists $A\mathbf{x} \in X$ such that
    $A^{-1}(A\mathbf{x}) = \mathbf{x}$ by Definitions 9.4.
  \end{enumerate}
\end{enumerate}
$\Box$ \\\\



%%%%%%%%%%%%%%%%%%%%%%%%%%%%%%%%%%%%%%%%%%%%%%%%%%%%%%%%%%%%%%%%%%%%%%%%%%%%%%%%



\textbf{Exercise 9.3.}
\emph{Assume $A \in L(X,Y)$ and $A\mathbf{x} = \mathbf{0}$ only when $\mathbf{x} = \mathbf{0}$.
Prove that $A$ is then $1$-$1$.} \\

\emph{Proof.}
Suppose $A\mathbf{x} = A\mathbf{y}$.
Since $A$ is a linear transformation,
$A(\mathbf{x}-\mathbf{y}) = A\mathbf{x} - A\mathbf{y} = \mathbf{0}$.
By assumption, $\mathbf{x}-\mathbf{y} = \mathbf{0}$
or $\mathbf{x} = \mathbf{y}$.
$\Box$ \\\\



%%%%%%%%%%%%%%%%%%%%%%%%%%%%%%%%%%%%%%%%%%%%%%%%%%%%%%%%%%%%%%%%%%%%%%%%%%%%%%%%



\textbf{Exercise 9.4.}
\emph{Prove (as asserted in Section 9.30) that null spaces and ranges of
linear transformations are vector spaces.} \\

\emph{Proof.}
Use the notation in Definitions 9.30.
Suppose $X$, $Y$ are vector spaces, and $A \in L(X,Y)$, as in Definition 9.6.
\begin{enumerate}
\item[(1)]
  \emph{Show that $\mathscr{N}(A)$ is a vector space in $X$.}
  \begin{enumerate}
  \item[(a)]
    Note that $\mathbf{0} \in X$.
    Since $A\mathbf{0} = \mathbf{0}$, $\mathbf{0} \in \mathscr{N}(A) \neq \varnothing$.

  \item[(b)]
    Suppose $\mathbf{x}_1, \mathbf{x}_2 \in \mathscr{N}(A)$.
    Then
    \begin{align*}
      A(\mathbf{x}_1+\mathbf{x}_2)
      &= A\mathbf{x}_1+A\mathbf{x}_2
        & (\text{$A$ is a linear transformation}) \\
      &= \mathbf{0}+\mathbf{0}
        & (\mathbf{x}_1, \mathbf{x}_2 \in \mathscr{N}(A)) \\
      &= \mathbf{0}.
    \end{align*}
    So $\mathbf{x}_1+\mathbf{x}_2 \in \mathscr{N}(A)$.

  \item[(c)]
    Suppose $\mathbf{x} \in \mathscr{N}(A)$ and $c$ is a scalar.
    Then
    \begin{align*}
      A(c\mathbf{x})
      &= cA\mathbf{x}
        & (\text{$A$ is a linear transformation}) \\
      &= c\mathbf{0}
        & (\mathbf{x} \in \mathscr{N}(A)) \\
      &= \mathbf{0}.
    \end{align*}
    So $c\mathbf{x} \in \mathscr{N}(A)$.
  \end{enumerate}
  By (a)(b)(c), $\mathscr{N}(A)$ is a vector space.

\item[(2)]
  \emph{Show that $\mathscr{R}(A)$ is a vector space in $Y$.}
  \begin{enumerate}
  \item[(a)]
    Note that $\mathbf{0} \in X$.
    So $A\mathbf{0} = \mathbf{0} \in \mathscr{R}(A) \neq \varnothing$.

  \item[(b)]
    Suppose $\mathbf{y}_1, \mathbf{y}_2 \in \mathscr{R}(A)$.
    Then there exist $\mathbf{x}_1, \mathbf{x}_2 \in X$
    such that $A\mathbf{x}_1 = \mathbf{y}_1$
    and $A\mathbf{x}_2 = \mathbf{y}_2$.
    Hence
    \begin{align*}
      \mathbf{y}_1+\mathbf{y}_2
      &= A\mathbf{x}_1+A\mathbf{x}_2 \\
      &= A(\mathbf{x}_1+\mathbf{x}_2)
        & (\text{$A$ is a linear transformation}).
    \end{align*}
    So $\mathbf{y}_1+\mathbf{y}_2 \in \mathscr{R}(A)$.

  \item[(c)]
    Suppose $\mathbf{y} \in \mathscr{R}(A)$ and $c$ is a scalar.
    Then there exists $\mathbf{x} \in X$ such that $A\mathbf{x} = \mathbf{y}$.
    Hence
    \begin{align*}
      c\mathbf{y}
      &= cA\mathbf{x} \\
      &= A(c\mathbf{x})
        & (\text{$A$ is a linear transformation}).
    \end{align*}
    So $c\mathbf{y} \in \mathscr{R}(A)$.
  \end{enumerate}
  By (a)(b)(c), $\mathscr{R}(A)$ is a vector space.
\end{enumerate}
$\Box$ \\\\



%%%%%%%%%%%%%%%%%%%%%%%%%%%%%%%%%%%%%%%%%%%%%%%%%%%%%%%%%%%%%%%%%%%%%%%%%%%%%%%%



\textbf{Exercise 9.5.}
\emph{Prove that to every $A \in L(\mathbb{R}^n, \mathbb{R}^1)$
corresponds a unique $\mathbf{y} \in \mathbb{R}^n$ such that
$A\mathbf{x} = \mathbf{x} \cdot \mathbf{y}$.
Prove also that $\norm{A} = |\mathbf{y}|$.
(Hint: Under certain conditions, equality holds in the Schwarz inequality.)} \\

\emph{Proof.}
\begin{enumerate}
\item[(1)]
  Recall that $\{ \mathbf{e}_1, \ldots, \mathbf{e}_n \}$
  is the standard basis of $\mathbb{R}^n$ (Definitions 9.1).
  Given any $\mathbf{x} \in \mathbb{R}^n$,
  write $\mathbf{x} = (x_1, \ldots, x_n)$ as $\mathbf{x} = \sum x_j \mathbf{e}_j$.

\item[(2)]
  \emph{Show that $\mathbf{y}$ exists.}
  Since $A$ is a linear transformation,
  \begin{align*}
    A\mathbf{x}
    &= A\left(\sum x_j \mathbf{e}_j\right) \\
    &= \sum x_j A\mathbf{e}_j \\
    &= (x_1, \ldots, x_n) \cdot (A\mathbf{e}_1, \ldots, A\mathbf{e}_n) \\
    &= \mathbf{x} \cdot \sum (A\mathbf{e}_j) \mathbf{e}_j.
  \end{align*}
  Define $\mathbf{y} = \sum (A\mathbf{e}_j) \mathbf{e}_j \in \mathbb{R}^n$
  so that $A\mathbf{x} = \mathbf{x} \cdot \mathbf{y}$.

\item[(3)]
  \emph{Show that $\mathbf{y}$ is unique.}
  Suppose there exists some $\mathbf{z} \in \mathbb{R}^n$
  such that $A\mathbf{x} = \mathbf{x} \cdot \mathbf{z}$.
  So
  \begin{align*}
    0
    &= A\mathbf{x} -  A\mathbf{x} \\
    &= \mathbf{x} \cdot \mathbf{y}-\mathbf{x} \cdot \mathbf{z} \\
    &= \mathbf{x} \cdot (\mathbf{y}-\mathbf{z})
  \end{align*}
  for any $\mathbf{x} \in \mathbb{R}^n$.
  In particular, take $\mathbf{x} = \mathbf{y}-\mathbf{z} \in \mathbb{R}^n$
  to get
  \[
    0
    = (\mathbf{y}-\mathbf{z}) \cdot (\mathbf{y}-\mathbf{z})
    = |\mathbf{y}-\mathbf{z}|^2
  \]
  or $\mathbf{y}-\mathbf{z} = \mathbf{0}$ or $\mathbf{y}=\mathbf{z}$.

\item[(4)]
  \emph{Show that $\norm{A} = |\mathbf{y}|$.}
  By the Schwarz inequality (Theorem 1.37(d)),
  \[
    |A\mathbf{x}| = |\mathbf{x} \cdot \mathbf{y}| \leq |\mathbf{x}||\mathbf{y}|
    \leq |\mathbf{y}|
  \]
  as $|\mathbf{x}| \leq 1$.
  Take the sup over all $|\mathbf{x}| \leq 1$ to get
  \[
    \norm{A} \leq |\mathbf{y}|.
  \]
  If $\mathbf{y} = \mathbf{0}$, then $\norm{A} = |\mathbf{y}| = 0$.
  If $\mathbf{y} \neq \mathbf{0}$,
  then the equality holds when
  $\mathbf{x} = \frac{\mathbf{y}}{|\mathbf{y}|} \in \mathbb{R}^n$.
  (Here $|\mathbf{x}| = 1$.)
\end{enumerate}
$\Box$ \\\\



%%%%%%%%%%%%%%%%%%%%%%%%%%%%%%%%%%%%%%%%%%%%%%%%%%%%%%%%%%%%%%%%%%%%%%%%%%%%%%%%



\textbf{Exercise 9.6.}
\emph{If $f(0,0) = 0$ and
\[
  f(x,y) = \frac{xy}{x^2+y^2}
  \qquad \text{if }
  (x,y) \neq (0,0),
\]
prove that
$(D_1 f)(x,y)$ and $(D_2 f)(x,y)$ exist at every point of $\mathbb{R}^2$,
although $f$ is not continuous at $(0,0)$.} \\

\emph{Proof.}
\begin{enumerate}
\item[(1)]
  \emph{Show that}
  \begin{equation*}
  (D_1 f)(x,y) =
    \begin{cases}
      0                              & \text{ if $(x,y)=(0,0)$}, \\
      \frac{y(y^2-x^2)}{(x^2+y^2)^2} & \text{ if $(x,y)\neq(0,0)$}.
    \end{cases}
  \end{equation*}
  Write
  \begin{align*}
    (D_1 f)(x,y)
    &= \lim_{t \to 0} \frac{f((x,y) + t(1,0)) - f(x,y)}{t} \\
    &= \lim_{t \to 0} \frac{f(x+t,y) - f(x,y)}{t}.
  \end{align*}
  If $(x,y) = (0,0)$,
  \[
    (D_1 f)(0,0)
    = \lim_{t \to 0} \frac{f(t,0) - f(0,0)}{t}
    = \lim_{t \to 0} \frac{0 - 0}{t}
    = 0.
  \]
  If $(x,y) \neq (0,0)$,
  \begin{align*}
    (D_1 f)(x,y)
    &= \lim_{t \to 0} \frac{f(x+t,y) - f(x,y)}{t} \\
    &= \lim_{t \to 0} \frac{\frac{(x+t)y}{(x+t)^2+y^2} - \frac{xy}{x^2+y^2}}{t} \\
    &= \lim_{t \to 0} \frac{y(y^2-x^2)-txy}{((x+t)^2+y^2)(x^2+y^2)} \\
    &= \frac{y(y^2-x^2)}{(x^2+y^2)^2}.
  \end{align*}

\item[(2)]
  \emph{Show that}
  \begin{equation*}
  (D_2 f)(x,y) =
    \begin{cases}
      0                              & \text{ if $(x,y)=(0,0)$}, \\
      \frac{x(x^2-y^2)}{(x^2+y^2)^2} & \text{ if $(x,y)\neq(0,0)$}.
    \end{cases}
  \end{equation*}
  Similar to (1).

\item[(3)]
  \emph{Show that $f$ is not continuous at $(0,0)$.}
  Note that
  \[
    \lim_{n \to \infty} f\left(\frac{1}{n},\frac{1}{n}\right)
    = \lim_{n \to \infty} \frac{\frac{1}{n} \cdot \frac{1}{n}}{\frac{1}{n^2}+\frac{1}{n^2}}
    = \lim_{n \to \infty} \frac{1}{2}
    = \frac{1}{2}
  \]
  and
  \[
    \lim_{n \to \infty} f\left(\frac{1}{n},0\right)
    = \lim_{n \to \infty} \frac{0}{\frac{1}{n^2} + 0}
    = \lim_{n \to \infty} 0
    = 0.
  \]
  Hence the limit $\lim_{(x,y) \to (0,0)} f(x,y)$ does not exist.
\end{enumerate}
$\Box$ \\\\



%%%%%%%%%%%%%%%%%%%%%%%%%%%%%%%%%%%%%%%%%%%%%%%%%%%%%%%%%%%%%%%%%%%%%%%%%%%%%%%%



\textbf{Exercise 9.7.}
\emph{Suppose that $f$ is a real-valued function
defined in an open set $E \subseteq \mathbb{R}^n$,
and that the partial derivatives $D_1 f, \ldots, D_n f$ are bounded in $E$.
Prove that $f$ is continuous in $E$.
(Hint: Proceed as in the proof of Theorem 9.21.)} \\

\emph{Proof.}
\begin{enumerate}
\item[(1)]
  Since $D_j f$ is bounded in $E$, there is a real number $M_j$ such that
  $|D_j f| \leq M_j$ in $E$.
  Take $M = \max_{1 \leq j \leq n} M_j$ so that $|D_j f| \leq M$ in $E$ for all $1 \leq j \leq n$.

\item[(2)]
  Fix $\mathbf{x} \in E$ and $\varepsilon > 0$.
  Since $E$ is open,
  there is an open neighborhood
  \[
    B(\mathbf{x};r)
    = \{ \mathbf{x}+\mathbf{h} \in E : |\mathbf{h}| < r \}
    \subseteq E
  \]
  with
  \[
    0 < r < \frac{\varepsilon}{n(M+1)}.
  \]

\item[(3)]
  Write $\mathbf{h} = \sum h_j \mathbf{e}_j$, $|\mathbf{h}| < r$,
  put $\mathbf{v}_0 = \mathbf{0}$,
  and $\mathbf{v}_k = h_1 \mathbf{e}_1 + \cdots + h_k \mathbf{e}_k$
  for $1 \leq k \leq n$.
  Then
  \[
    f(\mathbf{x}+\mathbf{h}) - f(\mathbf{x})
    = \sum_{j=1}^{n}
      [f(\mathbf{x}+\mathbf{v}_j) - f(\mathbf{x}+\mathbf{v}_{j-1})].
  \]
  Since $|\mathbf{v}_k| < r$ for $1 \leq k \leq n$ and
  since $B(\mathbf{x};r)$ is convex,
  the open interval with end points
  $\mathbf{x}+\mathbf{v}_{j-1}$ and $\mathbf{x}+\mathbf{v}_{j}$ lie in $B(\mathbf{x};r)$.
  Since $\mathbf{v}_{j} = \mathbf{v}_{j-1} - h_j\mathbf{e}_j$,
  the mean value theorem (Theorem 5.10) show that
  \[
    f(\mathbf{x}+\mathbf{v}_j) - f(\mathbf{x}+\mathbf{v}_{j-1})
    = h_j (D_j f)(\mathbf{x}+\mathbf{v}_{j-1} + \theta_j h_j \mathbf{e}_j)
  \]
  for some $\theta_j \in (0,1)$.

\item[(4)]
  Note that $\abs{h_j} \leq \abs{\mathbf{h}} < r < \frac{\varepsilon}{n(M+1)}$.
  Hence
  \begin{align*}
    \abs{f(\mathbf{x}+\mathbf{h}) - f(\mathbf{x})}
    &\leq \sum_{j=1}^{n}
      \abs{f(\mathbf{x}+\mathbf{v}_j) - f(\mathbf{x}+\mathbf{v}_{j-1})} \\
    &= \sum_{j=1}^{n}
      \abs{h_j} \abs{(D_j f)(\mathbf{x}+\mathbf{v}_{j-1} + \theta_j h_j \mathbf{e}_j)} \\
    &\leq \sum_{j=1}^{n}
      \frac{\varepsilon}{n(M+1)} \cdot M \\
    &< \varepsilon
  \end{align*}
  as $\abs{\mathbf{h}} < r < \frac{\varepsilon}{n(M+1)}$.
  Hence $f$ is continuous at all $\mathbf{x} \in E$.
\end{enumerate}
$\Box$ \\\\



%%%%%%%%%%%%%%%%%%%%%%%%%%%%%%%%%%%%%%%%%%%%%%%%%%%%%%%%%%%%%%%%%%%%%%%%%%%%%%%%



\textbf{Exercise 9.8.}
\emph{Suppose that $f$ is a differentiable real function in an open set
$E \subseteq \mathbb{R}^n$,
and that $f$ has a local maximum at a point $\mathbf{x} \in E$.
Prove that $f'(\mathbf{x}) = 0$.} \\

\emph{Proof (Theorem 5.8).}
\begin{enumerate}
\item[(1)]
Apply Theorem 5.8 to each $D_j f$ for $1 \leq j \leq n$.
Since $f$ has a local maximum at a point $\mathbf{x} \in E$,
there is an open neighborhood $B(\mathbf{x};r)$ of $\mathbf{x}$ in $E$
such that
\[
  f(\mathbf{y}) \leq f(\mathbf{x})
\]
for all $\mathbf{y} \in B(\mathbf{x};r)$.
Therefore,
\[
  f(\mathbf{x} + t\mathbf{e}_j) \leq f(\mathbf{x})
\]
for all $|t| < r$ and $1 \leq j \leq n$,
or $t \mapsto f(\mathbf{x} + t\mathbf{e}_j)$ has a local maximum at a point
$t = 0 \in (-r,r)$.


\item[(2)]
Since $f$ is a differentiable in $E$, each partial derivatives $D_j f$ exist (Theorem 9.21).
Hence Theorem 5.8 implies that $(D_j f)(\mathbf{x}) = 0$ for all $1 \leq j \leq n$.
So
\[
  f'(\mathbf{x})
  = [ (D_1 f)(\mathbf{x}) \cdots (D_k f)(\mathbf{x}) ]
  = [ 0 \cdots 0 ]
  = 0
\]
(as the zero matrix).
\end{enumerate}
$\Box$ \\\\



%%%%%%%%%%%%%%%%%%%%%%%%%%%%%%%%%%%%%%%%%%%%%%%%%%%%%%%%%%%%%%%%%%%%%%%%%%%%%%%%



\textbf{Exercise 9.9.}
\emph{If $\mathbf{f}$ is a differentiable mapping
of a connected open set $E \subseteq \mathbb{R}^n$,
and if $\mathbf{f}'(\mathbf{x}) = 0$ for every $\mathbf{x} \in E$,
prove that $\mathbf{f}$ is a constant in $E$.} \\

\emph{Proof.}
\begin{enumerate}
\item[(1)]
  \emph{Show that $\mathbf{f}$ is \textbf{locally constant}.}
  Given any $\mathbf{x} \in E$.
  Since $E$ is open, there exists an open neighborhood $B(\mathbf{x};r)$ of $\mathbf{x}$
  such that $B(\mathbf{x};r) \subseteq E$ and $r > 0$.
  Corollary to Theorem 9.19 implies that
  $\mathbf{f}$ is a constant on $B(\mathbf{x};r)$, that is,
  $\mathbf{f}$ is locally constant.

\item[(2)]
  \emph{Show that $\mathbf{f}$ is constant
  if $\mathbf{f}$ is locally constant in a connected set $E \subseteq \mathbb{R}^n$.}
  Might assume that $E \neq \varnothing$. (Otherwise there is nothing to do.)
  Take some $\mathbf{x}_0 \in E$.
  \begin{enumerate}
  \item[(a)]
    Let
    \[
      U = \{ \mathbf{y} \in E : \mathbf{f}(\mathbf{y}) = \mathbf{f}(\mathbf{x}_0) \}.
    \]

  \item[(b)]
    $U$ is open since $\mathbf{f}$ is locally constant (by (1)).
    (Take any $\mathbf{y} \in U$.
    Since $\mathbf{f}$ is locally constant,
    there is an open neighborhood $B(\mathbf{y}) \subseteq E$
    of $\mathbf{y}$ such that $f(\mathbf{z}) = f(\mathbf{y}) = f(\mathbf{x}_0)$
    whenever $\mathbf{z} \in B(\mathbf{y})$.
    So that $B(\mathbf{y}) \subseteq U$, or $U$ is open.)

  \item[(c)]
    Besides, since $\mathbf{f}$ is continuous (Remarks 9.13(c)),
    the set $U$ is closed.
    (The proof is the same as Proof (Definition 2.18(d)) in Exercise 4.3.)

  \item[(d)]
    So $U$ is open and closed.
    Write $E = U \cup (E - U)$.
    Here $U$ and $E - U$ are both open and closed.
    Hence $U \cap \overline{E-U} = U \cap (E-U) = \varnothing$
    and $\overline{U} \cap (E-U) = U \cap (E-U) = \varnothing$.
    Note that $\mathbf{x}_0 \in U \neq \varnothing$.
    By the connectedness of $E$, $E-U = \varnothing$, or $E = U$,
    or $\mathbf{f}$ is constant on $E$.
  \end{enumerate}

  \emph{Note.}
  \emph{The only subsets of a connected set $E$ which are both open and closed
  are $E$ and $\varnothing$.}

\end{enumerate}
$\Box$ \\\\



%%%%%%%%%%%%%%%%%%%%%%%%%%%%%%%%%%%%%%%%%%%%%%%%%%%%%%%%%%%%%%%%%%%%%%%%%%%%%%%%



\textbf{Exercise 9.10.}
\emph{If $f$ is a real function defined in a convex open set $E \subseteq \mathbb{R}^n$,
such that $(D_1 f)(\mathbf{x})$ = 0 for every $\mathbf{x} \in E$,
prove that $f(\mathbf{x})$ depends only on $x_2, \ldots, x_n$.
Show that the convexity of $E$ can be replaced by a weaker condition,
but that some condition is required.
For example, if $n=2$ and $E$ is shaped like a horseshoe,
the statement may be false.} \\

\emph{Proof.}
\begin{enumerate}
\item[(1)]
  It suffices to show that
  \[
    f(a, x_2, \ldots, x_n) = f(b, x_2, \ldots, x_n)
  \]
  whenever $\mathbf{x} = (a, x_2, \ldots, x_n) \in E$
  and $\mathbf{y} = (b, x_2, \ldots, x_n) \in E$
  if $(D_1 f)(\mathbf{x}) = 0$ in the convex open set $E$.

\item[(2)]
  Might assume that $a < b$.
  Since $g: t \mapsto f(t, x_2, \ldots, x_n)$ is a real continuous function
  on $[a,b]$ (by the openness of $E$) and differentiable in $(a,b)$
  (by the existence of $D_1 f$),
  \[
    g(b) - g(a) = (b - a)g'(\xi)
  \]
  for some $\xi \in (a,b)$.
  Note that
  \[
    g'(\xi) = (D_1 f)(\xi, x_2, \ldots, x_n) = 0
  \]
  by assumption. $g(b) = g(a)$ or $f(a, x_2, \ldots, x_n) = f(b, x_2, \ldots, x_n)$.

\item[(3)]
  (2) shows that the convexity of $E$ can be replaced by a weaker condition
  that $E \subseteq \mathbb{R}^n $ is convex in the first coordinate, say
  $E$ is open and
  \[
    \lambda \mathbf{x} + (1-\lambda) \mathbf{y}
    = (\lambda a + (1-\lambda) b, x_2, \ldots, x_n) \in E
  \]
  whenever
  $\mathbf{x} = (a, x_2, \ldots, x_n) \in E$,
  $\mathbf{y} = (b, x_2, \ldots, x_n) \in E$, and $0 < \lambda < 1$.

\item[(4)]
  \emph{Show that the convexity of $E$ or some weaker condition is required.}
  Define $f(x,y) = \mathrm{sgn}(x)$ on $E = \{ (x,y) \in \mathbb{R}^2 : x \neq 0 \}$.
  $E$ is open and $(D_1 f)(x,y) = 0$ in $E$.
  Note that $f(1989, 0) = 1$ and $f(-64, 0) = -1$,
  and thus $f(x,y)$ does not depend only on $y = 0$.
\end{enumerate}
$\Box$ \\\\



%%%%%%%%%%%%%%%%%%%%%%%%%%%%%%%%%%%%%%%%%%%%%%%%%%%%%%%%%%%%%%%%%%%%%%%%%%%%%%%%



\textbf{Exercise 9.11.}
\emph{If $f$ and $g$ are differentiable real functions in $\mathbb{R}^n$,
prove that
\[
  \nabla(fg) = f \nabla g + g \nabla f
\]
and that
\[
  \nabla\left(\frac{1}{f}\right) = -\frac{1}{f^2} \nabla f
\]
whenever $f \neq 0$.} \\

\emph{Proof.}
Recall Example 9.18:
\[
  (\nabla(f))(\mathbf{x}) = \sum_{i=1}^{n}(D_i f)(\mathbf{x})\mathbf{e}_i.
\]

\begin{enumerate}
\item[(1)]
  \emph{Show that $\nabla(fg) = f \nabla g + g \nabla f$.}
  For any $\mathbf{x} \in \mathbb{R}^n$,
  \begin{align*}
    (\nabla(fg))(\mathbf{x})
    &= \sum_{i=1}^{n}(D_i(fg))(\mathbf{x})\mathbf{e}_i \\
    &= \sum_{i=1}^{n}(g (D_i f) + f (D_i g))(\mathbf{x})\mathbf{e}_i
      &(\text{Theorem 5.3(b)}) \\
    &= \sum_{i=1}^{n} \left[ g(\mathbf{x}) (D_i f)(\mathbf{x})
      + f(\mathbf{x}) (D_i g)(\mathbf{x}) \right]\mathbf{e}_i \\
    &= g(\mathbf{x}) \sum_{i=1}^{n} (D_i f)(\mathbf{x})\mathbf{e}_i
      + f(\mathbf{x}) \sum_{i=1}^{n} (D_i g)(\mathbf{x})\mathbf{e}_i \\
    &= g(\mathbf{x}) (\nabla f)(\mathbf{x})
      + f(\mathbf{x}) (\nabla g)(\mathbf{x}) \\
    &= (f \nabla g + g \nabla f)(\mathbf{x}).
  \end{align*}

\item[(2)]
  \emph{Show that $$\nabla\left(\frac{1}{f}\right) = -\frac{1}{f^2} \nabla f$$
  whenever $f \neq 0$.}
  Note that $\nabla(1) = 0$
  since
  \[
    \nabla(1)(\mathbf{x})
    = \sum (D_i 1)(\mathbf{x})\mathbf{e}_i
    = \sum (0)(\mathbf{x})\mathbf{e}_i
    = \sum 0 \mathbf{e}_i
    = 0.
  \]
  Hence as $f \neq 0$, we have
  \begin{align*}
    0
    &= \nabla(1) \\
    &= \nabla\left(f \frac{1}{f}\right)
      & (f \neq 0) \\
    &= f \nabla\left(\frac{1}{f}\right) + \frac{1}{f} \nabla f
      & ((1)),
  \end{align*}
  or $\nabla\left(\frac{1}{f}\right) = -\frac{1}{f^2} \nabla f$.
\end{enumerate}
$\Box$ \\\\



%%%%%%%%%%%%%%%%%%%%%%%%%%%%%%%%%%%%%%%%%%%%%%%%%%%%%%%%%%%%%%%%%%%%%%%%%%%%%%%%



\textbf{Exercise 9.12.}
\emph{Fix two real numbers $a$ and $b$, $0<a<b$.
Define a mapping $\mathbf{f}=(f_1,f_2,f_3)$ of $\mathbb{R}^2$ into $\mathbb{R}^3$ by
\begin{align*}
  f_1(s,t) &= (b + a\cos s)\cos t \\
  f_2(s,t) &= (b + a\cos s)\sin t \\
  f_3(s,t) &= a \sin s.
\end{align*}
Describe the range $K$ if $\mathbf{f}$.
(It is a certain compact subset of $\mathbb{R}^3$.)}
\begin{enumerate}
\item[(a)]
  \emph{Show that there are exactly $4$ points $\mathbf{p} \in K$ such that
  \[
    (\nabla f_1)(\mathbf{f}^{-1}(\mathbf{p})) = \mathbf{0}.
  \]
  Find these points.}

\item[(b)]
  \emph{Determine the set of all $\mathbf{q} \in K$ such that
  \[
    (\nabla f_3)(\mathbf{f}^{-1}(\mathbf{q})) = \mathbf{0}.
  \]}
\item[(c)]
  \emph{Show that
  one of the point $\mathbf{p}$ found in part (a) corresponds to a local maximum of $f_1$,
  one corresponds to a local minimum,
  and that the other two are neither (they are so-called ``saddle points'').
  Which of the points $\mathbf{q}$ found in part (b) corresponds to maxima or minima?}

\item[(d)]
  \emph{Let $\lambda$ be an irrational real number, and define
  $\mathbf{g}(t) = \mathbf{f}(t,\lambda t)$.
  Prove that $\mathbf{g}$ is a one-to-one mapping of $\mathbb{R}^1$
  onto a dense subset of $K$.
  Prove that
  \[
    \abs{ \mathbf{g}'(t) }^2= a^2 + \lambda^2(b + a \cos t)^2.
  \]}
\end{enumerate}



\emph{Proof.}
\begin{enumerate}
\item[(1)]
  $K$ is a torus,
  where
  \begin{enumerate}
  \item[(a)]
    $s, t$ are angles which make a full circle
    (so that their values start and end at the same point).
  \item[(b)]
    $b$ is the distance from the center of the tube to the center of the torus.
  \item[(c)]
    $a$ is the radius of the tube.
  \end{enumerate}

\item[(2)]
  \emph{Show that $K$ is compact.}
  Since $\sin$ and $\cos$ are periodic (with period $2\pi$),
  $K = \mathbf{f}\left([0,2\pi]^2\right)$ is compact
  by the compactness of $[0,2\pi]^2$ and the continuity of $\mathbf{f}$ (Theorem 4.14).

\end{enumerate}
$\Box$ \\



\emph{Proof of (a).}
\begin{enumerate}
\item[(1)]
  \begin{align*}
    (\nabla f_1)(\mathbf{x})
    &= (D_1 f_1)(\mathbf{x})\mathbf{e}_1 + (D_2 f_1)(\mathbf{x})\mathbf{e}_2 \\
    &= ((D_1 f_1)(s,t), (D_2 f_1)(s,t)) \\
    &= (-a \sin s \cos t, -(b + a \cos t) \sin t)
  \end{align*}
  So
  $(\nabla f_1)(\mathbf{x}) = \mathbf{0}$
  if and only if
  \begin{align*}
    0 &= -a \sin s \cos t, \\
    0 &= -(b + a \cos t) \sin t.
  \end{align*}

\item[(2)]
  Note that $b + a \cos t > 0$ for any $b > a > 0$ and $t \in \mathbb{R}^1$.
  Hence $(\nabla f_1)(\mathbf{x}) = \mathbf{0}$ if and only if $\sin t = \sin s = 0$.
  Therefore,
  $\mathbf{p} = (\pm (b \pm a),0,0)$,
  or there are exactly $4$ points
  $\mathbf{p} = (b + a, 0, 0)$,
  $(b - a, 0, 0)$,
  $(-b - a, 0, 0)$, or
  $(-b + a, 0, 0) \in K$.
\end{enumerate}
$\Box$ \\



\emph{Proof of (b).}
\begin{enumerate}
\item[(1)]
  \begin{align*}
    (\nabla f_3)(\mathbf{x})
    &= (D_1 f_3)(\mathbf{x})\mathbf{e}_1 + (D_2 f_3)(\mathbf{x})\mathbf{e}_2 \\
    &= ((D_1 f_3)(s,t), (D_2 f_3)(s,t)) \\
    &= (a \cos s, 0)
  \end{align*}
  So
  $(\nabla f_1)(\mathbf{x}) = \mathbf{0}$
  if and only if $\cos s = 0$ (since $a > 0$).

\item[(2)]
  Therefore,
  $\mathbf{q} = (b\cos t, b\sin t, \pm a)$.
\end{enumerate}
$\Box$ \\



\emph{Proof of (c).}
\begin{enumerate}
\item[(1)]
  Since $-1 \leq \cos s \leq 1$ and $-1 \leq \cos t \leq 1$,
  $-b-a \leq f_1(s,t) \leq b+a$.

  \begin{enumerate}
  \item[(a)]
    $(b+a, 0, 0)$ corresponds to a local maximum of $f_1$.

  \item[(b)]
    $(-b-a, 0, 0)$ corresponds to a local minimum of $f_1$.

  \item[(c)]
    $(b - a, 0, 0)$ and $(-b + a, 0, 0)$ are saddle points
    by considering any open neighborhood of $(s,t)$
    at which $\cos s = \pm 1$ and $\cos t = \mp 1$.
  \end{enumerate}

\item[(2)]
  Since $-1 \leq \sin s \leq 1$,
  $-a \leq f_3(s,t) \leq a$.
  \begin{enumerate}
  \item[(a)]
    $(b\cos t, b\sin t, a)$ corresponds to a local maximum of $f_3$.

  \item[(b)]
    $(b\cos t, b\sin t, -a)$ corresponds to a local minimum of $f_3$.
  \end{enumerate}
\end{enumerate}
$\Box$ \\



\emph{Proof of (d).}
\begin{enumerate}
\item[(1)]
  \[
    \mathbf{g}(t)
    = \mathbf{f}(t,\lambda t)
    = ( (b+a\cos t)\cos(\lambda t), (b+a\cos t)\sin(\lambda t), a\sin t ).
  \]

\item[(2)]
  \emph{Show that $\mathbf{g}$ is a one-to-one mapping of $\mathbb{R}^1$.}
  It suffices to show that $\mathbf{g}(t) = \mathbf{g}(s)$ implies $t = s$.

  \begin{enumerate}
  \item[(a)]
    By $\mathbf{g}(t) = \mathbf{g}(s)$,
    \begin{align*}
      (b+a\cos t)\cos(\lambda t) &= (b+a\cos s)\cos(\lambda s) \tag{I}, \\
      (b+a\cos t)\sin(\lambda t) &= (b+a\cos s)\sin(\lambda s) \tag{II}, \\
      a\sin t &= a\sin s \tag{III}.
    \end{align*}
    (I) and (II) imply that $\cos t = \cos s$ (since $b > a > 0$).
    (III) implies that $\sin t = \sin s$.
    Hence
    \[
      t = s+2n\pi
    \]
    for some integer $n$.

  \item[(b)]
    Again, (I) and (II) imply that
    \[
      \cos(\lambda t) = \cos(\lambda s)
      \qquad \text{ and } \qquad
      \sin(\lambda t) = \sin(\lambda s).
    \]
    Hence
    \[
      \lambda t = \lambda s + 2m\pi
    \]
    for some integer $m$.
    By assumption that $t = s+2n\pi$, we have $m = n \lambda$.
    Since $\lambda$ is irrational, $m = n = 0$.
    Therefore $t = s$ holds.
  \end{enumerate}

\item[(3)]
  \emph{Show that $\mathbf{g}(\mathbb{R}^1)$ is dense in $K$.}
  Note that $\mathbf{f}\left([0,2\pi]^2\right) = K$.
  Use the notations $\{x\}$ in Exercise 4.16.
  It suffices to show that the set
  \[
    \left\{ \left(
      2\pi\left\{\frac{t}{2\pi}\right\},
      2\pi\left\{\frac{\lambda t}{2\pi}\right\}
    \right) : t \in \mathbb{R}^1 \right\}
  \]
  is dense in $[0,2\pi]^2$ (Exercise 4.4),
  or to show that
  \[
    \left\{ \left(\{ t \}, \{ \lambda t \} \right) : t \in \mathbb{R}^1 \right\}
  \]
  is dense in $[0,1]^2$, which is the conclusion of Exercise 4.25(b).

\item[(4)]
  \emph{Show that $\abs{ \mathbf{g}'(t) }^2= a^2 + \lambda^2(b + a \cos t)^2$.}
  By
  \begin{align*}
    \mathbf{g}'(t)
    = \left( \right.
        &-a\sin t \cos(\lambda t)-\lambda(b+a\cos t)\sin(\lambda t), \\
        &-a\sin t \sin(\lambda t)+\lambda(b+a\cos t)\cos(\lambda t), \\
        &a\cos t
      \left. \right),
  \end{align*}
  \begin{align*}
    \abs{ \mathbf{g}'(t) }^2
    =& \mathbf{g}'(t) \cdot \mathbf{g}'(t) \\
    =& (-a\sin t \cos(\lambda t)-\lambda(b+a\cos t)\sin(\lambda t))^2 \\
      &+ (-a\sin t \sin(\lambda t)+\lambda(b+a\cos t)\cos(\lambda t))^2 + (a\cos t)^2 \\
    =& \underbrace{a^2 \sin^2 t \cos^2(\lambda t) + a^2 \cos^2 t}_{= a^2} \\
      &+ \underbrace{\lambda^2(b+a\cos t)^2\sin^2(\lambda t)
        + \lambda^2(b+a\cos t)^2\cos^2(\lambda t)}_{= \lambda^2(b+a\cos t)^2} \\
      &+ 2a\lambda \sin t\cos(\lambda t)\lambda(b+a\cos t)\sin(\lambda t) \\
      &- 2a\lambda \sin t\sin(\lambda t)\lambda(b+a\cos t)\cos(\lambda t) \\
    =& a^2 + \lambda^2(b+a\cos t)^2.
  \end{align*}
\end{enumerate}
$\Box$ \\\\



%%%%%%%%%%%%%%%%%%%%%%%%%%%%%%%%%%%%%%%%%%%%%%%%%%%%%%%%%%%%%%%%%%%%%%%%%%%%%%%%



\textbf{Exercise 9.13.}
\emph{Suppose $\mathbf{f}$ is a differentiable mapping of $\mathbb{R}^1$ into
$\mathbb{R}^3$ such that $|\mathbf{f}(t)| = 1$ for every $t$.
Prove that $\mathbf{f}'(t) \cdot \mathbf{f}(t) = 0$.
Interpret this result geometrically.} \\

\emph{Proof.}
\begin{enumerate}
\item[(1)]
  Write $\mathbf{f} = (f_1, f_2, f_3)$ as a vector-valued function.
  By Remarks 5.16, $\mathbf{f}$ is differentiable if and only if each $f_1, f_2, f_3$
  is differentiable. So $\mathbf{f}' = (f_1', f_2', f_3)'$.
  Hence
  \begin{align*}
    &\text{$|\mathbf{f}(t)| = 1$ for every $t$} \\
    \Longleftrightarrow&
    \mathbf{f}(t) \cdot \mathbf{f}(t) = 1 \\
    \Longleftrightarrow&
    f_1(t)^2 + f_2(t)^2 + f_3(t)^2 = 1 \\
    \Longrightarrow&
    2 f_1(t) f_1'(t) + 2 f_2(t) f_2'(t) + 2 f_3(t) f_3'(t) = 0 \\
    \Longleftrightarrow&
    f_1(t) f_1'(t) + f_2(t) f_2'(t) + f_3(t) f_3'(t) = 0 \\
    \Longleftrightarrow&
    (f_1(t), f_2(t), f_3(t)) \cdot (f_1'(t), f_2'(t), f_3'(t)) = 0 \\
    \Longleftrightarrow&
    \mathbf{f}(t) \cdot \mathbf{f}'(t) = \mathbf{f}'(t) \cdot \mathbf{f}(t) = 0.
  \end{align*}

\item[(2)]
  The vector $\mathbf{f}'(t)$ is called the
  \emph{\textbf{tangent vector}} (or \emph{\textbf{velocity vector}})
  of $\mathbf{f}$ at $t$.
  Geometrically,
  given any mapping $\mathbf{f}$ lying on the sphere $S^2$,
  its tangent vector at $t$ is lying on the tangent plane of $S^2$ at $t$.
\end{enumerate}
$\Box$ \\\\



%%%%%%%%%%%%%%%%%%%%%%%%%%%%%%%%%%%%%%%%%%%%%%%%%%%%%%%%%%%%%%%%%%%%%%%%%%%%%%%%



\textbf{Exercise 9.14.}
\emph{Define $f(0,0) = 0$ and}
\[
  f(x,y) = \frac{x^3}{x^2 + y^2}
  \qquad \text{if }
  (x,y) \neq (0,0).
\]
\begin{enumerate}
\item[(a)]
  \emph{Prove that $D_1 f$ and $D_2 f$ are bounded functions in $\mathbb{R}^2$.
  (Hence $f$ is continuous.)}

\item[(b)]
  \emph{Let $\mathbf{u}$ be any unit vector in $\mathbb{R}^2$.
  Show that the directional derivative
  $(D_{\mathbf{u}}f)(0,0)$ exists, and that its absolute value is at most $1$.}

\item[(c)]
  \emph{Let $\gamma$ be a differentiable mapping of $\mathbb{R}^1$ into $\mathbb{R}^2$
  (in other words, $\gamma$ is a differentiable curve in $\mathbb{R}^2$),
  with $\gamma(t) = (0,0)$ and $\gamma'(t) \neq (0,0)$ for any $t \in \mathbb{R}^1$.
  Put $g(t) = f(\gamma(t))$ and prove that
  $g$ is differentiable for every $t \in \mathbb{R}^1$.
  If $\gamma \in \mathscr{C}'$, prove that $g \in \mathscr{C}'$.}

\item[(d)]
  \emph{In spite of this, prove that $f$ is not differentiable at $(0,0)$.} \\

\end{enumerate}



\emph{Proof of (a).}
\begin{enumerate}
\item[(1)]
  \emph{Show that}
    \begin{equation*}
    (D_1 f)(x,y) =
      \begin{cases}
        1
          & \text{ if $(x,y)=(0,0)$}, \\
        \frac{x^2(x^2 +3y^2)}{(x^2+y^2)^2}
          & \text{ if $(x,y)\neq(0,0)$}.
      \end{cases}
    \end{equation*}
    If $(x,y) = (0,0)$,
    \[
      (D_1 f)(0,0)
      = \lim_{t \to 0} \frac{f(t,0) - f(0,0)}{t}
      = \lim_{t \to 0} \frac{t - 0}{t}
      = 1.
    \]
    If $(x,y) \neq (0,0)$,
    \begin{align*}
      (D_1 f)(x,y)
      &= \lim_{t \to 0} \frac{f(x+t,y) - f(x,y)}{t} \\
      &= \lim_{t \to 0} \frac{\frac{(x+t)^3}{(x+t)^2+y^2} - \frac{x^3}{x^2+y^2}}{t} \\
      &= \lim_{t \to 0}
        \frac{x^2(x^2+3y^2) + tx(2x^2+3y^2) + t^2(x^2+y^2)}{((x+t)^2+y^2)(x^2+y^2)} \\
      &= \frac{x^2(x^2 +3y^2)}{(x^2+y^2)^2}.
    \end{align*}
    (Or differentiate directly.)

\item[(2)]
  \emph{Show that $(D_1 f)(x,y)$ is bounded.}
  It suffices to show that $(D_1 f)(x,y)$ is bounded if $(x,y) \neq (0,0)$.
  Write $x = r\cos\theta$ and $y = r\sin\theta$ in the polar coordinates.
  (Here $r > 0$.)
  Hence
  \[
    (D_1 f)(x,y)
    = \frac{x^2(x^2 +3y^2)}{(x^2+y^2)^2}
    = \cos^2\theta (\cos^2\theta + 3\sin^2\theta)
  \]
  is bounded by $1\cdot(1+3) = 4$.

\item[(3)]
  \emph{Show that}
    \begin{equation*}
    (D_2 f)(x,y) =
      \begin{cases}
        0
          & \text{ if $(x,y)=(0,0)$}, \\
        \frac{-2x^3y}{(x^2+y^2)^2}
          & \text{ if $(x,y)\neq(0,0)$}.
      \end{cases}
    \end{equation*}
    If $(x,y) = (0,0)$,
    \[
      (D_2 f)(0,0)
      = \lim_{t \to 0} \frac{f(0,t) - f(0,0)}{t}
      = \lim_{t \to 0} \frac{0 - 0}{t}
      = 0.
    \]
    If $(x,y) \neq (0,0)$,
    \begin{align*}
      (D_2 f)(x,y)
      &= \lim_{t \to 0} \frac{f(x,y+t) - f(x,y)}{t} \\
      &= \lim_{t \to 0} \frac{\frac{x^3}{x^2+(y+t)^2} - \frac{x^3}{x^2+y^2}}{t} \\
      &= \lim_{t \to 0}
        \frac{-2x^3y - tx^3}{(x^2+(y+t)^2)(x^2+y^2)} \\
      &= \frac{-2x^3y}{(x^2+y^2)^2}.
    \end{align*}
    (Or differentiate directly.)

\item[(4)]
  \emph{Show that $(D_2 f)(x,y)$ is bounded.}
  Similar to (2).

\item[(5)]
  \emph{Show that $f$ is continuous.}
  Apply Exercise 9.7 to (2)(4).
\end{enumerate}
$\Box$ \\



\emph{Proof of (b).}
\begin{enumerate}
\item[(1)]
  Write $\mathbf{u} = (u_1, u_2)$.
  The formula
  \[
    (D_{\mathbf{u}}f)(0,0) = (D_1 f)(0,0)u_1 + (D_2 f)(0,0)u_2 = u_1
  \]
  might be false since we don't know if $f$ is differentiable or not.
  Actually, we will show that $(D_{\mathbf{u}}f)(0,0) = u_1^3 \neq u_1$.

\item[(2)]
  \begin{align*}
    (D_{\mathbf{u}}f)(0,0)
    &= \lim_{t \to 0} \frac{f(tu_1,tu_2) - f(0,0)}{t} \\
    &= \lim_{t \to 0} \frac{\frac{t^3u_1^3}{t^2u_1^2 + t^2u_2^2} - 0}{t} \\
    &= \lim_{t \to 0} u_1^3
      & (|\mathbf{u}| = 1) \\
    &= u_1^3.
  \end{align*}
  Also $\abs{(D_{\mathbf{u}}f)(0,0)} = \abs{u_1}^3 \leq 1$
  since $|\mathbf{u}| = 1$.
\end{enumerate}
$\Box$ \\



\emph{Proof of (c).}
\begin{enumerate}
\item[(1)]
  Given any $t \in \mathbb{R}^1$.
  \[
    g'(t)
    = \lim_{x \to t} \frac{g(x)-g(t)}{x-t}
    = \lim_{x \to t} \frac{f(\gamma(x))-f(\gamma(t))}{x-t}.
  \]
  Write $\gamma(t) = (\gamma_1(t),\gamma_2(t))$.

\item[(2)]
  Suppose that $\gamma(t) \neq (0,0)$.
  Since $\gamma$ is differentiable, $\gamma$ is continuous.
  So there exists an open neighborhood $B(t) \subseteq \mathbb{R}^1$ of $t$
  such that $\gamma(x) \neq (0,0)$ whenever $x \in B(t)$.
  Hence
  \begin{align*}
    g'(t)
    &= \lim_{x \to t} \frac{\frac{\gamma_1(x)^3}{\gamma_1(x)^2+\gamma_2(x)^2}
      -\frac{\gamma_1(t)^3}{\gamma_1(t)^2+\gamma_2(t)^2}}{x-t} \\
    &= \frac{d}{dt}\left(\frac{\gamma_1(t)^3}{\gamma_1(t)^2+\gamma_2(t)^2}\right) \\
    &= \frac{3\gamma_1(t)^2\gamma_1'(t)}{\gamma_1(t)^2+\gamma_2(t)^2}
      - \frac{\gamma_1(t)^3(2\gamma_1(t)\gamma_1'(t)+2\gamma_2(t)\gamma_2'(t))}
        {(\gamma_1(t)^2+\gamma_2(t)^2)^2}.
  \end{align*}
  exists
  since $\gamma_1$ and $\gamma_2$ are differentiable.

\item[(3)]
  Suppose that $\gamma(t) = (0,0)$ and thus $\gamma'(t) \neq (0,0)$.
  So
  \[
    g'(t) = \lim_{x \to t} \frac{f(\gamma(x))}{x-t}
  \]
  Note that $\gamma(x) \neq (0,0)$ in some open neighborhood of $t$
  since
  \[
    \lim_{\substack{x \to t \\ \gamma(x) = (0,0)}} \frac{\gamma(x) - \gamma(t)}{x-t}
    = (0,0),
  \]
  contrary to the assumption that $\gamma'(t) \neq (0,0)$.
  Note that $\gamma_1(t) = \gamma_2(t) = 0$.
  So
  \begin{align*}
    g'(t)
    &= \lim_{x \to t} \frac{f(\gamma(x))}{x-t} \\
    &= \lim_{x \to t}
      \frac{\gamma_1(x)^3}{\gamma_1(x)^2+\gamma_2(x)^2}
      \cdot \frac{1}{x-t} \\
    &= \lim_{x \to t}
      \frac{(\gamma_1(x)-\gamma_1(t))^3}
        {(\gamma_1(x)-\gamma_1(t))^2+(\gamma_2(x)-\gamma_2(t))^2}
      \cdot \frac{1}{x-t} \\
    &= \lim_{x \to t}
      \frac{\left(\frac{\gamma_1(x)-\gamma_1(t)}{x-t}\right)^3}
        {\left(\frac{\gamma_1(x)-\gamma_1(t)}{x-t}\right)^2
        +\left(\frac{\gamma_2(x)-\gamma_2(t)}{x-t}\right)^2} \\
    &= \frac{\gamma_1'(t)^3}{\gamma_1'(t)^2+\gamma_2'(t)^2}
  \end{align*}
  since $\gamma'(t) \neq (0,0)$.

\item[(4)]
By (2)(3), $g'(t)$ exists and
\begin{equation*}
  g'(t) =
    \begin{cases}
      \frac{3\gamma_1(t)^2\gamma_1'(t)}{\gamma_1(t)^2+\gamma_2(t)^2}
        - \frac{\gamma_1(t)^3(2\gamma_1(t)\gamma_1'(t)+2\gamma_2(t)\gamma_2'(t))}
          {(\gamma_1(t)^2+\gamma_2(t)^2)^2}
        & \text{ if $\gamma(t) \neq (0,0)$}, \\
      \frac{\gamma_1'(t)^3}{\gamma_1'(t)^2+\gamma_2'(t)^2}
        & \text{ if $\gamma(t) = (0,0)$}.
    \end{cases}
\end{equation*}

\item[(5)]
  Now suppose $\gamma \in \mathscr{C}'$.
  To show $g' \in \mathscr{C}'$, it suffices to show that
  \[
    \lim_{x \to t} g'(x) = g'(t)
  \]
  if $\gamma(t) = (0,0)$ since $g'(t)$ is always continuous if $\gamma(t) \neq (0,0)$.
  Here all $\gamma_1, \gamma_2, \gamma_1', \gamma_2'$ are continuous
  and $\gamma_1(t)^2+\gamma_2(t)^2 \neq 0$ by assumption.
  So
  \begin{align*}
    &\lim_{x \to t}
      \frac{3\gamma_1(x)^2\gamma_1'(x)}{\gamma_1(x)^2+\gamma_2(x)^2} \\
    =& \lim_{x \to t}
      \frac{3\left(\frac{\gamma_1(x)-\gamma_1(t)}{x-t}\right)^2\gamma_1'(x)}
      {\left(\frac{\gamma_1(x)-\gamma_1(t)}{x-t}\right)^2
        +\left(\frac{\gamma_2(x)-\gamma_2(t)}{x-t}\right)^2} \\
    =& \frac{3\gamma_1'(t)^2 \cdot \gamma_1'(t)}{\gamma_1'(t)^2+\gamma_2'(t)^2} \\
    =& \frac{3\gamma_1'(t)^3}{\gamma_1'(t)^2+\gamma_2'(t)^2}
  \end{align*}
  and similarly
  \begin{align*}
    &\lim_{x \to t}
      \frac{\gamma_1(t)^3(2\gamma_1(t)\gamma_1'(t)+2\gamma_2(t)\gamma_2'(t))}
        {(\gamma_1(t)^2+\gamma_2(t)^2)^2} \\
    =& \lim_{x \to t}
      \frac{\left(\frac{\gamma_1(x)-\gamma_1(t)}{x-t}\right)^3
        \left( 2\frac{\gamma_1(x)-\gamma_1(t)}{x-t}\gamma_1'(t)
        +2\frac{\gamma_2(x)-\gamma_2(t)}{x-t}\gamma_2'(t) \right)}
      {\left( \left(\frac{\gamma_1(x)-\gamma_1(t)}{x-t}\right)^2
        +\left(\frac{\gamma_2(x)-\gamma_2(t)}{x-t}\right)^2 \right)^2} \\
    =& \frac{\gamma_1'(t)^3 \cdot (2\gamma_1'(t)\gamma_1'(t) + 2\gamma_2'(t)\gamma_2'(t))}
      {(\gamma_1'(t)^2+\gamma_2'(t)^2)^2} \\
    =& \frac{2\gamma_1'(t)^3}{\gamma_1'(t)^2+\gamma_2'(t)^2}.
  \end{align*}
  Hence
  \[
    \lim_{x \to t} g'(x)
    = \frac{3\gamma_1'(t)^3}{\gamma_1'(t)^2+\gamma_2'(t)^2}
      - \frac{2\gamma_1'(t)^3}{\gamma_1'(t)^2+\gamma_2'(t)^2}
    = g'(t).
  \]

\end{enumerate}
$\Box$ \\



\emph{Proof of (d).}
  (Reductio ad absurdum)
  If $f$ were differentiable,
  then
  \[
    (D_{\mathbf{u}}f)(0,0) = (D_1 f)(0,0)u_1 + (D_2 f)(0,0)u_2 = u_1
  \]
  (Formula (40) in Chapter 9), contrary to (b)
  if we take $\mathbf{u} = \left(\frac{1}{64}, \frac{\sqrt{4095}}{64} \right)$.
$\Box$ \\\\



%%%%%%%%%%%%%%%%%%%%%%%%%%%%%%%%%%%%%%%%%%%%%%%%%%%%%%%%%%%%%%%%%%%%%%%%%%%%%%%%



\textbf{Exercise 9.15.}
\emph{Define $f(0,0) = 0$, and put
\[
  f(x,y) = x^2+y^2-2x^2y-\frac{4x^6y^2}{(x^4+y^2)^2}
\]
if $(x,y) \neq (0,0)$.}
\begin{enumerate}
\item[(a)]
  \emph{Prove, for all $(x,y) \in \mathbb{R}^2$, that
  \[
    4x^4y^2 \leq (x^4+y^2)^2.
  \]
  Conclude that $f$ is continuous.}

\item[(b)]
  \emph{For $0 \leq \theta \leq 2\pi$, $-\infty < t < \infty$, define
  \[
    g_{\theta}(t) = f(t\cos\theta, t\sin\theta).
  \]
  Show that $g_{\theta}(0) = 0$, $g_{\theta}'(0) = 0$, $g_{\theta}''(0) = 2$.
  Each $g_{\theta}$ has therefore a strict local minimum at $t=0$.
  In other words, the restriction of $f$ to each line through $(0,0)$
  has a strict local minimum at $(0,0)$.}

\item[(c)]
  \emph{Show that $(0,0)$ is nevertheless not a local minimum for $f$,
  since $f(x,x^2)=-x^4$.} \\
\end{enumerate}



\emph{Proof of (a).}
\begin{enumerate}
\item[(1)]
  Since $t^2 \geq 0$ for all $t \in \mathbb{R}^1$,
  \[
    (x^4+y^2)^2 - 4x^4y^2 = (x^4-y^2)^2 \geq 0.
  \]
  Hence $4x^4y^2 \leq (x^4+y^2)^2$.
\item[(2)]
  $f(x,y)$ is continuous at $(x,y) \neq (0,0)$.
  Besides,
  \begin{align*}
    \abs{f(x,y)}
    &= \abs{x^2+y^2-2x^2y-\frac{4x^6y^2}{(x^4+y^2)^2}} \\
    &\leq \abs{x^2}+\abs{y^2}+\abs{2x^2y}+\abs{x^2}\abs{\frac{4x^4y^2}{(x^4+y^2)^2}} \\
    &\leq \abs{x^2}+\abs{y^2}+\abs{2x^2y}+\abs{x^2}.
  \end{align*}
  Hence
  $\abs{x^2}+\abs{y^2}+\abs{2x^2y}+\abs{x^2} \to 0$ as $(x,y) \to (0,0)$,
  or
  \[
    \lim_{(x,y) \to (0,0)} \abs{f(x,y)} = 0 = f(0,0),
  \]
  or $\lim_{(x,y) \to (0,0)} f(x,y) = f(0,0)$,
  or $f(x,y)$ is continuous at $(0,0)$.
\end{enumerate}
$\Box$ \\



\emph{Proof of (b).}
\begin{enumerate}
\item[(1)]
  \begin{equation*}
    g_{\theta}(t) =
      \begin{cases}
        t^2 - 2t^3 \cos^2\theta\sin\theta
          - \frac{4t^4\cos^6\theta\sin^2\theta}{(t^2\cos^4\theta+\sin^2\theta)^2}
          & \text{ if $t \neq 0$}, \\
        0
          & \text{ if $t = 0$}.
      \end{cases}
  \end{equation*}
  (Note that $\frac{4t^4\cos^6\theta\sin^2\theta}{(t^2\cos^4\theta+\sin^2\theta)^2}$
  is undefined as $t = 0$ and $\sin\theta = 0$.)

\item[(2)]
  $g_{\theta}(0) = 0$ by definition.

\item[(3)]
  \emph{Show that $g_{\theta}'(0) = 0$ for any $\theta \in [0,2\pi]$.}
  If $\sin\theta \neq 0$ ($\theta \neq 0, \pi, 2\pi$), then
  \begin{align*}
    g_{\theta}'(0)
    &= \lim_{t \to 0}
      \frac{t^2 - 2t^3 \cos^2\theta\sin\theta
        - \frac{4t^4\cos^6\theta\sin^2\theta}{(t^2\cos^4\theta+\sin^2\theta)^2} - 0}{t} \\
    &= \lim_{t \to 0}
      \left( t - 2t^2 \cos^2\theta\sin\theta
        - \frac{4t^3\cos^6\theta\sin^2\theta}{(t^2\cos^4\theta+\sin^2\theta)^2} \right) \\
    &= 0.
  \end{align*}
  If $\sin\theta = 0$, then
  \[
    g_{\theta}'(0)
    = \lim_{t \to 0} \frac{t^2 - 0}{t}
    = \lim_{t \to 0} t
    = 0.
  \]

\item[(4)]
  Combine (3) and a direct calculation for the case $t \neq 0$, we have
  \begin{equation*}
    g_{\theta}'(t) =
      \begin{cases}
        2t - 6t^2\cos^2\theta\sin\theta
          - \frac{16t^3\cos^6\theta\sin^4\theta}{(t^2\cos^4\theta+\sin^2\theta)^3}
          & \text{ if $t \neq 0$}, \\
        0
          & \text{ if $t = 0$}.
      \end{cases}
  \end{equation*}

\item[(5)]
  \emph{Show that $g_{\theta}''(0) = 2$ for any $\theta \in [0,2\pi]$.}
  If $\sin\theta \neq 0$ ($\theta \neq 0, \pi, 2\pi$), then
  \begin{align*}
    g_{\theta}''(0)
    &= \lim_{t \to 0}
      \frac{2t - 6t^2\cos^2\theta\sin\theta
        - \frac{16t^3\cos^6\theta\sin^4\theta}{(t^2\cos^4\theta+\sin^2\theta)^3} - 0}{t} \\
    &= \lim_{t \to 0}
      \left( t - 6t\cos^2\theta\sin\theta
        - \frac{16t^2\cos^6\theta\sin^4\theta}{(t^2\cos^4\theta+\sin^2\theta)^3} \right) \\
    &= 2.
  \end{align*}
  If $\sin\theta = 0$, then
  \[
    g_{\theta}''(0)
    = \lim_{t \to 0} \frac{2t - 0}{t}
    = \lim_{t \to 0} 2
    = 2.
  \]

\item[(6)]
  Since $g_{\theta}''(0) > 0$ and $g_{\theta}'(0) = 0$,
  $g_{\theta}$ has a strict local minimum at $t=0$.
  As $\theta$ is fixed, $f$ is restricted to some line through $(0,0)$.
  Hence, such restriction of $f$ has a strict local minimum at $t=0$.
\end{enumerate}
$\Box$ \\



\emph{Proof of (c).}
  Since $f(x,x^2) = -x^4 \leq 0 = f(0,0)$ in any open neighborhood of $(0,0)$,
  $f(0,0) = 0$ cannot be a local minimum for $f$.
$\Box$ \\\\



%%%%%%%%%%%%%%%%%%%%%%%%%%%%%%%%%%%%%%%%%%%%%%%%%%%%%%%%%%%%%%%%%%%%%%%%%%%%%%%%



\textbf{Exercise 9.16.}
\emph{Show that the continuity of $\mathbf{f}'$ at the point $\mathbf{a}$
is needed in the inverse function theorem, even in the case $n=1$: If
\[
  f(t) = t + 2t^2 \sin \frac{1}{t}
\]
for $t \neq 0$, and $f(0) = 0$, then $f'(0) = 1$, $f'$ is bouneded in $(-1,1)$,
but $f$ is not one-to-one in any neighborhood of $0$.} \\

\emph{Proof.}
\begin{enumerate}
\item[(1)]
  \emph{Show that}
  \begin{equation*}
    f'(t) =
      \begin{cases}
        1 + 4t\sin\frac{1}{t} - 2\cos\frac{1}{t}
          & \text{ if $t \neq 0$}, \\
        1
          & \text{ if $t = 0$}.
      \end{cases}
  \end{equation*}
  It suffices to show that $f'(0) = 1$.
  In fact,
  \[
    f'(0)
    = \lim_{t \to 0} \frac{t + 2t^2 \sin \frac{1}{t} - 0}{t - 0}
    = \lim_{t \to 0} \left( 1 + 2t \sin \frac{1}{t} \right)
    = 1
  \]
  (since $\sin \frac{1}{t}$ is bounded and $2t \to 0$ as $t \to 0$).
\end{enumerate}

\emph{Note.} $f'(t)$ is not continuous at $t = 0$.

\begin{enumerate}
\item[(2)]
  \emph{Show that $f'$ is bouneded in $(-1,1)$.}
  \[
    \abs{f'(t)}
    \leq 1 + 4\abs{t}\abs{\sin\frac{1}{t}} + 2\abs{\cos\frac{1}{t}}
    \leq 1 + 4 + 2
    = 7
  \]
  if $t \neq 0$.
  Hence $f'$ is bounded by $7$ in $(-1,1)$.

\item[(3)]
  \emph{Show that $f$ is not one-to-one in any neighborhood of $0$.}
  Take
  \[
    x_n = \frac{1}{2n\pi}
    \qquad
    \text{ and }
    \qquad
    y_n = \frac{1}{2n\pi+\pi}
  \]
  for $n = 1,2,3,\ldots$.
  So that
  \[
    f'(x_n) = -1 < 0
    \qquad
    \text{ and }
    \qquad
    f'(y_n) = 3 > 0.
  \]
  Since $f'(t)$ is continuous if $t \neq 0$,
  there exists $\xi_n \in (y_n, x_n)$ such that $f'(\xi_n) = 0$ (Theorem 4.23).
  Then Theorem 5.11 implies that $f$ has a local maximum at $\xi_n$,
  that is, $f$ is not one-to-one in the interval $[y_n, x_n]$
  (by applying Theorem 4.23 again).
  Since $x_n \to 0$ and $y_n \to 0$ as $n \to \infty$,
  $f$ is not one-to-one in any neighborhood of $0$.
\end{enumerate}
$\Box$ \\\\



%%%%%%%%%%%%%%%%%%%%%%%%%%%%%%%%%%%%%%%%%%%%%%%%%%%%%%%%%%%%%%%%%%%%%%%%%%%%%%%%



\textbf{Exercise 9.17.}
\emph{Let $\mathbf{f} = (f_1,f_2)$ be the mapping of $\mathbb{R}^2$ into $\mathbb{R}^2$
given by}
\[
  f_1(x,y) = e^x \cos y,
  \qquad
  f_2(x,y) = e^x \sin y.
\]
\begin{enumerate}
\item[(a)]
  \emph{What is the range of $\mathbf{f}$?}

\item[(b)]
  \emph{Show that the Jacobian of $\mathbf{f}$ is not zero at any point of $\mathbb{R}^2$.
  Thus every point of $\mathbb{R}^2$ has a neighborhood in which $\mathbf{f}$ is one-to-one.
  Nevertheless, $\mathbf{f}$ is not one-to-one on $\mathbb{R}^2$.}

\item[(c)]
  \emph{Put $\mathbf{a} = \left(0, \frac{\pi}{3}\right)$,
  $\mathbf{b} = \mathbf{f}(\mathbf{a})$, let $\mathbf{g}$ be the continuous inverse of $\mathbf{f}$,
  defined in a neighborhood of $\mathbf{b}$ such that $\mathbf{g}(\mathbf{b}) = \mathbf{a}$.
  Find an explicit formula for $\mathbf{g}$,
  compute $\mathbf{f}'(\mathbf{a})$ and $\mathbf{g}'(\mathbf{b})$,
  and verify the formula}
  \[
    \mathbf{g}'(\mathbf{y})
    = \{ \mathbf{f}'(\mathbf{g}(\mathbf{y})) \}^{-1}.
  \]

\item[(d)]
  \emph{What are the images under $\mathbf{f}$ of lines parallel to the coordinate axes?} \\
\end{enumerate}



\emph{Proof of (a).}
\begin{enumerate}
\item[(1)]
  The range of $\textbf{f}$ is $\mathbb{R}^2 - \{(0,0)\}$.

\item[(2)]
  If $(a,b) \neq (0,0)$,
  then
  $\mathbf{f}: \left(\log \sqrt{a^2+b^2}, \mathrm{atan2}(b,a)\right) \mapsto (a,b)$
  where
  \begin{equation*}
    \mathrm{atan2}(b,a) =
      \begin{cases}
        \arctan\left(\frac{b}{a}\right) & \text{ if $a > 0$}, \\
        \arctan\left(\frac{b}{a}\right) + \pi & \text{ if $a < 0$ and $b \geq 0$}, \\
        \arctan\left(\frac{b}{a}\right) - \pi & \text{ if $a < 0$ and $b < 0$}, \\
        \frac{\pi}{2} & \text{ if $a = 0$ and $b > 0$}, \\
        -\frac{\pi}{2} & \text{ if $a = 0$ and $b < 0$}.
      \end{cases}
  \end{equation*}
  (Or apply Theorem 8.7(d).)

\item[(3)]
  If $(a,b) = (0,0)$,
  then for any $(x,y) \in \mathbb{R}^2$
  we have
  $f_1(x,y)^2 + f_2(x,y)^2 = e^{2x} \neq 0$.
  So that there is no $(x,y)$ such that $\mathbf{f}: (x,y) \mapsto (0,0)$.
\end{enumerate}
$\Box$ \\



\emph{Proof of (b).}
\begin{enumerate}
\item[(1)]
  \[
    [\mathbf{f}'(x,y)]
    =
    \begin{bmatrix}
      e^x \cos y & -e^x \sin y \\
      e^x \sin y &  e^x \cos y
    \end{bmatrix}.
  \]
  So $\mathbf{f}'$ is continuous and
  \[
    J_{\mathbf{f}}(x,y)
    = \det \mathbf{f}'(x,y)
    = e^{2x}
    \neq 0.
  \]

\item[(2)]
  Since $J_{\mathbf{f}}(x,y) \neq 0$, $\mathbf{f}'(x,y)$ is invertible (Theorem 9.36).
  So the inverse function theorem (Theorem 9.24) implies that
  there exists an open neighborhood $B(x,y)$ of $(x,y)$ such that
  $\mathbf{f}$ is injective on $B(x,y)$.

\item[(3)]
  Note that
  \[
    \mathbf{f}(0,0) = \mathbf{f}(0,2\pi) = (1,0).
  \]
  So that $\mathbf{f}$ is not injective on the whole $\mathbb{R}^2$.
  (Injectivity of $\mathbf{f}$ is a local property.)
\end{enumerate}
$\Box$ \\



\emph{Proof of (c).}
\begin{enumerate}
\item[(1)]
  If $\mathbf{a} = \left(0, \frac{\pi}{3}\right)$, then
  $\mathbf{b} = \mathbf{f}(\mathbf{a}) = \left( \frac{1}{2}, \frac{\sqrt{3}}{2} \right)$.

\item[(2)]
  Similar to (2) in the proof of (a),
  define $\mathbf{g}: E \to \mathbb{R}^2$ by
  \[
    \mathbf{g}(x,y) = \left(\log \sqrt{x^2+y^2}, \arctan\left(\frac{y}{x}\right) \right).
  \]
  where $E$ is some open neighborhood of the point $\mathbf{b} \in \mathbb{R}^2$ described in (b).
  So $\mathbf{g}$ is a continuous inverse of $\mathbf{f}$.

\item[(3)]
  Since
  \[
    [\mathbf{f}'(x,y)]
    =
    \begin{bmatrix}
      e^x \cos y & -e^x \sin y \\
      e^x \sin y &  e^x \cos y
    \end{bmatrix},
  \]
  \[
    [\mathbf{f}'(\mathbf{a})]
    =
    \begin{bmatrix}
      \mathbf{f}'\left(0, \frac{\pi}{3}\right)
    \end{bmatrix}
    =
    \begin{bmatrix}
      e^0 \cos \frac{\pi}{3} & -e^0 \sin \frac{\pi}{3} \\
      e^0 \sin \frac{\pi}{3} &  e^0 \cos \frac{\pi}{3}
    \end{bmatrix}
    =
    \begin{bmatrix}
      \frac{1}{2} & -\frac{\sqrt{3}}{2} \\
      \frac{\sqrt{3}}{2} & \frac{1}{2}
    \end{bmatrix}.
  \]

\item[(4)]
  Since
  \[
    [\mathbf{g}'(x,y)]
    =
    \begin{bmatrix}
      \frac{x}{x^2+y^2} & \frac{y}{x^2+y^2} \\
      \frac{-y}{x^2+y^2} & \frac{x}{x^2+y^2}
    \end{bmatrix},
  \]
  \[
    [\mathbf{g}'(\mathbf{b})]
    =
    \begin{bmatrix}
      \mathbf{g}'\left( \frac{1}{2}, \frac{\sqrt{3}}{2} \right)
    \end{bmatrix} \\
    =
    \begin{bmatrix}
      \frac{1}{2} & \frac{\sqrt{3}}{2} \\
      -\frac{\sqrt{3}}{2} & \frac{1}{2}
    \end{bmatrix}.
  \]
  Here we can see $[\mathbf{f}'(\mathbf{a})][\mathbf{g}'(\mathbf{b})]
  = [\mathbf{g}'(\mathbf{b})][\mathbf{f}'(\mathbf{a})] = 1$.

\item[(5)]
  \begin{align*}
    [\mathbf{g}'(\mathbf{y})]
    &= [\mathbf{g}'(\mathbf{f}(\mathbf{x}))] \\
    &= [\mathbf{g}'(e^x \cos y, e^x \sin y)] \\
    &=
    \begin{bmatrix}
      \frac{e^x \cos y}{e^{2x}} & \frac{e^x \sin y}{e^{2x}} \\
      \frac{-e^x \sin y}{e^{2x}} & \frac{e^x \cos y}{e^{2x}}
    \end{bmatrix} \\
    &=
    \begin{bmatrix}
      e^{-x} \cos y & e^{-x} \sin y \\
      -e^{-x} \sin y & e^{-x} \cos y
    \end{bmatrix},
  \end{align*}
  and
  \[
    [\mathbf{f}'(\mathbf{g}(\mathbf{y}))]
    = [\mathbf{f}'(\mathbf{x})]
    =
    \begin{bmatrix}
      e^x \cos y & -e^x \sin y \\
      e^x \sin y &  e^x \cos y
    \end{bmatrix}.
  \]
  Note that
  \[
    \begin{bmatrix}
      e^{-x} \cos y & e^{-x} \sin y \\
      -e^{-x} \sin y & e^{-x} \cos y
    \end{bmatrix}
    \begin{bmatrix}
      e^x \cos y & -e^x \sin y \\
      e^x \sin y &  e^x \cos y
    \end{bmatrix}
    = 1.
  \]
  Therefore
  \[
    \mathbf{g}'(\mathbf{y})
    = \{ \mathbf{f}'(\mathbf{g}(\mathbf{y})) \}^{-1}
  \]
  on $\mathbf{g}(E)$.
\end{enumerate}
$\Box$ \\



\emph{Proof of (d).}
\begin{enumerate}
\item[(1)]
  The case $L_r = \{ (x,y) \in \mathbb{R}^2 : x = r \}$ parallel to $y$-axis
  where $r \in \mathbb{R}^1$ is constant.
  The image under $\mathbf{f}$ is
  \begin{align*}
    \mathbf{f}(L_r)
    &= \{ (e^r \cos y, e^r \sin y) \in \mathbb{R}^2 : y \in \mathbb{R}^1 \} \\
    &= \{ (s, t) \in \mathbb{R}^2 : s^2 + t^2 = (e^r)^2 \},
  \end{align*}
  a circle which is centered at the origin $(0,0) \in \mathbb{R}^2$ with radius $e^{r} > 0$.

\item[(2)]
  The case $L_{\theta} = \{ (x,y) \in \mathbb{R}^2 : y = \theta \}$ parallel to $x$-axis
  where $\theta \in \mathbb{R}^1$ is constant.
  The image under $\mathbf{f}$ is
  \begin{align*}
    \mathbf{f}(L_{\theta})
    &= \{ (e^x \cos \theta, e^x \sin \theta) \in \mathbb{R}^2 : x \in \mathbb{R}^1 \} \\
    &= \{ (y \cos \theta, y \sin \theta) \in \mathbb{R}^2 : y > 0 \},
  \end{align*}
  which is a ray from the origin $(0,0)$ (not included) to the infinity
  passing through a point $(\cos\theta, \sin\theta)$ in the unit circle.
\end{enumerate}
$\Box$ \\\\



%%%%%%%%%%%%%%%%%%%%%%%%%%%%%%%%%%%%%%%%%%%%%%%%%%%%%%%%%%%%%%%%%%%%%%%%%%%%%%%%



\textbf{Exercise 9.18.}
\emph{...} \\

\emph{Proof.}
\begin{enumerate}
\item[(1)]
\item[(2)]

\end{enumerate}
$\Box$ \\\\



%%%%%%%%%%%%%%%%%%%%%%%%%%%%%%%%%%%%%%%%%%%%%%%%%%%%%%%%%%%%%%%%%%%%%%%%%%%%%%%%



\textbf{Exercise 9.19.}
\emph{Show that the system of equations
\begin{align*}
  3x+y-z+u^2 &= 0 \\
  x-y+2z+u &= 0 \\
  2x+2y-3z+2u &= 0
\end{align*}
can solved
for $x,y,u$ in terms of $z$;
for $x,z,u$ in terms of $y$;
for $y,z,u$ in terms of $x$;
but not for $x,y,z$ in terms of $u.$} \\

\emph{Proof (Brute-force).}
\begin{enumerate}
\item[(1)]
  Denote
  \begin{align*}
    3x+y-z+u^2 &= 0 \tag{I} \\
    x-y+2z+u &= 0 \tag{II} \\
    2x+2y-3z+2u &= 0 \tag{III}
  \end{align*}
  So (I) - 3(II) implies that
  \[
    4y + u(u-3) = 7z \tag{IV},
  \]
  and (III) - 2(II) implies that
  \[
    4y = 7z \tag{V}.
  \]
  By (IV)(V), we have $u(u-3) = 0$.
  Hence $u = 0$ or $u = 3$ in any case.

\item[(2)]
  \emph{Show that (I)(II)(III) can be solve for $x,y,u$ in terms of $z$.}
  (V) implies that $y = \frac{7z}{4}$.
  Hence
  \[
    (x,y,u)
    =
    \left(-\frac{z}{4}, \frac{7z}{4}, 0 \right),
    \left(-\frac{z}{4}-3, \frac{7z}{4}, 3 \right).
  \]

\item[(3)]
  \emph{Show that (I)(II)(III) can be solve for $x,z,u$ in terms of $y$.}
  \[
    (x,z,u)
    =
    \left(-\frac{y}{7}, \frac{4y}{7}, 0 \right),
    \left(-\frac{y}{7}-3, \frac{4y}{7}, 3 \right).
  \]

\item[(4)]
  \emph{Show that (I)(II)(III) can be solve for $y,z,u$ in terms of $x$.}
  \[
    (y,z,u)
    =
    \left(-7x, -4x, 0 \right),
    \left(-7x-21, -4x-12, 3 \right).
  \]

\item[(5)]
  \emph{Show that (I)(II)(III) can not be solve for $x,y,z$ in terms of $u$.}
  Actually,
  \[
    (x,y,z)
    = \left(-t-u, 7t, 4t \right)
  \]
  for all $t \in \mathbb{R}^1$.
\end{enumerate}
$\Box$ \\



\emph{Proof (The implicit function theorem).}
\begin{enumerate}
\item[(1)]
  Define $\mathbf{f}$ be a $\mathscr{C}'$-mapping
  of $\mathbb{R}^{3+1}$ into $\mathbb{R}^3$ by
  \[
    \mathbf{f}(x,y,z,u)
    = (3x+y-z+u^2,x-y+2z+u,2x+2y-3z+2u).
  \]
  Note that $\mathbf{f}(0,0,0,0) = \mathbf{0}$
  and $\mathbf{f}(-3,0,0,3) = \mathbf{0}$.

\item[(2)]
  Since
  \[
    [\mathbf{f}'(x,y,z,u)]
    = \begin{bmatrix}
      3 &  1 & -1 & 2u \\
      1 & -1 &  2 &  1 \\
      2 &  2 & -3 &  2
    \end{bmatrix},
  \]
  $\mathbf{f}'$ is continuous,
  \[
    [\mathbf{f}'(0,0,0,0)]
    = \begin{bmatrix}
      3 &  1 & -1 &  0 \\
      1 & -1 &  2 &  1 \\
      2 &  2 & -3 &  2
    \end{bmatrix},
  \]
  and
  \[
    [\mathbf{f}'(-3,0,0,3)]
    = \begin{bmatrix}
      3 &  1 & -1 &  6 \\
      1 & -1 &  2 &  1 \\
      2 &  2 & -3 &  2
    \end{bmatrix}.
  \]

\item[(3)]
  The submatrix
  \[
    [\mathbf{f}'(0,0,0,0)]_x
    = \begin{bmatrix}
       1 & -1 &  0 \\
      -1 &  2 &  1 \\
       2 & -3 &  2
    \end{bmatrix}
  \]
  is invertiable since its determinant is $3 \neq 0$.
  By the implicit function theorem (Theorem 9.28),
  the system can be solved for $y,z,u$ in terms of $x$.
  Similar arguments to
  $[\mathbf{f}'(0,0,0,0)]_y$,
  $[\mathbf{f}'(0,0,0,0)]_z$,
  $[\mathbf{f}'(-3,0,0,3)]_x$,
  $[\mathbf{f}'(-3,0,0,3)]_y$, and
  $[\mathbf{f}'(-3,0,0,3)]_z$.

\item[(4)]
  Note that $[\mathbf{f}'(0,0,0,0)]_u$ and $[\mathbf{f}'(-3,0,0,3)]_u$
  are not invertible, we cannot apply the implicit function theorem (Theorem 9.28).
  We need to show by brute-force in this case.
\end{enumerate}
$\Box$ \\\\



%%%%%%%%%%%%%%%%%%%%%%%%%%%%%%%%%%%%%%%%%%%%%%%%%%%%%%%%%%%%%%%%%%%%%%%%%%%%%%%%



\textbf{Exercise 9.20.}
\emph{Take $n=m=1$ in the implicit function theorem,
and interpret the theorem (as well as its proof) graphically.} \\



\textbf{Implicit function theorem (for $n=m=1$).}
  Let $f(x,y)$ be a $\mathscr{C}'$-mapping of
  an open set $E \subseteq \mathbb{R}^2$ into $\mathbb{R}$,
  such that $f(a,b) = 0$ for some point $(a,b) \in E$.
  Assume that
  \[
    D_1 f(a,b) \neq 0.
  \]
  Then there exist open sets $U \subseteq E$ and $W \subseteq \mathbb{R}^1$,
  with $(a,b) \in U$ and $b \in W$, having the following property:
  \begin{quote}
    To every $y \in W$ corresponds a unique $x$ such that
    \[
      (x,y) \in U
      \qquad \text{ and } \qquad
      f(x,y) = 0.
    \]
    If this $x$ is defined to be $g(y)$,
    then $g$ is a $\mathscr{C}'$-mapping of $W$ into $\mathbb{R}^1$,
    $g(b) = a$,
    \[
      f(g(y),y) = 0
      \qquad
      (y \in W),
    \]
    and
    \[
      g'(b) = - \frac{D_2 f(a,b)}{D_1 f(a,b)}.
    \]
  \end{quote}


\emph{Proof.}
\begin{enumerate}
\item[(1)]
In the notations of Exercise 4.6,
define the graph of $f$ by the set
\[
  S = \{ (x,y) \in E : f(x,y) = 0 \}.
\]

\item[(2)]
Consider the graph $S$.
As $D_1 f(a,b) \neq 0$ and $f(x,y) \in \mathscr{C}'$,
there are an open neighborhood $U \subseteq E$ of $(a,b)$
and an open neighborhood $W$ of $b$
such that $x \mapsto f(x,y)$ is strictly monotonic whenever $y \in W$.
``Graphically'' by the monotony of $f(x,y)$, for any fixed $y$
there is a unique $x$ such that $f(x,y) = 0$.

\item[(3)]
``Graphically''
the tangent line passing through $(a,b)$ is
\[
  D_1 f(a,b)(x - a) + D_2 f(a,b)(y - b) = 0.
\]
Thus $g'(b) = -\frac{D_2 f(a,b)}{D_1 f(a,b)}$ if $D_1 f(a,b) \neq 0$.
\end{enumerate}
$\Box$ \\\\



%%%%%%%%%%%%%%%%%%%%%%%%%%%%%%%%%%%%%%%%%%%%%%%%%%%%%%%%%%%%%%%%%%%%%%%%%%%%%%%%



\textbf{Exercise 9.21.}
\emph{Define $f$ in $\mathbb{R}^2$ by}
\[
  f(x,y) = 2x^3-3x^2+2y^3+3y^2.
\]
\begin{enumerate}
\item[(a)]
  \emph{Find the four points in $\mathbb{R}^2$ at which the gradient of $f$ is zero.
  Show that $f$ has exactly one local maximum and one local minimum in $\mathbb{R}^1$.}

\item[(b)]
  \emph{Let $S$ ...} \\
\end{enumerate}



\emph{Proof of (a).}
\begin{enumerate}
\item[(1)]
\item[(2)]
\end{enumerate}
$\Box$ \\



\emph{Proof of (b).}
\begin{enumerate}
\item[(1)]
\item[(2)]
\end{enumerate}
$\Box$ \\\\



%%%%%%%%%%%%%%%%%%%%%%%%%%%%%%%%%%%%%%%%%%%%%%%%%%%%%%%%%%%%%%%%%%%%%%%%%%%%%%%%



\textbf{Exercise 9.22.}
\emph{Given a similar discussion for}
\[
  f(x,y) = 2x^3+6xy^2-3x^2+3y^2.
\]



\emph{Outline.}
\begin{enumerate}
\item[(a)]
  \emph{Find the two points in $\mathbb{R}^2$ at which the gradient of $f$ is zero.
  Show that $f$ has exactly one saddle point and one local minimum in $\mathbb{R}^1$.}

\item[(b)]
  \emph{Let $S$ ...} \\
\end{enumerate}



\emph{Proof of (a).}
\begin{enumerate}
\item[(1)]
\item[(2)]
\end{enumerate}
$\Box$ \\



\emph{Proof of (b).}
\begin{enumerate}
\item[(1)]
\item[(2)]
\end{enumerate}
$\Box$ \\\\



%%%%%%%%%%%%%%%%%%%%%%%%%%%%%%%%%%%%%%%%%%%%%%%%%%%%%%%%%%%%%%%%%%%%%%%%%%%%%%%%



\textbf{Exercise 9.23.}
\emph{...} \\

\emph{Proof.}
\begin{enumerate}
\item[(1)]
\item[(2)]

\end{enumerate}
$\Box$ \\\\



%%%%%%%%%%%%%%%%%%%%%%%%%%%%%%%%%%%%%%%%%%%%%%%%%%%%%%%%%%%%%%%%%%%%%%%%%%%%%%%%



\textbf{Exercise 9.24.}
\emph{For $(x,y) \neq (0,0)$, define $\mathbf{f} = (f_1,f_2)$ by
\[
  f_1(x,y) = \frac{x^2-y^2}{x^2+y^2},
  \qquad
  f_2(x,y) = \frac{xy}{x^2+y^2}.
\]
Compute the rank of $\mathbf{f}'(x,y)$, and find the range of $\mathbf{f}$.} \\

\emph{Proof.}
\begin{enumerate}
\item[(1)]
  \[
    [\mathbf{f}'(x,y)]
    =
    \begin{bmatrix}
      \frac{4xy^2}{(x^2+y^2)^2} & \frac{4x^2y}{(x^2+y^2)^2} \\
      \frac{-y(x^2-y^2)}{(x^2+y^2)^2} & \frac{x(x^2-y^2)}{(x^2+y^2)^2}
    \end{bmatrix}.
  \]

\item[(2)]
  \emph{Show that $\rank([\mathbf{f}'(x,y)]) \neq 2$.}
  It is equivalent to show that $\det[\mathbf{f}'(x,y)] = 0$.
  Actually,
  \[
    \det[\mathbf{f}'(x,y)]
    = \frac{4xy^2}{(x^2+y^2)^2} \cdot \frac{x(x^2-y^2)}{(x^2+y^2)^2}
      - \frac{4x^2y}{(x^2+y^2)^2} \cdot \frac{-y(x^2-y^2)}{(x^2+y^2)^2}
    = 0.
  \]

\item[(3)]
  \emph{Show that $\rank([\mathbf{f}'(x,y)]) \neq 0$.}
  \begin{align*}
    [\mathbf{f}'(x,y)]
    \begin{bmatrix}
      1 \\
      0
    \end{bmatrix}
    &=
    \begin{bmatrix}
      \frac{4xy^2}{(x^2+y^2)^2} & \frac{4x^2y}{(x^2+y^2)^2} \\
      \frac{-y(x^2-y^2)}{(x^2+y^2)^2} & \frac{x(x^2-y^2)}{(x^2+y^2)^2}
    \end{bmatrix}
    \begin{bmatrix}
      1 \\
      0
    \end{bmatrix} \\
    &=
    \begin{bmatrix}
      \frac{4xy^2}{(x^2+y^2)^2} \\
      \frac{-y(x^2-y^2)}{(x^2+y^2)^2}
    \end{bmatrix} \\
    &\neq
    \begin{bmatrix}
      0 \\
      0
    \end{bmatrix}
  \end{align*}
  for all $(x,y) \in \mathbb{R}^2 - \{(0,0)\}$.

\item[(4)]
  Since the rank of $\mathbf{f}'$ is the dimension of the subspace
  $\mathscr{R}(\mathbf{f}')$ in $\mathbb{R}^2$,
  $\rank([\mathbf{f}'(x,y)]) = 0, 1, 2$.
  By (2)(3), $\rank([\mathbf{f}'(x,y)]) = 1$.

\item[(5)]
  \emph{Show that the range of $\mathbf{f}$ is an ellipse}
  \[
    E = \{ (s,t) \in \mathbb{R}^2 : s^2 + 4t^2 = 1 \}.
  \]
  \begin{enumerate}
  \item[(a)]
    Clearly, $(f_1(x,y), f_2(x,y)) \in E$.

  \item[(b)]
    Conversely, for any $(s,t) \in E$ write
    \[
      s = \cos\theta
      \qquad \text{ and } \qquad
      t = \frac{1}{2} \sin\theta
    \]
    for some unique $\theta \in [0,2\pi)$ (Theorem 8.7(d)).
    By the tangent half-angle formula,
    \[
      s = \cos\theta = \frac{1-\tan^2\frac{\theta}{2}}{1+\tan^2\frac{\theta}{2}}
      \qquad \text{ and } \qquad
      t = \frac{1}{2} \sin\theta = \frac{\tan\frac{\theta}{2}}{1+\tan^2\frac{\theta}{2}}.
    \]
    Thus, there exists a point
    $\left(1, \tan\frac{\theta}{2}\right) \in \mathbb{R}^2$ such that
    \[
      f: \left(1, \tan\frac{\theta}{2}\right)
      \mapsto (s, t) \in E.
    \]

  \item[(c)]
    Or we can do a linear projection from a given point $P = (1,0)$,
    say for any $\lambda \in \mathbb{R}^1$ we define a line through $P$
    with slope $-\lambda$ meeting $E$ in a further point
    \[
      Q_{\lambda}
      = \left(\frac{\lambda^2-1}{\lambda^2+1},
        \frac{\lambda}{\lambda^2+1}\right).
    \]
    Might define $Q_{\infty} = P$.
    Graphically and informally,
    \[
      \{ Q_{\lambda} : \lambda \in \mathbb{P}^1(\mathbb{R}) = \mathbb{R} \cup \{\infty\} \} = E.
    \]
    Therefore, $f(1,0) = P$ and $f(\lambda, 1) \in E-\{P\}$.

  \end{enumerate}
  By (a)(b), the range of $\mathbf{f}$ is exactly the same as an ellipse $E$.

\end{enumerate}
$\Box$ \\\\



%%%%%%%%%%%%%%%%%%%%%%%%%%%%%%%%%%%%%%%%%%%%%%%%%%%%%%%%%%%%%%%%%%%%%%%%%%%%%%%%



\textbf{Exercise 9.25.}
\emph{...} \\

\emph{Proof.}
\begin{enumerate}
\item[(1)]
\item[(2)]

\end{enumerate}
$\Box$ \\\\



%%%%%%%%%%%%%%%%%%%%%%%%%%%%%%%%%%%%%%%%%%%%%%%%%%%%%%%%%%%%%%%%%%%%%%%%%%%%%%%%



\textbf{Exercise 9.26.}
\emph{...} \\

\emph{Proof.}
\begin{enumerate}
\item[(1)]
\item[(2)]

\end{enumerate}
$\Box$ \\\\



%%%%%%%%%%%%%%%%%%%%%%%%%%%%%%%%%%%%%%%%%%%%%%%%%%%%%%%%%%%%%%%%%%%%%%%%%%%%%%%%



\textbf{Exercise 9.27.}
\emph{Put $f(0,0) = 0$, and
\[
  f(x,y) = \frac{xy(x^2-y^2)}{x^2+y^2}
\]
if $(x,y) \neq (0,0)$. Prove that}
\begin{enumerate}
\item[(a)]
  \emph{$f$, $D_1 f$, $D_2 f$ are continuous in $\mathbb{R}^2$.}

\item[(b)]
  \emph{$D_{12}f$ and $D_{21}f$ exist at every point of $\mathbb{R}^2$,
  and are continuous except at $(0,0)$.}

\item[(c)]
  \emph{$(D_{12}f)(0,0) = 1$, and $(D_{21}f)(0,0) = -1$.} \\
\end{enumerate}



\emph{Proof of (a).}
\begin{enumerate}
\item[(1)]
  \emph{Show that $f$ is continuous in $\mathbb{R}^2$.}
  \begin{enumerate}
  \item[(a)]
    Clearly, $f(x,y)$ is continuous if $(x,y) \neq (0,0)$.
    So it suffices to show that
    \[
      \lim_{(x,y) \to (0,0)} f(x,y) = f(0,0) = 0.
    \]

  \item[(b)]
    Write $x = r\cos\theta$ and $y = r\sin\theta$ in the polar coordinates.
    (Here $r > 0$.)
    Hence
    \begin{align*}
      \lim_{(x,y) \to (0,0)} f(x,y)
      &= \lim_{(x,y) \to (0,0)} \frac{xy(x^2-y^2)}{x^2+y^2} \\
      &= \lim_{r \to 0} r^2 \cos\theta\sin\theta(\cos^2\theta-\sin^2\theta) \\
      &= 0
    \end{align*}
    since $\cos\theta\sin\theta(\cos^2\theta-\sin^2\theta)$ is bounded by $2$.
  \end{enumerate}

\item[(2)]
  \emph{Show that $D_1 f$ is continuous in $\mathbb{R}^2$.}
  \begin{enumerate}
  \item[(a)]
    $(x,y) \neq (0,0)$ implies that
    \[
      (D_1 f)(x,y) = \frac{y(x^4+4x^2y^2-y^4)}{(x^2+y^2)^2}.
    \]
    Besides,
    \begin{align*}
      (D_1 f)(0,0)
      &= \lim_{x \to 0} \frac{f(x,0) - f(0,0)}{x - 0} \\
      &= \lim_{x \to 0} \frac{0}{x} \\
      &= 0.
    \end{align*}

  \item[(b)]
    Clearly, $(D_1 f)(x,y)$ is continuous if $(x,y) \neq (0,0)$.
    So it suffices to show that
    \[
      \lim_{(x,y) \to (0,0)} (D_1 f)(x,y) = (D_1 f)(0,0) = 0.
    \]

  \item[(c)]
    Similar to (1)(b).
    Write $x = r\cos\theta$ and $y = r\sin\theta$ in the polar coordinates.
    (Here $r > 0$.)
    Hence
    \begin{align*}
      \lim_{(x,y) \to (0,0)} (D_1 f)(x,y)
      &= \lim_{(x,y) \to (0,0)} \frac{y(x^4+4x^2y^2-y^4)}{(x^2+y^2)^2} \\
      &= \lim_{r \to 0} r \sin\theta
        (\cos^4\theta+4\cos^2\theta\sin^2\theta-\sin^4\theta) \\
      &= 0
    \end{align*}
    since $\sin\theta(\cos^4\theta+4\cos^2\theta\sin^2\theta-\sin^4\theta)$ is bounded by $6$.
  \end{enumerate}

\item[(3)]
  Similar to (2).
  \emph{Show that $D_2 f$ is continuous in $\mathbb{R}^2$.}
  \begin{enumerate}
  \item[(a)]
    $(x,y) \neq (0,0)$ implies that
    \[
      (D_2 f)(x,y) = \frac{x(x^4-4x^2y^2-y^4)}{(x^2+y^2)^2}.
    \]
    Besides,
    \begin{align*}
      (D_2 f)(0,0)
      &= \lim_{y \to 0} \frac{f(0,y) - f(0,0)}{y - 0} \\
      &= \lim_{y \to 0} \frac{0}{y} \\
      &= 0.
    \end{align*}

  \item[(b)]
    Clearly, $(D_2 f)(x,y)$ is continuous if $(x,y) \neq (0,0)$.
    So it suffices to show that
    \[
      \lim_{(x,y) \to (0,0)} (D_2 f)(x,y) = (D_2 f)(0,0) = 0.
    \]

  \item[(c)]
    Similar to (1)(b).
    Write $x = r\cos\theta$ and $y = r\sin\theta$ in the polar coordinates.
    (Here $r > 0$.)
    Hence
    \begin{align*}
      \lim_{(x,y) \to (0,0)} (D_2 f)(x,y)
      &= \lim_{(x,y) \to (0,0)} \frac{x(x^4-4x^2y^2-y^4)}{(x^2+y^2)^2} \\
      &= \lim_{r \to 0} r \cos\theta
        (\cos^4\theta-4\cos^2\theta\sin^2\theta-\sin^4\theta) \\
      &= 0
    \end{align*}
    since $r \cos\theta(\cos^4\theta-4\cos^2\theta\sin^2\theta-\sin^4\theta)$ is bounded by $6$.
  \end{enumerate}
\end{enumerate}
$\Box$ \\



\emph{Proof of (b).}
\begin{enumerate}
\item[(1)]
  \emph{Show that $D_{12}f$ exists at every point of $\mathbb{R}^2$.}

\item[(2)]
  \emph{Show that $D_{12}f$ is continuous except at $(0,0)$.}

\item[(3)]
  \emph{Show that $D_{21}f$ exists at every point of $\mathbb{R}^2$.}

\item[(4)]
  \emph{Show that $D_{21}f$ is continuous except at $(0,0)$.}

\end{enumerate}
$\Box$ \\



\emph{Proof of (c).}
\begin{enumerate}
\item[(1)]
  \emph{Show that $(D_{12}f)(0,0) = 1$.}

\item[(2)]
  \emph{Show that $(D_{21}f)(0,0) = -1$.}

\end{enumerate}
$\Box$ \\\\



%%%%%%%%%%%%%%%%%%%%%%%%%%%%%%%%%%%%%%%%%%%%%%%%%%%%%%%%%%%%%%%%%%%%%%%%%%%%%%%%



\textbf{Exercise 9.28.}
\emph{...} \\

\emph{Proof.}
\begin{enumerate}
\item[(1)]
\item[(2)]

\end{enumerate}
$\Box$ \\\\



%%%%%%%%%%%%%%%%%%%%%%%%%%%%%%%%%%%%%%%%%%%%%%%%%%%%%%%%%%%%%%%%%%%%%%%%%%%%%%%%



\textbf{Exercise 9.29.}
\emph{...} \\

\emph{Proof.}
\begin{enumerate}
\item[(1)]
\item[(2)]

\end{enumerate}
$\Box$ \\\\



%%%%%%%%%%%%%%%%%%%%%%%%%%%%%%%%%%%%%%%%%%%%%%%%%%%%%%%%%%%%%%%%%%%%%%%%%%%%%%%%



\textbf{Exercise 9.30.}
\emph{...} \\

\emph{Proof.}
\begin{enumerate}
\item[(1)]
\item[(2)]

\end{enumerate}
$\Box$ \\\\



%%%%%%%%%%%%%%%%%%%%%%%%%%%%%%%%%%%%%%%%%%%%%%%%%%%%%%%%%%%%%%%%%%%%%%%%%%%%%%%%



\textbf{Exercise 9.31.}
\emph{...} \\

\emph{Proof.}
\begin{enumerate}
\item[(1)]
\item[(2)]

\end{enumerate}
$\Box$ \\\\



%%%%%%%%%%%%%%%%%%%%%%%%%%%%%%%%%%%%%%%%%%%%%%%%%%%%%%%%%%%%%%%%%%%%%%%%%%%%%%%%
%%%%%%%%%%%%%%%%%%%%%%%%%%%%%%%%%%%%%%%%%%%%%%%%%%%%%%%%%%%%%%%%%%%%%%%%%%%%%%%%



\end{document}