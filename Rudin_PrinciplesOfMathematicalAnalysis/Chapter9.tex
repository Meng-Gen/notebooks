\documentclass{article}
\usepackage{amsfonts}
\usepackage{amsmath}
\usepackage{amssymb}
\usepackage{bm}
\usepackage{hyperref}
\usepackage[none]{hyphenat}
\usepackage{mathrsfs}
\usepackage{physics}
\parindent=0pt

\def\upint{\mathchoice%
    {\mkern13mu\overline{\vphantom{\intop}\mkern7mu}\mkern-20mu}%
    {\mkern7mu\overline{\vphantom{\intop}\mkern7mu}\mkern-14mu}%
    {\mkern7mu\overline{\vphantom{\intop}\mkern7mu}\mkern-14mu}%
    {\mkern7mu\overline{\vphantom{\intop}\mkern7mu}\mkern-14mu}%
  \int}
\def\lowint{\mkern3mu\underline{\vphantom{\intop}\mkern7mu}\mkern-10mu\int}

\begin{document}



\textbf{\Large Chapter 9: Functions of Several Variables} \\\\



\emph{Author: Meng-Gen Tsai} \\
\emph{Email: plover@gmail.com} \\\\



%%%%%%%%%%%%%%%%%%%%%%%%%%%%%%%%%%%%%%%%%%%%%%%%%%%%%%%%%%%%%%%%%%%%%%%%%%%%%%%%
%%%%%%%%%%%%%%%%%%%%%%%%%%%%%%%%%%%%%%%%%%%%%%%%%%%%%%%%%%%%%%%%%%%%%%%%%%%%%%%%



\textbf{Exercise 9.1.}
\emph{If $S$ is a nonempty subset of a vector space $X$,
prove (as asserted in Section 9.1) that the span of $S$ is a vector space.} \\

Denote the span of $S$ by $\mathrm{span}(S)$. \\

\emph{Proof.}
\begin{enumerate}
\item[(1)]
  Since $S \neq \varnothing$, there is $\mathbf{z} \in S$.
  So $1\mathbf{z} = \mathbf{z} \in \mathrm{span}(S) \neq \varnothing$.
  (In fact, $\mathrm{span}(S) \supseteq S$.)

\item[(2)]
  If $\mathbf{x}, \mathbf{y} \in \mathrm{span}(S)$,
  then there exist elements
  $\mathbf{x}_1, \ldots, \mathbf{x}_m$, $\mathbf{y}_1, \ldots, \mathbf{y}_n \in S$
  and scalars $a_1, \ldots, a_m$, $b_1, \ldots, b_n$ such that
  \begin{align*}
    \mathbf{x} &= a_1 \mathbf{x}_1 + \cdots + a_m \mathbf{x}_m, \\
    \mathbf{y} &= b_1 \mathbf{y}_1 + \cdots + b_n \mathbf{y}_n.
  \end{align*}
  Then
  \[
    \mathbf{x}+\mathbf{y}
    = a_1 \mathbf{x}_1 + \cdots + a_m \mathbf{x}_m
      + b_1 \mathbf{y}_1 + \cdots + b_n \mathbf{y}_n
  \]
  is a linear combination of the elements of $S$.
  For any scalar $c$,
  \[
    c\mathbf{x} = (ca_1) \mathbf{x}_1 + \cdots + (ca_m) \mathbf{x}_m
  \]
  is again linear combination of the elements of $S$.

\item[(3)]
  By (1)(2), $\mathrm{span}(S)$ is a vector space.
\end{enumerate}
$\Box$ \\

\emph{Note.}
Any subspace of $X$ that contains $S$ must also contain $\mathrm{span}(S)$. \\\\



%%%%%%%%%%%%%%%%%%%%%%%%%%%%%%%%%%%%%%%%%%%%%%%%%%%%%%%%%%%%%%%%%%%%%%%%%%%%%%%%



\textbf{Exercise 9.2.}
\emph{Prove (as asserted in Section 9.6) that $BA$ is linear
if $A$ and $B$ are linear transformations.
Prove also that $A^{-1}$ is linear and invertible if $A$ is invertible.} \\

\emph{Proof.}
Use the notation in Definitions 9.6.
\begin{enumerate}
\item[(1)]
  \emph{Show that $BA$ is linear if $A$ and $B$ are linear transformations.}
  Let $X, Y, Z$ be vector spaces, $A \in L(X,Y)$ and $B \in L(Y,Z)$.
  \begin{enumerate}
  \item[(a)]
    Given any $\mathbf{x}_1, \mathbf{x}_2 \in X$.
    \begin{align*}
      (BA)(\mathbf{x}_1+\mathbf{x}_2)
      &= B(A(\mathbf{x}_1+\mathbf{x}_2)) \\
      &= B(A\mathbf{x}_1+A\mathbf{x}_2)
        & (\text{$A$ is a linear transformation}) \\
      &= B(A\mathbf{x}_1) + B(A\mathbf{x}_2)
        & (\text{$B$ is a linear transformation}) \\
      &= (BA)\mathbf{x}_1 + (BA)\mathbf{x}_2.
    \end{align*}

  \item[(b)]
    For any $\mathbf{x} \in X$ and scalar $c$,
    \begin{align*}
    (BA)(c\mathbf{x})
    &= B(A(c\mathbf{x})) \\
    &= B(cA\mathbf{x})
      & (\text{$A$ is a linear transformation}) \\
    &= cB(A\mathbf{x})
      & (\text{$B$ is a linear transformation}) \\
    &= c(BA)\mathbf{x}.
    \end{align*}
  \end{enumerate}
  By (a)(b), $BA \in L(X,Z)$.

\item[(2)]
  \emph{Show that $A^{-1}$ is linear if $A$ is invertible.}
  \begin{enumerate}
  \item[(a)]
    Given any $\mathbf{y}_1, \mathbf{y}_2 \in X$.
    Since $A$ is surjective,
    there exist $\mathbf{x}_1, \mathbf{x}_2 \in X$ such that
    \begin{align*}
      \mathbf{y}_1 &= A\mathbf{x}_1 \\
      \mathbf{y}_2 &= A\mathbf{x}_2.
    \end{align*}
    So
    \begin{align*}
      A^{-1}\mathbf{y}_1 &= A^{-1}(A\mathbf{x}_1) = \mathbf{x}_1 \\
      A^{-1}\mathbf{y}_2 &= A^{-1}(A\mathbf{x}_2) = \mathbf{x}_2
    \end{align*}
    (by Definitions 9.4).
    Hence
    \begin{align*}
      A^{-1}(\mathbf{y}_1+\mathbf{y}_2)
      &= A^{-1}(A\mathbf{x}_1+A\mathbf{x}_2) \\
      &= A^{-1}(A(\mathbf{x}_1+\mathbf{x}_2))
        & (\text{$A$ is a linear transformation}) \\
      &= \mathbf{x}_1+\mathbf{x}_2
        & (\text{Definitions 9.4}) \\
      &= A^{-1}\mathbf{y}_1+A^{-1}\mathbf{y}_2.
    \end{align*}

  \item[(b)]
    For any $\mathbf{y} \in X$ and scalar $c$,
    there is a corresponding $\mathbf{x} \in X$ such that $\mathbf{y} = A\mathbf{x}$
    since $A$ is surjective. So $A^{-1}\mathbf{y} = \mathbf{x}$ by Definition 9.4.
    Hence
    \begin{align*}
      A^{-1}(c\mathbf{y})
      &= A^{-1}(cA\mathbf{x}) \\
      &= A^{-1}(A(c\mathbf{x}))
        & (\text{$A$ is a linear transformation}) \\
      &= c\mathbf{x}
        & (\text{Definitions 9.4}) \\
      &= cA^{-1}\mathbf{y}.
    \end{align*}
  \end{enumerate}
  By (a)(b), $A^{-1} \in L(X)$.

\item[(3)]
  \emph{Show that $A^{-1}$ is invertible if $A$ is invertible.}
  It suffices to show that $A^{-1}$ is injective and surjective.
  \begin{enumerate}
  \item[(a)]
    \emph{Show that $A^{-1}$ is injective.}
    Given any $\mathbf{y}_1, \mathbf{y}_2 \in X$.
    Since $A$ is surjective,
    there exist $\mathbf{x}_1, \mathbf{x}_2 \in X$ such that
    \begin{align*}
      \mathbf{y}_1 &= A\mathbf{x}_1 \\
      \mathbf{y}_2 &= A\mathbf{x}_2.
    \end{align*}
    Suppose $A^{-1}\mathbf{y}_1 = A^{-1}\mathbf{y}_2$.
    So $A^{-1}(A\mathbf{x}_1) = A^{-1}(A\mathbf{x}_2)$,
    or $\mathbf{x}_1 = \mathbf{x}_2$,
    or $\mathbf{y}_1 = A\mathbf{x}_1 = A\mathbf{x}_2 = \mathbf{y}_2$.

  \item[(b)]
    \emph{Show that $A^{-1}$ is surjective.}
    For any $\mathbf{x} \in X$, there exists $A\mathbf{x} \in X$ such that
    $A^{-1}(A\mathbf{x}) = \mathbf{x}$ by Definitions 9.4.
  \end{enumerate}
\end{enumerate}
$\Box$ \\\\



%%%%%%%%%%%%%%%%%%%%%%%%%%%%%%%%%%%%%%%%%%%%%%%%%%%%%%%%%%%%%%%%%%%%%%%%%%%%%%%%



\textbf{Exercise 9.3.}
\emph{Assume $A \in L(X,Y)$ and $A\mathbf{x} = \mathbf{0}$ only when $\mathbf{x} = \mathbf{0}$.
Prove that $A$ is then $1$-$1$.} \\

\emph{Proof.}
Suppose $A\mathbf{x} = A\mathbf{y}$.
Since $A$ is a linear transformation,
$A(\mathbf{x}-\mathbf{y}) = A\mathbf{x} - A\mathbf{y} = \mathbf{0}$.
By assumption, $\mathbf{x}-\mathbf{y} = \mathbf{0}$
or $\mathbf{x} = \mathbf{y}$.
$\Box$ \\\\



%%%%%%%%%%%%%%%%%%%%%%%%%%%%%%%%%%%%%%%%%%%%%%%%%%%%%%%%%%%%%%%%%%%%%%%%%%%%%%%%



\textbf{Exercise 9.4.}
\emph{Prove (as asserted in Section 9.30) that null spaces and ranges of
linear transformations are vector spaces.} \\

\emph{Proof.}
Use the notation in Definitions 9.30.
Suppose $X$, $Y$ are vector spaces, and $A \in L(X,Y)$, as in Definition 9.6.
\begin{enumerate}
\item[(1)]
  \emph{Show that $\mathscr{N}(A)$ is a vector space in $X$.}
  \begin{enumerate}
  \item[(a)]
    Note that $\mathbf{0} \in X$.
    Since $A\mathbf{0} = \mathbf{0}$, $\mathbf{0} \in \mathscr{N}(A) \neq \varnothing$.

  \item[(b)]
    Suppose $\mathbf{x}_1, \mathbf{x}_2 \in \mathscr{N}(A)$.
    Then
    \begin{align*}
      A(\mathbf{x}_1+\mathbf{x}_2)
      &= A\mathbf{x}_1+A\mathbf{x}_2
        & (\text{$A$ is a linear transformation}) \\
      &= \mathbf{0}+\mathbf{0}
        & (\mathbf{x}_1, \mathbf{x}_2 \in \mathscr{N}(A)) \\
      &= \mathbf{0}.
    \end{align*}
    So $\mathbf{x}_1+\mathbf{x}_2 \in \mathscr{N}(A)$.

  \item[(c)]
    Suppose $\mathbf{x} \in \mathscr{N}(A)$ and $c$ is a scalar.
    Then
    \begin{align*}
      A(c\mathbf{x})
      &= cA\mathbf{x}
        & (\text{$A$ is a linear transformation}) \\
      &= c\mathbf{0}
        & (\mathbf{x} \in \mathscr{N}(A)) \\
      &= \mathbf{0}.
    \end{align*}
    So $c\mathbf{x} \in \mathscr{N}(A)$.
  \end{enumerate}
  By (a)(b)(c), $\mathscr{N}(A)$ is a vector space.

\item[(2)]
  \emph{Show that $\mathscr{R}(A)$ is a vector space in $Y$.}
  \begin{enumerate}
  \item[(a)]
    Note that $\mathbf{0} \in X$.
    So $A\mathbf{0} = \mathbf{0} \in \mathscr{R}(A) \neq \varnothing$.

  \item[(b)]
    Suppose $\mathbf{y}_1, \mathbf{y}_2 \in \mathscr{R}(A)$.
    Then there exist $\mathbf{x}_1, \mathbf{x}_2 \in X$
    such that $A\mathbf{x}_1 = \mathbf{y}_1$
    and $A\mathbf{x}_2 = \mathbf{y}_2$.
    Hence
    \begin{align*}
      \mathbf{y}_1+\mathbf{y}_2
      &= A\mathbf{x}_1+A\mathbf{x}_2 \\
      &= A(\mathbf{x}_1+\mathbf{x}_2)
        & (\text{$A$ is a linear transformation}).
    \end{align*}
    So $\mathbf{y}_1+\mathbf{y}_2 \in \mathscr{R}(A)$.

  \item[(c)]
    Suppose $\mathbf{y} \in \mathscr{R}(A)$ and $c$ is a scalar.
    Then there exists $\mathbf{x} \in X$ such that $A\mathbf{x} = \mathbf{y}$.
    Hence
    \begin{align*}
      c\mathbf{y}
      &= cA\mathbf{x} \\
      &= A(c\mathbf{x})
        & (\text{$A$ is a linear transformation}).
    \end{align*}
    So $c\mathbf{y} \in \mathscr{R}(A)$.
  \end{enumerate}
  By (a)(b)(c), $\mathscr{R}(A)$ is a vector space.
\end{enumerate}
$\Box$ \\\\



%%%%%%%%%%%%%%%%%%%%%%%%%%%%%%%%%%%%%%%%%%%%%%%%%%%%%%%%%%%%%%%%%%%%%%%%%%%%%%%%



\textbf{Exercise 9.5.}
\emph{Prove that to every $A \in L(\mathbb{R}^n, \mathbb{R}^1)$
corresponds a unique $\mathbf{y} \in \mathbb{R}^n$ such that
$A\mathbf{x} = \mathbf{x} \cdot \mathbf{y}$.
Prove also that $\norm{A} = |\mathbf{y}|$.
(Hint: Under certain conditions, equality holds in the Schwarz inequality.)} \\

\emph{Proof.}
\begin{enumerate}
\item[(1)]
  Recall that $\{ \mathbf{e}_1, \ldots, \mathbf{e}_n \}$
  is the standard basis of $\mathbb{R}^n$ (Definitions 9.1).
  Given any $\mathbf{x} \in \mathbb{R}^n$,
  write $\mathbf{x} = (x_1, \ldots, x_n)$ as $\mathbf{x} = \sum x_j \mathbf{e}_j$.

\item[(2)]
  \emph{Show that $\mathbf{y}$ exists.}
  Since $A$ is a linear transformation,
  \begin{align*}
    A\mathbf{x}
    &= A\left(\sum x_j \mathbf{e}_j\right) \\
    &= \sum x_j A\mathbf{e}_j \\
    &= (x_1, \ldots, x_n) \cdot (A\mathbf{e}_1, \ldots, A\mathbf{e}_n) \\
    &= \mathbf{x} \cdot \sum (A\mathbf{e}_j) \mathbf{e}_j.
  \end{align*}
  Define $\mathbf{y} = \sum (A\mathbf{e}_j) \mathbf{e}_j \in \mathbb{R}^n$
  so that $A\mathbf{x} = \mathbf{x} \cdot \mathbf{y}$.

\item[(3)]
  \emph{Show that $\mathbf{y}$ is unique.}
  Suppose there exists some $\mathbf{z} \in \mathbb{R}^n$
  such that $A\mathbf{x} = \mathbf{x} \cdot \mathbf{z}$.
  So
  \begin{align*}
    0
    &= A\mathbf{x} -  A\mathbf{x} \\
    &= \mathbf{x} \cdot \mathbf{y}-\mathbf{x} \cdot \mathbf{z} \\
    &= \mathbf{x} \cdot (\mathbf{y}-\mathbf{z})
  \end{align*}
  for any $\mathbf{x} \in \mathbb{R}^n$.
  In particular, take $\mathbf{x} = \mathbf{y}-\mathbf{z} \in \mathbb{R}^n$
  to get
  \[
    0
    = (\mathbf{y}-\mathbf{z}) \cdot (\mathbf{y}-\mathbf{z})
    = |\mathbf{y}-\mathbf{z}|^2
  \]
  or $\mathbf{y}-\mathbf{z} = \mathbf{0}$ or $\mathbf{y}=\mathbf{z}$.

\item[(4)]
  \emph{Show that $\norm{A} = |\mathbf{y}|$.}
  By the Schwarz inequality (Theorem 1.37(d)),
  \[
    |A\mathbf{x}| = |\mathbf{x} \cdot \mathbf{y}| \leq |\mathbf{x}||\mathbf{y}|
    \leq |\mathbf{y}|
  \]
  as $|\mathbf{x}| \leq 1$.
  Take the sup over all $|\mathbf{x}| \leq 1$ to get
  \[
    \norm{A} \leq |\mathbf{y}|.
  \]
  If $\mathbf{y} = \mathbf{0}$, then $\norm{A} = |\mathbf{y}| = 0$.
  If $\mathbf{y} \neq \mathbf{0}$,
  then the equality holds when
  $\mathbf{x} = \frac{\mathbf{y}}{|\mathbf{y}|} \in \mathbb{R}^n$.
  (Here $|\mathbf{x}| = 1$.)
\end{enumerate}
$\Box$ \\\\



%%%%%%%%%%%%%%%%%%%%%%%%%%%%%%%%%%%%%%%%%%%%%%%%%%%%%%%%%%%%%%%%%%%%%%%%%%%%%%%%



\textbf{Exercise 9.6.}
\emph{...} \\

\emph{Proof.}
\begin{enumerate}
\item[(1)]
\item[(2)]

\end{enumerate}
$\Box$ \\\\



%%%%%%%%%%%%%%%%%%%%%%%%%%%%%%%%%%%%%%%%%%%%%%%%%%%%%%%%%%%%%%%%%%%%%%%%%%%%%%%%



\textbf{Exercise 9.7.}
\emph{...} \\

\emph{Proof.}
\begin{enumerate}
\item[(1)]
\item[(2)]

\end{enumerate}
$\Box$ \\\\



%%%%%%%%%%%%%%%%%%%%%%%%%%%%%%%%%%%%%%%%%%%%%%%%%%%%%%%%%%%%%%%%%%%%%%%%%%%%%%%%



\textbf{Exercise 9.8.}
\emph{...} \\

\emph{Proof.}
\begin{enumerate}
\item[(1)]
\item[(2)]

\end{enumerate}
$\Box$ \\\\



%%%%%%%%%%%%%%%%%%%%%%%%%%%%%%%%%%%%%%%%%%%%%%%%%%%%%%%%%%%%%%%%%%%%%%%%%%%%%%%%



\textbf{Exercise 9.9.}
\emph{...} \\

\emph{Proof.}
\begin{enumerate}
\item[(1)]
\item[(2)]

\end{enumerate}
$\Box$ \\\\



%%%%%%%%%%%%%%%%%%%%%%%%%%%%%%%%%%%%%%%%%%%%%%%%%%%%%%%%%%%%%%%%%%%%%%%%%%%%%%%%



\textbf{Exercise 9.10.}
\emph{...} \\

\emph{Proof.}
\begin{enumerate}
\item[(1)]
\item[(2)]

\end{enumerate}
$\Box$ \\\\



%%%%%%%%%%%%%%%%%%%%%%%%%%%%%%%%%%%%%%%%%%%%%%%%%%%%%%%%%%%%%%%%%%%%%%%%%%%%%%%%



\textbf{Exercise 9.11.}
\emph{...} \\

\emph{Proof.}
\begin{enumerate}
\item[(1)]
\item[(2)]

\end{enumerate}
$\Box$ \\\\



%%%%%%%%%%%%%%%%%%%%%%%%%%%%%%%%%%%%%%%%%%%%%%%%%%%%%%%%%%%%%%%%%%%%%%%%%%%%%%%%



\textbf{Exercise 9.12.}
\emph{...} \\

\emph{Proof.}
\begin{enumerate}
\item[(1)]
\item[(2)]

\end{enumerate}
$\Box$ \\\\



%%%%%%%%%%%%%%%%%%%%%%%%%%%%%%%%%%%%%%%%%%%%%%%%%%%%%%%%%%%%%%%%%%%%%%%%%%%%%%%%



\textbf{Exercise 9.13.}
\emph{...} \\

\emph{Proof.}
\begin{enumerate}
\item[(1)]
\item[(2)]

\end{enumerate}
$\Box$ \\\\



%%%%%%%%%%%%%%%%%%%%%%%%%%%%%%%%%%%%%%%%%%%%%%%%%%%%%%%%%%%%%%%%%%%%%%%%%%%%%%%%



\textbf{Exercise 9.14.}
\emph{...} \\

\emph{Proof.}
\begin{enumerate}
\item[(1)]
\item[(2)]

\end{enumerate}
$\Box$ \\\\



%%%%%%%%%%%%%%%%%%%%%%%%%%%%%%%%%%%%%%%%%%%%%%%%%%%%%%%%%%%%%%%%%%%%%%%%%%%%%%%%



\textbf{Exercise 9.15.}
\emph{...} \\

\emph{Proof.}
\begin{enumerate}
\item[(1)]
\item[(2)]

\end{enumerate}
$\Box$ \\\\



%%%%%%%%%%%%%%%%%%%%%%%%%%%%%%%%%%%%%%%%%%%%%%%%%%%%%%%%%%%%%%%%%%%%%%%%%%%%%%%%



\textbf{Exercise 9.16.}
\emph{...} \\

\emph{Proof.}
\begin{enumerate}
\item[(1)]
\item[(2)]

\end{enumerate}
$\Box$ \\\\



%%%%%%%%%%%%%%%%%%%%%%%%%%%%%%%%%%%%%%%%%%%%%%%%%%%%%%%%%%%%%%%%%%%%%%%%%%%%%%%%



\textbf{Exercise 9.17.}
\emph{...} \\

\emph{Proof.}
\begin{enumerate}
\item[(1)]
\item[(2)]

\end{enumerate}
$\Box$ \\\\



%%%%%%%%%%%%%%%%%%%%%%%%%%%%%%%%%%%%%%%%%%%%%%%%%%%%%%%%%%%%%%%%%%%%%%%%%%%%%%%%



\textbf{Exercise 9.18.}
\emph{...} \\

\emph{Proof.}
\begin{enumerate}
\item[(1)]
\item[(2)]

\end{enumerate}
$\Box$ \\\\



%%%%%%%%%%%%%%%%%%%%%%%%%%%%%%%%%%%%%%%%%%%%%%%%%%%%%%%%%%%%%%%%%%%%%%%%%%%%%%%%



\textbf{Exercise 9.19.}
\emph{...} \\

\emph{Proof.}
\begin{enumerate}
\item[(1)]
\item[(2)]

\end{enumerate}
$\Box$ \\\\



%%%%%%%%%%%%%%%%%%%%%%%%%%%%%%%%%%%%%%%%%%%%%%%%%%%%%%%%%%%%%%%%%%%%%%%%%%%%%%%%



\textbf{Exercise 9.20.}
\emph{...} \\

\emph{Proof.}
\begin{enumerate}
\item[(1)]
\item[(2)]

\end{enumerate}
$\Box$ \\\\



%%%%%%%%%%%%%%%%%%%%%%%%%%%%%%%%%%%%%%%%%%%%%%%%%%%%%%%%%%%%%%%%%%%%%%%%%%%%%%%%



\textbf{Exercise 9.21.}
\emph{...} \\

\emph{Proof.}
\begin{enumerate}
\item[(1)]
\item[(2)]

\end{enumerate}
$\Box$ \\\\



%%%%%%%%%%%%%%%%%%%%%%%%%%%%%%%%%%%%%%%%%%%%%%%%%%%%%%%%%%%%%%%%%%%%%%%%%%%%%%%%



\textbf{Exercise 9.22.}
\emph{...} \\

\emph{Proof.}
\begin{enumerate}
\item[(1)]
\item[(2)]

\end{enumerate}
$\Box$ \\\\



%%%%%%%%%%%%%%%%%%%%%%%%%%%%%%%%%%%%%%%%%%%%%%%%%%%%%%%%%%%%%%%%%%%%%%%%%%%%%%%%



\textbf{Exercise 9.23.}
\emph{...} \\

\emph{Proof.}
\begin{enumerate}
\item[(1)]
\item[(2)]

\end{enumerate}
$\Box$ \\\\



%%%%%%%%%%%%%%%%%%%%%%%%%%%%%%%%%%%%%%%%%%%%%%%%%%%%%%%%%%%%%%%%%%%%%%%%%%%%%%%%



\textbf{Exercise 9.24.}
\emph{...} \\

\emph{Proof.}
\begin{enumerate}
\item[(1)]
\item[(2)]

\end{enumerate}
$\Box$ \\\\



%%%%%%%%%%%%%%%%%%%%%%%%%%%%%%%%%%%%%%%%%%%%%%%%%%%%%%%%%%%%%%%%%%%%%%%%%%%%%%%%



\textbf{Exercise 9.25.}
\emph{...} \\

\emph{Proof.}
\begin{enumerate}
\item[(1)]
\item[(2)]

\end{enumerate}
$\Box$ \\\\



%%%%%%%%%%%%%%%%%%%%%%%%%%%%%%%%%%%%%%%%%%%%%%%%%%%%%%%%%%%%%%%%%%%%%%%%%%%%%%%%



\textbf{Exercise 9.26.}
\emph{...} \\

\emph{Proof.}
\begin{enumerate}
\item[(1)]
\item[(2)]

\end{enumerate}
$\Box$ \\\\



%%%%%%%%%%%%%%%%%%%%%%%%%%%%%%%%%%%%%%%%%%%%%%%%%%%%%%%%%%%%%%%%%%%%%%%%%%%%%%%%



\textbf{Exercise 9.27.}
\emph{...} \\

\emph{Proof.}
\begin{enumerate}
\item[(1)]
\item[(2)]

\end{enumerate}
$\Box$ \\\\



%%%%%%%%%%%%%%%%%%%%%%%%%%%%%%%%%%%%%%%%%%%%%%%%%%%%%%%%%%%%%%%%%%%%%%%%%%%%%%%%



\textbf{Exercise 9.28.}
\emph{...} \\

\emph{Proof.}
\begin{enumerate}
\item[(1)]
\item[(2)]

\end{enumerate}
$\Box$ \\\\



%%%%%%%%%%%%%%%%%%%%%%%%%%%%%%%%%%%%%%%%%%%%%%%%%%%%%%%%%%%%%%%%%%%%%%%%%%%%%%%%



\textbf{Exercise 9.29.}
\emph{...} \\

\emph{Proof.}
\begin{enumerate}
\item[(1)]
\item[(2)]

\end{enumerate}
$\Box$ \\\\



%%%%%%%%%%%%%%%%%%%%%%%%%%%%%%%%%%%%%%%%%%%%%%%%%%%%%%%%%%%%%%%%%%%%%%%%%%%%%%%%



\textbf{Exercise 9.30.}
\emph{...} \\

\emph{Proof.}
\begin{enumerate}
\item[(1)]
\item[(2)]

\end{enumerate}
$\Box$ \\\\



%%%%%%%%%%%%%%%%%%%%%%%%%%%%%%%%%%%%%%%%%%%%%%%%%%%%%%%%%%%%%%%%%%%%%%%%%%%%%%%%



\textbf{Exercise 9.31.}
\emph{...} \\

\emph{Proof.}
\begin{enumerate}
\item[(1)]
\item[(2)]

\end{enumerate}
$\Box$ \\\\



%%%%%%%%%%%%%%%%%%%%%%%%%%%%%%%%%%%%%%%%%%%%%%%%%%%%%%%%%%%%%%%%%%%%%%%%%%%%%%%%
%%%%%%%%%%%%%%%%%%%%%%%%%%%%%%%%%%%%%%%%%%%%%%%%%%%%%%%%%%%%%%%%%%%%%%%%%%%%%%%%



\end{document}