\documentclass{article}
\usepackage{amsfonts}
\usepackage{amsmath}
\usepackage{amssymb}
\usepackage{bm}
\usepackage{hyperref}
\usepackage[none]{hyphenat}
\usepackage{mathrsfs}
\usepackage{physics}
\parindent=0pt

\def\upint{\mathchoice%
    {\mkern13mu\overline{\vphantom{\intop}\mkern7mu}\mkern-20mu}%
    {\mkern7mu\overline{\vphantom{\intop}\mkern7mu}\mkern-14mu}%
    {\mkern7mu\overline{\vphantom{\intop}\mkern7mu}\mkern-14mu}%
    {\mkern7mu\overline{\vphantom{\intop}\mkern7mu}\mkern-14mu}%
  \int}
\def\lowint{\mkern3mu\underline{\vphantom{\intop}\mkern7mu}\mkern-10mu\int}

\begin{document}



\textbf{\Large Chapter 9: Functions of Several Variables} \\\\



\emph{Author: Meng-Gen Tsai} \\
\emph{Email: plover@gmail.com} \\\\



% http://www2.math.uu.se/~astrombe/reellanalys2019/solutions_pub.pdf



%%%%%%%%%%%%%%%%%%%%%%%%%%%%%%%%%%%%%%%%%%%%%%%%%%%%%%%%%%%%%%%%%%%%%%%%%%%%%%%%
%%%%%%%%%%%%%%%%%%%%%%%%%%%%%%%%%%%%%%%%%%%%%%%%%%%%%%%%%%%%%%%%%%%%%%%%%%%%%%%%



\textbf{Exercise 9.1.}
\emph{If $S$ is a nonempty subset of a vector space $X$,
prove (as asserted in Section 9.1) that the span of $S$ is a vector space.} \\

Denote the span of $S$ by $\mathrm{span}(S)$. \\

\emph{Proof.}
\begin{enumerate}
\item[(1)]
  Since $S \neq \varnothing$, there is $\mathbf{z} \in S$.
  So $1\mathbf{z} = \mathbf{z} \in \mathrm{span}(S) \neq \varnothing$.
  (In fact, $\mathrm{span}(S) \supseteq S$.)

\item[(2)]
  If $\mathbf{x}, \mathbf{y} \in \mathrm{span}(S)$,
  then there exist elements
  $\mathbf{x}_1, \ldots, \mathbf{x}_m$, $\mathbf{y}_1, \ldots, \mathbf{y}_n \in S$
  and scalars $a_1, \ldots, a_m$, $b_1, \ldots, b_n$ such that
  \begin{align*}
    \mathbf{x} &= a_1 \mathbf{x}_1 + \cdots + a_m \mathbf{x}_m, \\
    \mathbf{y} &= b_1 \mathbf{y}_1 + \cdots + b_n \mathbf{y}_n.
  \end{align*}
  Then
  \[
    \mathbf{x}+\mathbf{y}
    = a_1 \mathbf{x}_1 + \cdots + a_m \mathbf{x}_m
      + b_1 \mathbf{y}_1 + \cdots + b_n \mathbf{y}_n
  \]
  is a linear combination of the elements of $S$.
  For any scalar $c$,
  \[
    c\mathbf{x} = (ca_1) \mathbf{x}_1 + \cdots + (ca_m) \mathbf{x}_m
  \]
  is again linear combination of the elements of $S$.

\item[(3)]
  By (1)(2), $\mathrm{span}(S)$ is a vector space.
\end{enumerate}
$\Box$ \\

\emph{Note.}
Any subspace of $X$ that contains $S$ must also contain $\mathrm{span}(S)$. \\\\



%%%%%%%%%%%%%%%%%%%%%%%%%%%%%%%%%%%%%%%%%%%%%%%%%%%%%%%%%%%%%%%%%%%%%%%%%%%%%%%%



\textbf{Exercise 9.2.}
\emph{Prove (as asserted in Section 9.6) that $BA$ is linear
if $A$ and $B$ are linear transformations.
Prove also that $A^{-1}$ is linear and invertible if $A$ is invertible.} \\

\emph{Proof.}
Use the notation in Definitions 9.6.
\begin{enumerate}
\item[(1)]
  \emph{Show that $BA$ is linear if $A$ and $B$ are linear transformations.}
  Let $X, Y, Z$ be vector spaces, $A \in L(X,Y)$ and $B \in L(Y,Z)$.
  \begin{enumerate}
  \item[(a)]
    Given any $\mathbf{x}_1, \mathbf{x}_2 \in X$.
    \begin{align*}
      (BA)(\mathbf{x}_1+\mathbf{x}_2)
      &= B(A(\mathbf{x}_1+\mathbf{x}_2)) \\
      &= B(A\mathbf{x}_1+A\mathbf{x}_2)
        & (\text{$A$ is a linear transformation}) \\
      &= B(A\mathbf{x}_1) + B(A\mathbf{x}_2)
        & (\text{$B$ is a linear transformation}) \\
      &= (BA)\mathbf{x}_1 + (BA)\mathbf{x}_2.
    \end{align*}

  \item[(b)]
    For any $\mathbf{x} \in X$ and scalar $c$,
    \begin{align*}
    (BA)(c\mathbf{x})
    &= B(A(c\mathbf{x})) \\
    &= B(cA\mathbf{x})
      & (\text{$A$ is a linear transformation}) \\
    &= cB(A\mathbf{x})
      & (\text{$B$ is a linear transformation}) \\
    &= c(BA)\mathbf{x}.
    \end{align*}
  \end{enumerate}
  By (a)(b), $BA \in L(X,Z)$.

\item[(2)]
  \emph{Show that $A^{-1}$ is linear if $A$ is invertible.}
  \begin{enumerate}
  \item[(a)]
    Given any $\mathbf{y}_1, \mathbf{y}_2 \in X$.
    Since $A$ is surjective,
    there exist $\mathbf{x}_1, \mathbf{x}_2 \in X$ such that
    \begin{align*}
      \mathbf{y}_1 &= A\mathbf{x}_1 \\
      \mathbf{y}_2 &= A\mathbf{x}_2.
    \end{align*}
    So
    \begin{align*}
      A^{-1}\mathbf{y}_1 &= A^{-1}(A\mathbf{x}_1) = \mathbf{x}_1 \\
      A^{-1}\mathbf{y}_2 &= A^{-1}(A\mathbf{x}_2) = \mathbf{x}_2
    \end{align*}
    (by Definitions 9.4).
    Hence
    \begin{align*}
      A^{-1}(\mathbf{y}_1+\mathbf{y}_2)
      &= A^{-1}(A\mathbf{x}_1+A\mathbf{x}_2) \\
      &= A^{-1}(A(\mathbf{x}_1+\mathbf{x}_2))
        & (\text{$A$ is a linear transformation}) \\
      &= \mathbf{x}_1+\mathbf{x}_2
        & (\text{Definitions 9.4}) \\
      &= A^{-1}\mathbf{y}_1+A^{-1}\mathbf{y}_2.
    \end{align*}

  \item[(b)]
    For any $\mathbf{y} \in X$ and scalar $c$,
    there is a corresponding $\mathbf{x} \in X$ such that $\mathbf{y} = A\mathbf{x}$
    since $A$ is surjective. So $A^{-1}\mathbf{y} = \mathbf{x}$ by Definition 9.4.
    Hence
    \begin{align*}
      A^{-1}(c\mathbf{y})
      &= A^{-1}(cA\mathbf{x}) \\
      &= A^{-1}(A(c\mathbf{x}))
        & (\text{$A$ is a linear transformation}) \\
      &= c\mathbf{x}
        & (\text{Definitions 9.4}) \\
      &= cA^{-1}\mathbf{y}.
    \end{align*}
  \end{enumerate}
  By (a)(b), $A^{-1} \in L(X)$.

\item[(3)]
  \emph{Show that $A^{-1}$ is invertible if $A$ is invertible.}
  It suffices to show that $A^{-1}$ is injective and surjective.
  \begin{enumerate}
  \item[(a)]
    \emph{Show that $A^{-1}$ is injective.}
    Given any $\mathbf{y}_1, \mathbf{y}_2 \in X$.
    Since $A$ is surjective,
    there exist $\mathbf{x}_1, \mathbf{x}_2 \in X$ such that
    \begin{align*}
      \mathbf{y}_1 &= A\mathbf{x}_1 \\
      \mathbf{y}_2 &= A\mathbf{x}_2.
    \end{align*}
    Suppose $A^{-1}\mathbf{y}_1 = A^{-1}\mathbf{y}_2$.
    So $A^{-1}(A\mathbf{x}_1) = A^{-1}(A\mathbf{x}_2)$,
    or $\mathbf{x}_1 = \mathbf{x}_2$,
    or $\mathbf{y}_1 = A\mathbf{x}_1 = A\mathbf{x}_2 = \mathbf{y}_2$.

  \item[(b)]
    \emph{Show that $A^{-1}$ is surjective.}
    For any $\mathbf{x} \in X$, there exists $A\mathbf{x} \in X$ such that
    $A^{-1}(A\mathbf{x}) = \mathbf{x}$ by Definitions 9.4.
  \end{enumerate}
\end{enumerate}
$\Box$ \\\\



%%%%%%%%%%%%%%%%%%%%%%%%%%%%%%%%%%%%%%%%%%%%%%%%%%%%%%%%%%%%%%%%%%%%%%%%%%%%%%%%



\textbf{Exercise 9.3.}
\emph{Assume $A \in L(X,Y)$ and $A\mathbf{x} = \mathbf{0}$ only when $\mathbf{x} = \mathbf{0}$.
Prove that $A$ is then $1$-$1$.} \\

\emph{Proof.}
Suppose $A\mathbf{x} = A\mathbf{y}$.
Since $A$ is a linear transformation,
$A(\mathbf{x}-\mathbf{y}) = A\mathbf{x} - A\mathbf{y} = \mathbf{0}$.
By assumption, $\mathbf{x}-\mathbf{y} = \mathbf{0}$
or $\mathbf{x} = \mathbf{y}$.
$\Box$ \\\\



%%%%%%%%%%%%%%%%%%%%%%%%%%%%%%%%%%%%%%%%%%%%%%%%%%%%%%%%%%%%%%%%%%%%%%%%%%%%%%%%



\textbf{Exercise 9.4.}
\emph{Prove (as asserted in Section 9.30) that null spaces and ranges of
linear transformations are vector spaces.} \\

\emph{Proof.}
Use the notation in Definitions 9.30.
Suppose $X$, $Y$ are vector spaces, and $A \in L(X,Y)$, as in Definition 9.6.
\begin{enumerate}
\item[(1)]
  \emph{Show that $\mathscr{N}(A)$ is a vector space in $X$.}
  \begin{enumerate}
  \item[(a)]
    Note that $\mathbf{0} \in X$.
    Since $A\mathbf{0} = \mathbf{0}$, $\mathbf{0} \in \mathscr{N}(A) \neq \varnothing$.

  \item[(b)]
    Suppose $\mathbf{x}_1, \mathbf{x}_2 \in \mathscr{N}(A)$.
    Then
    \begin{align*}
      A(\mathbf{x}_1+\mathbf{x}_2)
      &= A\mathbf{x}_1+A\mathbf{x}_2
        & (\text{$A$ is a linear transformation}) \\
      &= \mathbf{0}+\mathbf{0}
        & (\mathbf{x}_1, \mathbf{x}_2 \in \mathscr{N}(A)) \\
      &= \mathbf{0}.
    \end{align*}
    So $\mathbf{x}_1+\mathbf{x}_2 \in \mathscr{N}(A)$.

  \item[(c)]
    Suppose $\mathbf{x} \in \mathscr{N}(A)$ and $c$ is a scalar.
    Then
    \begin{align*}
      A(c\mathbf{x})
      &= cA\mathbf{x}
        & (\text{$A$ is a linear transformation}) \\
      &= c\mathbf{0}
        & (\mathbf{x} \in \mathscr{N}(A)) \\
      &= \mathbf{0}.
    \end{align*}
    So $c\mathbf{x} \in \mathscr{N}(A)$.
  \end{enumerate}
  By (a)(b)(c), $\mathscr{N}(A)$ is a vector space.

\item[(2)]
  \emph{Show that $\mathscr{R}(A)$ is a vector space in $Y$.}
  \begin{enumerate}
  \item[(a)]
    Note that $\mathbf{0} \in X$.
    So $A\mathbf{0} = \mathbf{0} \in \mathscr{R}(A) \neq \varnothing$.

  \item[(b)]
    Suppose $\mathbf{y}_1, \mathbf{y}_2 \in \mathscr{R}(A)$.
    Then there exist $\mathbf{x}_1, \mathbf{x}_2 \in X$
    such that $A\mathbf{x}_1 = \mathbf{y}_1$
    and $A\mathbf{x}_2 = \mathbf{y}_2$.
    Hence
    \begin{align*}
      \mathbf{y}_1+\mathbf{y}_2
      &= A\mathbf{x}_1+A\mathbf{x}_2 \\
      &= A(\mathbf{x}_1+\mathbf{x}_2)
        & (\text{$A$ is a linear transformation}).
    \end{align*}
    So $\mathbf{y}_1+\mathbf{y}_2 \in \mathscr{R}(A)$.

  \item[(c)]
    Suppose $\mathbf{y} \in \mathscr{R}(A)$ and $c$ is a scalar.
    Then there exists $\mathbf{x} \in X$ such that $A\mathbf{x} = \mathbf{y}$.
    Hence
    \begin{align*}
      c\mathbf{y}
      &= cA\mathbf{x} \\
      &= A(c\mathbf{x})
        & (\text{$A$ is a linear transformation}).
    \end{align*}
    So $c\mathbf{y} \in \mathscr{R}(A)$.
  \end{enumerate}
  By (a)(b)(c), $\mathscr{R}(A)$ is a vector space.
\end{enumerate}
$\Box$ \\\\



%%%%%%%%%%%%%%%%%%%%%%%%%%%%%%%%%%%%%%%%%%%%%%%%%%%%%%%%%%%%%%%%%%%%%%%%%%%%%%%%



\textbf{Exercise 9.5.}
\emph{Prove that to every $A \in L(\mathbb{R}^n, \mathbb{R}^1)$
corresponds a unique $\mathbf{y} \in \mathbb{R}^n$ such that
$A\mathbf{x} = \mathbf{x} \cdot \mathbf{y}$.
Prove also that $\norm{A} = |\mathbf{y}|$.
(Hint: Under certain conditions, equality holds in the Schwarz inequality.)} \\

\emph{Proof.}
\begin{enumerate}
\item[(1)]
  Recall that $\{ \mathbf{e}_1, \ldots, \mathbf{e}_n \}$
  is the standard basis of $\mathbb{R}^n$ (Definitions 9.1).
  Given any $\mathbf{x} \in \mathbb{R}^n$,
  write $\mathbf{x} = (x_1, \ldots, x_n)$ as $\mathbf{x} = \sum x_j \mathbf{e}_j$.

\item[(2)]
  \emph{Show that $\mathbf{y}$ exists.}
  Since $A$ is a linear transformation,
  \begin{align*}
    A\mathbf{x}
    &= A\left(\sum x_j \mathbf{e}_j\right) \\
    &= \sum x_j A\mathbf{e}_j \\
    &= (x_1, \ldots, x_n) \cdot (A\mathbf{e}_1, \ldots, A\mathbf{e}_n) \\
    &= \mathbf{x} \cdot \sum (A\mathbf{e}_j) \mathbf{e}_j.
  \end{align*}
  Define $\mathbf{y} = \sum (A\mathbf{e}_j) \mathbf{e}_j \in \mathbb{R}^n$
  so that $A\mathbf{x} = \mathbf{x} \cdot \mathbf{y}$.

\item[(3)]
  \emph{Show that $\mathbf{y}$ is unique.}
  Suppose there exists some $\mathbf{z} \in \mathbb{R}^n$
  such that $A\mathbf{x} = \mathbf{x} \cdot \mathbf{z}$.
  So
  \begin{align*}
    0
    &= A\mathbf{x} -  A\mathbf{x} \\
    &= \mathbf{x} \cdot \mathbf{y}-\mathbf{x} \cdot \mathbf{z} \\
    &= \mathbf{x} \cdot (\mathbf{y}-\mathbf{z})
  \end{align*}
  for any $\mathbf{x} \in \mathbb{R}^n$.
  In particular, take $\mathbf{x} = \mathbf{y}-\mathbf{z} \in \mathbb{R}^n$
  to get
  \[
    0
    = (\mathbf{y}-\mathbf{z}) \cdot (\mathbf{y}-\mathbf{z})
    = |\mathbf{y}-\mathbf{z}|^2
  \]
  or $\mathbf{y}-\mathbf{z} = \mathbf{0}$ or $\mathbf{y}=\mathbf{z}$.

\item[(4)]
  \emph{Show that $\norm{A} = |\mathbf{y}|$.}
  By the Schwarz inequality (Theorem 1.37(d)),
  \[
    |A\mathbf{x}| = |\mathbf{x} \cdot \mathbf{y}| \leq |\mathbf{x}||\mathbf{y}|
    \leq |\mathbf{y}|
  \]
  as $|\mathbf{x}| \leq 1$.
  Take the sup over all $|\mathbf{x}| \leq 1$ to get
  \[
    \norm{A} \leq |\mathbf{y}|.
  \]
  If $\mathbf{y} = \mathbf{0}$, then $\norm{A} = |\mathbf{y}| = 0$.
  If $\mathbf{y} \neq \mathbf{0}$,
  then the equality holds when
  $\mathbf{x} = \frac{\mathbf{y}}{|\mathbf{y}|} \in \mathbb{R}^n$.
  (Here $|\mathbf{x}| = 1$.)
\end{enumerate}
$\Box$ \\\\



%%%%%%%%%%%%%%%%%%%%%%%%%%%%%%%%%%%%%%%%%%%%%%%%%%%%%%%%%%%%%%%%%%%%%%%%%%%%%%%%



\textbf{Exercise 9.6.}
\emph{If $f(0,0) = 0$ and
\[
  f(x,y) = \frac{xy}{x^2+y^2}
  \qquad \text{if }
  (x,y) \neq (0,0),
\]
prove that
$(D_1 f)(x,y)$ and $(D_2 f)(x,y)$ exist at every point of $\mathbb{R}^2$,
although $f$ is not continuous at $(0,0)$.} \\

\emph{Proof.}
\begin{enumerate}
\item[(1)]
  \emph{Show that}
  \begin{equation*}
  (D_1 f)(x,y) =
    \begin{cases}
      0                              & \text{ if $(x,y)=(0,0)$}, \\
      \frac{y(y^2-x^2)}{(x^2+y^2)^2} & \text{ if $(x,y)\neq(0,0)$}.
    \end{cases}
  \end{equation*}
  Write
  \begin{align*}
    (D_1 f)(x,y)
    &= \lim_{t \to 0} \frac{f((x,y) + t(1,0)) - f(x,y)}{t} \\
    &= \lim_{t \to 0} \frac{f(x+t,y) - f(x,y)}{t}.
  \end{align*}
  If $(x,y) = (0,0)$,
  \[
    (D_1 f)(0,0)
    = \lim_{t \to 0} \frac{f(t,0) - f(0,0)}{t}
    = \lim_{t \to 0} \frac{0 - 0}{t}
    = 0.
  \]
  If $(x,y) \neq (0,0)$,
  \begin{align*}
    (D_1 f)(x,y)
    &= \lim_{t \to 0} \frac{f(x+t,y) - f(x,y)}{t} \\
    &= \lim_{t \to 0} \frac{\frac{(x+t)y}{(x+t)^2+y^2} - \frac{xy}{x^2+y^2}}{t} \\
    &= \lim_{t \to 0} \frac{y(y^2-x^2)-txy}{((x+t)^2+y^2)(x^2+y^2)} \\
    &= \frac{y(y^2-x^2)}{(x^2+y^2)^2}.
  \end{align*}

\item[(2)]
  \emph{Show that}
  \begin{equation*}
  (D_2 f)(x,y) =
    \begin{cases}
      0                              & \text{ if $(x,y)=(0,0)$}, \\
      \frac{x(x^2-y^2)}{(x^2+y^2)^2} & \text{ if $(x,y)\neq(0,0)$}.
    \end{cases}
  \end{equation*}
  Similar to (1).

\item[(3)]
  \emph{Show that $f$ is not continuous at $(0,0)$.}
  Note that
  \[
    \lim_{n \to \infty} f\left(\frac{1}{n},\frac{1}{n}\right)
    = \lim_{n \to \infty} \frac{\frac{1}{n} \cdot \frac{1}{n}}{\frac{1}{n^2}+\frac{1}{n^2}}
    = \lim_{n \to \infty} \frac{1}{2}
    = \frac{1}{2}
  \]
  and
  \[
    \lim_{n \to \infty} f\left(\frac{1}{n},0\right)
    = \lim_{n \to \infty} \frac{0}{\frac{1}{n^2} + 0}
    = \lim_{n \to \infty} 0
    = 0.
  \]
  Hence the limit $\lim_{(x,y) \to (0,0)} f(x,y)$ does not exist.
\end{enumerate}
$\Box$ \\\\



%%%%%%%%%%%%%%%%%%%%%%%%%%%%%%%%%%%%%%%%%%%%%%%%%%%%%%%%%%%%%%%%%%%%%%%%%%%%%%%%



\textbf{Exercise 9.7.}
\emph{Suppose that $f$ is a real-valued function
defined in an open set $E \subseteq \mathbb{R}^n$,
and that the partial derivatives $D_1 f, \ldots, D_n f$ are bounded in $E$.
Prove that $f$ is continuous in $E$.
(Hint: Proceed as in the proof of Theorem 9.21.)} \\

\emph{Proof.}
\begin{enumerate}
\item[(1)]
  Since $D_j f$ is bounded in $E$, there is a real number $M_j$ such that
  $|D_j f| \leq M_j$ in $E$.
  Take $M = \max_{1 \leq j \leq n} M_j$ so that $|D_j f| \leq M$ in $E$ for all $1 \leq j \leq n$.

\item[(2)]
  Fix $\mathbf{x} \in E$ and $\varepsilon > 0$.
  Since $E$ is open,
  there is an open neighborhood
  \[
    B(\mathbf{x};r)
    = \{ \mathbf{x}+\mathbf{h} \in E : |\mathbf{h}| < r \}
    \subseteq E
  \]
  with
  \[
    0 < r < \frac{\varepsilon}{n(M+1)}.
  \]

\item[(3)]
  Write $\mathbf{h} = \sum h_j \mathbf{e}_j$, $|\mathbf{h}| < r$,
  put $\mathbf{v}_0 = \mathbf{0}$,
  and $\mathbf{v}_k = h_1 \mathbf{e}_1 + \cdots + h_k \mathbf{e}_k$
  for $1 \leq k \leq n$.
  Then
  \[
    f(\mathbf{x}+\mathbf{h}) - f(\mathbf{x})
    = \sum_{j=1}^{n}
      [f(\mathbf{x}+\mathbf{v}_j) - f(\mathbf{x}+\mathbf{v}_{j-1})].
  \]
  Since $|\mathbf{v}_k| < r$ for $1 \leq k \leq n$ and
  since $B(\mathbf{x};r)$ is convex,
  the open interval with end points
  $\mathbf{x}+\mathbf{v}_{j-1}$ and $\mathbf{x}+\mathbf{v}_{j}$ lie in $B(\mathbf{x};r)$.
  Since $\mathbf{v}_{j} = \mathbf{v}_{j-1} - h_j\mathbf{e}_j$,
  the mean value theorem (Theorem 5.10) show that
  \[
    f(\mathbf{x}+\mathbf{v}_j) - f(\mathbf{x}+\mathbf{v}_{j-1})
    = h_j (D_j f)(\mathbf{x}+\mathbf{v}_{j-1} + \theta_j h_j \mathbf{e}_j)
  \]
  for some $\theta_j \in (0,1)$.

\item[(4)]
  Note that $\abs{h_j} \leq \abs{\mathbf{h}} < r < \frac{\varepsilon}{n(M+1)}$.
  Hence
  \begin{align*}
    \abs{f(\mathbf{x}+\mathbf{h}) - f(\mathbf{x})}
    &\leq \sum_{j=1}^{n}
      \abs{f(\mathbf{x}+\mathbf{v}_j) - f(\mathbf{x}+\mathbf{v}_{j-1})} \\
    &= \sum_{j=1}^{n}
      \abs{h_j} \abs{(D_j f)(\mathbf{x}+\mathbf{v}_{j-1} + \theta_j h_j \mathbf{e}_j)} \\
    &\leq \sum_{j=1}^{n}
      \frac{\varepsilon}{n(M+1)} \cdot M \\
    &< \varepsilon
  \end{align*}
  as $\abs{\mathbf{h}} < r < \frac{\varepsilon}{n(M+1)}$.
  Hence $f$ is continuous at all $\mathbf{x} \in E$.
\end{enumerate}
$\Box$ \\\\



%%%%%%%%%%%%%%%%%%%%%%%%%%%%%%%%%%%%%%%%%%%%%%%%%%%%%%%%%%%%%%%%%%%%%%%%%%%%%%%%



\textbf{Exercise 9.8.}
\emph{Suppose that $f$ is a differentiable real function in an open set
$E \subseteq \mathbb{R}^n$,
and that $f$ has a local maximum at a point $\mathbf{x} \in E$.
Prove that $f'(\mathbf{x}) = \mathbf{0}$.} \\

\emph{Proof (Theorem 5.8).}
\begin{enumerate}
\item[(1)]
Apply Theorem 5.8 to each $D_j f$ for $1 \leq j \leq n$.
Since $f$ has a local maximum at a point $\mathbf{x} \in E$,
there is an open neighborhood $B(\mathbf{x};r)$ of $\mathbf{x}$ in $E$
such that
\[
  f(\mathbf{y}) \leq f(\mathbf{x})
\]
for all $\mathbf{y} \in B(\mathbf{x};r)$.
Therefore,
\[
  f(\mathbf{x} + t\mathbf{e}_j) \leq f(\mathbf{x})
\]
for all $|t| < r$ and $1 \leq j \leq n$,
or $t \mapsto f(\mathbf{x} + t\mathbf{e}_j)$ has a local maximum at a point
$t = 0 \in (-r,r)$.


\item[(2)]
Since $f$ is a differentiable in $E$, each partial derivatives $D_j f$ exist (Theorem 9.21).
Hence Theorem 5.8 implies that $(D_j f)(\mathbf{x}) = 0$ for all $1 \leq j \leq n$.
So
\[
  f'(\mathbf{x})
  = [ (D_1 f)(\mathbf{x}) \cdots (D_k f)(\mathbf{x}) ]
  = [ 0 \cdots 0 ]
  = \mathbf{0}.
\]

\end{enumerate}
$\Box$ \\\\



%%%%%%%%%%%%%%%%%%%%%%%%%%%%%%%%%%%%%%%%%%%%%%%%%%%%%%%%%%%%%%%%%%%%%%%%%%%%%%%%



\textbf{Exercise 9.9.}
\emph{...} \\

\emph{Proof.}
\begin{enumerate}
\item[(1)]
\item[(2)]

\end{enumerate}
$\Box$ \\\\



%%%%%%%%%%%%%%%%%%%%%%%%%%%%%%%%%%%%%%%%%%%%%%%%%%%%%%%%%%%%%%%%%%%%%%%%%%%%%%%%



\textbf{Exercise 9.10.}
\emph{...} \\

\emph{Proof.}
\begin{enumerate}
\item[(1)]
\item[(2)]

\end{enumerate}
$\Box$ \\\\



%%%%%%%%%%%%%%%%%%%%%%%%%%%%%%%%%%%%%%%%%%%%%%%%%%%%%%%%%%%%%%%%%%%%%%%%%%%%%%%%



\textbf{Exercise 9.11.}
\emph{If $f$ and $g$ are differentiable real functions in $\mathbb{R}^n$,
prove that
\[
  \nabla(fg) = f \nabla g + g \nabla f
\]
and that
\[
  \nabla\left(\frac{1}{f}\right) = -\frac{1}{f^2} \nabla f
\]
whenever $f \neq 0$.} \\

\emph{Proof.}
Recall Example 9.18:
\[
  (\nabla(f))(\mathbf{x}) = \sum_{i=1}^{n}(D_i f)(\mathbf{x})\mathbf{e}_i.
\]

\begin{enumerate}
\item[(1)]
  \emph{Show that $\nabla(fg) = f \nabla g + g \nabla f$.}
  For any $\mathbf{x} \in \mathbb{R}^n$,
  \begin{align*}
    (\nabla(fg))(\mathbf{x})
    &= \sum_{i=1}^{n}(D_i(fg))(\mathbf{x})\mathbf{e}_i \\
    &= \sum_{i=1}^{n}(g (D_i f) + f (D_i g))(\mathbf{x})\mathbf{e}_i
      &(\text{Theorem 5.3(b)}) \\
    &= \sum_{i=1}^{n} \left[ g(\mathbf{x}) (D_i f)(\mathbf{x})
      + f(\mathbf{x}) (D_i g)(\mathbf{x}) \right]\mathbf{e}_i \\
    &= g(\mathbf{x}) \sum_{i=1}^{n} (D_i f)(\mathbf{x})\mathbf{e}_i
      + f(\mathbf{x}) \sum_{i=1}^{n} (D_i g)(\mathbf{x})\mathbf{e}_i \\
    &= g(\mathbf{x}) (\nabla f)(\mathbf{x})
      + f(\mathbf{x}) (\nabla g)(\mathbf{x}) \\
    &= (f \nabla g + g \nabla f)(\mathbf{x}).
  \end{align*}

\item[(2)]
  \emph{Show that $$\nabla\left(\frac{1}{f}\right) = -\frac{1}{f^2} \nabla f$$
  whenever $f \neq 0$.}
  Note that $\nabla(1) = 0$
  since
  \[
    \nabla(1)(\mathbf{x})
    = \sum (D_i 1)(\mathbf{x})\mathbf{e}_i
    = \sum (0)(\mathbf{x})\mathbf{e}_i
    = \sum 0 \mathbf{e}_i
    = 0.
  \]
  Hence as $f \neq 0$, we have
  \begin{align*}
    0
    &= \nabla(1) \\
    &= \nabla\left(f \frac{1}{f}\right)
      & (f \neq 0) \\
    &= f \nabla\left(\frac{1}{f}\right) + \frac{1}{f} \nabla f
      & ((1)),
  \end{align*}
  or $\nabla\left(\frac{1}{f}\right) = -\frac{1}{f^2} \nabla f$.
\end{enumerate}
$\Box$ \\\\



%%%%%%%%%%%%%%%%%%%%%%%%%%%%%%%%%%%%%%%%%%%%%%%%%%%%%%%%%%%%%%%%%%%%%%%%%%%%%%%%



\textbf{Exercise 9.12.}
\emph{...} \\

\emph{Proof.}
\begin{enumerate}
\item[(1)]
\item[(2)]

\end{enumerate}
$\Box$ \\\\



%%%%%%%%%%%%%%%%%%%%%%%%%%%%%%%%%%%%%%%%%%%%%%%%%%%%%%%%%%%%%%%%%%%%%%%%%%%%%%%%



\textbf{Exercise 9.13.}
\emph{Suppose $\mathbf{f}$ is a differentiable mapping of $\mathbb{R}^1$ into
$\mathbb{R}^3$ such that $|\mathbf{f}(t)| = 1$ for every $t$.
Prove that $\mathbf{f}'(t) \cdot \mathbf{f}(t) = 0$.
Interpret this result geometrically.} \\

\emph{Proof.}
\begin{enumerate}
\item[(1)]
  Write $\mathbf{f} = (f_1, f_2, f_3)$ as a vector-valued function.
  By Remarks 5.16, $\mathbf{f}$ is differentiable if and only if each $f_1, f_2, f_3$
  is differentiable. So $\mathbf{f}' = (f_1', f_2', f_3)'$.
  Hence
  \begin{align*}
    &\text{$|\mathbf{f}(t)| = 1$ for every $t$} \\
    \Longleftrightarrow&
    \mathbf{f}(t) \cdot \mathbf{f}(t) = 1 \\
    \Longleftrightarrow&
    f_1(t)^2 + f_2(t)^2 + f_3(t)^2 = 1 \\
    \Longrightarrow&
    2 f_1(t) f_1'(t) + 2 f_2(t) f_2'(t) + 2 f_3(t) f_3'(t) = 0 \\
    \Longleftrightarrow&
    f_1(t) f_1'(t) + f_2(t) f_2'(t) + f_3(t) f_3'(t) = 0 \\
    \Longleftrightarrow&
    (f_1(t), f_2(t), f_3(t)) \cdot (f_1'(t), f_2'(t), f_3'(t)) = 0 \\
    \Longleftrightarrow&
    \mathbf{f}(t) \cdot \mathbf{f}'(t) = \mathbf{f}'(t) \cdot \mathbf{f}(t) = 0.
  \end{align*}

\item[(2)]
  The vector $\mathbf{f}'(t)$ is called the
  \emph{\textbf{tangent vector}} (or \emph{\textbf{velocity vector}})
  of $\mathbf{f}$ at $t$.
  Geometrically,
  given any mapping $\mathbf{f}$ lying on the sphere $S^2$,
  its tangent vector at $t$ is lying on the tangent plane of $S^2$ at $t$.
\end{enumerate}
$\Box$ \\\\



%%%%%%%%%%%%%%%%%%%%%%%%%%%%%%%%%%%%%%%%%%%%%%%%%%%%%%%%%%%%%%%%%%%%%%%%%%%%%%%%



\textbf{Exercise 9.14.}
\emph{Define $f(0,0) = 0$ and}
\[
  f(x,y) = \frac{x^3}{x^2 + y^2}
  \qquad \text{if }
  (x,y) \neq (0,0).
\]
\begin{enumerate}
\item[(a)]
  \emph{Prove that $D_1 f$ and $D_2 f$ are bounded functions in $\mathbb{R}^2$.
  (Hence $f$ is continuous.)}

\item[(b)]
  \emph{Let $\mathbf{u}$ be any unit vector in $\mathbb{R}^2$.
  Show that the directional derivative
  $(D_{\mathbf{u}}f)(0,0)$ exists, and that its absolute value is at most $1$.}

\item[(c)]

\item[(d)]
  \emph{In spite of this, prove that $f$ is not differentiable at $(0,0)$.} \\

\end{enumerate}



\emph{Proof of (a).}
\begin{enumerate}
\item[(1)]
  \emph{Show that}
    \begin{equation*}
    (D_1 f)(x,y) =
      \begin{cases}
        1
          & \text{ if $(x,y)=(0,0)$}, \\
        \frac{x^2(x^2 +3y^2)}{(x^2+y^2)^2}
          & \text{ if $(x,y)\neq(0,0)$}.
      \end{cases}
    \end{equation*}
    If $(x,y) = (0,0)$,
    \[
      (D_1 f)(0,0)
      = \lim_{t \to 0} \frac{f(t,0) - f(0,0)}{t}
      = \lim_{t \to 0} \frac{t - 0}{t}
      = 1.
    \]
    If $(x,y) \neq (0,0)$,
    \begin{align*}
      (D_1 f)(x,y)
      &= \lim_{t \to 0} \frac{f(x+t,y) - f(x,y)}{t} \\
      &= \lim_{t \to 0} \frac{\frac{(x+t)^3}{(x+t)^2+y^2} - \frac{x^3}{x^2+y^2}}{t} \\
      &= \lim_{t \to 0}
        \frac{x^2(x^2+3y^2) + tx(2x^2+3y^2) + t^2(x^2+y^2)}{((x+t)^2+y^2)(x^2+y^2)} \\
      &= \frac{x^2(x^2 +3y^2)}{(x^2+y^2)^2}.
    \end{align*}
    (Or differentiate directly.)

\item[(2)]
  \emph{Show that $(D_1 f)(x,y)$ is bounded.}
  It suffices to show that $(D_1 f)(x,y)$ is bounded if $(x,y) \neq (0,0)$.
  Write $x = r\cos\theta$ and $y = r\sin\theta$ in the polar coordinates.
  (Here $r > 0$.)
  Hence
  \[
    (D_1 f)(x,y)
    = \frac{x^2(x^2 +3y^2)}{(x^2+y^2)^2}
    = \cos^2\theta (\cos^2\theta + 3\sin^2\theta)
  \]
  is bounded by $1\cdot(1+3) = 4$.

\item[(3)]
  \emph{Show that}
    \begin{equation*}
    (D_2 f)(x,y) =
      \begin{cases}
        0
          & \text{ if $(x,y)=(0,0)$}, \\
        \frac{-2x^3y}{(x^2+y^2)^2}
          & \text{ if $(x,y)\neq(0,0)$}.
      \end{cases}
    \end{equation*}
    If $(x,y) = (0,0)$,
    \[
      (D_2 f)(0,0)
      = \lim_{t \to 0} \frac{f(0,t) - f(0,0)}{t}
      = \lim_{t \to 0} \frac{0 - 0}{t}
      = 0.
    \]
    If $(x,y) \neq (0,0)$,
    \begin{align*}
      (D_2 f)(x,y)
      &= \lim_{t \to 0} \frac{f(x,y+t) - f(x,y)}{t} \\
      &= \lim_{t \to 0} \frac{\frac{x^3}{x^2+(y+t)^2} - \frac{x^3}{x^2+y^2}}{t} \\
      &= \lim_{t \to 0}
        \frac{-2x^3y - tx^3}{(x^2+(y+t)^2)(x^2+y^2)} \\
      &= \frac{-2x^3y}{(x^2+y^2)^2}.
    \end{align*}
    (Or differentiate directly.)

\item[(4)]
  \emph{Show that $(D_2 f)(x,y)$ is bounded.}
  Similar to (2).

\item[(5)]
  \emph{Show that $f$ is continuous.}
  Apply Exercise 9.7 to (2)(4).
\end{enumerate}
$\Box$ \\



\emph{Proof of (b).}
\begin{enumerate}
\item[(1)]
  Write $\mathbf{u} = (u_1, u_2)$.
  The formula
  \[
    (D_{\mathbf{u}}f)(0,0) = (D_1 f)(0,0)u_1 + (D_2 f)(0,0)u_2 = u_1
  \]
  might be false since we don't know if $f$ is differentiable or not.
  Actually, we will show that $(D_{\mathbf{u}}f)(0,0) = u_1^3 \neq u_1$.

\item[(2)]
  \begin{align*}
    (D_{\mathbf{u}}f)(0,0)
    &= \lim_{t \to 0} \frac{f(tu_1,tu_2) - f(0,0)}{t} \\
    &= \lim_{t \to 0} \frac{\frac{t^3u_1^3}{t^2u_1^2 + t^2u_2^2} - 0}{t} \\
    &= \lim_{t \to 0} u_1^3
      & (|\mathbf{u}| = 1) \\
    &= u_1^3.
  \end{align*}
  Also $\abs{(D_{\mathbf{u}}f)(0,0)} = \abs{u_1}^3 \leq 1$
  since $|\mathbf{u}| = 1$.
\end{enumerate}
$\Box$ \\



\emph{Proof of (d).}
  (Reductio ad absurdum)
  If $f$ were differentiable,
  then
  \[
    (D_{\mathbf{u}}f)(0,0) = (D_1 f)(0,0)u_1 + (D_2 f)(0,0)u_2 = u_1
  \]
  (Formula (40) in Chapter 9), contrary to (b).
$\Box$ \\\\



%%%%%%%%%%%%%%%%%%%%%%%%%%%%%%%%%%%%%%%%%%%%%%%%%%%%%%%%%%%%%%%%%%%%%%%%%%%%%%%%



\textbf{Exercise 9.15.}
\emph{...} \\

\emph{Proof.}
\begin{enumerate}
\item[(1)]
\item[(2)]

\end{enumerate}
$\Box$ \\\\



%%%%%%%%%%%%%%%%%%%%%%%%%%%%%%%%%%%%%%%%%%%%%%%%%%%%%%%%%%%%%%%%%%%%%%%%%%%%%%%%



\textbf{Exercise 9.16.}
\emph{...} \\

\emph{Proof.}
\begin{enumerate}
\item[(1)]
\item[(2)]

\end{enumerate}
$\Box$ \\\\



%%%%%%%%%%%%%%%%%%%%%%%%%%%%%%%%%%%%%%%%%%%%%%%%%%%%%%%%%%%%%%%%%%%%%%%%%%%%%%%%



\textbf{Exercise 9.17.}
\emph{...} \\

\emph{Proof.}
\begin{enumerate}
\item[(1)]
\item[(2)]

\end{enumerate}
$\Box$ \\\\



%%%%%%%%%%%%%%%%%%%%%%%%%%%%%%%%%%%%%%%%%%%%%%%%%%%%%%%%%%%%%%%%%%%%%%%%%%%%%%%%



\textbf{Exercise 9.18.}
\emph{...} \\

\emph{Proof.}
\begin{enumerate}
\item[(1)]
\item[(2)]

\end{enumerate}
$\Box$ \\\\



%%%%%%%%%%%%%%%%%%%%%%%%%%%%%%%%%%%%%%%%%%%%%%%%%%%%%%%%%%%%%%%%%%%%%%%%%%%%%%%%



\textbf{Exercise 9.19.}
\emph{...} \\

\emph{Proof.}
\begin{enumerate}
\item[(1)]
\item[(2)]

\end{enumerate}
$\Box$ \\\\



%%%%%%%%%%%%%%%%%%%%%%%%%%%%%%%%%%%%%%%%%%%%%%%%%%%%%%%%%%%%%%%%%%%%%%%%%%%%%%%%



\textbf{Exercise 9.20.}
\emph{...} \\

\emph{Proof.}
\begin{enumerate}
\item[(1)]
\item[(2)]

\end{enumerate}
$\Box$ \\\\



%%%%%%%%%%%%%%%%%%%%%%%%%%%%%%%%%%%%%%%%%%%%%%%%%%%%%%%%%%%%%%%%%%%%%%%%%%%%%%%%



\textbf{Exercise 9.21.}
\emph{...} \\

\emph{Proof.}
\begin{enumerate}
\item[(1)]
\item[(2)]

\end{enumerate}
$\Box$ \\\\



%%%%%%%%%%%%%%%%%%%%%%%%%%%%%%%%%%%%%%%%%%%%%%%%%%%%%%%%%%%%%%%%%%%%%%%%%%%%%%%%



\textbf{Exercise 9.22.}
\emph{...} \\

\emph{Proof.}
\begin{enumerate}
\item[(1)]
\item[(2)]

\end{enumerate}
$\Box$ \\\\



%%%%%%%%%%%%%%%%%%%%%%%%%%%%%%%%%%%%%%%%%%%%%%%%%%%%%%%%%%%%%%%%%%%%%%%%%%%%%%%%



\textbf{Exercise 9.23.}
\emph{...} \\

\emph{Proof.}
\begin{enumerate}
\item[(1)]
\item[(2)]

\end{enumerate}
$\Box$ \\\\



%%%%%%%%%%%%%%%%%%%%%%%%%%%%%%%%%%%%%%%%%%%%%%%%%%%%%%%%%%%%%%%%%%%%%%%%%%%%%%%%



\textbf{Exercise 9.24.}
\emph{...} \\

\emph{Proof.}
\begin{enumerate}
\item[(1)]
\item[(2)]

\end{enumerate}
$\Box$ \\\\



%%%%%%%%%%%%%%%%%%%%%%%%%%%%%%%%%%%%%%%%%%%%%%%%%%%%%%%%%%%%%%%%%%%%%%%%%%%%%%%%



\textbf{Exercise 9.25.}
\emph{...} \\

\emph{Proof.}
\begin{enumerate}
\item[(1)]
\item[(2)]

\end{enumerate}
$\Box$ \\\\



%%%%%%%%%%%%%%%%%%%%%%%%%%%%%%%%%%%%%%%%%%%%%%%%%%%%%%%%%%%%%%%%%%%%%%%%%%%%%%%%



\textbf{Exercise 9.26.}
\emph{...} \\

\emph{Proof.}
\begin{enumerate}
\item[(1)]
\item[(2)]

\end{enumerate}
$\Box$ \\\\



%%%%%%%%%%%%%%%%%%%%%%%%%%%%%%%%%%%%%%%%%%%%%%%%%%%%%%%%%%%%%%%%%%%%%%%%%%%%%%%%



\textbf{Exercise 9.27.}
\emph{...} \\

\emph{Proof.}
\begin{enumerate}
\item[(1)]
\item[(2)]

\end{enumerate}
$\Box$ \\\\



%%%%%%%%%%%%%%%%%%%%%%%%%%%%%%%%%%%%%%%%%%%%%%%%%%%%%%%%%%%%%%%%%%%%%%%%%%%%%%%%



\textbf{Exercise 9.28.}
\emph{...} \\

\emph{Proof.}
\begin{enumerate}
\item[(1)]
\item[(2)]

\end{enumerate}
$\Box$ \\\\



%%%%%%%%%%%%%%%%%%%%%%%%%%%%%%%%%%%%%%%%%%%%%%%%%%%%%%%%%%%%%%%%%%%%%%%%%%%%%%%%



\textbf{Exercise 9.29.}
\emph{...} \\

\emph{Proof.}
\begin{enumerate}
\item[(1)]
\item[(2)]

\end{enumerate}
$\Box$ \\\\



%%%%%%%%%%%%%%%%%%%%%%%%%%%%%%%%%%%%%%%%%%%%%%%%%%%%%%%%%%%%%%%%%%%%%%%%%%%%%%%%



\textbf{Exercise 9.30.}
\emph{...} \\

\emph{Proof.}
\begin{enumerate}
\item[(1)]
\item[(2)]

\end{enumerate}
$\Box$ \\\\



%%%%%%%%%%%%%%%%%%%%%%%%%%%%%%%%%%%%%%%%%%%%%%%%%%%%%%%%%%%%%%%%%%%%%%%%%%%%%%%%



\textbf{Exercise 9.31.}
\emph{...} \\

\emph{Proof.}
\begin{enumerate}
\item[(1)]
\item[(2)]

\end{enumerate}
$\Box$ \\\\



%%%%%%%%%%%%%%%%%%%%%%%%%%%%%%%%%%%%%%%%%%%%%%%%%%%%%%%%%%%%%%%%%%%%%%%%%%%%%%%%
%%%%%%%%%%%%%%%%%%%%%%%%%%%%%%%%%%%%%%%%%%%%%%%%%%%%%%%%%%%%%%%%%%%%%%%%%%%%%%%%



\end{document}