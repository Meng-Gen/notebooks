\documentclass{article}
\usepackage{amsfonts}
\usepackage{amsmath}
\usepackage{amssymb}
\usepackage{hyperref}
\usepackage[none]{hyphenat}
\usepackage{mathrsfs}
\usepackage{physics}
\parindent=0pt

\def\upint{\mathchoice%
    {\mkern13mu\overline{\vphantom{\intop}\mkern7mu}\mkern-20mu}%
    {\mkern7mu\overline{\vphantom{\intop}\mkern7mu}\mkern-14mu}%
    {\mkern7mu\overline{\vphantom{\intop}\mkern7mu}\mkern-14mu}%
    {\mkern7mu\overline{\vphantom{\intop}\mkern7mu}\mkern-14mu}%
  \int}
\def\lowint{\mkern3mu\underline{\vphantom{\intop}\mkern7mu}\mkern-10mu\int}

\begin{document}



\textbf{\Large Chapter 5: Differentiation} \\\\



\emph{Author: Meng-Gen Tsai} \\
\emph{Email: plover@gmail.com} \\\\



%%%%%%%%%%%%%%%%%%%%%%%%%%%%%%%%%%%%%%%%%%%%%%%%%%%%%%%%%%%%%%%%%%%%%%%%%%%%%%%%
%%%%%%%%%%%%%%%%%%%%%%%%%%%%%%%%%%%%%%%%%%%%%%%%%%%%%%%%%%%%%%%%%%%%%%%%%%%%%%%%



\textbf{Exercise 5.1.}
\emph{Let $f$ be defined for all real $x$, and suppose that
$$|f(x) - f(y)| \leq (x - y)^2$$
for all real $x$ and $y$. Prove that $f$ is a constant.} \\

\emph{Proof.}
\begin{enumerate}
\item[(1)]
Write
\[
  \left| \frac{f(x) - f(y)}{x - y} \right| \leq |x - y|
\]
if $x \neq y$.

\item[(2)]
Given any $y \in \mathbb{R}$,
\[
  \abs{ \frac{f(x) - f(y)}{x - y} } \to 0 \:\: \text{ as } \:\: x \to y,
\]
or $|f'(y)| = 0$.

\item[(3)]
Or using $\varepsilon$-$\delta$ argument. Fix $y \in \mathbb{R}$.
Given any $\varepsilon > 0$, there exists $\delta = \varepsilon > 0$ such that
$$\left| \frac{f(x) - f(y)}{x - y} - 0 \right| \leq |x - y| < \delta = \varepsilon$$
whenever $|x - y| < \delta$. That is, $|f'(y)| = 0$.

\item[(4)]
So $f'(y) = 0$ for any $y \in \mathbb{R}$.
By Theorem 5.11 (b), $f$ is a constant.
\end{enumerate}
$\Box$ \\\\



%%%%%%%%%%%%%%%%%%%%%%%%%%%%%%%%%%%%%%%%%%%%%%%%%%%%%%%%%%%%%%%%%%%%%%%%%%%%%%%%



\textbf{Exercise 5.2.}
\emph{Suppose $f'(x)>0$ in $(a,b)$.
Prove that $f$ is strictly increasing in $(a,b)$, and let $g$ be its inverse function.
Prove that $g$ is differentiable, and that
\[
  g'(f(x)) = \frac{1}{f'(x)} \:\:\:\:\:\:\:\: (a<x<b).
\]}

\emph{Proof.}
Let $E = (a,b)$.
\begin{enumerate}
\item[(1)]
Theorem 5.10 implies that
for any $a < p < q < b$ there exists $\xi \in (p,q)$ such that
\[
  f(p)-f(q) = (p-q)f'(\xi).
\]
Since $\xi \in (p,q) \subseteq E$, by assumption $f'(\xi) > 0$.
Hence $f(p)-f(q) = (p-q)f'(\xi) < 0$ (here $p-q < 0$), or
\[
  f(p) < f(q)
\]
if $p < q$.
Therefore, $f$ is strictly increasing in $(a,b)$.

\item[(2)]
\emph{Show that $f$ is one-to-one in $E$ if $f$ is strictly increasing in $E$.}
If $f(p) = f(q)$, then it cannot be $p > q$ or $p < q$ ((1)).
So that $p = q$, or $f$ is injective.

\item[(3)]
\emph{Show that $g$ is well-defined.}
Theorem 5.2 and Theorem 4.17.

\item[(4)]
\emph{Show that $ g'(f(x)) = \frac{1}{f'(x)}$.}
Given $y \in f(E)$, say $y = f(x)$ for some $x \in E$.
Given any $s \in f(E)$ with $s \neq y$.
Here $s = f(t)$ for some $t \in E$ and $t \neq x$.
\begin{align*}
  \lim_{s \to y} \frac{g(s) - g(y)}{s - y}
  &= \lim_{f(t) \to f(x)} \frac{g(f(t)) - g(f(x))}{f(t) - f(x)} \\
  &= \lim_{t \to x} \frac{t - x}{f(t) - f(x)} \\
  &= \lim_{t \to x} \frac{1}{\frac{f(t) - f(x)}{t - x}} \\
  &=\frac{1}{f'(x)}.
    &(f' > 0)
\end{align*}
Here $s \to y$ if and only if $t \to x$ since both $f$ and $g$
are continuous and one-to-one.
Hence $g$ is differentiable and $g'(f(x)) = \frac{1}{f'(x)}$.
\end{enumerate}
$\Box$ \\\\



%%%%%%%%%%%%%%%%%%%%%%%%%%%%%%%%%%%%%%%%%%%%%%%%%%%%%%%%%%%%%%%%%%%%%%%%%%%%%%%%



\textbf{Exercise 5.3.}
\emph{Suppose $g$ is a real function on $\mathbb{R}^1$,
with bounded derivative (say $|g'|\leq M$).
Fix $\varepsilon > 0$, and define $f(x)=x+\varepsilon g(x)$.
Prove that $f$ is one-to-one if $\varepsilon$ is small enough.
(A set of admissible values of $\varepsilon$ can be determined which depends only on $M$.)} \\

\emph{Proof.}
\begin{enumerate}
\item[(1)]
Note that
$f'(x) = 1 + \varepsilon g'(x)$ (Theorem 5.3).
Since $|g'|\leq M$,
\[
  1 - \varepsilon M \leq f'(x) \leq 1 + \varepsilon M.
\]

\item[(2)]
Pick
\[
  \varepsilon = \frac{1}{M+1} > 0.
\]
Thus,
\[
  f'(x) \geq \frac{1}{M+1} > 0.
\]
By Exercise 5.2, $f(x)$ is strictly increasing in $\mathbb{R}$ or one-to-one in $\mathbb{R}$.
\end{enumerate}
$\Box$ \\\\



%%%%%%%%%%%%%%%%%%%%%%%%%%%%%%%%%%%%%%%%%%%%%%%%%%%%%%%%%%%%%%%%%%%%%%%%%%%%%%%%



\textbf{Exercise 5.4.}
\emph{If
$$C_0 + \frac{C_1}{2} + \cdots + \frac{C_{n- 1}}{n} + \frac{C_n}{n + 1} = 0,$$
where $C_0, ..., C_n$ are real constants, prove that the equation
$$C_0 + C_1 x + \cdots + C_{n - 1} x^{n - 1} + C_n x^n = 0$$
has at least one real root between $0$ and $1$.} \\

\emph{Proof.}
Let
$$g(x) = C_0 x + \frac{C_1}{2} x^2 + \cdots + \frac{C_{n- 1}}{n} x^n + \frac{C_n}{n + 1} x^{n + 1}
\in \mathbb{R}[x].$$
Then $g(0) = g(1) = 0$, and
$g'(x) = C_0 + C_1 x + \cdots + C_{n - 1} x^{n - 1} + C_n x^n$.
By the mean value theorem (Theorem 5.10), there exists a point $\xi \in (0, 1)$ at which
$$g(1) - g(0) = g'(\xi)(1 - 0),$$
or $g'(\xi) = 0.$ That is, there exists a real root $x = \xi$ between $0$ and $1$
at which $C_0 + C_1 x + \cdots + C_{n - 1} x^{n - 1} + C_n x^n = 0$.
$\Box$ \\\\



%%%%%%%%%%%%%%%%%%%%%%%%%%%%%%%%%%%%%%%%%%%%%%%%%%%%%%%%%%%%%%%%%%%%%%%%%%%%%%%%



\textbf{Exercise 5.5.}
\emph{Suppose $f$ is defined and differentiable for every $x > 0$,
and $f'(x) \to 0$ as $x \to +\infty$.
Put $g(x) = f(x+1) - f(x)$.
Prove that $g(x) \to 0$ as $x \to +\infty$.} \\

\emph{Proof.}
Given any $x > 0$.
Since f is differentiable for every $x > 0$,
f is differentiable on $[x,x+1]$.
By Theorem 5.2 and Theorem 5.10 (the mean value theorem),
there is a point $\xi \in (x,x+1)$ at which
\[
  f(x+1) - f(x) = [(x+1) - x ]f'(\xi)
\]
or
\[
  g(x) = f'(\xi).
\]
As $x \to +\infty$, $\xi \to +\infty$.
Hence
\[
  \lim_{x \to +\infty} g(x)
  = \lim_{\xi \to +\infty} f'(\xi) = 0.
\]
$\Box$ \\\\



%%%%%%%%%%%%%%%%%%%%%%%%%%%%%%%%%%%%%%%%%%%%%%%%%%%%%%%%%%%%%%%%%%%%%%%%%%%%%%%%



\textbf{Exercise 5.6.}
\emph{Suppose}
\begin{enumerate}
  \item[(a)]
  \emph{$f$ is continuous for $x \geq 0$},

  \item[(b)]
  \emph{$f'(x)$ exists for $x > 0$},

  \item[(c)]
  \emph{$f(0) = 0$},

  \item[(d)]
  \emph{$f'$ is monotonically increasing.}
\end{enumerate}
\emph{Put
\[
  g(x) = \frac{f(x)}{x} \:\:\:\:\:\:\:\: (x>0)
\]
and prove that $g$ is monotonically increasing.} \\

\emph{Proof.}
\begin{enumerate}
  \item[(1)]
  It suffices to show that $g'(x) \geq 0$ for $x>0$ (Theorem 5.11(a)),
  that is, to show that
  \[
    g'(x) = \frac{x f'(x) - f(x)}{x^2} \geq 0 \:\:\:\:\:\:\:\: (x>0),
  \]
  or
  \[
    x f'(x) - f(x) \geq 0 \:\:\:\:\:\:\:\: (x>0)
  \]
  since $x^2 > 0$ for all nonzero $x$.

  \item[(2)]
  Given $x > 0$.
  By (a)(b), we apply the mean value theorem (Theorem 5.10) on $f$ to get
  \[
    f(x) - f(0) = (x - 0)f'(\xi)
  \]
  for some $\xi \in (0,x)$.
  By (c),
  \[
    f(x) = x f'(\xi).
  \]
  By (d),
  \[
    f(x) = x f'(\xi) \leq x f'(x).
  \]
  Hence $x f'(x) - f(x) \geq 0$,
  or $g$ is monotonically increasing.
\end{enumerate}
$\Box$ \\

\emph{Note.}
$g$ is increasing strictly if $f$ is increasing strictly. \\\\



%%%%%%%%%%%%%%%%%%%%%%%%%%%%%%%%%%%%%%%%%%%%%%%%%%%%%%%%%%%%%%%%%%%%%%%%%%%%%%%%


\textbf{Exercise 5.14.}
\emph{Let $f$ be a differentiable real function defined in $(a,b)$.
Prove that $f$ is convex if and only if $f'$ is monotonically increasing.
Assume next $f''(x)$ exists for every $x \in (a,b)$,
and prove that $f$ is convex if and only if $f''(x) \geq 0$ for all $x \in (a,b)$.} \\

\emph{Proof.}
\begin{enumerate}
\item[(1)]
  \emph{Show that $f'$ is monotonically increasing if $f$ is convex.}
  \begin{enumerate}
  \item[(a)]
  Since $f$ is convex, by definition (Exercise 4.23)
  \[
    f(\lambda x + (1-\lambda) y) \leq \lambda f(x) + (1-\lambda) f(y)
  \]
  whenever $a < x < b$, $a < y < b$, $0 < \lambda < 1$.

  \item[(b)]
  As $x \neq y$, we have
  \begin{align*}
    f(y) - f(x)
    &\geq \frac{f(x + \lambda(y-x)) - f(x)}{\lambda} \\
    &= \frac{f(x + \lambda(y-x)) - f(x)}{\lambda(y-x)} \cdot (y-x)
  \end{align*}
  and let $\lambda \to 0$ to get
  \[
    f(y) - f(x) \geq f'(x)(y - x)
  \]
  (since $f'(x)$ exists).
  Similarly, we have
  \[
    f(x) - f(y) \geq f'(y)(x - y).
  \]

  \item[(c)]
  Given any $y > x$, we have
  \[
    f'(y)(y - x) \geq f(y) - f(x) \geq f'(x)(y - x).
  \]
  Hence $f'(y) \geq f'(x)$ whenever $y > x$,
  or $f'$ is monotonically increasing.
  \end{enumerate}

\item[(2)]
  \emph{Show that $f$ is convex if $f'$ is monotonically increasing.}
  Given any $y > x$ and any $0 < \lambda < 1$.
  \begin{enumerate}
  \item[(a)]
  By Theorem 5.10 (the mean value theorem), there is a point $x < \xi < y$ such that
  \[
    f(y) - f(x) = f'(\xi)(y - x).
  \]
  Since $f'$ is monotonically increasing,
  \[
    f'(y)(y - x) \geq f(y) - f(x) \geq f'(x)(y - x).
  \]

  \item[(b)]
  Write $z = \lambda x + (1-\lambda)y$.
  Hence
  \begin{align*}
    f(y)-f(z) &\geq f'(z)(y-z), \\
    f(z)-f(x) &\leq f'(z)(z-x),
  \end{align*}
  or
  \begin{align*}
    f(y) &\geq f(z) + f'(z)(y-z), \\
    f(x) &\geq f(z) + f'(z)(x-z),
  \end{align*}
  or
  \begin{align*}
    \lambda f(x) + (1-\lambda)f(y)
    \geq&
    \lambda [f(z) + f'(z)(x-z)] \\
      &+ (1-\lambda)[f(z) + f'(z)(y-z)] \\
    =& f(z) \\
    =& f(\lambda x + (1-\lambda)y).
  \end{align*}
  Hence $f$ is convex.
  \end{enumerate}

\item[(3)]
  \emph{Show that $f''(x) \geq 0$ if $f$ is convex and $f''$ exists.}
  By (1), $f'$ is monotonically increasing since $f$ is convex.
  Given any $x \neq y$, we have
  \[
    \frac{f'(y)-f'(x)}{y - x} \geq 0.
  \]
  Let $y \to x$, we have $f''(x) \geq 0$ if $f''$ exists.

\item[(4)]
  \emph{Show that $f$ is convex if $f''$ exists and $f''(x) \geq 0$.}
  By Theorem 5.11(a), $f'$ is monotonically increasing.
  By (2), $f$ is convex.
\end{enumerate}
$\Box$ \\\\



%%%%%%%%%%%%%%%%%%%%%%%%%%%%%%%%%%%%%%%%%%%%%%%%%%%%%%%%%%%%%%%%%%%%%%%%%%%%%%%%
%%%%%%%%%%%%%%%%%%%%%%%%%%%%%%%%%%%%%%%%%%%%%%%%%%%%%%%%%%%%%%%%%%%%%%%%%%%%%%%%



\end{document}