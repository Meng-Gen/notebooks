\documentclass{article}
\usepackage{amsfonts}
\usepackage{amsmath}
\usepackage{amssymb}
\usepackage{hyperref}
\usepackage[none]{hyphenat}
\usepackage{mathrsfs}
\parindent=0pt

\def\upint{\mathchoice%
    {\mkern13mu\overline{\vphantom{\intop}\mkern7mu}\mkern-20mu}%
    {\mkern7mu\overline{\vphantom{\intop}\mkern7mu}\mkern-14mu}%
    {\mkern7mu\overline{\vphantom{\intop}\mkern7mu}\mkern-14mu}%
    {\mkern7mu\overline{\vphantom{\intop}\mkern7mu}\mkern-14mu}%
  \int}
\def\lowint{\mkern3mu\underline{\vphantom{\intop}\mkern7mu}\mkern-10mu\int}

\begin{document}

\textbf{\Large Chapter 4: Continuity} \\\\



\emph{Author: Meng-Gen Tsai} \\
\emph{Email: plover@gmail.com} \\\\



% http://my.fit.edu/~awelters/Teaching/2017/Fall/HW4SolnsFall2017MTH4101&5101.pdf
% https://www.math.purdue.edu/~dgarofal/hmw9(MA504).pdf
% https://carma.newcastle.edu.au/resources/jon/Preprints/Books/CUP/CUPold/np-convex.pdf



%%%%%%%%%%%%%%%%%%%%%%%%%%%%%%%%%%%%%%%%%%%%%%%%%%%%%%%%%%%%%%%%%%%%%%%%%%%%%%%%



\textbf{Exercise 4.1.}
\emph{Suppose $f$ is a real function define on $\mathbb{R}^1$ which satisfies
$$\lim_{h \to 0}[f(x+h)-f(x-h)] = 0$$
for every $x \in \mathbb{R}^1$.
Does this imply that $f$ is continuous?} \\

\emph{Proof.}
$\lim_{h \to 0}[f(x+h)-f(x-h)] = 0$ holds if $f$ is continuous.
But the converse of this statement and is not true.
For example, define $f: \mathbb{R}^1 \to \mathbb{R}^1$ by
\begin{equation*}
  f(x) =
    \begin{cases}
      1 & (x = 0), \\
      0 & (x \neq 0).
    \end{cases}
\end{equation*}
$f$ is not continuous at $x = 0$ but
$$\lim_{h \to 0}[f(x+h)-f(x-h)] = 0$$ for any $x \in \mathbb{R}^1$.
(The identity holds for $x \neq 0$ since $f$ is continuous on $\mathbb{R}^1 - \{0\}$.
Besides, $\lim_{h \to 0}[f(0+h)-f(0-h)] = \lim_{h \to 0}[0 - 0] = 0$.)
$\Box$ \\\\



%%%%%%%%%%%%%%%%%%%%%%%%%%%%%%%%%%%%%%%%%%%%%%%%%%%%%%%%%%%%%%%%%%%%%%%%%%%%%%%%



\textbf{Exercise 4.2.}
\emph{If $f$ is a continuous mapping of a metric space $X$
into a metric space $Y$,
prove that $f(\overline{E}) \subseteq \overline{f(E)}$
for every set $E \subseteq X$.
($\overline{E}$ denotes the closure of $E$.)
Show, by an example, that $f(\overline{E})$ can be a proper subset of $\overline{f(E)}$.} \\

\emph{Proof.}
\begin{enumerate}
\item[(1)]
Since $f$ is continuous and $\overline{f(E)}$ is closed,
$f^{-1}(\overline{f(E)})$ is closed.
Hence,
\begin{align*}
f^{-1}(\overline{f(E)})
&\supseteq f^{-1}(f(E))
  & \text{(Monotonicity of $f^{-1}$)} \\
&\supseteq E,
  & \text{(Note in Theorem 4.14)} \\
\overline{E}
&\subseteq f^{-1}(\overline{f(E)}),
  & \text{(Monotonicity of closure)} \\
f(\overline{E})
&\subseteq f(f^{-1}(\overline{f(E)}))
  & \text{(Monotonicity of $f$)} \\
&\subseteq \overline{f(E)}.
  & \text{(Note in Theorem 4.14)}
\end{align*}
\item[(2)]
Let $f: (0, \infty) \to \mathbb{R}$ be a continuous function
defined by $$f(x) = \frac{1}{x}.$$
Consider $E = \mathbb{Z}^+ \subseteq (0, \infty)$.
Then $f(E) = \left\{ \frac{1}{n} : n \in \mathbb{Z}^+ \right\}$,
and thus
\begin{align*}
  f(\overline{E})
  &= \left\{ \frac{1}{n} : n \in \mathbb{Z}^+ \right\}. \\
  \overline{f(E)}
  &= \left\{ \frac{1}{n} : n \in \mathbb{Z}^+ \right\} \bigcup \{0\}.
\end{align*}
\end{enumerate}
$\Box$ \\



\textbf{Supplement (Inverse image).}
\begin{enumerate}
\item[(1)]
\emph{$E \subseteq f^{-1}[f(E)]$ for $E \subseteq X$.}
\begin{align*}
  \forall \: x \in E
  &\Longrightarrow
  f(x) \in f(E)
    & \\
  &\Longleftrightarrow
  x \in f^{-1}[f(E)].
    &\text{(Definition of the inverse image)}
\end{align*}
$\Box$ \\
\item[(2)]
\emph{$f[f^{-1}(E)] \subseteq E$ for $E \subseteq Y$.}
\begin{align*}
  \forall \: y \in f[f^{-1}(E)]
  &\Longleftrightarrow
  \exists \: x \in f^{-1}(E) \text{ such that } y = f(x) \\
  &\Longleftrightarrow
  \exists \: x, f(x) \in E \text{ such that } y = f(x) \\
  &\Longrightarrow
  \exists \: x, y = f(x) \in E.
\end{align*}
$\Box$ \\
\end{enumerate}



\textbf{Supplement (Continuity).}
\emph{Let $f$ be a map from a topological space on $X$
to a topological space on $Y$.
Then, the following statements are equivalent:}
\begin{enumerate}
\item[(1)]
\emph{$f$ is continuous:
For each $x \in X$ and every neighborhood $V$ of $f(x)$,
there is a neighborhood $U$ of $x$ such that $f(U) \subseteq V$.}
\item[(2)]
\emph{For every open set $O$ in $Y$, the inverse image $f^{-1}(O)$
is open in $X$.}
\item[(3)]
\emph{For every closed set $C$ in $Y$, the inverse image $f^{-1}(C)$
is closed in $X$.}
\item[(4)]
\emph{$f(A)^{\circ} \subseteq f(A^{\circ})$ for every subset $A$ of $X$.}
\item[(5)]
\emph{$f^{-1}(B^{\circ}) \subseteq (f^{-1}(B))^{\circ}$ for every subset $B$ of $Y$.}
\item[(6)]
\emph{$f(\overline{A}) \subseteq \overline{f(A)}$ for every subset $A$ of $X$.}
\item[(7)]
\emph{$\overline{f^{-1}(B)} \subseteq f^{-1}(\overline{B})$ for every subset $B$ of $Y$.} \\\\
\end{enumerate}



%%%%%%%%%%%%%%%%%%%%%%%%%%%%%%%%%%%%%%%%%%%%%%%%%%%%%%%%%%%%%%%%%%%%%%%%%%%%%%%%



\textbf{Exercise 4.3.}
\emph{Let $f$ be a continuous real function on a metric space $X$.
Let $Z(f)$ (the zero set of $f$) be the set of all $p \in X$ at which $f(p) = 0$.
Prove that $Z(f)$ is closed.} \\

\emph{Proof (Corollary to Theorem 4.8).}
Since $f$ is continuous, $f^{-1}(\{0\}) = Z(f)$ is closed in $X$
for a closed subset $\{0\}$ in $\mathbb{R}^1$.
$\Box$ \\

Denote the complement of any set $E$ by $\widetilde{E}$. \\

\emph{Proof (Theorem 4.8).}
Consider the complement of $Z(f)$ in $X$,
\begin{align*}
\widetilde{Z(f)}
&= \{ x \in X : f(x) \neq 0 \} \\
&= f^{-1}((-\infty, 0) \cup (0, \infty)).
\end{align*}

Since $f$ is continuous, $f^{-1}((-\infty, 0) \cup (0, \infty)) = \widetilde{Z(f)}$
is open in $X$ for a open subset $(-\infty, 0) \cup (0, \infty)$ in $\mathbb{R}^1$.
$\Box$ \\

\emph{Proof (Definition 2.18(d)).}
Given any limit point $p$ of $Z(f)$.
\emph{Show that $f(p) = 0$ or $p \in Z(f)$.}
Since $f$ is continuous, given any $\epsilon > 0$ there exists a $\delta > 0$
such that $|f(x) - f(p)| < \epsilon$ for all $x \in X$ for which $d_X(x, p) < \delta$.
Since $p$ is a limit point of $Z(f)$, for such $\delta > 0$ we have a point $q \neq p$
such that $q \in Z(f)$, or $f(q) = 0$. So $|f(p)| < \epsilon$ for any $\epsilon > 0$.
$f(p) = 0$.
$\Box$ \\

\emph{Proof (Definition 2.18(f)).}
Consider the complement of $Z(f)$ in $X$,
$$\widetilde{Z(f)} = \{ x \in X : f(x) \neq 0 \} = \{f > 0\} \cup \{f < 0\}$$
where $\{f > 0\} = \{ x \in X : f(x) > 0 \}$ and $\{f < 0\} = \{ x \in X : f(x) < 0 \}$.
It suffices to show $\{f > 0\}$ is open. ($\{f < 0\}$ is similar.)
Given any point $p$ of $\{f > 0\}$ or $f(p) > 0$.
\emph{Want to show $p$ is an interior point of $\{f > 0\}$.}
Since $f$ is continuous, given any $\epsilon = \frac{f(p)}{2} > 0$
there exists a $\delta > 0$
such that $|f(x) - f(p)| < \frac{f(p)}{2}$ for all $x \in X$ for which $d_X(x, p) < \delta$.
For such $x$ with $d_X(x, p) < \delta$ we have
$$\frac{1}{2}f(p) < f(x) < \frac{3}{2}f(p).$$
That is, $N = \{ x : d_X(x, p) < \delta \}$ is a neighborhood $p$ such that
$N \subseteq \{f > 0\}$.
$\Box$ \\\\



%%%%%%%%%%%%%%%%%%%%%%%%%%%%%%%%%%%%%%%%%%%%%%%%%%%%%%%%%%%%%%%%%%%%%%%%%%%%%%%%



\textbf{Exercise 4.4.}
\emph{Let $f$ and $g$ be continuous mappings of a metric space $X$ into a metric space $Y$,
and let $E$ be a dense subset of $X$.
Prove that $f(E)$ is dense in $f(X)$.
If $g(p) = f(p)$ for all $p \in E$, prove that $g(p) = f(p)$ for all $p \in X$.
(In other words, a continuous mapping is determined by
its values on a dense subset of its domain.)} \\



%%%%%%%%%%%%%%%%%%%%%%%%%%%%%%%%%%%%%%%%%%%%%%%%%%%%%%%%%%%%%%%%%%%%%%%%%%%%%%%%



\textbf{Exercise 4.5.}
\emph{If $f$ is a real continuous function defined on a closed set $E \subset \mathbb{R}^1$,
prove that there exist continuous real function $g$ on $\mathbb{R}^1$ such that
$g(x) = f(x)$ for all $x \in E$.
(Such functions $g$ are called \textbf{continuous extensions} of $f$ from $E$ to $\mathbb{R}^1$.)
Show that the result becomes false if the word ``closed'' is omitted.
Extend the result to vector valued functions.
(Hint: Let the graph of $g$ be a straight line on each of the segments
which constitute the complement of $E$
(compare Exercise 2.29).
The result remains true if $\mathbb{R}^1$ is replaced by any metric space,
but the proof is not so simple.) } \\

\textbf{Supplement (Tietze's Extension Theorem).}
\emph{If $X$ is a normal topological space and
$f: A \to \mathbb{R}$ is a continuous map
from a closed subset $A$ of $X$ into the real numbers carrying the standard topology,
then there exists a continuous map
$g: X \to \mathbb{R}$ with $g(a) = f(a)$ for all $a \in A$.} \\\\



%%%%%%%%%%%%%%%%%%%%%%%%%%%%%%%%%%%%%%%%%%%%%%%%%%%%%%%%%%%%%%%%%%%%%%%%%%%%%%%%



\textbf{Exercise 4.23.}
\emph{A real-valued function $f$ defined in $(a,b)$
is said to be \textbf{convex} if
$$f(\lambda x + (1 - \lambda) y) \leq \lambda f(x) + (1 - \lambda) f(y)$$
whenever $a < x < b$, $a < y < b$, $0 < \lambda < 1$.
Prove that every convex function is continuous.
Prove that every increasing convex function of a convex function is convex.
(For example, if $f$ is convex, so is $e^f$.)} \\

\emph{If $f$ is convex in $(a,b)$ and if $a < s < t < u < b$,
show that
$$\frac{f(t)-f(s)}{t-s}
\leq \frac{f(u)-f(s)}{u-s}
\leq \frac{f(u)-f(t)}{u-t}.$$} \\

\emph{Proof.}
\begin{enumerate}
\item[(1)]
\emph{Show that
$\frac{f(t)-f(s)}{t-s}
\leq \frac{f(u)-f(s)}{u-s}
\leq \frac{f(u)-f(t)}{u-t}$.}
Since
\begin{align*}
  t
  &= \frac{t-s}{u-s} u + \left( 1-\frac{t-s}{u-s} \right) s \\
  &= \left( 1-\frac{u-t}{u-s} \right) u + \frac{u-t}{u-s} s
\end{align*}
and $0 < \frac{t-s}{u-s}, \frac{u-t}{u-s} < 1$,
by the convexity of $f$ we have
\begin{align*}
  f(t)
  &\leq \frac{t-s}{u-s} f(u) + \left( 1-\frac{t-s}{u-s} \right) f(s), \\
  f(t)
  &\leq \left( 1-\frac{u-t}{u-s} \right) f(u) + \frac{u-t}{u-s} f(s).
\end{align*}
It is equivalent to
$$\frac{f(t)-f(s)}{t-s}
\leq \frac{f(u)-f(s)}{u-s}
\leq \frac{f(u)-f(t)}{u-t}.$$
$\Box$\\
\item[(2)]
\emph{If $x, y, x', y'$ are points of $(a,b)$
with $x \leq x' < y'$ and $x < y \leq y'$,
then the chord over $(x',y')$ has larger slope than the chord over $(x,y)$; that is,
$$\frac{f(y)-f(x)}{y-x} \leq \frac{f(y')-f(x')}{y'-x'}.$$}
It is a corollary to (1).
\item[(3)]
\emph{Show that $f$ is continuous.}
Let $[c,d] \subseteq (a,b)$.
Then by (2),
$$\frac{f(c)-f(a)}{c-a}
\leq \frac{f(y) - f(x)}{y - x}
\leq \frac{f(b)-f(d)}{b-d}$$
for $x, y$ in $[c,d]$.
Thus $|f(y) - f(x)| \leq M|y - x|$ in $[c,d]$
(where $M = \max\left( |\frac{f(c)-f(a)}{c-a}|, |\frac{f(b)-f(d)}{b-d}| \right)$),
and so $f$ is absolutely continuous
on each closed subinterval of $(a,b)$.
Especially, $f$ is continuous.
\item[(4)]
\emph{Let $f$ be a convex function,
$g$ be an increasing convex function,
and $h = g \circ f$.
Show that $h$ is convex.}
\begin{align*}
f(\lambda x + (1-\lambda) y)
&\leq \lambda f(x) + (1-\lambda) f(y),
  &\text{(Convexity of $f$)} \\
g(f(\lambda x + (1-\lambda) y))
&\leq g(\lambda f(x) + (1-\lambda) f(y))
  &\text{(Increasing of $g$)} \\
&\leq \lambda g(f(x)) + (1-\lambda) g(f(y)),
  &\text{(Convexity of $g$)} \\
h(\lambda x + (1-\lambda) y)
&\leq \lambda h(x) + (1-\lambda) h(y).
\end{align*}
\end{enumerate}
$\Box$ \\\\



%%%%%%%%%%%%%%%%%%%%%%%%%%%%%%%%%%%%%%%%%%%%%%%%%%%%%%%%%%%%%%%%%%%%%%%%%%%%%%%%



\textbf{Exercise 4.24.}
\emph{Assume that $f$ is a continuous real function defined in $(a,b)$
such that
$$f\left( \frac{x+y}{2} \right)
\leq \frac{f(x)+f(y)}{2}$$
for all $x, y \in (a,b)$.
Prove that $f$ is convex.} \\

\emph{Proof.}
\begin{enumerate}
\item[(1)]
\emph{Show that
$$f\left( \frac{x_1 + \cdots + x_n}{n} \right)
\leq \frac{f(x_1) + \cdots + f(x_n)}{n}$$
whenever $a < x_i < b$ $(1 \leq i \leq n)$.}
Apply Cauchy induction and use the same argument in proving the AM-GM inequality.
As $n = 1, 2$, the inequality holds by assumption.
Suppose $n = 2^k$ $(k \geq 1)$ the inequality holds.
As $n = 2^{k+1}$,
\begin{align*}
&f\left( \frac{x_1 + \cdots + x_{2^{k+1}}}{2^{k+1}} \right) \\
=& f\left( \frac{1}{2} \left(\frac{x_1 + \cdots + x_{2^k}}{2^k}
  + \frac{x_{2^k+1} + \cdots + x_{2^{k+1}}}{2^k}\right) \right) \\
\leq& \frac{1}{2}
  \left(
    f\left(\frac{x_1 + \cdots + x_{2^k}}{2^k} \right)
    + f\left(\frac{x_{2^k+1} + \cdots + x_{2^{k+1}}}{2^k} \right)
  \right) \\
\leq& \frac{1}{2}
  \left(
    \frac{f(x_1) + \cdots + f(x_{2^k})}{2^k}
    + \frac{f(x_{2^k+1}) + \cdots + f(x_{2^{k+1}})}{2^k}
  \right) \\
=& \frac{f(x_1) + \cdots + f(x_{2^k}) + f(x_{2^k+1}) + \cdots + f(x_{2^{k+1}})}{2^{k+1}} \\
=& \frac{f(x_1) + \cdots + f(x_{2^{k+1}})}{2^{k+1}}.
\end{align*}
As $n$ is not a power of $2$,
then it is certainly less than some natural power of $2$, say $n < 2^m$ for some $m$.
Let
$$x_{n+1} = \cdots = x_{2^m} = \frac{x_1 + \cdots + x_n}{n} = \alpha.$$
Then by the induction hypothesis,
\begin{align*}
  f(\alpha)
  &= f\left( \frac{x_1 + \cdots + x_n + \alpha + \cdots + \alpha}{2^m} \right) \\
  &\leq \frac{f(x_1) + \cdots + f(x_n) + f(\alpha) + \cdots + f(\alpha)}{2^m} \\
  &\leq \frac{f(x_1) + \cdots + f(x_n) + (2^m - n)f(\alpha)}{2^m}, \\
  2^m f(\alpha)
  &\leq f(x_1) + \cdots + f(x_n) + (2^m - n)f(\alpha), \\
  n f(\alpha)
  &\leq f(x_1) + \cdots + f(x_n),
\end{align*}
or $f\left( \frac{1}{n} (x_1 + \cdots + x_n) \right)
\leq \frac{1}{n}(f(x_1) + \cdots f(x_n))$.
\item[(2)]
Hence,
$$f(\lambda x + (1 - \lambda) y) \leq \lambda f(x) + (1 - \lambda) f(y)$$
for any rational $\lambda$ in $(0,1)$.
(Given any positive integers $p < q$,
put $n = q$,
$x_1 = \cdots = x_p = x$
and $x_{p+1} = \cdots = x_n = y$ in (1).)
\item[(3)]
Given any real $\lambda \in (0,1)$,
there is a sequence of rational numbers $\{r_n\} \subseteq (0, 1)$
such that $r_n \to \lambda$.
By (2),
$$f(r_n x + (1 - r_n) y) \leq r_n f(x) + (1 - r_n) f(y)$$
for any rational $r_n$ in $(0,1)$.
Taking limit on the both sides and using the continuity of $f$,
we have
$$f(\lambda x + (1 - \lambda) y) \leq \lambda f(x) + (1 - \lambda) f(y).$$
\end{enumerate}
$\Box$ \\

\emph{Proof (Reductio ad absurdum).}
If $f$ were not convex,
then there is a subinterval $[c,d] \subseteq (a,b)$
such that
$$\frac{f(d)-f(c)}{d-c} < \frac{f(x_0)-f(c)}{x_0-c}$$
for some $x_0 \in [c, d]$.
Let
$$g(x) = f(x)-f(c) - \frac{f(d)-f(c)}{d-c}(x - c)$$
for $x \in [c,d]$.
Therefore,
\begin{enumerate}
\item[(1)]
$g(x)$ is continuous and midpoint convex.
\item[(2)]
$g(c) = g(d) = 0$.
\item[(3)]
Let $M = \sup\{g(x) : x \in [c,d]\}$.
$\infty > M > 0$ due to the continuity of $g$ and the existence of $x_0$.
And let $\xi = \inf \{ x \in [c,d] : g(x) = M \}$.
By the continuity of $g$, $g(\xi) = M$.
$\xi \in (c,d)$ by (2).
\item[(4)]
Since $(c,d)$ is open, there is $h > 0$ such that $(\xi-h,\xi+h) \subseteq (c,d)$.
By the minimality of $\xi$ and $M$, $g(\xi-h) < g(\xi)$ and $g(\xi+h) \leq g(\xi)$.
\end{enumerate}
Therefore,
\begin{align*}
g(\xi-h) + g(\xi+h)
&< 2 g(\xi), \\
\frac{g(\xi-h) + g(\xi+h)}{2}
&< g(h) \\
&= g\left( \frac{(\xi-h) + (\xi+h)}{2} \right),
\end{align*}
contrary to the midpoint convexity of $g$.
$\Box$ \\

The result becomes false if ``continuity of $f$'' is omitted. \\\\



%%%%%%%%%%%%%%%%%%%%%%%%%%%%%%%%%%%%%%%%%%%%%%%%%%%%%%%%%%%%%%%%%%%%%%%%%%%%%%%%



\textbf{Exercise 4.25.}
\emph{If $A \subset \mathbb{R}^k$ and $B \subset \mathbb{R}^k$,
define $A+B$ to be the set of all sums $\mathbf{x}+\mathbf{y}$ with
$\mathbf{x} \in A$, $\mathbf{y} \in B$.}

\begin{enumerate}
\item[(a)]
\emph{If $K$ is compact and $C$ is closed in $\mathbb{R}^k$,
prove that $K+C$ is closed.
(Hint: Take $\mathbf{z} \notin K+C$,
put $F = \mathbf{z}-C$,
the set of all $\mathbf{z}-\mathbf{y}$ with $\mathbf{y} \in C$.
Then $K$ and $F$ are disjoint.
Choose $\delta$ as in Exercise 4.21.
Show that the open ball with center $\mathbf{z}$ and radius $\delta$ does not intersect $K+C$.)}
\item[(b)]
\emph{Let $\alpha$ be an irrational real number.
Let $C_1$ be the set of all integers,
let $C_2$ be the set of all $n \alpha$ with $n \in C_1$.
Show that $C_1$ and $C_2$ are closed subsets of $\mathbb{R}^1$
whose sum $C_1 + C_2$ is not closed,
by showing that $C_1 + C_2$ is a countable dense subset of $\mathbb{R}^1$.} \\
\end{enumerate}



%%%%%%%%%%%%%%%%%%%%%%%%%%%%%%%%%%%%%%%%%%%%%%%%%%%%%%%%%%%%%%%%%%%%%%%%%%%%%%%%



\textbf{Exercise 4.26.}
\emph{Suppose $X$, $Y$, $Z$ are metric spaces, and $Y$ is compact.
Let $f$ map $X$ into $Y$,
let $g$ be a continuous one-to-one mapping of $Y$ into $Z$,
and put $h(x) = g(f(x))$ for $x \in X$.} \\

\emph{Prove that $f$ is uniformly continuous if $h$ is uniformly continuous.
(Hint: $g^{-1}$ has compact domain $g(Y)$, and $f(x) = g^{-1}(h(x))$.)} \\

\emph{Prove also that $f$ is continuous if $h$ is continuous. } \\

\emph{Show (by modifying Example 4.21, or by finding a different example) that
the compactness of $Y$ cannot be omitted from the hypotheses,
even when $X$ and $Z$ are compact.} \\



\end{document}