\documentclass{article}
\usepackage{amsfonts}
\usepackage{amsmath}
\usepackage{amssymb}
\usepackage{hyperref}
\usepackage[none]{hyphenat}
\usepackage{mathrsfs}
\parindent=0pt

\def\upint{\mathchoice%
    {\mkern13mu\overline{\vphantom{\intop}\mkern7mu}\mkern-20mu}%
    {\mkern7mu\overline{\vphantom{\intop}\mkern7mu}\mkern-14mu}%
    {\mkern7mu\overline{\vphantom{\intop}\mkern7mu}\mkern-14mu}%
    {\mkern7mu\overline{\vphantom{\intop}\mkern7mu}\mkern-14mu}%
  \int}
\def\lowint{\mkern3mu\underline{\vphantom{\intop}\mkern7mu}\mkern-10mu\int}

\begin{document}

\textbf{\Large Chapter 4: Continuity} \\\\



\emph{Author: Meng-Gen Tsai} \\
\emph{Email: plover@gmail.com} \\\\



% http://my.fit.edu/~awelters/Teaching/2017/Fall/HW4SolnsFall2017MTH4101&5101.pdf



\textbf{Exercise 4.1.}
\emph{Suppose $f$ is a real function define on $\mathbb{R}^1$ which satisfies
$$\lim_{h \rightarrow 0}[f(x+h)-f(x-h)] = 0$$
for every $x \in \mathbb{R}^1$.
Does this imply that $f$ is continuous?} \\

\emph{Proof.}
$\lim_{h \rightarrow 0}[f(x+h)-f(x-h)] = 0$ holds if $f$ is continuous.
But the converse of this statement and is not true.
For example, define $f: \mathbb{R}^1 \rightarrow \mathbb{R}^1$ by
\begin{equation*}
  f(x) =
    \begin{cases}
      1 & (x = 0), \\
      0 & (x \neq 0).
    \end{cases}
\end{equation*}
$f$ is not continuous at $x = 0$ but
$$\lim_{h \rightarrow 0}[f(x+h)-f(x-h)] = 0$$
for any $x \in \mathbb{R}^1$.
(The identity holds for $x \neq 0$ since $f$ is continuous on $\mathbb{R}^1 - \{0\}$.
Besides, $\lim_{h \rightarrow 0}[f(0+h)-f(0-h)] = \lim_{h \rightarrow 0}[0 - 0] = 0$.)
$\Box$ \\\\



\textbf{Exercise 4.2.}
\emph{If $f$ is a continuous mapping of a metric space $X$
into a metric space $Y$,
prove that $f(\overline{E}) \subseteq \overline{f(E)}$
for every set $E \subseteq X$.
($\overline{E}$ denotes the closure of $E$.)
Show, by an example, that $f(\overline{E})$ can be a proper subset of $\overline{f(E)}$.} \\

\emph{Proof.}
\begin{enumerate}
\item[(1)]
Since $f$ is continuous and $\overline{f(E)}$ is closed,
$f^{-1}(\overline{f(E)})$ is closed.
Hence,
\begin{align*}
f^{-1}(\overline{f(E)})
&\supseteq f^{-1}(f(E))
  & \text{(Monotonicity of $f^{-1}$)} \\
&\supseteq E,
  & \text{(Note in Theorem 4.14)} \\
\overline{E}
&\subseteq f^{-1}(\overline{f(E)}),
  & \text{(Monotonicity of closure)} \\
f(\overline{E})
&\subseteq f(f^{-1}(\overline{f(E)}))
  & \text{(Monotonicity of $f$)} \\
&\subseteq \overline{f(E)}.
  & \text{(Note in Theorem 4.14)}
\end{align*}
\item[(2)]
Let $f: (0, \infty) \rightarrow \mathbb{R}$ be a continuous function
defined by $$f(x) = \frac{1}{x}.$$
Consider $E = \mathbb{Z}^+ \subseteq (0, \infty)$.
Then $f(E) = \left\{ \frac{1}{n} : n \in \mathbb{Z}^+ \right\}$,
and thus
\begin{align*}
  f(\overline{E})
  &= \left\{ \frac{1}{n} : n \in \mathbb{Z}^+ \right\}. \\
  \overline{f(E)}
  &= \left\{ \frac{1}{n} : n \in \mathbb{Z}^+ \right\} \bigcup \{0\}.
\end{align*}
\end{enumerate}
$\Box$ \\



\textbf{Supplement (Inverse image).}
\begin{enumerate}
\item[(1)]
\emph{$E \subseteq f^{-1}[f(E)]$ for $E \subseteq X$.}
\begin{align*}
  \forall \: x \in E
  &\Longrightarrow
  f(x) \in f(E)
    & \\
  &\Longleftrightarrow
  x \in f^{-1}[f(E)].
    &\text{(Definition of the inverse image)}
\end{align*}
$\Box$ \\
\item[(2)]
\emph{$f[f^{-1}(E)] \subseteq E$ for $E \subseteq Y$.}
\begin{align*}
  \forall \: y \in f[f^{-1}(E)]
  &\Longleftrightarrow
  \exists \: x \in f^{-1}(E) \text{ such that } y = f(x) \\
  &\Longleftrightarrow
  \exists \: x, f(x) \in E \text{ such that } y = f(x) \\
  &\Longrightarrow
  \exists \: x, y = f(x) \in E.
\end{align*}
$\Box$ \\
\end{enumerate}



\textbf{Supplement (Continuity).}
\emph{Let $f$ be a map from a topological space on $X$
to a topological space on $Y$.
Then, the following statements are equivalent:}
\begin{enumerate}
\item[(1)]
\emph{$f$ is continuous:
For each $x \in X$ and every neighborhood $V$ of $f(x)$,
there is a neighborhood $U$ of $x$ such that $f(U) \subseteq V$.}
\item[(2)]
\emph{For every open set $O$ in $Y$, the inverse image $f^{-1}(O)$
is open in $X$.}
\item[(3)]
\emph{For every closed set $C$ in $Y$, the inverse image $f^{-1}(C)$
is closed in $X$.}
\item[(4)]
\emph{$f(A)^{\circ} \subseteq f(A^{\circ})$ for every subset $A$ of $X$.}
\item[(5)]
\emph{$f^{-1}(B^{\circ}) \subseteq (f^{-1}(B))^{\circ}$ for every subset $B$ of $Y$.}
\item[(6)]
\emph{$f(\overline{A}) \subseteq \overline{f(A)}$ for every subset $A$ of $X$.}
\item[(7)]
\emph{$\overline{f^{-1}(B)} \subseteq f^{-1}(\overline{B})$ for every subset $B$ of $Y$.} \\\\
\end{enumerate}



\textbf{Exercise 4.3.}
\emph{Let $f$ be a continuous real function on a metric space $X$.
Let $Z(f)$ (the zero set of $f$) be the set of all $p \in X$ at which $f(p) = 0$.
Prove that $Z(f)$ is closed.} \\

\emph{Proof (Corollary to Theorem 4.8).}
Since $f$ is continuous, $f^{-1}(\{0\}) = Z(f)$ is closed in $X$
for a closed subset $\{0\}$ in $\mathbb{R}^1$.
$\Box$ \\

\emph{Proof (Theorem 4.8).}
Consider the complement of $Z(f)$ in $X$,
\begin{align*}
\widetilde{Z(f)}
&= \{ x \in X : f(x) \neq 0 \} \\
&= f^{-1}((-\infty, 0) \cup (0, \infty)).
\end{align*}

Since $f$ is continuous, $f^{-1}((-\infty, 0) \cup (0, \infty)) = \widetilde{Z(f)}$
is open in $X$ for a open subset $(-\infty, 0) \cup (0, \infty)$ in $\mathbb{R}^1$.
$\Box$ \\

\emph{Proof (Definition 2.18(d)).}
Given any limit point $p$ of $Z(f)$.
\emph{Show that $f(p) = 0$ or $p \in Z(f)$.}
Since $f$ is continuous, given any $\epsilon > 0$ there exists a $\delta > 0$
such that $|f(x) - f(p)| < \epsilon$ for all $x \in X$ for which $d_X(x, p) < \delta$.
Since $p$ is a limit point of $Z(f)$, for such $\delta > 0$ we have a point $q \neq p$
such that $q \in Z(f)$, or $f(q) = 0$. So $|f(p)| < \epsilon$ for any $\epsilon > 0$.
$f(p) = 0$.
$\Box$ \\

\emph{Proof (Definition 2.18(f)).}
Consider the complement of $Z(f)$ in $X$,
$$\widetilde{Z(f)} = \{ x \in X : f(x) \neq 0 \} = \{f > 0\} \cup \{f < 0\}$$
where $\{f > 0\} = \{ x \in X : f(x) > 0 \}$ and $\{f < 0\} = \{ x \in X : f(x) < 0 \}$.
It suffices to show $\{f > 0\}$ is open. ($\{f < 0\}$ is similar.)
Given any point $p$ of $\{f > 0\}$ or $f(p) > 0$.
\emph{Want to show $p$ is an interior point of $\{f > 0\}$.}
Since $f$ is continuous, given any $\epsilon = \frac{f(p)}{2} > 0$
there exists a $\delta > 0$
such that $|f(x) - f(p)| < \frac{f(p)}{2}$ for all $x \in X$ for which $d_X(x, p) < \delta$.
For such $x$ with $d_X(x, p) < \delta$ we have
$$\frac{1}{2}f(p) < f(x) < \frac{3}{2}f(p).$$
That is, $N = \{ x : d_X(x, p) < \delta \}$ is a neighborhood $p$ such that
$N \subseteq \{f > 0\}$.
$\Box$ \\

\end{document}