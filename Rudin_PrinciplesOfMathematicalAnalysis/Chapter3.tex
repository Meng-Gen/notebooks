\documentclass{article}
\usepackage{amsfonts}
\usepackage{amsmath}
\usepackage{amssymb}
\usepackage{hyperref}
\usepackage[none]{hyphenat}
\usepackage{mathrsfs}
\usepackage{physics}
\parindent=0pt

\def\upint{\mathchoice%
    {\mkern13mu\overline{\vphantom{\intop}\mkern7mu}\mkern-20mu}%
    {\mkern7mu\overline{\vphantom{\intop}\mkern7mu}\mkern-14mu}%
    {\mkern7mu\overline{\vphantom{\intop}\mkern7mu}\mkern-14mu}%
    {\mkern7mu\overline{\vphantom{\intop}\mkern7mu}\mkern-14mu}%
  \int}
\def\lowint{\mkern3mu\underline{\vphantom{\intop}\mkern7mu}\mkern-10mu\int}

\begin{document}

\textbf{\Large Chapter 3: Numerical Sequences and Series} \\\\



\emph{Author: Meng-Gen Tsai} \\
\emph{Email: plover@gmail.com} \\\\



\textbf{Exercise 3.1.}
\emph{Prove that the convergence of $\{s_n\}$ implies
convergence of $\{\abs{s_n}\}$.  Is the converse true?} \\

\emph{Proof.}
\begin{enumerate}
\item[(1)]
Since $\{s_n\}$ is convergent, there is $s \in \mathbb{R}^1$
with the following property:
given any $\varepsilon > 0$, there is $N$ such that
$\abs{s_n - s} < \varepsilon$ whenever $n \geq N$.
So
$$\abs{\abs{s_n}-\abs{s}} \leq \abs{s_n - s}  < \varepsilon$$
(Exercise 1.13). That is,
$\{\abs{s_n}\}$ converges to $\abs{s}$.
\item[(2)]
The converse is not true by considering $s_n = (-1)^{n+1}$.
\end{enumerate}
$\Box$ \\\\



%%%%%%%%%%%%%%%%%%%%%%%%%%%%%%%%%%%%%%%%%%%%%%%%%%%%%%%%%%%%%%%%%%%%%%%%%%%%%%%%



\textbf{Exercise 3.2.}
\emph{Calculate $\lim_{n \to \infty}(\sqrt{n^2 + n} - n)$.} \\

\emph{Proof.}
$$\sqrt{n^2 + n} - n
= \frac{n}{\sqrt{n^2+n} + n}
= \frac{1}{\sqrt{1+\frac{1}{n}} + 1} \to \frac{1}{1+1} = \frac{1}{2}$$
as $n \to \infty$.
$\Box$ \\

\emph{Proof ($\varepsilon-N$ argument).}
Let $s_n = \sqrt{n^2 + n} - n$.
\emph{Show that the sequence $\{s_n\}$ converges to $s = \frac{1}{2}$.}
Given any $\varepsilon > 0$, there is $N > \frac{1}{\varepsilon}$ such that
\begin{align*}
\abs{s_n - s}
&= \abs{(\sqrt{n^2 + n} - n) - \frac{1}{2}}
= \abs{\frac{1}{\sqrt{1+\frac{1}{n}} + 1} - \frac{1}{2}} \\
&= \abs{
\frac{2-\left(\sqrt{1+\frac{1}{n}} + 1\right)}
{2\left(\sqrt{1+\frac{1}{n}} + 1\right)}
}
= \abs{
\frac{1-\sqrt{1+\frac{1}{n}}}
{2\left(\sqrt{1+\frac{1}{n}} + 1\right)}
} \\
&= \abs{
\frac{1 - \left(1- \frac{1}{n}\right)}
{2\left(\sqrt{1+\frac{1}{n}} + 1\right)^2}}
= \abs{
\frac{- \frac{1}{n}}
{2\left(\sqrt{1+\frac{1}{n}} + 1\right)^2}}
< \frac{1}{n}
\leq \frac{1}{N}
< \varepsilon
\end{align*}
wheneven $n \geq N$.
$\Box$ \\\\



%%%%%%%%%%%%%%%%%%%%%%%%%%%%%%%%%%%%%%%%%%%%%%%%%%%%%%%%%%%%%%%%%%%%%%%%%%%%%%%%



\textbf{Exercise 3.3.}
\emph{If $s_1 = \sqrt{2}$ and
$$s_{n+1} = \sqrt{2+\sqrt{s_n}} \:\: (n=1,2,3,...),$$
prove that $\{s_n\}$ converges, and that $s_n < 2$ for $n=1,2,3,...$.} \\

The convergence of $\{s_n\}$
implies there is $s \in \mathbb{R}$ such that $s_n \to s$
where $s = \sqrt{2+\sqrt{s}}$ and $\sqrt{2} < s \leq 2$.
WolframAlpha shows that
$$s = \frac{1}{3}
\left(
-1 + \sqrt[3]{\frac{1}{2}(79 - 3 \sqrt{249})}
   + \sqrt[3]{\frac{1}{2}(79 + 3 \sqrt{249})}
\right).$$ \\

\emph{Proof (Theorem 3.14).}
\begin{enumerate}
\item[(1)]
\emph{Show that $\{s_n\}$ is increasing (by mathematical induction).}
  \begin{enumerate}
  \item[(a)]
  \emph{Show that $s_2 > s_1$.}
  In fact,
  $$s_2 = \sqrt{2+\sqrt{s_1}} = \sqrt{2+\sqrt{\sqrt{2}}} < \sqrt{2} = s_1.$$
  \item[(a)]
  \emph{Show that $s_{n+1} > s_{n}$ if $s_{n} > s_{n-1}$.}
  $$s_{n+1} = \sqrt{2+\sqrt{s_{n}}} > \sqrt{2+\sqrt{s_{n-1}}} = s_n.$$
  \end{enumerate}
By mathematical induction, $\{s_n\}$ is (strictly) increasing.
\item[(2)]
\emph{Show that $\{s_n\}$ is bounded (by mathematical induction).}
  \begin{enumerate}
  \item[(a)]
  \emph{Show that $s_1 \leq 2$.}
  $\sqrt{2} \leq 2$.
  \item[(a)]
  \emph{Show that $s_{n+1} \leq 2$ if $s_{n} \leq 2$.}
  $$s_{n+1} = \sqrt{2+\sqrt{s_{n}}} \leq \sqrt{2+\sqrt{2}} < 2.$$
  \end{enumerate}
By mathematical induction, $\{s_n\}$ is bounded by $2$.
\end{enumerate}
Hence, $\{s_n\}$ converges since $\{s_n\}$ is increasing and bounded (Theorem 3.14).
$\Box$ \\\\



%%%%%%%%%%%%%%%%%%%%%%%%%%%%%%%%%%%%%%%%%%%%%%%%%%%%%%%%%%%%%%%%%%%%%%%%%%%%%%%%



\textbf{Exercise 3.4.}
\emph{Find the upper and lower limits of the sequences $\{s_n\}$ defined by
$$s_1 = 0; s_{2m} = \frac{s_{2m-1}}{2}; s_{2m+1} = \frac{1}{2} + s_{2m}.$$ } \\

Write out the first few terms of $\{s_n\}$:
$$0, 0, \frac{1}{2}, \frac{1}{4}, \frac{3}{4},
\frac{3}{8}, \frac{7}{8}, \frac{7}{16}, \frac{15}{16}, ...$$

It suggests us
\begin{align*}
s_{2m+1} &= 1 - \frac{1}{2^m} \:\: (m = 0, 1, 2, ...), \\
s_{2m} &= \frac{1}{2} - \frac{1}{2^m} \:\: (m = 1, 2, 3, ...). \\
\end{align*}

\emph{Proof.}
\begin{enumerate}
\item[(1)]
\emph{Show that
\begin{align*}
s_{2m+1} &= 1 - \frac{1}{2^m} \:\: (m = 0, 1, 2, ...), \\
s_{2m} &= \frac{1}{2} - \frac{1}{2^m}. \:\: (m = 1, 2, 3, ...)
\end{align*}}
Apply mathematical induction.
\item[(2)]
The upper limit is $1$.
\item[(3)]
The lower limit is $\frac{1}{2}$.
\end{enumerate}
$\Box$ \\\\



%%%%%%%%%%%%%%%%%%%%%%%%%%%%%%%%%%%%%%%%%%%%%%%%%%%%%%%%%%%%%%%%%%%%%%%%%%%%%%%%



\textbf{Exercise 3.5.}
\emph{For any two real sequences $\{a_n\}$, $\{b_n\}$, prove that
$$\limsup_{n \to \infty} (a_n + b_n)
\leq \limsup_{n \to \infty} a_n + \limsup_{n \to \infty} b_n$$
provided the sum of the right is not of the form $\infty - \infty$.} \\

\emph{Proof.}
Write
$\alpha = \limsup_{n \to \infty} a_n$
and
$\beta = \limsup_{n \to \infty} b_n$.
\begin{enumerate}
\item[(1)]
\emph{$\alpha = \infty$ and $\beta = \infty$.}
Nothing to do.
\item[(2)]
\emph{$\alpha = -\infty$ and $\beta = -\infty$.}
Since $\alpha = -\infty < \infty$,
there exists $M'$ such that $a_n < M'$ for all $n$.
For any real $M$, $a_n > M - M'$ for at most a finite number of values of $n$
(Theorem 3.17(a)).
Hence $a_n + b_n > M$ for at most a finite number of values of $n$.
Hence $\limsup_{n \to \infty} (a_n + b_n) = -\infty$,
or
$$\limsup_{n \to \infty} (a_n + b_n)
= \limsup_{n \to \infty} a_n + \limsup_{n \to \infty} b_n$$
in this case.
\item[(3)]
\emph{$\alpha$ and $\beta$ are finite.}
(Similar to the argument in Theorem 3.37.)
Choose $\alpha' > \alpha$ and $\beta' > \beta$.
There is an integer $N$ such that
$$\alpha' \geq a_n \text{ and } \beta' \geq b_n$$
whenever $n \geq N$.
Hence $$a_n + b_n \leq \alpha' + \beta'$$
whenever $n \geq N$. Take $\limsup$ to get
Hence $$\limsup_{n \to \infty} (a_n + b_n) \leq \alpha' + \beta'.$$
Since the inequality is true for every $\alpha' > \alpha$ and $\beta' > \beta$,
we have
$$\limsup_{n \to \infty} (a_n + b_n)
\leq \limsup_{n \to \infty} a_n + \limsup_{n \to \infty} b_n.$$
\end{enumerate}
$\Box$ \\\\



%%%%%%%%%%%%%%%%%%%%%%%%%%%%%%%%%%%%%%%%%%%%%%%%%%%%%%%%%%%%%%%%%%%%%%%%%%%%%%%%



\textbf{Exercise 3.7.}
\emph{Prove that the convergence of $\sum a_n$ implies the convergence of
$$\sum \frac{\sqrt{a_n}}{n},$$
if $a_n \geq 0$.} \\

\emph{Proof (Cauchy's inequatity).}
\begin{enumerate}
\item[(1)]
\emph{Show that $\sum\frac{\sqrt{a_n}}{n}$ is bounded.}
For any $k \in \mathbb{Z}^{+}$,
\begin{align*}
\left( \sum_{n=1}^{k} \frac{\sqrt{a_n}}{n} \right)^2
\leq&
\left( \sum_{n=1}^{k}{a_n} \right)
\left( \sum_{n=1}^{k}{\frac{1}{n^2}} \right)
  &(\text{Cauchy's inequatity}) \\
\leq& \left( \sum^{\infty}_{n=1}{a_n} \right)
\left( \sum^{\infty}_{n=1}{\frac{1}{n^2}} \right).
  &\left(\text{$\sum{a_n}, \sum{\frac{1}{n^2}}$: convergent}\right)
\end{align*}
Thus,
$\left( \sum_{n=1}^{k}\frac{\sqrt{a_n}}{n} \right)^2$ is bounded,
or $\sum_{n=1}^{k}\frac{\sqrt{a_n}}{n}$ is bounded.
\item[(2)]
\emph{Show that $\sum_{n=1}^{k} \frac{\sqrt{a_n}}{n}$ is increasing.}
It is clear due to $\frac{\sqrt{a_n}}{n} \geq 0$.
\end{enumerate}
By Theorem 3.14, $\sum_{n=1}^{\infty} \frac{\sqrt{a_n}}{n}$ converges.
$\Box$ \\

\emph{Proof (AM-GM inequality).}
\emph{Show that $\sum\frac{\sqrt{a_n}}{n}$ is bounded.}
\begin{align*}
\frac{\sqrt{a_n}}{n}
\leq&
\frac{1}{2}
\left( a_n + \frac{1}{n^2} \right)
  &(\text{AM-GM inequality}) \\
\sum_{n=1}^{k} \frac{\sqrt{a_n}}{n}
\leq&
\frac{1}{2}
\left( \sum_{n=1}^{k} a_n + \sum_{n=1}^{k} \frac{1}{n^2} \right) \\
\leq&
\frac{1}{2}
\left( \sum_{n=1}^{\infty} a_n + \sum_{n=1}^{\infty} \frac{1}{n^2} \right).
  &\left(\text{$\sum{a_n}, \sum{\frac{1}{n^2}}$: convergent}\right)
\end{align*}
Thus, $\sum_{n=1}^{k}\frac{\sqrt{a_n}}{n}$ is bounded.
The rest proof is the same as previous.
$\Box$ \\\\



%%%%%%%%%%%%%%%%%%%%%%%%%%%%%%%%%%%%%%%%%%%%%%%%%%%%%%%%%%%%%%%%%%%%%%%%%%%%%%%%



\textbf{Exercise 3.20.}
\emph{Suppose $\{p_n\}$ is a Cauchy sequence in a metric space $X$,
and some subsequence $\{p_{n_i}\}$ converges to a point $p \in X$.
Prove that the full sequence $\{p_n\}$ converges to $p$. } \\

\emph{Proof.}
Given any $\varepsilon > 0$.
\begin{enumerate}
\item[(1)]
Since $\{p_n\}$ is a Cauchy sequence, there exists a positive integer $N_1$ such that
$$d(p_n,p_m) < \frac{\varepsilon}{2} \text{ whenever } n, m \geq N_1.$$
\item[(2)]
Since the subsequence $\{p_{n_i}\}$ converges to a point $p \in X$,
there exists a positive integer $N_2$ such that
$$d(p_{n_i},p) < \frac{\varepsilon}{2} \text{ whenever } n_i \geq N_2.$$
\item[(3)]
Let $N = \max\{N_1, N_2\}$ be a positive integer.
So
\begin{align*}
d(p_n,p)
&\leq d(p_n,p_{n_i}) + d(p_{n_i}, p)
  &\text{(Definition 2.15(c))} \\
&< \frac{\varepsilon}{2} + \frac{\varepsilon}{2} \text{ whenever } n, n_i \geq N
  &\text{((1)(2))} \\
&= \varepsilon \text{ whenever } n \geq N.
\end{align*}
Hence the full sequence $\{p_n\}$ converges to $p$.
\end{enumerate}
$\Box$ \\\\



%%%%%%%%%%%%%%%%%%%%%%%%%%%%%%%%%%%%%%%%%%%%%%%%%%%%%%%%%%%%%%%%%%%%%%%%%%%%%%%%



\textbf{Exercise 3.21.}
\emph{Prove the following analogue of Theorem 3.10(b):
If $\{E_n\}$ is a sequence of closed and bounded sets in a complete metric space $X$,
if $E_n \supseteq E_{n+1}$, and if
$$\lim_{n \to \infty} \mathrm{diam}(E_n) = 0,$$
then $\bigcap_{n=1}^{\infty} E_n$ consists of exactly one point.} \\

Assume $E_n \neq \varnothing$. It is unnecessary to assume that $E_n$ is bounded
since we have the condition that $\lim_{n \to \infty} \mathrm{diam}(E_n) = 0$.\\

\emph{Note.}
Every compact metric space is complete, but complete spaces need not be compact.
In fact, a metric space is compact if and only if it is complete and totally bounded. \\

\emph{Proof.}
\begin{enumerate}
\item[(1)]
Pick $p_n \in E_n$ for $n = 1, 2, \ldots$.
\item[(2)]
\emph{Show that $\{p_n\}$ is a Cauchy sequence.}
Given any $\varepsilon > 0$.
There is a positive integer $N$ such that
$\mathrm{diam}(E_n) < \varepsilon$ whenever $n \geq N$.
Especially, $$\mathrm{diam}(E_N) < \varepsilon.$$
As $m, n \geq N$, $p_m \in E_m \subseteq E_N$ and $p_n \in E_n \subseteq E_N$.
By the definition of the diameter of $E_N$,
$$d(p_m,p_n) \leq \mathrm{diam}(E_N) < \varepsilon \text{ whenever } m,n \geq N.$$
\item[(3)]
Since $X$ is complete, $\{p_n\}$ converges to a point $p \in X$.
\item[(4)]
\emph{Show that $p \in \bigcap_{n=1}^{\infty} E_n$.}
(Reductio ad absurdum)
If there were some $n$ such that $p \not\in E_{n}$.
Consider the subsequence
$$p_{n}, p_{n+1}, p_{n+2}, \ldots.$$
Note that all $p_{n}, p_{n+1}, \ldots$ are in $E_n$.
By (3), it converges to $p$. Thus $p$ is a limit point of $E_n$.
Since $E_n$ is closed, $p \in E_n$, which is absurd.
\item[(5)]
\emph{Show that $\bigcap_{n=1}^{\infty} E_n = \{p\}$.}
(Reductio ad absurdum)
If there were $q \in \bigcap_{n=1}^{\infty} E_n$ with $q \neq p$,
then $d(p,q) > 0$ (Definition 2.15(a)).
It implies that
$$\mathrm{diam}(E_n)
\geq d(p,q) > 0 \text{ for all } n,$$
contrary to $\lim_{n \to \infty} \mathrm{diam}(E_n) = 0$.
\end{enumerate}
$\Box$ \\\\



%%%%%%%%%%%%%%%%%%%%%%%%%%%%%%%%%%%%%%%%%%%%%%%%%%%%%%%%%%%%%%%%%%%%%%%%%%%%%%%%



\textbf{Exercise 3.22 (Baire category theorem).}
\emph{Suppose $X$ is a complete metric space,
and $\{G_n\}$ is a sequence of dense open subsets of $X$.
Prove Baire's theorem, namely, that $\bigcap^\infty_1{G_n}$ is not empty.
(In fact, it is dense in $X$.)
(Hint: Find a shrinking sequence of neighborhoods $E_n$ such
that $\overline{E_n} \subseteq G_n$, and apply Exercise 3.21.) } \\

\emph{Proof.}
Given any open set $G_0$ in $X$,
will show that $$\bigcap_{n=0}^{\infty} G_n \neq \varnothing.$$
\begin{enumerate}
\item[(1)]
Since $G_1$ is dense, $G_0 \cap G_1$ is nonempty.
Take any one point $p_1$ in the open set $G_0 \cap G_1$,
then there exists a closed neighborhood
$$V_1
= \{ q \in X : d(q,p_1) < r_1 \}$$
of $p_1$ with $r_1 < 1$
such that
$$V_1 \subseteq G_0 \cap G_1.$$
Take $U_1 \subseteq E_1 \subseteq V_1$
such that
\begin{align*}
E_1 &= \left\{ q \in X : d(q,p_1) \leq \frac{r_1}{64} \right\} \subseteq V_1, \\
U_1 &= \left\{ q \in X : d(q,p_1) < \frac{r_1}{89} \right\} \subseteq E_1.
\end{align*}
\item[(2)]
Suppose $V_n, E_n, U_n$ have been constructed,
take any one point $p_{n+1}$ in the open set $U_n \cap G_{n+1}$,
there exists an open neighborhood
$$V_{n+1}
= \{ q \in X : d(q,p_{n+1}) < r_{n+1} \}$$
of $p_{n+1}$ with $r_{n+1}$ with $r_{n+1} < \frac{1}{n+1}$
such that
$$V_{n+1} \subseteq U_n \cap G_{n+1}.$$
Take $U_1 \subseteq E_1 \subseteq V_1$
such that
\begin{align*}
E_{n+1} &= \left\{ q \in X : d(q,p_{n+1}) \leq \frac{r_{n+1}}{64} \right\} \subseteq V_{n+1}, \\
U_{n+1} &= \left\{ q \in X : d(q,p_{n+1}) < \frac{r_{n+1}}{89} \right\} \subseteq E_{n+1}.
\end{align*}
\item[(3)]
Note that
  \begin{enumerate}
  \item[(a)]
  $E_n$ is closed and nonempty (since $p_n \in E_n$).
  \item[(b)]
  $\lim_{n \to \infty} \mathrm{diam}(E_n) = 0$
  (since $\mathrm{diam}(E_n) \leq 2 \cdot \frac{r_n}{64} < r_n < \frac{1}{n}$.)
  \item[(c)]
  $E_1 \supseteq E_2 \supseteq \cdots$
  (since
  $E_{n+1} \subseteq V_{n+1} \subseteq U_n \cap G_{n+1} \subseteq U_n \subseteq E_n$).
  \end{enumerate}
Since $X$ is complete, by Exercise 3.21,
$$\bigcap_{n=1}^{\infty} E_n = \{p\}$$
for some $p \in X$.
\item[(4)]
Hence
\begin{align*}
p \in \bigcap_{n=1}^{\infty} E_n
&\Longleftrightarrow
p \in E_n \text{ for all } n=1,2,3,\ldots \\
&\Longrightarrow
p \in E_1 \subseteq G_0 \cap G_1 \text{ and }
p \in E_{n+1} \subseteq U_n \cap G_{n+1} \subseteq G_{n+1} \\
&\Longrightarrow
p \in G_0 \cap G_1 \cap \cdots = \bigcap_{n=0}^{\infty} G_n \\
&\Longrightarrow
\bigcap_{n=0}^{\infty} G_n \neq \varnothing.
\end{align*}
\end{enumerate}
$\Box$ \\\\



%%%%%%%%%%%%%%%%%%%%%%%%%%%%%%%%%%%%%%%%%%%%%%%%%%%%%%%%%%%%%%%%%%%%%%%%%%%%%%%%



\textbf{Exercise 3.23.}



%%%%%%%%%%%%%%%%%%%%%%%%%%%%%%%%%%%%%%%%%%%%%%%%%%%%%%%%%%%%%%%%%%%%%%%%%%%%%%%%
%%%%%%%%%%%%%%%%%%%%%%%%%%%%%%%%%%%%%%%%%%%%%%%%%%%%%%%%%%%%%%%%%%%%%%%%%%%%%%%%



\end{document}