\documentclass{article}
\usepackage{amsfonts}
\usepackage{amsmath}
\usepackage{amssymb}
\usepackage{hyperref}
\usepackage[none]{hyphenat}
\usepackage{mathrsfs}
\usepackage{physics}
\parindent=0pt

\def\upint{\mathchoice%
    {\mkern13mu\overline{\vphantom{\intop}\mkern7mu}\mkern-20mu}%
    {\mkern7mu\overline{\vphantom{\intop}\mkern7mu}\mkern-14mu}%
    {\mkern7mu\overline{\vphantom{\intop}\mkern7mu}\mkern-14mu}%
    {\mkern7mu\overline{\vphantom{\intop}\mkern7mu}\mkern-14mu}%
  \int}
\def\lowint{\mkern3mu\underline{\vphantom{\intop}\mkern7mu}\mkern-10mu\int}

\begin{document}

\textbf{\Large Chapter 3: Numerical Sequences and Series} \\\\



\emph{Author: Meng-Gen Tsai} \\
\emph{Email: plover@gmail.com} \\\\



%%%%%%%%%%%%%%%%%%%%%%%%%%%%%%%%%%%%%%%%%%%%%%%%%%%%%%%%%%%%%%%%%%%%%%%%%%%%%%%%
%%%%%%%%%%%%%%%%%%%%%%%%%%%%%%%%%%%%%%%%%%%%%%%%%%%%%%%%%%%%%%%%%%%%%%%%%%%%%%%%



\textbf{Exercise 3.1.}
\emph{Prove that the convergence of $\{s_n\}$ implies
convergence of $\{\abs{s_n}\}$.  Is the converse true?} \\

\emph{Proof.}
\begin{enumerate}
\item[(1)]
Since $\{s_n\}$ is convergent, there is $s \in \mathbb{R}^1$
with the following property:
given any $\varepsilon > 0$, there is $N$ such that
$\abs{s_n - s} < \varepsilon$ whenever $n \geq N$.
So
$$\abs{\abs{s_n}-\abs{s}} \leq \abs{s_n - s}  < \varepsilon$$
(Exercise 1.13). That is,
$\{\abs{s_n}\}$ converges to $\abs{s}$.
\item[(2)]
The converse is not true by considering $s_n = (-1)^{n+1}$.
\end{enumerate}
$\Box$ \\\\



%%%%%%%%%%%%%%%%%%%%%%%%%%%%%%%%%%%%%%%%%%%%%%%%%%%%%%%%%%%%%%%%%%%%%%%%%%%%%%%%



\textbf{Exercise 3.2.}
\emph{Calculate $\lim_{n \to \infty}(\sqrt{n^2 + n} - n)$.} \\

\emph{Proof.}
$$\sqrt{n^2 + n} - n
= \frac{n}{\sqrt{n^2+n} + n}
= \frac{1}{\sqrt{1+\frac{1}{n}} + 1} \to \frac{1}{1+1} = \frac{1}{2}$$
as $n \to \infty$.
$\Box$ \\

\emph{Proof ($\varepsilon-N$ argument).}
Let $s_n = \sqrt{n^2 + n} - n$.
\emph{Show that the sequence $\{s_n\}$ converges to $s = \frac{1}{2}$.}
Given any $\varepsilon > 0$, there is $N > \frac{1}{\varepsilon}$ such that
\begin{align*}
\abs{s_n - s}
&= \abs{(\sqrt{n^2 + n} - n) - \frac{1}{2}}
= \abs{\frac{1}{\sqrt{1+\frac{1}{n}} + 1} - \frac{1}{2}} \\
&= \abs{
\frac{2-\left(\sqrt{1+\frac{1}{n}} + 1\right)}
{2\left(\sqrt{1+\frac{1}{n}} + 1\right)}
}
= \abs{
\frac{1-\sqrt{1+\frac{1}{n}}}
{2\left(\sqrt{1+\frac{1}{n}} + 1\right)}
} \\
&= \abs{
\frac{1 - \left(1- \frac{1}{n}\right)}
{2\left(\sqrt{1+\frac{1}{n}} + 1\right)^2}}
= \abs{
\frac{- \frac{1}{n}}
{2\left(\sqrt{1+\frac{1}{n}} + 1\right)^2}}
< \frac{1}{n}
\leq \frac{1}{N}
< \varepsilon
\end{align*}
wheneven $n \geq N$.
$\Box$ \\\\



%%%%%%%%%%%%%%%%%%%%%%%%%%%%%%%%%%%%%%%%%%%%%%%%%%%%%%%%%%%%%%%%%%%%%%%%%%%%%%%%



\textbf{Exercise 3.3.}
\emph{If $s_1 = \sqrt{2}$ and
$$s_{n+1} = \sqrt{2+\sqrt{s_n}} \:\: (n=1,2,3,...),$$
prove that $\{s_n\}$ converges, and that $s_n < 2$ for $n=1,2,3,...$.} \\

The convergence of $\{s_n\}$
implies there is $s \in \mathbb{R}$ such that $s_n \to s$
where $s = \sqrt{2+\sqrt{s}}$ and $\sqrt{2} < s \leq 2$.
WolframAlpha shows that
$$s = \frac{1}{3}
\left(
-1 + \sqrt[3]{\frac{1}{2}(79 - 3 \sqrt{249})}
   + \sqrt[3]{\frac{1}{2}(79 + 3 \sqrt{249})}
\right).$$ \\

\emph{Proof (Theorem 3.14).}
\begin{enumerate}
\item[(1)]
\emph{Show that $\{s_n\}$ is increasing (by mathematical induction).}
  \begin{enumerate}
  \item[(a)]
  \emph{Show that $s_2 > s_1$.}
  In fact,
  $$s_2 = \sqrt{2+\sqrt{s_1}} = \sqrt{2+\sqrt{\sqrt{2}}} < \sqrt{2} = s_1.$$
  \item[(a)]
  \emph{Show that $s_{n+1} > s_{n}$ if $s_{n} > s_{n-1}$.}
  $$s_{n+1} = \sqrt{2+\sqrt{s_{n}}} > \sqrt{2+\sqrt{s_{n-1}}} = s_n.$$
  \end{enumerate}
By mathematical induction, $\{s_n\}$ is (strictly) increasing.
\item[(2)]
\emph{Show that $\{s_n\}$ is bounded (by mathematical induction).}
  \begin{enumerate}
  \item[(a)]
  \emph{Show that $s_1 \leq 2$.}
  $\sqrt{2} \leq 2$.
  \item[(a)]
  \emph{Show that $s_{n+1} \leq 2$ if $s_{n} \leq 2$.}
  $$s_{n+1} = \sqrt{2+\sqrt{s_{n}}} \leq \sqrt{2+\sqrt{2}} < 2.$$
  \end{enumerate}
By mathematical induction, $\{s_n\}$ is bounded by $2$.
\end{enumerate}
Hence, $\{s_n\}$ converges since $\{s_n\}$ is increasing and bounded (Theorem 3.14).
$\Box$ \\\\



%%%%%%%%%%%%%%%%%%%%%%%%%%%%%%%%%%%%%%%%%%%%%%%%%%%%%%%%%%%%%%%%%%%%%%%%%%%%%%%%



\textbf{Exercise 3.4.}
\emph{Find the upper and lower limits of the sequences $\{s_n\}$ defined by
$$s_1 = 0; s_{2m} = \frac{s_{2m-1}}{2}; s_{2m+1} = \frac{1}{2} + s_{2m}.$$ } \\

Write out the first few terms of $\{s_n\}$:
$$0, 0, \frac{1}{2}, \frac{1}{4}, \frac{3}{4},
\frac{3}{8}, \frac{7}{8}, \frac{7}{16}, \frac{15}{16}, ...$$

It suggests us
\begin{align*}
s_{2m+1} &= 1 - \frac{1}{2^m} \:\: (m = 0, 1, 2, ...), \\
s_{2m} &= \frac{1}{2} - \frac{1}{2^m} \:\: (m = 1, 2, 3, ...). \\
\end{align*}

\emph{Proof.}
\begin{enumerate}
\item[(1)]
\emph{Show that
\begin{align*}
s_{2m+1} &= 1 - \frac{1}{2^m} \:\: (m = 0, 1, 2, ...), \\
s_{2m} &= \frac{1}{2} - \frac{1}{2^m}. \:\: (m = 1, 2, 3, ...)
\end{align*}}
Apply mathematical induction.
\item[(2)]
The upper limit is $1$.
\item[(3)]
The lower limit is $\frac{1}{2}$.
\end{enumerate}
$\Box$ \\\\



%%%%%%%%%%%%%%%%%%%%%%%%%%%%%%%%%%%%%%%%%%%%%%%%%%%%%%%%%%%%%%%%%%%%%%%%%%%%%%%%



\textbf{Exercise 3.5.}
\emph{For any two real sequences $\{a_n\}$, $\{b_n\}$, prove that
$$\limsup_{n \to \infty} (a_n + b_n)
\leq \limsup_{n \to \infty} a_n + \limsup_{n \to \infty} b_n$$
provided the sum of the right is not of the form $\infty - \infty$.} \\

\emph{Proof.}
Write
$\alpha = \limsup_{n \to \infty} a_n$
and
$\beta = \limsup_{n \to \infty} b_n$.
\begin{enumerate}
\item[(1)]
\emph{$\alpha = \infty$ and $\beta = \infty$.}
Nothing to do.
\item[(2)]
\emph{$\alpha = -\infty$ and $\beta = -\infty$.}
Since $\alpha = -\infty < \infty$,
there exists $M'$ such that $a_n < M'$ for all $n$.
For any real $M$, $a_n > M - M'$ for at most a finite number of values of $n$
(Theorem 3.17(a)).
Hence $a_n + b_n > M$ for at most a finite number of values of $n$.
Hence $\limsup_{n \to \infty} (a_n + b_n) = -\infty$,
or
$$\limsup_{n \to \infty} (a_n + b_n)
= \limsup_{n \to \infty} a_n + \limsup_{n \to \infty} b_n$$
in this case.
\item[(3)]
\emph{$\alpha$ and $\beta$ are finite.}
(Similar to the argument in Theorem 3.37.)
Choose $\alpha' > \alpha$ and $\beta' > \beta$.
There is an integer $N$ such that
$$\alpha' \geq a_n \text{ and } \beta' \geq b_n$$
whenever $n \geq N$.
Hence $$a_n + b_n \leq \alpha' + \beta'$$
whenever $n \geq N$. Take $\limsup$ to get
Hence $$\limsup_{n \to \infty} (a_n + b_n) \leq \alpha' + \beta'.$$
Since the inequality is true for every $\alpha' > \alpha$ and $\beta' > \beta$,
we have
$$\limsup_{n \to \infty} (a_n + b_n)
\leq \limsup_{n \to \infty} a_n + \limsup_{n \to \infty} b_n.$$
\end{enumerate}
$\Box$ \\\\



%%%%%%%%%%%%%%%%%%%%%%%%%%%%%%%%%%%%%%%%%%%%%%%%%%%%%%%%%%%%%%%%%%%%%%%%%%%%%%%%



\textbf{Exercise 3.6.}
\emph{Investigate the behavior (convergence or divergence) of $\sum a_n$ if}
\begin{enumerate}
\item[(a)]
\emph{$a_n = \sqrt{n+1} - \sqrt{n}$.}
\item[(b)]
\emph{$a_n = \frac{\sqrt{n+1} - \sqrt{n}}{n}$.}
\item[(c)]
\emph{$a_n = (\sqrt[n]{n} - 1)^n$.}
\item[(d)]
\emph{$a_n = \frac{1}{1+z^n}$ for complex values of $z$.} \\
\end{enumerate}

\emph{Proof of (a).}
\begin{enumerate}
\item[(1)]
Divergence.
\item[(2)]
$\sum_{n=1}^{k}a_n = \sqrt{k+1} - 1 \to \infty$ as $k \to \infty$.
\end{enumerate}
$\Box$ \\

\emph{Proof of (b).}
\begin{enumerate}
\item[(1)]
Convergence.
\item[(2)]
Since
$$|a_n|
= \frac{1}{n(\sqrt{n+1}+\sqrt{n})} < \frac{1}{2 n^{\frac{3}{2}}}$$
holds for all $n$ and $\sum \frac{1}{2 n^{\frac{3}{2}}}$ converges
(Theorem 3.28 and Theorem 3.3),
by comparison test (Theorem 3.25), $\sum a_n$ converges.
\end{enumerate}
$\Box$ \\

\emph{Proof of (c).}
\begin{enumerate}
\item[(1)]
Convergence.
\item[(2)]
Note that
$$\alpha
= \limsup_{n \to \infty} \sqrt[n]{|a_n|}
= \limsup_{n \to \infty} \sqrt[n]{n} - 1 = 0$$
(Theorem 3.20(c)).
Since $\alpha < 1$, $\sum a_n$ converges by root test (Theorem 3.33).
\end{enumerate}
$\Box$ \\

\emph{Proof of (d).}
\begin{enumerate}
\item[(1)]
Convergence if $|z| > 1$; divergence if $|z| \leq 1$.
\item[(2)]
Note that
$|z^n+1| + |-1| \geq |z^n|$
(Theorem 1.33(e)),
or
$$|z^n+1| \geq |z|^n - 1.$$
\item[(3)]
If $|z| > 1$, then there is an integer $N$ such that
$$|z|^n \geq 2 \text{ whenever } n \geq N.$$
Therefore, for $n \geq N$ we have
  \begin{align*}
    |a_n|
    &= \frac{1}{|z^n+1|} \\
    &\leq \frac{1}{|z|^n - 1}
      &((2)) \\
    &\leq \frac{1}{|z|^n - \frac{1}{2}|z|^n} \\
    &= \frac{2}{|z|^n}.
  \end{align*}
The geometric series $\sum \frac{2}{|z|^n}$ converges,
by comparison test (Theorem 3.25), $\sum a_n$ converges.
\item[(4)]
If $|z| \leq 1$, then $|a_n| \geq \frac{1}{2}$,
or $\lim a_n \neq 0$.
By Theorem 3.23 ($\lim a_n = 0$ if $\sum a_n$ converges),
$\sum a_n$ diverges.
\end{enumerate}
$\Box$ \\\\



%%%%%%%%%%%%%%%%%%%%%%%%%%%%%%%%%%%%%%%%%%%%%%%%%%%%%%%%%%%%%%%%%%%%%%%%%%%%%%%%



\textbf{Exercise 3.7.}
\emph{Prove that the convergence of $\sum a_n$ implies the convergence of
$$\sum \frac{\sqrt{a_n}}{n},$$
if $a_n \geq 0$.} \\

\emph{Proof (Cauchy's inequatity).}
\begin{enumerate}
\item[(1)]
\emph{Show that $\sum\frac{\sqrt{a_n}}{n}$ is bounded.}
For any $k \in \mathbb{Z}^{+}$,
\begin{align*}
\left( \sum_{n=1}^{k} \frac{\sqrt{a_n}}{n} \right)^2
\leq&
\left( \sum_{n=1}^{k}{a_n} \right)
\left( \sum_{n=1}^{k}{\frac{1}{n^2}} \right)
  &(\text{Cauchy's inequatity}) \\
\leq& \left( \sum^{\infty}_{n=1}{a_n} \right)
\left( \sum^{\infty}_{n=1}{\frac{1}{n^2}} \right).
  &\left(\text{$\sum{a_n}, \sum{\frac{1}{n^2}}$: convergent}\right)
\end{align*}
Thus,
$\left( \sum_{n=1}^{k}\frac{\sqrt{a_n}}{n} \right)^2$ is bounded,
or $\sum_{n=1}^{k}\frac{\sqrt{a_n}}{n}$ is bounded.
\item[(2)]
\emph{Show that $\sum_{n=1}^{k} \frac{\sqrt{a_n}}{n}$ is increasing.}
It is clear due to $\frac{\sqrt{a_n}}{n} \geq 0$.
\end{enumerate}
By Theorem 3.14, $\sum_{n=1}^{\infty} \frac{\sqrt{a_n}}{n}$ converges.
$\Box$ \\

\emph{Proof (AM-GM inequality).}
\emph{Show that $\sum\frac{\sqrt{a_n}}{n}$ is bounded.}
\begin{align*}
\frac{\sqrt{a_n}}{n}
\leq&
\frac{1}{2}
\left( a_n + \frac{1}{n^2} \right)
  &(\text{AM-GM inequality}) \\
\sum_{n=1}^{k} \frac{\sqrt{a_n}}{n}
\leq&
\frac{1}{2}
\left( \sum_{n=1}^{k} a_n + \sum_{n=1}^{k} \frac{1}{n^2} \right) \\
\leq&
\frac{1}{2}
\left( \sum_{n=1}^{\infty} a_n + \sum_{n=1}^{\infty} \frac{1}{n^2} \right).
  &\left(\text{$\sum{a_n}, \sum{\frac{1}{n^2}}$: convergent}\right)
\end{align*}
Thus, $\sum_{n=1}^{k}\frac{\sqrt{a_n}}{n}$ is bounded.
The rest proof is the same as previous.
$\Box$ \\\\



%%%%%%%%%%%%%%%%%%%%%%%%%%%%%%%%%%%%%%%%%%%%%%%%%%%%%%%%%%%%%%%%%%%%%%%%%%%%%%%%



\textbf{Exercise 3.10.}
\emph{Suppose that the coefficients of the power series $\sum a_n z^n$ are integers,
infinitely many of which are distinct from zero.
Prove that the radius of convergence is at most $1$.} \\

\emph{Proof (Theorem 3.39).}
$\alpha = \limsup_{n \to \infty} \sqrt[n]{|a_n|} \geq 1$ by assumption
that $\{a_n\}$ has infinitely many nonzero integers.
Hence the radius of convergence $R = \frac{1}{\alpha} \leq 1$.
$\Box$ \\\\



%%%%%%%%%%%%%%%%%%%%%%%%%%%%%%%%%%%%%%%%%%%%%%%%%%%%%%%%%%%%%%%%%%%%%%%%%%%%%%%%



\textbf{Exercise 3.12.}
\emph{Suppose $a_n > 0$ and $\sum a_n$ converges.
Put
$$r_n = \sum_{m=n}^{\infty} a_m.$$}
\begin{enumerate}
\item[(a)]
\emph{Prove that
$$\frac{a_m}{r_m} + \cdots + \frac{a_n}{r_n} > 1 - \frac{r_n}{r_m}$$
if $m < n$, and deduce that $\sum \frac{a_n}{r_n}$ diverges.}
\item[(b)]
\emph{Prove that
$$\frac{a_n}{\sqrt{r_n}} < 2(\sqrt{r_n} - \sqrt{r_{n+1}})$$
and deduce that $\sum \frac{a_n}{\sqrt{r_n}}$ converges.} \\
\end{enumerate}

\emph{Note.}
\begin{enumerate}
\item[(1)]
Each $r_n$ is positive and finite (since $a_n > 0$ and $\sum a_n$ converges).
\item[(2)]
$\{r_n\}$ is monotonic decreasing (since $a_n > 0$).
\item[(3)]
$\{r_n\}$ converges to $0$ (since $\sum a_n$ converges). \\
\end{enumerate}

\emph{Proof of (a).}
\begin{enumerate}
\item[(1)]
\begin{align*}
  \frac{a_m}{r_m} + \cdots + \frac{a_n}{r_n}
  &> \frac{a_m}{r_m} + \cdots + \frac{a_n}{r_m}
    &(\text{$r_m > r_k$ for $k=m+1,\ldots,n$}) \\
  &= \frac{a_m + \cdots + a_n}{r_m} \\
  &= \frac{r_m - r_{n+1}}{r_m}
    &(\text{Definition of $r_k$}) \\
  &> \frac{r_m - r_{n}}{r_m}
    &(r_n > r_{n+1}) \\
  &= 1 - \frac{r_n}{r_m}.
\end{align*}
\item[(2)]
(Reductio ad absurdum)
If $\sum \frac{a_n}{r_n}$ were converged,
then given $\varepsilon = \frac{1}{64} > 0$
there is an integer $N$ such that
$$\abs{\frac{a_m}{r_m} + \cdots + \frac{a_n}{r_n}} < \frac{1}{64}
\text{ whenever } n \geq m \geq N$$
(Theorem 3.22). By (1), let $m = N$ to get
$$1 - \frac{r_n}{r_N} < \frac{1}{64}
\text{ whenever } n \geq N,$$
or $$r_n > \frac{63}{64} r_N,$$
contrary to the assumption that
$\{r_n\}$ converges to $0$ (since $\sum a_n$ converges).
\end{enumerate}
$\Box$ \\

\emph{Proof of (b).}
\begin{enumerate}
\item[(1)]
Note that each $r_n$ is positive and finite, and thus
\begin{align*}
  \frac{a_n}{\sqrt{r_n}} < 2(\sqrt{r_n} - \sqrt{r_{n+1}})
  &\Longleftrightarrow
  \frac{r_n - r_{n+1}}{\sqrt{r_n}} < 2(\sqrt{r_n} - \sqrt{r_{n+1}}) \\
  &\Longleftrightarrow
  \frac{\sqrt{r_n} + \sqrt{r_{n+1}}}{\sqrt{r_n}} < 2 \\
  &\Longleftrightarrow
  \sqrt{r_n} + \sqrt{r_{n+1}} < 2 \sqrt{r_n} \\
  &\Longleftrightarrow
  \sqrt{r_{n+1}} < \sqrt{r_n} \\
  &\Longleftrightarrow
  r_{n+1} < r_n.
\end{align*}
The last statement holds since $\{r_n\}$ is monotonic decreasing.
\item[(2)]
  \begin{enumerate}
  \item[(a)]
  Each term $\frac{a_n}{\sqrt{r_n}}$ of $\sum \frac{a_n}{\sqrt{r_n}}$ is nonnegative.
  \item[(b)]
  The partial sum
  $$\sum_{k=1}^{n} \frac{a_k}{\sqrt{r_k}}
  < \sum_{k=1}^{n} 2(\sqrt{r_k} - \sqrt{r_{k+1}})
  = 2(\sqrt{r_1} - \sqrt{r_{n+1}})
  < 2\sqrt{r_1}$$
  is bounded by $2\sqrt{r_1}$.
  \end{enumerate}
  By (a)(b), $\sum \frac{a_n}{\sqrt{r_n}}$ converges (Theorem 3.24).
\end{enumerate}
$\Box$ \\\\



%%%%%%%%%%%%%%%%%%%%%%%%%%%%%%%%%%%%%%%%%%%%%%%%%%%%%%%%%%%%%%%%%%%%%%%%%%%%%%%%



\textbf{Exercise 3.13.}
\emph{Prove that the Cauchy product of two absolutely convergent series
converges absolutely.} \\

\emph{Proof.}
\begin{enumerate}
\item[(1)]
Given two absolutely convergent series $\sum a_n$ and $\sum b_n$.
The Cauchy product is $\sum c_n$
where
$$c_n = \sum_{k=0}^{n} a_k b_{n-k} \:\: (n=0,1,2,\ldots).$$
Let $\sum |a_n| = A < \infty$ and $\sum |b_n| = B < \infty$.
\item[(2)]
Each term $|c_k|$ of $\sum_{k=0}^{n}|c_k|$ is nonnegative.
\item[(3)]
Thus,
\begin{align*}
  \sum_{k=0}^{n}|c_k|
  &= \sum_{k=0}^{n} \abs{ \sum_{m=0}^{k} a_m b_{k-m} } \\
  &\leq \sum_{k=0}^{n} \sum_{m=0}^{k} |a_m| |b_{k-m}| \\
  &= \sum_{k=0}^{n} |a_k| \sum_{m=0}^{n-k} |b_m| \\
  &\leq \sum_{k=0}^{n} |a_k| B \\
  &\leq AB \\
  &< \infty.
\end{align*}
\item[(4)]
By (2)(3), $\sum_{k=0}^{n}|c_k|$ converges (Theorem 3.24),
or $\sum_{k=0}^{n} c_k$ converges absolutely.
\end{enumerate}
$\Box$ \\\\



%%%%%%%%%%%%%%%%%%%%%%%%%%%%%%%%%%%%%%%%%%%%%%%%%%%%%%%%%%%%%%%%%%%%%%%%%%%%%%%%



\textbf{Exercise 3.14 (Ces\`aro convergence).}
\emph{If $\{s_n\}$ is a complex sequence, define its arithmetic means $\sigma_n$ by
$$\sigma_n
= \frac{s_0 + s_1 + \cdots + s_n}{n+1} \:\: (n=0,1,2,\ldots).$$}
\begin{enumerate}
\item[(a)]
\emph{If $\lim s_n = s$, prove that $\lim \sigma_n = s$.}
\item[(b)]
\emph{Construct a sequence $\{s_n\}$ which does not converge, although $\lim \sigma_n = 0$.}
\item[(c)]
\emph{Can it happen that $s_n > 0$ for all $n$ and that $\limsup s_n = \infty$,
although $\lim \sigma_n = 0$?}
\item[(d)]
\emph{Put $a_n = s_n - s_{n-1}$, for $n \geq 1$.
Show that
$$s_n - \sigma_n = \frac{1}{n+1} \sum_{k=1}^{n} ka_k.$$
Assume that $\lim (na_n) = 0$ and that $\{\sigma_n\}$ converges.
Prove that $\{s_n\}$ converges.
[This gives a converse of (a), but under the additional assumption that $na_n \to 0$.]}
\item[(e)]
\emph{Derive the last conclusion from a weaker hypothesis:
Assume $M \leq \infty$, $|na_n| < M$ for all $n$, and $\lim \sigma_n = \sigma$.
Prove that $\lim s_n = \sigma$, by completing the following outline:}

\emph{If $m < n$, then
$$s_n - \sigma_n
= \frac{m+1}{n-m}(\sigma_n - \sigma_m)
  + \frac{1}{n-m}\sum_{i=m+1}^{n}(s_n - s_i).$$
For these $i$,
$$|s_n - s_i|
\leq \frac{(n-i)M}{i+1}
\leq \frac{(n-m-1)M}{m+2}.$$
Fix $\varepsilon > 0$ and associate with each $n$ the integer $m$ that satisfies
$$m \leq \frac{n-\varepsilon}{1+\varepsilon} < m+1.$$
Then $\frac{m+1}{n-m} \leq \frac{1}{\varepsilon}$ and $|s_n - s_i| < M\varepsilon$.
Hence
$$\limsup_{n \to \infty} |s_n - \sigma| \leq M\varepsilon.$$
Since $\varepsilon$ was arbitrary, $\lim s_n = \sigma$.} \\
\end{enumerate}



\emph{Proof of (a).}
Given any $\varepsilon > 0$.
\begin{enumerate}
\item[(1)]
For such $\varepsilon > 0$, there is an integer $N' \geq 1$ such that
$$|s_n-s| < \frac{\varepsilon}{64} \text{ whenever } n \geq N'.$$
\item[(2)]
For such $N'$, $\sum_{n=0}^{N'} |s_n-s|$ is finite.
Let $N''$ be an integer such that $$\sum_{n=0}^{N'} |s_n-s| < \frac{N''\varepsilon}{89}$$
(by taking $N'' = \left\lfloor \frac{89}{\varepsilon}\sum_{n=0}^{N'} |s_n-s| \right\rfloor + 1$).
\item[(3)]
Note that
  \begin{align*}
  |\sigma_n - s|
  &= \abs{ \left(\frac{1}{n+1}\sum_{k=0}^n s_k\right) - s } \\
  &= \abs{ \frac{1}{n+1}\sum_{k=0}^n (s_k - s) } \\
  &\leq \frac{1}{n+1}\sum_{k=0}^n |s_k - s|
  \end{align*}
holds for each $n=0,1,2,\ldots$.
In particular, for $n \geq N = \max\{N', N''\} \geq 1$, we have
  \begin{align*}
  |\sigma_{n} - s|
  &\leq \frac{1}{n+1}\sum_{k=0}^{n} |s_k - s| \\
  &\leq \left( \frac{1}{n+1}\sum_{k=0}^{N'}|s_k - s| \right)
    + \left( \frac{1}{n+1}\sum_{k=N'+1}^{n}|s_k - s| \right) \\
  &< \frac{1}{n+1} \cdot \frac{N''\varepsilon}{89}
    + \frac{1}{n+1} \cdot \frac{(n-N')\varepsilon}{64} \\
  &< \frac{\varepsilon}{89} + \frac{\varepsilon}{64} \\
  &< \varepsilon.
  \end{align*}
Therefore, $\lim \sigma_n = s$.
\end{enumerate}
$\Box$ \\



\emph{Proof of (b).}
Define $\{s_n\}$ by $s_n = (-1)^{n+1}$.
$\Box$ \\



\emph{Proof of (c).}
Yes.
Define
\begin{equation*}
  s_n =
    \begin{cases}
      \frac{1}{n!} + m^{63}
        & \text{if $n = m^{89}$ for some $m \in \mathbb{Z}$}, \\
      \frac{1}{n!}
        & \text{otherwise}.
    \end{cases}
\end{equation*}
\begin{enumerate}
\item[(1)]
Clearly, $\limsup s_n = \infty$.
\item[(2)]
Given any $n$, there is $m \in \mathbb{Z}$ satisfying $m^{89} \leq n < (m+1)^{89}$.
So
\begin{align*}
  0 < \sigma_n
  &= \frac{1}{n+1}\sum_{k=0}^{n} s_k \\
  &\leq \frac{1}{m^{89}+1}\sum_{k=0}^{n} s_k \\
  &= \frac{1}{m^{89}+1}
    \left( \sum_{k=0}^{n} \frac{1}{n!} + \sum_{k=0}^{m} k^{63} \right) \\
  &\leq \frac{1}{m^{89}+1}
    \left( \sum_{k=0}^{\infty} \frac{1}{n!} + \sum_{k=0}^{m} m^{63} \right) \\
  &= \frac{e + m \cdot m^{63}}{m^{89}+1} \\
  &= \frac{m^{64} + e}{m^{89}+1}.
\end{align*}
Let $n \to \infty$, then $m \to \infty$ and thus $\lim \sigma_n = 0$.
\end{enumerate}
$\Box$ \\



\emph{Proof of (d).}
\begin{enumerate}
\item[(1)]
  \begin{align*}
  \frac{1}{n+1}\sum_{k=1}^{n}ka_k
  &= \frac{1}{n+1}\sum_{k=1}^{n}k(s_k - s_{k-1}) \\
  &= \frac{1}{n+1}\left( \sum_{k=1}^{n}ks_k - \sum_{k=1}^{n}ks_{k-1} \right) \\
  &= \frac{1}{n+1}\left( \sum_{k=1}^{n}ks_k
    - \sum_{k=1}^{n}(k-1)s_{k-1} - \sum_{k=1}^{n}s_{k-1} \right) \\
  &= \frac{1}{n+1}\left( ns_n - \sum_{k=1}^{n}s_{k-1} \right) \\
  &= \frac{1}{n+1}\left( (n+1)s_n - \sum_{k=1}^{n+1}s_{k-1} \right) \\
  &= s_n - \sigma_n.
  \end{align*}
\item[(2)]
Write
$$s_n = \sigma_n + \frac{1}{n+1}\sum_{k=1}^{n}ka_k.$$
Since $\lim_{n \to \infty} (na_n) = 0$,
$\lim_{n \to \infty} \frac{1}{n+1}\sum_{k=1}^{n}ka_k = 0$ ((a)).
Since $\{\sigma_n\}$ converges,
$$\lim_{n \to \infty} s_n
= \lim_{n \to \infty} \sigma_n + \lim_{n \to \infty} \frac{1}{n+1}\sum_{k=1}^{n}ka_k
= \lim_{n \to \infty} \sigma_n$$
(Theorem 3.3(a)).
\end{enumerate}
$\Box$ \\



\emph{Proof of (e).}
\begin{enumerate}
\item[(1)]
If $m < n$, then
  \begin{align*}
  \sigma_n - \sigma_m
  &= \frac{1}{n+1}\sum_{k=0}^{n}s_{k} - \frac{1}{m+1}\sum_{k=0}^{m}s_{k} \\
  &= \frac{1}{n+1}\sum_{k=0}^{n}s_{k} - \frac{1}{m+1}\sum_{k=0}^{n}s_{k}
    + \frac{1}{m+1}\sum_{i=m+1}^{n}s_i \\
  &= \frac{m-n}{(m+1)(n+1)}\sum_{k=0}^{n}s_{k} + \frac{1}{m+1}\sum_{i=m+1}^{n}s_i \\
  &= \frac{m-n}{m+1}\sigma_n + \frac{1}{m+1}\sum_{i=m+1}^{n}s_i, \\
  \frac{m+1}{n-m}(\sigma_n - \sigma_m)
  &= -\sigma_n + \frac{1}{n-m}\sum_{i=m+1}^{n}s_i \\
  &= -\sigma_n - \frac{1}{n-m}\sum_{i=m+1}^{n}(-s_i) \\
  &= -\sigma_n - \left( \frac{1}{n-m}\sum_{i=m+1}^{n}(s_n - s_i) \right) + s_n, \\
  s_n - \sigma_n
  &= \frac{m+1}{n-m}(\sigma_n - \sigma_m) + \frac{1}{n-m}\sum_{i=m+1}^{n}(s_n - s_i).
  \end{align*}
\item[(2)]
For these $i$,
  \begin{align*}
  |s_n - s_i|
  &= \abs{ \sum_{k=i+1}^{n} a_k }
    &(s_n - s_i = \sum_{k=i+1}^{n} a_k) \\
  &\leq \sum_{k=i+1}^{n} |a_k|
    &\text{(Triangle inequality)} \\
  &< \sum_{k=i+1}^{n} \frac{M}{k}
    &(|ka_k| < M) \\
  &\leq \sum_{k=i+1}^{n} \frac{M}{i+1}
    &(k \geq i+1) \\
  &= \frac{(n-i)M}{i+1} \\
  &= \left( \frac{n-1}{i+1} - 1 \right) M \\
  &\leq \left( \frac{n-1}{m+2} - 1 \right) M
    &(i \geq m+1) \\
  &= \frac{(n-m-1)M}{m+2}.
  \end{align*}
\item[(3)]
Fix $1 > \varepsilon > 0$ and associate with each $n$ the integer $m$ that satisfies
$$m \leq \frac{n-\varepsilon}{1+\varepsilon} < m+1.$$
Clearly, $m \leq \frac{n-\varepsilon}{1+\varepsilon} < \frac{n}{1+\varepsilon} < n$.
Then
$$\frac{m+1}{n-m} \leq \frac{1}{\varepsilon}
\text{ and }
\frac{n-m-1}{m+2} < \varepsilon.$$
Hence $|s_n - s_i| < M\varepsilon$ by (2).
\item[(4)]
By (1)(3),
  \begin{align*}
  s_n - \sigma
  &=
  (\sigma_n - \sigma)
    + \frac{m+1}{n-m}(\sigma_n - \sigma_m)
    + \frac{1}{n-m}\sum_{i=m+1}^{n}(s_n - s_i), \\
  |s_n - \sigma|
  &\leq
  |\sigma_n - \sigma|
    + \frac{m+1}{n-m}|\sigma_n - \sigma_m|
    + \frac{1}{n-m}\sum_{i=m+1}^{n}|s_n - s_i| \\
  &<
  |\sigma_n - \sigma|
    + \frac{1}{\varepsilon} |\sigma_n - \sigma_m|
    + \frac{1}{n-m}\sum_{i=m+1}^{n} M\varepsilon \\
  &=
  |\sigma_n - \sigma|
    + \frac{1}{\varepsilon} |\sigma_n - \sigma_m|
    + M\varepsilon \\
  \end{align*}
holds for any $n$ and $m$ satisfying
$m \leq \frac{n-\varepsilon}{1+\varepsilon} < m+1.$
Since $\{ \sigma_n \}$ converges,
there is an integer $N$ such that
$$|\sigma_n - \sigma_m| < \varepsilon^2 \text{ whenever } m,n \geq N,$$
$$|\sigma_n - \sigma| < \varepsilon \text{ whenever } n \geq N.$$
So,
$$|s_n - \sigma| < (M+2) \varepsilon$$
holds for any $n \geq 2N+3$ (and the corresponding $m$
satisfying $m \leq \frac{n-\varepsilon}{1+\varepsilon} < m+1$
(which implies $m > \frac{n-\varepsilon}{1+\varepsilon} - 1
\geq \frac{n-1}{2} - 1 \geq N$)).
Take limit to get
$$\limsup_{n \to \infty} |s_n - \sigma| \leq (M+2) \varepsilon.$$
Since $\varepsilon$ was arbitrary, $\lim s_n = \sigma$.
\end{enumerate}
$\Box$ \\\\



%%%%%%%%%%%%%%%%%%%%%%%%%%%%%%%%%%%%%%%%%%%%%%%%%%%%%%%%%%%%%%%%%%%%%%%%%%%%%%%%



\textbf{Exercise 3.20.}
\emph{Suppose $\{p_n\}$ is a Cauchy sequence in a metric space $X$,
and some subsequence $\{p_{n_i}\}$ converges to a point $p \in X$.
Prove that the full sequence $\{p_n\}$ converges to $p$. } \\

\emph{Proof.}
Given any $\varepsilon > 0$.
\begin{enumerate}
\item[(1)]
Since $\{p_n\}$ is a Cauchy sequence, there exists a positive integer $N_1$ such that
$$d(p_n,p_m) < \frac{\varepsilon}{2} \text{ whenever } n, m \geq N_1.$$
\item[(2)]
Since the subsequence $\{p_{n_i}\}$ converges to a point $p \in X$,
there exists a positive integer $N_2$ such that
$$d(p_{n_i},p) < \frac{\varepsilon}{2} \text{ whenever } n_i \geq N_2.$$
\item[(3)]
Let $N = \max\{N_1, N_2\}$ be a positive integer.
So
\begin{align*}
d(p_n,p)
&\leq d(p_n,p_{n_i}) + d(p_{n_i}, p)
  &\text{(Definition 2.15(c))} \\
&< \frac{\varepsilon}{2} + \frac{\varepsilon}{2} \text{ whenever } n, n_i \geq N
  &\text{((1)(2))} \\
&= \varepsilon \text{ whenever } n \geq N.
\end{align*}
Hence the full sequence $\{p_n\}$ converges to $p$.
\end{enumerate}
$\Box$ \\\\



%%%%%%%%%%%%%%%%%%%%%%%%%%%%%%%%%%%%%%%%%%%%%%%%%%%%%%%%%%%%%%%%%%%%%%%%%%%%%%%%



\textbf{Exercise 3.21.}
\emph{Prove the following analogue of Theorem 3.10(b):
If $\{E_n\}$ is a sequence of closed and bounded sets in a complete metric space $X$,
if $E_n \supseteq E_{n+1}$, and if
$$\lim_{n \to \infty} \mathrm{diam}(E_n) = 0,$$
then $\bigcap_{n=1}^{\infty} E_n$ consists of exactly one point.} \\

Assume $E_n \neq \varnothing$. It is unnecessary to assume that $E_n$ is bounded
since we have the condition that $\lim_{n \to \infty} \mathrm{diam}(E_n) = 0$.\\

\emph{Note.}
Every compact metric space is complete, but complete spaces need not be compact.
In fact, a metric space is compact if and only if it is complete and totally bounded. \\

\emph{Proof.}
\begin{enumerate}
\item[(1)]
Pick $p_n \in E_n$ for $n = 1, 2, \ldots$.
\item[(2)]
\emph{Show that $\{p_n\}$ is a Cauchy sequence.}
Given any $\varepsilon > 0$.
There is a positive integer $N$ such that
$\mathrm{diam}(E_n) < \varepsilon$ whenever $n \geq N$.
Especially, $$\mathrm{diam}(E_N) < \varepsilon.$$
As $m, n \geq N$, $p_m \in E_m \subseteq E_N$ and $p_n \in E_n \subseteq E_N$.
By the definition of the diameter of $E_N$,
$$d(p_m,p_n) \leq \mathrm{diam}(E_N) < \varepsilon \text{ whenever } m,n \geq N.$$
\item[(3)]
Since $X$ is complete, $\{p_n\}$ converges to a point $p \in X$.
\item[(4)]
\emph{Show that $p \in \bigcap_{n=1}^{\infty} E_n$.}
(Reductio ad absurdum)
If there were some $n$ such that $p \not\in E_{n}$.
Consider the subsequence
$$p_{n}, p_{n+1}, p_{n+2}, \ldots.$$
Note that all $p_{n}, p_{n+1}, \ldots$ are in $E_n$.
By (3), it converges to $p$. Thus $p$ is a limit point of $E_n$.
Since $E_n$ is closed, $p \in E_n$, which is absurd.
\item[(5)]
\emph{Show that $\bigcap_{n=1}^{\infty} E_n = \{p\}$.}
(Reductio ad absurdum)
If there were $q \in \bigcap_{n=1}^{\infty} E_n$ with $q \neq p$,
then $d(p,q) > 0$ (Definition 2.15(a)).
It implies that
$$\mathrm{diam}(E_n)
\geq d(p,q) > 0 \text{ for all } n,$$
contrary to $\lim_{n \to \infty} \mathrm{diam}(E_n) = 0$.
\end{enumerate}
$\Box$ \\\\



%%%%%%%%%%%%%%%%%%%%%%%%%%%%%%%%%%%%%%%%%%%%%%%%%%%%%%%%%%%%%%%%%%%%%%%%%%%%%%%%



\textbf{Exercise 3.22 (Baire category theorem).}
\emph{Suppose $X$ is a complete metric space,
and $\{G_n\}$ is a sequence of dense open subsets of $X$.
Prove Baire's theorem, namely, that $\bigcap^\infty_1{G_n}$ is not empty.
(In fact, it is dense in $X$.)
(Hint: Find a shrinking sequence of neighborhoods $E_n$ such
that $\overline{E_n} \subseteq G_n$, and apply Exercise 3.21.) } \\

\emph{Proof.}
Given any open set $G_0$ in $X$,
will show that $$\bigcap_{n=0}^{\infty} G_n \neq \varnothing.$$
\begin{enumerate}
\item[(1)]
Since $G_1$ is dense, $G_0 \cap G_1$ is nonempty.
Take any one point $p_1$ in the open set $G_0 \cap G_1$,
then there exists a closed neighborhood
$$V_1
= \{ q \in X : d(q,p_1) < r_1 \}$$
of $p_1$ with $r_1 < 1$
such that
$$V_1 \subseteq G_0 \cap G_1.$$
Take $U_1 \subseteq E_1 \subseteq V_1$
such that
\begin{align*}
E_1 &= \left\{ q \in X : d(q,p_1) \leq \frac{r_1}{64} \right\} \subseteq V_1, \\
U_1 &= \left\{ q \in X : d(q,p_1) < \frac{r_1}{89} \right\} \subseteq E_1.
\end{align*}
\item[(2)]
Suppose $V_n, E_n, U_n$ have been constructed,
take any one point $p_{n+1}$ in the open set $U_n \cap G_{n+1}$,
there exists an open neighborhood
$$V_{n+1}
= \{ q \in X : d(q,p_{n+1}) < r_{n+1} \}$$
of $p_{n+1}$ with $r_{n+1}$ with $r_{n+1} < \frac{1}{n+1}$
such that
$$V_{n+1} \subseteq U_n \cap G_{n+1}.$$
Take $U_1 \subseteq E_1 \subseteq V_1$
such that
\begin{align*}
E_{n+1} &= \left\{ q \in X : d(q,p_{n+1}) \leq \frac{r_{n+1}}{64} \right\} \subseteq V_{n+1}, \\
U_{n+1} &= \left\{ q \in X : d(q,p_{n+1}) < \frac{r_{n+1}}{89} \right\} \subseteq E_{n+1}.
\end{align*}
\item[(3)]
Note that
  \begin{enumerate}
  \item[(a)]
  $E_n$ is closed and nonempty (since $p_n \in E_n$).
  \item[(b)]
  $\lim_{n \to \infty} \mathrm{diam}(E_n) = 0$
  (since $\mathrm{diam}(E_n) \leq 2 \cdot \frac{r_n}{64} < r_n < \frac{1}{n}$.)
  \item[(c)]
  $E_1 \supseteq E_2 \supseteq \cdots$
  (since
  $E_{n+1} \subseteq V_{n+1} \subseteq U_n \cap G_{n+1} \subseteq U_n \subseteq E_n$).
  \end{enumerate}
Since $X$ is complete, by Exercise 3.21,
$$\bigcap_{n=1}^{\infty} E_n = \{p\}$$
for some $p \in X$.
\item[(4)]
Hence
\begin{align*}
p \in \bigcap_{n=1}^{\infty} E_n
&\Longleftrightarrow
p \in E_n \text{ for all } n=1,2,3,\ldots \\
&\Longrightarrow
p \in E_1 \subseteq G_0 \cap G_1 \text{ and }
p \in E_{n+1} \subseteq U_n \cap G_{n+1} \subseteq G_{n+1} \\
&\Longrightarrow
p \in G_0 \cap G_1 \cap \cdots = \bigcap_{n=0}^{\infty} G_n \\
&\Longrightarrow
\bigcap_{n=0}^{\infty} G_n \neq \varnothing.
\end{align*}
\end{enumerate}
$\Box$ \\\\



%%%%%%%%%%%%%%%%%%%%%%%%%%%%%%%%%%%%%%%%%%%%%%%%%%%%%%%%%%%%%%%%%%%%%%%%%%%%%%%%



\textbf{Exercise 3.23.}
\emph{Suppose $\{p_n\}$ and $\{q_n\}$ are Cauchy sequences in a metric space $X$.
Show that the sequence $\{d(p_n,q_n)\}$ converges.
(Hint: For any $m, n$,
$$d(p_n,q_n) \leq d(p_n,p_m) + d(p_m,q_m) + d(q_m,q_n);$$
it follows that
$$|d(p_n,q_n) - d(p_m,q_m)|$$
is small if $m$ and $n$ are large.)} \\

\emph{Proof.}
Given any $\varepsilon > 0$.
\begin{enumerate}
\item[(1)]
Since $\{p_n\}$ and $\{q_n\}$ are Cauchy sequences,
there exists $N$ such that
$$d(p_n,p_m) < \frac{\varepsilon}{2} \text{ and }
d(q_m,q_n) < \frac{\varepsilon}{2}$$ whenever $m, n \geq N$.
\item[(2)]
Note that
$$d(p_n,q_n) \leq d(p_n,p_m) + d(p_m,q_m) + d(q_m,q_n).$$
It follows that
$$|d(p_n,q_n) - d(p_m,q_m)|
\leq d(p_n,p_m) + d(q_m,q_n)
< \frac{\varepsilon}{2} + \frac{\varepsilon}{2}
= \varepsilon.$$
Thus $\{d(p_n,q_n)\}$ is a Cauchy sequence in $\mathbb{R}^1$ (not in $X$).
\item[(3)]
Since $\mathbb{R}^1$ is a complete metric space, $\{d(p_n,q_n)\}$ converges.
\end{enumerate}
$\Box$ \\\\



%%%%%%%%%%%%%%%%%%%%%%%%%%%%%%%%%%%%%%%%%%%%%%%%%%%%%%%%%%%%%%%%%%%%%%%%%%%%%%%%



\textbf{Exercise 3.24.}
\emph{Let $X$ be a metric space.}
\begin{enumerate}
\item[(a)]
\emph{Call two Cauchy sequences $\{p_n\}$, $\{q_n\}$ in $X$ equivalent if
$$\lim_{n \to \infty}{d(p_n,q_n)} = 0.$$
Prove that this is an equivalence relation.}
\item[(b)]
\emph{Let $X^*$ be the set of all equivalence classes so obtained.
If $P \in X^*$, $Q \in X^*$, $\{p_n\} \in P$, $\{q_n\} \in Q$, define
$$\Delta(P,Q) = \lim_{n \to \infty} d(p_n,q_n);$$
by Exercise 3.23, this limit exists.
Show that the number
$\Delta(P,Q)$ is unchanged if $\{p_n\}$ and $\{q_n\}$ are replaced by equivalent sequences,
and hence that $\Delta$ is a distance function in $X^*$.}
\item[(c)]
\emph{Prove that the resulting metric space $X^*$ is complete.}
\item[(d)]
\emph{For each $p \in X$, there is a Cauchy sequence all of whose terms are $p$;
let $P_p$ be the element of $X^*$ which contains this sequence.
Prove that
$$\Delta(P_p,P_q) = d(p,q)$$
for all $p,q \in X$.
In other words, the mapping $\varphi$ defined by $\varphi(p) = P_p$
is an isometry
(i.e., a distance-preserving mapping) of $X$ into $X^*$.}
\item[(e)]
\emph{Prove that $\varphi(X)$ is dense in $X^*$, and that $\varphi(X) = X^*$ if $X$ is complete.
By (d), we may identify $X$ and $\varphi(X)$
and thus regard $X$ as embedded in the complete metric space $X^*$.
We call $X^*$ the \textbf{completion} of $X$.} \\
\end{enumerate}



\emph{Proof of (a).}
Given Cauchy sequences $\{p_n\}$, $\{q_n\}$, $\{r_n\}$ in $X$.
\begin{enumerate}
\item[(1)]
\emph{(Reflexivity)}
$$\lim_{n \to \infty}{d(p_n,q_n)} = \lim_{n \to \infty} 0 = 0$$
by the reflexivity of the metric function $d$.
\item[(2)]
\emph{(Symmetry)}
$$\lim_{n \to \infty}{d(p_n,q_n)} = \lim_{n \to \infty}{d(q_n,p_n)} = 0$$
by the symmetry of the metric function $d$.
\item[(3)]
\emph{(Transitivity)}
Suppose that $\lim_{n \to \infty}{d(p_n,q_n)} = \lim_{n \to \infty}{d(q_n,r_n)} = 0$.
By the triangle inequality of the metric function $d$,
we have
$$0 \leq d(p_n,r_n) \leq d(p_n,q_n)+d(q_n,r_n).$$
Take limit to get
\begin{align*}
0
&\leq \lim_{n \to \infty}d(p_n,r_n) \\
&\leq \lim_{n \to \infty}(d(p_n,q_n)+d(q_n,r_n)) \\
&= \lim_{n \to \infty}d(p_n,q_n) + \lim_{n \to \infty}d(q_n,r_n) \\
&= 0
\end{align*}
or $\lim_{n \to \infty}d(p_n,r_n) = 0$.
\end{enumerate}
$\Box$ \\



\emph{Proof of (b).}
\begin{enumerate}
\item[(1)]
\emph{Show that $\Delta$ is well-defined.}
Given any $\{p_n\}, \{p'_n\} \in P$ and $\{q_n\}, \{q'_n\} \in Q$.
  \begin{enumerate}
  \item[(a)]
  $\lim_{n \to \infty}d(p_n,p'_n) = 0$
  since $\{p_n\}$ and $\{p'_n\}$ are in the same equivalence class.
  \item[(b)]
  $\lim_{n \to \infty}d(q_n,q'_n) = 0$ (similar to (a)).
  \item[(c)]
  \emph{Show that $\lim_{n \to \infty}d(p_n,q_n) \leq \lim_{n \to \infty}d(p'_n,q'_n)$.}
  Since $d(p_n,q_n) \leq d(p_n,p'_n)+d(p'_n,q'_n)+d(q'_n,q_n)$,
  take limit to get
  \begin{align*}
    \lim_{n \to \infty} d(p_n,q_n)
    &\leq \lim_{n \to \infty}(d(p_n,p'_n)+d(p'_n,q'_n)+d(q'_n,q_n)) \\
    &= \lim_{n \to \infty}d(p_n,p'_n)
      + \lim_{n \to \infty}d(p'_n,q'_n)
      + \lim_{n \to \infty}d(q'_n,q_n) \\
    &= 0 + \lim_{n \to \infty}d(p'_n,q'_n) + 0 \\
    &= \lim_{n \to \infty}d(p'_n,q'_n)
  \end{align*}
  since (a)(b).
  \item[(d)]
  \emph{Show that $\lim_{n \to \infty}d(p_n,q_n) \geq \lim_{n \to \infty}d(p'_n,q'_n)$.}
  Similar to (c).
  \end{enumerate}
By (c)(d), $\lim_{n \to \infty}d(p_n,q_n) = \lim_{n \to \infty}d(p'_n,q'_n)$,
or $\Delta(P,Q)$ is well-defined.
\item[(2)]
\emph{Show that $\Delta$ is a metric.}
  \begin{enumerate}
  \item[(a)]
  \emph{Show that $\Delta(P,Q) > 0$ if $P \neq Q$; $\Delta(P,P) = 0$.}
  It is the definition of $\Delta$.
  \item[(b)]
  \emph{Show that $\Delta(P,Q) = \Delta(Q,P)$.}
  Similar to the argument in (a)(2).
  \item[(c)]
  \emph{Show that $\Delta(P,Q) \leq \Delta(P,R) + \Delta(R,Q)$.}
  Similar to the argument in (a)(3).
  \end{enumerate}
\end{enumerate}
$\Box$ \\



\emph{Proof of (c).}
\emph{Show that $\{P_k\}_{k=1}^{\infty}$ converges to $P$ in $(X^*, \Delta)$
for any given Cauchy sequence $\{P_k\}$.}
\begin{enumerate}
\item[(1)]
Take a Cauchy sequence $\{ p^{(k)}_n \}_{n=1}^{\infty}$ to represent $P_k$ for each $k$.
\emph{We will construct a Cauchy sequence $\{ p_k \}$ in $(X,d)$ such that
$\{P_k\}$ converges to $P$ which is the equivalent class of $\{ p_k \}$.}
\item[(2)]
For each $k$,
there exists $N_k$ such that
$$d\left(p^{(k)}_m,p^{(k)}_n\right) < \frac{1}{k} \text{ whenever } m,n \geq N_k.$$
Especially,
$$d\left(p^{(k)}_m,p^{(k)}_{N_k}\right) < \frac{1}{k} \text{ whenever } m \geq N_k.$$
Let $p_k = p^{(k)}_{N_k}$ and collect all $p_k$ as $\{ p_k \}_{k=1}^{\infty}$.
\item[(3)]
\emph{Show that $\{p_k\}$ is a Cauchy sequence in $(X,d)$.}
Note that for any $k$, we have
  \begin{align*}
    d(p_m,p_n)
    &= d\left(p^{(m)}_{N_m},p^{(n)}_{N_n}\right) \\
    &\leq d\left(p^{(m)}_{N_m},p^{(m)}_k\right)
      + d\left(p^{(m)}_k,p^{(n)}_k\right)
      + d\left(p^{(n)}_k,p^{(n)}_{N_n}\right).
  \end{align*}
Let $k \to \infty$, we have
  \begin{align*}
    d(p_m,p_n)
    &\leq \limsup_{k \to \infty}\left[ d\left(p^{(m)}_{N_m},p^{(m)}_k\right)
      + d\left(p^{(m)}_k,p^{(n)}_k\right)
      + d\left(p^{(n)}_k,p^{(n)}_{N_n}\right) \right] \\
    &\leq \frac{1}{m} + \Delta(P_m,P_n) + \frac{1}{n}
  \end{align*}
for any $m, n$ (by (2)).
Let $m, n \to \infty$, we establish the result (since $\{P_k\}$ is Cauchy).
\item[(4)]
\emph{Show that $\{P_k\}$ converges to $P \ni \{ p_k \}$.}
Given any $\varepsilon > 0$.
Since $\{p_k\}$ is Cauchy (3), there is $N > \frac{2}{\varepsilon}$ such that
$$d(p_m,p_n) < \frac{\varepsilon}{2} \text{ whenever } m,n \geq N.$$
Note that
  \begin{align*}
    d\left(p^{(k)}_n,p_n\right)
    &= d\left(p^{(k)}_n,p^{(n)}_{N_n}\right) \\
    &\leq d\left(p^{(k)}_n,p^{(k)}_{N_k}\right)
      + d\left(p^{(k)}_{N_k},p^{(n)}_{N_n}\right).
  \end{align*}
For any $k \geq N$, let $n \to \infty$ to get
  \begin{align*}
    \Delta(P_k,P)
    &= \lim_{n \to \infty} d\left(p^{(k)}_n,p_n\right) \\
    &\leq \limsup_{n \to \infty} d\left(p^{(k)}_n,p^{(k)}_{N_k}\right)
      + \limsup_{n \to \infty} d\left(p^{(k)}_{N_k},p^{(n)}_{N_n}\right) \\
    &< \frac{1}{k} + \frac{\varepsilon}{2} \\
    &\leq \frac{1}{N} + \frac{\varepsilon}{2} \\
    &< \frac{\varepsilon}{2} + \frac{\varepsilon}{2} \\
    &< \varepsilon.
  \end{align*}
\end{enumerate}
Hence, $(X^*, \Delta)$ is complete.
$\Box$ \\



\emph{Proof of (d).}
\begin{enumerate}
\item[(1)]
Define $\{p_n\}$ by $p_n = p$ $(n = 1,2,\ldots)$ for any $p \in X$.
\item[(2)]
\emph{Show that $\{p_n\}$ is a Cauchy sequence.}
$d(p_m,p_n) = d(p,p) = 0$.
\item[(3)]
Take $\{p\} \in P_p$ and $\{q\} \in P_q$.
Then
$$\Delta(P_p,P_q)
= \lim_{n \to \infty} d(p_n,q_n)
= \lim_{n \to \infty} d(p,q)
= d(p,q).$$
\end{enumerate}
$\Box$ \\



\emph{Proof of (e).}
\begin{enumerate}
\item[(1)]
\emph{Show that $\varphi(X)$ is dense in $X^*$.}
Given any $P \in X^*$, any $\{p_n\} \in P$ and any $\varepsilon > 0$.
Since $\{p_n\}$ is Cauchy, there is $N$ such that
$$d(p_m,p_n) < \frac{\varepsilon}{64} \text{ whenever } m,n \geq N.$$
Note that $p_N \in X$.
Pick $\{p_N\} \in P_{p_N} = \varphi(p_N) \in \varphi(X)$.
So
$$\Delta(P,P_{p_N})
= \lim_{n \to \infty} d(p_n,p_N)
\leq \frac{\varepsilon}{64}
< \varepsilon.$$
Hence $\varphi(X)$ is dense in $X^*$.
\item[(2)]
\emph{Show that $\varphi(X) = X^*$ if $X$ is complete.}
Given any $P \in X^* \ni \{p_n\}$.
Since $X$ is complete, a Cauchy sequence $\{p_n\}$ converges to $p \in X$.
Pick $\{p\} \in P_p = \varphi(p) \in \varphi(X)$.
So
$$\Delta(P,P_p)
= \lim_{n \to \infty} d(p_n,p)
= 0,$$
or $P = P_p$, or $\varphi(X) = X^*$.
\end{enumerate}
$\Box$ \\\\



%%%%%%%%%%%%%%%%%%%%%%%%%%%%%%%%%%%%%%%%%%%%%%%%%%%%%%%%%%%%%%%%%%%%%%%%%%%%%%%%



\textbf{Exercise 3.25.}
\emph{Let $X$ be the metric space whose points are rational numbers,
with the metric $d(x,y) = |x-y|$.
What is the completion of this space? (Compare Exercise 3.24.)} \\

\emph{Proof.}
By Exercise 3.24, we can identify one completion $(X^*,\Delta)$ with $(\mathbb{R},|\cdot|)$
(Theorem 3.11(c) and Theorem 1.20(b)).
$\Box$ \\



\textbf{Supplement (Uniqueness of completion).}
\emph{Show that a completion of a metric space is unique up to isometry.} \\

\emph{Outline.}
Suppose there are two completions $\{\varphi_i, (X^*_i,d^*_i) \}$ $(i=1,2)$ of $(X,d)$.
Let $$\psi = \varphi_2 \circ \varphi_1^{-1}: \varphi_1(X) \to \varphi_2(X)$$
be an isometry from $\varphi_1(X)$ into $\varphi_2(X)$
The sets $\varphi_i(X)$ $(i=1,2)$ are dense in $X^*_i$.
So we can extend $\psi$ (continuously) to a map $\psi: X^*_1 \to X^*_2$. \\

\emph{Proof.}
\begin{enumerate}
  \item[(1)]
  Given any $P \in X^*_1$, there is a Cauchy sequence
  $\{P_{p_n}\} = \{\varphi_1(p_n)\}$ in $\varphi_1(X)$ converging to $P$.
  Define $\psi(P)$ by
  $$\psi(P) = \lim_{n \to \infty} \psi(P_{p_n}).$$
  \item[(2)]
  \emph{Show that $\psi$ is well-defined.}
  Note that
  \begin{align*}
  \Delta_2(\psi(P_{p_m}), \psi(P_{p_n}))
  &= \Delta_2(\psi(\varphi_1(p_m)), \psi(\varphi_1(p_n))) \\
  &= \Delta_2(\varphi_2(p_m), \varphi_2(p_n)) \\
  &= d(p_n,p_m)
    &\text{($\varphi_2$ is isometric)} \\
  &= \Delta_1(\varphi_1(p_m), \varphi_1(p_n))
    &\text{($\varphi_1$ is isometric)} \\
  &= \Delta_1(P_{p_m}, P_{p_n}).
  \end{align*}
  So $\{ \psi(P_{p_n}) \}$ is a Cauchy sequence in $\varphi_2(X)$
  if (and only if) $\{P_{p_n}\}$ is a Cauchy sequence in $\varphi_1(X)$.
  Since $X^*_2$ is complete, $\{ \psi(P_{p_n}) \}$ converges to $\psi(P)$.
  The limit $\psi(P)$ is uniquely determined since $\Delta_2$ is a metric function.
  \item[(3)]
  Since $\psi$ is an isometry from $\varphi_1(X)$ into $\varphi_2(X)$,
  $$\psi^{-1} = \varphi_1 \circ \varphi_2^{-1}: \varphi_2(X) \to \varphi_1(X)$$
  is an isometry from $\varphi_2(X)$ into $\varphi_1(X)$.
  Besides, $\psi^{-1} \circ \psi = 1_{\varphi_1(X)}$
  and $\psi \circ \psi^{-1} = 1_{\varphi_2(X)}$.
  \item[(4)]
  \emph{Show that $\psi$ is surjective.}
  Given any $Q \in X^*_2$, there is a Cauchy sequence
  $\{P_{q_n}\} = \{\varphi_2(q_n)\}$ in $\varphi_2(X)$ converging to $Q$.
  Define
  $$P_{p_n} = \psi^{-1}(P_{q_n}) \in \varphi_1(X).$$
  $\psi(P_{p_n}) = 1_{\varphi_2(X)}(P_{q_n}) = P_{q_n}$.
  Besides, similar to argument in (2),
  $\{P_{p_n}\}$ is a Cauchy sequence in $\varphi_1(X)$.
  Since $X^*_1$ is complete, $\{P_{p_n}\}$ converges to $P \in X^*_1$.
  It is easy to verify that $\psi(P) = Q$.
  \item[(5)]
  \emph{Show that $\psi$ is injective.}
  Given any $P \in X^*_1$ and $Q \in X^*_1$,
  there are Cauchy sequences
  $$\{P_{p_n}\} = \{\varphi_1(p_n)\} \to P \text{ and }
  \{P_{q_n}\} = \{\varphi_1(q_n)\} \to Q.$$
  So
  \begin{align*}
     \psi(P) = \psi(Q)
     &\Longrightarrow
     \lim_{n \to \infty} \psi(P_{p_n}) = \lim_{n \to \infty} \psi(P_{q_n}) \\
     &\Longrightarrow
     0 = \lim_{n \to \infty} \Delta_2(\psi(P_{p_n}),\psi(P_{q_n})) \\
     &\Longrightarrow
     0 = \lim_{n \to \infty} \Delta_2(\psi(\varphi_1(p_n)),\psi(\varphi_1(q_n))) \\
     &\Longrightarrow
     0 = \lim_{n \to \infty} \Delta_2(\varphi_2(p_n),\varphi_2(q_n)) \\
     &\Longrightarrow
     0 = \lim_{n \to \infty} d(p_n,q_n).
       &\text{($\varphi_2$ is isometric)}
  \end{align*}
  Thus $\{p_n\} \in P$ and $\{q_n\} \in Q$ in the same equivalence class.
  Thus $P = Q$.
\end{enumerate}
$\Box$ \\\\



%%%%%%%%%%%%%%%%%%%%%%%%%%%%%%%%%%%%%%%%%%%%%%%%%%%%%%%%%%%%%%%%%%%%%%%%%%%%%%%%
%%%%%%%%%%%%%%%%%%%%%%%%%%%%%%%%%%%%%%%%%%%%%%%%%%%%%%%%%%%%%%%%%%%%%%%%%%%%%%%%



\end{document}