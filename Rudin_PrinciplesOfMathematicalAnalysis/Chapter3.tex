\documentclass{article}
\usepackage{amsfonts}
\usepackage{amsmath}
\usepackage{amssymb}
\usepackage{hyperref}
\usepackage[none]{hyphenat}
\usepackage{mathrsfs}
\usepackage{physics}
\parindent=0pt

\def\upint{\mathchoice%
    {\mkern13mu\overline{\vphantom{\intop}\mkern7mu}\mkern-20mu}%
    {\mkern7mu\overline{\vphantom{\intop}\mkern7mu}\mkern-14mu}%
    {\mkern7mu\overline{\vphantom{\intop}\mkern7mu}\mkern-14mu}%
    {\mkern7mu\overline{\vphantom{\intop}\mkern7mu}\mkern-14mu}%
  \int}
\def\lowint{\mkern3mu\underline{\vphantom{\intop}\mkern7mu}\mkern-10mu\int}

\begin{document}

\textbf{\Large Chapter 3: Numerical Sequences and Series} \\\\



\emph{Author: Meng-Gen Tsai} \\
\emph{Email: plover@gmail.com} \\\\



\textbf{Exercise 3.1.}
\emph{Prove that the convergence of $\{s_n\}$ implies
convergence of $\{\abs{s_n}\}$.  Is the converse true?} \\

\emph{Proof.}
\begin{enumerate}
\item[(1)]
Since $\{s_n\}$ is convergent, there is $s \in \mathbb{R}^1$
with the following property:
given any $\varepsilon > 0$, there is $N$ such that
$\abs{s_n - s} < \varepsilon$ whenever $n \geq N$.
So
$$\abs{\abs{s_n}-\abs{s}} \leq \abs{s_n - s}  < \varepsilon$$
(Exercise 1.13). That is,
$\{\abs{s_n}\}$ converges to $\abs{s}$.
\item[(2)]
The converse is not true by considering $s_n = (-1)^{n+1}$.
\end{enumerate}
$\Box$ \\\\



%%%%%%%%%%%%%%%%%%%%%%%%%%%%%%%%%%%%%%%%%%%%%%%%%%%%%%%%%%%%%%%%%%%%%%%%%%%%%%%%



\textbf{Exercise 3.2}
\emph{Calculate $\lim_{n \to \infty}(\sqrt{n^2 + n} - n)$.} \\

\emph{Proof.}
$$\sqrt{n^2 + n} - n
= \frac{n}{\sqrt{n^2+n} + n}
= \frac{1}{\sqrt{1+\frac{1}{n}} + 1} \to \frac{1}{1+1} = \frac{1}{2}$$
as $n \to \infty$.
$\Box$ \\

\emph{Proof ($\varepsilon-N$ argument).}
Let $s_n = \sqrt{n^2 + n} - n$.
\emph{Show that the sequence $\{s_n\}$ converges to $s = \frac{1}{2}$.}
Given any $\varepsilon > 0$, there is $N > \frac{1}{\varepsilon}$ such that
\begin{align*}
\abs{s_n - s}
&= \abs{(\sqrt{n^2 + n} - n) - \frac{1}{2}}
= \abs{\frac{1}{\sqrt{1+\frac{1}{n}} + 1} - \frac{1}{2}} \\
&= \abs{
\frac{2-\left(\sqrt{1+\frac{1}{n}} + 1\right)}
{2\left(\sqrt{1+\frac{1}{n}} + 1\right)}
}
= \abs{
\frac{1-\sqrt{1+\frac{1}{n}}}
{2\left(\sqrt{1+\frac{1}{n}} + 1\right)}
} \\
&= \abs{
\frac{1 - \left(1- \frac{1}{n}\right)}
{2\left(\sqrt{1+\frac{1}{n}} + 1\right)^2}}
= \abs{
\frac{- \frac{1}{n}}
{2\left(\sqrt{1+\frac{1}{n}} + 1\right)^2}}
< \frac{1}{n}
\leq \frac{1}{N}
< \varepsilon
\end{align*}
wheneven $n \geq N$.
$\Box$ \\\\



%%%%%%%%%%%%%%%%%%%%%%%%%%%%%%%%%%%%%%%%%%%%%%%%%%%%%%%%%%%%%%%%%%%%%%%%%%%%%%%%



\textbf{Exercise 3.3}
\emph{If $s_1 = \sqrt{2}$ and
$$s_{n+1} = \sqrt{2+\sqrt{s_n}} \:\: (n=1,2,3,...),$$
prove that $\{s_n\}$ converges, and that $s_n < 2$ for $n=1,2,3,...$.} \\

The convergence of $\{s_n\}$
implies there is $s \in \mathbb{R}$ such that $s_n \to s$
where $s = \sqrt{2+\sqrt{s}}$ and $\sqrt{2} < s \leq 2$.
WolframAlpha shows that
$$s = \frac{1}{3}
\left(
-1 + \sqrt[3]{\frac{1}{2}(79 - 3 \sqrt{249})}
   + \sqrt[3]{\frac{1}{2}(79 + 3 \sqrt{249})}
\right).$$ \\

\emph{Proof (Theorem 3.14).}
\begin{enumerate}
\item[(1)]
\emph{Show that $\{s_n\}$ is increasing (by mathematical induction).}
  \begin{enumerate}
  \item[(a)]
  \emph{Show that $s_2 > s_1$.}
  In fact,
  $$s_2 = \sqrt{2+\sqrt{s_1}} = \sqrt{2+\sqrt{\sqrt{2}}} < \sqrt{2} = s_1.$$
  \item[(a)]
  \emph{Show that $s_{n+1} > s_{n}$ if $s_{n} > s_{n-1}$.}
  $$s_{n+1} = \sqrt{2+\sqrt{s_{n}}} > \sqrt{2+\sqrt{s_{n-1}}} = s_n.$$
  \end{enumerate}
By mathematical induction, $\{s_n\}$ is (strictly) increasing.
\item[(2)]
\emph{Show that $\{s_n\}$ is bounded (by mathematical induction).}
  \begin{enumerate}
  \item[(a)]
  \emph{Show that $s_1 \leq 2$.}
  $\sqrt{2} \leq 2$.
  \item[(a)]
  \emph{Show that $s_{n+1} \leq 2$ if $s_{n} \leq 2$.}
  $$s_{n+1} = \sqrt{2+\sqrt{s_{n}}} \leq \sqrt{2+\sqrt{2}} < 2.$$
  \end{enumerate}
By mathematical induction, $\{s_n\}$ is bounded by $2$.
\end{enumerate}
Hence, $\{s_n\}$ converges since $\{s_n\}$ is increasing and bounded (Theorem 3.14).
$\Box$ \\\\



%%%%%%%%%%%%%%%%%%%%%%%%%%%%%%%%%%%%%%%%%%%%%%%%%%%%%%%%%%%%%%%%%%%%%%%%%%%%%%%%



\textbf{Exercise 3.4}
\emph{Find the upper and lower limits of the sequences $\{s_n\}$ defined by
$$s_1 = 0; s_{2m} = \frac{s_{2m-1}}{2}; s_{2m+1} = \frac{1}{2} + s_{2m}.$$ } \\

Write out the first few terms of $\{s_n\}$:
$$0, 0, \frac{1}{2}, \frac{1}{4}, \frac{3}{4},
\frac{3}{8}, \frac{7}{8}, \frac{7}{16}, \frac{15}{16}, ...$$

It suggests us
\begin{align*}
s_{2m+1} &= 1 - \frac{1}{2^m} \:\: (m = 0, 1, 2, ...), \\
s_{2m} &= \frac{1}{2} - \frac{1}{2^m} \:\: (m = 1, 2, 3, ...). \\
\end{align*}

\emph{Proof.}
\begin{enumerate}
\item[(1)]
\emph{Show that
\begin{align*}
s_{2m+1} &= 1 - \frac{1}{2^m} \:\: (m = 0, 1, 2, ...), \\
s_{2m} &= \frac{1}{2} - \frac{1}{2^m}. \:\: (m = 1, 2, 3, ...)
\end{align*}}
Apply mathematical induction.
\item[(2)]
The upper limit is $1$.
\item[(3)]
The lower limit is $\frac{1}{2}$.
\end{enumerate}

$\Box$ \\\\



%%%%%%%%%%%%%%%%%%%%%%%%%%%%%%%%%%%%%%%%%%%%%%%%%%%%%%%%%%%%%%%%%%%%%%%%%%%%%%%%



\textbf{Exercise 3.7}
\emph{Prove that the convergence of $\sum a_n$ implies the convergence of
$$\sum \frac{\sqrt{a_n}}{n},$$
if $a_n \geq 0$.} \\

\emph{Proof (Cauchy's inequatity).}
\begin{enumerate}
\item[(1)]
\emph{Show that $\sum\frac{\sqrt{a_n}}{n}$ is bounded.}
For any $k \in \mathbb{Z}^{+}$,
\begin{align*}
\left( \sum_{n=1}^{k} \frac{\sqrt{a_n}}{n} \right)^2
\leq&
\left( \sum_{n=1}^{k}{a_n} \right)
\left( \sum_{n=1}^{k}{\frac{1}{n^2}} \right)
  &(\text{Cauchy's inequatity}) \\
\leq& \left( \sum^{\infty}_{n=1}{a_n} \right)
\left( \sum^{\infty}_{n=1}{\frac{1}{n^2}} \right).
  &\left(\text{$\sum{a_n}, \sum{\frac{1}{n^2}}$: convergent}\right)
\end{align*}
Thus,
$\left( \sum_{n=1}^{k}\frac{\sqrt{a_n}}{n} \right)^2$ is bounded,
or $\sum_{n=1}^{k}\frac{\sqrt{a_n}}{n}$ is bounded.
\item[(2)]
\emph{Show that $\sum_{n=1}^{k} \frac{\sqrt{a_n}}{n}$ is increasing.}
It is clear due to $\frac{\sqrt{a_n}}{n} \geq 0$.
\end{enumerate}
By Theorem 3.14, $\sum_{n=1}^{\infty} \frac{\sqrt{a_n}}{n}$ converges.
$\Box$ \\

\emph{Proof (AM-GM inequality).}
\emph{Show that $\sum\frac{\sqrt{a_n}}{n}$ is bounded.}
\begin{align*}
\frac{\sqrt{a_n}}{n}
\leq&
\frac{1}{2}
\left( a_n + \frac{1}{n^2} \right)
  &(\text{AM-GM inequality}) \\
\sum_{n=1}^{k} \frac{\sqrt{a_n}}{n}
\leq&
\frac{1}{2}
\left( \sum_{n=1}^{k} a_n + \sum_{n=1}^{k} \frac{1}{n^2} \right) \\
\leq&
\frac{1}{2}
\left( \sum_{n=1}^{\infty} a_n + \sum_{n=1}^{\infty} \frac{1}{n^2} \right).
  &\left(\text{$\sum{a_n}, \sum{\frac{1}{n^2}}$: convergent}\right)
\end{align*}
Thus, $\sum_{n=1}^{k}\frac{\sqrt{a_n}}{n}$ is bounded.
The rest proof is the same as previous.
$\Box$ \\\\



%%%%%%%%%%%%%%%%%%%%%%%%%%%%%%%%%%%%%%%%%%%%%%%%%%%%%%%%%%%%%%%%%%%%%%%%%%%%%%%%



\end{document}