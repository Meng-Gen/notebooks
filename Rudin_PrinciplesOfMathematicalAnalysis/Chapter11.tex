\documentclass{article}
\usepackage{amsfonts}
\usepackage{amsmath}
\usepackage{amssymb}
\usepackage{hyperref}
\usepackage[none]{hyphenat}
\usepackage{mathrsfs}
\usepackage{physics}
\parindent=0pt

\def\upint{\mathchoice%
    {\mkern13mu\overline{\vphantom{\intop}\mkern7mu}\mkern-20mu}%
    {\mkern7mu\overline{\vphantom{\intop}\mkern7mu}\mkern-14mu}%
    {\mkern7mu\overline{\vphantom{\intop}\mkern7mu}\mkern-14mu}%
    {\mkern7mu\overline{\vphantom{\intop}\mkern7mu}\mkern-14mu}%
  \int}
\def\lowint{\mkern3mu\underline{\vphantom{\intop}\mkern7mu}\mkern-10mu\int}

\begin{document}

\textbf{\Large Chapter 11: The Lebesuge Theory} \\\\



\emph{Author: Meng-Gen Tsai} \\
\emph{Email: plover@gmail.com} \\\\



% http://www-users.math.umn.edu/~safon002/Archive/Math5616H/NOTES/B.pdf



%%%%%%%%%%%%%%%%%%%%%%%%%%%%%%%%%%%%%%%%%%%%%%%%%%%%%%%%%%%%%%%%%%%%%%%%%%%%%%%%
%%%%%%%%%%%%%%%%%%%%%%%%%%%%%%%%%%%%%%%%%%%%%%%%%%%%%%%%%%%%%%%%%%%%%%%%%%%%%%%%



\textbf{Exercise 11.1.}
\emph{If $f \geq 0$ and $\int_{E} f d\mu = 0$,
prove that $f(x) = 0$ almost everywhere on $E$.
(Hint: Let $E_n$ be the subset of $E$ on which $f(x) > \frac{1}{n}$.
Write $A = \bigcup E_n$.
Then $\mu(A) = 0$ if and only if $\mu(E_n) = 0$ for every $n$.)} \\

Might assume that $f$ is measurable on $E$. \\

\emph{Proof (Hint).}
\begin{enumerate}
\item[(1)]
Define $A = \{ x \in E : f(x) > 0 \}$.
So $f(x) = 0$ almost everywhere on $E$ if and only if $\mu(A) = 0$.

\item[(2)]
Define
\[
  E_n = \left\{ x \in E : f(x) > \frac{1}{n} \right\}
\]
for $n = 1,2,3,\ldots$.
Note that $E_1 \subseteq E_2 \subseteq E_3 \subseteq \cdots$ and
\[
  A = \bigcup_{n=1}^{\infty} E_n.
\]
Since $\mu$ is a measure,
\[
  \lim_{n \to \infty} \mu(E_n) = \mu(A)
\]
(Theorem 11.3).

\item[(3)]
(Reductio ad absurdum)
If $\mu(A)> 0$, there is an integer $N$ such that
$\mu(E_n) \geq \frac{\mu(A)}{2}$ whenever $n \geq N$ (by (2)).
In particular, take $n = N$ to get
\begin{align*}
  \int_E f d\mu
  &\geq \int_{E_N} f d\mu
    &\text{($\mu$ is a measure and $E_N \subseteq E$)}\\
  &\geq \frac{1}{N} \cdot \mu(E_N)
    &\text{(Remarks 11.23(b))} \\
  &\geq \frac{1}{N} \cdot \frac{\mu(A)}{2} \\
  & > 0,
\end{align*}
contrary to the assumption that $\int_{E} f d\mu = 0$.
\end{enumerate}
$\Box$ \\

\emph{Note.}
Compare to Exercise 6.2. \\\\



%%%%%%%%%%%%%%%%%%%%%%%%%%%%%%%%%%%%%%%%%%%%%%%%%%%%%%%%%%%%%%%%%%%%%%%%%%%%%%%%



\textbf{Exercise 11.2.}
\emph{If $\int_A f d\mu = 0$ for every measurable subset $A$ of a measurable set $E$,
then $f(x) = 0$ almost everywhere on $E$.} \\

Might assume that $f$ is measurable on $E$. \\

\emph{Proof.}
\begin{enumerate}
\item[(1)]
  Define
  \[
    A = \{ x \in E : f(x) \geq 0 \}
    \qquad \text{ and } \qquad
    B = \{ x \in E : f(x) \leq 0 \}.
  \]
  $A$ and $B$ are measurable subsets of a measurable set $E$ since $f$ is measurable.

\item[(2)]
  Apply Exercise 11.1 to the fact that $f \geq 0$ on $A$ (by construction)
  and $\int_A f d\mu = 0$ (by assumption),
  we have $f(x) = 0$ almost everywhere on $A$.

\item[(3)]
  Similarly,
  apply Exercise 11.1 to the fact that $-f \geq 0$ on $B$
  and $\int_B (-f) d\mu = -\int_B f d\mu = 0$,
  we have $f(x) = 0$ almost everywhere on $B$.

\item[(4)]
  As $E = A \cup B$, $f(x) = 0$ almost everywhere on $E$ by (2)(3).
\end{enumerate}
$\Box$ \\\\



%%%%%%%%%%%%%%%%%%%%%%%%%%%%%%%%%%%%%%%%%%%%%%%%%%%%%%%%%%%%%%%%%%%%%%%%%%%%%%%%



\textbf{Exercise 11.3.}
\emph{If $\{f_n\}$ is a sequence of measurable functions,
prove that the set of points $x$ at which $\{f_n(x)\}$ converges is measurable.} \\

\emph{Proof.}
\begin{enumerate}
\item[(1)]
  It suffices to show that
  \[
    E
    = \{ x : \{ f_n(x) \} \text{ is convergent } \}
    = \{ x : \{ f_n(x) \} \text{ is Cauchy } \}
  \]
  is measurable (since $\mathbb{R}^1$ is complete).

\item[(2)]
  Write
  \[
    E
    =
    \bigcap_{k=1}^{\infty}
    \bigcup_{N=1}^{\infty}
    \bigcap_{n,m \geq N} \left\{ x : |f_n(x)-f_m(x)| \leq \frac{1}{k} \right\}
  \]
  Since $\{f_n\}$ is a sequence of measurable functions,
  $x \mapsto |f_n(x)-f_m(x)|$ is measurable
  (Theorem 11.16 and Theorem 11.18).
  Hence
  \[
    \left\{ x : |f_n(x)-f_m(x)| \leq \frac{1}{k} \right\}
  \]
  is measurable (Theorem 11.15).
  Therefore $E$ is measurable.
\end{enumerate}
$\Box$ \\\\



%%%%%%%%%%%%%%%%%%%%%%%%%%%%%%%%%%%%%%%%%%%%%%%%%%%%%%%%%%%%%%%%%%%%%%%%%%%%%%%%



\textbf{Exercise 11.4.}
\emph{If $f \in \mathscr{L}(\mu)$ on $E$ and $g$ is bounded and measurable on $E$,
then $fg \in \mathscr{L}(\mu)$ on $E$.} \\

\emph{Proof (Theorem 11.27).}
\begin{enumerate}
\item[(1)]
  $fg$ is measurable since both $f$ and $g$ are measurable (Theorem 11.18).

\item[(2)]
  $|g| \leq M$ for some real $M \in \mathbb{R}^1$ by the boundedness of $g$.
  Hence
  \[
    |fg| \leq M|f|
  \]
  on $E$.

\item[(3)]
  To apply Theorem 11.27, it suffices to show that
  $M|f| \in \mathscr{L}(\mu)$ on $E$.
  Theorem 11.26 implies that $|f| \in \mathscr{L}(\mu)$ if $f \in \mathscr{L}(\mu)$.
  And Remarks 11.23(d) implies that $M|f| \in \mathscr{L}(\mu)$ if $|f| \in \mathscr{L}(\mu)$.
\end{enumerate}
$\Box$ \\

\emph{Note} (Riemann integral).
  \emph{If $f \in \mathscr{R}$ on $[a,b]$ and $g$ is bounded and measurable on $[a,b]$,
  then $fg$ might be not Riemann integrable.} \\\\



%%%%%%%%%%%%%%%%%%%%%%%%%%%%%%%%%%%%%%%%%%%%%%%%%%%%%%%%%%%%%%%%%%%%%%%%%%%%%%%%



\textbf{Exercise 11.5.}
\emph{Put
  \begin{equation*}
    g(x) =
    \begin{cases}
      0 & (0 \leq x \leq \frac{1}{2}), \\
      1 & (\frac{1}{2} < x \leq 1),
    \end{cases}
  \end{equation*}
  and
  \begin{align*}
    f_{2k}(x) &= g(x) &(0 \leq x \leq 1), \\
    f_{2k+1}(x) &= g(1-x) &(0 \leq x \leq 1).
  \end{align*}
Show that
\[
  \liminf_{n \to \infty} f_n(x) = 0
  \qquad
  (0 \leq x \leq 1),
\]
but
\[
  \int_{0}^{1} f_n(x)dx = \frac{1}{2}.
\]
(Compare with the Fatou's theorem.)} \\

\emph{Proof.}
\begin{enumerate}
\item[(1)]
  \emph{Show that $\liminf_{n \to \infty} f_n(x) = 0$.}
  Note that
  \begin{equation*}
    g(1-x) =
    \begin{cases}
      1 & (0 \leq x < \frac{1}{2}), \\
      0 & (\frac{1}{2} < x \leq 1).
    \end{cases}
  \end{equation*}
  Since $f_n(x) \geq 0$ by definition,
  $\liminf_{n \to \infty} f_n(x) \geq 0$.
  Since $f_{2k}(0) = f_{2k+1}(1) = 0$ for all positive integers $k$,
  $\liminf_{n \to \infty} f_n(x) \leq 0$.
  Therefore the result is established.

\item[(2)]
  \emph{Show that $\int_{0}^{1} f_n(x)dx = \frac{1}{2}$.}
  Since
  \begin{align*}
    \int_{0}^{1} f_{2k}(x)dx
    &= \int_{0}^{1} g(x)dx = \frac{1}{2}, \\
    \int_{0}^{1} f_{2k+1}(x)dx
    &= \int_{0}^{1} g(1-x)dx = \frac{1}{2},
  \end{align*}
  in any case $\int_{0}^{1} f_n(x)dx = \frac{1}{2}$ for all positive integers $n$.

\item[(3)]
  This example shows that we may have the strict inequality in the Fatou's theorem.
\end{enumerate}
$\Box$ \\



\textbf{Supplement (Similar exercise).}
  \emph{Consider the sequence $\{f_n\}$ defined by $f_n(x) = 1$ if $n \leq x < n+1$,
  with $f_n(x) = 0$ otherwise.
  Show that we may have the strict inequality in the Fatou's theorem.} \\\\



%%%%%%%%%%%%%%%%%%%%%%%%%%%%%%%%%%%%%%%%%%%%%%%%%%%%%%%%%%%%%%%%%%%%%%%%%%%%%%%%



\textbf{Exercise 11.6.}
\emph{Let
\begin{equation*}
f_n(x) =
  \begin{cases}
    \frac{1}{n}
      & (|x| \leq n), \\
    0
      & (|x| > n).
  \end{cases}
\end{equation*}
Then $f_n(x) \to 0$ uniformly on $\mathbb{R}^1$,
but
\[
  \int_{-\infty}^{\infty} f_n(x) dx = 2
  \qquad
  (n = 1,2,3,\ldots).
\]
(We write $\int_{-\infty}^{\infty}$ in place of $\int_{\mathbb{R}^1}$.)
Thus uniform convergence does not imply dominated convergence
in the sense of Theorem 11.32.
However, on sets of finite measure,
uniformly convergent sequences of bounded functions do satisfy Theorem 11.32.} \\



\emph{Proof.}
\begin{enumerate}
\item[(1)]
  \emph{Show that $f_n(x) \to 0$ uniformly on $\mathbb{R}^1$.}
  Given any $\varepsilon > 0$, there is an integer $N > \frac{1}{\varepsilon}$
  such that
  \[
    |f_n(x) - 0| \leq \frac{1}{n} \leq \frac{1}{N} < \varepsilon
  \]
  whenever $n \geq N$ and $x \in \mathbb{R}^1$.
  Hence $f_n(x) \to 0$ uniformly.

\item[(2)]
  \emph{Show that $\int_{-\infty}^{\infty} f_n(x) dx = 2$.}
  \[
    \int_{-\infty}^{\infty} f_n(x) dx
    = \int_{-n}^{n} \frac{1}{n} dx
    = 2.
  \]

\item[(3)]
  By (1)(2),
  \[
    \lim_{n \to \infty} \int_{-\infty}^{\infty} f_n(x) dx
    \neq \int_{-\infty}^{\infty} \lim_{n \to \infty} f_n(x) dx
  \]
  suggests that the Lebesgue's dominated convergence theorem (Theorem 11.32)
  does not hold in this case.
  In fact,
  if there were $g \in \mathscr{L}$ such that $|f_n(x)| \leq g(x)$,
  then
  \begin{align*}
    \int_{-\infty}^{\infty} g(x) dx
    &\geq \int_{0}^{\infty} g(x) dx
      &(\text{Theorem 11.24}) \\
    &= \sum_{n=1}^{\infty} \int_{n-1}^{n} g(x) dx
      &(\text{Theorem 11.24}) \\
    &\geq \sum_{n=1}^{\infty} \int_{n-1}^{n} |f_n(x)| dx \\
    &= \sum_{n=1}^{\infty} \int_{n-1}^{n} \frac{1}{n} dx \\
    &= \sum_{n=1}^{\infty} \frac{1}{n} \\
    &= \infty,
  \end{align*}
  which is absurd.

\item[(4)]
  \emph{Show that on sets of finite measure,
  uniformly convergent sequences of bounded functions $\{f_n\}$
  do satisfy Theorem 11.32.}
  \begin{enumerate}
  \item[(a)]
    Since $\{f_n\}$ is uniformly convergent,
    $\{f_n\}$ is uniformly bounded (Exercise 7.1), or
    there exists a real number $M$ such that
    \[
      |f_n(x)| \leq M
    \]
    for all positive integer $n$ and $x \in E$.

  \item[(b)]
    Define $g(x) = M$ on $E$.
    It is clear that
    \[
      \int_E g(x) dx = M \mu(E) < +\infty.
    \]
    Now we can apply the Lebesgue's dominated convergence theorem (Theorem 11.32)
    to get
    \[
      \lim_{n \to \infty} \int_E f_n d\mu = \int_E \lim_{n \to \infty} f_n d\mu.
    \]
  \end{enumerate}
\end{enumerate}
$\Box$ \\\\



%%%%%%%%%%%%%%%%%%%%%%%%%%%%%%%%%%%%%%%%%%%%%%%%%%%%%%%%%%%%%%%%%%%%%%%%%%%%%%%%



\textbf{Exercise 11.7.}
\emph{Find a necessary and sufficient condition that $f \in \mathscr{R}(\alpha)$ on $[a,b]$.
(Hint: Consider Example 11.6(b) and Theorem 11.33.)} \\

\emph{Proof.}
\begin{enumerate}
\item[(1)]
  Defines the regular measure $\mu$ by
  \begin{align*}
    \mu([a,b)) &= \alpha(b-) - \alpha(a-) \\
    \mu([a,b]) &= \alpha(b+) - \alpha(a-) \\
    \mu((a,b]) &= \alpha(b+) - \alpha(a+) \\
    \mu((a,b)) &= \alpha(b-) - \alpha(a+)
  \end{align*}
  where $\alpha: \mathbb{R}^1 \to \mathbb{R}^1$ might be defined by
  \begin{equation*}
    \alpha(x) =
      \begin{cases}
        \alpha(a) & \text{ if $x < a$}, \\
        \alpha(x) & \text{ if $a \leq x \leq b$}, \\
        \alpha(b) & \text{ if $x > b$}.
      \end{cases}
  \end{equation*}
  (Example 11.6(b)).

\item[(2)]
  \emph{Suppose $f$ is bounded on $[a,b]$.
  Then $f \in \mathscr{R}(\alpha)$ on $[a,b]$ if and only if
  $f$ and $\alpha$ satisfy both properties (I) and (II) below.}
  \begin{enumerate}
  \item[(I)]
    \emph{$f$ and $\alpha$
    cannot be both left-discontinuous, or both right-discontinuous at same point.}
  \item[(II)]
    \emph{$f$ is continuous almost everywhere with respect to $\mu$ on $[a,b] - D_{\alpha}$
    where $D_{\alpha}$ is the set of discontinuities of $\alpha$ on $[a,b]$.}
  \end{enumerate}

\item[(3)]
  Similar to Theorem 11.33.
  By Definition 6.2 and Theorem 6.4 there is a sequence $\{P_k\}$ of partitions of $[a,b]$,
  such that $P_{k+1}$ is a refinement of $P_k$,
  such that the distance between adjacent points of $P_k$ is less than $\frac{1}{k}$,
  and such that
  \[
    \lim_{k \to \infty} L(P_k,f,\alpha) = \mathscr{R}\lowint f d\alpha,
    \qquad
    \lim_{k \to \infty} L(P_k,f,\alpha) = \mathscr{R}\upint f d\alpha.
  \]
  (In this proof, all integrals are taken over $[a,b]$.)

\item[(4)]
  If $P_k = \{ x_0, x_1, \ldots, x_n \}$, with $x_0 = a$, $x_n = b$, define
  \[
    L_k(a) = U_k(a) = f(a);
  \]
  put $U_k(x) = M_i$ and $L_k(x) = m_i$ for $x_{i-1} < x \leq x_{i}$, $1 \leq i \leq n$,
  using the notation introduced in Definition 6.1.
  Then
  \[
    L(P_k,f,\alpha) = \int L_k d\mu,
    \qquad
    U(P_k,f,\alpha) = \int U_k d\mu
  \]
  and
  \[
    L_1(x) \leq L_2(x) \leq \cdots \leq f(x) \leq \cdots \leq U_2(x) \leq U_1(x)
  \]
  for all $x \in [a,b]$, since $P_{k+1}$ refines $P_k$.

\item[(5)]
  So there exist
  \[
    L(x) = \lim_{k \to \infty} L_k(x),
    \qquad
    U(x) = \lim_{k \to \infty} U_k(x).
  \]
  Observe that $L$ and $U$ are bounded $\mu$-measurable function on $[a,b]$,
  that
  \[
    L(x) \leq f(x) \leq U(x)
    \qquad
    (a \leq x \leq b),
  \]
  and that
  \begin{align*}
    \int L d\mu &= \lim \int L_k d\mu = \lim L(P_k,f,\alpha) = \mathscr{R}\lowint f d\alpha, \\
    \int U d\mu &= \lim \int U_k d\mu = \lim U(P_k,f,\alpha) = \mathscr{R}\upint f d\alpha
  \end{align*}
  by the Lebesgue's monotone convergence theorem (Theorem 11.28).

\item[(6)]
  So $f \in \mathscr{R}(\alpha)$ on $[a,b]$
  if and only if
  $\int L d\mu = \int U d\mu$
  if and only if
  \begin{enumerate}
  \item[(III)]
    $L(x) = U(x)$ almost everywhere with respect to $\mu$
  \end{enumerate}
  by Exercise 11.1 and the fact that $U(x) - L(x) \geq 0$.

\item[(7)]
  \emph{Show that $f \in \mathscr{R}(\alpha)$ on $[a,b]$ implies the property (I).}
  It is independent of the Lebesgue theory.
  \begin{enumerate}
  \item[(a)]
    Suppose both $f$ and $\alpha$ are discontinuous from the right at $x = c$;
    that is,
    suppose that there exists an $\varepsilon > 0$ such that for every $\delta > 0$
    there are values of $x, y \in (c,c+\delta) \subseteq [a,b]$ for which
    \[
      |f(x)-f(c)| \geq \sqrt{\varepsilon},
      \qquad
      |\alpha(y)-\alpha(c)| \geq \sqrt{\varepsilon}.
    \]

  \item[(b)]
    Let $P = \{x_0 < \cdots < x_n \}$ be any partition of $[a,b]$ containing $c$,
    say $c = x_{i-1}$ for some $i = 1,\ldots,n$.
    Then
    \begin{align*}
      U(P,f,\alpha) - L(P,f,\alpha)
      &= \sum_{j=1}^{n} (M_j - m_j)(\alpha(x_j) - \alpha(x_{j-1})) \\
      &\geq (M_i - m_i)(\alpha(x_i) - \alpha(x_{i-1})).
    \end{align*}
    Take $\delta = x_i - x_{i-1}$. $x_i = x_{i-1}+\delta = c+\delta$.
    Then
    \[
      \alpha(x_i) - \alpha(x_{i-1})
      = \alpha(c+\delta) - \alpha(c)
      \geq \alpha(y) - \alpha(c)
      \geq \sqrt{\varepsilon}
    \]
    (by the monotonicity of $\alpha$).
    Besides,
    \[
      M_i - m_i \geq |f(x)-f(c)| \geq \sqrt{\varepsilon}.
    \]
    Hence,
    \[
      U(P,f,\alpha) - L(P,f,\alpha) \geq \varepsilon.
    \]
    Therefore,
    Theorem 6.6 implies that $f \not\in \mathscr{R}(\alpha)$ on $[a,b]$.

  \item[(c)]
    The argument is similar if $c$ is a common discontinuity from the left.
  \end{enumerate}

\item[(8)]
  \emph{Show that (III) implies (II).}
  \begin{enumerate}
  \item[(a)]
    \emph{Show that $f$ is continuous at $x \in [a,b] - D_{\alpha}$ if
    $U(x) = L(x)$ and $x \not\in \bigcup_{k=1}^{\infty} P_k$.}
    (Reductio ad absurdum)
    If $f$ were not continuous at $x$,
    then there exists an $\varepsilon > 0$ such that
    there is a sequence $\{ y_m \}$ in $[a,b]$
    such that $|y_m - x| < \frac{1}{m}$ but
    \[
      |f(y_m) - f(x)| > \varepsilon
    \]
    for $m = 1,2,3,\ldots$ (Theorem 4.2).

  \item[(b)]
    Given any $P_k$.
    Since $x \not \in P_k$, $x \in (x_{i-1},x_{i})$ for some $i$.
    Since $(x_{i-1},x_{i})$ is open,
    there exists an integer $N_k$ such that $y_m \in (x_{i-1},x_{i})$ whenever $m \geq N_k$.
    Hence,
    \[
      U_k(x) - L_k(x) = M_i - m_i \geq |f(y_m) - f(x)| > \varepsilon.
    \]
    Take the limit to get
    \[
      U(x) - L(x) \geq \varepsilon,
    \]
    which is absurd. Therefore, the statement in (a) is proved.

  \item[(c)]
    Now it suffices to show that both sets
    \begin{align*}
      E_1 &= \left\{ x \in [a,b] - D_{\alpha} : L(x) \neq U(x) \right\} \\
      E_2 &= \left\{ x \in [a,b] - D_{\alpha} : x \in \bigcup_{k=1}^{\infty} P_k \right\}
    \end{align*}
    are $\mu$-measure zero.
    $E_1$ is $\mu$-measure zero by (III).
    $E_2$ is $\mu$-measure zero since $E_2$ is countable and defined on $[a,b] - D_{\alpha}$.
  \end{enumerate}
  Therefore, (II) is established.

\item[(9)]
  \emph{Show that (I)(II) implies (III).}
  Use the notation in (8).
  \begin{enumerate}
  \item[(a)]
    It suffices to show that
    \begin{align*}
      &\{ x \in [a,b] : L(x) \neq U(x) \} \\
      =& \underbrace{\{ x \in [a,b] - D_{\alpha} : L(x) \neq U(x) \}}_{= E_1}
      \bigcup
        \underbrace{\{ x \in D_{\alpha} : L(x) \neq U(x) \}}_{= E_3}
    \end{align*}
    is $\mu$-measure zero.

  \item[(b)]
    Note that $E_2$ is $\mu$-measure zero.
    Hence (II) and (8)(a) imply that $E_1$ is $\mu$-measure zero.
    So it suffices to show that $E_3$ is $\mu$-measure zero.
    In fact, we will show that $E_3 = \varnothing$,
    or $L(x) = U(x)$ whenever $x \in D_{\alpha}$.

  \item[(c)]
    Write
    \[
      D_{\alpha} = \{ y_1, \ldots, y_m, \ldots \}
    \]
    since $D_{\alpha}$ is at most countable (Theorem 4.30).
    (Set $y_m = y_{m+1} = \cdots$ if $D_{\alpha}$ is finite.)
    Define a refinement of $P_k$ by
    \[
      P_k \bigcup \{ y_1, \ldots, y_k \}
    \]
    and use the same symbol $P_k$ to denote this refinement.

  \item[(d)]
    Given any $x \in D_{\alpha}$.
    Suppose $\alpha$ is discontinuous from the right at $x$.
    By the construction in (c),
    there is an integer $N_1$ such that
    $x = x_{i-1}$ is in some subinterval $[x_{i-1},x_{i}]$ of $P_k$
    whenever $k \geq N_1$.

  \item[(e)]
    Note that $f$ is continuous from the right at $x$ by (I).
    Given an $\varepsilon > 0$, there exists a $\delta > 0$ such that
    \[
      |f(y) - f(x)| \leq \frac{\varepsilon}{2}
    \]
    whenever $y \in [x,x+\delta)$.
    So
    \[
      |f(y) - f(z)|
      \leq |f(y) - f(x)| + |f(x) - f(z)|
      \leq \frac{\varepsilon}{2} + \frac{\varepsilon}{2}
      = \varepsilon
    \]
    whenever $y, z \in [x,x+\delta)$.

  \item[(f)]
    Take an integer $N_2 > \frac{1}{\delta}$ such that
    for any any subinterval $[x_{i-1},x_{i}]$ of $P_k$ we have
    $[x_{i-1},x_{i}] \subseteq [x,x+\delta)$ whenever $k \geq N_2$.

  \item[(g)]
    Take $N = \max\{ N_1, N_2 \}$.
    As $k \geq N$, $[x_{i-1},x_{i}] \subseteq [x,x+\delta)$ and
    \[
      U_k(x) - L_k(x)
      = M_i - m_i
      = \sup_{y,z \in [x_{i-1},x_{i}]} |f(y) - f(z)|
      \leq \varepsilon.
    \]
    Take the limit to get
    \[
      U(x) - L(x) \leq \varepsilon.
    \]
    Since $\varepsilon$ is arbitrary, $L(x) = U(x)$.

  \item[(h)]
    The argument is similar if $\alpha$ is discontinuous from the left at $x$.
  \end{enumerate}
\end{enumerate}
$\Box$ \\\\



%%%%%%%%%%%%%%%%%%%%%%%%%%%%%%%%%%%%%%%%%%%%%%%%%%%%%%%%%%%%%%%%%%%%%%%%%%%%%%%%



\textbf{Exercise 11.8.}
\emph{If $f \in \mathscr{R}$ on $[a,b]$ and if $F(x) = \int_{a}^{x}f(t)dt$,
prove that $F'(x) = f(x)$ almost everywhere on $[a,b]$.} \\

\emph{Proof.}
\begin{enumerate}
\item[(1)]
  Theorem 6.20 implies that
  $F'(x_0) = f(x_0)$ if $f$ is continuous at $x_0 \in [a,b]$.

\item[(2)]
  Since $f \in \mathscr{R}$ on $[a,b]$, $f$ is bounded on $[a,b]$.
  Theorem 11.33 implies that
  $f$ is continuous almost everywhere on $[a,b]$.
\end{enumerate}
By (1)(2), $F'(x) = f(x)$ almost everywhere on $[a,b]$.
$\Box$ \\\\



%%%%%%%%%%%%%%%%%%%%%%%%%%%%%%%%%%%%%%%%%%%%%%%%%%%%%%%%%%%%%%%%%%%%%%%%%%%%%%%%



\textbf{Exercise 11.9.}
\emph{Prove that the function $F$ given by
\[
  F(x) = \int_{a}^{x} f dt
  \qquad
  (a \leq x \leq b)
\]
(where $f \in \mathscr{L}$ on $[a,b]$) is continuous on $[a,b]$.} \\



\emph{Proof.}
\begin{enumerate}
\item[(1)]
  \emph{Let $f \in \mathscr{L}$ on $E$.
  Show that given any $\varepsilon > 0$ there is a $\delta > 0$ such that
  \[
    \int_{A} f d\mu < \varepsilon
  \]
  whenever $A \subseteq E$ with $\mu(A) < \delta$.}
  \begin{enumerate}
  \item[(a)]
    Define $f_n(x) = \min\{ f(x), n \}$ on $E$ for $n=1,2,3,\ldots$.
    Then $\{f_n\}$ is a sequence of measurable functions such that
    \[
      0 \leq f_1(x) \leq f_2(x) \leq \cdots.
    \]
    Also, $f_n \to f$.
    Then by the Lebesuge's monotone convergenece theorem (Theorem 11.28),
    \[
      \lim_{n \to \infty} \int_{E} f_n d\mu = \int_{E} f d\mu.
    \]

  \item[(b)]
    For such $\varepsilon > 0$, there is an integer $N \geq 1$ such that
    \[
      \int_{E} (f - f_N) d\mu < \frac{\varepsilon}{2}.
    \]
    Choose $\delta > 0$ so that $\delta < \frac{\varepsilon}{2N}$.
    If $\mu(A) < \delta$, we have
    \begin{align*}
      \int_{A} f d\mu
      &= \int_{A} (f-f_N) d\mu + \int_{A} f_N d\mu \\
      &\leq \int_{E} (f-f_N) d\mu + N \mu(A) \\
      &< \frac{\varepsilon}{2} + \frac{\varepsilon}{2} \\
      &= \varepsilon.
    \end{align*}
  \end{enumerate}

\item[(2)]
  Apply (1) to $f^{+}$ and $f^{-}$ on $E = [a,b]$.
  Given any $\varepsilon > 0$, there is a common $\delta > 0$
  such that
  \[
    \abs{ \int_{x}^{y} f^{+} dt } < \frac{\varepsilon}{2}
    \qquad \text{and} \qquad
    \abs{ \int_{x}^{y} f^{-} dt } < \frac{\varepsilon}{2}
  \]
  whenever $|y - x| < \delta$.
  So
  \[
    \abs{ F(y) - F(x) }
    \leq \abs{\int_{x}^{y} f^{+} dt} + \abs{\int_{x}^{y} f^{-} dt}
    < \varepsilon
  \]
  whenever $|y - x| < \delta$.
  Hence $F$ is uniformly continuous.
  (In fact, $F$ is absolutely continuous by the same argument.)
\end{enumerate}
$\Box$ \\

\emph{Note.}
Compare to Theorem 6.20. \\\\



%%%%%%%%%%%%%%%%%%%%%%%%%%%%%%%%%%%%%%%%%%%%%%%%%%%%%%%%%%%%%%%%%%%%%%%%%%%%%%%%



\textbf{Exercise 11.10.}
\emph{If $\mu(X) < +\infty$ and $f \in \mathscr{L}^2(\mu)$ on $X$,
prove that $f \in \mathscr{L}$ on $X$.
If
\[
  \mu(X) = +\infty,
\]
this is false. For instance, if
\[
  f(x) = \frac{1}{1+|x|},
\]
then $f^2 \in \mathscr{L}$ on $\mathbb{R}^1$,
but $f \not\in \mathscr{L}$ on $\mathbb{R}^1$.} \\

\emph{Proof.}
\begin{enumerate}
\item[(1)]
Since $\mu(X) < +\infty$, $1 \in \mathscr{L}^2(\mu)$ on $X$.
By Theorem 11.35, $f \in \mathscr{L}(\mu)$, and
\[
  \int_X |f| d\mu
  \leq \norm{f} \norm{1}.
\]

\item[(2)]
  \emph{Show that $f^2 \in \mathscr{L}$ on $\mathbb{R}^1$.}
  To apply Theorem 11.33,
  we might restrict the measure space $X = \mathbb{R}^1$ to some interval $[a,b]$.
  Then apply the Lebesgue's monotone convergence theorem (Theorem 11.28) to get the conclusion.
  \begin{enumerate}
  \item[(a)]
  Write
  \[
    f(x)^2
    = \left(\frac{1}{1+|x|}\right)^2
    = \frac{1}{1 + 2|x| + x^2}
    \leq \frac{1}{1+x^2}.
  \]
  By Theorem 11.27,
  \emph{it suffices to show that $\frac{1}{1+x^2} \in \mathscr{L}$ on $\mathbb{R}^1$.}

  \item[(b)]
  Consider the sequence $\{f_n\}$ defined by
  \[
    f_n(x) = \frac{1}{1+x^2} \chi_{[-n,n]}(x).
  \]
  (Here $\chi_{[-n,n]} = K_{[-n,n]}$ is the characteristic function of $[-n,n]$
  defined in Definition 11.19.)
  By construction,
  \[
    0 \leq f_1(x) \leq f_2(x) \leq \cdots
    \qquad
    (x \in \mathbb{R}^1)
  \]
  and
  \[
    f_n(x) \to \frac{1}{1+x^2}
    \qquad
    (x \in \mathbb{R}^1).
  \]

  \item[(c)]
  Hence
  \begin{align*}
    \int_{\mathbb{R}^1} \frac{1}{1+x^2} dx
    &= \lim_{n \to \infty} \int_{\mathbb{R}^1} f_n(x) dx
      &(\text{Theorem 11.28}) \\
    &= \lim_{n \to \infty} \int_{\mathbb{R}^1} \frac{1}{1+x^2} \chi_{[-n,n]}(x) dx \\
    &= \lim_{n \to \infty} \int_{-n}^{n} \frac{1}{1+x^2} dx \\
    &= \lim_{n \to \infty} \mathscr{R}\int_{-n}^{n} \frac{1}{1+x^2} dx
      &(\text{Theorem 11.33}) \\
    &= \lim_{n \to \infty} 2 \arctan(n) \\
    &= \pi < \infty.
  \end{align*}
  \end{enumerate}

\item[(4)]
  \emph{Show that $f \not\in \mathscr{L}$ on $\mathbb{R}^1$.}
  \begin{enumerate}
  \item[(a)]
  Consider the sequence $\{f_n\}$ defined by
  \[
    f_n(x) = f(x) \chi_{[-n,n]}(x) = \frac{1}{1+|x|} \chi_{[-n,n]}(x).
  \]
  By construction,
  \[
    0 \leq f_1(x) \leq f_2(x) \leq \cdots
    \qquad
    (x \in \mathbb{R}^1)
  \]
  and
  \[
    f_n(x) \to f(x)
    \qquad
    (x \in \mathbb{R}^1).
  \]

  \item[(b)]
  Hence
  \begin{align*}
    \int_{\mathbb{R}^1} f(x) dx
    &= \lim_{n \to \infty} \int_{\mathbb{R}^1} f_n(x) dx
      &(\text{Theorem 11.28}) \\
    &= \lim_{n \to \infty} \int_{\mathbb{R}^1} \frac{1}{1+|x|} \chi_{[-n,n]}(x) dx \\
    &= \lim_{n \to \infty} \int_{-n}^{n} \frac{1}{1+|x|} dx \\
    &= \lim_{n \to \infty} \mathscr{R}\int_{-n}^{n} \frac{1}{1+|x|} dx
      &(\text{Theorem 11.33}) \\
    &= \lim_{n \to \infty} 2 \log(n+1) \\
    &= \infty,
  \end{align*}
  or $f \not\in \mathscr{L}$ on $\mathbb{R}^1$.
  \end{enumerate}
\end{enumerate}
$\Box$ \\



\emph{Note.}
Compare to Exercise 6.5. \\\\



%%%%%%%%%%%%%%%%%%%%%%%%%%%%%%%%%%%%%%%%%%%%%%%%%%%%%%%%%%%%%%%%%%%%%%%%%%%%%%%%



\textbf{Exercise 11.11.}
\emph{If $f, g \in \mathscr{L}(\mu)$ on $X$, defined the distance between $f$ and $g$ by
\[
  \int_{X} |f-g| d\mu.
\]
Prove that $\mathscr{L}(\mu)$ is a complete metric space.} \\



\emph{Proof.}
\begin{enumerate}
\item[(1)]
  Define
  \[
    \norm{f-g}_1 = \int_{X} |f-g| d\mu.
  \]
  Thus $\norm{f-g}_1 = 0$ if and only if $f = g$ almost everywhere on $X$ (Exercise 11.1).
  As in Remark 11.37, we identify two functions to be equivalent
  if they are equal almost everywhere.

\item[(2)]
  \emph{Show that $\mathscr{L}(\mu)$ is a metric space.}
  \begin{enumerate}
  \item[(a)]
    By definition, $\norm{f-g}_1 \geq 0$.
    Besides, $\norm{f-g}_1 = 0$ if and only if $f = g$ almost everywhere by (1).

  \item[(b)]
    $\norm{f-g}_1 = \norm{g-f}_1$ since $|f(x)-g(x)| = |g(x)-f(x)|$ for all $x \in X$.

  \item[(c)]
    Since $|f(x)-g(x)| \leq |f(x)-h(x)| + |h(x)-g(x)|$ for all $x \in X$,
    Remarks 11.23(c) and Theorem 11.29 imply that
    \[
      \norm{f-g}_1 \leq \norm{f-h}_1 + \norm{h-g}_1.
    \]
  \end{enumerate}

\item[(3)]
  \emph{Show that $\mathscr{L}(\mu)$ is complete.}
  Similar to the proof of Theorem 11.42.
  \begin{enumerate}
  \item[(a)]
    \emph{Let $\{f_n\}$ be a Cauchy sequence in $\mathscr{L}(\mu)$,
    show that there exists a function $f \in \mathscr{L}(\mu)$ such that
    $\{f_n\}$ converges to $f \in \mathscr{L}(\mu)$.}

  \item[(b)]
    Since $\{f_n\}$ is a Cauchy sequence,
    we can find a sequence $\{n_k\}$, $k = 1,2,3,\ldots$, such that
    \[
      \norm{ f_{n_k} - f_{n_{k+1}} }_1
      = \int_{X} \abs{f_{n_k} - f_{n_{k+1}}} d\mu
      < \frac{1}{2^k}
      \qquad
      (k = 1,2,3,\ldots).
    \]
    Hence
    \[
      \sum_{k=1}^{\infty} \int_{X} \abs{f_{n_k} - f_{n_{k+1}}} d\mu
      \leq \sum_{k=1}^{\infty} \frac{1}{2^k}
      = 1
      < +\infty.
    \]

  \item[(c)]
    By Theorem 11.30, we may interchange the summation and integration to get
    \[
      \int_{X} \sum_{k=1}^{\infty} \abs{f_{n_k} - f_{n_{k+1}}} d\mu < +\infty,
    \]
    or
    \[
      \sum_{k=1}^{\infty} \abs{f_{n_k}(x) - f_{n_{k+1}}(x)}
      = \sum_{k=1}^{\infty} \abs{f_{n_{k+1}}(x) - f_{n_{k}}(x)}
      < +\infty
    \]
    almost everywhere on $X$.

  \item[(d)]
    Since the $k$th partial sum of the series
    \[
      \sum_{k=1}^{\infty} (f_{n_{k+1}}(x) - f_{n_{k}}(x))
    \]
    which converges almost everywhere on $X$ (Theorem 3.45), is
    \[
      f_{n_{k+1}}(x) - f_{n_{1}}(x),
    \]
    we see that the equation
    \[
      f(x) = \lim_{k \to \infty} f_{n_k}(x)
    \]
    defines $f(x)$ for almost all $x \in X$,
    and it does not matter how we define $f(x)$ at the remaining points of $X$.

  \item[(e)]
    We shall now show that this function $f$ has the desired properties.
    Let $\varepsilon > 0$ be given, and choose $N$ such that
    \[
      \norm{ f_n-f_m }_1 \leq \varepsilon
    \]
    whenever $n, m \geq N$.
    If $n_k > N$, Fatou's theorem shows that
    \[
      \norm{ f-f_{n_k} }_1
      \leq \liminf_{i \to \infty} \norm{ f_{n_i}-f_{n_k} }_1
      \leq \varepsilon.
    \]
    Thus $f-f_{n_k} \in \mathscr{L}(\mu)$,
    and since $f = (f-f_{n_k}) + f_{n_k} \in \mathscr{L}(\mu)$,
    we see that $f \in \mathscr{L}(\mu)$.
    Also, since $\varepsilon$ is arbitrary,
    \[
      \lim_{k \to \infty} \norm{f-f_{n_k}}_1 = 0.
    \]

  \item[(f)]
    Finally, the inequality
    \[
      \norm{f-f_n}_1 \leq \norm{f-f_{n_k}}_1 + \norm{f_{n_k}-f_n}_1
    \]
    shows that $\{f_n\}$ converges to $f \in \mathscr{L}(\mu)$;
    for if we take $n$ and $n_k$ large enough,
    each of the two terms can be made arbitrary small.
  \end{enumerate}
\end{enumerate}
$\Box$ \\\\



%%%%%%%%%%%%%%%%%%%%%%%%%%%%%%%%%%%%%%%%%%%%%%%%%%%%%%%%%%%%%%%%%%%%%%%%%%%%%%%%



\textbf{Exercise 11.12.}
\emph{Suppose}
\begin{enumerate}
\item[(a)]
  \emph{$|f(x,y)| \leq 1$ if $0 \leq x \leq 1$, $0 \leq y \leq 1$.}

\item[(b)]
  \emph{for fixed $x$, $f(x,y)$ is a continuous function of $y$.}

\item[(c)]
  \emph{for fixed $y$, $f(x,y)$ is a continuous function of $x$.}
\end{enumerate}
\emph{Put
\[
  g(x) = \int_{0}^{1} f(x,y) dy
  \qquad
  (0 \leq x \leq 1).
\]
Is $g$ continuous?} \\

\emph{Proof.}
\begin{enumerate}
\item[(1)]
  \emph{Show that $g$ is continuous.}

\item[(2)]
  Let $\{ x_n \}$ be a sequence in $[0,1]$ such that
  $x_n \neq x$ and $\lim x_n = x$.
  It suffices to show that
  \begin{align*}
    \lim_{n \to \infty} g(x_n)
    &= \lim_{n \to \infty} \int_{0}^{1} f(x_n,y) dy \\
    &= \int_{0}^{1} \lim_{n \to \infty} f(x_n,y) dy \\
    &= \int_{0}^{1} f(x,y) dy \\
    &= g(x)
  \end{align*}
  (Theorem 4.2).
  Since $\lim_{n \to \infty} f(x_n,y) = f(x,y)$ for any fixed $y$ (by (c)),
  it suffices to show that
  \[
    \lim_{n \to \infty} \int_{0}^{1} f(x_n,y) dy
    = \int_{0}^{1} \lim_{n \to \infty} f(x_n,y) dy.
  \]

\item[(3)]
  Define $\{f_n\}$ by $f_n(y) = f(x_n,y)$.
  $f_n(y)$ is a continuous function of $y$ for every fixed $n$ (by (b)).
  Thus $f_n(y)$ is measurable (Example 11.14).
  Besides, $|f_n(y)| \leq 1$ and $1 \in \mathscr{L}$ on $[0,1]$ (by (a)).
  The Lebesgue's dominated convergence theorem (Theorem 11.32) implies that
  \[
    \lim_{n \to \infty} \int_{0}^{1} f(x_n,y) dy
    = \int_{0}^{1} \lim_{n \to \infty} f(x_n,y) dy.
  \]
\end{enumerate}
$\Box$ \\



\textbf{Supplement (Similar exercise).}
\emph{Suppose}
\begin{enumerate}
\item[(a)]
  \emph{$|f(x,y)| \leq g(y)$ if $0 \leq x \leq 1$, $0 \leq y \leq 1$,
  where $g \in \mathscr{L}$ on $[0,1]$.}

\item[(b)]
  \emph{for fixed $x$, $f(x,y)$ is a measurable function of $y$.}

\item[(c)]
  \emph{for fixed $y$, $f(x,y)$ is a continuous function of $x$.}
\end{enumerate}
\emph{Show that
\[
  h(x) = \int_{0}^{1} f(x,y) dy
  \qquad
  (0 \leq x \leq 1).
\]
is continuous.} \\\\



%%%%%%%%%%%%%%%%%%%%%%%%%%%%%%%%%%%%%%%%%%%%%%%%%%%%%%%%%%%%%%%%%%%%%%%%%%%%%%%%



\textbf{Exercise 11.13.}
\emph{Consider the functions
\[
  f_n(x) = \sin(nx)
  \qquad
  (n=1,2,3,\ldots, -\pi \leq x \leq \pi)
\]
as points of $\mathscr{L}^2$.
Prove that the set of these points is closed and bounded, but not compact.} \\


\emph{Proof.}
Define $E = \{ f_n \}$ as a set in $\mathscr{L}^2$.
\begin{enumerate}
\item[(1)]
  \emph{Show that $E$ is bounded.}
  Note that
  \[
    \norm{f_n}_2
    = \left( \int_{-\pi}^{\pi} \sin(nx)^2 dx \right)^{\frac{1}{2}}
    = \sqrt{\pi}
  \]
  for all $n$ (Definition 8.10).
  So $E$ is bounded by $\sqrt{\pi}$.

\item[(2)]
  \emph{Show that $E$ is closed.}
  It suffices to show that $E$ has no limit points.
  \begin{align*}
    \norm{f_n - f_m}_2
    &= \left( \int_{-\pi}^{\pi} (\sin(nx)-\sin(mx))^2 dx \right)^{\frac{1}{2}} \\
    &= \left( \int_{-\pi}^{\pi}
      \sin(nx)^2 - 2\sin(nx)\sin(mx) + \sin(mx)^2 dx \right)^{\frac{1}{2}} \\
    &= (\pi + 0 + \pi)^{\frac{1}{2}} \\
    &= \sqrt{2\pi}
  \end{align*}
  for all $n \neq m$ (Definition 8.10).
  So all points of $E$ are isolated.

\item[(3)]
  \emph{Show that $E$ is not compact.}
  \begin{enumerate}
  \item[(a)]
    Take a collection
    \[
      \mathscr{G} = \left\{ G_n = B\left(f_n;1\right) \right\}
    \]
    of open subsets ($n = 1,2,3,\ldots$).

  \item[(b)]
    $\mathscr{G}$ is an open covering of $E \subseteq \mathscr{L}^2$
    since $f_n \in G_n$ for each $n = 1,2,3,\ldots$.

  \item[(c)]
    \emph{Show that there is no finite subcoverings of $\mathscr{G}$.}
    (Reductio ad absurdum)
    If
    \[
      \mathscr{G}' = \left\{ G_{n_1}, G_{n_2}, \ldots, G_{n_k} \right\}
    \]
    were a finite subcovering of $\mathscr{G}$ with $n_1 < n_2 < \cdots < n_k$,
    then $f_{n_k+1}$ is not in any open sets from $\mathscr{G}'$ (by (2)),
    which is absurd.
  \end{enumerate}
\end{enumerate}
$\Box$ \\\\



%%%%%%%%%%%%%%%%%%%%%%%%%%%%%%%%%%%%%%%%%%%%%%%%%%%%%%%%%%%%%%%%%%%%%%%%%%%%%%%%



\textbf{Exercise 11.14.}
\emph{Prove that a complex function $f$ is measurable
if and only if $f^{-1}(V)$ is measurable for every open set $V$ in the plane.} \\

\emph{Proof.}
\begin{enumerate}
\item[(1)]
  Write $f=u+iv$, where $u$ and $v$ are real.
  $f$ is measurable if and only if both $u$ and $v$ are measurable.

\item[(2)]
  \emph{Show that $f$ is measurable
  if $f^{-1}(V)$ is measurable for every open set $V \subseteq \mathbb{C}$.}
  For every real $a$,
  both sets
  \begin{align*}
    f^{-1}((a,\infty) \times \mathbb{R}^1) &= \{ x : u(x) > a \}, \\
    f^{-1}(\mathbb{R}^1 \times (a,\infty)) &= \{ x : v(x) > a \}
  \end{align*}
  are measurable.
  By Definition 11.13, both $u$ and $v$ are measurable, or $f$ is measurable.

\item[(3)]
  \emph{Show that $f^{-1}(V)$ is measurable for every open set $V \subseteq \mathbb{C}$
  if $f$ is measurable.}
  Since $\mathbb{R}^2$ is separable (Exercise 2.22),
  $V$ is the union of a subcollection of a countable base (Exercise 2.23).
  In particular,
  define
  \[
    I((a,b);r)
    = (a-r,a+r) \times (b-r,b+r)
    \subseteq \mathbb{R}^2
  \]
  for all $a, b, r \in \mathbb{Q}$.
  The collection
  \[
    \{ I((a,b);r) \}
  \]
  is a countable base for $\mathbb{R}^2$.
  Write $V$ as a countable union of $I((a,b);r)$:
  \[
    V = \bigcup I((a,b);r).
  \]

\item[(4)]
  \emph{Show that $f^{-1}(V)$ is measurable for every open set $V \subseteq \mathbb{C}$.}
  \begin{align*}
    f^{-1}(V)
    &= f^{-1} \left( \bigcup I((a,b);r) \right) \\
    &= \bigcup f^{-1}(I((a,b);r)) \\
    &= \bigcup \{ x : f(x) \in I((a,b);r) \} \\
    &= \bigcup \{ x : (u(x), v(x)) \in (a-r,a+r) \times (b-r,b+r) \} \\
    &= \bigcup \left( \{ x : u(x) \in (a-r,a+r) \} \bigcap \{ x : v(x) \in (b-r,b+r) \} \right) \\
    &= \bigcup \left( u^{-1}((a-r,a+r)) \bigcap v^{-1}((b-r,b+r)) \right).
  \end{align*}
  Since both $u$ and $v$ are measurable,
  \[
    u^{-1}((a-r,a+r)) \bigcap v^{-1}((b-r,b+r))
  \]
  is measurable, and thus $f^{-1}(V)$ is measurable (since the union is countable).
\end{enumerate}
$\Box$ \\\\



%%%%%%%%%%%%%%%%%%%%%%%%%%%%%%%%%%%%%%%%%%%%%%%%%%%%%%%%%%%%%%%%%%%%%%%%%%%%%%%%



\textbf{Exercise 11.15.}
\emph{Let $\mathscr{R}$ be the ring of all elementary subsets of $(0,1]$.
If $0 < a \leq b \leq 1$, define
\[
  \phi([a,b])
  = \phi([a,b))
  = \phi((a,b])
  = \phi((a,b))
  = b-a,
\]
but define
\[
  \phi((0,b)) = \phi((0,b]) = 1+b
\]
if $0 < b \leq 1$.
Show that this gives an additive set function $\phi$ on $\mathscr{R}$,
which is not regular and
which cannot be extended to a countably additive set function on a $\sigma$-ring.} \\

\emph{Proof.}
\begin{enumerate}
\item[(1)]
  Define $\phi: \mathscr{R} \to \mathbb{R} \cup \{ \pm \infty \}$ by
  \[
    \phi(A) = \sum_{i=1}^{n} \phi(I_i)
  \]
  where $A$ is a finite number of disjoint intervals $I_1, \ldots, I_n$ (Definition 11.4).

\item[(2)]
  \emph{Show that $\phi$ is an additive set function.}
  Given any two elementary sets $A, B \in \mathscr{R}$ with $A \bigcap B = \varnothing$.
  By Definition 11.4,
  \[
    A = \bigcup_{i=1}^{n} I_i,
    \qquad
    B = \bigcup_{j=1}^{m} J_j
  \]
  where $I_i \bigcap J_j = \varnothing$ for all $1 \leq i \leq n$ and $1 \leq j \leq m$
  (since $A \bigcap B = \varnothing$).
  Hence,
  \begin{align*}
    \phi\left(A \bigcup B\right)
    &= \phi\left( \left\{ \bigcup_{i=1}^{n} I_i \right\}
      \bigcup \left\{ \bigcup_{j=1}^{m} J_j \right\} \right) \\
    &= \sum_{i=1}^{n} \phi(I_i) + \sum_{j=1}^{m} \phi(J_j) \\
    &= \phi(A) + \phi(B).
  \end{align*}

\item[(3)]
  \emph{Show that $\phi$ is not countably additive.}
  Write
  \[
    (0,1]
    = \bigcup_{i=1}^{\infty} A_i
  \]
  where $A_i = \left( 2^{-i},2^{-i+1} \right]$.
  Note that $A_i \bigcap A_j = \varnothing$ if $i \neq j$.
  So
  \[
    \sum_{i=1}^{\infty} \phi(A_i)
    = \sum_{i=1}^{\infty} 2^{-i}
    = 1
    \neq 2
    = \phi((0,1]).
  \]

\item[(4)]
  \emph{Show that $\phi$ is not regular.}
  \begin{enumerate}
  \item[(a)]
    \emph{Given any closed set $F \in \mathscr{R}$. Show that $\phi(F) \leq 1$.}
    Write $F = \bigcup_{i=1}^{n} I_i$ where each $I_i$ are disjoint by Definition 11.4.
    Here every $I_i$ is never of the form $(0,b]$ or $(0,b)$ where $b > 0$.
    (Otherwise $0$ is a limit point of $F$, or $0 \in \overline{F} = F$,
    which is absurd.)
    Hence $\phi(F) = \sum \phi(I_i) \leq 1$.

  \item[(b)]
    Take $A = \left(0,\frac{1}{2}\right] \in \mathscr{R}$
    and $\varepsilon = \frac{1}{64} > 0$.
    Then for every closed set $F \in \mathscr{R}$,
    we have
    \[
      \phi(A) = \frac{3}{2} > 1 + \frac{1}{64} \geq \phi(F) + \varepsilon.
    \]
    That is, $\phi$ cannot be regular (Definition 11.5).
  \end{enumerate}
\end{enumerate}
$\Box$ \\\\



%%%%%%%%%%%%%%%%%%%%%%%%%%%%%%%%%%%%%%%%%%%%%%%%%%%%%%%%%%%%%%%%%%%%%%%%%%%%%%%%



\textbf{Exercise 11.16.}
\emph{Suppose $\{n_k\}$ is an increasing sequence of positive integers and
$E$ is the set of all $x \in (-\pi,\pi)$ at which $\{ \sin(n_k x) \}$ converges.
Prove that $m(E) = 0$.
(Hint: For every $A \subseteq E$,
\[
  \int_A \sin(n_k x) dx \to 0,
\]
and
\[
  2 \int_A (\sin(n_k x))^2 dx = \int_A (1-\cos(2n_k x)) dx \to m(A)
\]
as $k \to \infty$.)} \\

\emph{Proof (Hint).}
\begin{enumerate}
\item[(1)]
  Define $\{f_k\}$ by $f_k(x) = \sin(n_k x)$ on $[-\pi,\pi]$ for $k = 1,2,3,\ldots$.
  $\{f_k\}$ is a sequence of measurable functions on $[-\pi,\pi]$
  since each $f_k: x \to \sin(n_k x)$ is continuous (Example 11.14).
  By Exercise 11.3, $E$ is measurable.
  Given any measurable subset $A$ of $E$,
  $\{f_k\}$ is a sequence of measurable functions on $A$
  and $f(x) = \lim_{k \to \infty} f_k(x)$ is well-defined by the definition of $A \subseteq E$.

\item[(2)]
  Apply the Bessel inequality (Theorem 8.12 and Definition 11.39) to
  the function $\chi_{A} \in \mathscr{L}^2$ on $[-\pi,\pi]$,
  we have
  \[
    c_{-n}
    = \int_{[-\pi,\pi]} \chi_{A} e^{inx} dx
    \to 0
  \]
  as $n \to \infty$.
  Hence
  \[
    \lim_{k \to \infty} \int_{A} \sin(n_k x) dx
    = 0
  \]
  for any measurable subset $A$ of $E$.

\item[(3)]
  \emph{Show that $f(x) = 0$ almost everywhere on $E$.}
  Note that
  \[
    |f_k(x)| = |\sin(n_k x)| \leq 1
  \]
  on $A$ and
  \[
    \int_{A} dx
    = m(A)
    \leq m([-\pi,\pi])
    = 2\pi
    < \infty.
  \]
  By (2) and the Lebesgue's dominated convergence theorem (Theorem 11.32),
  \[
    \int_{A} f dx
    = \lim_{k \to \infty} \int_{A} f_k dx
    = \lim_{k \to \infty} \int_{A} \sin(n_k x) dx
    = 0
  \]
  for any measurable subset $A$ of $E$.
  By Exercise 11.2, the conclusion holds.

\item[(4)]
  Apply (1)(2)(3) to the sequence of measurable functions $\{f_k^2\}$ on $[-\pi,\pi]$,
  we have
  \begin{align*}
    0
    &= 2 \int_{A} f^2 dx
      &(\text{$f^2(x) = 0$ a.e. on $A$}) \\
    &= \lim_{k \to \infty} 2 \int_{A} f_k^2 dx \\
    &= \lim_{k \to \infty} 2 \int_{A} \sin(n_k x)^2 dx \\
    &= \lim_{k \to \infty} \int_A (1-\cos(2n_k x)) dx \\
    &= m(A) - \lim_{k \to \infty} \int_A \cos(2n_k x) dx \\
    &= m(A)
  \end{align*}
  for any measurable subset $A$ of $E$.
  In particular, take $A = E$ to get $m(E) = 0$.
\end{enumerate}
$\Box$ \\



%%%%%%%%%%%%%%%%%%%%%%%%%%%%%%%%%%%%%%%%%%%%%%%%%%%%%%%%%%%%%%%%%%%%%%%%%%%%%%%%



\textbf{Exercise 11.17.}
\emph{Suppose $E \subseteq (-\pi,\pi)$, $m(E) > 0$, $\delta > 0$.
Use the Bessel inequality to prove that there are at most finitely many integers $n$
such that $\sin(nx) \geq \delta$ for all $x \in E$.} \\

\emph{Proof.}
\begin{enumerate}
\item[(1)]
  (Reductio ad absurdum)
  If there were infinitely many integers $n$
  such that $\sin(nx) \geq \delta$ for all $x \in E$,
  then there exists an increasing sequence of positive integers $\{n_k\}$
  such that $\sin(n_k x) \geq \delta$ for all $x \in E$ and $n_k$.

\item[(2)]
  Since $E$ is measurable,
  we apply the Bessel inequality (Theorem 8.12 and Definition 11.39) to
  the function $\chi_{E} \in \mathscr{L}^2$ on $[-\pi,\pi]$:
  \[
    c_{-n}
    = \int_{[-\pi,\pi]} \chi_{E} e^{inx} dx
    \to 0
  \]
  as $n \to \infty$.
  Hence
  \[
    \lim_{k \to \infty} \int_{E} \sin(n_k x) dx
    = 0.
  \]

\item[(3)]
  Note that for all $n_k$, we have
  \[
    \int_{E} \sin(n_k x) dx
    \geq m(E) \delta.
  \]
  Here $m(E) \delta > 0$ is a constant, contrary to
  $\lim_{k \to \infty} \int_{E} \sin(n_k x) dx = 0$.
\end{enumerate}
$\Box$ \\\\



%%%%%%%%%%%%%%%%%%%%%%%%%%%%%%%%%%%%%%%%%%%%%%%%%%%%%%%%%%%%%%%%%%%%%%%%%%%%%%%%



\textbf{Exercise 11.18.}
\emph{Suppose $f \in \mathscr{L}^2(\mu)$, $g \in \mathscr{L}^2(\mu)$.
Prove that
\[
  \abs{ \int f\overline{g} d\mu }^2 = \int |f|^2 d\mu \int |g|^2 d\mu
\]
if and only if $f(x) = 0$ almost everywhere or there is a constant $c$ such that
$g(x) = cf(x)$ almost everywhere.
(Compare Theorem 11.35.)} \\



\emph{Proof.}
\begin{enumerate}
\item[(1)]
  \emph{Show that
  \[
    \abs{ \int f\overline{g} d\mu }^2 = \int |f|^2 d\mu \int |g|^2 d\mu
  \]
  if $f(x) = 0$ almost everywhere or
  there is a constant $c$ such that $g(x) = cf(x)$ almost everywhere.}
  \begin{enumerate}
  \item[(a)]
    If $f(x) = 0$ almost everywhere,
    then
    \[
      \abs{ \int f\overline{g} d\mu }^2 = 0
    \]
    and
    \[
      \int |f|^2 d\mu \int |g|^2 d\mu = 0.
    \]

  \item[(b)]
    If there is a constant $c \in \mathbb{C}$ such that $g(x) = cf(x)$ almost everywhere,
    then
    \[
      \abs{ \int f\overline{g} d\mu }^2
      = \abs{ \int \overline{c}|f|^2 d\mu }^2
      = |\overline{c}|^2 \left\{ \int |f|^2 d\mu \right\}^2
    \]
    and
    \[
      \int |f|^2 d\mu \int |g|^2 d\mu
      = \int |f|^2 d\mu \int |c|^2 |f|^2 d\mu
      = |c|^2 \left\{ \int |f|^2 d\mu \right\}^2.
    \]
    Since $|\overline{c}| = |c|$, the conclusion holds.
  \end{enumerate}

\item[(2)]
  \emph{Show that $f(x) = 0$ almost everywhere
  or there is a constant $c$ such that $g(x) = cf(x)$ almost everywhere
  if}
  \[
    \abs{ \int f\overline{g} d\mu }^2 = \int |f|^2 d\mu \int |g|^2 d\mu.
  \]
  \begin{enumerate}
  \item[(a)]
    We might assume that $\int |f|^2 d\mu > 0$.
    (The case $\int |f|^2 d\mu = 0$ implies that $f(x) = 0$ almost everywhere (Exercise 11.1).)

  \item[(b)]
    Let
    \[
      c = \frac{\overline{\int f\overline{g} d\mu}}{\int |f|^2 d\mu} \in \mathbb{C}.
    \]
    $c$ is well-defined since $\int |f|^2 d\mu \neq 0$ by (a).
    Then it suffices to show that $\int |g-cf|^2 d\mu = 0$
    since this conclusion implies that $g(x) = cf(x)$ almost everywhere by Exercise 11.1.

  \item[(c)]
    \emph{Show that $\int |c|^2|f|^2 d\mu = \int |g|^2 d\mu$.}
    \begin{align*}
      \int |c|^2|f|^2 d\mu
      &= |c|^2 \int |f|^2 d\mu \\
      &= \frac{\abs{ \int f\overline{g} d\mu }^2}{ \left\{ \int |f|^2 d\mu \right\}^2 }
        \int |f|^2 d\mu \\
      &= \frac{\abs{ \int f\overline{g} d\mu }^2}{\int |f|^2 d\mu} \\
      &= \int |g|^2 d\mu
    \end{align*}

  \item[(d)]
    \emph{Show that $\int \Re(cf\overline{g}) d\mu = \int |g|^2 d\mu$.}
    \begin{align*}
      \int \Re(cf\overline{g}) d\mu
      &= \Re\left( \int cf\overline{g} d\mu \right) \\
      &= \Re\left( c \int f\overline{g} d\mu \right) \\
      &= \Re\left( \frac{\overline{\int f\overline{g} d\mu}}{\int |f|^2 d\mu}
        \int f\overline{g} d\mu \right) \\
      &= \Re\left( \frac{\abs{\int f\overline{g} d\mu}^2}{\int |f|^2 d\mu} \right) \\
      &= \Re\left( \int |g|^2 d\mu \right) \\
      &= \int |g|^2 d\mu.
    \end{align*}

  \item[(e)]
    By (c)(d), we have
    \begin{align*}
      \int |g-cf|^2 d\mu
      &= \int |g|^2 - 2\Re(cf\overline{g}) + |c|^2|f|^2 d\mu \\
      &= \int |g|^2 d\mu - 2 \int \Re(cf\overline{g}) d\mu + \int |c|^2|f|^2 d\mu \\
      &= \int |g|^2 d\mu - 2 \int |g|^2 d\mu + \int |g|^2 d\mu \\
      &= 0.
    \end{align*}
  \end{enumerate}
\end{enumerate}
$\Box$ \\



\emph{Note.}
Compare Exercise 1.15. \\\\



%%%%%%%%%%%%%%%%%%%%%%%%%%%%%%%%%%%%%%%%%%%%%%%%%%%%%%%%%%%%%%%%%%%%%%%%%%%%%%%%
%%%%%%%%%%%%%%%%%%%%%%%%%%%%%%%%%%%%%%%%%%%%%%%%%%%%%%%%%%%%%%%%%%%%%%%%%%%%%%%%



\end{document}