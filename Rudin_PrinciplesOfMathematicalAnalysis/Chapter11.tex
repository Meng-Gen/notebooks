\documentclass{article}
\usepackage{amsfonts}
\usepackage{amsmath}
\usepackage{amssymb}
\usepackage{hyperref}
\usepackage[none]{hyphenat}
\usepackage{mathrsfs}
\usepackage{physics}
\parindent=0pt

\def\upint{\mathchoice%
    {\mkern13mu\overline{\vphantom{\intop}\mkern7mu}\mkern-20mu}%
    {\mkern7mu\overline{\vphantom{\intop}\mkern7mu}\mkern-14mu}%
    {\mkern7mu\overline{\vphantom{\intop}\mkern7mu}\mkern-14mu}%
    {\mkern7mu\overline{\vphantom{\intop}\mkern7mu}\mkern-14mu}%
  \int}
\def\lowint{\mkern3mu\underline{\vphantom{\intop}\mkern7mu}\mkern-10mu\int}

\begin{document}

\textbf{\Large Chapter 11: The Lebesuge Theory} \\\\



\emph{Author: Meng-Gen Tsai} \\
\emph{Email: plover@gmail.com} \\\\



%%%%%%%%%%%%%%%%%%%%%%%%%%%%%%%%%%%%%%%%%%%%%%%%%%%%%%%%%%%%%%%%%%%%%%%%%%%%%%%%
%%%%%%%%%%%%%%%%%%%%%%%%%%%%%%%%%%%%%%%%%%%%%%%%%%%%%%%%%%%%%%%%%%%%%%%%%%%%%%%%



\textbf{Exercise 11.1.}
\emph{If $f \geq 0$ and $\int_{E} f d\mu = 0$,
prove that $f(x) = 0$ almost everywhere on $E$.
(Hint: Let $E_n$ be the subset of $E$ on which $f(x) > \frac{1}{n}$.
Write $A = \bigcup E_n$.
Then $\mu(A) = 0$ if and only if $\mu(E_n) = 0$ for every $n$.)} \\

Might assume that $f$ is measurable on $E$. \\

\emph{Proof (Hint).}
\begin{enumerate}
\item[(1)]
Define $A = \{ x \in E : f(x) > 0 \}$.
So $f(x) = 0$ almost everywhere on $E$ if and only if $\mu(A) = 0$.

\item[(2)]
Define
\[
  E_n = \left\{ x \in E : f(x) > \frac{1}{n} \right\}
\]
for $n = 1,2,3,\ldots$.
Note that $E_1 \subseteq E_2 \subseteq E_3 \subseteq \cdots$ and
\[
  A = \bigcup_{n=1}^{\infty} E_n.
\]
Since $\mu$ is a measure,
\[
  \lim_{n \to \infty} \mu(E_n) = \mu(A)
\]
(Theorem 11.3).

\item[(3)]
(Reductio ad absurdum)
If $\mu(A)> 0$, there is an integer $N$ such that
$\mu(E_n) \geq \frac{\mu(A)}{2}$ whenever $n \geq N$ (by (2)).
In particular, take $n = N$ to get
\begin{align*}
  \int_E f d\mu
  &\geq \int_{E_N} f d\mu
    &\text{($\mu$ is a measure and $E_N \subseteq E$)}\\
  &\geq \frac{1}{N} \cdot \mu(E_N)
    &\text{(Remarks 11.23(b))} \\
  &\geq \frac{1}{N} \cdot \frac{\mu(A)}{2} \\
  & > 0,
\end{align*}
contrary to the assumption that $\int_{E} f d\mu = 0$.
\end{enumerate}
$\Box$ \\

\emph{Note.}
Compare to Exercise 6.2. \\\\



%%%%%%%%%%%%%%%%%%%%%%%%%%%%%%%%%%%%%%%%%%%%%%%%%%%%%%%%%%%%%%%%%%%%%%%%%%%%%%%%



\textbf{Exercise 11.2.}
\emph{If $\int_A f d\mu = 0$ for every measurable subset $A$ of a measurable set $E$,
then $f(x) = 0$ almost everywhere on $E$.} \\

Might assume that $f$ is measurable on $E$. \\

\emph{Proof.}
\begin{enumerate}
\item[(1)]
  Define
  \[
    A = \{ x \in E : f(x) \geq 0 \}
    \qquad \text{ and } \qquad
    B = \{ x \in E : f(x) \leq 0 \}.
  \]
  $A$ and $B$ are measurable subsets of a measurable set $E$ since $f$ is measurable.

\item[(2)]
  Apply Exercise 11.1 to the fact that $f \geq 0$ on $A$ (by construction)
  and $\int_A f d\mu = 0$ (by assumption),
  we have $f(x) = 0$ almost everywhere on $A$.

\item[(3)]
  Similarly,
  apply Exercise 11.1 to the fact that $-f \geq 0$ on $B$
  and $\int_B (-f) d\mu = -\int_B f d\mu = 0$,
  we have $f(x) = 0$ almost everywhere on $B$.

\item[(4)]
  As $E = A \cup B$, $f(x) = 0$ almost everywhere on $E$ by (2)(3).
\end{enumerate}
$\Box$ \\\\



%%%%%%%%%%%%%%%%%%%%%%%%%%%%%%%%%%%%%%%%%%%%%%%%%%%%%%%%%%%%%%%%%%%%%%%%%%%%%%%%



\textbf{Exercise 11.3.}
\emph{If $\{f_n\}$ is a sequence of measurable functions,
prove that the set of points $x$ at which $\{f_n(x)\}$ converges is measurable.} \\

\emph{Proof.}
\begin{enumerate}
\item[(1)]
  It suffices to show that
  \[
    E
    = \{ x : \{ f_n(x) \} \text{ is convergent } \}
    = \{ x : \{ f_n(x) \} \text{ is Cauchy } \}
  \]
  is measurable (since $\mathbb{R}^1$ is complete).

\item[(2)]
  Write
  \[
    E
    =
    \bigcap_{k=1}^{\infty}
    \bigcup_{N=1}^{\infty}
    \bigcap_{n,m \geq N} \left\{ x : |f_n(x)-f_m(x)| \leq \frac{1}{k} \right\}
  \]
  Since $\{f_n\}$ is a sequence of measurable functions,
  $x \mapsto |f_n(x)-f_m(x)|$ is measurable
  (Theorem 11.16 and Theorem 11.18).
  Hence
  \[
    \left\{ x : |f_n(x)-f_m(x)| \leq \frac{1}{k} \right\}
  \]
  is measurable (Theorem 11.15).
  Therefore $E$ is measurable.
\end{enumerate}
$\Box$ \\\\



%%%%%%%%%%%%%%%%%%%%%%%%%%%%%%%%%%%%%%%%%%%%%%%%%%%%%%%%%%%%%%%%%%%%%%%%%%%%%%%%



\textbf{Exercise 11.4.}
\emph{If $f \in \mathscr{L}(\mu)$ on $E$ and $g$ is bounded and measurable on $E$,
then $fg \in \mathscr{L}(\mu)$ on $E$.} \\

\emph{Proof (Theorem 11.27).}
\begin{enumerate}
\item[(1)]
  $fg$ is measurable since both $f$ and $g$ are measurable (Theorem 11.18).

\item[(2)]
  $|g| \leq M$ for some real $M \in \mathbb{R}^1$ by the boundedness of $g$.
  Hence
  \[
    |fg| \leq M|f|
  \]
  on $E$.

\item[(3)]
  To apply Theorem 11.27, it suffices to show that
  $M|f| \in \mathscr{L}(\mu)$ on $E$.
  Theorem 11.26 implies that $|f| \in \mathscr{L}(\mu)$ if $f \in \mathscr{L}(\mu)$.
  And Remarks 11.23(d) implies that $M|f| \in \mathscr{L}(\mu)$ if $|f| \in \mathscr{L}(\mu)$.
\end{enumerate}
$\Box$ \\

\emph{Note} (Riemann integral).
  \emph{If $f \in \mathscr{R}$ on $[a,b]$ and $g$ is bounded and measurable on $[a,b]$,
  then $fg$ might be not Riemann integrable.} \\\\



%%%%%%%%%%%%%%%%%%%%%%%%%%%%%%%%%%%%%%%%%%%%%%%%%%%%%%%%%%%%%%%%%%%%%%%%%%%%%%%%



\textbf{Exercise 11.5.}
\emph{Put
  \begin{equation*}
    g(x) =
    \begin{cases}
      0 & (0 \leq x \leq \frac{1}{2}), \\
      1 & (\frac{1}{2} < x \leq 1),
    \end{cases}
  \end{equation*}
  and
  \begin{align*}
    f_{2k}(x) &= g(x) &(0 \leq x \leq 1), \\
    f_{2k+1}(x) &= g(1-x) &(0 \leq x \leq 1).
  \end{align*}
Show that
\[
  \liminf_{n \to \infty} f_n(x) = 0
  \qquad
  (0 \leq x \leq 1),
\]
but
\[
  \int_{0}^{1} f_n(x)dx = \frac{1}{2}.
\]
(Compare with the Fatou's theorem.)} \\

\emph{Proof.}
\begin{enumerate}
\item[(1)]
  \emph{Show that $\liminf_{n \to \infty} f_n(x) = 0$.}
  Note that
  \begin{equation*}
    g(1-x) =
    \begin{cases}
      1 & (0 \leq x < \frac{1}{2}), \\
      0 & (\frac{1}{2} < x \leq 1).
    \end{cases}
  \end{equation*}
  Since $f_n(x) \geq 0$ by definition,
  $\liminf_{n \to \infty} f_n(x) \geq 0$.
  Since $f_{2k}(0) = f_{2k+1}(1) = 0$ for all positive integers $k$,
  $\liminf_{n \to \infty} f_n(x) \leq 0$.
  Therefore the result is established.

\item[(2)]
  \emph{Show that $\int_{0}^{1} f_n(x)dx = \frac{1}{2}$.}
  Since
  \begin{align*}
    \int_{0}^{1} f_{2k}(x)dx
    &= \int_{0}^{1} g(x)dx = \frac{1}{2}, \\
    \int_{0}^{1} f_{2k+1}(x)dx
    &= \int_{0}^{1} g(1-x)dx = \frac{1}{2},
  \end{align*}
  in any case $\int_{0}^{1} f_n(x)dx = \frac{1}{2}$ for all positive integers $n$.

\item[(3)]
  This example shows that we may have the strict inequality in the Fatou's theorem.
\end{enumerate}
$\Box$ \\



\textbf{Supplement (Similar exercise).}
  \emph{Consider the sequence $\{f_n\}$ defined by $f_n(x) = 1$ if $n \leq x < n+1$,
  with $f_n(x) = 0$ otherwise.
  Show that we may have the strict inequality in the Fatou's theorem.}\\\\



%%%%%%%%%%%%%%%%%%%%%%%%%%%%%%%%%%%%%%%%%%%%%%%%%%%%%%%%%%%%%%%%%%%%%%%%%%%%%%%%



\textbf{Exercise 11.6.}
\emph{Let
\begin{equation*}
f_n(x) =
  \begin{cases}
    \frac{1}{n}
      & (|x| \leq n), \\
    0
      & (|x| > n).
  \end{cases}
\end{equation*}
Then $f_n(x) \to 0$ uniformly on $\mathbb{R}^1$,
but
\[
  \int_{-\infty}^{\infty} f_n(x) dx = 2
  \qquad
  (n = 1,2,3,\ldots).
\]
(We write $\int_{-\infty}^{\infty}$ in place of $\int_{\mathbb{R}^1}$.)
Thus uniform convergence does not imply dominated convergence
in the sense of Theorem 11.32.
However, on sets of finite measure,
uniformly convergent sequences of bounded functions do satisfy Theorem 11.32.} \\



\emph{Proof.}
\begin{enumerate}
\item[(1)]
  \emph{Show that $f_n(x) \to 0$ uniformly on $\mathbb{R}^1$.}
  Given any $\varepsilon > 0$, there is an integer $N > \frac{1}{\varepsilon}$
  such that
  \[
    |f_n(x) - 0| \leq \frac{1}{n} \leq \frac{1}{N} < \varepsilon
  \]
  whenever $n \geq N$ and $x \in \mathbb{R}^1$.
  Hence $f_n(x) \to 0$ uniformly.

\item[(2)]
  \emph{Show that $\int_{-\infty}^{\infty} f_n(x) dx = 2$.}
  \[
    \int_{-\infty}^{\infty} f_n(x) dx
    = \int_{-n}^{n} \frac{1}{n} dx
    = 2.
  \]

\item[(3)]
  By (1)(2),
  \[
    \lim_{n \to \infty} \int_{-\infty}^{\infty} f_n(x) dx
    \neq \int_{-\infty}^{\infty} \lim_{n \to \infty} f_n(x) dx
  \]
  suggests that the Lebesgue's dominated convergence theorem (Theorem 11.32)
  does not hold in this case.
  In fact,
  if there were $g \in \mathscr{L}$ such that $|f_n(x)| \leq g(x)$,
  then
  \begin{align*}
    \int_{-\infty}^{\infty} g(x) dx
    &\geq \int_{0}^{\infty} g(x) dx
      &(\text{Theorem 11.24}) \\
    &= \sum_{n=1}^{\infty} \int_{n-1}^{n} g(x) dx
      &(\text{Theorem 11.24}) \\
    &\geq \sum_{n=1}^{\infty} \int_{n-1}^{n} |f_n(x)| dx \\
    &= \sum_{n=1}^{\infty} \int_{n-1}^{n} \frac{1}{n} dx \\
    &= \sum_{n=1}^{\infty} \frac{1}{n} \\
    &= \infty,
  \end{align*}
  which is absurd.

\item[(4)]
  \emph{Show that on sets of finite measure,
  uniformly convergent sequences of bounded functions $\{f_n\}$
  do satisfy Theorem 11.32.}
  \begin{enumerate}
  \item[(a)]
    Since $\{f_n\}$ is uniformly convergent,
    $\{f_n\}$ is uniformly bounded (Exercise 7.1), or
    there exists a real number $M$ such that
    \[
      |f_n(x)| \leq M
    \]
    for all positive integer $n$ and $x \in E$.

  \item[(b)]
    Define $g(x) = M$ on $E$.
    It is clear that
    \[
      \int_E g(x) dx = M \mu(E) < +\infty.
    \]
    Now we can apply the Lebesgue's dominated convergence theorem (Theorem 11.32)
    to get
    \[
      \lim_{n \to \infty} \int_E f_n d\mu = \int_E \lim_{n \to \infty} f_n d\mu.
    \]
  \end{enumerate}
\end{enumerate}
$\Box$ \\\\



%%%%%%%%%%%%%%%%%%%%%%%%%%%%%%%%%%%%%%%%%%%%%%%%%%%%%%%%%%%%%%%%%%%%%%%%%%%%%%%%



\textbf{Exercise 11.7.}
\emph{...} \\

\emph{Proof.}
\begin{enumerate}
\item[(1)]
\item[(2)]

\end{enumerate}
$\Box$ \\\\



%%%%%%%%%%%%%%%%%%%%%%%%%%%%%%%%%%%%%%%%%%%%%%%%%%%%%%%%%%%%%%%%%%%%%%%%%%%%%%%%



\textbf{Exercise 11.8.}
\emph{If $f \in \mathscr{R}$ on $[a,b]$ and if $F(x) = \int_{a}^{x}f(t)dt$,
prove that $F'(x) = f(x)$ almost everywhere on $[a,b]$.} \\

\emph{Proof.}
\begin{enumerate}
\item[(1)]
  Theorem 6.20 implies that
  $F'(x_0) = f(x_0)$ if $f$ is continuous at $x_0 \in [a,b]$.

\item[(2)]
  Since $f \in \mathscr{R}$ on $[a,b]$, $f$ is bounded on $[a,b]$.
  Theorem 11.33 implies that
  $f$ is continuous almost everywhere on $[a,b]$.
\end{enumerate}
By (1)(2), $F'(x) = f(x)$ almost everywhere on $[a,b]$.
$\Box$ \\\\



%%%%%%%%%%%%%%%%%%%%%%%%%%%%%%%%%%%%%%%%%%%%%%%%%%%%%%%%%%%%%%%%%%%%%%%%%%%%%%%%



\textbf{Exercise 11.9.}
\emph{Prove that the function $F$ given by
\[
  F(x) = \int_{a}^{x} f dt
  \qquad
  (a \leq x \leq b)
\]
(where $f \in \mathscr{L}$ on $[a,b]$) is continuous on $[a,b]$.} \\



\emph{Proof.}
\begin{enumerate}
\item[(1)]
  \emph{Let $f \in \mathscr{L}$ on $E$.
  Show that given any $\varepsilon > 0$ there is a $\delta > 0$ such that
  \[
    \int_{A} f d\mu < \varepsilon
  \]
  whenever $A \subseteq E$ with $\mu(A) < \delta$.}
  \begin{enumerate}
  \item[(a)]
    Define $f_n(x) = \min\{ f(x), n \}$ on $E$ for $n=1,2,3,\ldots$.
    Then $\{f_n\}$ is a sequence of measurable functions such that
    \[
      0 \leq f_1(x) \leq f_2(x) \leq \cdots.
    \]
    Also, $f_n \to f$.
    Then by the Lebesuge's monotone convergenece theorem (Theorem 11.28),
    \[
      \lim_{n \to \infty} \int_{E} f_n d\mu = \int_{E} f d\mu.
    \]

  \item[(b)]
    For such $\varepsilon > 0$, there is an integer $N \geq 1$ such that
    \[
      \int_{E} (f - f_N) d\mu < \frac{\varepsilon}{2}.
    \]
    Choose $\delta > 0$ so that $\delta < \frac{\varepsilon}{2N}$.
    If $\mu(A) < \delta$, we have
    \begin{align*}
      \int_{A} f d\mu
      &= \int_{A} (f-f_N) d\mu + \int_{A} f_N d\mu \\
      &\leq \int_{E} (f-f_N) d\mu + N \mu(A) \\
      &< \frac{\varepsilon}{2} + \frac{\varepsilon}{2} \\
      &= \varepsilon.
    \end{align*}
  \end{enumerate}

\item[(2)]
  Apply (1) to $f^{+}$ and $f^{-}$ on $E = [a,b]$.
  Given any $\varepsilon > 0$, there is a common $\delta > 0$
  such that
  \[
    \abs{ \int_{x}^{y} f^{+} dt } < \frac{\varepsilon}{2}
    \qquad \text{and} \qquad
    \abs{ \int_{x}^{y} f^{-} dt } < \frac{\varepsilon}{2}
  \]
  whenever $|y - x| < \delta$.
  So
  \[
    \abs{ F(y) - F(x) }
    \leq \abs{\int_{x}^{y} f^{+} dt} + \abs{\int_{x}^{y} f^{-} dt}
    < \varepsilon
  \]
  whenever $|y - x| < \delta$.
  Hence $F$ is uniformly continuous.
  (In fact, $F$ is absolutely continuous by the same argument.)
\end{enumerate}
$\Box$ \\

\emph{Note.}
Compare to Theorem 6.20. \\\\



%%%%%%%%%%%%%%%%%%%%%%%%%%%%%%%%%%%%%%%%%%%%%%%%%%%%%%%%%%%%%%%%%%%%%%%%%%%%%%%%



\textbf{Exercise 11.10.}
\emph{If $\mu(X) < +\infty$ and $f \in \mathscr{L}^2(\mu)$ on $X$,
prove that $f \in \mathscr{L}$ on $X$.
If
\[
  \mu(X) = +\infty,
\]
this is false. For instance, if
\[
  f(x) = \frac{1}{1+|x|},
\]
then $f^2 \in \mathscr{L}$ on $\mathbb{R}^1$,
but $f \not\in \mathscr{L}$ on $\mathbb{R}^1$.} \\

\emph{Proof.}
\begin{enumerate}
\item[(1)]
Since $\mu(X) < +\infty$, $1 \in \mathscr{L}^2(\mu)$ on $X$.
By Theorem 11.35, $f \in \mathscr{L}(\mu)$, and
\[
  \int_X |f| d\mu
  \leq \norm{f} \norm{1}.
\]

\item[(2)]
  \emph{Show that $f^2 \in \mathscr{L}$ on $\mathbb{R}^1$.}
  To apply Theorem 11.33,
  we might restrict the measure space $X = \mathbb{R}^1$ to some interval $[a,b]$.
  Then apply the Lebesgue's monotone convegence theorem (Theorem 11.28) to get the conclusion.
  \begin{enumerate}
  \item[(a)]
  Write
  \[
    f(x)^2
    = \left(\frac{1}{1+|x|}\right)^2
    = \frac{1}{1 + 2|x| + x^2}
    \leq \frac{1}{1+x^2}.
  \]
  By Theorem 11.27,
  \emph{it suffices to show that $\frac{1}{1+x^2} \in \mathscr{L}$ on $\mathbb{R}^1$.}

  \item[(b)]
  Consider the sequence $\{f_n\}$ defined by
  \[
    f_n(x) = \frac{1}{1+x^2} \chi_{[-n,n]}(x).
  \]
  (Here $\chi_{[-n,n]} = K_{[-n,n]}$ is the characteristic function of $[-n,n]$
  defined in Definition 11.19.)
  By construction,
  \[
    0 \leq f_1(x) \leq f_2(x) \leq \cdots
    \qquad
    (x \in \mathbb{R}^1)
  \]
  and
  \[
    f_n(x) \to \frac{1}{1+x^2}
    \qquad
    (x \in \mathbb{R}^1).
  \]

  \item[(c)]
  Hence
  \begin{align*}
    \int_{\mathbb{R}^1} \frac{1}{1+x^2} dx
    &= \lim_{n \to \infty} \int_{\mathbb{R}^1} f_n(x) dx
      &(\text{Theorem 11.28}) \\
    &= \lim_{n \to \infty} \int_{\mathbb{R}^1} \frac{1}{1+x^2} \chi_{[-n,n]}(x) dx \\
    &= \lim_{n \to \infty} \int_{-n}^{n} \frac{1}{1+x^2} dx \\
    &= \lim_{n \to \infty} \mathscr{R}\int_{-n}^{n} \frac{1}{1+x^2} dx
      &(\text{Theorem 11.33}) \\
    &= \lim_{n \to \infty} 2 \arctan(n) \\
    &= \pi < \infty.
  \end{align*}
  \end{enumerate}

\item[(4)]
  \emph{Show that $f \not\in \mathscr{L}$ on $\mathbb{R}^1$.}
  \begin{enumerate}
  \item[(a)]
  Consider the sequence $\{f_n\}$ defined by
  \[
    f_n(x) = f(x) \chi_{[-n,n]}(x) = \frac{1}{1+|x|} \chi_{[-n,n]}(x).
  \]
  By construction,
  \[
    0 \leq f_1(x) \leq f_2(x) \leq \cdots
    \qquad
    (x \in \mathbb{R}^1)
  \]
  and
  \[
    f_n(x) \to f(x)
    \qquad
    (x \in \mathbb{R}^1).
  \]

  \item[(b)]
  Hence
  \begin{align*}
    \int_{\mathbb{R}^1} f(x) dx
    &= \lim_{n \to \infty} \int_{\mathbb{R}^1} f_n(x) dx
      &(\text{Theorem 11.28}) \\
    &= \lim_{n \to \infty} \int_{\mathbb{R}^1} \frac{1}{1+|x|} \chi_{[-n,n]}(x) dx \\
    &= \lim_{n \to \infty} \int_{-n}^{n} \frac{1}{1+|x|} dx \\
    &= \lim_{n \to \infty} \mathscr{R}\int_{-n}^{n} \frac{1}{1+|x|} dx
      &(\text{Theorem 11.33}) \\
    &= \lim_{n \to \infty} 2 \log(n+1) \\
    &= \infty,
  \end{align*}
  or $f \not\in \mathscr{L}$ on $\mathbb{R}^1$.
  \end{enumerate}
\end{enumerate}
$\Box$ \\



\emph{Note.}
Compare to Exercise 6.5. \\\\



%%%%%%%%%%%%%%%%%%%%%%%%%%%%%%%%%%%%%%%%%%%%%%%%%%%%%%%%%%%%%%%%%%%%%%%%%%%%%%%%



\textbf{Exercise 11.11.}
\emph{If $f, g \in \mathscr{L}(\mu)$ on $X$, defined the distance between $f$ and $g$ by
\[
  \int_{X} |f-g| d\mu.
\]
Prove that $\mathscr{L}(\mu)$ is a complete metric space.} \\



\emph{Proof.}
\begin{enumerate}
\item[(1)]
  Define
  \[
    \norm{f-g}_1 = \int_{X} |f-g| d\mu.
  \]
  Thus $\norm{f-g}_1 = 0$ if and only if $f = g$ almost everywhere on $X$ (Exercise 11.1).
  As in Remark 11.37, we identify two functions to be equivalent
  if they are equal almost everywhere.

\item[(2)]
  \emph{Show that $\mathscr{L}(\mu)$ is a metric space.}
  \begin{enumerate}
  \item[(a)]
    By definition, $\norm{f-g}_1 \geq 0$.
    Besides, $\norm{f-g}_1 = 0$ if and only if $f = g$ almost everywhere by (1).

  \item[(b)]
    $\norm{f-g}_1 = \norm{g-f}_1$ since $|f(x)-g(x)| = |g(x)-f(x)|$ for all $x \in X$.

  \item[(c)]
    Since $|f(x)-g(x)| \leq |f(x)-h(x)| + |h(x)-g(x)|$ for all $x \in X$,
    Remarks 11.23(c) and Theorem 11.29 imply that
    \[
      \norm{f-g}_1 \leq \norm{f-h}_1 + \norm{h-g}_1.
    \]
  \end{enumerate}

\item[(3)]
  \emph{Show that $\mathscr{L}(\mu)$ is complete.}
  Similar to the proof of Theorem 11.42.
  \begin{enumerate}
  \item[(a)]
    \emph{Let $\{f_n\}$ be a Cauchy sequence in $\mathscr{L}(\mu)$,
    show that there exists a function $f \in \mathscr{L}(\mu)$ such that
    $\{f_n\}$ converges to $f \in \mathscr{L}(\mu)$.}

  \item[(b)]
    Since $\{f_n\}$ is a Cauchy sequence,
    we can find a sequence $\{n_k\}$, $k = 1,2,3,\ldots$, such that
    \[
      \norm{ f_{n_k} - f_{n_{k+1}} }_1
      = \int_{X} \abs{f_{n_k} - f_{n_{k+1}}} d\mu
      < \frac{1}{2^k}
      \qquad
      (k = 1,2,3,\ldots).
    \]
    Hence
    \[
      \sum_{k=1}^{\infty} \int_{X} \abs{f_{n_k} - f_{n_{k+1}}} d\mu
      \leq \sum_{k=1}^{\infty} \frac{1}{2^k}
      = 1
      < +\infty.
    \]

  \item[(c)]
    By Theorem 11.30, we may interchange the summation and integration to get
    \[
      \int_{X} \sum_{k=1}^{\infty} \abs{f_{n_k} - f_{n_{k+1}}} d\mu < +\infty,
    \]
    or
    \[
      \sum_{k=1}^{\infty} \abs{f_{n_k}(x) - f_{n_{k+1}}(x)}
      = \sum_{k=1}^{\infty} \abs{f_{n_{k+1}}(x) - f_{n_{k}}(x)}
      < +\infty
    \]
    almost everywhere on $X$.

  \item[(d)]
    Since the $k$th partial sum of the series
    \[
      \sum_{k=1}^{\infty} (f_{n_{k+1}}(x) - f_{n_{k}}(x))
    \]
    which converges almost everywhere on $X$ (Theorem 3.45), is
    \[
      f_{n_{k+1}}(x) - f_{n_{1}}(x),
    \]
    we see that the equation
    \[
      f(x) = \lim_{k \to \infty} f_{n_k}(x)
    \]
    defines $f(x)$ for almost all $x \in X$,
    and it does not matter how we define $f(x)$ at the remaining points of $X$.

  \item[(e)]
    We shall now show that this function $f$ has the desired properties.
    Let $\varepsilon > 0$ be given, and choose $N$ such that
    \[
      \norm{ f_n-f_m }_1 \leq \varepsilon
    \]
    whenever $n, m \geq N$.
    If $n_k > N$, Fatou's theorem shows that
    \[
      \norm{ f-f_{n_k} }_1
      \leq \liminf_{i \to \infty} \norm{ f_{n_i}-f_{n_k} }_1
      \leq \varepsilon.
    \]
    Thus $f-f_{n_k} \in \mathscr{L}(\mu)$,
    and since $f = (f-f_{n_k}) + f_{n_k} \in \mathscr{L}(\mu)$,
    we see that $f \in \mathscr{L}(\mu)$.
    Also, since $\varepsilon$ is arbitrary,
    \[
      \lim_{k \to \infty} \norm{f-f_{n_k}}_1 = 0.
    \]

  \item[(f)]
    Finally, the inequality
    \[
      \norm{f-f_n}_1 \leq \norm{f-f_{n_k}}_1 + \norm{f_{n_k}-f_n}_1
    \]
    shows that $\{f_n\}$ converges to $f \in \mathscr{L}(\mu)$;
    for if we take $n$ and $n_k$ large enough,
    each of the two terms can be made arbitrary small.
  \end{enumerate}
\end{enumerate}
$\Box$ \\\\



%%%%%%%%%%%%%%%%%%%%%%%%%%%%%%%%%%%%%%%%%%%%%%%%%%%%%%%%%%%%%%%%%%%%%%%%%%%%%%%%



\textbf{Exercise 11.12.}
\emph{...} \\

\emph{Proof.}
\begin{enumerate}
\item[(1)]
\item[(2)]

\end{enumerate}
$\Box$ \\\\



%%%%%%%%%%%%%%%%%%%%%%%%%%%%%%%%%%%%%%%%%%%%%%%%%%%%%%%%%%%%%%%%%%%%%%%%%%%%%%%%



\textbf{Exercise 11.13.}
\emph{...} \\

\emph{Proof.}
\begin{enumerate}
\item[(1)]
\item[(2)]

\end{enumerate}
$\Box$ \\\\



%%%%%%%%%%%%%%%%%%%%%%%%%%%%%%%%%%%%%%%%%%%%%%%%%%%%%%%%%%%%%%%%%%%%%%%%%%%%%%%%



\textbf{Exercise 11.14.}
\emph{...} \\

\emph{Proof.}
\begin{enumerate}
\item[(1)]
\item[(2)]

\end{enumerate}
$\Box$ \\\\



%%%%%%%%%%%%%%%%%%%%%%%%%%%%%%%%%%%%%%%%%%%%%%%%%%%%%%%%%%%%%%%%%%%%%%%%%%%%%%%%



\textbf{Exercise 11.15.}
\emph{...} \\

\emph{Proof.}
\begin{enumerate}
\item[(1)]
\item[(2)]

\end{enumerate}
$\Box$ \\\\



%%%%%%%%%%%%%%%%%%%%%%%%%%%%%%%%%%%%%%%%%%%%%%%%%%%%%%%%%%%%%%%%%%%%%%%%%%%%%%%%



\textbf{Exercise 11.16.}
\emph{...} \\

\emph{Proof.}
\begin{enumerate}
\item[(1)]
\item[(2)]

\end{enumerate}
$\Box$ \\\\



%%%%%%%%%%%%%%%%%%%%%%%%%%%%%%%%%%%%%%%%%%%%%%%%%%%%%%%%%%%%%%%%%%%%%%%%%%%%%%%%



\textbf{Exercise 11.17.}
\emph{...} \\

\emph{Proof.}
\begin{enumerate}
\item[(1)]
\item[(2)]

\end{enumerate}
$\Box$ \\\\



%%%%%%%%%%%%%%%%%%%%%%%%%%%%%%%%%%%%%%%%%%%%%%%%%%%%%%%%%%%%%%%%%%%%%%%%%%%%%%%%



\textbf{Exercise 11.18.}
\emph{...} \\

\emph{Proof.}
\begin{enumerate}
\item[(1)]
\item[(2)]

\end{enumerate}
$\Box$ \\\\



%%%%%%%%%%%%%%%%%%%%%%%%%%%%%%%%%%%%%%%%%%%%%%%%%%%%%%%%%%%%%%%%%%%%%%%%%%%%%%%%
%%%%%%%%%%%%%%%%%%%%%%%%%%%%%%%%%%%%%%%%%%%%%%%%%%%%%%%%%%%%%%%%%%%%%%%%%%%%%%%%



\end{document}