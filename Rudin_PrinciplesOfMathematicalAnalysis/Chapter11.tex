\documentclass{article}
\usepackage{amsfonts}
\usepackage{amsmath}
\usepackage{amssymb}
\usepackage{hyperref}
\usepackage[none]{hyphenat}
\usepackage{mathrsfs}
\usepackage{physics}
\parindent=0pt

\def\upint{\mathchoice%
    {\mkern13mu\overline{\vphantom{\intop}\mkern7mu}\mkern-20mu}%
    {\mkern7mu\overline{\vphantom{\intop}\mkern7mu}\mkern-14mu}%
    {\mkern7mu\overline{\vphantom{\intop}\mkern7mu}\mkern-14mu}%
    {\mkern7mu\overline{\vphantom{\intop}\mkern7mu}\mkern-14mu}%
  \int}
\def\lowint{\mkern3mu\underline{\vphantom{\intop}\mkern7mu}\mkern-10mu\int}

\begin{document}

\textbf{\Large Chapter 11: The Lebesuge Theory} \\\\



\emph{Author: Meng-Gen Tsai} \\
\emph{Email: plover@gmail.com} \\\\



%%%%%%%%%%%%%%%%%%%%%%%%%%%%%%%%%%%%%%%%%%%%%%%%%%%%%%%%%%%%%%%%%%%%%%%%%%%%%%%%
%%%%%%%%%%%%%%%%%%%%%%%%%%%%%%%%%%%%%%%%%%%%%%%%%%%%%%%%%%%%%%%%%%%%%%%%%%%%%%%%



\textbf{Exercise 11.1.}
\emph{If $f \geq 0$ and $\int_{E} f d\mu = 0$,
prove that $f(x) = 0$ almost everywhere on $E$.
(Hint: Let $E_n$ be the subset of $E$ on which $f(x) > \frac{1}{n}$.
Write $A = \bigcup E_n$.
Then $\mu(A) = 0$ if and only if $\mu(E_n) = 0$ for every $n$.)} \\

\emph{Proof (Hint).}
\begin{enumerate}
\item[(1)]
Define $A = \{ x \in E : f(x) > 0 \}$.
So $f(x) = 0$ almost everywhere on $E$ if and only if $\mu(A) = 0$.

\item[(2)]
Define
\[
  E_n = \left\{ x \in E : f(x) > \frac{1}{n} \right\}
\]
for $n = 1,2,3,\ldots$.
Note that $E_1 \subseteq E_2 \subseteq E_3 \subseteq \cdots$ and
\[
  A = \bigcup_{n=1}^{\infty} E_n.
\]
Since $\mu$ is a measure,
\[
  \lim_{n \to \infty} \mu(E_n) = \mu(A)
\]
(Theorem 11.3).

\item[(3)]
(Reductio ad absurdum)
If $\mu(A)> 0$, there is an integer $N$ such that
$\mu(E_n) \geq \frac{\mu(A)}{2}$ whenever $n \geq N$ (by (2)).
In particular, take $n = N$ to get
\begin{align*}
  \int_E f d\mu
  &\geq \int_{E_N} f d\mu
    &\text{($\mu$ is a measure and $E_N \subseteq E$)}\\
  &\geq \frac{1}{N} \cdot \mu(E_N)
    &\text{(Remarks 11.23(b))} \\
  &\geq \frac{1}{N} \cdot \frac{\mu(A)}{2} \\
  & > 0,
\end{align*}
contrary to the assumption that $\int_{E} f d\mu = 0$.
\end{enumerate}
$\Box$ \\

\emph{Note.}
Compare to Exercise 6.2. \\\\



%%%%%%%%%%%%%%%%%%%%%%%%%%%%%%%%%%%%%%%%%%%%%%%%%%%%%%%%%%%%%%%%%%%%%%%%%%%%%%%%
%%%%%%%%%%%%%%%%%%%%%%%%%%%%%%%%%%%%%%%%%%%%%%%%%%%%%%%%%%%%%%%%%%%%%%%%%%%%%%%%



\end{document}