\documentclass{article}
\usepackage{amsfonts}
\usepackage{amsmath}
\usepackage{amssymb}
\usepackage{hyperref}
\usepackage[none]{hyphenat}
\usepackage{mathrsfs}
\usepackage{physics}
\parindent=0pt

\def\upint{\mathchoice%
    {\mkern13mu\overline{\vphantom{\intop}\mkern7mu}\mkern-20mu}%
    {\mkern7mu\overline{\vphantom{\intop}\mkern7mu}\mkern-14mu}%
    {\mkern7mu\overline{\vphantom{\intop}\mkern7mu}\mkern-14mu}%
    {\mkern7mu\overline{\vphantom{\intop}\mkern7mu}\mkern-14mu}%
  \int}
\def\lowint{\mkern3mu\underline{\vphantom{\intop}\mkern7mu}\mkern-10mu\int}

\begin{document}

\textbf{\Large Chapter 11: The Lebesuge Theory} \\\\



\emph{Author: Meng-Gen Tsai} \\
\emph{Email: plover@gmail.com} \\\\



%%%%%%%%%%%%%%%%%%%%%%%%%%%%%%%%%%%%%%%%%%%%%%%%%%%%%%%%%%%%%%%%%%%%%%%%%%%%%%%%
%%%%%%%%%%%%%%%%%%%%%%%%%%%%%%%%%%%%%%%%%%%%%%%%%%%%%%%%%%%%%%%%%%%%%%%%%%%%%%%%



\textbf{Exercise 11.1.}
\emph{If $f \geq 0$ and $\int_{E} f d\mu = 0$,
prove that $f(x) = 0$ almost everywhere on $E$.
(Hint: Let $E_n$ be the subset of $E$ on which $f(x) > \frac{1}{n}$.
Write $A = \bigcup E_n$.
Then $\mu(A) = 0$ if and only if $\mu(E_n) = 0$ for every $n$.)} \\

Might assume that $f$ is measurable on $E$. \\

\emph{Proof (Hint).}
\begin{enumerate}
\item[(1)]
Define $A = \{ x \in E : f(x) > 0 \}$.
So $f(x) = 0$ almost everywhere on $E$ if and only if $\mu(A) = 0$.

\item[(2)]
Define
\[
  E_n = \left\{ x \in E : f(x) > \frac{1}{n} \right\}
\]
for $n = 1,2,3,\ldots$.
Note that $E_1 \subseteq E_2 \subseteq E_3 \subseteq \cdots$ and
\[
  A = \bigcup_{n=1}^{\infty} E_n.
\]
Since $\mu$ is a measure,
\[
  \lim_{n \to \infty} \mu(E_n) = \mu(A)
\]
(Theorem 11.3).

\item[(3)]
(Reductio ad absurdum)
If $\mu(A)> 0$, there is an integer $N$ such that
$\mu(E_n) \geq \frac{\mu(A)}{2}$ whenever $n \geq N$ (by (2)).
In particular, take $n = N$ to get
\begin{align*}
  \int_E f d\mu
  &\geq \int_{E_N} f d\mu
    &\text{($\mu$ is a measure and $E_N \subseteq E$)}\\
  &\geq \frac{1}{N} \cdot \mu(E_N)
    &\text{(Remarks 11.23(b))} \\
  &\geq \frac{1}{N} \cdot \frac{\mu(A)}{2} \\
  & > 0,
\end{align*}
contrary to the assumption that $\int_{E} f d\mu = 0$.
\end{enumerate}
$\Box$ \\

\emph{Note.}
Compare to Exercise 6.2. \\\\



%%%%%%%%%%%%%%%%%%%%%%%%%%%%%%%%%%%%%%%%%%%%%%%%%%%%%%%%%%%%%%%%%%%%%%%%%%%%%%%%



\textbf{Exercise 11.2.}
\emph{If $\int_A f d\mu = 0$ for every measurable subset $A$ of a measurable set $E$,
then $f(x) = 0$ almost everywhere on $E$.} \\

Might assume that $f$ is measurable on $E$. \\

\emph{Proof.}
\begin{enumerate}
\item[(1)]
  Define
  \[
    A = \{ x \in E : f(x) \geq 0 \}
    \qquad \text{ and } \qquad
    B = \{ x \in E : f(x) \leq 0 \}.
  \]
  $A$ and $B$ are measurable subsets of a measurable set $E$ since $f$ is measurable.

\item[(2)]
  Apply Exercise 11.1 to the fact that $f \geq 0$ on $A$ (by construction)
  and $\int_A f d\mu = 0$ (by assumption),
  we have $f(x) = 0$ almost everywhere on $A$.

\item[(3)]
  Similarly,
  apply Exercise 11.1 to the fact that $-f \geq 0$ on $B$
  and $\int_B (-f) d\mu = -\int_B f d\mu = 0$,
  we have $f(x) = 0$ almost everywhere on $B$.

\item[(4)]
  As $E = A \cup B$, $f(x) = 0$ almost everywhere on $E$ by (2)(3).
\end{enumerate}
$\Box$ \\\\



%%%%%%%%%%%%%%%%%%%%%%%%%%%%%%%%%%%%%%%%%%%%%%%%%%%%%%%%%%%%%%%%%%%%%%%%%%%%%%%%



\textbf{Exercise 11.3.}
\emph{If $\{f_n\}$ is a sequence of measurable functions,
prove that the set of points $x$ at which $\{f_n(x)\}$ converges is measurable.} \\

\emph{Proof.}
\begin{enumerate}
\item[(1)]
  It suffices to show that
  \[
    E
    = \{ x : \{ f_n(x) \} \text{ is convergent } \}
    = \{ x : \{ f_n(x) \} \text{ is Cauchy } \}
  \]
  is measurable (since $\mathbb{R}^1$ is complete).

\item[(2)]
  Write
  \[
    E
    =
    \bigcap_{k=1}^{\infty}
    \bigcup_{N=1}^{\infty}
    \bigcap_{n,m \geq N} \left\{ x : |f_n(x)-f_m(x)| \leq \frac{1}{k} \right\}
  \]
  Since $\{f_n\}$ is a sequence of measurable functions,
  $x \mapsto |f_n(x)-f_m(x)|$ is measurable
  (Theorem 11.16 and Theorem 11.18).
  Hence
  \[
    \left\{ x : |f_n(x)-f_m(x)| \leq \frac{1}{k} \right\}
  \]
  is measurable (Theorem 11.15).
  Therefore $E$ is measurable.
\end{enumerate}
$\Box$ \\\\



%%%%%%%%%%%%%%%%%%%%%%%%%%%%%%%%%%%%%%%%%%%%%%%%%%%%%%%%%%%%%%%%%%%%%%%%%%%%%%%%



\textbf{Exercise 11.4.}
\emph{If $f \in \mathscr{L}(\mu)$ on $E$ and $g$ is bounded and measurable on $E$,
then $fg \in \mathscr{L}(\mu)$ on $E$.} \\

\emph{Proof (Theorem 11.27).}
\begin{enumerate}
\item[(1)]
  $fg$ is measurable since both $f$ and $g$ are measurable (Theorem 11.18).

\item[(2)]
  $|g| \leq M$ for some real $M \in \mathbb{R}^1$ by the boundedness of $g$.
  Hence
  \[
    |fg| \leq M|f|
  \]
  on $E$.

\item[(3)]
  To apply Theorem 11.27, it suffices to show that
  $M|f| \in \mathscr{L}(\mu)$ on $E$.
  Theorem 11.26 implies that $|f| \in \mathscr{L}(\mu)$ if $f \in \mathscr{L}(\mu)$.
  And Remarks 11.23(d) implies that $M|f| \in \mathscr{L}(\mu)$ if $|f| \in \mathscr{L}(\mu)$.
\end{enumerate}
$\Box$ \\

\emph{Note.}
  It is not true for Riemann integrable functions:
  \emph{If $f \in \mathscr{R}$ on $[a,b]$ and $g$ is bounded and measurable on $[a,b]$,
  then $fg$ might be not Riemann integrable.} \\\\



%%%%%%%%%%%%%%%%%%%%%%%%%%%%%%%%%%%%%%%%%%%%%%%%%%%%%%%%%%%%%%%%%%%%%%%%%%%%%%%%



\textbf{Exercise 11.5.}
\emph{Put
  \begin{equation*}
    g(x) =
    \begin{cases}
      0 & (0 \leq x \leq \frac{1}{2}), \\
      1 & (\frac{1}{2} < x \leq 1),
    \end{cases}
  \end{equation*}
  and
  \begin{align*}
    f_{2k}(x) &= g(x) &(0 \leq x \leq 1), \\
    f_{2k+1}(x) &= g(1-x) &(0 \leq x \leq 1).
  \end{align*}
Show that
\[
  \liminf_{n \to \infty} f_n(x) = 0
  \qquad
  (0 \leq x \leq 1),
\]
but
\[
  \int_{0}^{1} f_n(x)dx = \frac{1}{2}.
\]
(Compare with the Fatou's theorem.)} \\

\emph{Proof.}
\begin{enumerate}
\item[(1)]
  \emph{Show that $\liminf_{n \to \infty} f_n(x) = 0$.}
  Note that
  \begin{equation*}
    g(1-x) =
    \begin{cases}
      1 & (0 \leq x < \frac{1}{2}), \\
      0 & (\frac{1}{2} < x \leq 1).
    \end{cases}
  \end{equation*}
  Since $f_n(x) \geq 0$ by definition,
  $\liminf_{n \to \infty} f_n(x) \geq 0$.
  Since $f_{2k}(0) = f_{2k+1}(1) = 0$ for all positive integers $k$,
  $\liminf_{n \to \infty} f_n(x) \leq 0$.
  Therefore the result is established.

\item[(2)]
  \emph{Show that $\int_{0}^{1} f_n(x)dx = \frac{1}{2}$.}
  Since
  \begin{align*}
    \int_{0}^{1} f_{2k}(x)dx
    &= \int_{0}^{1} g(x)dx = \frac{1}{2}, \\
    \int_{0}^{1} f_{2k+1}(x)dx
    &= \int_{0}^{1} g(1-x)dx = \frac{1}{2},
  \end{align*}
  in any case $\int_{0}^{1} f_n(x)dx = \frac{1}{2}$ for all positive integers $n$.

\item[(3)]
  This example shows that we may have the strict inequality in the Fatou's theorem.
\end{enumerate}
$\Box$ \\



\textbf{Supplement (Similar exercise).}
  \emph{Consider the sequence $\{f_n\}$ defined by $f_n(x) = 1$ if $n \leq x < n+1$,
  with $f_n(x) = 0$ otherwise.
  Show that we may have the strict inequality in the Fatou's theorem.}\\\\



%%%%%%%%%%%%%%%%%%%%%%%%%%%%%%%%%%%%%%%%%%%%%%%%%%%%%%%%%%%%%%%%%%%%%%%%%%%%%%%%



\textbf{Exercise 11.6.}
\emph{...} \\

\emph{Proof.}
\begin{enumerate}
\item[(1)]
\item[(2)]

\end{enumerate}
$\Box$ \\\\



%%%%%%%%%%%%%%%%%%%%%%%%%%%%%%%%%%%%%%%%%%%%%%%%%%%%%%%%%%%%%%%%%%%%%%%%%%%%%%%%



\textbf{Exercise 11.7.}
\emph{...} \\

\emph{Proof.}
\begin{enumerate}
\item[(1)]
\item[(2)]

\end{enumerate}
$\Box$ \\\\



%%%%%%%%%%%%%%%%%%%%%%%%%%%%%%%%%%%%%%%%%%%%%%%%%%%%%%%%%%%%%%%%%%%%%%%%%%%%%%%%



\textbf{Exercise 11.8.}
\emph{...} \\

\emph{Proof.}
\begin{enumerate}
\item[(1)]
\item[(2)]

\end{enumerate}
$\Box$ \\\\



%%%%%%%%%%%%%%%%%%%%%%%%%%%%%%%%%%%%%%%%%%%%%%%%%%%%%%%%%%%%%%%%%%%%%%%%%%%%%%%%



\textbf{Exercise 11.9.}
\emph{...} \\

\emph{Proof.}
\begin{enumerate}
\item[(1)]
\item[(2)]

\end{enumerate}
$\Box$ \\\\



%%%%%%%%%%%%%%%%%%%%%%%%%%%%%%%%%%%%%%%%%%%%%%%%%%%%%%%%%%%%%%%%%%%%%%%%%%%%%%%%



\textbf{Exercise 11.10.}
\emph{If $\mu(X) < +\infty$ and $f \in \mathscr{L}^2(\mu)$ on $X$,
prove that $f \in \mathscr{L}$ on $X$.
If
\[
  \mu(X) = +\infty,
\]
this is false. For instance, if
\[
  f(x) = \frac{1}{1+|x|},
\]
then $f^2 \in \mathscr{L}$ on $\mathbb{R}^1$,
but $f \not\in \mathscr{L}$ on $\mathbb{R}^1$.} \\

\emph{Proof.}
\begin{enumerate}
\item[(1)]
Since $\mu(X) < +\infty$, $1 \in \mathscr{L}^2(\mu)$ on $X$.
By Theorem 11.35, $f \in \mathscr{L}(\mu)$, and
\[
  \int_X |f| d\mu
  \leq \norm{f} \norm{1}.
\]

\item[(2)]

\end{enumerate}
$\Box$ \\\\



%%%%%%%%%%%%%%%%%%%%%%%%%%%%%%%%%%%%%%%%%%%%%%%%%%%%%%%%%%%%%%%%%%%%%%%%%%%%%%%%



\textbf{Exercise 11.11.}
\emph{...} \\

\emph{Proof.}
\begin{enumerate}
\item[(1)]
\item[(2)]

\end{enumerate}
$\Box$ \\\\



%%%%%%%%%%%%%%%%%%%%%%%%%%%%%%%%%%%%%%%%%%%%%%%%%%%%%%%%%%%%%%%%%%%%%%%%%%%%%%%%



\textbf{Exercise 11.12.}
\emph{...} \\

\emph{Proof.}
\begin{enumerate}
\item[(1)]
\item[(2)]

\end{enumerate}
$\Box$ \\\\



%%%%%%%%%%%%%%%%%%%%%%%%%%%%%%%%%%%%%%%%%%%%%%%%%%%%%%%%%%%%%%%%%%%%%%%%%%%%%%%%



\textbf{Exercise 11.13.}
\emph{...} \\

\emph{Proof.}
\begin{enumerate}
\item[(1)]
\item[(2)]

\end{enumerate}
$\Box$ \\\\



%%%%%%%%%%%%%%%%%%%%%%%%%%%%%%%%%%%%%%%%%%%%%%%%%%%%%%%%%%%%%%%%%%%%%%%%%%%%%%%%



\textbf{Exercise 11.14.}
\emph{...} \\

\emph{Proof.}
\begin{enumerate}
\item[(1)]
\item[(2)]

\end{enumerate}
$\Box$ \\\\



%%%%%%%%%%%%%%%%%%%%%%%%%%%%%%%%%%%%%%%%%%%%%%%%%%%%%%%%%%%%%%%%%%%%%%%%%%%%%%%%



\textbf{Exercise 11.15.}
\emph{...} \\

\emph{Proof.}
\begin{enumerate}
\item[(1)]
\item[(2)]

\end{enumerate}
$\Box$ \\\\



%%%%%%%%%%%%%%%%%%%%%%%%%%%%%%%%%%%%%%%%%%%%%%%%%%%%%%%%%%%%%%%%%%%%%%%%%%%%%%%%



\textbf{Exercise 11.16.}
\emph{...} \\

\emph{Proof.}
\begin{enumerate}
\item[(1)]
\item[(2)]

\end{enumerate}
$\Box$ \\\\



%%%%%%%%%%%%%%%%%%%%%%%%%%%%%%%%%%%%%%%%%%%%%%%%%%%%%%%%%%%%%%%%%%%%%%%%%%%%%%%%



\textbf{Exercise 11.17.}
\emph{...} \\

\emph{Proof.}
\begin{enumerate}
\item[(1)]
\item[(2)]

\end{enumerate}
$\Box$ \\\\



%%%%%%%%%%%%%%%%%%%%%%%%%%%%%%%%%%%%%%%%%%%%%%%%%%%%%%%%%%%%%%%%%%%%%%%%%%%%%%%%



\textbf{Exercise 11.18.}
\emph{...} \\

\emph{Proof.}
\begin{enumerate}
\item[(1)]
\item[(2)]

\end{enumerate}
$\Box$ \\\\



%%%%%%%%%%%%%%%%%%%%%%%%%%%%%%%%%%%%%%%%%%%%%%%%%%%%%%%%%%%%%%%%%%%%%%%%%%%%%%%%
%%%%%%%%%%%%%%%%%%%%%%%%%%%%%%%%%%%%%%%%%%%%%%%%%%%%%%%%%%%%%%%%%%%%%%%%%%%%%%%%



\end{document}