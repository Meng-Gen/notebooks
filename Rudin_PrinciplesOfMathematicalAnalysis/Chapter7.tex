\documentclass{article}
\usepackage{amsfonts}
\usepackage{amsmath}
\usepackage{amssymb}
\usepackage{bm}
\usepackage{hyperref}
\usepackage[none]{hyphenat}
\usepackage{mathrsfs}
\usepackage{physics}
\parindent=0pt

\def\upint{\mathchoice%
    {\mkern13mu\overline{\vphantom{\intop}\mkern7mu}\mkern-20mu}%
    {\mkern7mu\overline{\vphantom{\intop}\mkern7mu}\mkern-14mu}%
    {\mkern7mu\overline{\vphantom{\intop}\mkern7mu}\mkern-14mu}%
    {\mkern7mu\overline{\vphantom{\intop}\mkern7mu}\mkern-14mu}%
  \int}
\def\lowint{\mkern3mu\underline{\vphantom{\intop}\mkern7mu}\mkern-10mu\int}

\begin{document}



\textbf{\Large Chapter 7: Sequences and Series of Functions} \\\\



\emph{Author: Meng-Gen Tsai} \\
\emph{Email: plover@gmail.com} \\\\



% http://math.ucsd.edu/~lni/math140/HW140B_5_solutions.pdf
% https://minds.wisconsin.edu/bitstream/handle/1793/67009/rudin%20ch%207.pdf
% http://www.math.ust.hk/~majhu/Math204/Homework204_Set03.pdf
% https://docs.google.com/viewer?a=v&pid=sites&srcid=ZGVmYXVsdGRvbWFpbnxtYXRoc29sdXRpb25ndWlkZXN8Z3g6MzIxZGJmOTAzNzQzZTAyYw
% https://www.csie.ntu.edu.tw/~b89089/old/math/pma/pma7.html



%%%%%%%%%%%%%%%%%%%%%%%%%%%%%%%%%%%%%%%%%%%%%%%%%%%%%%%%%%%%%%%%%%%%%%%%%%%%%%%%
%%%%%%%%%%%%%%%%%%%%%%%%%%%%%%%%%%%%%%%%%%%%%%%%%%%%%%%%%%%%%%%%%%%%%%%%%%%%%%%%



\textbf{Exercise 7.1.}
\emph{Prove that every uniformly convergent sequence of bounded functions
is uniformly bounded.} \\

\emph{Proof (Cauchy criterion).}
Let $\{f_n\}$ be a uniformly convergent sequence of bounded functions.
\begin{enumerate}
\item[(1)]
Since $f_n$ is bounded, there exists a $M_n$ such that $|f_n(x)| \leq M_n$.

\item[(2)]
Since $\{f_n\}$ converges uniformly, given $1 > 0$ there exists an integer $N$
such that
\[
  |f_n(x) - f_m(x)| \leq 1 \text{ whenever } n, m \geq N
\]
(Theorem 7.8 (Cauchy criterion for uniformly convergence)).
Especially,
\[
  |f_n(x)| \leq |f_n(x) - f_N(x)| + |f_N(x)| \leq 1 + M_N \text{ whenever } n \geq N.
\]

\item[(3)]
Thus, $\{f_n\}$ is uniformly bounded by $M = \max\{ M_1, \ldots, M_{N-1}, M_{N}+1 \}$.
\end{enumerate}
$\Box$ \\\\



%%%%%%%%%%%%%%%%%%%%%%%%%%%%%%%%%%%%%%%%%%%%%%%%%%%%%%%%%%%%%%%%%%%%%%%%%%%%%%%%



\textbf{Exercise 7.2.}
\emph{If $\{f_n\}$ and $\{g_n\}$ converge uniformly on a set $E$,
prove that $\{f_n + g_n\}$ converge uniformly on $E$.
If, in addition, $\{f_n\}$ and $\{g_n\}$ are sequences of bounded functions,
prove that $\{f_n g_n\}$ converges uniformly on $E$.} \\

\emph{Proof.}
Let $f_n \to f$ uniformly and $g_n \to g$ uniformly.
\begin{enumerate}
  \item[(1)]
  \emph{Show that $\{f_n + g_n\}$ converges uniformly.}
    Given $\varepsilon > 0$.
    Since $f_n \to f$ uniformly and $g_n \to g$ uniformly,
    there exist two integers $N_1$ and $N_2$ such that
    \begin{align*}
      |f_n(x) - f(x)| \leq \frac{\varepsilon}{2}
      &\text{ whenever }
      n \geq N_1, x \in E \\
      |g_n(x) - g(x)| \leq \frac{\varepsilon}{2}
      &\text{ whenever }
      n \geq N_2, x \in E.
    \end{align*}
    Take $N = \max\{N_1,N_2\}$, we have
    \begin{align*}
      &|(f_n(x)+ g_n(x)) - (f(x) + g(x))| \\
      =& |(f_n(x) - f(x)) + (g_n(x) - g(x))| \\
      \leq& |f_n(x) - f(x)| + |g_n(x) - g(x)| \\
      \leq& \frac{\varepsilon}{2} + \frac{\varepsilon}{2} \\
      =& \varepsilon
    \end{align*}
    whenever $n \geq N$, $x \in E$.
    Hence $f_n + g_n \to f+g$ uniformly on $E$.

  \item[(2)]
  \emph{Show that $\{f_n g_n\}$ converges uniformly
  if, in addition, $\{f_n\}$ and $\{g_n\}$ are sequences of bounded functions.}
  Given $\varepsilon > 0$.
  \begin{enumerate}
    \item[(a)]
    By Exercise 7.1, both $\{f_n\}$ and $\{g_n\}$ are uniformly bounded.
    So there exist $M_1$ and $M_2$
    such that
    \[
      |f_n(x)| \leq M_1 \text{ and } |g_n(x)| \leq M_2
    \]
    for all $n$ and $x \in E$.
    Also, $|f(x)| \leq M_1 + 1$ and $|g(x)| \leq M_2 + 1$.

    \item[(b)]
    Since $f_n \to f$ uniformly and $g_n \to g$ uniformly,
    there exist two integers $N_1$ and $N_2$ such that
    \begin{align*}
      |f_n(x) - f(x)| \leq \frac{\varepsilon}{2(M_2 + 1)}
      &\text{ whenever }
      n \geq N_1, x \in E \\
      |g_n(x) - g(x)| \leq \frac{\varepsilon}{2(M_1 + 1)}
      &\text{ whenever }
      n \geq N_2, x \in E.
    \end{align*}
    (Note that each denominator of $\frac{\varepsilon}{2(M_j + 1)}$ $(j=1,2)$
    is well-defined and positive!)
    Take $N = \max\{N_1,N_2\}$, we have
    \begin{align*}
      &|f_n(x)g_n(x) - f(x)g(x)| \\
      =& |[f_n(x) - f(x)]g_n(x) + f(x)[g_n(x) - g(x)]| \\
      \leq& |f_n(x) - f(x)||g_n(x)| + |f(x)||g_n(x) - g(x)| \\
      \leq& \frac{\varepsilon}{2(M_2 + 1)} \cdot M_2
        + (M_1 + 1) \cdot \frac{\varepsilon}{2(M_1 + 1)} \\
      \leq& \varepsilon
    \end{align*}
    whenever $n \geq N$, $x \in E$.
    Hence $f_n g_n \to fg$ uniformly on $E$.
  \end{enumerate}
\end{enumerate}
$\Box$ \\

\emph{Proof (Cauchy criterion).}
\begin{enumerate}
  \item[(1)]
  \emph{Show that $\{f_n + g_n\}$ converges uniformly.}
    Given $\varepsilon > 0$.
    Since $\{f_n\}$ and $\{g_n\}$ converge uniformly,
    there exist two integers $N_1$ and $N_2$ such that
    \begin{align*}
      |f_n(x) - f_m(x)| \leq \frac{\varepsilon}{2}
      &\text{ whenever }
      n,m \geq N_1, x \in E \\
      |g_n(x) - g_m(x)| \leq \frac{\varepsilon}{2}
      &\text{ whenever }
      n,m \geq N_2, x \in E.
    \end{align*}
    Take $N = \max\{N_1,N_2\}$, we have
    \begin{align*}
      &|(f_n(x)+ g_n(x)) - (f_m(x) + g_m(x))| \\
      =& |(f_n(x) - f_n(x)) + (g_n(x) - g_m(x))| \\
      \leq& |f_n(x) - f_n(x)| + |g_n(x) - g_m(x)| \\
      \leq& \frac{\varepsilon}{2} + \frac{\varepsilon}{2} \\
      =& \varepsilon
    \end{align*}
    whenever $n,m \geq N$, $x \in E$.
    Hence $\{f_n + g_n\}$ converges uniformly on $E$.

  \item[(2)]
  \emph{Show that $\{f_n g_n\}$ converges uniformly
  if, in addition, $\{f_n\}$ and $\{g_n\}$ are sequences of bounded functions.}
  Given $\varepsilon > 0$.
  \begin{enumerate}
    \item[(a)]
    By Exercise 7.1, both $\{f_n\}$ and $\{g_n\}$ are uniformly bounded.
    So there exist $M_1$ and $M_2$
    such that
    \[
      |f_n(x)| \leq M_1 \text{ and } |g_n(x)| \leq M_2
    \]
    for all $n$ and $x \in E$.
    Also, $|f(x)| \leq M_1 + 1$ and $|g(x)| \leq M_2 + 1$.

    \item[(b)]
    Since $\{f_n\} \to f$ uniformly and $\{g_n\} \to g$ uniformly,
    there exist two integers $N_1$ and $N_2$ such that
    \begin{align*}
      |f_n(x) - f_m(x)| \leq \frac{\varepsilon}{2(M_2 + 1)}
      &\text{ whenever }
      n,m \geq N_1, x \in E \\
      |g_n(x) - g_m(x)| \leq \frac{\varepsilon}{2(M_1 + 1)}
      &\text{ whenever }
      n,m \geq N_2, x \in E.
    \end{align*}
    Take $N = \max\{N_1,N_2\}$, we have
    \begin{align*}
      &|f_n(x)g_n(x) - f_m(x)g_m(x)| \\
      =& |[f_n(x) - f_m(x)]g_n(x) + f_m(x)[g_n(x) - g_m(x)]| \\
      \leq& |f_n(x) - f_m(x)||g_n(x)| + |f_m(x)||g_n(x) - g_m(x)| \\
      \leq& \frac{\varepsilon}{2(M_2 + 1)} \cdot M_2
        + M_1 \cdot \frac{\varepsilon}{2(M_1 + 1)} \\
      \leq& \varepsilon
    \end{align*}
    whenever $n \geq N$, $x \in E$.
    Hence $\{f_n g_n\}$ converges uniformly on $E$.
  \end{enumerate}
\end{enumerate}
$\Box$ \\

\emph{Note.} It proved that $f_n g_n \to fg$ in Theorem 7.29.
\\\\



%%%%%%%%%%%%%%%%%%%%%%%%%%%%%%%%%%%%%%%%%%%%%%%%%%%%%%%%%%%%%%%%%%%%%%%%%%%%%%%%



\textbf{Exercise 7.3.}
\emph{Construct sequences $\{f_n\}$, $\{g_n\}$ which converge uniformly on some set $E$,
but such that $\{f_n g_n\}$ does not converge uniformly on $E$
(of course, $\{f_n g_n\}$ must converge on $E$).} \\

We provides some examples here. \\

\emph{Proof ($f_n(x) = x + \frac{1}{n}$).}
\begin{enumerate}
  \item[(1)]
  Define $\{f_n(x)\}$ on $E = \mathbb{R}$ by $f_n(x) = x + \frac{1}{n}$ and $f(x) = x$.
  Clearly, $\{f_n(x)\}$ converges to $f(x)$ pointwise.

  \item[(2)]
  \emph{Show that $\{f_n\}$ converges uniformly.}
  Given $\varepsilon > 0$.
  There exists an integer $N \geq \frac{1}{\varepsilon}$ such that
  \[
    |f_n(x) - f(x)| = \frac{1}{n} \leq \frac{1}{N} \leq \varepsilon
  \]
  whenever $n \geq N$ and $x \in E$.
  Hence $\{f_n\} \to f$ uniformly.

  \item[(3)]
  \emph{Show that $\{f_n^2\}$ does not converge uniformly.}
  Clearly, $\{f_n(x)^2\}$ converges to $f(x)^2$ pointwise.
  Hence
  \[
    \sup_{x \in E} |f_n(x)^2 - f(x)^2|
    = \sup_{x \in E} \abs{\frac{2x}{n} + \frac{1}{n^2}}
    \to \infty
  \]
  as $n \to \infty$ (by considering $x = n^2 \in E$).
  Hence $\{f_n^2 \}$ does not converge uniformly (Theorem 7.9).
\end{enumerate}
$\Box$ \\

\emph{Proof ($f_n(x) = \frac{1}{x}$, $g_n(x) = \frac{1}{n}$).}
\begin{enumerate}
  \item[(1)]
  Let $E = (0,1)$.
  Let $\{f_n(x)\}$ on $E$ be $f_n(x) = \frac{1}{x}$
  and $\{g_n(x)\}$ on $E$ be $g_n(x) = \frac{1}{n}$.
  Clearly, $\{f_n(x)\}$ converges to $f(x) = \frac{1}{x}$ pointwise
  and $\{g_n(x)\}$ converges to $g(x) = 0$ pointwise.

  \item[(2)]
  \emph{Show that $\{f_n\}$ converges uniformly.}
  Given $\varepsilon > 0$.
  There exists an integer $N = 1$ such that
  \[
    |f_n(x) - f(x)| = 0 \leq \varepsilon
  \]
  whenever $n \geq N$ and $x \in E$.
  Hence $\{f_n\} \to f$ uniformly.

  \item[(3)]
  \emph{Show that $\{g_n\}$ converges uniformly.}
  Given $\varepsilon > 0$.
  There exists an integer $N \geq \frac{1}{\varepsilon}$ such that
  \[
    |g_n(x) - g(x)| = \frac{1}{n} \leq \frac{1}{N} \leq \varepsilon
  \]
  whenever $n \geq N$ and $x \in E$.
  Hence $\{g_n\} \to g$ uniformly.

  \item[(4)]
  \emph{Show that $\{f_n g_n\}$ does not converge uniformly.}
  Clearly, $\{f_n(x) g_n(x) \}$ converges to $f(x) g(x) = 0$ pointwise.
  Hence
  \[
    \sup_{x \in E} |f_n(x) g_n(x) - 0|
    = \sup_{x \in E} \abs{\frac{1}{nx}}
    \to \infty
  \]
  as $n \to \infty$ (by considering $x = \frac{1}{n^2} \in E$).
  Hence $\{f_n g_n \}$ does not converge uniformly (Theorem 7.9).
\end{enumerate}
$\Box$ \\

\emph{Proof (Exercise 9.2 in Tom M. Apostol, Mathematical Analysis, 2nd edition).}
\begin{enumerate}
  \item[(1)]
  Let $E = [\alpha,\beta] \subseteq \mathbb{R}$ be a bounded interval.
  Define two sequences $\{f_n\}$ and $\{g_n\}$ on $E$ as follows:
  \[
    f_n(x) = x \left( 1+\frac{1}{n} \right)
    \text{ if $x \in \mathbb{R}$, $n = 1,2,\cdots$},
  \]
  \begin{equation*}
  g_n(x) =
    \begin{cases}
      \frac{1}{n}   & \text{ if $x=0$ or if $x$ is irrational}, \\
      b+\frac{1}{n} & \text{ if $x$ is rational $\neq 0$, say $x=\frac{a}{b}$, $b>0$}.
    \end{cases}
  \end{equation*}
  Here we assume that $\mathrm{gcd}(a,b) = 1$.
  Clearly, $f(x) = x$ and
  \begin{equation*}
  g(x) =
    \begin{cases}
      0 & \text{ if $x=0$ or if $x$ is irrational}, \\
      b & \text{ if $x$ is rational $\neq 0$, say $x=\frac{a}{b}$, $b>0$}.
    \end{cases}
  \end{equation*}
  Let $M = \max\{|\alpha|,|\beta|\} \geq 0$.

  \item[(2)]
  \emph{Show that $\{f_n\}$ converges uniformly.}
  Given $\varepsilon > 0$.
  There exists an integer $N \geq \frac{M}{\varepsilon}$ such that
  \[
    |f_n(x) - f(x)| = \frac{|x|}{n} \leq \frac{M}{N} \leq \varepsilon
  \]
  whenever $n \geq N$ and $x \in E$.
  Hence $\{f_n\} \to f$ uniformly.

  \item[(3)]
  \emph{Show that $\{g_n\}$ converges uniformly.}
  Given $\varepsilon > 0$.
  There exists an integer $N \geq \frac{1}{\varepsilon}$ such that
  \[
    |g_n(x) - g(x)| = \frac{1}{n} \leq \frac{1}{N} \leq \varepsilon
  \]
  whenever $n \geq N$ and $x \in E$.
  Hence $\{g_n\} \to g$ uniformly.

  \item[(4)]
  \emph{Show that $\{f_n g_n\}$ does not converge uniformly.}
  \begin{enumerate}
    \item[(a)]
      Clearly, $\{f_n(x) g_n(x) \}$ converges to $f(x)g(x)$ pointwise
      where
      \begin{equation*}
      f(x) g(x) =
        \begin{cases}
          0 & \text{ if $x=0$ or if $x$ is irrational}, \\
          a & \text{ if $x=\frac{a}{b}$ is rational $\neq 0$, $b>0$}.
        \end{cases}
      \end{equation*}

    \item[(b)]
      Note that
      \begin{equation*}
      f_n(x) g_n(x) =
        \begin{cases}
          \frac{x}{n} \left( 1 + \frac{1}{n} \right)
            & \text{ if $x=0$ or if $x$ is irrational}, \\
          \left( a + \frac{x}{n} \right)\left( 1 + \frac{1}{n} \right)
            & \text{ if $x=\frac{a}{b}$ is rational $\neq 0$, $b>0$}.
        \end{cases}
      \end{equation*}
      Therefore,
      \begin{equation*}
      f_n(x) g_n(x) - f(x) g(x) =
        \begin{cases}
          \frac{x}{n} \left( 1 + \frac{1}{n} \right)
            & \text{ if $x=0$ or if $x$ is irrational}, \\
          \frac{x}{n} \left( 1 + b + \frac{1}{n} \right)
            & \text{ if $x=\frac{a}{b}$ is rational $\neq 0$, $b>0$}.
        \end{cases}
      \end{equation*}

    \item[(c)]
      Hence
      \begin{align*}
        \sup_{x \in E} |f_n(x) g_n(x) - f(x) g(x)|
        &\geq \sup_{x \in E \cap \mathbb{Q}} |f_n(x) g_n(x) - f(x) g(x)| \\
        &= \sup_{x \in E \cap \mathbb{Q}}
          \abs{a} \left( \frac{1}{n} + \frac{1}{bn} + \frac{1}{bn^2} \right) \\
        &\geq \sup_{x \in E \cap \mathbb{Q}}
          \abs{a} \left( \frac{1}{n} \right) \\
        &= \sup_{x \in E \cap \mathbb{Q}} \frac{\abs{a}}{n}.
      \end{align*}

    \item[(d)]
      \emph{Given any irrational number $\gamma \in E$,
      there exists a sequence
      \[
        \left\{ r_m = \frac{a_m}{b_m} \right\}
      \]
      of nonzero rational numbers in $E$ such that $\lim r_m = \gamma$.
      Show that $\{a_m\}$ is unbounded.}
      If it is true, we can find $x_n = r_{m_n} = \frac{a_{m_n}}{b_{m_n}}$
      such that $|a_{m_n}| \geq n^2$ and
      \[
        \sup_{x \in E} |f_n(x) g_n(x) - f(x) g(x)|
        \geq \sup_{x \in E \cap \mathbb{Q}} \frac{\abs{a}}{n}
        \geq \frac{n^2}{n}
        = n \to \infty
      \]
      as $n \to \infty$.

    \item[(e)]
      (Reductio ad absurdum)
      If $\{a_m\}$ were bounded, then there exists
      a \textbf{constant} subsequence of $\{a_{m_k}\}$
      such that $\lim a_{m_k} = a \in \mathbb{Z}$.
      Since $\lim_{m \to \infty} r_m = \gamma$, $\lim_{k \to \infty} r_{m_k} = \gamma$ or
      \[
        \lim_{k \to \infty} b_{m_k}
        = \lim_{k \to \infty} \frac{a_{m_k}}{r_{m_k}}
        = \frac{a}{\gamma}
      \]
      (it is well-defined since $r_{m_k}$ and $\gamma$ cannot be zero).
      Since all $b_{m_k}$ are positive integers,
      the limit $\lim b_{m_k} = b$ is a positive integer too,
      or $b = \frac{a}{\gamma} \in \mathbb{Z}^+$, or $\gamma = \frac{a}{b} \in \mathbb{Z}$,
      which is absurd.
  \end{enumerate}
  Therefore, $\{f_n g_n\}$ does not converge uniformly.
\end{enumerate}
$\Box$ \\\\



%%%%%%%%%%%%%%%%%%%%%%%%%%%%%%%%%%%%%%%%%%%%%%%%%%%%%%%%%%%%%%%%%%%%%%%%%%%%%%%%



\textbf{Exercise 7.4.}
\emph{Consider
\[
  f(x) = \sum_{n=1}^{\infty} \frac{1}{1+n^2x}.
\]
For what values of $x$ does the series converge absolutely?
On what intervals does it converge uniformly?
On what intervals does it fail to converge uniformly?
Is $f$ continuous whenever the series converges?
Is $f$ bounded?} \\

\emph{Note.}
In the sense of Definition 2.17, the interval is always closed.
We might consider all closed intervals, open intervals, and half-open intervals
together. \\

\emph{Proof.}
\begin{enumerate}
\item[(1)]
  $f(x)$ is well-defined on
  \[
    E = \mathbb{R} - \left\{ -\frac{1}{m^2} : m = 1,2,3,\ldots \right\}.
  \]

\item[(2)]
  \emph{Show that $f(x)$ converges absolutely if and only if $x \in E - \{0\}$.}
  \begin{enumerate}
    \item[(a)]
    The case $x > 0$.
    Consider
    \[
      \sum \abs{\frac{1}{1+n^2x}}
      = \sum \frac{1}{1+n^2x}
      = x^{-1} \sum \frac{1}{n^2 + x^{-1}}.
    \]
    Since $\frac{1}{n^2 + x^{-1}} < \frac{1}{n^2}$ and $\sum \frac{1}{n^2} < \infty$,
    $\sum \abs{\frac{1}{1+n^2x}}$ converges
    (Theorem 3.25(a)(the comparison test) and Theorem 3.28).

    \item[(b)]
    The case $x = 0$.
    $f(0) = \sum 1 = \infty$ diverges.

    \item[(c)]
    The case $x < 0$ and $x \neq -\frac{1}{m^2}$ for any integer $m > 0$.
    There is an integer $N > 0$ such that
    \[
      \frac{1}{n^2 + x^{-1}} > 0
    \]
    whenever $n \geq N$.
    (We might take $N > (-x^{-1})^{\frac{1}{2}}$.)
    Now consider
    \begin{align*}
      \sum_{n=1}^{\infty} \abs{\frac{1}{1+n^2x}}
      &= \sum_{n=1}^{N-1} \abs{\frac{1}{1+n^2x}}
        + \sum_{n=N}^{\infty} \abs{\frac{1}{1+n^2x}} \\
      &= \sum_{n=1}^{N-1} \abs{\frac{1}{1+n^2x}}
        + \abs{x^{-1}} \sum_{n=N}^{\infty} \frac{1}{n^2 + x^{-1}}.
    \end{align*}
    Here $\sum_{n=1}^{N-1} \abs{\frac{1}{1+n^2x}}$ is a fixed number
    and $\sum_{n=N}^{\infty} \abs{\frac{1}{1+n^2x}}$ converges
    (as in (a)).
    Hence $\sum_{n=1}^{\infty} \abs{\frac{1}{1+n^2x}}$ converges.
  \end{enumerate}

\item[(3)]
  \emph{Show that $f$ is unbounded on $(0,b) \subseteq E$ where $b > 0$.}
  For any integer $N > 0$,
  \begin{align*}
    f\left(\frac{1}{N^2}\right)
    &= \sum_{n=1}^{\infty} \frac{N^2}{n^2 + N^2} \\
    &> \sum_{n=1}^{N} \frac{N^2}{n^2 + N^2} \\
    &\geq \sum_{n=1}^{N} \frac{N^2}{N^2 + N^2} \\
    &= \frac{N}{2}.
  \end{align*}
  So $\frac{1}{N^2} \in (0,b)$ as $N \to \infty$ and
  $f\left(\frac{1}{N^2}\right) \to \infty$ as $N \to \infty$.

\item[(4)]
  To find on what intervals, say $I$, $f(x)$ converges uniformly, by (1)(2)
  there are only three possible cases:
  \begin{enumerate}
    \item[(a)]
      $I \subseteq (0,\infty)$.
      \begin{enumerate}
        \item[(i)]
          \emph{Show that $f$ does not converge uniformly on $(0,b)$ if $b > 0$.}
          Consider the $N$th partial sum
          \[
            s_N(x) = \sum_{n=1}^{N} \frac{1}{1+n^2x}
          \]
          on $(0,b)$.
          $s_N(x)$ is bounded by $N$ since each term $\frac{1}{1+n^2x}$ is less than $1$.
          Since $f$ is unbounded on $(0,b)$ (by (3)),
          $f$ does not converge uniformly on $(0,b)$ (Exercise 7.1).

        \item[(ii)]
          \emph{Show that $f$ converges uniformly on $[a,\infty)$ if $a > 0$.}
          For each term $\frac{1}{1+n^2x}$ of $f(x)$ on $[a,\infty)$, we have
          \[
            \abs{\frac{1}{1+n^2x}}
            \leq \frac{1}{1+n^2a}
            < \frac{1}{n^2a}.
          \]
          Since $\sum \frac{1}{n^2a} = a \sum \frac{1}{n^2}$ converges,
          $f$ converges uniformly (Theorem 7.10).
      \end{enumerate}

    \item[(b)]
      $I \subseteq \left(-\frac{1}{m^2}, -\frac{1}{(m+1)^2}\right)$
      for some integer $m > 0$.
      \emph{Show that $f$ converges uniformly on
      $\left(-\frac{1}{m^2}, -\frac{1}{(m+1)^2}\right)$ for all integer $m > 0$.}
      For $n$th term $\frac{1}{1+n^2x}$ of $f(x)$ on $(-\infty,-1)$, we have
      \[
        \abs{\frac{1}{1+n^2x}}
        \leq \frac{(m+1)^2}{n^2 - (m+1)^2}
        \leq \frac{(m+1)^2}{(n-(m+1))^2}
      \]
      if $n > m+1$.
      Since
      $\sum_{n=m+2}^{\infty} \frac{(m+1)^2}{(n-(m+1))^2} = \frac{(m+1)^2\pi^2}{6}$
      converges,
      \[
        f(x) = \sum_{n=1}^{m+1} \frac{1}{1+n^2x} + \sum_{n=m+2}^{\infty} \frac{1}{1+n^2x}
      \]
      converges uniformly (Theorem 7.10).

    \item[(c)]
      $I \subseteq (-\infty,-1)$.
      \emph{Show that $f$ converges uniformly on $(-\infty,-1)$.}
      Similar to (b).
      For $n$th term $\frac{1}{1+n^2x}$ of $f(x)$ on $(-\infty,-1)$, we have
      \[
        \abs{\frac{1}{1+n^2x}}
        \leq \frac{1}{n^2 - 1}
        \leq \frac{1}{(n-1)^2}
      \]
      if $n > 1$.
      Since $\sum_{n=2}^{\infty} \frac{1}{(n-1)^2} = \frac{\pi^2}{6}$ converges,
      \[
        f(x) = \frac{1}{1+x} + \sum_{n=2}^{\infty} \frac{1}{1+n^2x}
      \]
      converges uniformly (Theorem 7.10).

  \end{enumerate}

\item[(4)]
  \emph{Show that $f$ is continuous whenever the series converges.}

\end{enumerate}
$\Box$ \\\\



%%%%%%%%%%%%%%%%%%%%%%%%%%%%%%%%%%%%%%%%%%%%%%%%%%%%%%%%%%%%%%%%%%%%%%%%%%%%%%%%



\textbf{Exercise 7.5.}
\emph{Let
\begin{equation*}
  f_n(x) =
    \begin{cases}
      0                    & (x < \frac{1}{n+1}), \\
      \sin^2 \frac{\pi}{x} & (\frac{1}{n+1} \leq x \leq \frac{1}{n}), \\
      0                    & (\frac{1}{n} < x).
    \end{cases}
\end{equation*}
Show that $\{f_n\}$ converges to a continuous function, but not uniformly.
Use the series $\sum f_n$ to show that absolute convergence, even for all $x$,
does not imply uniform convergence.} \\

\emph{Proof.}
\begin{enumerate}
\item[(1)]
\emph{Show that $\lim_{n \to \infty} f_n(x) = 0$.
Hence $\{f_n\}$ converges to a continuous function $0$ pointwise.}
Clearly, $f_n(x) = 0$ for all $x \not\in (0,1)$.
Next, for any fixed $x \in (0,1)$, there exists an integer $N > \frac{1}{x}$
such that
\[
  x > \frac{1}{N} \geq \frac{1}{n}
\]
whenever $n \geq N$.
Hence $f_n(x) = 0$ whenever $n \geq N$.

\item[(2)]
\emph{Show that $f_n \to f = 0$ not uniformly.}
Let
\[
  x_n = \frac{1}{n+\frac{1}{2}} \to 0
\]
for all $n=1,2,3,\ldots$.
Thus, $f_m(x_n) = \delta_{mn}$, where $\delta_{mn}$ is Kronecker delta.
  \begin{enumerate}
  \item[(a)]
  \emph{(Definition 7.7.)}
  (Reductio ad absurdum)
  If $\{f_n\}$ were convergent uniformly, then
  given $\varepsilon = \frac{1}{64} > 0$,
  there exists an integer $N$ such that $n \geq N$ implies
  \[
    |f_n(x) - f(x)| \leq \frac{1}{64}
  \]
  for all real $x$.
  However,
  \[
    |f_N(x_N) - f(x_N)| =  1 > \frac{1}{64},
  \]
  which is absurd.

  \item[(b)]
  \emph{(Theorem 7.8)}
  (Reductio ad absurdum)
  If $\{f_n\}$ were convergent uniformly, then
  given $\varepsilon = \frac{1}{64} > 0$,
  there exists an integer $N$ such that $n,m \geq N$ implies
  \[
    |f_n(x) - f_m(x)| \leq \frac{1}{64}
  \]
  for all real $x$.
  However,
  \[
    |f_N(x_{N}) - f_{N+1}(x_{N})| =  1 > \frac{1}{64},
  \]
  which is absurd.

  \item[(c)]
  \emph{(Theorem 7.9)}
  Since
  \[
    M_n
    = \sup_{x \in \mathbb{R}}|f_n(x) - f(x)|
    \geq |f_n(x_n) - f(x_n)| = 1,
  \]
  $f_n \to f$ not uniformly.

  \item[(d)]
  \emph{(Exercise 7.9.)}
  Since
  each $f_n$ is continuous and
  \[
    \lim_{n \to \infty} f_n(x_n) = \lim_{n \to \infty} 1 = 1 \neq 0 = f(0),
  \]
  $f_n \to f = 0$ not uniformly.
  \end{enumerate}

\item[(3)]
\emph{Show that $\sum f_n$ converges absolutely.}
Write $F_n = \sum_{k=1}^{n} f_k$ and $F = \sum f_n$.
Clearly,
\begin{equation*}
  F(x) =
    \begin{cases}
      0                    & (x \leq 0), \\
      \sin^2 \frac{\pi}{x} & (0 < x \leq 1), \\
      0                    & (x \geq 1).
    \end{cases}
\end{equation*}
Note that $f_n \geq 0$ for each $n$.
Hence $\sum f_n$ converges absolutely.

\item[(4)]
\emph{Show that $\sum f_n$ does not converge uniformly.}
Similar to (2).
Let
\[
  x_n = \frac{1}{n+\frac{1}{2}} \to 0
\]
for all $n=1,2,3,\ldots$.
Thus
\begin{equation*}
  F_m(x_n) =
    \begin{cases}
      1 & (m \geq n), \\
      0 & (m < n).
    \end{cases}
\end{equation*}

  \begin{enumerate}
  \item[(a)]
  \emph{(Definition 7.7.)}
  (Reductio ad absurdum)
  If $\{F_n\}$ were convergent uniformly, then
  given $\varepsilon = \frac{1}{64} > 0$,
  there exists an integer $N$ such that $n \geq N$ implies
  \[
    |F_n(x) - F(x)| \leq \frac{1}{64}
  \]
  for all real $x$.
  However,
  \[
    |F_N(x_{N+1}) - F(x_{N+1})| =  1 > \frac{1}{64},
  \]
  which is absurd.

  \item[(b)]
  \emph{(Theorem 7.8)}
  (Reductio ad absurdum)
  If $\{F_n\}$ were convergent uniformly, then
  given $\varepsilon = \frac{1}{64} > 0$,
  there exists an integer $N$ such that $n,m \geq N$ implies
  \[
    |F_n(x) - F_m(x)| \leq \frac{1}{64}
  \]
  for all real $x$.
  However,
  \[
    |F_N(x_{N+1}) - F_{N+1}(x_{N+1})| =  1 > \frac{1}{64},
  \]which is absurd.

  \item[(c)]
  \emph{(Theorem 7.9)}
  Since
  \[
    M_n
    = \sup_{x \in \mathbb{R}}|F_n(x) - F(x)|
    \geq |F_n(x_{n+1}) - F(x_{n+1})| = 1,
  \]
  $F_n \to F$ not uniformly.

  \item[(d)]
  \emph{(Exercise 7.9.)}
  Since
  each $F_n$ is continuous and
  \[
    \lim_{n \to \infty} F_n(x_{n+1}) = \lim_{n \to \infty} 0 \neq 1 = F(x_{n+1}),
  \]
  $F_n \to F$ not uniformly.

  \item[(e)]
  \emph{(Theorem 7.12.)}
  (Reductio ad absurdum)
  If $\{F_n\}$ were converging to $F$ uniformly, then
  $F$ were continuous since each $F_n$ is continuous by Theorem 7.12.
  However, $F$ is not continuous at $x = 0$.
  \end{enumerate}
\end{enumerate}
$\Box$ \\\\



%%%%%%%%%%%%%%%%%%%%%%%%%%%%%%%%%%%%%%%%%%%%%%%%%%%%%%%%%%%%%%%%%%%%%%%%%%%%%%%%



\textbf{Exercise 7.6.}
\emph{Prove that the series
\[
  \sum_{n=1}^{\infty} (-1)^n \frac{x^2+n}{n^2}
\]
converges uniformly in every bounded interval,
but does not converge absolutely for any value of $x$.} \\

\emph{Proof (Dirichlet's test).}
Given any bounded interval $E = [\alpha,\beta] \subseteq \mathbb{R}$.
Write $f_n(x) = (-1)^n$ on $E$ and $g_n(x) = \frac{x^2+n}{n^2}$ on $E$.
\begin{enumerate}
  \item[(1)]
  The partial sums $F_n(x)$ of $\sum f_n(x)$ form a uniformly bounded sequence.

  \item[(2)]
  $g_1(x) \geq g_2(x) \geq \cdots$ since
  \[
    g_{n+1}(x)
    = \frac{x^2}{(n+1)^2} + \frac{1}{n+1}
    < \frac{x^2}{n^2} + \frac{1}{n}
    = g_n(x).
  \]

  \item[(3)]
  Write $M = \max\{|\alpha|,|\beta|\}$.
  Since
  \[
    |g_n(x)|
    = \frac{x^2}{n^2} + \frac{1}{n}
    \leq \frac{M^2}{n^2} + \frac{1}{n} \to \infty
  \]
  as $n \to \infty$,
  $\lim_{n \to \infty} g_n(x) = 0$.
  By Dirichlet's test (Exercise 7.11),
  $\sum_{n=1}^{\infty} f_n(x) g_n(x) = \sum_{n=1}^{\infty} (-1)^n \frac{x^2+n}{n^2}$
  converges.

  \item[(4)]
  \begin{align*}
    \sum |f_n(x)|
    &= \sum \frac{x^2+n}{n^2} \\
    &\geq \sum \frac{n}{n^2} \\
    &= \sum \frac{1}{n} \to \log n + \gamma
  \end{align*}
  (Exercise 8.9).
  Hence $\sum (-1)^n \frac{x^2+n}{n^2}$ does not converge absolutely for any value of $x$.
\end{enumerate}
$\Box$ \\\\



%%%%%%%%%%%%%%%%%%%%%%%%%%%%%%%%%%%%%%%%%%%%%%%%%%%%%%%%%%%%%%%%%%%%%%%%%%%%%%%%



\textbf{Exercise 7.7.}
\emph{For $n=1,2,3,\ldots$, $x$ real, put
\[
  f_n(x) = \frac{x}{1+nx^2}.
\]
Show that $\{f_n\}$ converges uniformly to a function $f$,
and that the equation
\[
  f'(x) = \lim_{n \to \infty} f_n'(x)
\]
is correct if $x \neq 0$, but false if $x = 0$.} \\

$f_n(x)$ is defined on $\mathbb{R}$. \\

\emph{Proof.}
\begin{enumerate}
  \item[(1)]
  Since
  \[
    \abs{ f_n(x) }
    = \abs{ \frac{x}{1+nx^2} }
    \leq \frac{|x|}{\sqrt{n}|x|}
    = \frac{1}{\sqrt{n}} \to \infty
  \]
  as $n \to \infty$, $f_n \to 0$ uniformly (Theorem 7.9).

  \item[(2)]
  Clearly, $f'(x) = 0$.
  Since
  \[
    f_n'(x) = \frac{1-nx^2}{(1+nx^2)^2},
  \]
  \begin{equation*}
  \lim_{n \to \infty} f_n'(x) =
    \begin{cases}
      1 & (x = 0), \\
      0 & (x \neq 0).
    \end{cases}
  \end{equation*}
  So that the equation
  \[
    f'(x) = \lim_{n \to \infty} f_n'(x)
  \]
  is correct if $x \neq 0$, but false if $x = 0$.
\end{enumerate}
$\Box$ \\

\emph{Note.}
$f_n'(x)$ does not converge uniformly by considering
\[
  \lim_{n \to \infty} f_n'\left(\frac{1}{n}\right)
  = \lim_{n \to \infty} \frac{1-\frac{1}{n}}{(1+\frac{1}{n})^2}
  = 1.
\]
\\\\



%%%%%%%%%%%%%%%%%%%%%%%%%%%%%%%%%%%%%%%%%%%%%%%%%%%%%%%%%%%%%%%%%%%%%%%%%%%%%%%%



\textbf{Exercise 7.8.}
\emph{If
  \begin{equation*}
  I(x) =
    \begin{cases}
      0 & (x \leq 0), \\
      1 & (x > 0),
    \end{cases}
  \end{equation*}
if $\{ x_n \}$ is a sequence of distinct points of $(a,b)$,
and if $\sum|c_n|$ converges,
prove that the series
\[
  f(x) = \sum_{n=1}^{\infty} c_n I(x-x_n)
  \qquad
  (a \leq x \leq b)
\]
converges uniformly,
and that $f$ is continuous for every $x \neq x_n$.} \\

\emph{Proof.}
\begin{enumerate}
\item[(1)]
Define $f_n(x) = c_n I(x-x_n)$ on $(a,b)$. So
\[
  |f_n(x)| = |c_n| |I(x-x_n)| \leq |c_n|
  \qquad
  (x \in (a,b), n = 1,2,3,\ldots).
\]
Since $\sum|c_n|$ converges, $f = \sum f_n$ converges uniformly (Theorem 7.10).

\item[(2)]
Given any $p \in (a,b)$ with $p \neq x_n$ for all $n=1,2,3,\ldots$.
So each $I(x-x_n)$ is continuous at $x=p$, and thus
each partial sum $\sum_{n=1}^{N} f_n(x)$ is continuous.

\item[(3)]
By Theorem 7.11
\begin{align*}
  \lim_{x \to p} f(x)
  &= \lim_{x \to p} \sum_{n=1}^{\infty} f_n(x) \\
  &= \lim_{N \to \infty} \left( \lim_{x \to p} \sum_{n=1}^{N} f_n(x) \right) \\
  &= \lim_{N \to \infty} \sum_{n=1}^{N} f_n(p) \\
  &= \sum_{n=1}^{\infty} f_n(p) \\
  &= f(p).
\end{align*}
$f(x)$ is continuous at $x=p$ too.
\end{enumerate}
$\Box$ \\\\



%%%%%%%%%%%%%%%%%%%%%%%%%%%%%%%%%%%%%%%%%%%%%%%%%%%%%%%%%%%%%%%%%%%%%%%%%%%%%%%%



\textbf{Exercise 7.9.}
\emph{Let $\{f_n\}$ be a sequence of continuous functions
which converges uniformly to a function $f$ on a set $E$.
Prove that
\[
  \lim_{n \to \infty} f_n(x_n) = f(x)
\]
for every sequence of points $x_n \in E$ such that $x_n \to x$,
and $x \in E$.
Is the converse of this true?} \\

\emph{Proof.}
\begin{enumerate}
  \item[(1)]
  Given any $x \in E$ and any $\varepsilon > 0$.
  Since each $f_n$ is continuous and $f_n \to f$ uniformly,
  $f$ is continuous (Theorem 7.12).
  Hence as $x_n \to x$, there exists an integer $N_1$
  such that
  \[
    |f(x_n) - f(x)| \leq \frac{\varepsilon}{2}
    \text{ whenever } n \geq N_1
  \]
  (Theorem 4.2).
  Also, $f_n \to f$ uniformly implies that there exists an integer $N_2$
  such that
  \[
    |f_n(x_n) - f(x_n)| \leq \frac{\varepsilon}{2}
    \text{ whenever } n \geq N_2.
  \]
  Let $N = \max\{N_1,N_2\}$ be an integer.
  Then
  \[
    |f_n(x_n) - f(x)|
    \leq |f_n(x_n) - f(x_n)| + |f(x_n) - f(x)|
    \leq \frac{\varepsilon}{2} + \frac{\varepsilon}{2}
    = \varepsilon
  \]
  whenever $n \geq N$.
  Therefore, $\lim_{n \to \infty} f_n(x_n) = f(x)$.

  \item[(2)]
  \emph{Show that the converse is false.}
  Let $E = (0,1)$ and $f_n = \frac{1}{nx}$ on $E$.
  Given any $x \in E$.
  First,
  \[
    f(x) = \lim_{n \to \infty} f_n = \lim_{n \to \infty} \frac{1}{nx} = 0
  \]
  Next, for each sequence of points $x_n \in E$ such that $x_n \to x$
  (note that each $x_n \neq 0$ and $x \neq 0$), we have
  \[
    \lim_{n \to \infty} f_n(x_n)
    = \lim_{n \to \infty} \frac{1}{nx_n}
    = \lim_{n \to \infty} \frac{1}{n} \lim_{n \to \infty} \frac{1}{x_n}
    = 0 \cdot \frac{1}{x}
    = 0.
  \]
  Hence $\lim_{n \to \infty} f_n(x_n) = f(x) = 0$.
  However, $\{f_n\}$ does not converge uniformly.
  (See \emph{Proof ($f_n(x) = \frac{1}{x}$, $g_n(x) = \frac{1}{n}$)} in Exercise 7.3.)
\end{enumerate}
$\Box$ \\\\



%%%%%%%%%%%%%%%%%%%%%%%%%%%%%%%%%%%%%%%%%%%%%%%%%%%%%%%%%%%%%%%%%%%%%%%%%%%%%%%%



\textbf{Exercise 7.10.}
\emph{Letting $(x)$ denote the fractional part of the real number $x$
(see Exercise 4.16 for the definition),
consider the function
\[
  f(x) = \sum_{n=1}^{\infty} \frac{(nx)}{n^2}
  \qquad
  (x \in \mathbb{R}).
\]
Find all discontinuities of $f$, and show that they form a countable dense set.
Show that $f$ is nevertheless Riemann-integrable on every bounded interval.} \\

\emph{Proof.}
Let $f_n(x) = \frac{(nx)}{n^2}$ on $\mathbb{R}$,
$F_n(x) = \sum_{k=1}^{n} f_k(x)$ on $\mathbb{R}$.
\begin{enumerate}
\item[(1)]
Since
\[
  \abs{f_n(x)} = \abs{ \frac{(nx)}{n^2} } \leq \frac{1}{n^2}
\]
for all $x \in \mathbb{R}$ and $n=1,2,3,\ldots$
and $\sum \frac{1}{n^2}$ converges (to $\frac{\pi^2}{6}$),
$F_n = \sum f_k$ converges uniformly to $f$ on $\mathbb{R}$ (Theorem 7.10).

\item[(2)]
Note that $(x)$ is continuous on $\mathbb{R} - \mathbb{Z}$
and not continuous on $\mathbb{Z}$ (Exercise 4.16).
Now we define $E_n = \{ x \in \mathbb{R} : nx \in \mathbb{Z}\}$.
So $E_1 = \mathbb{Z}$, and
\[
  \bigcup_{n=1}^{\infty} E_n = \mathbb{Q}.
\]
So $f_n$ is continuous on $\mathbb{R} - E_n$
and not continuous on $E_n$.
So $F_n = \sum f_k$ is continuous on
$\mathbb{R} - \bigcup_{k=1}^{n} E_k
\supseteq \mathbb{R}$ - $\mathbb{Q}$.

\item[(3)]
\emph{Show that $f(x)$ is continuous on $\mathbb{R}$ - $\mathbb{Q}$.}
Since
$\{F_n\}$ is a sequence of continuous functions on $\mathbb{R}$ - $\mathbb{Q}$ (by (2))
and $F_n \to f$ uniformly (by (1)),
$f$ is continuous on $\mathbb{R}$ - $\mathbb{Q}$ (Theorem 7.12).

\item[(4)]
\emph{Show that $f(x)$ is not continuous on $\mathbb{Q}$,
which is a countable dense set of $\mathbb{R}$.}
  \begin{enumerate}
  \item[(a)]
  (Reductio ad absurdum)
  If there were $p = \frac{a}{b} \in \mathbb{Q}$
  with $a,b \in \mathbb{Z}$, $(a,b) = 1$ and $b > 0$
  such that $f(x)$ is continuous at $x = p$,
  then
  \[
    \lim_{x \to p^{-}} f(x) = \lim_{x \to p^{+}} f(x).
  \]

  \item[(b)]
  As $b \mid n$, say $n = bq$ for some $q \in \mathbb{Z}^{+}$, we have
  \begin{align*}
    \lim_{x \to p^{-}} f_n(x)
    &= \lim_{x \to p^{-}} \frac{1}{b^2 q^2}
    = \frac{1}{b^2 q^2}, \\
    \lim_{x \to p^{+}} f_n(x)
    &= \lim_{x \to p^{+}} \frac{0}{b^2 q^2}
    = 0.
  \end{align*}
  As $b \nmid n$,
  \[
    \lim_{x \to p^{-}} f_n(x) = \lim_{x \to p^{+}} f_n(x) = f_n(p).
  \]
  Thus,
  \[
    \lim_{x \to p^{-}} F_n(x) - \lim_{x \to p^{+}} F_n(x)
    = \frac{1}{b^2} \sum_{q = 1}^{[\frac{n}{b}]} \frac{1}{q^2}.
  \]

  \item[(c)]
  Since $F_n \to f$ uniformly, given $\varepsilon = \frac{64}{1989 b^2} > 0$,
  there exists an integer $N'$ such that
  \[
    \abs{ \sum_{n=m}^{\infty} f_n(x) }
    = \sum_{n=m}^{\infty} f_n(x)
    \leq \frac{64}{1989 b^2}
  \]
  whenever $m \geq N'$.

  \item[(d)]
  Take $N = \max\{N', b\}$.
  \begin{align*}
    &\abs{ \underbrace{\lim_{x \to p^{-}} f(x)}_{\text{exists}}
      - \underbrace{\lim_{x \to p^{+}} f(x)}_{\text{exists}} } \\
    =& \abs{
      \underbrace{\lim_{x \to p^{-}} F_N(x)}_{\text{exists}}
      - \underbrace{\lim_{x \to p^{+}} F_N(x)}_{\text{exists}}
      + \underbrace{\lim_{x \to p^{-}} \sum_{n=N+1}^{\infty} f_n(x)}_{\text{exists}}
      - \underbrace{\lim_{x \to p^{+}} \sum_{n=N+1}^{\infty} f_n(x)}_{\text{exists}} } \\
    \geq& \abs{ \lim_{x \to p^{-}} F_N(x) - \lim_{x \to p^{+}} F_N(x) }
      - \abs{  \lim_{x \to p^{-}} \sum_{n=N+1}^{\infty} f_n(x) }
      - \abs{  \lim_{x \to p^{+}} \sum_{n=N+1}^{\infty} f_n(x) } \\
    \geq& \frac{1}{b^2} \sum_{q = 1}^{[\frac{n}{b}]} \frac{1}{q^2}
      - \frac{64}{1989 b^2} - \frac{64}{1989 b^2} \\
    \geq& \frac{1}{q^2} - \frac{64}{1989 b^2} - \frac{64}{1989 b^2} \\
    =& \frac{1861}{1989 b^2} \\
    >& 0,
  \end{align*}
  which is absurd.
  \end{enumerate}

  \item[(4)]
  \emph{Show that $f$ is nevertheless Riemann-integrable on every bounded interval.}
  Since each $f_n \in \mathscr{R}$ on every bounded interval,
  $F_n \in \mathscr{R}$ on every bounded interval.
  Since $F_n \to f$ uniformly,
  $f \in \mathscr{R}$ on every bounded interval by Theorem 7.16.
\end{enumerate}
$\Box$ \\\\



%%%%%%%%%%%%%%%%%%%%%%%%%%%%%%%%%%%%%%%%%%%%%%%%%%%%%%%%%%%%%%%%%%%%%%%%%%%%%%%%



\textbf{Exercise 7.11 (Dirichlet's test).}
\emph{Suppose $\{f_n\}$, $\{g_n\}$ are defined on $E$, and}
\begin{enumerate}
  \item[(a)]
  \emph{$\sum f_n(x)$ has uniformly bounded partial sums;}
  \item[(b)]
  \emph{$g_n(x) \to 0$ uniformly on $E$;}
  \item[(b)]
  \emph{$g_1(x) \geq g_2(x) \geq g_3(x) \geq \cdots$ for every $x \in E$.}
\end{enumerate}
\emph{Prove that $\sum f_n(x) g_n(x)$ converges uniformly on $E$.
(Hint: Compare with Theorem 3.42.)} \\



\emph{Theorem 3.42 (Dirichlet's test).}
Suppose
\begin{enumerate}
  \item[(a)]
  the partial sums $A_n$ of $\sum a_n$ form a bounded sequence;
  \item[(b)]
  $b_0 \geq b_1 \geq b_2 \geq \cdots$;
  \item[(c)]
  $\lim_{n \to \infty} b_n = 0$.
\end{enumerate}
Then $\sum a_n b_n$ converges. \\



\emph{Proof (Theorem 3.42).}
Let $F_n(x) = \sum_{k=1}^{n} f_k(x)$.
Choose $M$ such that $|F_n(x)| \leq M$ for all $n$, all $x \in E$.
Given $\varepsilon > 0$,
there is an integer $N$ such that $g_N(x) \leq \frac{\varepsilon}{2(M+1)}$ for all $x \in E$.
For $N \leq p \leq q$, we have
\begin{align*}
  &\abs{\sum_{n=p}^{q} f_n(x) g_n(x)} \\
  =& \abs{\sum_{n=p}^{q-1} F_n(x)(g_n(x)-g_{n+1}(x)) + F_q(x)g_q(x) - F_{p-1}(x)g_p(x)} \\
  \leq& M \abs{\sum_{n=p}^{q-1}(g_n(x)-g_{n+1}(x)) + g_q(x) + g_p(x)} \\
  =& 2M g_p(x) \\
  \leq& 2M g_N(x) \\
  \leq& \varepsilon
\end{align*}
for all $x \in E$.
Uniformly convergence now follows from the Cauchy criterion (Theorem 7.8).
Note that the first inequality in the above chain depends of course on the fact that
$g_n(x) - g_{n+1}(x) \geq 0$.
$\Box$ \\\\



%%%%%%%%%%%%%%%%%%%%%%%%%%%%%%%%%%%%%%%%%%%%%%%%%%%%%%%%%%%%%%%%%%%%%%%%%%%%%%%%



\textbf{Exercise 7.12.}
\emph{Suppose $g$ and $f_n$ ($n=1,2,3\ldots$) are defined on $(0,\infty)$,
are Riemann-integrable on $[t,T]$ whenever $0 < t < T < \infty$,
$|f_n| \leq g$, $f_n \to f$ uniformly on every compact subset of $(0,\infty)$),
and
\[
  \int_{0}^{\infty} g(x)dx < \infty.
\]
Prove that
\[
  \lim_{n \to \infty} \int_{0}^{\infty} f_n(x)dx = \int_{0}^{\infty} f(x)dx.
\]
(See Exercises 6.7 and 6.8 for the relevant definitions.)
This is a rather weak form of Lebesgue's dominated convergence theorem (Theorem 11.32).
Even in the context of the Riemann integral,
uniform convergence can be replaced by pointwise convergence if
it is assumed that $f \in \mathscr{R}$.
(See the articles by F. Cunningham in Math. Mag., vol. 40, 1967, pp. 179-186,
and by H. Kestelman in Amer. Math. Monthly, vol. 77, 1970, pp. 182-187.)} \\



\emph{Proof.}
\begin{enumerate}
\item[(1)]
It is equivalent to show that
\[
  \lim_{n \to \infty} \int_{0}^{1} f_n(x)dx = \int_{0}^{1} f(x)dx
\]
and
\[
  \lim_{n \to \infty} \int_{1}^{\infty} f_n(x)dx = \int_{1}^{\infty} f(x)dx
\]
in the sense of Exercises 6.7 and 6.8.

\item[(2)]
\emph{Show that $\int_{0}^{1} f_n(x)dx$ ($n=1,2,3\ldots$) and $\int_{0}^{1} f(x)dx$ are
convergent (well-defined) in in the sense of Exercises 6.7.}
By assumption, as $0 < t < 1$ we have
\[
  \abs{\int_{t}^{1} f_n(x)dx}
  \leq \int_{t}^{1} \abs{f_n(x)}dx
  \leq \int_{t}^{1} g(x)dx.
\]
Note that
\[
  \lim_{t \to 0} \int_{t}^{1} g(x)dx = \int_{0}^{1} g(x)dx < \infty
\]
(Exercises 6.7).
Hence
\[
  \lim_{t \to 0}\abs{\int_{t}^{1} f_n(x)dx}
  = \abs{\lim_{t \to 0} \int_{t}^{1} f_n(x)dx}
  \leq \int_{0}^{1} g(x)dx
  < \infty.
\]
Also,
since $|f_n(x)| \leq g(x)$ and $f_n \to f$ uniformly, $f(x) \leq g(x)$ pointwise.
Apply the same argument to get
\[
  \abs{\lim_{t \to 0} \int_{t}^{1} f(x)dx}
  < \infty.
\]
Here $\int_{t}^{1} f(x)dx$ exists by Theorem 7.16.

\item[(3)]
Given any integer $n > 0$ and $t \in (0,1]$, we have
\begin{align*}
  \abs{ \int_{0}^{1} f_n(x)dx - \int_{0}^{1} f(x)dx }
  \leq&
  \abs{ \int_{0}^{1} f_n(x)dx - \int_{t}^{1} f_n(x)dx } \\
    &+ \abs{ \int_{t}^{1} f_n(x)dx - \int_{t}^{1} f(x)dx } \\
    &+ \abs{ \int_{t}^{1} f(x)dx - \int_{0}^{1} f(x)dx } \\
  \leq&
  \abs{ \int_{0}^{t} f_n(x)dx } \\
    &+ \int_{t}^{1} \abs{ f_n(x) - f(x) } dx \\
    &+ \abs{ \int_{0}^{t} f(x)dx } \\
\end{align*}

\item[(4)]
Given $\varepsilon > 0$.
Apply the similar argument in (2),
we have
\begin{align*}
  \abs{ \int_{0}^{t} f_n(x)dx }
  &\leq \int_{0}^{t} |f_n(x)|dx
  \leq \int_{0}^{t} g(x)dx, \\
  \abs{ \int_{0}^{t} f(x)dx }
  &\leq \int_{0}^{t} |f(x)|dx
  \leq \int_{0}^{t} g(x)dx.
\end{align*}
Since $\int_{0}^{t} g(x)dx < \infty$,
there exists a real number $c \in (0,1)$ such that
\[
  \int_{0}^{t} g(x)dx < \frac{\varepsilon}{3}
\]
whenever $0 < t \leq c$.
In particular, for any integer $n > 0$ we have
\begin{align*}
  \abs{ \int_{0}^{c} f_n(x)dx }
  &\leq \frac{\varepsilon}{3}, \\
  \abs{ \int_{0}^{c} f(x)dx }
  &\leq \frac{\varepsilon}{3}.
\end{align*}

\item[(5)]
For such $c \in (0,1)$ in (4), there is an integer $N$ such that
\[
  \abs{f_n(x) - f(x)} < \frac{\varepsilon}{3(1-c)}
\]
whenever $n \geq N$ and $x \in [c,1]$
since $f_n \to f$ uniformly on a compact set $[c,1]$.

\item[(6)]
By (3)(4)(5),
\begin{align*}
  &\abs{ \int_{0}^{1} f_n(x)dx - \int_{0}^{1} f(x)dx } \\
  \leq&
  \abs{ \int_{0}^{c} f_n(x)dx }
    + \int_{c}^{1} \abs{ f_n(x) - f(x) } dx
    + \abs{ \int_{0}^{c} f(x)dx } \\
  <&
  \frac{\varepsilon}{3}
    + (1-c) \cdot \frac{\varepsilon}{3(1-c)}
    + \frac{\varepsilon}{3} \\
  =&
  \varepsilon
\end{align*}
whenever $n \geq N$.
Therefore
\[
  \lim_{n \to \infty} \int_{0}^{1} f_n(x)dx = \int_{0}^{1} f(x)dx.
\]
Similarly,
\[
  \lim_{n \to \infty} \int_{1}^{\infty} f_n(x)dx = \int_{1}^{\infty} f(x)dx.
\]
Hence
\[
  \lim_{n \to \infty} \int_{0}^{\infty} f_n(x)dx = \int_{0}^{\infty} f(x)dx.
\]
\end{enumerate}
$\Box$ \\



\textbf{Supplement (Tannery's convergence theorem for Riemann integrals).}
\emph{(Exercise 10.7 of the book
T. M. Apostol, Mathematical Analysis, Second Edition.)
Prove Tannery's convergence theorem for Riemann integrals:
Given a sequence of functions $\{f_n\}$ and an increasing sequence $\{p_n\}$
of real numbers such that $p_n \to +\infty$ as $n \to \infty$.
Assume that}
\begin{enumerate}
\item[(a)]
\emph{$f_n \to f$ uniformly on $[a,b]$ for every $b \geq a$.}

\item[(b)]
\emph{$f_n$ is Riemann-integrable on $[a,b]$ for every $b \geq a$.}

\item[(c)]
\emph{$|f_n(x)| \leq g(x)$ on $[a,+\infty)$,
where $g$ is improper Riemann-integrable on $[a,+\infty)$.}
\end{enumerate}

\emph{Then both $f$ and $|f|$ are improper Riemann-integrable on $[a,+\infty)$,
the sequence $\{\int_{a}^{p_n}f_n(x)dx\}$ converges, and
\[
  \int_{a}^{\infty} f(x)dx = \lim_{n \to \infty}\int_{a}^{p_n}f_n(x)dx.
\]}
\begin{enumerate}
\item[(d)]
\emph{Use Tannery's theorem to prove that
\[
  \lim_{n \to \infty} \int_{0}^{n} \left( 1-\frac{x}{n} \right)^n x^p dx
  = \int_{0}^{\infty} e^{-x}x^p dx,
\]
if $p > -1$.} \\\\
\end{enumerate}



%%%%%%%%%%%%%%%%%%%%%%%%%%%%%%%%%%%%%%%%%%%%%%%%%%%%%%%%%%%%%%%%%%%%%%%%%%%%%%%%



\textbf{Exercise 7.13.}
\emph{Assume that $\{f_n\}$ is a sequence of monotonically increasing functions on $\mathbb{R}^1$
with $0 \leq  f_n(x) \leq 1$ for all $x$ and all $n$.}
\begin{enumerate}
\item[(a)]
  \emph{Prove that there is a function $f$ and a sequence $\{n_k\}$ such that
  \[
    f(x) = \lim_{k \to \infty} f_{n_k}(x)
  \]
  for every $x \in \mathbb{R}^1$.
  (The existence of such a pointwise convergent subsequence is usually
  called \textbf{Helly's selection theorem}.)}

\item[(b)]
  \emph{If, moreover, $f$ is continuous, does $f_{n_k} \to f$ uniformly on $\mathbb{R}^1$
  or on any bounded subset $E$ of $\mathbb{R}^1$?}
\end{enumerate}

\emph{(Hint:}
\begin{enumerate}
\item[(i)]
\emph{Some subsequence $\{f_{n_i}\}$ converges at all rational points $r$, say, to $f(r)$.}

\item[(ii)]
\emph{Define $f(x)$, for any $x \in \mathbb{R}^1$, to be $\sup f(r)$,
the sup being taken over all $r \leq x$.}

\item[(iii)]
\emph{Show that $f_{n_i}(x) \to f(x)$ at every $x$ at which $f$ is continuous.
(This is where monotonicity is strongly used.)}

\item[(iv)]
\emph{A subsequence of $\{f_{n_i}\}$ converges at every point of discontinuity of $f$
since there are at most countably many such points.}

\end{enumerate}
\emph{This proves (a).
To prove (b), modify your proof of (iii) appropriately.)} \\

\emph{Proof of (a).}
\begin{enumerate}
\item[(1)]
  \emph{Show that there is a subsequence $\{f_{n_i}\}$ converges at all rational points $r$,
  say, to $f(r)$.}
  Let $E = \mathbb{Q}$ be a countable subset of $\mathbb{R}^1$ in Theorem 7.23.

\item[(2)]
  Define $f(x)$, for any $x \in \mathbb{R}^1$, to be $\sup f(r)$,
  the sup being taken over all $r \leq x$.
  It is well-defined since $f(x) = \sup f(r) \leq 1$
  and the construction of $\mathbb{R}^1$ (Theorem 1.19).
  Note that $f$ is monotonically increasing.

\item[(3)]
  \emph{Show that $f_{n_i}(x) \to f(x)$ at every $x$ at which $f$ is continuous.}
  \begin{enumerate}
  \item[(a)]
    Given any $x$ at which $f$ is continuous.
    Given any $\varepsilon > 0$.
    Since $f$ is continuous at $x$, there exists a $\delta > 0$
    such that
    \[
      f(x)-\frac{\varepsilon}{89} < f(r) < f(x)+\frac{\varepsilon}{89}
    \]
    whenever $r \in (x-\delta,x+\delta)$.

  \item[(b)]
    Given any $r \in \mathbb{Q}$.
    By (1), there is an integer $N$ such that
    \[
      f(r)-\frac{\varepsilon}{64} < f_{n_i}(r) < f(r)+\frac{\varepsilon}{64}
    \]
    whenever $i \geq N$.

  \item[(c)]
    As $r \in (x,x+\delta) \bigcap \mathbb{Q} \neq \varnothing$
    (since $\mathbb{Q}$ is dense in $\mathbb{R}$)
    and $i \geq N$, we have
      \begin{align*}
        f_{n_i}(x)
        &\leq f_{n_i}(r)
          &\text{($f_{n_i}$: increasing)} \\
        &< f(r) + \frac{\varepsilon}{64}
          &\text{((b))} \\
        &< f(x)+\frac{\varepsilon}{89} + \frac{\varepsilon}{64}
          &\text{((a))} \\
        &< f(x) + \varepsilon.
      \end{align*}
    Similarly,
    \[
      f_{n_i}(x) > f(x) - \varepsilon.
    \]
    Therefore
    \[
      \abs{ f_{n_i}(x) - f(x) } < \varepsilon
    \]
    whenever $i \geq N$.
  \end{enumerate}

\item[(4)]
  \emph{Show that there is a subsequence of $\{f_{n_i}\}$ converging
  at every point of discontinuity of $f$
  since there are at most countably many such points.}
  \begin{enumerate}
  \item[(a)]
    By construction of $f$, $f$ is monotonically increasing and $0 \leq f(x) \leq 1$.

  \item[(b)]
    Theorem 4.30 implies that there are at most countably many discontinuity points of $f$.

  \item[(c)]
    Apply Theorem 7.23 again to get there is
    a subsequence of $\{f_{n_i}\}$ converging
    at every point of discontinuity of $f$.
  \end{enumerate}

\item[(5)]
  Since any subsequence of $\{f_{n_i}\}$ converges at every point of continuity of $f$,
  there exists a subsequence $\{f_{n_k}\}$ of $\{f_n\}$ such that
  \[
    \lim_{k \to \infty} f_{n_k}(x) = f(x)
  \]
  for $x \in \mathbb{R}^1$ (by (3)(4)).
\end{enumerate}
$\Box$ \\



\emph{Proof of (b).}
\begin{enumerate}
\item[(1)]
  \emph{Show that the result does not hold on $\mathbb{R}^1$.}
  (Using sigmoid functions.)
  \begin{enumerate}
  \item[(a)]
    Define
    \begin{equation*}
    f_n(x) =
      \begin{cases}
        \frac{1}{2(1+e^{-x})} & \text{ if $x < n$}, \\
        1 & \text{ if $x \geq n$}.
      \end{cases}
    \end{equation*}

  \item[(b)]
    $\{f_n\}$ is a sequence of monotonically increasing functions on $\mathbb{R}^1$
    with $0 \leq f_n(x) \leq 1$ for all $x$ and all $n$.

  \item[(c)]
    Define a continuous function $f(x)$ on $\mathbb{R}^1$ by
    \[
      f(x) = \frac{1}{2(1+e^{-x})}.
    \]
    So for every subsequence $\{f_{n_k}\}$ of $\{f_n\}$, we have
    \[
      \lim_{k \to \infty} f_{n_k}(x) = f(x)
    \]
    for all $x \in \mathbb{R}^1$, but
    \[
      \abs{ f_{n}(n) - f(n) }
      = \abs{ 1 - \frac{1}{2(1+e^{-n})} }
      \geq 1 - \frac{1}{2}
      = \frac{1}{2}.
    \]
    So that no subsequence can converge uniformly on $\mathbb{R}^1$
    (by using the similar argument in Example 7.21).
  \end{enumerate}

\item[(2)]
  \emph{Show that the result holds on any bounded subset $E$ of $\mathbb{R}^1$.}
  Might assume that $E = [a,b]$ with $a \neq -\infty$ and $b \neq \infty$.
  \begin{enumerate}
  \item[(a)]
    Given any $\varepsilon > 0$.
    Since $f$ is continuous on a compact set $E$,
    $f$ is continuous uniformly on $E$,
    and thus there exists a $\delta > 0$ such that
    \[
      |f(x) - f(y)| < \frac{\varepsilon}{89}
    \]
    whenever $x,y \in K$ and $|x-y| < \delta$.

  \item[(b)]
    For such $\delta > 0$,
    define a partition $P = \{x_0, x_1, \ldots, x_m\}$ of $[a,b]$ such that
    \[
      \Delta x_j = x_j - x_{j-1} < \frac{\delta}{64}
    \]
    for all $1 \leq j \leq m$.

  \item[(c)]
    Since $f_{n_k} \to f$ (pointwise), for each $x_j$ in the partition $P$
    there exist integers $N_j$ such that
    \[
      \abs{ f_{n_k}(x_j) - f(x_j) } < \frac{\varepsilon}{1989}
    \]
    whenever $k \geq N_j$.
    Take an integer $N = \max\{N_0, N_1, \ldots, N_m\}$. Thus
    \[
      \abs{ f_{n_k}(x_j) - f(x_j) } < \frac{\varepsilon}{1989}
    \]
    whenever $0 \leq j \leq m$ and $k \geq N$.

  \item[(d)]
    As $0 \leq j \leq m$ and $k \geq N$, we have
      \begin{align*}
        &\abs{f_{n_k}(x_j) - f_{n_k}(x_{j-1})} \\
        \leq&
        \abs{f_{n_k}(x_j) - f(x_j)}
          + \abs{f(x_j) - f(x_{j-1})}
          + \abs{f(x_{j-1}) - f_{n_k}(x_{j-1})} \\
        <&
        \frac{\varepsilon}{1989}
          + \frac{\varepsilon}{89}
          + \frac{\varepsilon}{1989}
      \end{align*}
    by (a)(c).

  \item[(e)]
    Now given any $x \in [a,b]$, by (b) there is a subinterval $[x_{i-1},x_i]$
    such that $x \in [x_{i-1},x_i]$.
    Hence
    \begin{align*}
      \abs{f_{n_k}(x) - f(x)}
      \leq&
      \abs{f_{n_k}(x) - f_{n_k}(x_{i-1})} \\
        &+ \abs{f_{n_k}(x_{i-1}) - f(x_{i-1})} \\
        &+ \abs{f(x_{i-1}) - f(x)} \\
      \leq&
      \abs{f_{n_k}(x_i) - f_{n_k}(x_{i-1})}
          &\text{($f_{n_k}$: increasing)} \\
        &+ \abs{f_{n_k}(x_{i-1}) - f(x_{i-1})} \\
        &+ \abs{f(x_{i-1}) - f(x_i)}
          &\text{($f$: increasing)} \\
      <&
      \frac{\varepsilon}{1989}
          + \frac{\varepsilon}{89}
          + \frac{\varepsilon}{1989}
          &\text{((d))} \\
        &+ \frac{\varepsilon}{1989}
          &\text{((c))} \\
        &+ \frac{\varepsilon}{89}
          &\text{((a))} \\
      <& \varepsilon
    \end{align*}
    whenever $k \geq N$.
    The above inequality holds for any $x \in [a,b]$
    and thus $f_{n_k} \to f$ uniformly.
  \end{enumerate}
\end{enumerate}
$\Box$ \\\\



%%%%%%%%%%%%%%%%%%%%%%%%%%%%%%%%%%%%%%%%%%%%%%%%%%%%%%%%%%%%%%%%%%%%%%%%%%%%%%%%



\textbf{Exercise 7.14.}
\emph{Let $f$ be a continuous real function on $\mathbb{R}^1$
with the following properties:
$0 \leq f(t) \leq 1$, $f(t+2)=f(t)$ for every $t$, and
\begin{equation*}
  f(t) =
    \begin{cases}
      0 & \left(0 \leq t \leq \frac{1}{3}\right) \\
      1 & \left(\frac{2}{3} \leq t \leq 1\right).
    \end{cases}
\end{equation*}
Put $\Phi(t) = (x(t),y(t))$, where
\begin{align*}
  x(t) &= \sum_{n=1}^{\infty} 2^{-n} f(3^{2n-1}t), \\
  y(t) &= \sum_{n=1}^{\infty} 2^{-n} f(3^{2n}t).
\end{align*}
Prove that $\Phi(t)$ is continuous and
that $\Phi(t)$ maps $I = [0,1]$ onto the unit square $I^2 \subseteq \mathbb{R}^2$.
If fact, show that $\Phi(t)$ maps the Cantor set onto $I^2$.}
\emph{(Hint: Each $(x_0,y_0) \in I^2$ has the form
\begin{align*}
  x_0 &= \sum_{n=1}^{\infty} 2^{-n} a_{2n-1}, \\
  y_0 &= \sum_{n=1}^{\infty} 2^{-n} a_{2n}
\end{align*}
where each $a_i$ is $0$ or $1$.
If
\[
  t_0 = \sum_{i=1}^{\infty} 3^{-i-1}(2a_i)
\]
show that $f(3^kt_0) = a_k$, and hence that $x(t_0) = x_0$, $y(t_0) = y_0$.}
\emph{(This simple example of a so-called ``space-filling curve''
is due to I. J. Schoenberg, Bull. A.M.S., vol. 44, 1938, pp. 519.)} \\

\emph{Proof (Hint).}
\begin{enumerate}
\item[(1)]
  \emph{Show that $\Phi(t)$ is continuous.}
  $x(t)$ and $y(t)$ is well-defined (Theorem 7.10).
  Thus $\Phi(t)$ is well-defined too.
  Note that each term of $x(t)$ or $y(t)$ is continuous
  (since $f$ itself is continuous),
  $x(t)$ or $y(t)$ is continuous (Theorem 7.11).
  Hence $\Phi(t)$ is continuous (Theorem 4.10).

\item[(2)]
  For each $(x_0,y_0) \in I^2$, we can write
  \begin{align*}
    x_0 &= \sum_{n=1}^{\infty} 2^{-n} a_{2n-1}, \\
    y_0 &= \sum_{n=1}^{\infty} 2^{-n} a_{2n}
  \end{align*}
  where each $a_i$ is $0$ or $1$
  (by the same argument in Decimals 1.22).
  For such $\{a_i\}$ define
  \[
    t_0
    = \sum_{i=1}^{\infty} 3^{-i-1}(2a_i).
  \]
  By Exercise 3.19, $t_0$ is precisely in the Cantor set.

\item[(3)]
  \emph{Show that $f(3^k t_0) = a_k$.}
  Write
  \begin{align*}
    3^k t_0
    &= \sum_{i=1}^{\infty} 3^{k-i-1}(2a_i) \\
    &= \sum_{i=1}^{k-1} 3^{k-i-1}(2a_i)
      + 3^{-1}(2a_k)
      + \sum_{i=k+1}^{\infty} 3^{k-i-1}(2a_i) \\
    &= \underbrace{2 \sum_{i=1}^{k-1} 3^{k-i-1} a_i}_{\text{define as } \alpha(k)}
      + \underbrace{\frac{2a_k}{3}}_{\text{define as } \beta(k)}
      + \underbrace{\sum_{i=k+1}^{\infty} \frac{2a_i}{3^{i-k+1}}}_{\text{define as } \gamma(k)}.
  \end{align*}
  Here $\alpha(k)$ is an even integer, and $\gamma(k) \in \left[0,\frac{1}{3}\right]$.
  Since $f$ is of a period $2$, we have
  \[
    f(3^k t_0)
    = f(\alpha(k) + \beta(k) + \gamma(k))
    = f(\beta(k) + \gamma(k)).
  \]
  Now we consider two possible cases of $a_k$.
  \begin{enumerate}
    \item[(a)]
      If $a_k = 0$, then $\beta(k) = 0$ and
      $\beta(k) + \gamma(k) \in \left[0,\frac{1}{3}\right]$ and
      $f(3^k t_0) = 0$.
    \item[(b)]
      If $a_k = 1$, then $\beta(k) = \frac{2}{3}$ and
      $\beta(k) + \gamma(k) \in \left[\frac{2}{3}, 1\right]$ and
      $f(3^k t_0) = 1$.
  \end{enumerate}
  In any case, $f(3^k t_0) = a_k$.

\item[(4)]
  \emph{Show that $x(t_0) = x_0$, $y(t_0) = y_0$,
  and $\Phi(t)$ maps the Cantor set onto $I^2$.}
  By (2)(3),
  \begin{align*}
    x(t_0)
    &= \sum_{n=1}^{\infty} 2^{-n} f(3^{2n-1}t_0)
    = \sum_{n=1}^{\infty} 2^{-n} a_{2n-1}
    = x_0, \\
    y(t_0)
    &= \sum_{n=1}^{\infty} 2^{-n} f(3^{2n}t_0)
    = \sum_{n=1}^{\infty} 2^{-n} a_{2n}
    = y_0.
  \end{align*}
  Hence $\Phi(t)$ maps the Cantor set onto $I^2$.
\end{enumerate}
$\Box$ \\\\



%%%%%%%%%%%%%%%%%%%%%%%%%%%%%%%%%%%%%%%%%%%%%%%%%%%%%%%%%%%%%%%%%%%%%%%%%%%%%%%%



\textbf{Exercise 7.15.}
\emph{Suppose that $f$ is a real continuous function on $\mathbb{R}^1$,
$f_n(t)= f(nt)$ for $n=1,2,3,\ldots$,
and $\{f_n\}$ is equicontinuous on $[0,1]$.
What conclusion can you draw about $f$?} \\

\emph{Proof.}
\begin{enumerate}
\item[(1)]
  \emph{Show that $f$ is constant on $[0,\infty)$.}

\item[(2)]
  Given any $\varepsilon > 0$.
  Since $\{f_n\}$ is equicontinuous on $[0,1]$,
  there exists a $1 > \delta > 0$ such that
  \[
    |f_n(x)-f_n(y)| < \varepsilon
  \]
  whenever $n \in \mathbb{Z}^{+}$, $|x-y| < \delta < 1$, $x \in [0,1]$ and $y \in [0,1]$.
  Take $x = t \in [0,1]$ and $y = 0$.
  Note that $f_n(t) = f(nt)$ for any $n \in \mathbb{Z}^{+}$ and
  $t \in \mathbb{R}^1$.
  Hence
  \[
    |f(nt) - f(0)| < \varepsilon
  \]
  for all integer $n > 0$ and $0 \leq t < \delta < 1$.

\item[(3)]
  Given any $x \in [0,\infty)$.
  There is an integer $N > 0$ such that $0 \leq x < N\delta$
  (by taking $N > \frac{\delta}{x}$).
  Let $t = \frac{x}{N}$.
  So that $0 \leq t < \delta$.
  Hence
  \[
    |f(x) - f(0)| = |f(Nt) - f(0)| < \varepsilon.
  \]
  Since $\varepsilon > 0$ is arbitrary,
  $f(x) = f(0)$ for any $x \in [0,+\infty)$.
  Therefore $f$ is constant on $[0,\infty)$.
\end{enumerate}
$\Box$ \\\\



%%%%%%%%%%%%%%%%%%%%%%%%%%%%%%%%%%%%%%%%%%%%%%%%%%%%%%%%%%%%%%%%%%%%%%%%%%%%%%%%



\textbf{Exercise 7.16.}
\emph{Suppose $\{f_n\}$ is an equicontinuous sequence of functions on a compact set $K$,
and $\{f_n\}$ converges pointwise on $K$.
Prove that $\{f_n\}$ converges uniformly on $K$.} \\

(Assume that $\{f_n\}$ is a sequence of complex-valued functions.) \\

\emph{Proof.}
Given any $\varepsilon > 0$.
\begin{enumerate}
  \item[(1)]
  Since $\{f_n\}$ is equicontinuous, there is $\delta > 0$ such that
  \[
    |f_n(x) - f_n(y)| < \frac{\varepsilon}{3}
  \]
  whenever $x,y \in K$, $|x-y| < \delta$, $n = 1,2,3,\ldots$
  (where $d$ is the metric function).

  \item[(2)]
  (Similar to Exercise 4.8.)
  For such $\delta > 0$, we construct an open covering of $K$.
  Pick a collection $\mathscr{C}$ of open balls
  $B(a;\delta)$
  where $a$ runs over all elements of $K$.
  Since $\mathscr{C}$ is an open covering of a compact set $K$,
  there is a finite subcollection $\mathscr{C}'$ of $\mathscr{C}$
  also covers $K$, say
  \[
    \mathscr{C}'
    = \left\{B(a_1;\delta), B(a_2;\delta), \ldots, B(a_m;\delta) \right\}.
  \]

  \item[(3)]
  Since $f_n$ converges pointwise on $K$,
  for each $i$ there is an integer $N_i$ such that
  \[
    |f_n(a_i)-f_m(a_i)| < \frac{\varepsilon}{3}
  \]
  whenever $n,m \geq N_i$.

  \item[(4)]
  Now given any $x \in K$, by (2) there exist $a_j$ $(1 \leq j \leq m)$
  such that $x \in B(a_j;\delta)$.
  Take $N = \max\{N_1,\ldots,N_m\}$.
  Hence
  \begin{align*}
    |f_n(x)-f_m(x)|
    &\leq
    |f_n(x)-f_n(a_j)| + |f_n(a_j)-f_m(a_j)| + |f_m(a_j)-f_m(x)| \\
    &<
    \frac{\varepsilon}{3} + \frac{\varepsilon}{3} + \frac{\varepsilon}{3} \\
    &=
    \varepsilon.
  \end{align*}
  whenever $n,m \geq N$.
  Hence $\{f_n\}$ converges uniformly (Theorem 7.8).
\end{enumerate}
$\Box$ \\\\



%%%%%%%%%%%%%%%%%%%%%%%%%%%%%%%%%%%%%%%%%%%%%%%%%%%%%%%%%%%%%%%%%%%%%%%%%%%%%%%%



\textbf{Exercise 7.17.}
\emph{Define the notions of uniform convergence and equicontinuous
for mappings into any metric space.
Show that Theorems 7.9 and 7.12 are valid for mappings into any metric space,
that Theorems 7.8 and 7.11 are valid for mappings into any complete metric space,
and that Theorems 7.10, 7.16, 7.17, 7.24, and 7.25 hold for
vector-valued functions, that is, for mappings into any $\mathbb{R}^k$.} \\



\textbf{Definition 7.7 over metric spaces.}
\emph{Suppose $(X,d_X)$ and $(Y,d_Y)$ are metric spaces, $E \subseteq X$.
We say that a sequence of functions $\{f_n\}$ of $f_n: E \to Y$,
$n = 1,2,3,\ldots$,
\textbf{converges uniformly} on $E$ to a function $f$ mapping from $E$ to $Y$
if for every $\varepsilon > 0$ there is an integer $N$ such that $n \geq N$ implies
\[
  d_Y(f_n(x),f(x)) \leq \varepsilon
\]
for all $x \in E$.}
$\Box$ \\



\textbf{Theorem 7.8 over complete metric spaces.}
\emph{Suppose $(X,d_X)$ is a metric space,
$(Y,d_Y)$ is a complete metric spaces, $E \subseteq X$.
The sequence of functions $\{f_n\}$ of $f_n: E \to Y$,
converges uniformly on $E$ if and only if
for every $\varepsilon > 0$
there is an integer $N$ such that $m \geq N$, $n \geq N$, $x \in E$ implies}
\[
  d_Y(f_n(x),f_m(x)) \leq \varepsilon.
\]

\emph{Proof (Theorem 7.8).}
\begin{enumerate}
  \item[(1)]
  Suppose $\{f_n\}$ converges uniformly on $E$, and let $f$ be the limit function.
  Then there is an integer $N$ such that $n \geq N$, $x \in E$ implies
  \[
    d_Y(f_n(x),f(x)) \leq \frac{\varepsilon}{2},
  \]
  so that
  \[
    d_Y(f_n(x),f_m(x))
    \leq d_Y(f_n(x),f(x)) + d_Y(f(x),f_m(x))
    \leq \varepsilon
  \]
  if $n \geq N$, $m \geq N$, $x \in E$.

  \item[(2)]
  Conversely, suppose the Cauchy condition holds.
  By Theorem 3.11,
  the sequence $\{f_n(x)\}$ converges, for every $x$, to a limit which we may call $f(x)$.
  Thus the sequence $\{f_n\}$ converges on $E$, to $f$.
  We have to prove that the convergence is uniformly.
  Let $\varepsilon > 0$ be given, and choose $N$ such that
  \[
    d_Y(f_n(x),f_m(x)) \leq \varepsilon.
  \]
  Fix $n$, and let $m \to \infty$.
  Since $f_m(x) \to f(x)$ as $m \to \infty$, this gives
  \[
    d_Y(f_n(x),f(x)) \leq \varepsilon
  \]
  for every $n \geq N$ and every $x \in E$, which completes the proof.
\end{enumerate}
$\Box$ \\



\textbf{Theorem 7.9 over metric spaces.}
\emph{Suppose $(X,d_X)$ and $(Y,d_Y)$ are metric spaces, $E \subseteq X$.
Let $\{f_n\}$ be a sequence of functions of $f_n: E \to Y$.
Suppose
\[
  \lim_{n \to \infty}f_n(x) = f(x)
  \qquad
  (x \in E).
\]
Put
\[
  M_n = \sup_{x \in E} d_Y(f_n(x),f(x)).
\]
Then $f_n \to f$ uniformly on $E$ if and only if $M_n \to 0$ as $n \to \infty$.} \\

\emph{Proof (Theorem 7.9).}
Given any $\varepsilon > 0$.
\begin{enumerate}
  \item[(1)]
  Suppose $\{f_n\}$ converges uniformly on $E$.
  Then there is an integer $N$ such that $n \geq N$, $x \in E$ implies
  \[
    d_Y(f_n(x),f(x)) \leq \varepsilon.
  \]
  Take sup over all $x \in E$ to get
  \[
    M_n = \sup_{x \in E} d_Y(f_n(x),f(x)) \leq \varepsilon
  \]
  whenever $n \geq N$.

  \item[(2)]
  Conversely, suppose $M_n \to 0$ as $n \to \infty$.
  Then there is an integer $N$ such that $n \geq N$ implies
  \[
    M_n = \sup_{x \in E} d_Y(f_n(x),f(x)) \leq \varepsilon
  \]
  whenever $n \geq N$.
  Hence
  \[
    d_Y(f_n(x),f(x)) \leq \sup_{x \in E} d_Y(f_n(x),f(x)) \leq \varepsilon
  \]
  whenever $n \geq N$ and $x \in E$.
\end{enumerate}
$\Box$ \\



\textbf{Theorem 7.10 over $\mathbb{R}^k$.}
\emph{Suppose $E \subseteq \mathbb{R}^1$,
$\{\mathbf{f}_n\}$ is a sequence of functions of $\mathbf{f}_n: E \to \mathbb{R}^k$,
and
\[
  |\mathbf{f}_n(x)| \leq M_n
  \qquad
  (x \in E, n = 1,2,3,\ldots).
\]
Then $\sum \mathbf{f}_n$ converges uniformly on $E$ if $\sum M_n$ converges.} \\

\emph{Proof (Brute-force).}
If $\sum M_n$ converges, then for arbitrary $\varepsilon > 0$,
\[
  \abs{\sum_{i=n}^{m}\mathbf{f}_i(x)} \leq \sum_{i=n}^{m}M_i \leq \varepsilon
  \qquad
  (x \in E),
\]
provided $m$ and $n$ are large enough.
Uniform convergence now follows from Theorem 7.8 over complete metric spaces
(since $\mathbb{R}^1$ is complete).
$\Box$ \\



\textbf{Theorem 7.11 over complete metric spaces.}
\emph{Suppose $(X,d_X)$ is a metric space,
$(Y,d_Y)$ is a complete metric spaces, $E \subseteq X$.
Suppose the sequence of functions $\{f_n\}$ of $f_n: E \to Y$
converges uniformly on $E$.
Let $x$ be a limit point of $E$, and suppose that
\[
  \lim_{t \to x} f_n(t) = A_n
  \qquad
  (n=1,2,3,\ldots).
\]
Then $\{A_n\}$ converges, and
\[
  \lim_{t \to x}f(t) = \lim_{n \to \infty} A_n.
\]
In other words, the conclusion is that}
\[
  \lim_{t \to x} \lim_{n \to \infty} f_n(t)
  = \lim_{n \to \infty} \lim_{t \to x} f_n(t).
\]

\emph{Proof (Theorem 7.11).}
\begin{enumerate}
\item[(1)]
  Let $\varepsilon > 0$ be given.
  By the uniform convergence of $\{f_n\}$, there exists an integer $N$
  such that $n \geq N$, $m \geq N$, $t \in E$ imply
  \[
    d_Y(f_n(t),f_m(t)) \leq \varepsilon.
  \]
  Letting $t \to x$, we obtain
  \[
    d_Y(A_n,A_m) \leq \varepsilon
  \]
  for $n \geq N$, $m \geq N$, so that $\{A_n\}$ is a Cauchy sequence in $Y$
  and therefore converges, say to $A$ (since $Y$ is complete).

\item[(2)]
  Next,
  \[
    d_Y(f(t),A)
    \leq
    d_Y(f(t),f_n(t)) + d_Y(f_n(t),A_n) + d_Y(A_n,A).
  \]

\item[(3)]
  We choose $N_1$ such that
  \[
    d_Y(f(t),f_n(t)) \leq \frac{\varepsilon}{3}
  \]
  whenever $n \geq N_1$ and $t \in E$
  (by the uniform convergence of $\{f_n\}$).

\item[(4)]
  We choose $N_2$ such that
  \[
    d_Y(A_n,A) \leq \frac{\varepsilon}{3}
  \]
  whenever $n \geq N_2$.

\item[(5)]
  In particular, we choose $n = \max\{N_1,N_2\}$ in (3)(4).
  For such $n$, there is an open neighborhood $B(x)$ of $x$ such that
  \[
    d_Y(f_n(t),A_n) \leq \frac{\varepsilon}{3}
  \]
  if $t \in B(x) \bigcap E - \{x\}$.

\item[(6)]
  Substituting the inequalities (3)(4)(5) into (2), we see that
  \[
    d_Y(f(t),A) \leq \varepsilon,
  \]
  provided $t \in B(x) \bigcap E - \{x\}$.
\end{enumerate}
$\Box$ \\



\textbf{Theorem 7.12 over metric spaces.}
\emph{Suppose $(X,d_X)$ and $(Y,d_Y)$ are metric spaces, $E \subseteq X$.
If $\{f_n\}$ is a sequence of continuous functions of $f_n: E \to Y$,
and if $f_n \to f$ uniformly on $E$,
then $f$ is continuous on $E$.} \\

\emph{Note.}
It is not a corollary of Theorem 7.11 over complete metric spaces
since $Y$ might not be complete. \\

\emph{Proof.}
\begin{enumerate}
\item[(1)]
  Suppose $x \in E$.
  If $x$ is an isolated point of $E$, there is nothing to do.
  We might assume that $x$ is a limit point of $E$.
  Let $\varepsilon > 0$ be given.

\item[(2)]
  Next, write
  \[
    d_Y(f(y),f(x))
    \leq
    d_Y(f(y),f_n(y)) + d_Y(f_n(y),f_n(x)) + d_Y(f_n(x),f(x))
  \]
  if $y \in E$.

\item[(3)]
  By the uniform convergence of $\{f_n\}$, there exists an integer $N$
  such that $n \geq N$, $t \in E$ imply
  \[
    d_Y(f_n(t),f(t)) \leq \frac{\varepsilon}{3}.
  \]

\item[(4)]
  In particular, we choose $n = N$ in (3).
  For such $n$, by the continuity of $f_n$
  there is a $\delta > 0$ such that
  \[
    d_Y(f_n(t),f_n(x)) < \frac{\varepsilon}{3}
  \]
  if $0 < d_X(t,x) < \delta$ and $t \in E$.

\item[(5)]
  Substituting the inequalities (3)(4) into (2), we see that
  \[
    d_Y(f(y),f(x)) < \varepsilon,
  \]
  provided $0 < d_X(y,x) < \delta$ and $y \in E$.
\end{enumerate}
$\Box$ \\



\textbf{Theorem 7.16 over $\mathbb{R}^k$.}
\emph{Let $\alpha$ be monotonically increasing on $[a,b]$.
Suppose $\mathbf{f}_n \in \mathscr{R}(\alpha)$ on $[a,b]$,
for $n = 1,2,3,\ldots$,
and suppose $\mathbf{f}_n \to \mathbf{f}$ uniformly on $[a,b]$.
Then $\mathbf{f} \in \mathscr{R}(\alpha)$ on $[a,b]$,
and
\[
  \int_{a}^{b} \mathbf{f} d\alpha
  = \lim_{n \to \infty} \int_{a}^{b} \mathbf{f}_n d\alpha.
\]
(The existence of the limit is part of the conclusion.)} \\

\emph{Proof (Theorem 7.16).}
\begin{enumerate}
\item[(1)]
  Write
  \[
    \mathbf{f}_n = \left(f_{n(1)}, \ldots, f_{n(k)}\right)
    \qquad
    \text{ and }
    \qquad
    \mathbf{f} = \left(f_{(1)}, \ldots, f_{(k)}\right)
  \]
  be the corresponding mappings of $[a,b]$ into $\mathbb{R}^k$.
  Since $\mathbf{f}_n \to \mathbf{f}$ uniformly on $[a,b]$,
  $f_{n(j)} \to f_{(j)}$ uniformly on $[a,b]$ for $j = 1,\ldots,k$
  by noting that
  \[
    \abs{ f_{n(j)}(x) - f_{(j)}(x) }
    \leq
    \abs{ \mathbf{f}_n(x) - \mathbf{f}(x) }
  \]
  for all $j$.

\item[(2)]
  By Definition 6.23,
  $\mathbf{f}_n \in \mathscr{R}(\alpha)$ on $[a,b]$
  means that
  $f_{n(j)} \in \mathscr{R}(\alpha)$ on $[a,b]$ for all $j$.
  By Theorem 7.16, $f_{(j)} \in \mathscr{R}(\alpha)$ on $[a,b]$ and
  \[
    \int_{a}^{b} f_{(j)} d\alpha
    = \lim_{n \to \infty} \int_{a}^{b} f_{n(j)} d\alpha
  \]
  for all $j$.

\item[(3)]
  By Definition 6.23, $\mathbf{f} \in \mathscr{R}(\alpha)$ on $[a,b]$, and
  \begin{align*}
    \int_{a}^{b} \mathbf{f} d\alpha
    &= \left(
        \int_{a}^{b} f_{(1)} d\alpha,
        \ldots,
        \int_{a}^{b} f_{(k)} d\alpha
      \right) \\
    &= \left(
        \lim_{n \to \infty} \int_{a}^{b} f_{n(1)} d\alpha,
        \ldots,
        \lim_{n \to \infty} \int_{a}^{b} f_{n(k)} d\alpha
      \right) \\
    &= \lim_{n \to \infty} \left(
        \int_{a}^{b} f_{n(1)} d\alpha,
        \ldots,
        \int_{a}^{b} f_{n(k)} d\alpha
      \right) \\
    &= \lim_{n \to \infty} \int_{a}^{b} \mathbf{f}_n d\alpha.
  \end{align*}
\end{enumerate}
$\Box$ \\



\textbf{Theorem 7.17 over $\mathbb{R}^k$.}
\emph{Suppose $\{\mathbf{f}_n\}$ is a sequence of vector-valued functions,
differentiable on $[a,b]$ and such that
$\{\mathbf{f}_n(x_0)\}$ converges for some point $x_0 \in [a,b]$.
If $\{\mathbf{f}'_n\}$ converges uniformly on $[a,b]$,
then $\{\mathbf{f}_n\}$ converges uniformly on $[a,b]$ to a function $\mathbf{f}$,
and}
\[
  \mathbf{f}'(x) = \lim_{n \to \infty} \mathbf{f}'_n(x)
  \qquad
  (a \leq x \leq b).
\]
\emph{Proof (Theorem 7.17).}
\begin{enumerate}
\item[(1)]
  Write
  \[
    \mathbf{f}_n = \left(f_{n(1)}, \ldots, f_{n(k)}\right)
  \]
  be the corresponding mappings of $[a,b]$ into $\mathbb{R}^k$.

\item[(2)]
  $\{\mathbf{f}_n\}$ is a sequence of differentiable functions on $[a,b]$
  if and only if
  $\{f_{n(j)}\}$ is a sequence of differentiable functions on $[a,b]$ for all $j=1,\ldots,k$
  (Remarks 5.16).

\item[(3)]
  $\{\mathbf{f}_n(x_0)\}$ converges for some point $x_0 \in [a,b]$
  if and only if
  $\{f_{n(j)}(x_0)\}$ converges for some point $x_0 \in [a,b]$ and all $j=1,\ldots,k$.

\item[(4)]
  $\{\mathbf{f}'_n\}$ converges uniformly on $[a,b]$
  if and only if
  $\{ f'_{n(j)} \}$ converges uniformly on $[a,b]$ for all $j=1,\ldots,k$.

\item[(5)]
  By Theorem 7.17, $\{f_{n(j)}\}$ converges uniformly on $[a,b]$, to a function $f_{(j)}$,
  and
  \[
    f'_{(j)}(x) = \lim_{n \to \infty} f'_{n(j)}(x)
    \qquad
    (a \leq x \leq b).
  \]
  for all $j$.
  Define
  \[
    \mathbf{f} = \left(f_{(1)}, \ldots, f_{(k)}\right).
  \]
  Hence $\{\mathbf{f}_n\}$ converges uniformly on $[a,b]$ to $\mathbf{f}$,
  and
  \begin{align*}
    \mathbf{f}'(x)
    &= \left( f'_{(1)}(x), \ldots, f'_{(k)}(x) \right) \\
    &= \left( \lim_{n \to \infty} f'_{n(1)}(x), \ldots, \lim_{n \to \infty} f'_{n(k)}(x) \right) \\
    &= \lim_{n \to \infty} \left( f'_{n(1)}(x), \ldots, f'_{n(k)}(x) \right) \\
    &= \lim_{n \to \infty} \mathbf{f}'_n(x).
  \end{align*}
\end{enumerate}
$\Box$ \\



\textbf{Definition 7.22 over metric spaces.}
\emph{Suppose $(X,d_X)$ and $(Y,d_Y)$ are metric spaces, $E \subseteq X$.
A family $\mathscr{F}$ of functions $f: E \to Y$ is
said to be \textbf{equicontinuous} on $E$
if for every $\varepsilon > 0$ there is a $\delta > 0$ such that
\[
  d_Y(f(y),f(x)) < \varepsilon
\]
whenever $d_X(x,y) < \delta$, $x \in E$, $y \in E$, and $f \in \mathscr{F}$.}
$\Box$ \\



\textbf{Theorem 7.24 over $\mathbb{R}^k$.}
\emph{If $K$ is a compact metric space,
if $\mathbf{f}_n \in \mathscr{C}(K)$ for $n=1,2,3,\ldots$,
and if $\{\mathbf{f}_n\}$ converges uniformly on $K$,
then $\{\mathbf{f}_n\}$ is equicontinuous on $K$.} \\

\emph{Proof (Brute-force).}
\begin{enumerate}
\item[(1)]
  Note that Definition 7.14 works for vector-valued functions.

\item[(2)]
  Similar to the proof of Theorem 7.24.
  Let $\varepsilon > 0$ be given.
  Since $\{\mathbf{f}_n\}$ converges uniformly, there is an integer $N$ such that
  \[
    \norm{ \mathbf{f}_n - \mathbf{f}_N } < \frac{\varepsilon}{3}
    \qquad
    (n > N).
  \]

\item[(3)]
  Since continuous functions are uniformly continuous on compact set,
  there is a $\delta > 0$ such that
  \[
    \abs{ \mathbf{f}_i(x) - \mathbf{f}_i(y) } < \frac{\varepsilon}{3}
  \]
  if $1 \leq i \leq N$ and $d(x,y) < \delta$.

\item[(4)]
  If $n > N$ and $d(x,y) < \delta$, it follows that
  \begin{align*}
    &\abs{\mathbf{f}_n(x)-\mathbf{f}_n(y)} \\
    \leq& \abs{\mathbf{f}_n(x)-\mathbf{f}_N(x)}
     + \abs{\mathbf{f}_N(x)-\mathbf{f}_N(y)}
     + \abs{\mathbf{f}_N(y)-\mathbf{f}_n(y)} \\
    \leq& \norm{ \mathbf{f}_n - \mathbf{f}_N }
     + \abs{\mathbf{f}_N(x)-\mathbf{f}_N(y)}
     + \norm{ \mathbf{f}_n - \mathbf{f}_N } \\
    <& \frac{\varepsilon}{3} + \frac{\varepsilon}{3} + \frac{\varepsilon}{3} \\
    =& \varepsilon.
  \end{align*}

\item[(5)]
By (3)(4), the result is established.
\end{enumerate}
$\Box$ \\



\emph{Proof (Theorem 7.24).}
\begin{enumerate}
\item[(1)]
  Write
  \[
    \mathbf{f}_n = \left(f_{n(1)}, \ldots, f_{n(k)}\right)
  \]
  be the corresponding mappings of $K$ into $\mathbb{R}^k$.
  Each $f_{n(j)} \in \mathscr{C}(K)$ too.

\item[(2)]
  $\{\mathbf{f}_n\}$ converges uniformly on $K$
  if and only if
  $\{ f_{n(j)} \}$ converges uniformly on $K$ for all $j=1,\ldots,k$.

\item[(3)]
  By Theorem 7.24,
  $\{ f_{n(j)} \}$ is equicontinuous on $K$ for all $j=1,\ldots,k$.
  Let $\varepsilon > 0$ be given.
  There exist $\delta_j > 0$ such that
  \[
    \abs{f_{n(j)}(x) - f_{n(j)}(y)} < \frac{\varepsilon}{\sqrt{k}}
  \]
  whenever $d(x,y) < \delta_j$, $x \in K$ and $y \in K$.
  Take $\delta = \min\{\delta_1, \ldots, \delta_k\} > 0$.
  Hence
  \begin{align*}
    &\abs{\mathbf{f}_n(x) - \mathbf{f}(y)} \\
    =& \left\{ \abs{f_{n(1)}(x) - f_{n(1)}(y)}^2
      + \ldots
      + \abs{f_{n(k)}(x) - f_{n(k)}(y)}^2 \right\}^{\frac{1}{2}} \\
    <& \left\{ k \cdot \left(\frac{\varepsilon}{\sqrt{k}}\right)^2 \right\}^{\frac{1}{2}} \\
    =& \varepsilon.
  \end{align*}
\end{enumerate}
$\Box$ \\



\textbf{Theorem 7.25 over $\mathbb{R}^k$ (Arzel\`{a}-Ascoli theorem).}
\emph{If $K$ is compact, if $\mathbf{f}_n \in \mathscr{C}(K)$ for $n = 1,2,3,\ldots$,
and if $\{\mathbf{f}_n\}$ is pointwise bounded and equicontinuous on $K$, then}
\begin{enumerate}
\item[(a)]
  \emph{$\{\mathbf{f}_n\}$ is uniformly bounded on $K$.}

\item[(b)]
  \emph{$\{\mathbf{f}_n\}$ contains a uniformly convergent subsequence.} \\
\end{enumerate}

\emph{Proof of (a)(Theorem 7.25).}
\begin{enumerate}
\item[(1)]
  Write
  \[
    \mathbf{f}_n = \left(f_{n(1)}, \ldots, f_{n(k)}\right)
  \]
  be the corresponding mappings of $K$ into $\mathbb{R}^k$.
  Each $f_{n(j)} \in \mathscr{C}(K)$ too.

\item[(2)]
  $\{\mathbf{f}_n\}$ is pointwise bounded on $K$
  if and only if
  $\{ f_{n(j)} \}$ is pointwise bounded on $K$ for all $j=1,\ldots,k$.

\item[(3)]
  $\{\mathbf{f}_n\}$ is equicontinuous on $K$
  if and only if
  $\{ f_{n(j)} \}$ is equicontinuous on $K$ for all $j=1,\ldots,k$.

\item[(4)]
  By Theorem 7.25(a),
  $\{ f_{n(j)} \}$ is uniformly bounded on $K$,
  say $\abs{f_{n(j)}(x)} \leq M_j$ ($x \in K$), for all $j=1,\ldots,k$.
  So
  \begin{align*}
    \abs{\mathbf{f}_n(x)}
    &= \left\{ f_{n(1)}(x)^2 + \cdots + f_{n(k)}(x)^2 \right\}^{\frac{1}{2}} \\
    &\leq \left\{ M_1^2 + \cdots + M_k^2 \right\}^{\frac{1}{2}}.
  \end{align*}
  Here $\left\{ M_1^2 + \cdots + M_k^2 \right\}^{\frac{1}{2}}$ is a constant
  and thus $\{\mathbf{f}_n\}$ is uniformly bounded on $K$.
  \end{enumerate}
$\Box$ \\



\emph{Proof of (b)(Theorem 7.25).}
\begin{enumerate}
\item[(1)]
  Write
  \[
    \mathbf{f}_n = \left(f_{n(1)}, \ldots, f_{n(k)}\right)
  \]
  be the corresponding mappings of $K$ into $\mathbb{R}^k$.
  Each $f_{n(j)} \in \mathscr{C}(K)$ too.

\item[(2)]
  $\{\mathbf{f}_n\}$ is pointwise bounded on $K$
  if and only if
  $\{ f_{n(j)} \}$ is pointwise bounded on $K$ for all $j=1,\ldots,k$.

\item[(3)]
  $\{\mathbf{f}_n\}$ is equicontinuous on $K$
  if and only if
  $\{ f_{n(j)} \}$ is equicontinuous on $K$ for all $j=1,\ldots,k$.

\item[(4)]
  By Theorem 7.25(b),
  $\{ f_{n(1)} \}$ contains a uniformly convergent subsequence,
  say
  \[
    \left\{ f_{n_{m(1)}(1)} \right\}.
  \]

\item[(5)]
  Again,
  $\{ f_{n_{m(1)}(2)} \}$ contains a uniformly convergent subsequence,
  say
  \[
    \left\{ f_{n_{m(1),m(2)}(2)} \right\}.
  \]
  Note that
  \[
    \left\{ f_{n_{m(1),m(2)}(1)} \right\}
  \]
  is also a subsequence of $\left\{ f_{n_{m(1)}(1)} \right\}$ or $\{f_{n(1)}\}$
  and thus is uniformly convergent.

\item[(6)]
  Continue this process again and again.
  So we can get a uniformly convergent subsequence of $\{ f_{n(j)} \}$,
  say
  \[
    \left\{ f_{n_{m(1),\ldots,m(k)}(j)} \right\},
  \]
  which is uniformly convergent for all $j=1,\ldots,k$.
  Therefore,
  $\{\mathbf{f}_n\}$ contains a uniformly convergent subsequence
  \[
    \left\{ \mathbf{f}_{n_{m(1),\ldots,m(k)}} \right\}.
  \]
  \end{enumerate}
$\Box$ \\\\



%%%%%%%%%%%%%%%%%%%%%%%%%%%%%%%%%%%%%%%%%%%%%%%%%%%%%%%%%%%%%%%%%%%%%%%%%%%%%%%%



\textbf{Exercise 7.18.}
\emph{Let $\{f_n\}$ be a uniformly bounded sequence of functions which are
Riemann-integrable on $[a,b]$, and put
\[
  F_n(x) = \int_{a}^{x} f_n(t)dt
  \qquad
  (a \leq x \leq b).
\]
Prove that there exists a subsequence $\{F_{n_k}\}$ which
converges uniformly on $[a,b]$.} \\

\emph{Proof (Theorem 7.25).}
\begin{enumerate}
\item[(1)]
  Since $\{f_n\}$ is uniformly bounded,
  there exists a $M$ such that $|f_n(x)| \leq M$ for all $n$ and $x \in [a,b]$.
  Note that $[a,b]$ is compact.

\item[(2)]
  \emph{Show that $\{F_n\}$ is uniformly bounded
  (and thus pointwise bounded) on $[a,b]$.}
  \begin{align*}
    \abs{F_n(x)}
    &= \abs{ \int_{a}^{x} f_n(t)dt } \\
    &\leq \int_{a}^{x} \abs{f_n(t)}dt \\
    &\leq \int_{a}^{x} M dt \\
    &= (x-a)M \\
    &\leq (b-a)M.
  \end{align*}
  Here $(b-a)M$ is constant.

\item[(3)]
  \emph{Show that $\{F_n\}$ is equicontinuous on $[a,b]$.}
  Similar to (2).
  Given any $\varepsilon > 0$,
  there is a $\delta = \frac{\varepsilon}{M+1} > 0$ such that
  \begin{align*}
    \abs{F_n(x)-F_n(y)}
    &= \abs{ \int_{a}^{x} f_n(t)dt - \int_{a}^{y} f_n(t)dt } \\
    &= \abs{ \int_{y}^{x} f_n(t)dt } \\
    &\leq \int_{\min\{x,y\}}^{\max\{x,y\}} \abs{f_n(t)}dt \\
    &\leq \int_{\min\{x,y\}}^{\max\{x,y\}} M dt \\
    &\leq |x-y|M \\
    &\leq \frac{\varepsilon}{M+1} \cdot M \\
    &< \varepsilon
  \end{align*}
  whenever $|x-y| < \delta$, $x \in [a,b]$ and $y \in [a,b]$.
  Hence $\{F_n\}$ is equicontinuous on $[a,b]$.

\item[(4)]
  By Theorem 7.25(b)(Arzel\`{a}-Ascoli theorem),
  $\{F_n\}$ contains a uniformly convergent sequence.
\end{enumerate}
$\Box$ \\\\



%%%%%%%%%%%%%%%%%%%%%%%%%%%%%%%%%%%%%%%%%%%%%%%%%%%%%%%%%%%%%%%%%%%%%%%%%%%%%%%%



\textbf{Exercise 7.19.}
\emph{Let $K$ be a compact metric space,
let $S$ be a subset of $\mathscr{C}(K)$.
Prove that $S$ is compact (with respect to the metric defined in Section 7.14)
if and only if
$S$ is uniformly closed, pointwise bounded, and equicontinuous.
(If $S$ is not equicontinuous,
then $S$ contains a sequence which has no equicontinuous subsequence,
hence has no subsequence that converges uniformly on $K$.)} \\

\emph{Proof.}
\begin{enumerate}
\item[(1)]
  \emph{Show that $S$ is compact
  if $S$ is uniformly closed, pointwise bounded, and equicontinuous.}
  \begin{enumerate}
  \item[(a)]
    By Exercise 2.26, it suffices to
    \emph{show that every infinite subset $E$ of $S$ has a limit point.}
    From such infinite subset $E$ we can take
    $f_n \in E \subseteq S \subseteq \mathscr{C}(K)$ for $n = 1,2,3,\ldots$.
    Consider the sequence $\{f_n\}$.

  \item[(b)]
    $\{f_n\}$ is uniformly (pointwise) bounded since $S$ is uniformly bounded.

  \item[(c)]
    $\{f_n\}$ is equicontinuous since $S$ is equicontinuous.

  \item[(d)]
    Since $K$ is compact, we can apply Theorem 7.25(b)(Arzel\`{a}-Ascoli theorem)
    to (b)(c) to get that
    $\{f_n\}$ contains a uniformly convergent subsequence
    $\{f_{n_k}\}$ converging to a limit point $f$.
    Since $f_{n_k} \in S$ and $S$ is uniformly closed, $f \in S$.
  \end{enumerate}

\item[(2)]
  \emph{Show that $S$ is uniformly closed if $S$ is compact.}
  $S$ is closed since $S$ is compact (Theorem 2.34).
  By Definition 7.14, $S$ is also called uniformly closed.

\item[(3)]
  \emph{Show that $S$ is uniformly (pointwise) bounded if $S$ is compact.}
  Let
  \[
    G_n = \{ f \in S : \norm{f} < n \}
  \]
  for $n = 1,2,3,\ldots$.
  Note that
  \[
    \mathscr{C} = \{ G_n \}_{n=1,2,3,\ldots}
  \]
  is an open covering of $S$ since each $f \in S \subseteq \mathscr{C}(K)$ is bounded.
  Since $S$ is compact, there exists a finite subcovering $\mathscr{C}'$ of $\mathscr{C}$,
  say
  \[
    \mathscr{C}' = \left\{ G_{n_1}, \ldots, G_{n_k} \right\}.
  \]
  Let $N = \max\{n_1, \ldots, n_k\}$.
  Then
  \[
    S \subseteq G_{n_1} \cup \cdots \cup G_{n_k} = G_N,
  \]
  or $S$ is uniformly (pointwise) bounded by $N$.

\item[(4)]
  \emph{Show that $S$ is equicontinuous if $S$ is compact.}
  (Reductio ad absurdum)
  If $S$ were not equicontinuous,
  then $S$ contains a sequence $\{f_n\}$ which has no equicontinuous subsequence,
  hence has no subsequence that converges uniformly on $K$.
  However, $\{f_n\} \subseteq S$ and the compactness of $S$ imply that
  some subsequence of $\{f_n\}$ converges to a point of $S$ (Theorem 3.6(a)),
  which is absurd.
\end{enumerate}
$\Box$ \\\\



%%%%%%%%%%%%%%%%%%%%%%%%%%%%%%%%%%%%%%%%%%%%%%%%%%%%%%%%%%%%%%%%%%%%%%%%%%%%%%%%



\textbf{Exercise 7.20.}
\emph{If $f$ is continuous on $[0,1]$ and if
\[
  \int_{0}^{1} f(x) x^n dx = 0
  \qquad
  (n=0,1,2,\ldots),
\]
prove that $f(x) = 0$ on $[0,1]$.
(Hint: The integral of the product of $f$ with any polynomial is zero.
Use the Weierstrass theorem to show that
$\int_{0}^{1} f^2(x) dx = 0$.)} \\

\emph{Proof.}
\begin{enumerate}
\item[(1)]
Since $\int_{0}^{1} f(x) x^n dx = 0$ for all $n = 0,1,2,\ldots$,
\[
  \int_{0}^{1} f(x) P(x) dx = 0 \text{ for all } P(x) \in \mathbb{R}[x].
\]

\item[(2)]
By Theorem 7.26 (Stone-Weierstrass Theorem),
there exists a sequence of $P_n(x) \in \mathbb{R}[x]$ such that
\[
  P_n(x) \to f(x)
\]
uniformly on $[0,1]$.
Since $f(x)$ is continuous on the compact set $[0,1]$, $f(x)$ is bounded on $[0,1]$.
Hence
\[
  f(x) P_n(x) \to f^2(x)
\]
uniformly on $[0,1]$.

\item[(3)]
Since each $f(x) P_n(x)$ is continuous,
$f(x) P_n(x) \in \mathscr{R}$ on $[0,1]$ (Theorem 6.8).
By Theorem 7.16,
\[
  \int_{0}^{1} f^2(x) dx
  = \lim_{n \to \infty} \int_{0}^{1} f(x) P_n(x) dx
  = \lim_{n \to \infty} 0
  = 0.
\]

\item[(4)]
Since $f^2(x)$ is continuous,
$f^2(x) = 0$ or $f(x) = 0$ by (3) and Exercise 6.2.
\end{enumerate}
$\Box$ \\\\



%%%%%%%%%%%%%%%%%%%%%%%%%%%%%%%%%%%%%%%%%%%%%%%%%%%%%%%%%%%%%%%%%%%%%%%%%%%%%%%%



\textbf{Exercise 7.21.}
\emph{Let $K$ be the unit circle in the complex plane
(i.e., the set of all $z$ with $|z|=1$),
and let $\mathscr{A}$ be the algebra of all functions of the form
\[
  f(e^{i\theta}) = \sum_{n=0}^{N} c_n e^{in\theta}.
  \qquad
  (\theta \text{ real}).
\]
Then $\mathscr{A}$ separates points on $K$ and
$\mathscr{A}$ vanishes at no point of $K$.
But nevertheless there are continuous functions on $K$ which
are not in the uniform closure of $\mathscr{A}$.
(Hint: For every $f \in \mathscr{A}$
\[
  \int_{0}^{2\pi} f(e^{i\theta})e^{i\theta}d\theta = 0,
\]
and this is also true for every $f$ in the clousure of $\mathscr{A}$.)} \\

\emph{Note.}
$\mathscr{A}$ is not a self-adjoint algebra of complex continuous functions
on $K$, and thus Theorem 7.33 does not hold. \\

\emph{Proof.}
\begin{enumerate}
\item[(1)]
  \emph{Show that $\mathscr{A}$ separates points on $K$ and
  $\mathscr{A}$ vanishes at no point of $K$.}
  Take $f \in \mathscr{A}$ defined by
  \[
    f(e^{i\theta}) = e^{i\theta}.
  \]
  Since $f(z) = z$ for every $z \in K$,
  $f(z_1) \neq f(z_2)$ for every pair of distinct points $z_1, z_2 \in K$.
  Hence $\mathscr{A}$ separates points on $K$.
  Besides, $f(z) \neq 0$ for all $z \in K = \{ z \in \mathbb{C} : |z|=1 \}$.
  Hence $\mathscr{A}$ vanishes at no point of $K$.

\item[(2)]
  \emph{Show that for every $f \in \mathscr{A}$}
  \[
    \int_{0}^{2\pi} f(e^{i\theta})e^{i\theta}d\theta = 0.
  \]
  Similar to Definition 8.9. Given $f \in \mathscr{A}$.
  \begin{align*}
    \int_{0}^{2\pi} f(e^{i\theta})e^{i\theta}d\theta
    &= \int_{0}^{2\pi} \sum_{n=0}^{N} c_n e^{in\theta}e^{i\theta}d\theta \\
    &= \sum_{n=0}^{N} c_n \int_{0}^{2\pi} e^{i(n+1)\theta} d\theta \\
    &= \sum_{n=0}^{N} c_n \cdot 0 \\
    &= 0.
  \end{align*}

\item[(3)]
  \emph{Show that for every $f$ in the clousure of $\mathscr{A}$}
  \[
    \int_{0}^{2\pi} f(e^{i\theta})e^{i\theta}d\theta = 0.
  \]
  Given $f$ in the clousure of $\mathscr{A}$.
  Then $f$ is the limit function of some uniformly convergent sequences $\{f_n\}$
  of members of $\mathscr{A}$.
  Note that $e^{i\theta}$ is bounded on $K$
  and thus $f_n(e^{i\theta})e^{i\theta} \to f(e^{i\theta})e^{i\theta}$ uniformly (Theorem 7.9).
  So Theorem 7.16 implies that
  \[
    \int_{0}^{2\pi} f(e^{i\theta})e^{i\theta}d\theta
    = \lim_{n \to \infty} \int_{0}^{2\pi} f_n(e^{i\theta})e^{i\theta}d\theta
    = \lim_{n \to \infty} 0
    = 0.
  \]

\item[(4)]
  Consider $g(e^{i\theta}) = e^{-i\theta}$ on $K$, or $g(z) = \frac{1}{z}$.
  By definition, $g$ is continuous on $K$ but
  \[
    \int_{0}^{2\pi} g(e^{i\theta})e^{i\theta}d\theta
    = \int_{0}^{2\pi} d\theta
    = 2\pi
    \neq 0.
  \]
  Hence there are continuous functions on $K$ which
  are not in the uniform closure of $\mathscr{A}$.
\end{enumerate}
$\Box$ \\\\



%%%%%%%%%%%%%%%%%%%%%%%%%%%%%%%%%%%%%%%%%%%%%%%%%%%%%%%%%%%%%%%%%%%%%%%%%%%%%%%%



\textbf{Exercise 7.22.}
\emph{Assume $f \in \mathscr{R}(\alpha)$ on $[a,b]$,
and prove that there are polynomials $P_n$ such that
\[
  \lim_{n \to \infty} \int_{a}^{b} |f-P_n|^2 d\alpha = 0.
\]
(Compare with Exercise 6.12.)} \\

\emph{Notation.}
For $u \in \mathscr{R}(\alpha)$ on $[a,b]$, define
  \[
    \norm{u}_2 = \left\{ \int_{a}^{b}|u|^2 d\alpha \right\}^{\frac{1}{2}}.
  \] \\

\emph{Proof.}
Given any $\varepsilon = \frac{1}{n} > 0$ $(n=1,2,3,\ldots$).
\begin{enumerate}
\item[(1)]
  By Exercise 6.12, there exists a continuous function $g_n$ on $[a,b]$
  such that
  \[
    \norm{f-g_n}_2 < \frac{1}{n}.
  \]

\item[(2)]
  By Theorem 7.26 (Stone-Weierstrass Theorem),
  there is a polynomial $P_n$ such that
  \[
    |g_n(x)-P_n(x)| < \frac{1}{n}
  \]
  for all $x \in [a,b]$.
  Thus
  \[
    \norm{g_n-P_n}_2
    \leq
    \left\{ \int_{a}^{b}\left(\frac{1}{n}\right)^2 d\alpha \right\}^{\frac{1}{2}}
    =
    \frac{(\alpha(b)-\alpha(a))^{\frac{1}{2}}}{n}.
  \]

\item[(3)]
  By Exercise 6.11,
  \[
    \norm{f-P_n}_2
    \leq
    \norm{f-g_n}_2 + \norm{g_n-P_n}_2
    \leq
    \frac{1+(\alpha(b)-\alpha(a))^{\frac{1}{2}}}{n},
  \]
  or
  \[
    0
    \leq
    \int_{a}^{b} |f-P_n|^2 d\alpha
    \leq
    \frac{[1+(\alpha(b)-\alpha(a))^{\frac{1}{2}}]^2}{n^2}.
  \]
  As $n \to \infty$, $\int_{a}^{b} |f-P_n|^2 d\alpha \to 0$.
\end{enumerate}
$\Box$ \\\\



%%%%%%%%%%%%%%%%%%%%%%%%%%%%%%%%%%%%%%%%%%%%%%%%%%%%%%%%%%%%%%%%%%%%%%%%%%%%%%%%



\textbf{Exercise 7.23.}
\emph{Put $P_0 = 0$, and define, for $n = 0,1,2,\ldots$,
\[
  P_{n+1}(x) = P_n(x) + \frac{x^2-P_n^2(x)}{2}.
\]
Prove that
\[
  \lim_{n \to \infty} P_n(x) = |x|,
\]
uniformly on $[-1,1]$.
(This makes it possible to prove the Stone-Weierstrass theorem without
first proving Theorem 7.26.)
(Hint: Use the identity
\[
  |x| - P_{n+1} = [|x| - P_n(x)]\left[1-\frac{|x|+P_n(x)}{2}\right]
\]
to prove that $0 \leq P_n(x) \leq P_{n+1}(x) \leq |x|$ if $|x| \leq 1$, and that
\[
  |x| - P_n(x) \leq |x| \left(1-\frac{|x|}{2}\right)^n < \frac{2}{n+1}
\]
if $|x| \leq 1$.)} \\

\emph{Proof (Hint).}
\begin{enumerate}
\item[(1)]
\begin{align*}
  |x| - P_{n+1}(x)
  &= |x| - P_n(x) - \frac{|x|^2 - P_n^2(x)}{2} \\
  &= |x| - P_n(x) - \frac{(|x| + P_n(x))(|x| - P_n(x))}{2} \\
  &= [|x| - P_n(x)]\left[1-\frac{|x|+P_n(x)}{2}\right].
\end{align*}

\item[(2)]
\emph{Show that $0 \leq P_n(x) \leq |x|$ if $|x| \leq 1$.}
Induction on $n$.
  \begin{enumerate}
  \item[(a)]
  If $n = 0$, then $P_n(x) = P_0(x) = 0$ and thus $0 \leq P_0(x) \leq |x|$.

  \item[(b)]
  Assume the induction hypothesis that for the single case $n = k$ holds,
  and thus $0 \leq P_k(x) \leq |x|$ if $|x| \leq 1$.
  So
  \begin{align*}
    0 &\leq |x|-P_k(x) \leq |x|, \\
    0 \leq 1-|x| &\leq 1-\frac{|x|+P_k(x)}{2} \leq 1-\frac{|x|}{2} \leq 1
  \end{align*}
  if $|x| \leq 1$.
  Hence
  \[
    0 \leq [|x| - P_k(x)]\left[1-\frac{|x|+P_k(x)}{2}\right] \leq |x|.
  \]
  By (1),
  \[
    0 \leq |x| - P_{k+1}(x) \leq |x|
  \]
  or $0 \leq P_{k+1}(x) \leq |x|$ if $|x| \leq 1$

  \item[(c)]
  Since both the base case in (a) and
  the inductive step in (b) have been proved as true,
  by mathematical induction the result holds.
  \end{enumerate}

\item[(3)]
\emph{Show that $0 \leq P_n(x) \leq P_{n+1}(x) \leq |x|$ if $|x| \leq 1$.}
By (2), it suffices to show that $P_{n}(x) \leq P_{n+1}(x)$.
By (1)(2), we have
\begin{align*}
  |x| - P_{n+1}(x)
  &= [|x| - P_n(x)]\left[1-\frac{|x|+P_n(x)}{2}\right] \\
  &\leq |x| - P_n(x)
\end{align*}
or $P_{n}(x) \leq P_{n+1}(x)$.

\item[(4)]
\emph{Define $f_n(t) = t(1-t)^n$ on $\left[0,\frac{1}{2}\right]$ for $n=1,2,3,\ldots$.
Show that $f_n(t) \leq \frac{1}{n+1}$.}
Since
\[
  f'_n(t) = (1-t)^{n-1}(1 - (n+1)t)
\]
$f'_n(t) = 0$ on $\left[0,\frac{1}{2}\right]$ if and only if $t = \frac{1}{n+1}$.
By Theorem 5.11, $f_n(t)$ reaches its maximum at $t = \frac{1}{n+1}$.
Hence
\[
  f_n(t)
  \leq f_n\left(\frac{1}{n+1}\right)
  = \frac{1}{n+1} \left(\frac{n}{n+1}\right)^n
  < \frac{1}{n+1}.
\]

\item[(5)]
\emph{Show that
\[
  |x| - P_n(x) \leq |x| \left(1-\frac{|x|}{2}\right)^n < \frac{2}{n+1}
\]
if $|x| \leq 1$.}
Note that
\begin{align*}
  |x| - P_n(x)
  &\leq [ |x| - P_0(x) ] \prod_{k=0}^{n-1}\left[1-\frac{|x|+P_k(x)}{2}\right]
    & ((1)) \\
  &\leq |x| \prod_{k=0}^{n-1}\left[1-\frac{|x|}{2}\right]
    & ((2)) \\
  &\leq |x| \left[1-\frac{|x|}{2}\right]^n \\
  & < \frac{2}{n+1}
    & \text{(Put $t=\frac{|x|}{2}$ in (4))}.
\end{align*}

\item[(6)]
(5) implies that
\[
  \lim_{n \to \infty} P_n(x) = |x|
\]
and
\[
  \lim_{n \to \infty} \sup_{x \in [-1,1]}\abs{ P_n(x) - |x| } = 0.
\]
By Theorem 7.9, $P_n(x) \to |x|$ uniformly on $[-1,1]$.
\end{enumerate}
$\Box$ \\\\



%%%%%%%%%%%%%%%%%%%%%%%%%%%%%%%%%%%%%%%%%%%%%%%%%%%%%%%%%%%%%%%%%%%%%%%%%%%%%%%%



\textbf{Exercise 7.24.}
\emph{Let $X$ be a metric space, with metric $d$.
Fix a point $a \in X$.
Assign to each $p \in X$ the function $f_p$ defined by
\[
  f_p(x) = d(x,p) - d(x,a)
  \qquad
  (x \in X).
\]
Prove that $\abs{ f_p(x) } \leq d(a,p)$ for all $x \in X$,
and that therefore, $f_p \in \mathscr{C}(X)$.
Prove that
\[
  \norm{ f_p - f_q } = d(p,q)
\]
for all $p,q \in X$.
If $\Phi(p) = f_p$ it follows that $\Phi$ is an \textbf{isometry}
(a distance-preserving mapping)
of $X$ onto $\Phi(X) \subseteq \mathscr{C}(X)$.
Let $Y$ be the closure of $\Phi(X)$ in $\mathscr{C}(X)$.
Show that $Y$ is complete.
Conclusion: $X$ is isometric to a dense subset of a complete metric space $Y$.
(Exercise 3.24 contains a different proof of this.)} \\

\emph{Proof.}
\begin{enumerate}
\item[(1)]
  \emph{Show that $\abs{ f_p(x) } \leq d(a,p)$ for all $x \in X$.}
  Since $d$ is a metric,
  \begin{align*}
    d(x,p) - d(x,a) &\leq d(a,p),
    d(x,a) - d(x,p) &\leq d(a,p)
  \end{align*}
  for any $x \in X$.
  Thus $\abs{ f_p(x) } \leq d(a,p)$ for any $x \in X$.

\item[(2)]
  \emph{Show that $f_p \in \mathscr{C}(X)$, i.e, $f_p$ is continuous on $X$.}
  Given any $\varepsilon > 0$, there exists a $\delta = \frac{\varepsilon}{2}$ such that
  \begin{align*}
    \abs{ f_p(x) - f_p(y) }
    &= \abs{ (d(x,p) - d(x,a)) - (d(y,p) - d(y,a)) } \\
    &= \abs{ (d(x,p) - d(y,p)) - (d(x,a) - d(y,a)) } \\
    &\leq \abs{ d(x,p) - d(y,p) } + \abs{ d(x,a) - d(y,a) } \\
    &\leq d(x,y) + d(x,y) \\
    &< \delta + \delta \\
    &= \varepsilon
  \end{align*}
  whenever $d(x,y) < \delta$ and $x,y \in X$.
  Hence $f_p$ is (uniformly) continuous on $X$.

\item[(3)]
  \emph{Show that
  \[
    \norm{ f_p - f_q } = d(p,q)
  \]
  for all $p,q \in X$.}
  \begin{enumerate}
  \item[(a)]
    \emph{Show that $\norm{ f_p - f_q } \leq d(p,q)$.}
    Given any $x \in X$, we have
    \begin{align*}
      \abs{f_p(x)-f_q(x)}
      &= \abs{(d(x,p) - d(x,a))-(d(x,q) - d(x,a))} \\
      &= \abs{d(x,p) - d(x,q)} \\
      &\leq d(p,q).
    \end{align*}
    $d(p,q)$ is an upper bound of $\left\{ \abs{f_p(x)-f_q(x)} : x \in E \right\}$.
    Take sup over all $x \in E$ to get $\norm{ f_p - f_q } \leq d(p,q)$.

  \item[(b)]
    \emph{Show that $\norm{ f_p - f_q } \geq d(p,q)$.}
    Note that
    \begin{align*}
      \norm{ f_p - f_q }
      &= \sup_{x \in E} \abs{f_p(x)-f_q(x)} \\
      &\geq \sup_{x \in \{p,q\}} \abs{f_p(x)-f_q(x)} \\
      &= \sup_{x \in \{p,q\}} d(p,q) \\
      &= d(p,q).
    \end{align*}
  \end{enumerate}

\item[(4)]
  \emph{Show that $Y$ is complete.}
  By Theorem 3.11 and Definition 3.12,
  every closed subset of a complete metric space is complete.
  Note that $Y$ is a closed subset (by construction)
  of a complete metric space $\mathscr{C}(X)$ (Theorem 7.15).
  Therefore $Y$ is complete.
\end{enumerate}
$\Box$ \\\\



%%%%%%%%%%%%%%%%%%%%%%%%%%%%%%%%%%%%%%%%%%%%%%%%%%%%%%%%%%%%%%%%%%%%%%%%%%%%%%%%



\textbf{Exercise 7.25.}
\emph{Suppose $\phi$ is a continuous bounded real function in the strip defined by
$0 \leq x \leq 1$, $-\infty < y < \infty$.
Prove that the initial-value problem
\[
  y' = \phi(x, y), \qquad y(0) = c
\]
has a solution.
(Note that the hypothesis of the existence theorem are less stringent than
those of the corresponding uniqueness theorem; see Exercise 5.27.)}
\emph{(Hint: Fix $n$.
For $i = 0,\ldots,n$, put $x_i = \frac{i}{n}$.
Let $f_n$ be a continuous function on $[0, 1]$ such that $f_n(0) = c$,
\[
  f'_n(t) = \phi(x_i, f_n(x_i))
  \qquad \text{if } x_i < t < x_{i+1}
\]
and put
\[
  \Delta_n(t) = f_n'(t) - \phi(t, f_n(t)),
\]
except at the points $x_i$, where $\Delta_n(t) = 0$.
Then
\[
  f_n(x) = c + \int_{0}^{x} [\phi(t, f_n(t)) + \Delta_n(t)] dt.
\]
Choose $M < \infty$ so that $|\phi| \leq M$.
Verify the following assertions.}
\begin{enumerate}
  \item[(a)]
  \emph{$|f_n'| \leq M$, $|\Delta_n| \leq 2M$, $\Delta \in \mathscr{R}$,
  and $|f_n| \leq |c| + M = M_1$, say, on $[0, 1]$, for all $n$.}

  \item[(b)]
  \emph{$\{f_n\}$ is equicontinuous on $[0, 1]$, since $|f_n'| \leq M$.}

  \item[(c)]
  \emph{Some $\{f_{n_k}\}$ converges to some $f$, uniformly on $[0, 1]$.}

  \item[(d)]
  \emph{Since $\phi$ is uniformly continuous on the rectangle $0 \leq x \leq 1$, $|y| \leq M_1$,
  \[
    \phi(t, f_{n_k}(t)) \to \phi(t, f(t))
  \]
  uniformly on $[0, 1]$.}

  \item[(e)]
  \emph{$\Delta_n(t) \to 0$ uniformly on $[0, 1]$,
  since
  \[
    \Delta_n(t) = \phi(x_i, f_n(x_i)) - \phi(t, f_n(t))
  \]
  in $(x_i, x_{i+1})$.}

  \item[(f)]
  \emph{Hence}
  \[
    f(x) = c + \int_{0}^{x} \phi(t, f(t)) dt.
  \]
\end{enumerate}
\emph{This $f$ is a solution of the given problem.)} \\



%%%%%%%%%%%%%%%%%%%%%%%%%%%%%%%%%%%%%%%%%%%%%%%%%%%%%%%%%%%%%%%%%%%%%%%%%%%%%%%%



\textbf{Exercise 7.26.}
\emph{Prove an analogous existence theorem for the initial-value problem
\[
  \mathbf{y}' = \bm{\Phi}(x, \mathbf{y}), \qquad \mathbf{y}(0) = \mathbf{c},
\]
where now $\mathbf{c} \in \mathbb{R}^k$, $\mathbf{y} \in \mathbb{R}^k$,
and $\bm{\Phi}$ is a continuous bounded mapping of the part of $\mathbb{R}^{k+1}$
defined by $0 \leq x \leq 1$, $\mathbf{y} \in \mathbb{R}^k$ into $\mathbb{R}^k$.
(Compare Exercise 5.28.)
(Hint: Use the vector-valued version of Theorem 7.25.)} \\



%%%%%%%%%%%%%%%%%%%%%%%%%%%%%%%%%%%%%%%%%%%%%%%%%%%%%%%%%%%%%%%%%%%%%%%%%%%%%%%%
%%%%%%%%%%%%%%%%%%%%%%%%%%%%%%%%%%%%%%%%%%%%%%%%%%%%%%%%%%%%%%%%%%%%%%%%%%%%%%%%



\end{document}