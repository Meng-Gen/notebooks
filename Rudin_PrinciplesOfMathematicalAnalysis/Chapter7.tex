\documentclass{article}
\usepackage{amsfonts}
\usepackage{amsmath}
\usepackage{amssymb}
\usepackage{hyperref}
\usepackage[none]{hyphenat}
\usepackage{mathrsfs}
\usepackage{physics}
\parindent=0pt

\def\upint{\mathchoice%
    {\mkern13mu\overline{\vphantom{\intop}\mkern7mu}\mkern-20mu}%
    {\mkern7mu\overline{\vphantom{\intop}\mkern7mu}\mkern-14mu}%
    {\mkern7mu\overline{\vphantom{\intop}\mkern7mu}\mkern-14mu}%
    {\mkern7mu\overline{\vphantom{\intop}\mkern7mu}\mkern-14mu}%
  \int}
\def\lowint{\mkern3mu\underline{\vphantom{\intop}\mkern7mu}\mkern-10mu\int}

\begin{document}



\textbf{\Large Chapter 7: Sequences and Series of Functions} \\\\



\emph{Author: Meng-Gen Tsai} \\
\emph{Email: plover@gmail.com} \\\\



%%%%%%%%%%%%%%%%%%%%%%%%%%%%%%%%%%%%%%%%%%%%%%%%%%%%%%%%%%%%%%%%%%%%%%%%%%%%%%%%
%%%%%%%%%%%%%%%%%%%%%%%%%%%%%%%%%%%%%%%%%%%%%%%%%%%%%%%%%%%%%%%%%%%%%%%%%%%%%%%%



\textbf{Exercise 7.1.}
\emph{Prove that every uniformly convergent sequence of bounded functions
is uniformly bounded.} \\

\emph{Proof (Cauchy criterion).}
Let $\{f_n\}$ be a uniformly convergent sequence of bounded functions.
\begin{enumerate}
\item[(1)]
Since $f_n$ is bounded, there exists $M_n$ such that $|f_n(x)| \leq M_n$.

\item[(2)]
Since $\{f_n\}$ converges uniformly, given $1 > 0$ there exists an integer $N$
such that
\[
  |f_n(x) - f_m(x)| \leq 1 \text{ whenever } n, m \geq N
\]
(Theorem 7.8 (Cauchy criterion for uniformly convergence)).
Especially,
\[
  |f_n(x)| \leq |f_n(x) - f_N(x)| + |f_N(x)| \leq 1 + M_N \text{ whenever } n \geq N.
\]

\item[(3)]
Thus, $\{f_n\}$ is uniformly bounded by $M = \max\{ M_1, \ldots, M_{N-1}, M_{N}+1 \}$.
\end{enumerate}
$\Box$ \\\\



%%%%%%%%%%%%%%%%%%%%%%%%%%%%%%%%%%%%%%%%%%%%%%%%%%%%%%%%%%%%%%%%%%%%%%%%%%%%%%%%



\textbf{Exercise 7.2.}
\emph{If $\{f_n\}$ and $\{g_n\}$ converge uniformly on a set $E$,
prove that $\{f_n + g_n\}$ converge uniformly on $E$.
If, in addition, $\{f_n\}$ and $\{g_n\}$ are sequences of bounded functions,
prove that $\{f_n g_n\}$ converges uniformly on $E$.} \\

\emph{Proof.}
Let $f_n \to f$ uniformly and $g_n \to g$ uniformly.
\begin{enumerate}
  \item[(1)]
  \emph{Show that $\{f_n + g_n\}$ converges uniformly.}
    Given $\varepsilon > 0$.
    Since $f_n \to f$ uniformly and $g_n \to g$ uniformly,
    there exist two integers $N_1$ and $N_2$ such that
    \begin{align*}
      |f_n(x) - f(x)| \leq \frac{\varepsilon}{2}
      &\text{ whenever }
      n \geq N_1, x \in E \\
      |g_n(x) - g(x)| \leq \frac{\varepsilon}{2}
      &\text{ whenever }
      n \geq N_2, x \in E.
    \end{align*}
    Take $N = \max\{N_1,N_2\}$, we have
    \begin{align*}
      &|(f_n(x)+ g_n(x)) - (f(x) + g(x))| \\
      =& |(f_n(x) - f(x)) + (g_n(x) - g(x))| \\
      \leq& |f_n(x) - f(x)| + |g_n(x) - g(x)| \\
      \leq& \frac{\varepsilon}{2} + \frac{\varepsilon}{2} \\
      =& \varepsilon
    \end{align*}
    whenever $n \geq N$, $x \in E$.
    Hence $f_n + g_n \to f+g$ uniformly on $E$.

  \item[(2)]
  \emph{Show that $\{f_n g_n\}$ converges uniformly
  if, in addition, $\{f_n\}$ and $\{g_n\}$ are sequences of bounded functions.}
  Given $\varepsilon > 0$.
  \begin{enumerate}
    \item[(a)]
    By Exercise 7.1, both $\{f_n\}$ and $\{g_n\}$ are uniformly bounded.
    So there exist $M_1$ and $M_2$
    such that
    \[
      |f_n(x)| \leq M_1 \text{ and } |g_n(x)| \leq M_2
    \]
    for all $n$ and $x \in E$.
    Also, $|f(x)| \leq M_1 + 1$ and $|g(x)| \leq M_2 + 1$.

    \item[(b)]
    Since $f_n \to f$ uniformly and $g_n \to g$ uniformly,
    there exist two integers $N_1$ and $N_2$ such that
    \begin{align*}
      |f_n(x) - f(x)| \leq \frac{\varepsilon}{2(M_2 + 1)}
      &\text{ whenever }
      n \geq N_1, x \in E \\
      |g_n(x) - g(x)| \leq \frac{\varepsilon}{2(M_1 + 1)}
      &\text{ whenever }
      n \geq N_2, x \in E.
    \end{align*}
    (Note that each denominator of $\frac{\varepsilon}{2(M_j + 1)}$ $(j=1,2)$
    is well-defined and positive!)
    Take $N = \max\{N_1,N_2\}$, we have
    \begin{align*}
      &|f_n(x)g_n(x) - f(x)g(x)| \\
      =& |[f_n(x) - f(x)]g_n(x) + f(x)[g_n(x) - g(x)]| \\
      \leq& |f_n(x) - f(x)||g_n(x)| + |f(x)||g_n(x) - g(x)| \\
      \leq& \frac{\varepsilon}{2(M_2 + 1)} \cdot M_2
        + (M_1 + 1) \cdot \frac{\varepsilon}{2(M_1 + 1)} \\
      \leq& \varepsilon
    \end{align*}
    whenever $n \geq N$, $x \in E$.
    Hence $f_n g_n \to fg$ uniformly on $E$.
  \end{enumerate}
\end{enumerate}
$\Box$ \\

\emph{Proof (Cauchy criterion).}
\begin{enumerate}
  \item[(1)]
  \emph{Show that $\{f_n + g_n\}$ converges uniformly.}
    Given $\varepsilon > 0$.
    Since $\{f_n\}$ and $\{g_n\}$ converge uniformly,
    there exist two integers $N_1$ and $N_2$ such that
    \begin{align*}
      |f_n(x) - f_m(x)| \leq \frac{\varepsilon}{2}
      &\text{ whenever }
      n,m \geq N_1, x \in E \\
      |g_n(x) - g_m(x)| \leq \frac{\varepsilon}{2}
      &\text{ whenever }
      n,m \geq N_2, x \in E.
    \end{align*}
    Take $N = \max\{N_1,N_2\}$, we have
    \begin{align*}
      &|(f_n(x)+ g_n(x)) - (f_m(x) + g_m(x))| \\
      =& |(f_n(x) - f_n(x)) + (g_n(x) - g_m(x))| \\
      \leq& |f_n(x) - f_n(x)| + |g_n(x) - g_m(x)| \\
      \leq& \frac{\varepsilon}{2} + \frac{\varepsilon}{2} \\
      =& \varepsilon
    \end{align*}
    whenever $n,m \geq N$, $x \in E$.
    Hence $\{f_n + g_n\}$ converges uniformly on $E$.

  \item[(2)]
  \emph{Show that $\{f_n g_n\}$ converges uniformly
  if, in addition, $\{f_n\}$ and $\{g_n\}$ are sequences of bounded functions.}
  Given $\varepsilon > 0$.
  \begin{enumerate}
    \item[(a)]
    By Exercise 7.1, both $\{f_n\}$ and $\{g_n\}$ are uniformly bounded.
    So there exist $M_1$ and $M_2$
    such that
    \[
      |f_n(x)| \leq M_1 \text{ and } |g_n(x)| \leq M_2
    \]
    for all $n$ and $x \in E$.
    Also, $|f(x)| \leq M_1 + 1$ and $|g(x)| \leq M_2 + 1$.

    \item[(b)]
    Since $\{f_n\} \to f$ uniformly and $\{g_n\} \to g$ uniformly,
    there exist two integers $N_1$ and $N_2$ such that
    \begin{align*}
      |f_n(x) - f_m(x)| \leq \frac{\varepsilon}{2(M_2 + 1)}
      &\text{ whenever }
      n,m \geq N_1, x \in E \\
      |g_n(x) - g_m(x)| \leq \frac{\varepsilon}{2(M_1 + 1)}
      &\text{ whenever }
      n,m \geq N_2, x \in E.
    \end{align*}
    Take $N = \max\{N_1,N_2\}$, we have
    \begin{align*}
      &|f_n(x)g_n(x) - f_m(x)g_m(x)| \\
      =& |[f_n(x) - f_m(x)]g_n(x) + f_m(x)[g_n(x) - g_m(x)]| \\
      \leq& |f_n(x) - f_m(x)||g_n(x)| + |f_m(x)||g_n(x) - g_m(x)| \\
      \leq& \frac{\varepsilon}{2(M_2 + 1)} \cdot M_2
        + M_1 \cdot \frac{\varepsilon}{2(M_1 + 1)} \\
      \leq& \varepsilon
    \end{align*}
    whenever $n \geq N$, $x \in E$.
    Hence $\{f_n g_n\}$ converges uniformly on $E$.
  \end{enumerate}
\end{enumerate}
$\Box$ \\\\



%%%%%%%%%%%%%%%%%%%%%%%%%%%%%%%%%%%%%%%%%%%%%%%%%%%%%%%%%%%%%%%%%%%%%%%%%%%%%%%%



\textbf{Exercise 7.3.}
\emph{Construct sequences $\{f_n\}$, $\{g_n\}$ which converge uniformly on some set $E$,
but such that $\{f_n g_n\}$ does not converge uniformly on $E$
(of course, $\{f_n g_n\}$ must converge on $E$).} \\

We provides some examples here. \\

\emph{Proof ($f_n(x) = x + \frac{1}{n}$).}
\begin{enumerate}
  \item[(1)]
  Define $\{f_n(x)\}$ on $E = \mathbb{R}$ by $f_n(x) = x + \frac{1}{n}$ and $f(x) = x$.
  Clearly, $\{f_n(x)\}$ converges to $f(x)$ pointwise.

  \item[(2)]
  \emph{Show that $\{f_n\}$ converges uniformly.}
  Given $\varepsilon > 0$.
  There exists an integer $N \geq \frac{1}{\varepsilon}$ such that
  \[
    |f_n(x) - f(x)| = \frac{1}{n} \leq \frac{1}{N} \leq \varepsilon
  \]
  whenever $n \geq N$ and $x \in E$.
  Hence $\{f_n\} \to f$ uniformly.

  \item[(3)]
  \emph{Show that $\{f_n^2\}$ does not converge uniformly.}
  Clearly, $\{f_n(x)^2\}$ converges to $f(x)^2$ pointwise.
  Hence
  \[
    \sup_{x \in E} |f_n(x)^2 - f(x)^2|
    = \sup_{x \in E} \abs{\frac{2x}{n} + \frac{1}{n^2}}
    \to \infty
  \]
  as $n \to \infty$ (by considering $x = n^2 \in E$).
  Hence $\{f_n^2 \}$ does not converge uniformly (Theorem 7.9).
\end{enumerate}
$\Box$ \\

\emph{Proof ($f_n(x) = \frac{1}{x}$, $g_n(x) = \frac{1}{n}$).}
\begin{enumerate}
  \item[(1)]
  Let $E = (0,1)$.
  Let $\{f_n(x)\}$ on $E$ be $f_n(x) = \frac{1}{x}$
  and $\{g_n(x)\}$ on $E$ be $g_n(x) = \frac{1}{n}$.
  Clearly, $\{f_n(x)\}$ converges to $f(x) = \frac{1}{x}$ pointwise
  and $\{g_n(x)\}$ converges to $g(x) = 0$ pointwise.

  \item[(2)]
  \emph{Show that $\{f_n\}$ converges uniformly.}
  Given $\varepsilon > 0$.
  There exists an integer $N = 1$ such that
  \[
    |f_n(x) - f(x)| = 0 \leq \varepsilon
  \]
  whenever $n \geq N$ and $x \in E$.
  Hence $\{f_n\} \to f$ uniformly.

  \item[(3)]
  \emph{Show that $\{g_n\}$ converges uniformly.}
  Given $\varepsilon > 0$.
  There exists an integer $N \geq \frac{1}{\varepsilon}$ such that
  \[
    |g_n(x) - g(x)| = \frac{1}{n} \leq \frac{1}{N} \leq \varepsilon
  \]
  whenever $n \geq N$ and $x \in E$.
  Hence $\{g_n\} \to g$ uniformly.

  \item[(4)]
  \emph{Show that $\{f_n g_n\}$ does not converge uniformly.}
  Clearly, $\{f_n(x) g_n(x) \}$ converges to $f(x) g(x) = 0$ pointwise.
  Hence
  \[
    \sup_{x \in E} |f_n(x) g_n(x) - 0|
    = \sup_{x \in E} \abs{\frac{1}{nx}}
    \to \infty
  \]
  as $n \to \infty$ (by considering $x = \frac{1}{n^2} \in E$).
  Hence $\{f_n g_n \}$ does not converge uniformly (Theorem 7.9).
\end{enumerate}
$\Box$ \\

\emph{Proof (Exercise 9.2 in Tom M. Apostol, Mathematical Analysis, 2nd edition).}
\begin{enumerate}
  \item[(1)]
  Let $E = [\alpha,\beta] \subseteq \mathbb{R}$ be a bounded interval.
  Define two sequences $\{f_n\}$ and $\{g_n\}$ on $E$ as follows:
  \[
    f_n(x) = x \left( 1+\frac{1}{n} \right)
    \text{ if $x \in \mathbb{R}$, $n = 1,2,\cdots$},
  \]
  \begin{equation*}
  g_n(x) =
    \begin{cases}
      \frac{1}{n}   & \text{ if $x=0$ or if $x$ is irrational}, \\
      b+\frac{1}{n} & \text{ if $x$ is rational $\neq 0$, say $x=\frac{a}{b}$, $b>0$}.
    \end{cases}
  \end{equation*}
  Here we assume that $\mathrm{gcd}(a,b) = 1$.
  Clearly, $f(x) = x$ and
  \begin{equation*}
  g(x) =
    \begin{cases}
      0 & \text{ if $x=0$ or if $x$ is irrational}, \\
      b & \text{ if $x$ is rational $\neq 0$, say $x=\frac{a}{b}$, $b>0$}.
    \end{cases}
  \end{equation*}
  Let $M = \max\{|\alpha|,|\beta|\} \geq 0$.

  \item[(2)]
  \emph{Show that $\{f_n\}$ converges uniformly.}
  Given $\varepsilon > 0$.
  There exists an integer $N \geq \frac{M}{\varepsilon}$ such that
  \[
    |f_n(x) - f(x)| = \frac{|x|}{n} \leq \frac{M}{N} \leq \varepsilon
  \]
  whenever $n \geq N$ and $x \in E$.
  Hence $\{f_n\} \to f$ uniformly.

  \item[(3)]
  \emph{Show that $\{g_n\}$ converges uniformly.}
  Given $\varepsilon > 0$.
  There exists an integer $N \geq \frac{1}{\varepsilon}$ such that
  \[
    |g_n(x) - g(x)| = \frac{1}{n} \leq \frac{1}{N} \leq \varepsilon
  \]
  whenever $n \geq N$ and $x \in E$.
  Hence $\{g_n\} \to g$ uniformly.

  \item[(4)]
  \emph{Show that $\{f_n g_n\}$ does not converge uniformly.}
  \begin{enumerate}
    \item[(a)]
      Clearly, $\{f_n(x) g_n(x) \}$ converges to $f(x)g(x)$ pointwise
      where
      \begin{equation*}
      f(x) g(x) =
        \begin{cases}
          0 & \text{ if $x=0$ or if $x$ is irrational}, \\
          a & \text{ if $x=\frac{a}{b}$ is rational $\neq 0$, $b>0$}.
        \end{cases}
      \end{equation*}

    \item[(b)]
      Note that
      \begin{equation*}
      f_n(x) g_n(x) =
        \begin{cases}
          \frac{x}{n} \left( 1 + \frac{1}{n} \right)
            & \text{ if $x=0$ or if $x$ is irrational}, \\
          \left( a + \frac{x}{n} \right)\left( 1 + \frac{1}{n} \right)
            & \text{ if $x=\frac{a}{b}$ is rational $\neq 0$, $b>0$}.
        \end{cases}
      \end{equation*}
      Therefore,
      \begin{equation*}
      f_n(x) g_n(x) - f(x) g(x) =
        \begin{cases}
          \frac{x}{n} \left( 1 + \frac{1}{n} \right)
            & \text{ if $x=0$ or if $x$ is irrational}, \\
          \frac{x}{n} \left( 1 + b + \frac{1}{n} \right)
            & \text{ if $x=\frac{a}{b}$ is rational $\neq 0$, $b>0$}.
        \end{cases}
      \end{equation*}

    \item[(c)]
      Hence
      \begin{align*}
        \sup_{x \in E} |f_n(x) g_n(x) - f(x) g(x)|
        &\geq \sup_{x \in E \cap \mathbb{Q}} |f_n(x) g_n(x) - f(x) g(x)| \\
        &= \sup_{x \in E \cap \mathbb{Q}}
          \abs{a} \left( \frac{1}{n} + \frac{1}{bn} + \frac{1}{bn^2} \right) \\
        &\geq \sup_{x \in E \cap \mathbb{Q}}
          \abs{a} \left( \frac{1}{n} \right) \\
        &= \sup_{x \in E \cap \mathbb{Q}} \frac{\abs{a}}{n}.
      \end{align*}

    \item[(d)]
      \emph{Given any irrational number $\gamma \in E$,
      there exists a sequence
      \[
        \left\{ r_m = \frac{a_m}{b_m} \right\}
      \]
      of nonzero rational numbers in $E$ such that $\lim r_m = \gamma$.
      Show that $\{a_m\}$ is unbounded.}
      If it is true, we can find $x_n = r_{m_n} = \frac{a_{m_n}}{b_{m_n}}$
      such that $|a_{m_n}| \geq n^2$ and
      \[
        \sup_{x \in E} |f_n(x) g_n(x) - f(x) g(x)|
        \geq \sup_{x \in E \cap \mathbb{Q}} \frac{\abs{a}}{n}
        \geq \frac{n^2}{n}
        = n \to \infty
      \]
      as $n \to \infty$.

    \item[(e)]
      (Reductio ad absurdum)
      If $\{a_m\}$ were bounded, then there exists
      a \textbf{constant} subsequence of $\{a_{m_k}\}$
      such that $\lim a_{m_k} = a \in \mathbb{Z}$.
      Since $\lim_{m \to \infty} r_m = \gamma$, $\lim_{k \to \infty} r_{m_k} = \gamma$ or
      \[
        \lim_{k \to \infty} b_{m_k}
        = \lim_{k \to \infty} \frac{a_{m_k}}{r_{m_k}}
        = \frac{a}{\gamma}
      \]
      (it is well-defined since $r_{m_k}$ and $\gamma$ cannot be zero).
      Since all $b_{m_k}$ are positive integers,
      the limit $\lim b_{m_k} = b$ is a positive integer too,
      or $b = \frac{a}{\gamma} \in \mathbb{Z}^+$, or $\gamma = \frac{a}{b} \in \mathbb{Z}$,
      which is absurd.
  \end{enumerate}
  Therefore, $\{f_n g_n\}$ does not converge uniformly.
\end{enumerate}
$\Box$ \\\\



%%%%%%%%%%%%%%%%%%%%%%%%%%%%%%%%%%%%%%%%%%%%%%%%%%%%%%%%%%%%%%%%%%%%%%%%%%%%%%%%



\textbf{Exercise 7.4.}
\emph{Consider
\[
  f(x) = \sum_{n=1}^{\infty} \frac{1}{1+n^2x}.
\]
For what values of $x$ does the series converge absolutely?
On what intervals does it converge uniformly?
On what intervals does it fail to converge uniformly?
Is $f$ continuous whenever the series converges?
Is $f$ bounded?} \\

\emph{Proof.}
Clearly, $f(x)$ is defined on
$\mathbb{R} - \{ -1, -\frac{1}{4}, -\frac{1}{9}, \ldots \}$.
\begin{enumerate}
  \item[(1)]
\end{enumerate}

PLACEHOLDER \\\\



%%%%%%%%%%%%%%%%%%%%%%%%%%%%%%%%%%%%%%%%%%%%%%%%%%%%%%%%%%%%%%%%%%%%%%%%%%%%%%%%



\textbf{Exercise 7.5.}

PLACEHOLDER \\\\



%%%%%%%%%%%%%%%%%%%%%%%%%%%%%%%%%%%%%%%%%%%%%%%%%%%%%%%%%%%%%%%%%%%%%%%%%%%%%%%%



\textbf{Exercise 7.6.}
\emph{Prove that the series
\[
  \sum_{n=1}^{\infty} (-1)^n \frac{x^2+n}{n^2}
\]
converges uniformly in every bounded interval,
but does not converge absolutely for any value of $x$.} \\

\emph{Proof (Dirichlet's test).}
Given any bounded interval $E = [\alpha,\beta] \subseteq \mathbb{R}$.
Write $f_n(x) = (-1)^n$ on $E$ and $g_n(x) = \frac{x^2+n}{n^2}$ on $E$.
\begin{enumerate}
  \item[(1)]
  The partial sums $F_n(x)$ of $\sum f_n(x)$ form a uniformly bounded sequence.

  \item[(2)]
  $g_1(x) \geq g_2(x) \geq \cdots$ since
  \[
    g_{n+1}(x)
    = \frac{x^2}{(n+1)^2} + \frac{1}{n+1}
    < \frac{x^2}{n^2} + \frac{1}{n}
    = g_n(x).
  \]

  \item[(3)]
  Write $M = \max\{|\alpha|,|\beta|\}$.
  Since
  \[
    |g_n(x)|
    = \frac{x^2}{n^2} + \frac{1}{n}
    \leq \frac{M^2}{n^2} + \frac{1}{n} \to \infty
  \]
  as $n \to \infty$,
  $\lim_{n \to \infty} g_n(x) = 0$.
  By Dirichlet's test (Exercise 7.11),
  $\sum_{n=1}^{\infty} f_n(x) g_n(x) = \sum_{n=1}^{\infty} (-1)^n \frac{x^2+n}{n^2}$
  converges.

  \item[(4)]
  \begin{align*}
    \sum |f_n(x)|
    &= \sum \frac{x^2+n}{n^2} \\
    &\geq \sum \frac{n}{n^2} \\
    &= \sum \frac{1}{n} \to \log n + \gamma
  \end{align*}
  (Exercise 8.9).
  Hence $\sum (-1)^n \frac{x^2+n}{n^2}$ does not converge absolutely for any value of $x$
\end{enumerate}
$\Box$ \\\\



%%%%%%%%%%%%%%%%%%%%%%%%%%%%%%%%%%%%%%%%%%%%%%%%%%%%%%%%%%%%%%%%%%%%%%%%%%%%%%%%



\textbf{Exercise 7.7.}
\emph{For $n=1,2,3,\ldots$, $x$ real, put
\[
  f_n(x) = \frac{x}{1+nx^2}.
\]
Show that $\{f_n\}$ converges uniformly to a function $f$,
and that the equation
\[
  f'(x) = \lim_{n \to \infty} f_n'(x)
\]
is correct if $x \neq 0$, but false if $x = 0$.} \\

$f_n(x)$ is defined on $\mathbb{R}$. \\

\emph{Proof.}
\begin{enumerate}
  \item[(1)]
  Since
  \[
    \abs{ f_n(x) }
    = \abs{ \frac{x}{1+nx^2} }
    \leq \frac{|x|}{\sqrt{n}|x|}
    = \frac{1}{\sqrt{n}} \to \infty
  \]
  as $n \to \infty$, $f_n \to 0$ uniformly (Theorem 7.9).

  \item[(2)]
  Clearly, $f'(x) = 0$.
  Since
  \[
    f_n'(x) = \frac{1-nx^2}{(1+nx^2)^2},
  \]
  \begin{equation*}
  \lim_{n \to \infty} f_n'(x) =
    \begin{cases}
      1 & (x = 0), \\
      0 & (x \neq 0).
    \end{cases}
  \end{equation*}
  So that the equation
  \[
    f'(x) = \lim_{n \to \infty} f_n'(x)
  \]
  is correct if $x \neq 0$, but false if $x = 0$.
\end{enumerate}
$\Box$ \\

\emph{Note.}
$f_n'(x)$ does not converge uniformly by considering
\[
  \lim_{n \to \infty} f_n'\left(\frac{1}{n}\right)
  = \lim_{n \to \infty} \frac{1-\frac{1}{n}}{(1+\frac{1}{n})^2}
  = 1.
\]
\\\\



%%%%%%%%%%%%%%%%%%%%%%%%%%%%%%%%%%%%%%%%%%%%%%%%%%%%%%%%%%%%%%%%%%%%%%%%%%%%%%%%



\textbf{Exercise 7.8.}
\emph{If
  \begin{equation*}
  I(x) =
    \begin{cases}
      0 & (x \leq 0), \\
      1 & (x > 0),
    \end{cases}
  \end{equation*}
if $\{ x_n \}$ is a sequence of distinct points of $(a,b)$,
and if $\sum|c_n|$ converges,
prove that the series
\[
  f(x) = \sum_{n=1}^{\infty} c_n I(x-x_n)
  \:\:\:\:\:\:\:\:
  (a \leq x \leq b)
\]
converges uniformly,
and that $f$ is continuous for every $x \neq x_n$.} \\

\emph{Proof.}
\begin{enumerate}
\item[(1)]
Define $f_n(x) = c_n I(x-x_n)$ on $(a,b)$. So
\[
  |f_n(x)| = |c_n| |I(x-x_n)| \leq |c_n|
  \:\:\:\:\:\:\:\:
  (x \in (a,b), n = 1,2,3,\ldots).
\]
Since $\sum|c_n|$ converges, $f = \sum f_n$ converges uniformly (Theorem 7.10).

\item[(2)]
Given any $p \in (a,b)$ with $p \neq x_n$ for all $n=1,2,3,\ldots$.
So each $I(x-x_n)$ is continuous at $x=p$, and thus
each partial sum $\sum_{n=1}^{N} f_n(x)$ is continuous.

\item[(3)]
By Theorem 7.11
\begin{align*}
  \lim_{x \to p} f(x)
  &= \lim_{x \to p} \sum_{n=1}^{\infty} f_n(x) \\
  &= \lim_{N \to \infty} \left( \lim_{x \to p} \sum_{n=1}^{N} f_n(x) \right) \\
  &= \lim_{N \to \infty} \sum_{n=1}^{N} f_n(p) \\
  &= \sum_{n=1}^{\infty} f_n(p) \\
  &= f(p).
\end{align*}
$f(x)$ is continuous at $x=p$ too.
\end{enumerate}
$\Box$ \\\\



%%%%%%%%%%%%%%%%%%%%%%%%%%%%%%%%%%%%%%%%%%%%%%%%%%%%%%%%%%%%%%%%%%%%%%%%%%%%%%%%



\textbf{Exercise 7.9.}

PLACEHOLDER \\\\



%%%%%%%%%%%%%%%%%%%%%%%%%%%%%%%%%%%%%%%%%%%%%%%%%%%%%%%%%%%%%%%%%%%%%%%%%%%%%%%%



\textbf{Exercise 7.10.}

PLACEHOLDER \\\\



%%%%%%%%%%%%%%%%%%%%%%%%%%%%%%%%%%%%%%%%%%%%%%%%%%%%%%%%%%%%%%%%%%%%%%%%%%%%%%%%



\textbf{Exercise 7.11 (Dirichlet's test).}
\emph{Suppose $\{f_n\}$, $\{g_n\}$ are defined on $E$, and}
\begin{enumerate}
  \item[(a)]
  \emph{$\sum f_n(x)$ has uniformly bounded partial sums;}
  \item[(b)]
  \emph{$g_n(x) \to 0$ uniformly on $E$;}
  \item[(b)]
  \emph{$g_1(x) \geq g_2(x) \geq g_3(x) \geq \cdots$ for every $x \in E$.}
\end{enumerate}
\emph{Prove that $\sum f_n(x) g_n(x)$ converges uniformly on $E$.
(Hint: Compare with Theorem 3.42.)} \\



\emph{Theorem 3.42 (Dirichlet's test).}
Suppose
\begin{enumerate}
  \item[(a)]
  the partial sums $A_n$ of $\sum a_n$ form a bounded sequence;
  \item[(b)]
  $b_0 \geq b_1 \geq b_2 \geq \cdots$;
  \item[(c)]
  $\lim_{n \to \infty} b_n = 0$.
\end{enumerate}
Then $\sum a_n b_n$ converges. \\



\emph{Proof (Theorem 3.42).}
Let $F_n(x) = \sum_{k=1}^{n} f_k(x)$.
Choose $M$ such that $|F_n(x)| \leq M$ for all $n$, all $x \in E$.
Given $\varepsilon > 0$,
there is an integer $N$ such that $g_N(x) \leq \frac{\varepsilon}{2(M+1)}$ for all $x \in E$.
For $N \leq p \leq q$, we have
\begin{align*}
  &\abs{\sum_{n=p}^{q} f_n(x) g_n(x)} \\
  =& \abs{\sum_{n=p}^{q-1} F_n(x)(g_n(x)-g_{n+1}(x)) + F_q(x)g_q(x) - F_{p-1}(x)g_p(x)} \\
  \leq& M \abs{\sum_{n=p}^{q-1}(g_n(x)-g_{n+1}(x)) + g_q(x) + g_p(x)} \\
  =& 2M g_p(x) \\
  \leq& 2M g_N(x) \\
  \leq& \varepsilon
\end{align*}
for all $x \in E$.
Uniformly convergence now follows from the Cauchy criterion (Theorem 7.8).
Note that the first inequality in the above chain depends of course on the fact that
$g_n(x) - g_{n+1}(x) \geq 0$.
$\Box$ \\\\



%%%%%%%%%%%%%%%%%%%%%%%%%%%%%%%%%%%%%%%%%%%%%%%%%%%%%%%%%%%%%%%%%%%%%%%%%%%%%%%%



\textbf{Exercise 7.12.}
PLACEHOLDER



%%%%%%%%%%%%%%%%%%%%%%%%%%%%%%%%%%%%%%%%%%%%%%%%%%%%%%%%%%%%%%%%%%%%%%%%%%%%%%%%



\textbf{Exercise 7.13.}
PLACEHOLDER



%%%%%%%%%%%%%%%%%%%%%%%%%%%%%%%%%%%%%%%%%%%%%%%%%%%%%%%%%%%%%%%%%%%%%%%%%%%%%%%%



\textbf{Exercise 7.14.}
PLACEHOLDER



%%%%%%%%%%%%%%%%%%%%%%%%%%%%%%%%%%%%%%%%%%%%%%%%%%%%%%%%%%%%%%%%%%%%%%%%%%%%%%%%



\textbf{Exercise 7.15.}
PLACEHOLDER



%%%%%%%%%%%%%%%%%%%%%%%%%%%%%%%%%%%%%%%%%%%%%%%%%%%%%%%%%%%%%%%%%%%%%%%%%%%%%%%%



\textbf{Exercise 7.16.}
PLACEHOLDER



%%%%%%%%%%%%%%%%%%%%%%%%%%%%%%%%%%%%%%%%%%%%%%%%%%%%%%%%%%%%%%%%%%%%%%%%%%%%%%%%



\textbf{Exercise 7.17.}
PLACEHOLDER



%%%%%%%%%%%%%%%%%%%%%%%%%%%%%%%%%%%%%%%%%%%%%%%%%%%%%%%%%%%%%%%%%%%%%%%%%%%%%%%%



\textbf{Exercise 7.18.}
PLACEHOLDER



%%%%%%%%%%%%%%%%%%%%%%%%%%%%%%%%%%%%%%%%%%%%%%%%%%%%%%%%%%%%%%%%%%%%%%%%%%%%%%%%



\textbf{Exercise 7.19.}

PLACEHOLDER \\\\



%%%%%%%%%%%%%%%%%%%%%%%%%%%%%%%%%%%%%%%%%%%%%%%%%%%%%%%%%%%%%%%%%%%%%%%%%%%%%%%%



\textbf{Exercise 7.20.}
\emph{If $f$ is continuous on $[0,1]$ and if
\[
  \int_{0}^{1} f(x) x^n dx = 0
  \:\:\:\:\:\:\:\:
  (n=0,1,2,\ldots),
\]
prove that $f(x) = 0$ on $[0,1]$.
(Hint: The integral of the product of $f$ with any polynomial is zero.
Use the Weierstrass theorem to show that
$\int_{0}^{1} f^2(x) dx = 0$.)} \\

\emph{Proof.}
\begin{enumerate}
\item[(1)]
Since $\int_{0}^{1} f(x) x^n dx = 0$ for all $n = 0,1,2,\ldots$,
\[
  \int_{0}^{1} f(x) P(x) dx = 0 \text{ for all } P(x) \in \mathbb{R}[x].
\]

\item[(2)]
By Theorem 7.26 (Stone-Weierstrass Theorem),
there exists a sequence of $P_n(x) \in \mathbb{R}[x]$ such that
\[
  P_n(x) \to f(x)
\]
uniformly on $[0,1]$.
Since $f(x)$ is continuous on the compact set $[0,1]$, $f(x)$ is bounded on $[0,1]$.
Hence
\[
  f(x) P_n(x) \to f^2(x)
\]
uniformly on $[0,1]$.

\item[(3)]
Since each $f(x) P_n(x)$ is continuous,
$f(x) P_n(x) \in \mathscr{R}$ on $[0,1]$ (Theorem 6.8).
By Theorem 7.16,
\[
  \int_{0}^{1} f^2(x) dx
  = \lim_{n \to \infty} \int_{0}^{1} f(x) P_n(x) dx
  = \lim_{n \to \infty} 0
  = 0.
\]

\item[(4)]
Since $f^2(x)$ is continuous,
$f^2(x) = 0$ or $f(x) = 0$ by (3) and Exercise 6.2.
\end{enumerate}
$\Box$ \\\\



%%%%%%%%%%%%%%%%%%%%%%%%%%%%%%%%%%%%%%%%%%%%%%%%%%%%%%%%%%%%%%%%%%%%%%%%%%%%%%%%



\textbf{Exercise 7.21.}

PLACEHOLDER \\\\



%%%%%%%%%%%%%%%%%%%%%%%%%%%%%%%%%%%%%%%%%%%%%%%%%%%%%%%%%%%%%%%%%%%%%%%%%%%%%%%%



\textbf{Exercise 7.22.}
PLACEHOLDER



%%%%%%%%%%%%%%%%%%%%%%%%%%%%%%%%%%%%%%%%%%%%%%%%%%%%%%%%%%%%%%%%%%%%%%%%%%%%%%%%



\textbf{Exercise 7.23.}
PLACEHOLDER



%%%%%%%%%%%%%%%%%%%%%%%%%%%%%%%%%%%%%%%%%%%%%%%%%%%%%%%%%%%%%%%%%%%%%%%%%%%%%%%%



\textbf{Exercise 7.24.}
PLACEHOLDER



%%%%%%%%%%%%%%%%%%%%%%%%%%%%%%%%%%%%%%%%%%%%%%%%%%%%%%%%%%%%%%%%%%%%%%%%%%%%%%%%



\textbf{Exercise 7.25.}
PLACEHOLDER



%%%%%%%%%%%%%%%%%%%%%%%%%%%%%%%%%%%%%%%%%%%%%%%%%%%%%%%%%%%%%%%%%%%%%%%%%%%%%%%%



\textbf{Exercise 7.26.}
PLACEHOLDER



%%%%%%%%%%%%%%%%%%%%%%%%%%%%%%%%%%%%%%%%%%%%%%%%%%%%%%%%%%%%%%%%%%%%%%%%%%%%%%%%
%%%%%%%%%%%%%%%%%%%%%%%%%%%%%%%%%%%%%%%%%%%%%%%%%%%%%%%%%%%%%%%%%%%%%%%%%%%%%%%%



\end{document}