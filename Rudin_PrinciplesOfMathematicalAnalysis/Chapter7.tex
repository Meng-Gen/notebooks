\documentclass{article}
\usepackage{amsfonts}
\usepackage{amsmath}
\usepackage{amssymb}
\usepackage{hyperref}
\usepackage[none]{hyphenat}
\usepackage{mathrsfs}
\usepackage{physics}
\parindent=0pt

\def\upint{\mathchoice%
    {\mkern13mu\overline{\vphantom{\intop}\mkern7mu}\mkern-20mu}%
    {\mkern7mu\overline{\vphantom{\intop}\mkern7mu}\mkern-14mu}%
    {\mkern7mu\overline{\vphantom{\intop}\mkern7mu}\mkern-14mu}%
    {\mkern7mu\overline{\vphantom{\intop}\mkern7mu}\mkern-14mu}%
  \int}
\def\lowint{\mkern3mu\underline{\vphantom{\intop}\mkern7mu}\mkern-10mu\int}

\begin{document}



\textbf{\Large Chapter 7: Sequences and Series of Functions} \\\\



\emph{Author: Meng-Gen Tsai} \\
\emph{Email: plover@gmail.com} \\\\



% http://math.ucsd.edu/~lni/math140/HW140B_5_solutions.pdf
% https://minds.wisconsin.edu/bitstream/handle/1793/67009/rudin%20ch%207.pdf
% http://www.math.ust.hk/~majhu/Math204/Homework204_Set03.pdf
% https://docs.google.com/viewer?a=v&pid=sites&srcid=ZGVmYXVsdGRvbWFpbnxtYXRoc29sdXRpb25ndWlkZXN8Z3g6MzIxZGJmOTAzNzQzZTAyYw
% https://www.csie.ntu.edu.tw/~b89089/old/math/pma/pma7.html



%%%%%%%%%%%%%%%%%%%%%%%%%%%%%%%%%%%%%%%%%%%%%%%%%%%%%%%%%%%%%%%%%%%%%%%%%%%%%%%%
%%%%%%%%%%%%%%%%%%%%%%%%%%%%%%%%%%%%%%%%%%%%%%%%%%%%%%%%%%%%%%%%%%%%%%%%%%%%%%%%



\textbf{Exercise 7.1.}
\emph{Prove that every uniformly convergent sequence of bounded functions
is uniformly bounded.} \\

\emph{Proof (Cauchy criterion).}
Let $\{f_n\}$ be a uniformly convergent sequence of bounded functions.
\begin{enumerate}
\item[(1)]
Since $f_n$ is bounded, there exists a $M_n$ such that $|f_n(x)| \leq M_n$.

\item[(2)]
Since $\{f_n\}$ converges uniformly, given $1 > 0$ there exists an integer $N$
such that
\[
  |f_n(x) - f_m(x)| \leq 1 \text{ whenever } n, m \geq N
\]
(Theorem 7.8 (Cauchy criterion for uniformly convergence)).
Especially,
\[
  |f_n(x)| \leq |f_n(x) - f_N(x)| + |f_N(x)| \leq 1 + M_N \text{ whenever } n \geq N.
\]

\item[(3)]
Thus, $\{f_n\}$ is uniformly bounded by $M = \max\{ M_1, \ldots, M_{N-1}, M_{N}+1 \}$.
\end{enumerate}
$\Box$ \\\\



%%%%%%%%%%%%%%%%%%%%%%%%%%%%%%%%%%%%%%%%%%%%%%%%%%%%%%%%%%%%%%%%%%%%%%%%%%%%%%%%



\textbf{Exercise 7.2.}
\emph{If $\{f_n\}$ and $\{g_n\}$ converge uniformly on a set $E$,
prove that $\{f_n + g_n\}$ converge uniformly on $E$.
If, in addition, $\{f_n\}$ and $\{g_n\}$ are sequences of bounded functions,
prove that $\{f_n g_n\}$ converges uniformly on $E$.} \\

\emph{Proof.}
Let $f_n \to f$ uniformly and $g_n \to g$ uniformly.
\begin{enumerate}
  \item[(1)]
  \emph{Show that $\{f_n + g_n\}$ converges uniformly.}
    Given $\varepsilon > 0$.
    Since $f_n \to f$ uniformly and $g_n \to g$ uniformly,
    there exist two integers $N_1$ and $N_2$ such that
    \begin{align*}
      |f_n(x) - f(x)| \leq \frac{\varepsilon}{2}
      &\text{ whenever }
      n \geq N_1, x \in E \\
      |g_n(x) - g(x)| \leq \frac{\varepsilon}{2}
      &\text{ whenever }
      n \geq N_2, x \in E.
    \end{align*}
    Take $N = \max\{N_1,N_2\}$, we have
    \begin{align*}
      &|(f_n(x)+ g_n(x)) - (f(x) + g(x))| \\
      =& |(f_n(x) - f(x)) + (g_n(x) - g(x))| \\
      \leq& |f_n(x) - f(x)| + |g_n(x) - g(x)| \\
      \leq& \frac{\varepsilon}{2} + \frac{\varepsilon}{2} \\
      =& \varepsilon
    \end{align*}
    whenever $n \geq N$, $x \in E$.
    Hence $f_n + g_n \to f+g$ uniformly on $E$.

  \item[(2)]
  \emph{Show that $\{f_n g_n\}$ converges uniformly
  if, in addition, $\{f_n\}$ and $\{g_n\}$ are sequences of bounded functions.}
  Given $\varepsilon > 0$.
  \begin{enumerate}
    \item[(a)]
    By Exercise 7.1, both $\{f_n\}$ and $\{g_n\}$ are uniformly bounded.
    So there exist $M_1$ and $M_2$
    such that
    \[
      |f_n(x)| \leq M_1 \text{ and } |g_n(x)| \leq M_2
    \]
    for all $n$ and $x \in E$.
    Also, $|f(x)| \leq M_1 + 1$ and $|g(x)| \leq M_2 + 1$.

    \item[(b)]
    Since $f_n \to f$ uniformly and $g_n \to g$ uniformly,
    there exist two integers $N_1$ and $N_2$ such that
    \begin{align*}
      |f_n(x) - f(x)| \leq \frac{\varepsilon}{2(M_2 + 1)}
      &\text{ whenever }
      n \geq N_1, x \in E \\
      |g_n(x) - g(x)| \leq \frac{\varepsilon}{2(M_1 + 1)}
      &\text{ whenever }
      n \geq N_2, x \in E.
    \end{align*}
    (Note that each denominator of $\frac{\varepsilon}{2(M_j + 1)}$ $(j=1,2)$
    is well-defined and positive!)
    Take $N = \max\{N_1,N_2\}$, we have
    \begin{align*}
      &|f_n(x)g_n(x) - f(x)g(x)| \\
      =& |[f_n(x) - f(x)]g_n(x) + f(x)[g_n(x) - g(x)]| \\
      \leq& |f_n(x) - f(x)||g_n(x)| + |f(x)||g_n(x) - g(x)| \\
      \leq& \frac{\varepsilon}{2(M_2 + 1)} \cdot M_2
        + (M_1 + 1) \cdot \frac{\varepsilon}{2(M_1 + 1)} \\
      \leq& \varepsilon
    \end{align*}
    whenever $n \geq N$, $x \in E$.
    Hence $f_n g_n \to fg$ uniformly on $E$.
  \end{enumerate}
\end{enumerate}
$\Box$ \\

\emph{Proof (Cauchy criterion).}
\begin{enumerate}
  \item[(1)]
  \emph{Show that $\{f_n + g_n\}$ converges uniformly.}
    Given $\varepsilon > 0$.
    Since $\{f_n\}$ and $\{g_n\}$ converge uniformly,
    there exist two integers $N_1$ and $N_2$ such that
    \begin{align*}
      |f_n(x) - f_m(x)| \leq \frac{\varepsilon}{2}
      &\text{ whenever }
      n,m \geq N_1, x \in E \\
      |g_n(x) - g_m(x)| \leq \frac{\varepsilon}{2}
      &\text{ whenever }
      n,m \geq N_2, x \in E.
    \end{align*}
    Take $N = \max\{N_1,N_2\}$, we have
    \begin{align*}
      &|(f_n(x)+ g_n(x)) - (f_m(x) + g_m(x))| \\
      =& |(f_n(x) - f_n(x)) + (g_n(x) - g_m(x))| \\
      \leq& |f_n(x) - f_n(x)| + |g_n(x) - g_m(x)| \\
      \leq& \frac{\varepsilon}{2} + \frac{\varepsilon}{2} \\
      =& \varepsilon
    \end{align*}
    whenever $n,m \geq N$, $x \in E$.
    Hence $\{f_n + g_n\}$ converges uniformly on $E$.

  \item[(2)]
  \emph{Show that $\{f_n g_n\}$ converges uniformly
  if, in addition, $\{f_n\}$ and $\{g_n\}$ are sequences of bounded functions.}
  Given $\varepsilon > 0$.
  \begin{enumerate}
    \item[(a)]
    By Exercise 7.1, both $\{f_n\}$ and $\{g_n\}$ are uniformly bounded.
    So there exist $M_1$ and $M_2$
    such that
    \[
      |f_n(x)| \leq M_1 \text{ and } |g_n(x)| \leq M_2
    \]
    for all $n$ and $x \in E$.
    Also, $|f(x)| \leq M_1 + 1$ and $|g(x)| \leq M_2 + 1$.

    \item[(b)]
    Since $\{f_n\} \to f$ uniformly and $\{g_n\} \to g$ uniformly,
    there exist two integers $N_1$ and $N_2$ such that
    \begin{align*}
      |f_n(x) - f_m(x)| \leq \frac{\varepsilon}{2(M_2 + 1)}
      &\text{ whenever }
      n,m \geq N_1, x \in E \\
      |g_n(x) - g_m(x)| \leq \frac{\varepsilon}{2(M_1 + 1)}
      &\text{ whenever }
      n,m \geq N_2, x \in E.
    \end{align*}
    Take $N = \max\{N_1,N_2\}$, we have
    \begin{align*}
      &|f_n(x)g_n(x) - f_m(x)g_m(x)| \\
      =& |[f_n(x) - f_m(x)]g_n(x) + f_m(x)[g_n(x) - g_m(x)]| \\
      \leq& |f_n(x) - f_m(x)||g_n(x)| + |f_m(x)||g_n(x) - g_m(x)| \\
      \leq& \frac{\varepsilon}{2(M_2 + 1)} \cdot M_2
        + M_1 \cdot \frac{\varepsilon}{2(M_1 + 1)} \\
      \leq& \varepsilon
    \end{align*}
    whenever $n \geq N$, $x \in E$.
    Hence $\{f_n g_n\}$ converges uniformly on $E$.
  \end{enumerate}
\end{enumerate}
$\Box$ \\

\emph{Note.} It proved that $f_n g_n \to fg$ in Theorem 7.29.
\\\\



%%%%%%%%%%%%%%%%%%%%%%%%%%%%%%%%%%%%%%%%%%%%%%%%%%%%%%%%%%%%%%%%%%%%%%%%%%%%%%%%



\textbf{Exercise 7.3.}
\emph{Construct sequences $\{f_n\}$, $\{g_n\}$ which converge uniformly on some set $E$,
but such that $\{f_n g_n\}$ does not converge uniformly on $E$
(of course, $\{f_n g_n\}$ must converge on $E$).} \\

We provides some examples here. \\

\emph{Proof ($f_n(x) = x + \frac{1}{n}$).}
\begin{enumerate}
  \item[(1)]
  Define $\{f_n(x)\}$ on $E = \mathbb{R}$ by $f_n(x) = x + \frac{1}{n}$ and $f(x) = x$.
  Clearly, $\{f_n(x)\}$ converges to $f(x)$ pointwise.

  \item[(2)]
  \emph{Show that $\{f_n\}$ converges uniformly.}
  Given $\varepsilon > 0$.
  There exists an integer $N \geq \frac{1}{\varepsilon}$ such that
  \[
    |f_n(x) - f(x)| = \frac{1}{n} \leq \frac{1}{N} \leq \varepsilon
  \]
  whenever $n \geq N$ and $x \in E$.
  Hence $\{f_n\} \to f$ uniformly.

  \item[(3)]
  \emph{Show that $\{f_n^2\}$ does not converge uniformly.}
  Clearly, $\{f_n(x)^2\}$ converges to $f(x)^2$ pointwise.
  Hence
  \[
    \sup_{x \in E} |f_n(x)^2 - f(x)^2|
    = \sup_{x \in E} \abs{\frac{2x}{n} + \frac{1}{n^2}}
    \to \infty
  \]
  as $n \to \infty$ (by considering $x = n^2 \in E$).
  Hence $\{f_n^2 \}$ does not converge uniformly (Theorem 7.9).
\end{enumerate}
$\Box$ \\

\emph{Proof ($f_n(x) = \frac{1}{x}$, $g_n(x) = \frac{1}{n}$).}
\begin{enumerate}
  \item[(1)]
  Let $E = (0,1)$.
  Let $\{f_n(x)\}$ on $E$ be $f_n(x) = \frac{1}{x}$
  and $\{g_n(x)\}$ on $E$ be $g_n(x) = \frac{1}{n}$.
  Clearly, $\{f_n(x)\}$ converges to $f(x) = \frac{1}{x}$ pointwise
  and $\{g_n(x)\}$ converges to $g(x) = 0$ pointwise.

  \item[(2)]
  \emph{Show that $\{f_n\}$ converges uniformly.}
  Given $\varepsilon > 0$.
  There exists an integer $N = 1$ such that
  \[
    |f_n(x) - f(x)| = 0 \leq \varepsilon
  \]
  whenever $n \geq N$ and $x \in E$.
  Hence $\{f_n\} \to f$ uniformly.

  \item[(3)]
  \emph{Show that $\{g_n\}$ converges uniformly.}
  Given $\varepsilon > 0$.
  There exists an integer $N \geq \frac{1}{\varepsilon}$ such that
  \[
    |g_n(x) - g(x)| = \frac{1}{n} \leq \frac{1}{N} \leq \varepsilon
  \]
  whenever $n \geq N$ and $x \in E$.
  Hence $\{g_n\} \to g$ uniformly.

  \item[(4)]
  \emph{Show that $\{f_n g_n\}$ does not converge uniformly.}
  Clearly, $\{f_n(x) g_n(x) \}$ converges to $f(x) g(x) = 0$ pointwise.
  Hence
  \[
    \sup_{x \in E} |f_n(x) g_n(x) - 0|
    = \sup_{x \in E} \abs{\frac{1}{nx}}
    \to \infty
  \]
  as $n \to \infty$ (by considering $x = \frac{1}{n^2} \in E$).
  Hence $\{f_n g_n \}$ does not converge uniformly (Theorem 7.9).
\end{enumerate}
$\Box$ \\

\emph{Proof (Exercise 9.2 in Tom M. Apostol, Mathematical Analysis, 2nd edition).}
\begin{enumerate}
  \item[(1)]
  Let $E = [\alpha,\beta] \subseteq \mathbb{R}$ be a bounded interval.
  Define two sequences $\{f_n\}$ and $\{g_n\}$ on $E$ as follows:
  \[
    f_n(x) = x \left( 1+\frac{1}{n} \right)
    \text{ if $x \in \mathbb{R}$, $n = 1,2,\cdots$},
  \]
  \begin{equation*}
  g_n(x) =
    \begin{cases}
      \frac{1}{n}   & \text{ if $x=0$ or if $x$ is irrational}, \\
      b+\frac{1}{n} & \text{ if $x$ is rational $\neq 0$, say $x=\frac{a}{b}$, $b>0$}.
    \end{cases}
  \end{equation*}
  Here we assume that $\mathrm{gcd}(a,b) = 1$.
  Clearly, $f(x) = x$ and
  \begin{equation*}
  g(x) =
    \begin{cases}
      0 & \text{ if $x=0$ or if $x$ is irrational}, \\
      b & \text{ if $x$ is rational $\neq 0$, say $x=\frac{a}{b}$, $b>0$}.
    \end{cases}
  \end{equation*}
  Let $M = \max\{|\alpha|,|\beta|\} \geq 0$.

  \item[(2)]
  \emph{Show that $\{f_n\}$ converges uniformly.}
  Given $\varepsilon > 0$.
  There exists an integer $N \geq \frac{M}{\varepsilon}$ such that
  \[
    |f_n(x) - f(x)| = \frac{|x|}{n} \leq \frac{M}{N} \leq \varepsilon
  \]
  whenever $n \geq N$ and $x \in E$.
  Hence $\{f_n\} \to f$ uniformly.

  \item[(3)]
  \emph{Show that $\{g_n\}$ converges uniformly.}
  Given $\varepsilon > 0$.
  There exists an integer $N \geq \frac{1}{\varepsilon}$ such that
  \[
    |g_n(x) - g(x)| = \frac{1}{n} \leq \frac{1}{N} \leq \varepsilon
  \]
  whenever $n \geq N$ and $x \in E$.
  Hence $\{g_n\} \to g$ uniformly.

  \item[(4)]
  \emph{Show that $\{f_n g_n\}$ does not converge uniformly.}
  \begin{enumerate}
    \item[(a)]
      Clearly, $\{f_n(x) g_n(x) \}$ converges to $f(x)g(x)$ pointwise
      where
      \begin{equation*}
      f(x) g(x) =
        \begin{cases}
          0 & \text{ if $x=0$ or if $x$ is irrational}, \\
          a & \text{ if $x=\frac{a}{b}$ is rational $\neq 0$, $b>0$}.
        \end{cases}
      \end{equation*}

    \item[(b)]
      Note that
      \begin{equation*}
      f_n(x) g_n(x) =
        \begin{cases}
          \frac{x}{n} \left( 1 + \frac{1}{n} \right)
            & \text{ if $x=0$ or if $x$ is irrational}, \\
          \left( a + \frac{x}{n} \right)\left( 1 + \frac{1}{n} \right)
            & \text{ if $x=\frac{a}{b}$ is rational $\neq 0$, $b>0$}.
        \end{cases}
      \end{equation*}
      Therefore,
      \begin{equation*}
      f_n(x) g_n(x) - f(x) g(x) =
        \begin{cases}
          \frac{x}{n} \left( 1 + \frac{1}{n} \right)
            & \text{ if $x=0$ or if $x$ is irrational}, \\
          \frac{x}{n} \left( 1 + b + \frac{1}{n} \right)
            & \text{ if $x=\frac{a}{b}$ is rational $\neq 0$, $b>0$}.
        \end{cases}
      \end{equation*}

    \item[(c)]
      Hence
      \begin{align*}
        \sup_{x \in E} |f_n(x) g_n(x) - f(x) g(x)|
        &\geq \sup_{x \in E \cap \mathbb{Q}} |f_n(x) g_n(x) - f(x) g(x)| \\
        &= \sup_{x \in E \cap \mathbb{Q}}
          \abs{a} \left( \frac{1}{n} + \frac{1}{bn} + \frac{1}{bn^2} \right) \\
        &\geq \sup_{x \in E \cap \mathbb{Q}}
          \abs{a} \left( \frac{1}{n} \right) \\
        &= \sup_{x \in E \cap \mathbb{Q}} \frac{\abs{a}}{n}.
      \end{align*}

    \item[(d)]
      \emph{Given any irrational number $\gamma \in E$,
      there exists a sequence
      \[
        \left\{ r_m = \frac{a_m}{b_m} \right\}
      \]
      of nonzero rational numbers in $E$ such that $\lim r_m = \gamma$.
      Show that $\{a_m\}$ is unbounded.}
      If it is true, we can find $x_n = r_{m_n} = \frac{a_{m_n}}{b_{m_n}}$
      such that $|a_{m_n}| \geq n^2$ and
      \[
        \sup_{x \in E} |f_n(x) g_n(x) - f(x) g(x)|
        \geq \sup_{x \in E \cap \mathbb{Q}} \frac{\abs{a}}{n}
        \geq \frac{n^2}{n}
        = n \to \infty
      \]
      as $n \to \infty$.

    \item[(e)]
      (Reductio ad absurdum)
      If $\{a_m\}$ were bounded, then there exists
      a \textbf{constant} subsequence of $\{a_{m_k}\}$
      such that $\lim a_{m_k} = a \in \mathbb{Z}$.
      Since $\lim_{m \to \infty} r_m = \gamma$, $\lim_{k \to \infty} r_{m_k} = \gamma$ or
      \[
        \lim_{k \to \infty} b_{m_k}
        = \lim_{k \to \infty} \frac{a_{m_k}}{r_{m_k}}
        = \frac{a}{\gamma}
      \]
      (it is well-defined since $r_{m_k}$ and $\gamma$ cannot be zero).
      Since all $b_{m_k}$ are positive integers,
      the limit $\lim b_{m_k} = b$ is a positive integer too,
      or $b = \frac{a}{\gamma} \in \mathbb{Z}^+$, or $\gamma = \frac{a}{b} \in \mathbb{Z}$,
      which is absurd.
  \end{enumerate}
  Therefore, $\{f_n g_n\}$ does not converge uniformly.
\end{enumerate}
$\Box$ \\\\



%%%%%%%%%%%%%%%%%%%%%%%%%%%%%%%%%%%%%%%%%%%%%%%%%%%%%%%%%%%%%%%%%%%%%%%%%%%%%%%%



\textbf{Exercise 7.4.}
\emph{Consider
\[
  f(x) = \sum_{n=1}^{\infty} \frac{1}{1+n^2x}.
\]
For what values of $x$ does the series converge absolutely?
On what intervals does it converge uniformly?
On what intervals does it fail to converge uniformly?
Is $f$ continuous whenever the series converges?
Is $f$ bounded?} \\

\emph{Proof.}
Clearly, $f(x)$ is defined on
$\mathbb{R} - \{ -1, -\frac{1}{4}, -\frac{1}{9}, \ldots \}$.
\begin{enumerate}
  \item[(1)]
\end{enumerate}

PLACEHOLDER \\\\



%%%%%%%%%%%%%%%%%%%%%%%%%%%%%%%%%%%%%%%%%%%%%%%%%%%%%%%%%%%%%%%%%%%%%%%%%%%%%%%%



\textbf{Exercise 7.5.}
\emph{Let
\begin{equation*}
  f_n(x) =
    \begin{cases}
      0                    & (x < \frac{1}{n+1}), \\
      \sin^2 \frac{\pi}{x} & (\frac{1}{n+1} \leq x \leq \frac{1}{n}), \\
      0                    & (\frac{1}{n} < x).
    \end{cases}
\end{equation*}
Show that $\{f_n\}$ converges to a continuous function, but not uniformly.
Use the series $\sum f_n$ to show that absolute convergence, even for all $x$,
does not imply uniform convergence.} \\

\emph{Proof.}
\begin{enumerate}
\item[(1)]
\emph{Show that $\lim_{n \to \infty} f_n(x) = 0$.
Hence $\{f_n\}$ converges to a continuous function $0$ pointwise.}
Clearly, $f_n(x) = 0$ for all $x \not\in (0,1)$.
Next, for any fixed $x \in (0,1)$, there exists an integer $N > \frac{1}{x}$
such that
\[
  x > \frac{1}{N} \geq \frac{1}{n}
\]
whenever $n \geq N$.
Hence $f_n(x) = 0$ whenever $n \geq N$.

\item[(2)]
\emph{Show that $f_n \to f = 0$ not uniformly.}
Let
\[
  x_n = \frac{1}{n+\frac{1}{2}} \to 0
\]
for all $n=1,2,3,\ldots$.
Thus, $f_m(x_n) = \delta_{mn}$, where $\delta_{mn}$ is Kronecker delta.
  \begin{enumerate}
  \item[(a)]
  \emph{(Definition 7.7.)}
  (Reductio ad absurdum)
  If $\{f_n\}$ were convergent uniformly, then
  given $\varepsilon = \frac{1}{64} > 0$,
  there exists an integer $N$ such that $n \geq N$ implies
  \[
    |f_n(x) - f(x)| \leq \frac{1}{64}
  \]
  for all real $x$.
  However,
  \[
    |f_N(x_N) - f(x_N)| =  1 > \frac{1}{64},
  \]
  which is absurd.

  \item[(b)]
  \emph{(Theorem 7.8)}
  (Reductio ad absurdum)
  If $\{f_n\}$ were convergent uniformly, then
  given $\varepsilon = \frac{1}{64} > 0$,
  there exists an integer $N$ such that $n,m \geq N$ implies
  \[
    |f_n(x) - f_m(x)| \leq \frac{1}{64}
  \]
  for all real $x$.
  However,
  \[
    |f_N(x_{N}) - f_{N+1}(x_{N})| =  1 > \frac{1}{64},
  \]
  which is absurd.

  \item[(c)]
  \emph{(Theorem 7.9)}
  Since
  \[
    M_n
    = \sup_{x \in \mathbb{R}}|f_n(x) - f(x)|
    \geq |f_n(x_n) - f(x_n)| = 1,
  \]
  $f_n \to f$ not uniformly.

  \item[(d)]
  \emph{(Exercise 7.9.)}
  Since
  each $f_n$ is continuous and
  \[
    \lim_{n \to \infty} f_n(x_n) = \lim_{n \to \infty} 1 = 1 \neq 0 = f(0),
  \]
  $f_n \to f = 0$ not uniformly.
  \end{enumerate}

\item[(3)]
\emph{Show that $\sum f_n$ converges absolutely.}
Write $F_n = \sum_{k=1}^{n} f_k$ and $F = \sum f_n$.
Clearly,
\begin{equation*}
  F(x) =
    \begin{cases}
      0                    & (x \leq 0), \\
      \sin^2 \frac{\pi}{x} & (0 < x \leq 1), \\
      0                    & (x \geq 1).
    \end{cases}
\end{equation*}
Note that $f_n \geq 0$ for each $n$.
Hence $\sum f_n$ converges absolutely.

\item[(4)]
\emph{Show that $\sum f_n$ does not converge uniformly.}
Similar to (2).
Let
\[
  x_n = \frac{1}{n+\frac{1}{2}} \to 0
\]
for all $n=1,2,3,\ldots$.
Thus
\begin{equation*}
  F_m(x_n) =
    \begin{cases}
      1 & (m \geq n), \\
      0 & (m < n).
    \end{cases}
\end{equation*}

  \begin{enumerate}
  \item[(a)]
  \emph{(Definition 7.7.)}
  (Reductio ad absurdum)
  If $\{F_n\}$ were convergent uniformly, then
  given $\varepsilon = \frac{1}{64} > 0$,
  there exists an integer $N$ such that $n \geq N$ implies
  \[
    |F_n(x) - F(x)| \leq \frac{1}{64}
  \]
  for all real $x$.
  However,
  \[
    |F_N(x_{N+1}) - F(x_{N+1})| =  1 > \frac{1}{64},
  \]
  which is absurd.

  \item[(b)]
  \emph{(Theorem 7.8)}
  (Reductio ad absurdum)
  If $\{F_n\}$ were convergent uniformly, then
  given $\varepsilon = \frac{1}{64} > 0$,
  there exists an integer $N$ such that $n,m \geq N$ implies
  \[
    |F_n(x) - F_m(x)| \leq \frac{1}{64}
  \]
  for all real $x$.
  However,
  \[
    |F_N(x_{N+1}) - F_{N+1}(x_{N+1})| =  1 > \frac{1}{64},
  \]which is absurd.

  \item[(c)]
  \emph{(Theorem 7.9)}
  Since
  \[
    M_n
    = \sup_{x \in \mathbb{R}}|F_n(x) - F(x)|
    \geq |F_n(x_{n+1}) - F(x_{n+1})| = 1,
  \]
  $F_n \to F$ not uniformly.

  \item[(d)]
  \emph{(Exercise 7.9.)}
  Since
  each $F_n$ is continuous and
  \[
    \lim_{n \to \infty} F_n(x_{n+1}) = \lim_{n \to \infty} 0 \neq 1 = F(x_{n+1}),
  \]
  $F_n \to F$ not uniformly.

  \item[(e)]
  \emph{(Theorem 7.12.)}
  (Reductio ad absurdum)
  If $\{F_n\}$ were converging to $F$ uniformly, then
  $F$ were continuous since each $F_n$ is continuous by Theorem 7.12.
  However, $F$ is not continuous at $x = 0$.
  \end{enumerate}
\end{enumerate}
$\Box$ \\\\



%%%%%%%%%%%%%%%%%%%%%%%%%%%%%%%%%%%%%%%%%%%%%%%%%%%%%%%%%%%%%%%%%%%%%%%%%%%%%%%%



\textbf{Exercise 7.6.}
\emph{Prove that the series
\[
  \sum_{n=1}^{\infty} (-1)^n \frac{x^2+n}{n^2}
\]
converges uniformly in every bounded interval,
but does not converge absolutely for any value of $x$.} \\

\emph{Proof (Dirichlet's test).}
Given any bounded interval $E = [\alpha,\beta] \subseteq \mathbb{R}$.
Write $f_n(x) = (-1)^n$ on $E$ and $g_n(x) = \frac{x^2+n}{n^2}$ on $E$.
\begin{enumerate}
  \item[(1)]
  The partial sums $F_n(x)$ of $\sum f_n(x)$ form a uniformly bounded sequence.

  \item[(2)]
  $g_1(x) \geq g_2(x) \geq \cdots$ since
  \[
    g_{n+1}(x)
    = \frac{x^2}{(n+1)^2} + \frac{1}{n+1}
    < \frac{x^2}{n^2} + \frac{1}{n}
    = g_n(x).
  \]

  \item[(3)]
  Write $M = \max\{|\alpha|,|\beta|\}$.
  Since
  \[
    |g_n(x)|
    = \frac{x^2}{n^2} + \frac{1}{n}
    \leq \frac{M^2}{n^2} + \frac{1}{n} \to \infty
  \]
  as $n \to \infty$,
  $\lim_{n \to \infty} g_n(x) = 0$.
  By Dirichlet's test (Exercise 7.11),
  $\sum_{n=1}^{\infty} f_n(x) g_n(x) = \sum_{n=1}^{\infty} (-1)^n \frac{x^2+n}{n^2}$
  converges.

  \item[(4)]
  \begin{align*}
    \sum |f_n(x)|
    &= \sum \frac{x^2+n}{n^2} \\
    &\geq \sum \frac{n}{n^2} \\
    &= \sum \frac{1}{n} \to \log n + \gamma
  \end{align*}
  (Exercise 8.9).
  Hence $\sum (-1)^n \frac{x^2+n}{n^2}$ does not converge absolutely for any value of $x$.
\end{enumerate}
$\Box$ \\\\



%%%%%%%%%%%%%%%%%%%%%%%%%%%%%%%%%%%%%%%%%%%%%%%%%%%%%%%%%%%%%%%%%%%%%%%%%%%%%%%%



\textbf{Exercise 7.7.}
\emph{For $n=1,2,3,\ldots$, $x$ real, put
\[
  f_n(x) = \frac{x}{1+nx^2}.
\]
Show that $\{f_n\}$ converges uniformly to a function $f$,
and that the equation
\[
  f'(x) = \lim_{n \to \infty} f_n'(x)
\]
is correct if $x \neq 0$, but false if $x = 0$.} \\

$f_n(x)$ is defined on $\mathbb{R}$. \\

\emph{Proof.}
\begin{enumerate}
  \item[(1)]
  Since
  \[
    \abs{ f_n(x) }
    = \abs{ \frac{x}{1+nx^2} }
    \leq \frac{|x|}{\sqrt{n}|x|}
    = \frac{1}{\sqrt{n}} \to \infty
  \]
  as $n \to \infty$, $f_n \to 0$ uniformly (Theorem 7.9).

  \item[(2)]
  Clearly, $f'(x) = 0$.
  Since
  \[
    f_n'(x) = \frac{1-nx^2}{(1+nx^2)^2},
  \]
  \begin{equation*}
  \lim_{n \to \infty} f_n'(x) =
    \begin{cases}
      1 & (x = 0), \\
      0 & (x \neq 0).
    \end{cases}
  \end{equation*}
  So that the equation
  \[
    f'(x) = \lim_{n \to \infty} f_n'(x)
  \]
  is correct if $x \neq 0$, but false if $x = 0$.
\end{enumerate}
$\Box$ \\

\emph{Note.}
$f_n'(x)$ does not converge uniformly by considering
\[
  \lim_{n \to \infty} f_n'\left(\frac{1}{n}\right)
  = \lim_{n \to \infty} \frac{1-\frac{1}{n}}{(1+\frac{1}{n})^2}
  = 1.
\]
\\\\



%%%%%%%%%%%%%%%%%%%%%%%%%%%%%%%%%%%%%%%%%%%%%%%%%%%%%%%%%%%%%%%%%%%%%%%%%%%%%%%%



\textbf{Exercise 7.8.}
\emph{If
  \begin{equation*}
  I(x) =
    \begin{cases}
      0 & (x \leq 0), \\
      1 & (x > 0),
    \end{cases}
  \end{equation*}
if $\{ x_n \}$ is a sequence of distinct points of $(a,b)$,
and if $\sum|c_n|$ converges,
prove that the series
\[
  f(x) = \sum_{n=1}^{\infty} c_n I(x-x_n)
  \:\:\:\:\:\:\:\:
  (a \leq x \leq b)
\]
converges uniformly,
and that $f$ is continuous for every $x \neq x_n$.} \\

\emph{Proof.}
\begin{enumerate}
\item[(1)]
Define $f_n(x) = c_n I(x-x_n)$ on $(a,b)$. So
\[
  |f_n(x)| = |c_n| |I(x-x_n)| \leq |c_n|
  \:\:\:\:\:\:\:\:
  (x \in (a,b), n = 1,2,3,\ldots).
\]
Since $\sum|c_n|$ converges, $f = \sum f_n$ converges uniformly (Theorem 7.10).

\item[(2)]
Given any $p \in (a,b)$ with $p \neq x_n$ for all $n=1,2,3,\ldots$.
So each $I(x-x_n)$ is continuous at $x=p$, and thus
each partial sum $\sum_{n=1}^{N} f_n(x)$ is continuous.

\item[(3)]
By Theorem 7.11
\begin{align*}
  \lim_{x \to p} f(x)
  &= \lim_{x \to p} \sum_{n=1}^{\infty} f_n(x) \\
  &= \lim_{N \to \infty} \left( \lim_{x \to p} \sum_{n=1}^{N} f_n(x) \right) \\
  &= \lim_{N \to \infty} \sum_{n=1}^{N} f_n(p) \\
  &= \sum_{n=1}^{\infty} f_n(p) \\
  &= f(p).
\end{align*}
$f(x)$ is continuous at $x=p$ too.
\end{enumerate}
$\Box$ \\\\



%%%%%%%%%%%%%%%%%%%%%%%%%%%%%%%%%%%%%%%%%%%%%%%%%%%%%%%%%%%%%%%%%%%%%%%%%%%%%%%%



\textbf{Exercise 7.9.}
\emph{Let $\{f_n\}$ be a sequence of continuous functions
which converges uniformly to a function $f$ on a set $E$.
Prove that
\[
  \lim_{n \to \infty} f_n(x_n) = f(x)
\]
for every sequence of points $x_n \in E$ such that $x_n \to x$,
and $x \in E$.
Is the converse of this true?} \\

\emph{Proof.}
\begin{enumerate}
  \item[(1)]
  Given any $x \in E$ and any $\varepsilon > 0$.
  Since each $f_n$ is continuous and $f_n \to f$ uniformly,
  $f$ is continuous (Theorem 7.12).
  Hence as $x_n \to x$, there exists an integer $N_1$
  such that
  \[
    |f(x_n) - f(x)| \leq \frac{\varepsilon}{2}
    \text{ whenever } n \geq N_1
  \]
  (Theorem 4.2).
  Also, $f_n \to f$ uniformly implies that there exists an integer $N_2$
  such that
  \[
    |f_n(x_n) - f(x_n)| \leq \frac{\varepsilon}{2}
    \text{ whenever } n \geq N_2.
  \]
  Let $N = \max\{N_1,N_2\}$ be an integer.
  Then
  \[
    |f_n(x_n) - f(x)|
    \leq |f_n(x_n) - f(x_n)| + |f(x_n) - f(x)|
    \leq \frac{\varepsilon}{2} + \frac{\varepsilon}{2}
    = \varepsilon
  \]
  whenever $n \geq N$.
  Therefore, $\lim_{n \to \infty} f_n(x_n) = f(x)$.

  \item[(2)]
  \emph{Show that the converse is false.}
  Let $E = (0,1)$ and $f_n = \frac{1}{nx}$ on $E$.
  Given any $x \in E$.
  First,
  \[
    f(x) = \lim_{n \to \infty} f_n = \lim_{n \to \infty} \frac{1}{nx} = 0
  \]
  Next, for each sequence of points $x_n \in E$ such that $x_n \to x$
  (note that each $x_n \neq 0$ and $x \neq 0$), we have
  \[
    \lim_{n \to \infty} f_n(x_n)
    = \lim_{n \to \infty} \frac{1}{nx_n}
    = \lim_{n \to \infty} \frac{1}{n} \lim_{n \to \infty} \frac{1}{x_n}
    = 0 \cdot \frac{1}{x}
    = 0.
  \]
  Hence $\lim_{n \to \infty} f_n(x_n) = f(x) = 0$.
  However, $\{f_n\}$ does not converge uniformly.
  (See \emph{Proof ($f_n(x) = \frac{1}{x}$, $g_n(x) = \frac{1}{n}$)} in Exercise 7.3.)
\end{enumerate}
$\Box$ \\\\



%%%%%%%%%%%%%%%%%%%%%%%%%%%%%%%%%%%%%%%%%%%%%%%%%%%%%%%%%%%%%%%%%%%%%%%%%%%%%%%%



\textbf{Exercise 7.10.}
\emph{Letting $(x)$ denote the fractional part of the real number $x$
(see Exercise 4.16 for the definition),
consider the function
\[
  f(x) = \sum_{n=1}^{\infty} \frac{(nx)}{n^2}
  \:\:\:\:\:\:\:\:
  (x \in \mathbb{R}).
\]
Find all discontinuities of $f$, and show that they form a countable dense set.
Show that $f$ is nevertheless Riemann-integrable on every bounded interval.} \\

\emph{Proof.}
Let $f_n(x) = \frac{(nx)}{n^2}$ on $\mathbb{R}$,
$F_n(x) = \sum_{k=1}^{n} f_k(x)$ on $\mathbb{R}$.
\begin{enumerate}
\item[(1)]
Since
\[
  \abs{f_n(x)} = \abs{ \frac{(nx)}{n^2} } \leq \frac{1}{n^2}
\]
for all $x \in \mathbb{R}$ and $n=1,2,3,\ldots$
and $\sum \frac{1}{n^2}$ converges (to $\frac{\pi^2}{6}$),
$F_n = \sum f_k$ converges uniformly to $f$ on $\mathbb{R}$ (Theorem 7.10).

\item[(2)]
Note that $(x)$ is continuous on $\mathbb{R} - \mathbb{Z}$
and not continuous on $\mathbb{Z}$ (Exercise 4.16).
Now we define $E_n = \{ x \in \mathbb{R} : nx \in \mathbb{Z}\}$.
So $E_1 = \mathbb{Z}$, and
\[
  \bigcup_{n=1}^{\infty} E_n = \mathbb{Q}.
\]
So $f_n$ is continuous on $\mathbb{R} - E_n$
and not continuous on $E_n$.
So $F_n = \sum f_k$ is continuous on
$\mathbb{R} - \bigcup_{k=1}^{n} E_k
\supseteq \mathbb{R}$ - $\mathbb{Q}$.

\item[(3)]
\emph{Show that $f(x)$ is continuous on $\mathbb{R}$ - $\mathbb{Q}$.}
Since
$\{F_n\}$ is a sequence of continuous functions on $\mathbb{R}$ - $\mathbb{Q}$ (by (2))
and $F_n \to f$ uniformly (by (1)),
$f$ is continuous on $\mathbb{R}$ - $\mathbb{Q}$ (Theorem 7.12).

\item[(4)]
\emph{Show that $f(x)$ is not continuous on $\mathbb{Q}$,
which is a countable dense set of $\mathbb{R}$.}
  \begin{enumerate}
  \item[(a)]
  (Reductio ad absurdum)
  If there were $p = \frac{a}{b} \in \mathbb{Q}$
  with $a,b \in \mathbb{Z}$, $(a,b) = 1$ and $b > 0$
  such that $f(x)$ is continuous at $x = p$,
  then
  \[
    \lim_{x \to p^{-}} f(x) = \lim_{x \to p^{+}} f(x).
  \]

  \item[(b)]
  As $b \mid n$, say $n = bq$ for some $q \in \mathbb{Z}^{+}$, we have
  \begin{align*}
    \lim_{x \to p^{-}} f_n(x)
    &= \lim_{x \to p^{-}} \frac{1}{b^2 q^2}
    = \frac{1}{b^2 q^2}, \\
    \lim_{x \to p^{+}} f_n(x)
    &= \lim_{x \to p^{+}} \frac{0}{b^2 q^2}
    = 0.
  \end{align*}
  As $b \nmid n$,
  \[
    \lim_{x \to p^{-}} f_n(x) = \lim_{x \to p^{+}} f_n(x) = f_n(p).
  \]
  Thus,
  \[
    \lim_{x \to p^{-}} F_n(x) - \lim_{x \to p^{+}} F_n(x)
    = \frac{1}{b^2} \sum_{q = 1}^{[\frac{n}{b}]} \frac{1}{q^2}.
  \]

  \item[(c)]
  Since $F_n \to f$ uniformly, given $\varepsilon = \frac{64}{1989 b^2} > 0$,
  there exists an integer $N'$ such that
  \[
    \abs{ \sum_{n=m}^{\infty} f_n(x) }
    = \sum_{n=m}^{\infty} f_n(x)
    \leq \frac{64}{1989 b^2}
  \]
  whenever $m \geq N'$.

  \item[(d)]
  Take $N = \max\{N', b\}$.
  \begin{align*}
    &\abs{ \underbrace{\lim_{x \to p^{-}} f(x)}_{\text{exists}}
      - \underbrace{\lim_{x \to p^{+}} f(x)}_{\text{exists}} } \\
    =& \abs{
      \underbrace{\lim_{x \to p^{-}} F_N(x)}_{\text{exists}}
      - \underbrace{\lim_{x \to p^{+}} F_N(x)}_{\text{exists}}
      + \underbrace{\lim_{x \to p^{-}} \sum_{n=N+1}^{\infty} f_n(x)}_{\text{exists}}
      - \underbrace{\lim_{x \to p^{+}} \sum_{n=N+1}^{\infty} f_n(x)}_{\text{exists}} } \\
    \geq& \abs{ \lim_{x \to p^{-}} F_N(x) - \lim_{x \to p^{+}} F_N(x) }
      - \abs{  \lim_{x \to p^{-}} \sum_{n=N+1}^{\infty} f_n(x) }
      - \abs{  \lim_{x \to p^{+}} \sum_{n=N+1}^{\infty} f_n(x) } \\
    \geq& \frac{1}{b^2} \sum_{q = 1}^{[\frac{n}{b}]} \frac{1}{q^2}
      - \frac{64}{1989 b^2} - \frac{64}{1989 b^2} \\
    \geq& \frac{1}{q^2} - \frac{64}{1989 b^2} - \frac{64}{1989 b^2} \\
    =& \frac{1861}{1989 b^2} \\
    >& 0,
  \end{align*}
  which is absurd.
  \end{enumerate}

  \item[(4)]
  \emph{Show that $f$ is nevertheless Riemann-integrable on every bounded interval.}
  Since each $f_n \in \mathscr{R}$ on every bounded interval,
  $F_n \in \mathscr{R}$ on every bounded interval.
  Since $F_n \to f$ uniformly,
  $f \in \mathscr{R}$ on every bounded interval by Theorem 7.16.
\end{enumerate}
$\Box$ \\\\



%%%%%%%%%%%%%%%%%%%%%%%%%%%%%%%%%%%%%%%%%%%%%%%%%%%%%%%%%%%%%%%%%%%%%%%%%%%%%%%%



\textbf{Exercise 7.11 (Dirichlet's test).}
\emph{Suppose $\{f_n\}$, $\{g_n\}$ are defined on $E$, and}
\begin{enumerate}
  \item[(a)]
  \emph{$\sum f_n(x)$ has uniformly bounded partial sums;}
  \item[(b)]
  \emph{$g_n(x) \to 0$ uniformly on $E$;}
  \item[(b)]
  \emph{$g_1(x) \geq g_2(x) \geq g_3(x) \geq \cdots$ for every $x \in E$.}
\end{enumerate}
\emph{Prove that $\sum f_n(x) g_n(x)$ converges uniformly on $E$.
(Hint: Compare with Theorem 3.42.)} \\



\emph{Theorem 3.42 (Dirichlet's test).}
Suppose
\begin{enumerate}
  \item[(a)]
  the partial sums $A_n$ of $\sum a_n$ form a bounded sequence;
  \item[(b)]
  $b_0 \geq b_1 \geq b_2 \geq \cdots$;
  \item[(c)]
  $\lim_{n \to \infty} b_n = 0$.
\end{enumerate}
Then $\sum a_n b_n$ converges. \\



\emph{Proof (Theorem 3.42).}
Let $F_n(x) = \sum_{k=1}^{n} f_k(x)$.
Choose $M$ such that $|F_n(x)| \leq M$ for all $n$, all $x \in E$.
Given $\varepsilon > 0$,
there is an integer $N$ such that $g_N(x) \leq \frac{\varepsilon}{2(M+1)}$ for all $x \in E$.
For $N \leq p \leq q$, we have
\begin{align*}
  &\abs{\sum_{n=p}^{q} f_n(x) g_n(x)} \\
  =& \abs{\sum_{n=p}^{q-1} F_n(x)(g_n(x)-g_{n+1}(x)) + F_q(x)g_q(x) - F_{p-1}(x)g_p(x)} \\
  \leq& M \abs{\sum_{n=p}^{q-1}(g_n(x)-g_{n+1}(x)) + g_q(x) + g_p(x)} \\
  =& 2M g_p(x) \\
  \leq& 2M g_N(x) \\
  \leq& \varepsilon
\end{align*}
for all $x \in E$.
Uniformly convergence now follows from the Cauchy criterion (Theorem 7.8).
Note that the first inequality in the above chain depends of course on the fact that
$g_n(x) - g_{n+1}(x) \geq 0$.
$\Box$ \\\\



%%%%%%%%%%%%%%%%%%%%%%%%%%%%%%%%%%%%%%%%%%%%%%%%%%%%%%%%%%%%%%%%%%%%%%%%%%%%%%%%



\textbf{Exercise 7.12.}
\emph{Suppose $g$ and $f_n$ ($n=1,2,3\ldots$) are defined on $(0,\infty)$,
are Riemann-integrable on $[t,T]$ whenever $0 < t < T < \infty$,
$|f_n| \leq g$, $f_n \to f$ uniformly on every compact subset of $(0,\infty)$),
and
\[
  \int_{0}^{\infty} g(x)dx < \infty.
\]
Prove that
\[
  \lim_{n \to \infty} \int_{0}^{\infty} f_n(x)dx = \int_{0}^{\infty} f(x)dx.
\]
(See Exercises 6.7 and 6.8 for the relevant definitions.)
This is a rather weak form of Lebesgue's dominated convergence theorem (Theorem 11.32).
Even in the context of the Riemann integral,
uniform convergence can be replaced by pointwise convergence if
it is assumed that $f \in \mathscr{R}$.
(See the articles by F. Cunningham in Math. Mag., vol. 40, 1967, pp. 179-186,
and by H. Kestelman in Amer. Math. Monthly, vol. 77, 1970, pp. 182-187.)} \\



\emph{Proof.}
\begin{enumerate}
\item[(1)]
It is equivalent to show that
\[
  \lim_{n \to \infty} \int_{0}^{1} f_n(x)dx = \int_{0}^{1} f(x)dx
\]
and
\[
  \lim_{n \to \infty} \int_{1}^{\infty} f_n(x)dx = \int_{1}^{\infty} f(x)dx
\]
in the sense of Exercises 6.7 and 6.8.

\item[(2)]
\emph{Show that $\int_{0}^{1} f_n(x)dx$ ($n=1,2,3\ldots$) and $\int_{0}^{1} f(x)dx$ are
convergent (well-defined) in in the sense of Exercises 6.7.}
By assumption, as $0 < t < 1$ we have
\[
  \abs{\int_{t}^{1} f_n(x)dx}
  \leq \int_{t}^{1} \abs{f_n(x)}dx
  \leq \int_{t}^{1} g(x)dx.
\]
Note that
\[
  \lim_{t \to 0} \int_{t}^{1} g(x)dx = \int_{0}^{1} g(x)dx < \infty
\]
(Exercises 6.7).
Hence
\[
  \lim_{t \to 0}\abs{\int_{t}^{1} f_n(x)dx}
  = \abs{\lim_{t \to 0} \int_{t}^{1} f_n(x)dx}
  \leq \int_{0}^{1} g(x)dx
  < \infty.
\]
Also,
since $|f_n(x)| \leq g(x)$ and $f_n \to f$ uniformly, $f(x) \leq g(x)$ pointwise.
Apply the same argument to get
\[
  \abs{\lim_{t \to 0} \int_{t}^{1} f(x)dx}
  < \infty.
\]
Here $\int_{t}^{1} f(x)dx$ exists by Theorem 7.16.

\item[(3)]
Given any integer $n > 0$ and $t \in (0,1]$, we have
\begin{align*}
  \abs{ \int_{0}^{1} f_n(x)dx - \int_{0}^{1} f(x)dx }
  \leq&
  \abs{ \int_{0}^{1} f_n(x)dx - \int_{t}^{1} f_n(x)dx } \\
    &+ \abs{ \int_{t}^{1} f_n(x)dx - \int_{t}^{1} f(x)dx } \\
    &+ \abs{ \int_{t}^{1} f(x)dx - \int_{0}^{1} f(x)dx } \\
  \leq&
  \abs{ \int_{0}^{t} f_n(x)dx } \\
    &+ \int_{t}^{1} \abs{ f_n(x) - f(x) } dx \\
    &+ \abs{ \int_{0}^{t} f(x)dx } \\
\end{align*}

\item[(4)]
Given $\varepsilon > 0$.
Apply the similar argument in (2),
we have
\begin{align*}
  \abs{ \int_{0}^{t} f_n(x)dx }
  &\leq \int_{0}^{t} |f_n(x)|dx
  \leq \int_{0}^{t} g(x)dx, \\
  \abs{ \int_{0}^{t} f(x)dx }
  &\leq \int_{0}^{t} |f(x)|dx
  \leq \int_{0}^{t} g(x)dx.
\end{align*}
Since $\int_{0}^{t} g(x)dx < \infty$,
there exists a real number $c \in (0,1)$ such that
\[
  \int_{0}^{t} g(x)dx < \frac{\varepsilon}{3}
\]
whenever $0 < t \leq c$.
In particular, for any integer $n > 0$ we have
\begin{align*}
  \abs{ \int_{0}^{c} f_n(x)dx }
  &\leq \frac{\varepsilon}{3}, \\
  \abs{ \int_{0}^{c} f(x)dx }
  &\leq \frac{\varepsilon}{3}.
\end{align*}

\item[(5)]
For such $c \in (0,1)$ in (4), there is an integer $N$ such that
\[
  \abs{f_n(x) - f(x)} < \frac{\varepsilon}{3(1-c)}
\]
whenever $n \geq N$ and $x \in [c,1]$
since $f_n \to f$ uniformly on a compact set $[c,1]$.

\item[(6)]
By (3)(4)(5),
\begin{align*}
  &\abs{ \int_{0}^{1} f_n(x)dx - \int_{0}^{1} f(x)dx } \\
  \leq&
  \abs{ \int_{0}^{c} f_n(x)dx }
    + \int_{c}^{1} \abs{ f_n(x) - f(x) } dx
    + \abs{ \int_{0}^{c} f(x)dx } \\
  <&
  \frac{\varepsilon}{3}
    + (1-c) \cdot \frac{\varepsilon}{3(1-c)}
    + \frac{\varepsilon}{3} \\
  =&
  \varepsilon
\end{align*}
whenever $n \geq N$.
Therefore
\[
  \lim_{n \to \infty} \int_{0}^{1} f_n(x)dx = \int_{0}^{1} f(x)dx.
\]
Similarly,
\[
  \lim_{n \to \infty} \int_{1}^{\infty} f_n(x)dx = \int_{1}^{\infty} f(x)dx.
\]
Hence
\[
  \lim_{n \to \infty} \int_{0}^{\infty} f_n(x)dx = \int_{0}^{\infty} f(x)dx.
\]
\end{enumerate}
$\Box$ \\



\textbf{Supplement (Tannery's convergence theorem for Riemann integrals).}
\emph{(Exercise 10.7 of the book
T. M. Apostol, Mathematical Analysis, Second Edition.)
Prove Tannery's convergence theorem for Riemann integrals:
Given a sequence of functions $\{f_n\}$ and an increasing sequence $\{p_n\}$
of real numbers such that $p_n \to +\infty$ as $n \to \infty$.
Assume that}
\begin{enumerate}
\item[(a)]
\emph{$f_n \to f$ uniformly on $[a,b]$ for every $b \geq a$.}

\item[(b)]
\emph{$f_n$ is Riemann-integrable on $[a,b]$ for every $b \geq a$.}

\item[(c)]
\emph{$|f_n(x)| \leq g(x)$ on $[a,+\infty)$,
where $g$ is improper Riemann-integrable on $[a,+\infty)$.}
\end{enumerate}

\emph{Then both $f$ and $|f|$ are improper Riemann-integrable on $[a,+\infty)$,
the sequence $\{\int_{a}^{p_n}f_n(x)dx\}$ converges, and
\[
  \int_{a}^{\infty} f(x)dx = \lim_{n \to \infty}\int_{a}^{p_n}f_n(x)dx.
\]}
\begin{enumerate}
\item[(d)]
\emph{Use Tannery's theorem to prove that
\[
  \lim_{n \to \infty} \int_{0}^{n} \left( 1-\frac{x}{n} \right)^n x^p dx
  = \int_{0}^{\infty} e^{-x}x^p dx,
\]
if $p > -1$.} \\\\
\end{enumerate}



%%%%%%%%%%%%%%%%%%%%%%%%%%%%%%%%%%%%%%%%%%%%%%%%%%%%%%%%%%%%%%%%%%%%%%%%%%%%%%%%



\textbf{Exercise 7.13.}
\emph{Assume that $\{f_n\}$ is a sequence of monotonically increasing functions on $\mathbb{R}^1$
with $0 \leq  f_n(x) \leq 1$ for all $x$ and all $n$.}
\begin{enumerate}
\item[(a)]
  \emph{Prove that there is a function $f$ and a sequence $\{n_k\}$ such that
  \[
    f(x) = \lim_{k \to \infty} f_{n_k}(x)
  \]
  for every $x \in \mathbb{R}^1$.
  (The existence of such a pointwise convergent subsequence is usually
  called \textbf{Helly's selection theorem}.)}

\item[(b)]
  \emph{If, moreover, $f$ is continuous, does $f_{n_k} \to f$ uniformly on $\mathbb{R}^1$
  or on any bounded subset $E$ of $\mathbb{R}^1$?}
\end{enumerate}

\emph{(Hint:}
\begin{enumerate}
\item[(i)]
\emph{Some subsequence $\{f_{n_i}\}$ converges at all rational points $r$, say, to $f(r)$.}

\item[(ii)]
\emph{Define $f(x)$, for any $x \in \mathbb{R}^1$, to be $\sup f(r)$,
the sup being taken over all $r \leq x$.}

\item[(iii)]
\emph{Show that $f_{n_i}(x) \to f(x)$ at every $x$ at which $f$ is continuous.
(This is where monotonicity is strongly used.)}

\item[(iv)]
\emph{A subsequence of $\{f_{n_i}\}$ converges at every point of discontinuity of $f$
since there are at most countably many such points.}

\end{enumerate}
\emph{This proves (a).
To prove (b), modify your proof of (iii) appropriately.)} \\

\emph{Proof of (a).}
\begin{enumerate}
\item[(1)]
  \emph{Show that there is a subsequence $\{f_{n_i}\}$ converges at all rational points $r$,
  say, to $f(r)$.}
  Let $E = \mathbb{Q}$ be a countable subset of $\mathbb{R}^1$ in Theorem 7.23.

\item[(2)]
  Define $f(x)$, for any $x \in \mathbb{R}^1$, to be $\sup f(r)$,
  the sup being taken over all $r \leq x$.
  It is well-defined since $f(x) = \sup f(r) \leq 1$
  and the construction of $\mathbb{R}^1$ (Theorem 1.19).
  Note that $f$ is monotonically increasing.

\item[(3)]
  \emph{Show that $f_{n_i}(x) \to f(x)$ at every $x$ at which $f$ is continuous.}
  \begin{enumerate}
  \item[(a)]
    Given any $x$ at which $f$ is continuous.
    Given any $\varepsilon > 0$.
    Since $f$ is continuous at $x$, there exists a $\delta > 0$
    such that
    \[
      f(x)-\frac{\varepsilon}{89} < f(r) < f(x)+\frac{\varepsilon}{89}
    \]
    whenever $r \in (x-\delta,x+\delta)$.

  \item[(b)]
    Given any $r \in \mathbb{Q}$.
    By (1), there is an integer $N$ such that
    \[
      f(r)-\frac{\varepsilon}{64} < f_{n_i}(r) < f(r)+\frac{\varepsilon}{64}
    \]
    whenever $i \geq N$.

  \item[(c)]
    As $r \in (x,x+\delta) \bigcap \mathbb{Q} \neq \varnothing$
    (since $\mathbb{Q}$ is dense in $\mathbb{R}$)
    and $i \geq N$, we have
      \begin{align*}
        f_{n_i}(x)
        &\leq f_{n_i}(r)
          &\text{($f_{n_i}$: increasing)} \\
        &< f(r) + \frac{\varepsilon}{64}
          &\text{((b))} \\
        &< f(x)+\frac{\varepsilon}{89} + \frac{\varepsilon}{64}
          &\text{((a))} \\
        &< f(x) + \varepsilon.
      \end{align*}
    Similarly,
    \[
      f_{n_i}(x) > f(x) - \varepsilon.
    \]
    Therefore
    \[
      \abs{ f_{n_i}(x) - f(x) } < \varepsilon
    \]
    whenever $i \geq N$.
  \end{enumerate}

\item[(4)]
  \emph{Show that there is a subsequence of $\{f_{n_i}\}$ converging
  at every point of discontinuity of $f$
  since there are at most countably many such points.}
  \begin{enumerate}
  \item[(a)]
    By construction of $f$, $f$ is monotonically increasing and $0 \leq f(x) \leq 1$.

  \item[(b)]
    Theorem 4.30 implies that there are at most countably many discontinuity points of $f$.

  \item[(c)]
    Apply Theorem 7.23 again to get there is
    a subsequence of $\{f_{n_i}\}$ converging
    at every point of discontinuity of $f$.
  \end{enumerate}

\item[(5)]
  Since any subsequence of $\{f_{n_i}\}$ converges at every point of continuity of $f$,
  there exists a subsequence $\{f_{n_k}\}$ of $\{f_n\}$ such that
  \[
    \lim_{k \to \infty} f_{n_k}(x) = f(x)
  \]
  for $x \in \mathbb{R}^1$ (by (3)(4)).
\end{enumerate}
$\Box$ \\



\emph{Proof of (b).}
\begin{enumerate}
\item[(1)]
  \emph{Show that the result does not hold on $\mathbb{R}^1$.}
  (Using sigmoid functions.)
  \begin{enumerate}
  \item[(a)]
    Define
    \begin{equation*}
    f_n(x) =
      \begin{cases}
        \frac{1}{2(1+e^{-x})} & \text{ if $x < n$}, \\
        1 & \text{ if $x \geq n$}.
      \end{cases}
    \end{equation*}

  \item[(b)]
    $\{f_n\}$ is a sequence of monotonically increasing functions on $\mathbb{R}^1$
    with $0 \leq f_n(x) \leq 1$ for all $x$ and all $n$.

  \item[(c)]
    Define a continuous function $f(x)$ on $\mathbb{R}^1$ by
    \[
      f(x) = \frac{1}{2(1+e^{-x})}.
    \]
    So for every subsequence $\{f_{n_k}\}$ of $\{f_n\}$, we have
    \[
      \lim_{k \to \infty} f_{n_k}(x) = f(x)
    \]
    for all $x \in \mathbb{R}^1$, but
    \[
      \abs{ f_{n}(n) - f(n) }
      = \abs{ 1 - \frac{1}{2(1+e^{-n})} }
      \geq 1 - \frac{1}{2}
      = \frac{1}{2}.
    \]
    So that no subsequence can converge uniformly on $\mathbb{R}^1$
    (by using the similar argument in Example 7.21).
  \end{enumerate}

\item[(2)]
  \emph{Show that the result holds on any bounded subset $E$ of $\mathbb{R}^1$.}
  Might assume that $E = [a,b]$ with $a \neq -\infty$ and $b \neq \infty$.
  \begin{enumerate}
  \item[(a)]
    Given any $\varepsilon > 0$.
    Since $f$ is continuous on a compact set $E$,
    $f$ is continuous uniformly on $E$,
    and thus there exists a $\delta > 0$ such that
    \[
      |f(x) - f(y)| < \frac{\varepsilon}{89}
    \]
    whenever $x,y \in K$ and $|x-y| < \delta$.

  \item[(b)]
    For such $\delta > 0$,
    define a partition $P = \{x_0, x_1, \ldots, x_m\}$ of $[a,b]$ such that
    \[
      \Delta x_j = x_j - x_{j-1} < \frac{\delta}{64}
    \]
    for all $1 \leq j \leq m$.

  \item[(c)]
    Since $f_{n_k} \to f$ (pointwise), for each $x_j$ in the partition $P$
    there exist integers $N_j$ such that
    \[
      \abs{ f_{n_k}(x_j) - f(x_j) } < \frac{\varepsilon}{1989}
    \]
    whenever $k \geq N_j$.
    Take an integer $N = \max\{N_0, N_1, \ldots, N_m\}$. Thus
    \[
      \abs{ f_{n_k}(x_j) - f(x_j) } < \frac{\varepsilon}{1989}
    \]
    whenever $0 \leq j \leq m$ and $k \geq N$.

  \item[(d)]
    As $0 \leq j \leq m$ and $k \geq N$, we have
      \begin{align*}
        &\abs{f_{n_k}(x_j) - f_{n_k}(x_{j-1})} \\
        \leq&
        \abs{f_{n_k}(x_j) - f(x_j)}
          + \abs{f(x_j) - f(x_{j-1})}
          + \abs{f(x_{j-1}) - f_{n_k}(x_{j-1})} \\
        <&
        \frac{\varepsilon}{1989}
          + \frac{\varepsilon}{89}
          + \frac{\varepsilon}{1989}
      \end{align*}
    by (a)(c).

  \item[(e)]
    Now given any $x \in [a,b]$, by (b) there is a subinterval $[x_{i-1},x_i]$
    such that $x \in [x_{i-1},x_i]$.
    Hence
    \begin{align*}
      \abs{f_{n_k}(x) - f(x)}
      \leq&
      \abs{f_{n_k}(x) - f_{n_k}(x_{i-1})} \\
        &+ \abs{f_{n_k}(x_{i-1}) - f(x_{i-1})} \\
        &+ \abs{f(x_{i-1}) - f(x)} \\
      \leq&
      \abs{f_{n_k}(x_i) - f_{n_k}(x_{i-1})}
          &\text{($f_{n_k}$: increasing)} \\
        &+ \abs{f_{n_k}(x_{i-1}) - f(x_{i-1})} \\
        &+ \abs{f(x_{i-1}) - f(x_i)}
          &\text{($f$: increasing)} \\
      <&
      \frac{\varepsilon}{1989}
          + \frac{\varepsilon}{89}
          + \frac{\varepsilon}{1989}
          &\text{((d))} \\
        &+ \frac{\varepsilon}{1989}
          &\text{((c))} \\
        &+ \frac{\varepsilon}{89}
          &\text{((a))} \\
      <& \varepsilon
    \end{align*}
    whenever $k \geq N$.
    The above inequality holds for any $x \in [a,b]$
    and thus $f_{n_k} \to f$ uniformly.
  \end{enumerate}
\end{enumerate}
$\Box$ \\\\



%%%%%%%%%%%%%%%%%%%%%%%%%%%%%%%%%%%%%%%%%%%%%%%%%%%%%%%%%%%%%%%%%%%%%%%%%%%%%%%%



\textbf{Exercise 7.14.}
PLACEHOLDER



%%%%%%%%%%%%%%%%%%%%%%%%%%%%%%%%%%%%%%%%%%%%%%%%%%%%%%%%%%%%%%%%%%%%%%%%%%%%%%%%



\textbf{Exercise 7.15.}
\emph{Suppose that $f$ is a real continuous function on $\mathbb{R}^1$,
$f_n(t)= f(nt)$ for $n=1,2,3,\ldots$,
and $\{f_n\}$ is equicontinuous on $[0,1]$.
What conclusion can you draw about $f$?} \\

\emph{Proof.}
\begin{enumerate}
\item[(1)]
  \emph{Show that $f$ is constant on $[0,\infty)$.}

\item[(2)]
  Given any $\varepsilon > 0$.
  Since $\{f_n\}$ is equicontinuous on $[0,1]$,
  there exists a $1 > \delta > 0$ such that
  \[
    |f_n(x)-f_n(y)| < \varepsilon
  \]
  whenever $n \in \mathbb{Z}^{+}$, $x,y \in [0,1]$ and $|x-y| < \delta < 1$.
  Take $x = t \in [0,1]$ and $y = 0$.
  Note that $f_n(t) = f(nt)$ for any $n \in \mathbb{Z}^{+}$ and
  $t \in \mathbb{R}^1$.
  Hence
  \[
    |f(nt) - f(0)| < \varepsilon
  \]
  for all integer $n > 0$ and $0 \leq t < \delta < 1$.

\item[(3)]
  Given any $x \in [0,\infty)$.
  There is an integer $N > 0$ such that $0 \leq x < N\delta$
  (by taking $N > \frac{\delta}{x}$).
  Let $t = \frac{x}{N}$.
  So that $0 \leq t < \delta$.
  Hence
  \[
    |f(x) - f(0)| = |f(Nt) - f(0)| < \varepsilon.
  \]
  Since $\varepsilon > 0$ is arbitrary,
  $f(x) = f(0)$ for any $x \in [0,+\infty)$.
  Therefore $f$ is constant on $[0,\infty)$.
\end{enumerate}
$\Box$ \\\\



%%%%%%%%%%%%%%%%%%%%%%%%%%%%%%%%%%%%%%%%%%%%%%%%%%%%%%%%%%%%%%%%%%%%%%%%%%%%%%%%



\textbf{Exercise 7.16.}
\emph{Suppose $\{f_n\}$ is an equicontinuous sequence of functions on a compact set $K$,
and $\{f_n\}$ converges pointwise on $K$.
Prove that $\{f_n\}$ converges uniformly on $K$.} \\

(Assume that $\{f_n\}$ is a sequence of complex-valued functions.) \\

\emph{Proof.}
Given any $\varepsilon > 0$.
\begin{enumerate}
  \item[(1)]
  Since $\{f_n\}$ is equicontinuous, there is $\delta > 0$ such that
  \[
    |f_n(x) - f_n(y)| < \frac{\varepsilon}{3}
  \]
  whenever $x,y \in K$, $|x-y| < \delta$, $n = 1,2,3,\ldots$
  (where $d$ is the metric function).

  \item[(2)]
  (Similar to Exercise 4.8.)
  For such $\delta > 0$, we construct an open covering of $K$.
  Pick a collection $\mathscr{C}$ of open balls
  $B(a;\delta)$
  where $a$ runs over all elements of $K$.
  Since $\mathscr{C}$ is an open covering of a compact set $K$,
  there is a finite subcollection $\mathscr{C}'$ of $\mathscr{C}$
  also covers $K$, say
  \[
    \mathscr{C}'
    = \left\{B(a_1;\delta), B(a_2;\delta), \ldots, B(a_m;\delta) \right\}.
  \]

  \item[(3)]
  Since $f_n$ converges pointwise on $K$,
  for each $i$ there is an integer $N_i$ such that
  \[
    |f_n(a_i)-f_m(a_i)| < \frac{\varepsilon}{3}
  \]
  whenever $n,m \geq N_i$.

  \item[(4)]
  Now given any $x \in K$, by (2) there exist $a_j$ $(1 \leq j \leq m)$
  such that $x \in B(a_j;\delta)$.
  Take $N = \max\{N_1,\ldots,N_m\}$.
  Hence
  \begin{align*}
    |f_n(x)-f_m(x)|
    &\leq
    |f_n(x)-f_n(a_j)| + |f_n(a_j)-f_m(a_j)| + |f_m(a_j)-f_m(x)| \\
    &<
    \frac{\varepsilon}{3} + \frac{\varepsilon}{3} + \frac{\varepsilon}{3} \\
    &=
    \varepsilon.
  \end{align*}
  whenever $n,m \geq N$.
  Hence $\{f_n\}$ converges uniformly (Theorem 7.8).
\end{enumerate}
$\Box$ \\\\



%%%%%%%%%%%%%%%%%%%%%%%%%%%%%%%%%%%%%%%%%%%%%%%%%%%%%%%%%%%%%%%%%%%%%%%%%%%%%%%%



\textbf{Exercise 7.17.}
PLACEHOLDER



%%%%%%%%%%%%%%%%%%%%%%%%%%%%%%%%%%%%%%%%%%%%%%%%%%%%%%%%%%%%%%%%%%%%%%%%%%%%%%%%



\textbf{Exercise 7.18.}
PLACEHOLDER



%%%%%%%%%%%%%%%%%%%%%%%%%%%%%%%%%%%%%%%%%%%%%%%%%%%%%%%%%%%%%%%%%%%%%%%%%%%%%%%%



\textbf{Exercise 7.19.}

PLACEHOLDER \\\\



%%%%%%%%%%%%%%%%%%%%%%%%%%%%%%%%%%%%%%%%%%%%%%%%%%%%%%%%%%%%%%%%%%%%%%%%%%%%%%%%



\textbf{Exercise 7.20.}
\emph{If $f$ is continuous on $[0,1]$ and if
\[
  \int_{0}^{1} f(x) x^n dx = 0
  \:\:\:\:\:\:\:\:
  (n=0,1,2,\ldots),
\]
prove that $f(x) = 0$ on $[0,1]$.
(Hint: The integral of the product of $f$ with any polynomial is zero.
Use the Weierstrass theorem to show that
$\int_{0}^{1} f^2(x) dx = 0$.)} \\

\emph{Proof.}
\begin{enumerate}
\item[(1)]
Since $\int_{0}^{1} f(x) x^n dx = 0$ for all $n = 0,1,2,\ldots$,
\[
  \int_{0}^{1} f(x) P(x) dx = 0 \text{ for all } P(x) \in \mathbb{R}[x].
\]

\item[(2)]
By Theorem 7.26 (Stone-Weierstrass Theorem),
there exists a sequence of $P_n(x) \in \mathbb{R}[x]$ such that
\[
  P_n(x) \to f(x)
\]
uniformly on $[0,1]$.
Since $f(x)$ is continuous on the compact set $[0,1]$, $f(x)$ is bounded on $[0,1]$.
Hence
\[
  f(x) P_n(x) \to f^2(x)
\]
uniformly on $[0,1]$.

\item[(3)]
Since each $f(x) P_n(x)$ is continuous,
$f(x) P_n(x) \in \mathscr{R}$ on $[0,1]$ (Theorem 6.8).
By Theorem 7.16,
\[
  \int_{0}^{1} f^2(x) dx
  = \lim_{n \to \infty} \int_{0}^{1} f(x) P_n(x) dx
  = \lim_{n \to \infty} 0
  = 0.
\]

\item[(4)]
Since $f^2(x)$ is continuous,
$f^2(x) = 0$ or $f(x) = 0$ by (3) and Exercise 6.2.
\end{enumerate}
$\Box$ \\\\



%%%%%%%%%%%%%%%%%%%%%%%%%%%%%%%%%%%%%%%%%%%%%%%%%%%%%%%%%%%%%%%%%%%%%%%%%%%%%%%%



\textbf{Exercise 7.21.}

PLACEHOLDER \\\\



%%%%%%%%%%%%%%%%%%%%%%%%%%%%%%%%%%%%%%%%%%%%%%%%%%%%%%%%%%%%%%%%%%%%%%%%%%%%%%%%



\textbf{Exercise 7.22.}
\emph{Assume $f \in \mathscr{R}(\alpha)$ on $[a,b]$,
and prove that there are polynomials $P_n$ such that
\[
  \lim_{n \to \infty} \int_{a}^{b} |f-P_n|^2 d\alpha = 0.
\]
(Compare with Exercise 6.12.)} \\

\emph{Notation.}
For $u \in \mathscr{R}(\alpha)$ on $[a,b]$, define
  \[
    \norm{u}_2 = \left\{ \int_{a}^{b}|u|^2 d\alpha \right\}^{\frac{1}{2}}.
  \] \\

\emph{Proof.}
Given any $\varepsilon = \frac{1}{n} > 0$ $(n=1,2,3,\ldots$).
\begin{enumerate}
\item[(1)]
  By Exercise 6.12, there exists a continuous function $g_n$ on $[a,b]$
  such that
  \[
    \norm{f-g_n}_2 < \frac{1}{n}.
  \]

\item[(2)]
  By Theorem 7.26 (Stone-Weierstrass Theorem),
  there is a polynomial $P_n$ such that
  \[
    |g_n(x)-P_n(x)| < \frac{1}{n}
  \]
  for all $x \in [a,b]$.
  Thus
  \[
    \norm{g_n-P_n}_2
    \leq
    \left\{ \int_{a}^{b}\left(\frac{1}{n}\right)^2 d\alpha \right\}^{\frac{1}{2}}
    =
    \frac{(\alpha(b)-\alpha(a))^{\frac{1}{2}}}{n}.
  \]

\item[(3)]
  By Exercise 6.11,
  \[
    \norm{f-P_n}_2
    \leq
    \norm{f-g_n}_2 + \norm{g_n-P_n}_2
    \leq
    \frac{1+(\alpha(b)-\alpha(a))^{\frac{1}{2}}}{n},
  \]
  or
  \[
    0
    \leq
    \int_{a}^{b} |f-P_n|^2 d\alpha
    \leq
    \frac{[1+(\alpha(b)-\alpha(a))^{\frac{1}{2}}]^2}{n^2}.
  \]
  As $n \to \infty$, $\int_{a}^{b} |f-P_n|^2 d\alpha \to 0$.
\end{enumerate}
$\Box$ \\\\



%%%%%%%%%%%%%%%%%%%%%%%%%%%%%%%%%%%%%%%%%%%%%%%%%%%%%%%%%%%%%%%%%%%%%%%%%%%%%%%%



\textbf{Exercise 7.23.}
\emph{Put $P_0 = 0$, and define, for $n = 0,1,2,\ldots$,
\[
  P_{n+1}(x) = P_n(x) + \frac{x^2-P_n^2(x)}{2}.
\]
Prove that
\[
  \lim_{n \to \infty} P_n(x) = |x|,
\]
uniformly on $[-1,1]$.
(This makes it possible to prove the Stone-Weierstrass theorem without
first proving Theorem 7.26.)
(Hint: Use the identity
\[
  |x| - P_{n+1} = [|x| - P_n(x)]\left[1-\frac{|x|+P_n(x)}{2}\right]
\]
to prove that $0 \leq P_n(x) \leq P_{n+1}(x) \leq |x|$ if $|x| \leq 1$, and that
\[
  |x| - P_n(x) \leq |x| \left(1-\frac{|x|}{2}\right)^n < \frac{2}{n+1}
\]
if $|x| \leq 1$.)} \\

\emph{Proof (Hint).}
\begin{enumerate}
\item[(1)]
\begin{align*}
  |x| - P_{n+1}(x)
  &= |x| - P_n(x) - \frac{|x|^2 - P_n^2(x)}{2} \\
  &= |x| - P_n(x) - \frac{(|x| + P_n(x))(|x| - P_n(x))}{2} \\
  &= [|x| - P_n(x)]\left[1-\frac{|x|+P_n(x)}{2}\right].
\end{align*}

\item[(2)]
\emph{Show that $0 \leq P_n(x) \leq |x|$ if $|x| \leq 1$.}
Induction on $n$.
  \begin{enumerate}
  \item[(a)]
  If $n = 0$, then $P_n(x) = P_0(x) = 0$ and thus $0 \leq P_0(x) \leq |x|$.

  \item[(b)]
  Assume the induction hypothesis that for the single case $n = k$ holds,
  and thus $0 \leq P_k(x) \leq |x|$ if $|x| \leq 1$.
  So
  \begin{align*}
    0 &\leq |x|-P_k(x) \leq |x|, \\
    0 \leq 1-|x| &\leq 1-\frac{|x|+P_k(x)}{2} \leq 1-\frac{|x|}{2} \leq 1
  \end{align*}
  if $|x| \leq 1$.
  Hence
  \[
    0 \leq [|x| - P_k(x)]\left[1-\frac{|x|+P_k(x)}{2}\right] \leq |x|.
  \]
  By (1),
  \[
    0 \leq |x| - P_{k+1}(x) \leq |x|
  \]
  or $0 \leq P_{k+1}(x) \leq |x|$ if $|x| \leq 1$

  \item[(c)]
  Since both the base case in (a) and
  the inductive step in (b) have been proved as true,
  by mathematical induction the result holds.
  \end{enumerate}

\item[(3)]
\emph{Show that $0 \leq P_n(x) \leq P_{n+1}(x) \leq |x|$ if $|x| \leq 1$.}
By (2), it suffices to show that $P_{n}(x) \leq P_{n+1}(x)$.
By (1)(2), we have
\begin{align*}
  |x| - P_{n+1}(x)
  &= [|x| - P_n(x)]\left[1-\frac{|x|+P_n(x)}{2}\right] \\
  &\leq |x| - P_n(x)
\end{align*}
or $P_{n}(x) \leq P_{n+1}(x)$.

\item[(4)]
\emph{Define $f_n(t) = t(1-t)^n$ on $\left[0,\frac{1}{2}\right]$ for $n=1,2,3,\ldots$.
Show that $f_n(t) \leq \frac{1}{n+1}$.}
Since
\[
  f'_n(t) = (1-t)^{n-1}(1 - (n+1)t)
\]
$f'_n(t) = 0$ on $\left[0,\frac{1}{2}\right]$ if and only if $t = \frac{1}{n+1}$.
By Theorem 5.11, $f_n(t)$ reaches its maximum at $t = \frac{1}{n+1}$.
Hence
\[
  f_n(t)
  \leq f_n\left(\frac{1}{n+1}\right)
  = \frac{1}{n+1} \left(\frac{n}{n+1}\right)^n
  < \frac{1}{n+1}.
\]

\item[(5)]
\emph{Show that
\[
  |x| - P_n(x) \leq |x| \left(1-\frac{|x|}{2}\right)^n < \frac{2}{n+1}
\]
if $|x| \leq 1$.}
Note that
\begin{align*}
  |x| - P_n(x)
  &\leq [ |x| - P_0(x) ] \prod_{k=0}^{n-1}\left[1-\frac{|x|+P_k(x)}{2}\right]
    & ((1)) \\
  &\leq |x| \prod_{k=0}^{n-1}\left[1-\frac{|x|}{2}\right]
    & ((2)) \\
  &\leq |x| \left[1-\frac{|x|}{2}\right]^n \\
  & < \frac{2}{n+1}
    & \text{(Put $t=\frac{|x|}{2}$ in (4))}.
\end{align*}

\item[(6)]
(5) implies that
\[
  \lim_{n \to \infty} P_n(x) = |x|
\]
and
\[
  \lim_{n \to \infty} \sup_{x \in [-1,1]}\abs{ P_n(x) - |x| } = 0.
\]
By Theorem 7.9, $P_n(x) \to |x|$ uniformly on $[-1,1]$.
\end{enumerate}
$\Box$ \\\\



%%%%%%%%%%%%%%%%%%%%%%%%%%%%%%%%%%%%%%%%%%%%%%%%%%%%%%%%%%%%%%%%%%%%%%%%%%%%%%%%



\textbf{Exercise 7.24.}
PLACEHOLDER



%%%%%%%%%%%%%%%%%%%%%%%%%%%%%%%%%%%%%%%%%%%%%%%%%%%%%%%%%%%%%%%%%%%%%%%%%%%%%%%%



\textbf{Exercise 7.25.}
PLACEHOLDER



%%%%%%%%%%%%%%%%%%%%%%%%%%%%%%%%%%%%%%%%%%%%%%%%%%%%%%%%%%%%%%%%%%%%%%%%%%%%%%%%



\textbf{Exercise 7.26.}
PLACEHOLDER



%%%%%%%%%%%%%%%%%%%%%%%%%%%%%%%%%%%%%%%%%%%%%%%%%%%%%%%%%%%%%%%%%%%%%%%%%%%%%%%%
%%%%%%%%%%%%%%%%%%%%%%%%%%%%%%%%%%%%%%%%%%%%%%%%%%%%%%%%%%%%%%%%%%%%%%%%%%%%%%%%



\end{document}