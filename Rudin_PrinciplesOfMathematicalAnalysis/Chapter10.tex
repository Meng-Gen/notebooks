\documentclass{article}
\usepackage{amsfonts}
\usepackage{amsmath}
\usepackage{amssymb}
\usepackage{hyperref}
\usepackage[none]{hyphenat}
\usepackage{mathrsfs}
\usepackage{physics}
\parindent=0pt

\def\upint{\mathchoice%
    {\mkern13mu\overline{\vphantom{\intop}\mkern7mu}\mkern-20mu}%
    {\mkern7mu\overline{\vphantom{\intop}\mkern7mu}\mkern-14mu}%
    {\mkern7mu\overline{\vphantom{\intop}\mkern7mu}\mkern-14mu}%
    {\mkern7mu\overline{\vphantom{\intop}\mkern7mu}\mkern-14mu}%
  \int}
\def\lowint{\mkern3mu\underline{\vphantom{\intop}\mkern7mu}\mkern-10mu\int}

\begin{document}

\textbf{\Large Chapter 10: Integration of Differential Forms} \\\\



\emph{Author: Meng-Gen Tsai} \\
\emph{Email: plover@gmail.com} \\\\



%http://pages.cs.wisc.edu/~wentaowu/other-docs/POMA_Solution_Sheet.pdf



%%%%%%%%%%%%%%%%%%%%%%%%%%%%%%%%%%%%%%%%%%%%%%%%%%%%%%%%%%%%%%%%%%%%%%%%%%%%%%%%
%%%%%%%%%%%%%%%%%%%%%%%%%%%%%%%%%%%%%%%%%%%%%%%%%%%%%%%%%%%%%%%%%%%%%%%%%%%%%%%%



\textbf{Exercise 10.1.}
\emph{...} \\

\emph{Proof.}
\begin{enumerate}
\item[(1)]
\item[(2)]

\end{enumerate}
$\Box$ \\\\



%%%%%%%%%%%%%%%%%%%%%%%%%%%%%%%%%%%%%%%%%%%%%%%%%%%%%%%%%%%%%%%%%%%%%%%%%%%%%%%%



\textbf{Exercise 10.2.}
\emph{For $i=1,2,3,\ldots$, let $\varphi_i \in \mathscr{C}(\mathbb{R}^1)$ have support
in $(2^{-i},2^{1-i})$, such that $\int \varphi_i = 1$.
Put
\[
  f(x,y) = \sum_{i=1}^{\infty}[ \varphi_i(x)-\varphi_{i+1}(x) ] \varphi_i(y)
\]
Then $f$ has compact support in $\mathbb{R}^2$,
$f$ is continuous except at $(0,0)$,
and
\[
  \int dy \int f(x,y) dx = 0
  \qquad
  \text{ but }
  \qquad
  \int dx \int f(x,y) dy = 1.
\]
Observe that $f$ is unbounded in every neighborhood of $(0,0)$.} \\

\emph{Proof.}
\begin{enumerate}
\item[(1)]
\item[(2)]

\end{enumerate}
$\Box$ \\\\



%%%%%%%%%%%%%%%%%%%%%%%%%%%%%%%%%%%%%%%%%%%%%%%%%%%%%%%%%%%%%%%%%%%%%%%%%%%%%%%%



\textbf{Exercise 10.3.}
\emph{...} \\

\emph{Proof.}
\begin{enumerate}
\item[(1)]
\item[(2)]

\end{enumerate}
$\Box$ \\\\



%%%%%%%%%%%%%%%%%%%%%%%%%%%%%%%%%%%%%%%%%%%%%%%%%%%%%%%%%%%%%%%%%%%%%%%%%%%%%%%%



\textbf{Exercise 10.4.}
\emph{For $(x,y) \in \mathbb{R}^2$, define
\[
  \mathbf{F}(x,y) = (e^x \cos y - 1, e^x \sin y)
\]
Prove that $\mathbf{F} = \mathbf{G}_2 \circ \mathbf{G}_1$, where
\begin{align*}
  \mathbf{G}_1(x,y) &= (e^x \cos y - 1, y) \\
  \mathbf{G}_2(u,v) &= (u, (1+u) \tan v)
\end{align*}
are primitive in some neighborhood of $(0,0)$.
Compute the Jacobians of $\mathbf{G}_1$, $\mathbf{G}_2$, $\mathbf{F}$ at $(0,0)$.
Define
\[
  \mathbf{H}_2(x,y) = (x, e^x \sin y)
\]
and find
\[
  \mathbf{H}_1(u,v) = (h(u,v),v)
\]
so that $\mathbf{F} = \mathbf{H}_1 \circ \mathbf{H}_2$
is in some neighborhood of $(0,0)$.} \\



\emph{Proof.}
\begin{enumerate}
\item[(1)]
  By Definition 10.5,
  \begin{align*}
    \mathbf{G}_1(x,y) &= (e^x \cos y - 1) \mathbf{e}_1 + y \mathbf{e}_2, \\
    \mathbf{G}_2(u,v) &= u \mathbf{e}_1 + ((1+u) \tan v) \mathbf{e}_2
  \end{align*}
  are primitive in some neighborhood of $(0,0)$.

\item[(2)]
  \emph{Show that $\mathbf{F} = \mathbf{G}_2 \circ \mathbf{G}_1$.}
  Given any $(x,y) \in \mathbb{R}^2$, we have
  \begin{align*}
    (\mathbf{G}_2 \circ \mathbf{G}_1)(x,y)
    &= \mathbf{G}_2(\mathbf{G}_1(x,y)) \\
    &= \mathbf{G}_2(e^x \cos y - 1, y) \\
    &= (e^x \cos y - 1, (1+(e^x \cos y - 1)) \tan y) \\
    &= (e^x \cos y - 1, e^x \sin y) \\
    &= \mathbf{F}(x,y).
  \end{align*}

\item[(3)]
  Since
  \begin{align*}
    J_{\mathbf{G}_1}(x,y)
    &=
    \begin{bmatrix}
      e^x \cos y & -e^x \sin y \\
      0 & 1
    \end{bmatrix} \\
    J_{\mathbf{G}_2}(x,y)
    &=
    \begin{bmatrix}
      1 & 0 \\
      \tan y & (1+x)\sec^2 y
    \end{bmatrix} \\
    J_{\mathbf{F}}(x,y)
    &=
    \begin{bmatrix}
      e^x \cos y & -e^x \sin y \\
      e^x \sin y & e^x \cos y
    \end{bmatrix},
  \end{align*}
  \begin{align*}
    J_{\mathbf{G}_1}(0,0)
    &=
    \begin{bmatrix}
      1 & 0 \\
      0 & 1
    \end{bmatrix} \\
     J_{\mathbf{G}_2}(0,0)
    &=
    \begin{bmatrix}
      1 & 0 \\
      0 & 1
    \end{bmatrix} \\
     J_{\mathbf{F}}(0,0)
    &=
    \begin{bmatrix}
      1 & 0 \\
      0 & 1
    \end{bmatrix}.
  \end{align*}

\item[(4)]
  Define $h(u,v) = \sqrt{e^{2u} - v^{2}} - 1$ on
  \[
    B\left((0,0);\frac{1}{64}\right) \subseteq \mathbb{R}^2.
  \]
  $h(u,v)$ is well-defined since $e^{2u}-v^2 > 0$
  for all $(u,v) \in B\left((0,0);\frac{1}{64}\right)$.

\item[(5)]
  Given any $(x,y) \in \mathbb{R}^2$, we have
  \begin{align*}
    (\mathbf{H}_1 \circ \mathbf{H}_2)(x,y)
    &= \mathbf{H}_1(\mathbf{H}_2(x,y)) \\
    &= \mathbf{H}_1(x, e^x \sin y) \\
    &= (\sqrt{e^{2x} - (e^x \sin y)^2} - 1, e^x \sin y) \\
    &= (e^x \cos y - 1, e^x \sin y) \\
    &= \mathbf{F}(x,y).
  \end{align*}

\end{enumerate}
$\Box$ \\\\



%%%%%%%%%%%%%%%%%%%%%%%%%%%%%%%%%%%%%%%%%%%%%%%%%%%%%%%%%%%%%%%%%%%%%%%%%%%%%%%%



\textbf{Exercise 10.5.}
\emph{...} \\

\emph{Proof.}
\begin{enumerate}
\item[(1)]
\item[(2)]

\end{enumerate}
$\Box$ \\\\



%%%%%%%%%%%%%%%%%%%%%%%%%%%%%%%%%%%%%%%%%%%%%%%%%%%%%%%%%%%%%%%%%%%%%%%%%%%%%%%%



\textbf{Exercise 10.6.}
\emph{...} \\

\emph{Proof.}
\begin{enumerate}
\item[(1)]
\item[(2)]

\end{enumerate}
$\Box$ \\\\



%%%%%%%%%%%%%%%%%%%%%%%%%%%%%%%%%%%%%%%%%%%%%%%%%%%%%%%%%%%%%%%%%%%%%%%%%%%%%%%%



\textbf{Exercise 10.7.}
\emph{...} \\

\emph{Proof.}
\begin{enumerate}
\item[(1)]
\item[(2)]

\end{enumerate}
$\Box$ \\\\



%%%%%%%%%%%%%%%%%%%%%%%%%%%%%%%%%%%%%%%%%%%%%%%%%%%%%%%%%%%%%%%%%%%%%%%%%%%%%%%%



\textbf{Exercise 10.8.}
\emph{Let $H$ be the parallelogram in $\mathbb{R}^2$ whose vertices are
$(1,1)$, $(3,2)$, $(4,5)$, $(2,4)$.
Find the affine map $T$ which sends
$(0,0)$ to $(1,1)$, $(1,0)$ to $(3,2)$, $(1,1)$ to $(4,5)$, $(0,1)$ to $(2,4)$.
Show that $J_{T} = 5$.
Use $T$ to convert the integral
\[
  \alpha = \int_{H} e^{x-y} dxdy
\]
to an integral over $I^2$ and thus compute $\alpha$.} \\

\emph{Proof.}
\begin{enumerate}
\item[(1)]
  By Affine simplexes 10.26,
  \[
    T(\mathbf{x}) = T(\mathbf{0}) + A\mathbf{x},
  \]
  where $A \in L(\mathbb{R}^2, \mathbb{R}^2)$, say
  $A = \begin{bmatrix}
    a & b \\
    c & d
  \end{bmatrix}$.
  Note that $T:
  \begin{bmatrix}
    0 \\
    0
  \end{bmatrix} \mapsto
  \begin{bmatrix}
    1 \\
    1
  \end{bmatrix}$.
  Thus
  \[
    T:
    \begin{bmatrix}
      x \\
      y
    \end{bmatrix} \mapsto
    \begin{bmatrix}
      1 \\
      1
    \end{bmatrix}
    +
    \begin{bmatrix}
      a & b \\
      c & d
    \end{bmatrix}
    \begin{bmatrix}
      x \\
      y
    \end{bmatrix}
    =
    \begin{bmatrix}
      1+ax+by \\
      1+cx+dy
    \end{bmatrix}.
  \]

\item[(2)]
  By $T: (1,0) \mapsto (3,2)$ and $T: (0,1) \mapsto (2,4)$,
  we can solve $A$ as
  \[
    A = \begin{bmatrix}
      2 & 1 \\
      1 & 3
    \end{bmatrix}.
  \]
  It is easy to verify such
  \[
    T:
    \underbrace{\begin{bmatrix}
      x \\
      y
    \end{bmatrix}}_{\mathbf{x}}
    \mapsto
    \underbrace{\begin{bmatrix}
      1 \\
      1
    \end{bmatrix}}_{T(\mathbf{0})}
    +
    \underbrace{\begin{bmatrix}
      2 & 1 \\
      1 & 3
    \end{bmatrix}}_{A}
    \underbrace{\begin{bmatrix}
      x \\
      y
    \end{bmatrix}}_{\mathbf{x}}
    =
    \begin{bmatrix}
      1+2x+y \\
      1+x+3y
    \end{bmatrix}
  \]
  satisfying our requirement.

\item[(3)]
  \[
    J_T
    =
    \det\begin{bmatrix}
      2 & 1 \\
      1 & 3
    \end{bmatrix}
    = 5.
  \]

\item[(4)]
  We cannot apply Theorem 10.9 directly as Exercise 10.9 to 10.13.
  Luckily, we might use the first part of the proof in Theorem 10.9 to
  show that
  \[
    \int_{\mathbb{R}^2} f(\mathbf{y}) d\mathbf{y}
    = \int_{\mathbb{R}^2} f(T(\mathbf{x})) \abs{J_T(\mathbf{x})} d\mathbf{x}
  \]
  holds for affine maps by Theorem 10.2 again.
  Hence
  \begin{align*}
    \int_{H} e^{x-y} dxdy
    &= \int_{[0,1]^2} e^{(1+2u+v)-(1+u+3v)} \abs{J_T} du dv \\
    &= 5 \int_{[0,1]^2} e^{u-2v} du dv \\
    &= 5 \left\{ \int_{0}^{1} e^u du \right\}\left\{ \int_{0}^{1} e^{-2v} dv \right\}
      &(\text{Theorem 10.2}) \\
    &= \frac{5}{2}(e-1)(1-e^{-2}).
  \end{align*}
\end{enumerate}
$\Box$ \\\\



%%%%%%%%%%%%%%%%%%%%%%%%%%%%%%%%%%%%%%%%%%%%%%%%%%%%%%%%%%%%%%%%%%%%%%%%%%%%%%%%



\textbf{Exercise 10.9.}
\emph{...} \\

\emph{Proof.}
\begin{enumerate}
\item[(1)]
\item[(2)]

\end{enumerate}
$\Box$ \\\\



%%%%%%%%%%%%%%%%%%%%%%%%%%%%%%%%%%%%%%%%%%%%%%%%%%%%%%%%%%%%%%%%%%%%%%%%%%%%%%%%



\textbf{Exercise 10.10.}
\emph{...} \\

\emph{Proof.}
\begin{enumerate}
\item[(1)]
\item[(2)]

\end{enumerate}
$\Box$ \\\\



%%%%%%%%%%%%%%%%%%%%%%%%%%%%%%%%%%%%%%%%%%%%%%%%%%%%%%%%%%%%%%%%%%%%%%%%%%%%%%%%



\textbf{Exercise 10.11.}
\emph{...} \\

\emph{Proof.}
\begin{enumerate}
\item[(1)]
\item[(2)]

\end{enumerate}
$\Box$ \\\\



%%%%%%%%%%%%%%%%%%%%%%%%%%%%%%%%%%%%%%%%%%%%%%%%%%%%%%%%%%%%%%%%%%%%%%%%%%%%%%%%



\textbf{Exercise 10.12.}
\emph{...} \\

\emph{Proof.}
\begin{enumerate}
\item[(1)]
\item[(2)]

\end{enumerate}
$\Box$ \\\\



%%%%%%%%%%%%%%%%%%%%%%%%%%%%%%%%%%%%%%%%%%%%%%%%%%%%%%%%%%%%%%%%%%%%%%%%%%%%%%%%



\textbf{Exercise 10.13.}
\emph{...} \\

\emph{Proof.}
\begin{enumerate}
\item[(1)]
\item[(2)]

\end{enumerate}
$\Box$ \\\\



%%%%%%%%%%%%%%%%%%%%%%%%%%%%%%%%%%%%%%%%%%%%%%%%%%%%%%%%%%%%%%%%%%%%%%%%%%%%%%%%



\textbf{Exercise 10.14.}
\emph{...} \\

\emph{Proof.}
\begin{enumerate}
\item[(1)]
\item[(2)]

\end{enumerate}
$\Box$ \\\\



%%%%%%%%%%%%%%%%%%%%%%%%%%%%%%%%%%%%%%%%%%%%%%%%%%%%%%%%%%%%%%%%%%%%%%%%%%%%%%%%



\textbf{Exercise 10.15.}
\emph{If $\omega$ and $\lambda$ are $k$- and $m$-forms, respectively,
prove that}
\[
  \omega \wedge \lambda = (-1)^{km} \lambda \wedge \omega.
\]

\emph{Proof.}
\begin{enumerate}
\item[(1)]
  Write
  \[
    \omega = \sum_I b_I(\mathbf{x}) dx_I,
    \qquad
    \lambda = \sum_J c_J(\mathbf{x}) dx_J
  \]
  in the stardard presentations,
  where $I$ and $J$ range over all increasing $k$-indices
  and over all increasing $m$-indices taken from the set $\{1,\ldots,n\}$.

\item[(2)]
  \emph{Show that $dx_I \wedge dx_J = (-1)^{km} dx_J \wedge dx_I$.}
  \begin{align*}
    dx_I \wedge dx_J
    &= dx_{i_1} \wedge \cdots \wedge dx_{i_k}
      \wedge dx_J \\
    &= (-1)^m dx_{i_1} \wedge \cdots \wedge dx_{i_{k-1}}
      \wedge dx_J \wedge dx_{i_{k}} \\
    &= (-1)^{2m} dx_{i_1} \wedge \cdots \wedge dx_{i_{k-2}}
      \wedge dx_J \wedge dx_{i_{k-1}} \wedge dx_{i_{k}} \\
    &\cdots \\
    &= (-1)^{km} dx_J
      \wedge dx_{i_1} \wedge \cdots \wedge dx_{i_k} \\
    &= (-1)^{km} dx_J \wedge dx_I.
  \end{align*}

\item[(3)]
  \begin{align*}
    \omega \wedge \lambda
    &= \sum_{I,J} b_I(\mathbf{x}) c_J(\mathbf{x}) dx_I \wedge dx_J \\
    &= (-1)^{km} \sum_{J,I} c_J(\mathbf{x}) b_I(\mathbf{x}) dx_J \wedge dx_I \\
    &= (-1)^{km} \lambda \wedge \omega.
  \end{align*}
\end{enumerate}
$\Box$ \\\\



%%%%%%%%%%%%%%%%%%%%%%%%%%%%%%%%%%%%%%%%%%%%%%%%%%%%%%%%%%%%%%%%%%%%%%%%%%%%%%%%



\textbf{Exercise 10.16.}
\emph{...} \\

\emph{Proof.}
\begin{enumerate}
\item[(1)]
\item[(2)]

\end{enumerate}
$\Box$ \\\\



%%%%%%%%%%%%%%%%%%%%%%%%%%%%%%%%%%%%%%%%%%%%%%%%%%%%%%%%%%%%%%%%%%%%%%%%%%%%%%%%



\textbf{Exercise 10.17.}
\emph{...} \\

\emph{Proof.}
\begin{enumerate}
\item[(1)]
\item[(2)]

\end{enumerate}
$\Box$ \\\\



%%%%%%%%%%%%%%%%%%%%%%%%%%%%%%%%%%%%%%%%%%%%%%%%%%%%%%%%%%%%%%%%%%%%%%%%%%%%%%%%



\textbf{Exercise 10.18.}
\emph{...} \\

\emph{Proof.}
\begin{enumerate}
\item[(1)]
\item[(2)]

\end{enumerate}
$\Box$ \\\\



%%%%%%%%%%%%%%%%%%%%%%%%%%%%%%%%%%%%%%%%%%%%%%%%%%%%%%%%%%%%%%%%%%%%%%%%%%%%%%%%



\textbf{Exercise 10.19.}
\emph{...} \\

\emph{Proof.}
\begin{enumerate}
\item[(1)]
\item[(2)]

\end{enumerate}
$\Box$ \\\\



%%%%%%%%%%%%%%%%%%%%%%%%%%%%%%%%%%%%%%%%%%%%%%%%%%%%%%%%%%%%%%%%%%%%%%%%%%%%%%%%



\textbf{Exercise 10.20.}
\emph{...} \\

\emph{Proof.}
\begin{enumerate}
\item[(1)]
\item[(2)]

\end{enumerate}
$\Box$ \\\\



%%%%%%%%%%%%%%%%%%%%%%%%%%%%%%%%%%%%%%%%%%%%%%%%%%%%%%%%%%%%%%%%%%%%%%%%%%%%%%%%



\textbf{Exercise 10.21.}
\emph{...} \\

\emph{Proof.}
\begin{enumerate}
\item[(1)]
\item[(2)]

\end{enumerate}
$\Box$ \\\\



%%%%%%%%%%%%%%%%%%%%%%%%%%%%%%%%%%%%%%%%%%%%%%%%%%%%%%%%%%%%%%%%%%%%%%%%%%%%%%%%



\textbf{Exercise 10.22.}
\emph{...} \\

\emph{Proof.}
\begin{enumerate}
\item[(1)]
\item[(2)]

\end{enumerate}
$\Box$ \\\\



%%%%%%%%%%%%%%%%%%%%%%%%%%%%%%%%%%%%%%%%%%%%%%%%%%%%%%%%%%%%%%%%%%%%%%%%%%%%%%%%



\textbf{Exercise 10.23.}
\emph{...} \\

\emph{Proof.}
\begin{enumerate}
\item[(1)]
\item[(2)]

\end{enumerate}
$\Box$ \\\\



%%%%%%%%%%%%%%%%%%%%%%%%%%%%%%%%%%%%%%%%%%%%%%%%%%%%%%%%%%%%%%%%%%%%%%%%%%%%%%%%



\textbf{Exercise 10.24.}
\emph{...} \\

\emph{Proof.}
\begin{enumerate}
\item[(1)]
\item[(2)]

\end{enumerate}
$\Box$ \\\\



%%%%%%%%%%%%%%%%%%%%%%%%%%%%%%%%%%%%%%%%%%%%%%%%%%%%%%%%%%%%%%%%%%%%%%%%%%%%%%%%



\textbf{Exercise 10.25.}
\emph{...} \\

\emph{Proof.}
\begin{enumerate}
\item[(1)]
\item[(2)]

\end{enumerate}
$\Box$ \\\\



%%%%%%%%%%%%%%%%%%%%%%%%%%%%%%%%%%%%%%%%%%%%%%%%%%%%%%%%%%%%%%%%%%%%%%%%%%%%%%%%



\textbf{Exercise 10.26.}
\emph{...} \\

\emph{Proof.}
\begin{enumerate}
\item[(1)]
\item[(2)]

\end{enumerate}
$\Box$ \\\\



%%%%%%%%%%%%%%%%%%%%%%%%%%%%%%%%%%%%%%%%%%%%%%%%%%%%%%%%%%%%%%%%%%%%%%%%%%%%%%%%



\textbf{Exercise 10.27.}
\emph{...} \\

\emph{Proof.}
\begin{enumerate}
\item[(1)]
\item[(2)]

\end{enumerate}
$\Box$ \\\\



%%%%%%%%%%%%%%%%%%%%%%%%%%%%%%%%%%%%%%%%%%%%%%%%%%%%%%%%%%%%%%%%%%%%%%%%%%%%%%%%



\textbf{Exercise 10.28.}
\emph{...} \\

\emph{Proof.}
\begin{enumerate}
\item[(1)]
\item[(2)]

\end{enumerate}
$\Box$ \\\\



%%%%%%%%%%%%%%%%%%%%%%%%%%%%%%%%%%%%%%%%%%%%%%%%%%%%%%%%%%%%%%%%%%%%%%%%%%%%%%%%



\textbf{Exercise 10.29.}
\emph{...} \\

\emph{Proof.}
\begin{enumerate}
\item[(1)]
\item[(2)]

\end{enumerate}
$\Box$ \\\\



%%%%%%%%%%%%%%%%%%%%%%%%%%%%%%%%%%%%%%%%%%%%%%%%%%%%%%%%%%%%%%%%%%%%%%%%%%%%%%%%



\textbf{Exercise 10.30.}
\emph{...} \\

\emph{Proof.}
\begin{enumerate}
\item[(1)]
\item[(2)]

\end{enumerate}
$\Box$ \\\\


%%%%%%%%%%%%%%%%%%%%%%%%%%%%%%%%%%%%%%%%%%%%%%%%%%%%%%%%%%%%%%%%%%%%%%%%%%%%%%%%



\textbf{Exercise 10.31.}
\emph{...} \\

\emph{Proof.}
\begin{enumerate}
\item[(1)]
\item[(2)]

\end{enumerate}
$\Box$ \\\\


%%%%%%%%%%%%%%%%%%%%%%%%%%%%%%%%%%%%%%%%%%%%%%%%%%%%%%%%%%%%%%%%%%%%%%%%%%%%%%%%



\textbf{Exercise 10.32.}
\emph{...} \\

\emph{Proof.}
\begin{enumerate}
\item[(1)]
\item[(2)]

\end{enumerate}
$\Box$ \\\\



%%%%%%%%%%%%%%%%%%%%%%%%%%%%%%%%%%%%%%%%%%%%%%%%%%%%%%%%%%%%%%%%%%%%%%%%%%%%%%%%
%%%%%%%%%%%%%%%%%%%%%%%%%%%%%%%%%%%%%%%%%%%%%%%%%%%%%%%%%%%%%%%%%%%%%%%%%%%%%%%%



\end{document}