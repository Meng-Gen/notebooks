\documentclass{article}
\usepackage{amsfonts}
\usepackage{amsmath}
\usepackage{amssymb}
\usepackage{hyperref}
\usepackage{mathrsfs}
\parindent=0pt

\def\upint{\mathchoice%
    {\mkern13mu\overline{\vphantom{\intop}\mkern7mu}\mkern-20mu}%
    {\mkern7mu\overline{\vphantom{\intop}\mkern7mu}\mkern-14mu}%
    {\mkern7mu\overline{\vphantom{\intop}\mkern7mu}\mkern-14mu}%
    {\mkern7mu\overline{\vphantom{\intop}\mkern7mu}\mkern-14mu}%
  \int}
\def\lowint{\mkern3mu\underline{\vphantom{\intop}\mkern7mu}\mkern-10mu\int}

\begin{document}

\textbf{\Large Chapter 6: The Riemann-Stieltjes Integral} \\\\



\textbf{Exercise 6.1.}
\emph{Suppose $\alpha$ increases on $[a, b]$, $a \leq ≤ x_0 \leq b$,
$\alpha$ is continuous at $x_0$, $f(x_0) = 1$, and $f(x) = 0$ if $x \neq x_0$.
Prove that $f \in \mathscr{R}(\alpha)$ and that $\int f d \alpha = 0$.} \\

Given any partition $P = \{a = p_0, p_1, ..., p_{n-1}, p_n = b \}$,
where $a = p_0 \leq p_1 \leq \cdots \leq p_{n-1} \leq p_n = b$.
We might compute $L(P, f, \alpha)$ and $U(P, f, \alpha)$ by using $\epsilon$-$\delta$ argument
since we are hinted by the condition that $\alpha$ is continuous.
A function which is continuous at $x_0$ has a nice property near $x_0$
and this property would help us estimate $U(P, f, \alpha)$ near $x_0$.
On the contrary, if both $f$ and $\alpha$ are discontinuous at $x_0$,
it might be $f \not\in \mathscr{R}(\alpha)$.\\

\textbf{Claim 1.}
\emph{$L(P, f, \alpha) = 0$.} \\
\emph{Proof of Claim 1.}
$m_i = 0$ since $\inf f(x) = 0$ on any subinterval of $[a, b]$.
So $L(P, f, \alpha) = \sum m_i \Delta \alpha_i = 0$.
Here we don't need the condition that $\alpha$ is continuous at $x_0$.
$\Box$ \\

\textbf{Claim 2.}
\emph{For any $\epsilon > 0$,
there exists a partition $P$ such that $U(P, f, \alpha) < \epsilon$.} \\
\emph{Proof of Claim 2.}
Let $x_0 \in [p_{i_0 - 1}, p_{i_0}]$ for some $i_0$. Then
$M_i = \sup_{p_{i - 1} \leq x \leq p_i} f(x) = 0$ if $i \neq i_0$, and $M_{i_0} = 1$.
So
$$U(P, f, \alpha) = \sum M_i \Delta \alpha_i = \Delta \alpha_{i_0}.$$
It is not true for any arbitrary $\alpha$. (For example, $\alpha$ has a jump on $x = x_0$.)
In fact, Exercise 6.3 shows this.
Luckily, $\alpha$ is continuous at $x_0$. So for $\epsilon > 0$,
there exists $\delta > 0$ such that $|\alpha(x) - \alpha(x_0)| < \frac{\epsilon}{2}$
whenever $|x - x_0| < \delta$ (and $x \in [a, b]$).
Now we pick a nice partition
$$P = \{ a, x_0 - \delta_1, x_0 + \delta_2, b \},$$
where $\delta_1 = \min(\delta, x_0 - a) \geq 0$
and $\delta_2 = \min(\delta, b - x_0) \geq 0$.
(It is a trick about resizing ``$\delta$''
to avoid considering the edge cases $x_0 = a$ or $x_0 = b$ or $a = b$.)
Then $x_0 \in [x_0 - \delta_1, x_0 + \delta_2]$
and $\Delta \alpha$ on $[x_0 - \delta_1, x_0 + \delta_2]$ is
\begin{align*}
\alpha(x_0 + \delta_2) - \alpha(x_0 - \delta_1)
&= (\alpha(x_0 + \delta_2) - \alpha(x_0)) + (\alpha(x_0) - \alpha(x_0 - \delta_1)) \\
&< \frac{\epsilon}{2} + \frac{\epsilon}{2} = \epsilon.
\end{align*}
Therefore, $U(P, f, \alpha) < \epsilon$.
$\Box$ \\

\emph{Proof (Definition 6.2).}
By Claim 1 and 2 and notice that $U(P, f, \alpha) \geq 0$ for any partition $P$,
\begin{align*}
\upint_a^b f d\alpha &= \inf U(P, f, \alpha) = 0, \\
\lowint_a^b f d\alpha &= \sup L(P, f, \alpha) = 0,
\end{align*}
the inf and sup again being taken over all partitions.
Hence $f \in \mathscr{R}(\alpha)$ and that $\int f d \alpha = 0$ by Definition 6.2.
$\Box$ \\

\emph{Proof (Theorem 6.5).}
By Claim 1 and 2,
$$0 \leq U(P, f, \alpha) - L(P, f, \alpha) < \epsilon.$$
Hence $f \in \mathscr{R}(\alpha)$ by Theorem 6.5.
Furthermore,
$$\int f d \alpha = \lowint_a^b f d\alpha = \sup L(P, f, \alpha) = 0.$$
$\Box$ \\

\emph{Proof (Theorem 6.10).}
$f \in \mathscr{R}(\alpha)$ by Theorem 6.10.
Thus, by Claim 1
$$\int f d \alpha = \lowint_a^b f d\alpha = \sup L(P, f, \alpha) = 0.$$
$\Box$ \\

\end{document}