\documentclass{article}
\usepackage{amsfonts}
\usepackage{amsmath}
\usepackage{amssymb}
\usepackage{hyperref}
\usepackage{mathrsfs}
\parindent=0pt

\def\upint{\mathchoice%
    {\mkern13mu\overline{\vphantom{\intop}\mkern7mu}\mkern-20mu}%
    {\mkern7mu\overline{\vphantom{\intop}\mkern7mu}\mkern-14mu}%
    {\mkern7mu\overline{\vphantom{\intop}\mkern7mu}\mkern-14mu}%
    {\mkern7mu\overline{\vphantom{\intop}\mkern7mu}\mkern-14mu}%
  \int}
\def\lowint{\mkern3mu\underline{\vphantom{\intop}\mkern7mu}\mkern-10mu\int}

\begin{document}

\textbf{\Large Chapter 6: The Riemann-Stieltjes Integral} \\\\



\emph{Author: Meng-Gen Tsai} \\
\emph{Email: plover@gmail.com} \\\\



\textbf{Supplement.} Another definition of Riemann-Stieltjes integral. \\
\emph{(Exercise 7.3, 7.4 of the book
T. M. Apostol, Mathematical Analysis, Second Edition.)
Let $P$ be a partition of $[a, b]$.
The norm of a partition $P$ is the length of the largest subinterval $[x_{i-1}, x_i]$
of $P$ and is denoted by $\Vert P \Vert$.} \\

\emph{We say $f \in \mathscr{R}(\alpha)$
if there exists $A \in \mathbb{R}$ having the property that
for any $\varepsilon > 0$, there exists $\delta > 0$ such that
for any partition $P$ of $[a, b]$ with norm $\Vert P \Vert < \delta$
and for any choice of $t_i \in [x_{i-1}, x_i]$,
we have $|\sum_{i = 1}^{n} f(t_i) \Delta \alpha_i - A| < \varepsilon$.} \\

\textbf{Cliam.}
\emph{$f \in \mathscr{R}$ in the sense of Definition 6.2
implies that
$f \in \mathscr{R}$ in the sense of this another definition.} \\
\emph{Proof of Claim.}
Let $A = \int f dx$, $M > 0$ be one upper bound of $|f|$ on $[a, b]$.
Given $\varepsilon > 0$, there exists a partition
$P_0 = \{a = x_0, x_1, ..., x_{N-1}, x_N = b \}$
such that
$U(P_0, f) \leq A + \frac{\varepsilon}{2}$.
Let $\delta = \frac{\varepsilon}{2MN} > 0$.
Then for any partition $P$ with norm $\Vert P \Vert < \delta$, write
$$U(P, f) = \sum_{i = 1}^{n} M_i \Delta x_i = S_1 + S_2,$$
where
$S_1$ is the sum of terms arising from those subintervals of $P$ containing no point of $P_0$,
$S_2$ is the sum of the remaining terms.
Then
\begin{align*}
S_1 &\leq U(P_0, f) < A + \frac{\varepsilon}{2}, \\
S_2 &\leq NM \Vert P \Vert < NM \delta < \frac{\varepsilon}{2}.
\end{align*}
Therefore, $U(P, f) < A + \varepsilon$.
Similarly, $L(P, f) > A - \varepsilon$ whenever $\Vert P \Vert < \delta'$.
Hence, $|\sum_{i = 1}^{n} f(t_i) \Delta x_i - A| < \varepsilon$
whenever $\Vert P \Vert < \min(\delta, \delta')$.
(Copy Apostol's hint and ensure $M > 0$. $M$ in Apostol's hint might be zero if $f = 0$.)
$\Box$ \\



This supplement will be used in computing
$\int_0^{\infty} (\frac{\sin x}{x})^2 dx = \frac{\pi}{2}$ in Exercise 8.12. \\\\


\textbf{Exercise 6.1.}
\emph{Suppose $\alpha$ increases on $[a, b]$, $a \leq ≤ x_0 \leq b$,
$\alpha$ is continuous at $x_0$, $f(x_0) = 1$, and $f(x) = 0$ if $x \neq x_0$.
Prove that $f \in \mathscr{R}(\alpha)$ and that $\int f d \alpha = 0$.} \\

Given any partition $P = \{a = p_0, p_1, ..., p_{n-1}, p_n = b \}$,
where $a = p_0 \leq p_1 \leq \cdots \leq p_{n-1} \leq p_n = b$.
We might compute $L(P, f, \alpha)$ and $U(P, f, \alpha)$ by using $\varepsilon$-$\delta$ argument
since we are hinted by the condition that $\alpha$ is continuous.
A function which is continuous at $x_0$ has a nice property near $x_0$
and this property would help us estimate $U(P, f, \alpha)$ near $x_0$.
On the contrary, if both $f$ and $\alpha$ are discontinuous at $x_0$,
it might be $f \not\in \mathscr{R}(\alpha)$.
Besides, if $f$ has too many points of discontinuity
($f(x) = 0$ if $x \in \mathbb{Q}$ and $f(x) = 1$ otherwise, for example),
then $f$ might not be Riemann-integrable on $[0, 1]$. \\

\textbf{Claim 1.}
\emph{$L(P, f, \alpha) = 0$.} \\
\emph{Proof of Claim 1.}
$m_i = 0$ since $\inf f(x) = 0$ on any subinterval of $[a, b]$.
So $L(P, f, \alpha) = \sum m_i \Delta \alpha_i = 0$.
Here we don't need the condition that $\alpha$ is continuous at $x_0$.
$\Box$ \\

\textbf{Claim 2.}
\emph{For any $\varepsilon > 0$,
there exists a partition $P$ such that $U(P, f, \alpha) < \varepsilon$.} \\
\emph{Proof of Claim 2.}
Let $x_0 \in [p_{i_0 - 1}, p_{i_0}]$ for some $i_0$. Then
$M_i = \sup_{p_{i - 1} \leq x \leq p_i} f(x) = 0$ if $i \neq i_0$, and $M_{i_0} = 1$.
So
$$U(P, f, \alpha) = \sum M_i \Delta \alpha_i = \Delta \alpha_{i_0}.$$
It is not true for any arbitrary $\alpha$. (For example, $\alpha$ has a jump on $x = x_0$.)
In fact, Exercise 6.3 shows this.
Luckily, $\alpha$ is continuous at $x_0$. So for $\varepsilon > 0$,
there exists $\delta > 0$ such that $|\alpha(x) - \alpha(x_0)| < \frac{\varepsilon}{2}$
whenever $|x - x_0| < \delta$ (and $x \in [a, b]$).
Now we pick a nice partition
$$P = \{ a, x_0 - \delta_1, x_0 + \delta_2, b \},$$
where $\delta_1 = \min(\delta, x_0 - a) \geq 0$
and $\delta_2 = \min(\delta, b - x_0) \geq 0$.
(It is a trick about resizing ``$\delta$''
to avoid considering the edge cases $x_0 = a$ or $x_0 = b$ or $a = b$.)
Then $x_0 \in [x_0 - \delta_1, x_0 + \delta_2]$
and $\Delta \alpha$ on $[x_0 - \delta_1, x_0 + \delta_2]$ is
\begin{align*}
\alpha(x_0 + \delta_2) - \alpha(x_0 - \delta_1)
&= (\alpha(x_0 + \delta_2) - \alpha(x_0)) + (\alpha(x_0) - \alpha(x_0 - \delta_1)) \\
&< \frac{\varepsilon}{2} + \frac{\varepsilon}{2} = \varepsilon.
\end{align*}
Therefore, $U(P, f, \alpha) < \varepsilon$.
$\Box$ \\

\emph{Proof (Definition 6.2).}
By Claim 1 and 2 and notice that $U(P, f, \alpha) \geq 0$ for any partition $P$,
\begin{align*}
\upint_a^b f d\alpha &= \inf U(P, f, \alpha) = 0, \\
\lowint_a^b f d\alpha &= \sup L(P, f, \alpha) = 0,
\end{align*}
the inf and sup again being taken over all partitions.
Hence $f \in \mathscr{R}(\alpha)$ and that $\int f d \alpha = 0$ by Definition 6.2.
$\Box$ \\

\emph{Proof (Theorem 6.5).}
By Claim 1 and 2,
$$0 \leq U(P, f, \alpha) - L(P, f, \alpha) < \varepsilon.$$
Hence $f \in \mathscr{R}(\alpha)$ by Theorem 6.5.
Furthermore,
$$\int f d \alpha = \lowint_a^b f d\alpha = \sup L(P, f, \alpha) = 0.$$
$\Box$ \\

\emph{Proof (Theorem 6.10).}
$f \in \mathscr{R}(\alpha)$ by Theorem 6.10.
Thus, by Claim 1
$$\int f d \alpha = \lowint_a^b f d\alpha = \sup L(P, f, \alpha) = 0.$$
$\Box$ \\

\end{document}