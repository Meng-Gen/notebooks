\documentclass{article}
\usepackage{amsfonts}
\usepackage{amsmath}
\usepackage{amssymb}
\usepackage{hyperref}
\usepackage[none]{hyphenat}
\usepackage{mathrsfs}
\usepackage{physics}
\parindent=0pt

\def\upint{\mathchoice%
    {\mkern13mu\overline{\vphantom{\intop}\mkern7mu}\mkern-20mu}%
    {\mkern7mu\overline{\vphantom{\intop}\mkern7mu}\mkern-14mu}%
    {\mkern7mu\overline{\vphantom{\intop}\mkern7mu}\mkern-14mu}%
    {\mkern7mu\overline{\vphantom{\intop}\mkern7mu}\mkern-14mu}%
  \int}
\def\lowint{\mkern3mu\underline{\vphantom{\intop}\mkern7mu}\mkern-10mu\int}

\begin{document}



\textbf{\Large Chapter 2: Basic Topology} \\\\



\emph{Author: Meng-Gen Tsai} \\
\emph{Email: plover@gmail.com} \\\\



% http://assets.press.princeton.edu/chapters/s8008.pdf
% http://www.math.pitt.edu/~sparling/091/15391/15391homework4s.pdf



%%%%%%%%%%%%%%%%%%%%%%%%%%%%%%%%%%%%%%%%%%%%%%%%%%%%%%%%%%%%%%%%%%%%%%%%%%%%%%%%



\textbf{Notation.}
\begin{enumerate}
\item[(1)]
$E^{\circ}$ or $\text{int}(E)$ is the interior of $E$.
\item[(2)]
$\overline{E}$ is the closure of $E$.
\item[(3)]
$\widetilde{E}$ is the complement of $E$.
\item[(4)]
$B(p;r)$ or $B(p)$ is the set of all points $q$ in a metric space $(M,d)$
such that $d_M(p,q) < r$. \\\\
\end{enumerate}



%%%%%%%%%%%%%%%%%%%%%%%%%%%%%%%%%%%%%%%%%%%%%%%%%%%%%%%%%%%%%%%%%%%%%%%%%%%%%%%%



\textbf{Exercise 2.1.}
\emph{Prove that the empty set is a subset of every set.} \\

\emph{Proof.}
By Definitions 1.3,
\begin{enumerate}
\item[(1)]
The set which contains no element will be called the \textbf{empty set},
\item[(2)]
If $A$ and $B$ are sets, and if every element of $A$ is an element of $B$,
we say that $A$ is a \textbf{subset} of $B$,
\end{enumerate}
every element of the empty set (there are none) belongs to every set.
That is, the empty set is a subset of every set.
$\Box$ \\\\



%%%%%%%%%%%%%%%%%%%%%%%%%%%%%%%%%%%%%%%%%%%%%%%%%%%%%%%%%%%%%%%%%%%%%%%%%%%%%%%%



\textbf{Exercise 2.2.}
\emph{A complex number $z$ is said to be algebraic if there are integers
$a_0, ..., a_n$, not all zero, such that
$$a_0 z^n + a_1 z^{n-1} + \cdots + a_{n-1} z + a_n = 0.$$
Prove that the set of all algebraic numbers is countable.
(Hint: For every positive integer $N$ there are only finitely many equations with
$$n + |a_0| + |a_1| + \cdots + |a_n| = N.$$}

Might assume $a_0 \neq 0$. \\

For example, all rational numbers are algebraic
since $p = \frac{\alpha}{\beta}$ (where $\alpha, \beta \in \mathbb{Z}$)
is a root of $\beta z - \alpha = 0$. \\

Besides, $z = \sqrt{2} + \sqrt{3}$ is algebraic since $z^4 - 10z^2 + 1 = 0$.
In fact, $z = \pm\sqrt{2} \pm\sqrt{3}$ are also algebraic since
$z^4 - 10z^2 + 1 =
(z - \sqrt{2} - \sqrt{3})(z + \sqrt{2} - \sqrt{3})
(z - \sqrt{2} + \sqrt{3})(z + \sqrt{2} + \sqrt{3})$. \\

\textbf{Lemma.}
\emph{The set of all polynomials over $\mathbb{Z}$ is countable implies that
the set of algebraic numbers is countable.} \\

\emph{Proof of Lemma.}
By definition, we write the set of algebraic numbers as
$$S = \bigcup_{f(x) \in \mathbb{Z}[x]} \{ z \in \mathbb{C} : f(z) = 0 \}.$$
Since each polynomial of degree $n$ has at most $n$ roots,
$\{ z \in \mathbb{C} : f(z) = 0 \}$ is finite for each given $f(x) \in \mathbb{Z}[x]$.
So $S$ is a countable union (by assumption) of finite sets, and hence at most countable.
$S$ is infinite since every integer $\alpha$ is a root of $f(z) = z - \alpha$.
So $S$ is countable.
$\Box$ \\

Thus, it suffices to show that
\emph{the set of all polynomials over $\mathbb{Z}$ is countable.} \\

\emph{Proof (Hint).}
For every positive integer $N$ there are only finitely many equations with
$n + |a_0| + |a_1| + \cdots + |a_n| = N.$
Write
$$P_N = \{ f(x) \in \mathbb{Z}[x] : n + |a_0| + |a_1| + \cdots + |a_n| = N \}$$
where $f(x) = a_0 z^n + a_1 z^{n-1} + \cdots + a_{n-1} z + a_n$ with $a_0 \neq 0$,
and
$$P = \bigcup_{N = 1}^{\infty} P_N.$$
$P$ is the set of all polynomials over $\mathbb{Z}$. \\

Each $P_N$ is finite for given $N$
(since the equation $n + |a_0| + |a_1| + \cdots + |a_n| = N$
has finitely many solutions
$(n, a_0, a_1, ..., a_n) \in \mathbb{Z}^{n+2}$).
So $P$ is a countable union of finite sets, and hence at most countable.
$P$ is infinite since $\mathbb{Z}$ is a subring of $\mathbb{Z}[x]$.
So $P$ is countable.
$\Box$ \\

\emph{Proof (Theorem 2.13).}
\begin{enumerate}
\item[(1)]
\emph{$\mathbb{Z}^N$ is countable for any integer $N > 0$.}
Theorem 2.13.
\item[(2)]
\emph{The set of all polynomials over $\mathbb{Z}$ is countable.}
Let
$$P_n = \{ f \in \mathbb{Z}[x] : \deg f = n \},$$
and
$$P = \bigcup_{n = 1}^{\infty} P_n = \mathbb{Z}[x].$$

\emph{Claim: $P_n$ is countable.}
Define a 1-1 map $\varphi_n: P_n \rightarrow \mathbb{Z}^{n+1}$ by
$$\varphi_n(a_0 z^n + a_1 z^{n-1} + \cdots + a_n)
= (a_0, a_1, ..., a_{n-1}, a_n).$$
By (1) and Theorem 2.8, $P_n$ is countable.
($P_n$ is infinite since $a_n \in \mathbb{Z}$.)
Now $P$ is a countable union of countable sets,
and hence countable by Theorem 2.12.
\end{enumerate}
$\Box$ \\

\emph{Proof (Unique factorization theorem).}
\begin{enumerate}
\item[(1)]
\emph{The set of prime numbers is countable.}
Write all primes in the ascending order as $p_1, p_2, ..., p_n, ...$
where $p_1 = 2, p_2 = 3, ..., p_{10001} = 104743, ...$
(See \href{https://projecteuler.net/problem=7}{ProjectEuler 7: 10001st prime}.
Use sieve of Eratosthenes to get $p_{10001}$.)
\item[(2)]
\emph{The set of all polynomials over $\mathbb{Z}$ is countable.}
Let
$$P_n = \{ f \in \mathbb{Z}[x] : \deg f = n \},$$
and
$$P = \bigcup_{n = 1}^{\infty} P_n = \mathbb{Z}[x].$$

\emph{Claim: $P_n$ is countable.}
Define a map $\varphi_n: P_n \rightarrow \mathbb{Z}^+$ by
$$\varphi_n(a_0 z^n + a_1 z^{n-1} + \cdots + a_n)
= p_1^{\psi(a_0)} p_2^{\psi(a_1)} \cdots p_{n+1}^{\psi(a_n)},$$
where $\psi$ is a 1-1 correspondence from $\mathbb{Z}$ to $\mathbb{Z}^+$ (Example 2.5).
By the unique factorization theorem, $\varphi_n$ is 1-1.
So $P_n$ is countable by Theorem 2.8.
($P_n$ is infinite since $a_n \in \mathbb{Z}$.)
Now $P$ is a countable union of countable sets,
and hence countable by Theorem 2.12.
\end{enumerate}
$\Box$ \\\\



%%%%%%%%%%%%%%%%%%%%%%%%%%%%%%%%%%%%%%%%%%%%%%%%%%%%%%%%%%%%%%%%%%%%%%%%%%%%%%%%



\textbf{Exercise 2.3.}
\emph{Prove that there exist real numbers which are not algebraic.} \\

\emph{Proof (Exercise 2.2).}
If all real numbers were algebraic, then $\mathbb{R}$ is countable by Exercise 2.2,
contrary to the fact that
$\mathbb{R}$ is uncountable (Corollary to Theorem 2.43).
$\Box$ \\

\emph{Proof (Liouville, 1844).}
\begin{enumerate}
\item[(1)]
\textbf{Lemma.}
\emph{If $\xi$ is a real algebraic number of degree $n > 1$,
then there is a constant $A > 0$ (depending on $\xi$) such that
$$\left| \xi - \frac{h}{k} \right| \geq \frac{A}{k^n}$$
for all rational numbers $\frac{h}{k}$.}
\begin{enumerate}
\item[(a)]
If $\left| \xi - \frac{h}{k} \right| \geq 1$, pick $A = 1 > 0$.
\item[(b)]
If $\left| \xi - \frac{h}{k} \right| < 1$,
let $f(x) = a_0 + a_1 x + \cdots + a_n x^n$
be an irreducible polynomial of degree $n > 1$ over $\mathbb{Z}$ such that
$f(\xi) = 0$.
By the mean value theorem,
$$f(\xi) - f\left( \frac{h}{k} \right)
= \left( \xi - \frac{h}{k} \right) f'(c)$$
for some
$c
\in \left( \xi - \frac{h}{k}, \xi + \frac{h}{k} \right)
\subseteq (\xi - 1, \xi + 1)$.
Notice that
\begin{enumerate}
\item[(i)]
$f(\xi) = 0$ by definition.
\item[(ii)]
$f\left( \frac{h}{k} \right) \neq 0$ since $\frac{h}{k}$ cannot be a root of $f(x)$.
Otherwise $f$ is of degree $1$, contrary to the assumption of $f$.
\item[(iii)]
$\left| f\left( \frac{h}{k} \right) \right| \geq \frac{1}{k^n}$
since
\begin{align*}
  f\left( \frac{h}{k} \right)
  &= a_0 + a_1 \left( \frac{h}{k} \right) + \cdots + a_n \left( \frac{h}{k} \right)^n
  \neq 0, \\
  k^n f\left( \frac{h}{k} \right)
  &= a_0 k^n + h k^{n-1} a_1 + \cdots + h^n a_n
  \neq 0, \\
  k^n \left| f\left( \frac{h}{k} \right) \right|
  &\geq 1.
\end{align*}
\item[(iv)]
$|f'(c)| \leq \sup_{x \in [\xi - 1, \xi + 1]}|f'(x)|$ since
$c \in [\xi - 1, \xi + 1]$
and $f'(x)$ is continuous or bounded on a compact set $[\xi - 1, \xi + 1]$.
\end{enumerate}
By (i)-(iv),
\begin{align*}
  \left| f(\xi) - f\left( \frac{h}{k} \right) \right|
  &= \left| \left( \xi - \frac{h}{k} \right) f'(c) \right|, \\
  \frac{1}{k^n} \leq \left| f\left( \frac{h}{k} \right) \right|
  &= \left| \xi - \frac{h}{k} \right| |f'(c)|
  \leq \left| \xi - \frac{h}{k} \right| \cdot \sup_{x \in [\xi - 1, \xi + 1]}|f'(x)|.
\end{align*}
Pick $A = (1 + \sup_{x \in [\xi - 1, \xi + 1]}|f'(x)|)^{-1} > 0$.
\end{enumerate}
By (a)(b), we arrange
$A = \min(1, (1 + \sup_{x \in [\xi - 1, \xi + 1]}|f'(x)|)^{-1}) > 0$
to fit the inequality.
\item[(2)]
\emph{$\xi = \sum_{n=0}^{\infty} 10^{-n!}$ is transcendental.}
\begin{enumerate}
\item[(a)]
Let $k_j = 10^{j!}$, $h_j = 10^{j!} \sum_{n=0}^{j} 10^{-n!}$.
Then
$$\left| \xi - \frac{h_j}{k_j} \right|
= \sum_{n=j+1}^{\infty} 10^{-n!}
< \sum_{n=(j+1)!}^{\infty} 10^{-n}
= \frac{A_j}{k_j^{j}}
$$
where $A_j = \frac{10}{9} \cdot 10^{-j!}$.
\item[(b)]
If $\xi$ were a real algebraic number of degree $d > 1$,
then by Lemma and (a),
$$\frac{A}{k_j^{d}}
< \left| \xi - \frac{h_j}{k_j} \right|
< \frac{A_j}{k_j^{j}}
< \frac{A_j}{k_j^{d}}$$
for some $A > 0$ and $j \geq d$,
or $0 < A < A_j$.
Since $j$ is arbitrary,
$A_j \rightarrow 0$ as $j \rightarrow \infty$,
contrary to $A > 0$.
\item[(c)]
If $\xi$ were a real algebraic number of degree $d = 1$,
$\xi = \frac{h}{k}$ is a rational number.
So
$$\left| \xi - \frac{h_j}{k_j} \right|
= \left| \frac{h}{k} - \frac{h_j}{k_j} \right|
= \left| \frac{h k_j - k h_j}{k k_j} \right|
\geq \left| \frac{1}{k k_j} \right|
= \frac{|k|^{-1}}{k_j}$$
for all $j$.
(It is impossible that $h k_j - k h_j = 0$ or $\frac{h}{k} = \frac{h_i}{k_j}$
since $| \frac{h}{k} - \frac{h_j}{k_j} | = \sum_{n=j+1}^{\infty} 10^{-n!} > 0$ for all $j$.)
Again by (a),
$$\frac{|k|^{-1}}{k_j}
\leq \left| \xi - \frac{h_j}{k_j} \right|
< \frac{A_j}{k_j^{j}}
< \frac{A_j}{k_j},$$
or $0 < |k|^{-1} < A_j$.
(Similar to (b).)
Since $j$ is arbitrary,
$A_j \rightarrow 0$ as $j \rightarrow \infty$,
contrary to $|k|^{-1} > 0$.
\end{enumerate}
\end{enumerate}
$\Box$ \\\\



%%%%%%%%%%%%%%%%%%%%%%%%%%%%%%%%%%%%%%%%%%%%%%%%%%%%%%%%%%%%%%%%%%%%%%%%%%%%%%%%



\textbf{Exercise 2.4.}
\emph{Is the set of all irrational real numbers countable?} \\

\emph{Proof (Reductio ad absurdum).}
If $\mathbb{R}-\mathbb{Q}$ were countable,
then $\mathbb{R} = \mathbb{Q} \bigcup (\mathbb{R}-\mathbb{Q})$ is countable
(Theorem 2.12),
contrary to the fact that
$\mathbb{R}$ is uncountable (Corollary to Theorem 2.43).
$\Box$ \\

\emph{Proof (Exercise 2.18).}
Exercise 2.18 provides
some examples of uncountable subset $E$ of irrational real numbers.
\begin{enumerate}
  \item[(1)]
  Let $A$ be the set of all $y \in [0,1]$
  whose decimal expansion contains only the digits $4$ and $7$.
  Let $\xi = \sum_{n=0}^{\infty} 10^{-n!}$ and $$E = \{ y + \xi : y \in A \}.$$
  \item[(2)]
  Let $E$ be a subset of Liouville numbers as
  $$E = \left\{ \sum_{n=0}^{\infty} \frac{a_n}{10^{n!}} : a_n \in \{4, 7\} \right\}.$$
  \item[(3)]
  Let
  $$E = \left\{ \sum_{n=1989}^{\infty} \frac{a_n}{n!} : a_n \in \{6, 4\} \right\}.$$
\end{enumerate}
We can apply the same argument of Theorem 2.14 to prove that each $E$ is uncountable.
Then use Theorem 2.8 to get all irrational real numbers cannot be countable.
$\Box$ \\\\



%%%%%%%%%%%%%%%%%%%%%%%%%%%%%%%%%%%%%%%%%%%%%%%%%%%%%%%%%%%%%%%%%%%%%%%%%%%%%%%%



\textbf{Exercise 2.5.}
\emph{Construct a bounded set of real numbers with exactly three limit points.} \\

\emph{Proof (Exercise 2.12).}
Let
$$K_{p} =
\{ p \}
\bigcup
\left\{ p + \frac{1}{n} : n \in \mathbb{Z}^+ \right\} \subseteq \mathbb{R}^1$$
be a compact set of $\mathbb{R}^1$ with exactly one limit point $p \in \mathbb{R}^1$
(Exercise 2.12).
Then
$$K_{1989} \cup K_{6} \cup K_{4}$$
is a compact set of $\mathbb{R}^1$ with exactly three limit points
$1989, 6, 4 \in \mathbb{R}^1$.
$\Box$ \\\\



%%%%%%%%%%%%%%%%%%%%%%%%%%%%%%%%%%%%%%%%%%%%%%%%%%%%%%%%%%%%%%%%%%%%%%%%%%%%%%%%



\textbf{Exercise 2.6.}
\emph{Let $E'$ be the set of all limit points of a set $E$.
Prove that $E'$ is closed.
Prove that $E$ and $\overline{E}$ have the same limit points.
(Recall that $\overline{E} = E \cup E'$.)
Do $E$ and $E'$ always have the same limit points?} \\

\emph{Proof.}
\begin{enumerate}
\item[(1)]
\emph{Show that $E'$ is closed.}
  \begin{enumerate}
  \item[(a)]
  \emph{Use Definition 2.18 (d).}
    \begin{enumerate}
    \item[(i)]
    It suffices to show every limit point of $E'$ is a limit point of $E$.
    Given a limit point $p$ of $E'$, so that every open neighborhood $U$ of $p$
    contains a point $q_0 \neq p$ such that $q_0 \in E'$.
    \item[(ii)]
    Since $q_0$ is a limit point of $E$,
    there is an open neighborhood $V$ of $q_0$ contains a point $q \neq q_0$
    such that $q \in E$, where
    $$V = U \cap B\left( q_0; \frac{1}{2}d(p,q_0) \right) \subseteq U$$
    ($B(x;r)$ is the open ball with center at $x$ and radius $r$).
    \item[(iii)]
    By the construction of $V$,
    for such open neighborhood $U$ of $p$,
    there is $q \neq p$ and $q \in V \subseteq U$ and $q \in E$.
    That is, $p$ is a limit point of $E$.
    \end{enumerate}
  \item[(b)]
  \emph{Use Definition 2.18 (e).}
    \begin{enumerate}
    \item[(i)]
    To show $E'$ is closed or $X-E'$ is open,
    it suffices to show every point of $X-E'$ is an interior point of $X-E'$.
    \item[(ii)]
    Given a point $p \in X-E'$, or $p$ is not a limit point of $E$.
    There is an open neighborhood $U$ of $p$ contains no point $q \neq p$ such that $q \in E$.
    \item[(iii)]
    To show $U$ is an open neighborhood of $p$ such that $U \subseteq X-E'$,
    it suffices to no point $q \neq p$ such that $q \in E'$.
    If there were a limit point $q$ of $E$ such that $q \neq p$ and $q \in U$,
    then
    $$V = U \cap B\left( q; \frac{1}{2}d(p,q) \right) \subseteq U$$
    is an open neighborhood of $q$ contains no point of $E$,
    contrary to the assumption $q \in E'$.
    So $U \subseteq X-E'$ is an open neighborhood of $p \in X-E'$.
    \end{enumerate}
  \end{enumerate}
\item[(2)]
\emph{Show that $E' = \overline{E}'$.}
It suffices to show $E' \supseteq \overline{E}'$.
($E' \subseteq \overline{E}'$ holds trivially since $E \subseteq \overline{E}$).
Given a limit point $p$ of $\overline{E} = E \cup E'$.
  \begin{enumerate}
  \item[(a)]
  $p$ is a limit point of $E$. Nothing to do.
  \item[(b)]
  $p$ is a limit point of $E'$.
  Since $p$ is a limit point of $E'$ and $E'$ is a closed set,
  $p \in E'$, or $p$ is a limit point of $E$.
  \end{enumerate}
In any case, $E' \supseteq \overline{E}'$.
\item[(3)]
\emph{$E$ and $E'$ might not have the same limit points.}
Let
$$E = \left\{ \frac{1}{n} : n \in \mathbb{Z}^+ \right\} \subseteq \mathbb{R}^1.$$
Then $E' = \{0\}$ and thus $(E')' = \varnothing$.
\end{enumerate}
$\Box$ \\\\



%%%%%%%%%%%%%%%%%%%%%%%%%%%%%%%%%%%%%%%%%%%%%%%%%%%%%%%%%%%%%%%%%%%%%%%%%%%%%%%%



\textbf{Exercise 2.7.}
\emph{Let $A_1, A_2, A_3, ...$ be subsets of a metric space.}
\begin{enumerate}
\item[(a)]
\emph{If $B_n = \bigcup^{n}_{i=1} A_i$, prove that
$\overline{B_n} = \bigcup^{n}_{i=1}{\overline{A_i}}$, for $n = 1, 2, 3, ...$}
\item[(b)]
\emph{If $B = \bigcup^{\infty}_{i=1} A_i$,
prove that $\overline{B} \supseteq \bigcup^{\infty}_{i=1} \overline{A_i}$.}
\end{enumerate}
\emph{Show, by an example, that this inclusion can be proper. } \\

\emph{Proof of (a).}
\begin{enumerate}
  \item[(1)]
  \emph{Show that $\overline{B_n} \subseteq \bigcup^{n}_{i=1}{\overline{A_i}}$.}
  Since $A_i \subseteq \overline{A_i}$ for any $i$, we have
  $$B_n = \bigcup^{n}_{i=1} A_i \subseteq \bigcup^{n}_{i=1} \overline{A_i}.$$
  Since $\bigcup^{n}_{i=1} \overline{A_i}$ is a union of finitely many
  closed set $\overline{A_i}$, $\bigcup^{n}_{i=1} \overline{A_i}$ is closed
  (Theorem 2.24(d)).
  By Theorem 2.27(c), $\overline{B_n} \subseteq \bigcup^{n}_{i=1} \overline{A_i}$.
  \item[(2)]
  \emph{Show that $\overline{B_n} \supseteq \bigcup^{n}_{i=1}{\overline{A_i}}$.}
  Same argument in the proof of (b).
\end{enumerate}
$\Box$\\

\emph{Proof of (b).}
Since $\bigcup^{\infty}_{j=1} A_j \supseteq A_i$ for any $i$,
by the monotonicity of closure, we have
$\overline{\bigcup^{\infty}_{j=1} A_j} \supseteq \overline{A_i}$ for any $i$,
or $\overline{B} \supseteq \bigcup^{\infty}_{i=1} \overline{A_i}$.
$\Box$\\

\emph{Proof of proper inclusion in (b).}
Let
$$A_{n} = \left( \frac{1}{n}, \infty \right) \subseteq \mathbb{R}^1$$
for any $n \in \mathbb{Z}^+$.
Then
\begin{align*}
\bigcup^{\infty}_{n=1} A_n = (0, \infty)
&\Longrightarrow
\overline{\bigcup^{\infty}_{n=1} A_n} = \overline{(0, \infty)} = [0, \infty), \\
\overline{A_n} = \left[ \frac{1}{n}, \infty \right)
&\Longrightarrow
\bigcup^{\infty}_{n=1} \overline{A_n}
= \bigcup^{\infty}_{n=1} \left[ \frac{1}{n}, \infty \right)
= (0, \infty).
\end{align*}
$\Box$ \\\\



%%%%%%%%%%%%%%%%%%%%%%%%%%%%%%%%%%%%%%%%%%%%%%%%%%%%%%%%%%%%%%%%%%%%%%%%%%%%%%%%



\textbf{Exercise 2.8.}
\emph{Is every point of every open set $E \subseteq \mathbb{R}^2$ a limit point of $E$?
Answer the same question for closed sets in $\mathbb{R}^2$. } \\

It is not true for all metric spaces $X$.
The (discrete) metric in Exercise 2.10 implies no limit point exists in $X$. \\

\emph{Proof.}
\begin{enumerate}
\item[(1)]
\emph{Show that for every open set $E \subseteq \mathbb{R}^k$, $E \subseteq E'$.}
Given any point $\mathbf{p} \in E$, we shall show $\mathbf{p}$ is a limit point of $E$.
  \begin{enumerate}
  \item[(a)]
  Since $E$ is open, there is an open neighborhood $B(\mathbf{p};r_0) \subseteq E$
  for some $r_0 > 0$.
  \item[(b)]
  \emph{In particular, given any $s \in \mathbb{R}$ such that $0 < s < r_0$, we can find
  $$\mathbf{q} \in B(\mathbf{p};s) \subseteq B(\mathbf{p};r_0) \subseteq E$$
  such that $\mathbf{q} \neq \mathbf{p}$.}
  Explicitly, write $$\mathbf{p} = (p_1, \ldots, p_k)$$ and
  choose
  $$\mathbf{q} = \left(p_1 + \frac{s}{89}, p_2, \ldots, p_k\right) \neq \mathbf{p}$$
  (since $s > 0$).
  Clearly, $\mathbf{q}$ is well-defined in $\mathbb{R}^k$ and
  $|\mathbf{q} - \mathbf{p}| = \frac{s}{89} < s$ or $\mathbf{q} \in B(\mathbf{p};s)$.
  \item[(c)]
  Now given every open neighborhood $B(\mathbf{p}, r)$ of $\mathbf{p}$.
  We can choose $s \in \mathbb{R}$ such that $0 < s < \min\{r_0,r\} \leq r_0$.
  (might pick $s = \frac{1}{64}\min\{r_0,r\}$.)
  By (b), there exists $\mathbf{q} \neq \mathbf{p}$ such that
  $$\mathbf{q} \in B(\mathbf{p};s) \subseteq B(\mathbf{p};r) \subseteq E.$$
  \end{enumerate}
\item[(2)]
\emph{Give an example of a closed set $E \subseteq \mathbb{R}^k$ such that $E \not\subseteq E'$.}
Pick $E = \{ \mathbf{0} \}$.
So $E' = \varnothing$ and thus $E \not\subseteq E'$.
\end{enumerate}
$\Box$ \\\\



%%%%%%%%%%%%%%%%%%%%%%%%%%%%%%%%%%%%%%%%%%%%%%%%%%%%%%%%%%%%%%%%%%%%%%%%%%%%%%%%



\textbf{Exercise 2.9.}
\emph{Let $E^\circ$ denote the set of all interior points of a set $E$.
[See Definition 2.18(e); $E^\circ$ is called the interior of $E$.]}
\begin{enumerate}
\item[(a)]
\emph{Prove that $E^\circ$ is always open.}
\item[(b)]
\emph{Prove that $E$ is open if and only if $E^\circ = E$.}
\item[(c)]
\emph{If $G$ is contained in $E$ and $G$ is open,
prove that $G$ is contained in $E^\circ$.}
\item[(d)]
\emph{Prove that the complement of $E^\circ$ is the closure of the
complement of $E$.}
\item[(e)]
\emph{Do $E$ and $\overline{E}$ always have the same interiors?}
\item[(f)]
\emph{Do $E$ and $E^\circ$ always have the same closures?} \\
\end{enumerate}

Similar to Theorem 2.27. \\

\emph{Proof of (a).}
\emph{It is equivalent to show that
$E^{\circ} \subseteq (E^{\circ})^{\circ}.$}
\begin{enumerate}
\item[(1)]
Given any point $x \in E^{\circ}$, there is $r > 0$ such that $B(x;r) \subseteq E$.
\item[(2)]
It suffices to show that $B\left(x;\frac{2}{r}\right) \subseteq E^{\circ}$.
Given any point $y \in B\left(x;\frac{2}{r}\right)$,
we will show that there is an open neighborhood $B\left(y;\frac{2}{r}\right)$ of $y$
such that $B\left(y;\frac{2}{r}\right) \subseteq E$.
\item[(3)]
Given any point $z \in B\left(y;\frac{2}{r}\right)$, we have
$$d(z,x) \leq d(z,y) + d(y,x) < \frac{2}{r} + \frac{2}{r} = r,$$
or $z \in B(x;r) \subseteq E$.
Therefore, $B\left(y;\frac{2}{r}\right) \subseteq E$,
or $y \in E^{\circ}$,
or $B\left(x;\frac{2}{r}\right) \subseteq E^{\circ}$,
or $x \in (E^{\circ})^{\circ}$,
or $E^{\circ} \subseteq (E^{\circ})^{\circ}$.
\end{enumerate}
$\Box$ \\

\emph{Proof of (b).}
\begin{enumerate}
\item[(1)]
($\Longrightarrow$)(Definition 2.18)
Since $E$ is open, every point of $E$ is an interior point of $E$.
Hence $E \subseteq E^{\circ}$.
Note that $E^{\circ} \subseteq E$ is trivial, and thus $E^\circ = E$.
\item[(2)]
($\Longleftarrow$)((a))
By (a), $E = E^{\circ}$ is always open.
\item[(3)]
($\Longleftarrow$)(Definition 2.18)
Every point of $E$ is an interior point of $E$ since $E = E^\circ$.
Hence $E$ is open by Definition 2.18(f).
\end{enumerate}
$\Box$ \\

\emph{Proof of (c).}
$G \subseteq E$ implies $G^{\circ} \subseteq E^{\circ}$.
$G = G^{\circ}$ since $G$ is open ((b)).
Hence $G = G^{\circ} \subseteq E^{\circ}$, that is,
$E^{\circ}$ is the largest open set contained in $E$.
(Similarly, $\overline{E}$ is the smallest closed set containing $E$.)
$\Box$ \\

\emph{Proof of (d).}
\emph{Show that
$X - E^{\circ} = \overline{X - E}$ and
$(X-E)^{\circ} = X - \overline{E}.$}
\begin{enumerate}
\item[(1)]
(Theorem 2.27 and (c))
\begin{align*}
X - E^{\circ}
&= X - \bigcup_{\text{Open } V \subseteq E} V \\
&= \bigcap_{\text{Open } V \subseteq E} (X - V) \\
&= \bigcap_{\text{Closed } W \supseteq X-E} W \\
&= \overline{X - E}. \\
X - \overline{E}
&= X - \bigcap_{\text{Closed } W \supseteq E} W \\
&= \bigcup_{\text{Closed } W \supseteq E} (X - W) \\
&= \bigcup_{\text{Open } V \subseteq X-E} V \\
&= (X - E)^{\circ}.
\end{align*}
\item[(2)]
(Brute-force)
\begin{align*}
x \in E^{\circ}
\Longleftrightarrow&
\exists r > 0 \text{ such that } B(x;r) \subseteq E \\
\Longleftrightarrow&
\exists r > 0 \text{ such that } B(x;r) \cap (X-E) = \varnothing \\
\Longleftrightarrow&
x \not\in \overline{X-E} \\
\Longleftrightarrow&
x \in X - \overline{X-E}. \\
x \in (X-E)^{\circ}
\Longleftrightarrow&
\exists r > 0 \text{ such that } B(x;r) \subseteq (X-E) \\
\Longleftrightarrow&
\exists r > 0 \text{ such that } B(x;r) \cap E = \varnothing \\
\Longleftrightarrow&
x \not\in \overline{E} \\
\Longleftrightarrow&
x \in X - \overline{E}.
\end{align*}
\end{enumerate}
Note that $X - E^{\circ} = \overline{X - E}$ is equivalent to
$(X-E)^{\circ} = X - \overline{E}$ by mapping $E \mapsto X-E$.
$\Box$ \\

\emph{Proof of (e).}
No.
\begin{enumerate}
\item[(1)]
Let $X = \mathbb{R}$ equipped with the Euclidean metric, and $E = \mathbb{Q} \subseteq X$.
\item[(2)]
$E^\circ = \varnothing$ since $\widetilde{\mathbb{Q}}$ is dense in $\mathbb{R}$.
\item[(3)]
$(\overline{E})^{\circ} = (\mathbb{R})^{\circ} = \mathbb{R}$
since $\mathbb{Q}$ is dense in $\mathbb{R}$ and $\mathbb{R}$ is open.
\end{enumerate}
$\Box$ \\

\emph{Proof of (f).}
No.
\begin{enumerate}
\item[(1)]
Let $X = \mathbb{R}$ equipped with the Euclidean metric, and $E = \mathbb{Q} \subseteq X$.
\item[(2)]
$\overline{E} = \mathbb{R}$ since $\mathbb{Q}$ is dense in $\mathbb{R}$.
\item[(3)]
$\overline{E^\circ} = \overline{\varnothing} = \varnothing$
since $\widetilde{\mathbb{Q}}$ is dense in $\mathbb{R}$.
\end{enumerate}
$\Box$ \\\\



%%%%%%%%%%%%%%%%%%%%%%%%%%%%%%%%%%%%%%%%%%%%%%%%%%%%%%%%%%%%%%%%%%%%%%%%%%%%%%%%



\textbf{Exercise 2.10.}
\emph{Let $X$ be an infinite set. For $p \in X$ and $q \in X$, define
\begin{equation*}
  d(p, q) =
    \begin{cases}
      1 & (\text{if } p \neq q) \\
      0 & (\text{if } p = q).
    \end{cases}
\end{equation*}
Prove that this is a metric.
Which subsets of the resulting metric space are open?
Which are closed?
Which are compact?} \\

(The statement holds even if $X$ is finite.)
We called $d$ the discrete metric,
and the corresponding topology on $X$ induces the discrete topology.
Conversely, if $X$ has the discrete topology, $X$ is always metrizable by the discrete metric. \\

\emph{Proof.}
\begin{enumerate}
\item[(1)] \emph{$d(p, q)$ is a metric.}
\begin{enumerate}
\item[(a)] \emph{$d(p, q) > 0$ if $p \neq q$; $d(p, p) = 0$.} Trivial.
\item[(b)] \emph{$d(p, q) = d(q, p)$.} Trivial.
\item[(c)] \emph{$d(p, q) \leq d(p, r) + d(r, q)$ for any $r \in X$.}
If $p = q$, nothing to do. If $p \neq q$, $r \neq p$ or $r \neq q$ for any $r \in X$.
(Assume not true, $r = p$ and $r = q$ implies that $p = q$ which is a contradiction.)
In any case $d(p, r) + d(r, q) \geq 1 = d(p, q)$.
\end{enumerate}
\item[(2)] \emph{Every subset is open.}
Let $E$ be any subset of $X$.
Then every point $p \in E$ is an interior point of $E$.
In fact, we can pick one open neighborhood $U = B\left(p;\frac{1}{2}\right)$ of $p$
containing only one point $p \in E$ or $U = \{ p \}$,
and such open neighborhood $U$ is a subset of $E$.
So every subset of $X$ is open.
\item[(3)] \emph{Every subset is closed.}
Since every subset is open, every subset is closed by Theorem 2.23. \\

\textbf{Supplement.}
Might use Definition 2.18 (d) to prove directly since there are no limit points in $X$
if we consider one open neighborhood $U = B\left(p;\frac{1}{2}\right)$ of $p$.
Therefore, every subset is closed.
Again we apply Theorem 2.23 to get that every subset is open
without using Definition 2.18 (f).
\item[(4)] \emph{A subset is compact iff it is finite.}
\begin{enumerate}
\item[(a)]
\emph{Any finite subset is compact.}
Say $E = \{p_1, p_2, ..., p_k\}$, and $\{ G_{\alpha} \}$ be an open covering of $E$.
From $\{ G_{\alpha} \}$
we pick
$G_{\alpha_1}$ containing $p_1$,
$G_{\alpha_2}$ containing $p_2$, ..., and
$G_{\alpha_k}$ containing $p_k$.
This process can be done in the finitely many steps.
Therefore,
$$\{ G_{\alpha_1}, G_{\alpha_2}, ..., G_{\alpha_k} \}$$
is a finite subcovering of $\{ G_{\alpha} \}$ covering $E$.
\item[(b)]
\emph{Any infinite subset is not compact.}
Take a collection
$$\mathscr{G} = \left\{ G_p = \{ p \} \right\}$$
of open subsets where $p$ runs all points in $E$.
Clearly, $\{ G_p \}$ is an open covering.
Assume
$$\mathscr{G}' = \left\{ G_{p_1}, G_{p_2}, ..., G_{p_k} \right\}$$
is any finite subcovering of $\mathscr{G}$.
Since $E$ is infinite, there exist a point $p \in E$ such that
$p \neq p_1$,
$p \neq p_2$, ...,
$p \neq p_k$.
Therefore, $\mathscr{G}'$ does not cover $p$,
or $\mathscr{G}$ does not contains any finite subcovering $\mathscr{G}'$.
\end{enumerate}
\end{enumerate}
$\Box$ \\

Notice that every subset is bounded.
Therefore, every subset is closed and bounded,
but only finite subset is compact, i.e.,
Heine-Borel theorem is not true in the infinite discrete topology. \\\\



%%%%%%%%%%%%%%%%%%%%%%%%%%%%%%%%%%%%%%%%%%%%%%%%%%%%%%%%%%%%%%%%%%%%%%%%%%%%%%%%



\textbf{Exercise 2.11.}
\emph{For $x \in \mathbb{R}^1$ and $y \in \mathbb{R}^1$, define
\begin{align*}
  d_1(x,y) &= (x-y)^2, \\
  d_2(x,y) &= \sqrt{|x-y|}, \\
  d_3(x,y) &= |x^2 - y^2|, \\
  d_4(x,y) &= |x - 2y|, \\
  d_5(x,y) &= \frac{|x-y|}{1+|x-y|}.
\end{align*}
Determine, for each of these, whether it is a metric or not.} \\

\emph{Proof.}
\begin{enumerate}
\item[(1)]
\emph{$d = d_1$ is not a metric.}
(Reductio ad absurdum) If $d$ were a metric,
then $$d(0,2) > d(0,1)+d(1,2),$$
contrary to Definition 2.15(c) that $d(p,q) \leq d(p,r)+d(r,q)$.
\item[(2)]
\emph{$d = d_2$ is a metric.}
\emph{It suffices to show that
$d'(x,y) = \sqrt{d(x,y)}$ is a metric if $d(x,y)$ is a metric.}
  For any $p, q, r \in \mathbb{R}^1$,
  \begin{enumerate}
  \item[(a)]
  $d'(p,q) = \sqrt{d(p,q)} > 0$ if $p \neq q$;
  $d'(p,p) = \sqrt{d(p,p)} = 0$.
  \item[(b)]
  $d'(p,q) = \sqrt{d(p,q)} = \sqrt{d(q,p)} = d'(q,p)$.
  \item[(c)]
    \begin{align*}
    &\sqrt{d(p,r)+d(r,q)}
    \leq \sqrt{d(p,r)}+ \sqrt{d(r,q)} \\
    \Longleftrightarrow&
    (\sqrt{d(p,r)+d(r,q)})^2
    \leq (\sqrt{d(p,r)}+ \sqrt{d(r,q)})^2 \\
    \Longleftrightarrow&
    d(p,r)+d(r,q)
    \leq d(p,r)+d(r,q) + 2\sqrt{d(p,r)}\sqrt{d(r,q)} \\
    \Longleftrightarrow&
    0
    \leq 2\sqrt{d(p,r)}\sqrt{d(r,q)}.
    \end{align*}
  \item[(d)]
    \begin{align*}
    d'(p,q) &= \sqrt{d(p,q)} \\
    &\leq \sqrt{d(p,r)+d(r,q)}
      &\text{(Triangle inequality)} \\
    &\leq \sqrt{d(p,r)}+ \sqrt{d(r,q)}
      &\text{((c))} \\
    &= d'(p,r) + d'(r,q).
    \end{align*}
  \end{enumerate}
  By Definition 2.15, $d'$ is a metric.
\item[(3)]
\emph{$d = d_3$ is not a metric.}
(Reductio ad absurdum) If $d$ were a metric,
then $$d(1,-1) = 0,$$
contrary to Definition 2.15(a): $d(p,q) > 0$ if $p \neq q$; $d(p,p) = 0$.
\item[(4)]
\emph{$d = d_4$ is not a metric.}
(Reductio ad absurdum) If $d$ were a metric,
then $$d(1,1) = 1,$$
contrary to Definition 2.15(a): $d(p,q) > 0$ if $p \neq q$; $d(p,p) = 0$.
\item[(5)]
\emph{$d = d_5$ is a metric.}
\emph{It suffices to show that
$d'(x,y) = \frac{d(x,y)}{1+d(x,y)}$ is a metric if $d(x,y)$ is a metric.}
  For any $p, q, r \in \mathbb{R}^1$,
  \begin{enumerate}
  \item[(a)]
  $d'(p,q) = \frac{d(p,q)}{1+d(p,q)} > 0$ if $p \neq q$;
  $d'(p,p) = \frac{d(p,p)}{1+d(p,p)} = 0$.
  \item[(b)]
  $d'(p,q) = \frac{d(p,q)}{1+d(p,q)}
  = \frac{d(q,p)}{1+d(q,p)} = d'(q,p)$.
  \item[(c)]
  Write $x = d(p,q)$, $y = d(p,r)$ and $z = d(r,q)$.
  So $x, y, z \geq 0$ and
    \begin{align*}
    &x \leq y + z \\
    \Longleftrightarrow&
    x + x(y+z) \leq y+z + x(y+z) \\
    \Longleftrightarrow&
    x(1+y+z) \leq (1+x)(y+z) \\
    \Longleftrightarrow&
    \frac{x}{1+x} \leq \frac{y+z}{1+y+z}.
    \end{align*}
  \item[(d)]
    \begin{align*}
    d'(p,q) &= \frac{d(p,q)}{1+d(p,q)} \\
    &\leq \frac{d(p,r)+d(r,q)}{1+d(p,r)+d(r,q)}
      &\text{((c))} \\
    &= \frac{d(p,r)}{1+d(p,r)+d(r,q)} + \frac{d(r,q)}{1+d(p,r)+d(r,q)} \\
    &= \frac{d(p,r)}{1+d(p,r)} + \frac{d(r,q)}{1+d(r,q)} \\
    &= d'(p,r) + d'(r,q).
    \end{align*}
  \item[(e)]
  Or we can show $d'(p,q) \leq d'(p,r) + d'(r,q)$ by
    \begin{align*}
    &\frac{x}{1+x} \leq \frac{y}{1+y} + \frac{z}{1+z} \\
    \Longleftrightarrow&
    x(1+y)(1+z) \leq y(1+z)(1+x) + z(1+x)(1+y) \\
    \Longleftrightarrow&
    x+xy+xz+xyz \\
    &\leq (y+xy+yz+xyz) + (z+xz+yz+xyz) \\
    \Longleftrightarrow&
    x \leq y+z+2yz+xyz \\
    \Longleftarrow&
    x \leq y+z
      &\text{($d$ is nonnegative)}
    \end{align*}
  \end{enumerate}
  By Definition 2.15, $d'$ is a metric.
\end{enumerate}
$\Box$ \\\\



%%%%%%%%%%%%%%%%%%%%%%%%%%%%%%%%%%%%%%%%%%%%%%%%%%%%%%%%%%%%%%%%%%%%%%%%%%%%%%%%



\textbf{Exercise 2.12.}
\emph{Let $K \subseteq \mathbb{R}^1$
consist of $0$ and the numbers $\frac{1}{n}$, for $n = 1, 2, 3, ...$.
Prove that $K$ is compact directly from the definition
(without using the Heine-Borel theorem).} \\

\emph{Proof.}
Let $\{ G_{\alpha} \}$ be an open covering of $K$.
There is an open set $G_0 \in \{ G_{\alpha} \}$ containing $0$.
So there exists an open neighborhood $U = B(0;r)$ of $0$ such that $U \subseteq G_0$.
So $U$ contains all points $q = \frac{1}{n}$ of $K$ whenever $n > \frac{1}{r}$.
To construct a finite subcovering of $\{ G_{\alpha} \}$,
we need to pick finitely many open sets from $\{ G_{\alpha} \}$
to cover the remaining points $q = \frac{1}{n}$ where $n = 1, 2, ..., \left[\frac{1}{r}\right]$,
say $G_1$ contains $q = \frac{1}{1}$, $G_2$ contains $q = \frac{1}{2}$, ...,
$G_{\left[\frac{1}{r}\right]}$ contains $q = \frac{1}{\left[\frac{1}{r}\right]}$.
(Might be duplicated.)
Hence,
$$\left\{ G_0, G_1, G_2, ..., G_{\left[\frac{1}{r}\right]} \right\}$$
is a finite subcovering of $\{ G_{\alpha} \}$ covering $K$.
$\Box$ \\

\emph{Proof (Heine-Borel theorem).}
\begin{enumerate}
\item[(1)]
$K$ is closed. In fact, the only limit point of $K$ is $0$, which is in $K$.
\begin{enumerate}
\item[(a)]
\emph{$p = 0$ is a limit point.}
Given $r > 0$.
There always exists $n \in \mathbb{Z}^{+}$ such that $r > \frac{1}{n}$.
So any open neighborhood $B(0;r)$ of $p = 0$
contains at least one point $q = \frac{1}{n} \neq 0$ in $K$.
\item[(b)]
\emph{$p < 0$ is not a limit point.}
Pick an open neighborhood $B(p;r)$ of $p$ where $r = |p| > 0$.
Then $B(p;r) \cap K = \varnothing$.
\item[(c)]
\emph{$p > 0$ is not a limit point.}
There always exists $m \in \mathbb{Z}^+$ such that $p > \frac{1}{n}$ whenever $n \geq m$.
Pick an open neighborhood $B(p;r)$ of $p$ where $r = p - \frac{1}{m} > 0$.
Then $B(p;r)$ does not have all points $q = \frac{1}{n} \in K$ whenever $n \geq m$.
By Theorem 2.20, $p$ cannot be a limit point of $K$.
\end{enumerate}
\item[(2)]
$K$ is bounded. There is a real number $M = 2$ and a point $q = 0 \in \mathbb{R}^1$
such that $|p - q| = |p| < 2$ for all $p \in K$.
\end{enumerate}
By Heine-Borel theorem, $K$ is compact in $\mathbb{R}^1$.
$\Box$ \\\\



%%%%%%%%%%%%%%%%%%%%%%%%%%%%%%%%%%%%%%%%%%%%%%%%%%%%%%%%%%%%%%%%%%%%%%%%%%%%%%%%



\textbf{Exercise 2.13.}
\emph{Construct a compact set of real numbers whose limit points form a countable set.} \\

\emph{Proof (Exercise 2.12).}
Let $K(p;r) \subseteq \mathbb{R}^1$ be
$$K(p;r) = \left\{ p + \frac{r}{n} : n = 2, 3, \ldots \right\} \bigcup \{p\}$$
and
$$K
= \left(\bigcup_{i = 0}^{\infty} K(2^{-i};2^{-i})\right) \bigcup \{0\}.$$
\begin{enumerate}
\item[(1)]
\emph{The set of limit points of $K$ is $K' = \{ 2^{-i} : i = 0, 1, 2, \ldots \} \bigcup \{0\}$,
which is (infinitely) countable.}
  \begin{enumerate}
  \item[(a)]
  The unique limit point of $K(2^{-i};2^{-i})$ is $2^{-i}$ for each $i = 0, 1, 2, \ldots$
  (Exercise 2.12).
  \item[(b)]
  $0$ is a limit point of $K$.
  \item[(c)]
  No other limit points of $K$.
  Similar to the argument of the proof of Exercise 2.12.
  \end{enumerate}
\item[(2)]
$K$ is closed. All limit points are in $K$.
\item[(3)]
$K$ is bounded. There is a real number $M = 2$ and a point $q = 0 \in \mathbb{R}^1$
such that $|p - q| = |p| < 2$ for all $p \in K$.
\end{enumerate}
By Heine-Borel theorem, $K$ is compact in $\mathbb{R}^1$,
and has infinitely countable limit points.
$\Box$ \\\\



%%%%%%%%%%%%%%%%%%%%%%%%%%%%%%%%%%%%%%%%%%%%%%%%%%%%%%%%%%%%%%%%%%%%%%%%%%%%%%%%



\textbf{Exercise 2.14.}
\emph{Give an example of an open cover of the segment $(0, 1)$
which has no finite subcover.} \\

\emph{Proof.}
In $\mathbb{R}^1$, take a collection
$$\mathscr{G} = \left\{ G_n = \left(\frac{1}{n}, 1\right) \right\}$$
of open subsets where $n \in \mathbb{Z}^+$.
\begin{enumerate}
\item[(1)]
\emph{$\mathscr{G}$ is an open covering of $(0, 1) \subseteq \mathbb{R}^1$.}
Actually, given $x \in (0, 1)$, there exists an positive integer $n$ such that $x > \frac{1}{n}$.
That is, $x \in \left(\frac{1}{n}, 1\right) = G_n$.
\item[(2)]
\emph{There is no finite subcovering of $\mathscr{G}$.}
Assume $$\mathscr{G}' = \left\{ G_{n_1}, G_{n_2}, ..., G_{n_k} \right\}$$
is any finite subcovering of $\mathscr{G}$ where $n_1 < n_2 < ... < n_k$.
Take $x \in \left(0, \frac{1}{n_k}\right) \neq \varnothing$,
$x = \frac{1}{2 n_k}$ for example.
Then $x \not\in G_{n_1}$, $x \not\in G_{n_1}$, ..., $x \not\in G_{n_k}$,
which contradicts that $\mathscr{G}'$ is a finite subcovering of $\mathscr{G}$ covering $(0, 1)$.
\end{enumerate}
$\Box$ \\\\



%%%%%%%%%%%%%%%%%%%%%%%%%%%%%%%%%%%%%%%%%%%%%%%%%%%%%%%%%%%%%%%%%%%%%%%%%%%%%%%%



\textbf{Exercise 2.15.}
\emph{
Show that Theorem 2.36 and its Corollary become false
(in $\mathbb{R}^1$, for example) if the word ``compact''
is replaced by ``closed'' or by ``bounded.''} \\

\emph{Recall:}
\begin{enumerate}
\item[(1)]
Theorem 2.36:
\emph{If $\{K_\alpha\}$ is a collection of compact subsets of a metric space $X$
such that the intersection of every finite subcollection of $\{K_\alpha\}$ is nonempty,
then $\bigcap K_\alpha$ is nonempty.}
\item[(2)]
Corollary:
\emph{If $\{K_n\}$ is a sequence of nonempty compact sets such that $K_n$
contains $K_{n+1}$ $(n=1,2,3,\ldots)$, then $\bigcap K_n$ is not empty.} \\
\end{enumerate}

\emph{Proof.}
Let $X = \mathbb{R}^1$ with the usual Euclidean metric.
\begin{enumerate}
\item[(1)]
For the closeness, let $K_n = [n,\infty) \subseteq X$.
\item[(2)]
For the boundedness, let $K_n = \left(0, \frac{1}{n}\right) \subseteq X$.
\end{enumerate}
In any case, $K_1 \supseteq K_2 \supseteq \cdots$ and $\bigcap K_n = \varnothing$.
$\Box$ \\\\



%%%%%%%%%%%%%%%%%%%%%%%%%%%%%%%%%%%%%%%%%%%%%%%%%%%%%%%%%%%%%%%%%%%%%%%%%%%%%%%%



\textbf{Exercise 2.16.}
\emph{Regard $\mathbb{Q}$, the set of all rational numbers, as a metric space,
with $d(p,q) = |p-q|$.
Let $E$ be the set of all $p \in \mathbb{Q}$ such that $2 < p^2 < 3$.
Show that $E$ is closed and bounded in $Q$,
but that $E$ is not compact.
Is $E$ open in $\mathbb{Q}$?} \\

\textbf{Lemma.}
\emph{Assume $S \subseteq T \subseteq M$.
Then $S$ is compact in $(M,d)$ if, and only if,
$S$ is compact in the metric subspace $(T,d)$. } \\

\emph{Proof of Lemma.}
\begin{enumerate}
\item[(1)]
$(\Longrightarrow)$
Let $\mathscr{F}$ be an open covering of $S$ in $(T,d)$, say
$S \subseteq \bigcup_{A \in \mathscr{F}} A$ where $A$ is open in $T$.
Then $A = B \cap T$ for some open set $B$ in $M$ (Theorem 3.33).
Let $\mathscr{G}$ be the collection of $B$.
Then
$$S \subseteq
\bigcup_{A \in \mathscr{F}} A
= \bigcup_{B \in \mathscr{G}} (B \cap T)
\subseteq \bigcup_{B \in \mathscr{G}} B,$$
or $\mathscr{G}$ be an open covering of $S$ in $(M,d)$.
Since $S$ is compact in $(M,d)$,
$\mathscr{G}$ contains a finite subcovering, say
$$S \subseteq B_1 \cap \cdots \cap B_p.$$
So $$S \cap T \subseteq (B_1\cap T) \cap \cdots \cap (B_p \cap T),$$
or $$S \subseteq A_1 \cap \cdots \cap A_p$$ (since $S \subseteq T$ or $S \cap T = S$).
So there is a finite subcovering of $\mathscr{F}$ covering $S$,
or $S$ is compact in $(T,d)$.
\item[(2)]
$(\Longleftarrow)$
Let $\mathscr{G}$ be an open covering of $S$ in $(M,d)$, say
$S \subseteq \bigcup_{B \in \mathscr{G}} B$ where $B$ is open in $M$.
Then $A = B \cap T$ is open in $T$.
Let $\mathscr{F}$ be the collection of $A$.
Then
$$S \cap T \subseteq
\bigcup_{B \in \mathscr{G}} (B \cap T)
= \bigcup_{A \in \mathscr{F}} A,$$
or $\mathscr{F}$ be an open covering of $S \cap T = S$ in $(T,d)$.
Since $S$ is compact in $(T,d)$,
$\mathscr{F}$ contains a finite subcovering, say
$$S \subseteq A_1 \cap \cdots \cap A_p.$$
Clearly, $S \subseteq B_1 \cap \cdots \cap B_p$ since $A = B \cap T \subseteq B$.
So there is a finite subcovering of $\mathscr{G}$ covering $S$,
or $S$ is compact in $(M,d)$.
\end{enumerate}
$\Box$ \\

\emph{Proof.}
Write $E_0 = (\sqrt{2},\sqrt{3}) \bigcup (-\sqrt{3},-\sqrt{2})$,
and $E = E_0 \bigcap \mathbb{Q}$.
\begin{enumerate}
\item[(1)]
$E$ is a subset of $\mathbb{Q}$.
\item[(2)]
\emph{Show that $E$ is bounded in $\mathbb{Q}$.}
Since $\mathbb{Q}$ is dense in $\mathbb{R}$,
there is $p \in \mathbb{Q}$ such that $\sqrt{2} < p < \sqrt{3}$, or $p \in E$.
Let $r = p + \sqrt{3} > 0$.
Therefore, $E \subseteq B(p;r)$ for some $r > 0$ and $p \in E$,
or $E$ is bounded.
\item[(3)]
\emph{Show that $E$ is closed in $\mathbb{Q}$.}
It suffices to show its complement is open in $\mathbb{Q}$.
Given any
$$p \in \widetilde{E}
= ((-\infty, -\sqrt{3}] \cup [-\sqrt{2},\sqrt{2}] \cup [\sqrt{3}, \infty)) \cap \mathbb{Q}.$$
$p \leq -\sqrt{3}$ or $-\sqrt{2} \leq p \leq \sqrt{2}$ or $p \geq \sqrt{3}$.
  \begin{enumerate}
  \item[(a)]
  $p \leq -\sqrt{3}$. $p \neq -\sqrt{3}$ since $p \in \mathbb{Q}$ and $-\sqrt{3}$ is irrational.
  So $p < -\sqrt{3}$ and thus there exists $q \in \mathbb{Q}$ such that $p < q < -\sqrt{3}$
  since $\mathbb{Q}$ is dense in $\mathbb{R}$.
  Let $r = \max\{-\sqrt{3} - q, q - p\} > 0$.
  The ball $B(q;r)$ is contained in $\widetilde{E}$.
  \item[(b)]
  $-\sqrt{2} \leq p \leq \sqrt{2}$. Similar to (a).
  \item[(c)]
  $p \geq \sqrt{3}$. Similar to (a).
  \end{enumerate}
  By (a)(b), $\widetilde{E}$ is open in $\mathbb{Q}$, or $E$ is closed in $\mathbb{Q}$.
\item[(4)]
\emph{Show that $E$ is not compact in $\mathbb{Q}$.}
(Reductio ad absurdum)
If $E_0$ were compact in the metric space $\mathbb{Q}$,
$E_0$ is compact in the metric space $\mathbb{R}$ (Lemma),
which is absurd.
\item[(5)]
\emph{Show that $E$ is open.}
Similar to (3).
\end{enumerate}
$\Box$ \\\\


%%%%%%%%%%%%%%%%%%%%%%%%%%%%%%%%%%%%%%%%%%%%%%%%%%%%%%%%%%%%%%%%%%%%%%%%%%%%%%%%



\textbf{Exercise 2.17.}
\emph{Let $E$ be the set of all $x \in [0,1]$
whose decimal expansion contains only the digits $4$ and $7$.
Is $E$ countable? Is $E$ dense in $[0,1]$? Is $E$ compact? Is $E$ perfect?} \\

\emph{Proof.}
\begin{enumerate}
\item[(1)]
\emph{Show that $E$ is uncountable.}
Same as Theorem 2.14.
Or show that $E$ is perfect and then apply Theorem 2.43.
\item[(2)]
\emph{Show that $E$ is not dense in $[0,1]$.}
Note that $E \subseteq \left[\frac{4}{9}, \frac{7}{9}\right]$.
So $$B\left(0;\frac{1}{64}\right) \bigcap E
\subseteq B\left(0;\frac{1}{64}\right) \bigcap \left[\frac{4}{9}, \frac{7}{9}\right]
= \varnothing$$ or $0$ is not a limit point of $E$.
Hence $E$ is not dense in $[0,1]$.
\item[(3)]
\emph{Show that $E$ is compact.}
\emph{It is equivalent to show that $E$ is closed and bounded (Theorem 2.41).}
  Let a decimal expansion of $x \in (0,1)$ be $0.x_1 x_2 \cdots$.
  \begin{enumerate}
  \item[(a)]
  \emph{Show that $\widetilde{E}$ is open.}
  Since $E \subseteq \left[\frac{4}{9}, \frac{7}{9}\right]$,
  it suffices to show that every point $x \in (0,1) \cap \widetilde{E}$ is
  an interior point of $\widetilde{E}$.
  Say a decimal expansion of $x$ containing at least one digit $x_n \neq 4, 7$.
  Note that
  $$|x-y| \geq 10^{-n} > 0$$
  for any $y = 0.y_1y_2 \cdots \in E$.
  Hence there is an open neighborhood $B(x;10^{-n})$ of $x$ such that
  $B(x;10^{-n}) \cap E = \varnothing$, or $B(x;10^{-n}) \subseteq \widetilde{E}$,
  or $x$ is an interior point of $\widetilde{E}$.
  \item[(b)]
  \emph{Show that $E$ is closed.}
  Given any limit point $x \in \mathbb{R}^1$ of $E$, we want to show that $x \in E$.
  (Reductio ad absurdum) Similar to (a).
  \item[(c)]
  \emph{Show that $E$ is bounded.}
  $E \subseteq B(0;1)$.
  \end{enumerate}
\item[(4)]
\emph{Show that $E$ is perfect.}
  \begin{enumerate}
  \item[(a)]
  $E$ is closed (by (3)).
  \item[(b)]
  \emph{Show that every point of $E$ is a limit point of $E$.}
  Given any $x \in E$.
  Given any open neighborhood $B(x;r)$ of $x$, there is a positive integer $n$
  such that $$\frac{3}{10^n} < r.$$
  For such $n$, pick $y = 0.x_1x_2 \cdots x_{n-1} y_n \cdots x_{n+1} \cdots \in E$
  where
  \begin{equation*}
  y_n =
    \begin{cases}
      4 & (x_n = 7), \\
      7 & (x_n = 4).
    \end{cases}
  \end{equation*}
  $y \neq x$, and $|y-x| = \frac{3}{10^{n}} < r$.
  So that there is $y \neq x$ such that $y \in B(x;r)$, or $x$ is a limit point of $E$.
  \end{enumerate}
\end{enumerate}
$\Box$ \\\\



%%%%%%%%%%%%%%%%%%%%%%%%%%%%%%%%%%%%%%%%%%%%%%%%%%%%%%%%%%%%%%%%%%%%%%%%%%%%%%%%



\textbf{Exercise 2.18.}
\emph{Is there a nonempty perfect set in $\mathbb{R}^1$ which contains no rational number?} \\

Yes. \\

\textbf{Lemma.}
\emph{$x \in \mathbb{Q}$ if and only if has repeating decimal expansion.} \\

\emph{Proof of Lemma.}
\begin{enumerate}
\item[(1)]
$(\Longleftarrow)$
Given any repeating decimal
$$x = x_0.x_1 x_2 \cdots x_n \overline{x_{n+1} \cdots x_{n+m}}$$
where $x_0 \in \mathbb{Z}$ and $x_1, \ldots, x_{n+m} \in \{ 0,1,2,3,4,5,6,7,8,9 \}$.
Thus $x = p/q$
where
$$p = (10^m-1) \sum_{i=0}^{n}10^{n-i} x_i
+ \sum_{j=1}^{m} 10^{m-j} x_{n+j} \in \mathbb{Z}$$
and
$$q = 10^n(10^m-1) \in \mathbb{Z}.$$
\item[(2)]
$(\Longrightarrow)$
(Euler's totient function)
Given any $x = p/q$ where $p, q \in \mathbb{Z}$, $q > 0$.
  \begin{enumerate}
  \item[(a)]
  Write $q = 2^{a} 5^{b} q_1$ where $a, b$ are nonnegative integers and
  $(q_1, 10) = 1$ (Unique factorization theorem).
  \item[(b)]
  Let $n = \max\{a,b\}$. Then $2^{n-a} 5^{n-b} q = 10^n q_1$.
  \item[(c)]
  Since $(q_1, 10) = 1$,
  $10^m \equiv 1 \pmod{q_1}$ where $m = \varphi(q_1)$ is Euler's totient function of $q_1$.
  Hence
  $10^m - 1 = q_1 q_2$ for some $q_2 \in \mathbb{Z}$,
  or
  $$2^{n-a} 5^{n-b} q_2 q = 10^n(10^m-1).$$
  Here $2^{n-a} 5^{n-b} q_2$, $n, m$ are nonnegative integers.
  \item[(d)]
  Now write
  $$x = \frac{p}{q}
  = \frac{2^{n-a} 5^{n-b} q_2 p}{10^n(10^m-1)}
  = \frac{(10^m-1) q_3 + r}{10^n(10^m-1)}
  = \frac{q_3}{10^n} + \frac{r}{10^n(10^m-1)}$$
  where $q_3, r \in \mathbb{Z}$ with $0 \leq r < 10^m-1$.
  Might assume $q_3 \geq 0$.
  (If $q_3 < 0$, apply the same argument to $-q_3$
  and then add the minus symbol ``$-$'' in the front of a decimal expansion.)
  Hence
  $$x = x_0.x_1 x_2 \cdots x_n \overline{x_{n+1} \cdots x_{n+m}}$$
  where
  \begin{align*}
    x_0 &= \left\lfloor \frac{q_3}{10^n} \right\rfloor \\
    x_i &= \text{last digit of } \left\lfloor \frac{q_3}{10^{n-i}} \right\rfloor
      &\text{($1 \leq i \leq n$)} \\
    x_{n+j} &= \text{last digit of } \left\lfloor \frac{r}{10^{m-j}} \right\rfloor
      &\text{($1 \leq j \leq m$)}
  \end{align*}
  \end{enumerate}
\item[(3)]
$(\Longrightarrow)$
(Pigeonhole principle)
Given any $x = p/q$ where $p, q \in \mathbb{Z}$, $q > 0$.
  \begin{enumerate}
  \item[(a)]
  Might assume $p \geq 0$.
  (If $p < 0$, apply the same argument to $-p$
  and then add the minus symbol ``$-$'' in the front of the decimal expansion.)
  Write $$x = x_0.x_1 x_2 \cdots.$$
  \item[(b)]
  Apply Euclidean algorithm to get
  $$p = x_0 q + r_0 \qquad \text{with} \qquad 0 \leq r_0 < q.$$
  $x_0$ is the integer part of $x = p/q$.
  Continue Euclidean algorithm to get $x_1$ by
  $$10 r_0 = x_1 q + r_1 \qquad \text{with} \qquad 0 \leq r_1 < q.$$
  In general, for $n \geq 1$, $x_n$ is given by
  $$10 r_{i-1} = x_i q + r_i \qquad \text{with} \qquad 0 \leq r_i < q.$$
  \item[(c)]
  The pigeonhole principle shows that there must be two equal remainders,
  that is, $$r_n = r_{n+m} \qquad \text{with} \qquad m > 0.$$
  By induction, $r_{n+k} = r_{n+m+k}$ for any $k \geq 0$.
  Thus $x_{n+k} = x_{n+m+k}$ holds for any $k > 0$, that is,
  $x$ has a decimal expansion
  $$x = x_0.x_1 x_2 \cdots x_n \overline{x_{n+1} \cdots x_{n+m}}.$$
  \end{enumerate}
\end{enumerate}
$\Box$ \\



\emph{Proof (Exercise 2.17).}
Let $A$ be the set of all $y \in [0,1]$
whose decimal expansion contains only the digits $4$ and $7$.
Though $A \cap \mathbb{Q} \neq \varnothing$ since $\frac{4}{9} \in A$,
we can shift $A$ by a number $\xi = \sum_{n=0}^{\infty} 10^{-n!}$ (Exercise 2.3),
that is, we construct
$$E = \{ y + \xi : y \in A \}$$
and show that $E$ is our desired nonempty perfect set in $\mathbb{R}-\mathbb{Q}$.
\begin{enumerate}
\item[(1)]
Any number $x \in E$ has decimal expansion
$x = 0.x_1 x_2 \cdots$ with
$x_n \in \{5,8\}$ if $n$ is a factorial number; otherwise $x_n \in \{4,7\}$.
\item[(2)]
$E$ is a perfect set (Exercise 2.17).
\item[(3)]
$E \subseteq \mathbb{R}-\mathbb{Q}$.
It suffices to show that each $x \in E$ has no repeating decimal expansions (Lemma).
It is clear by the construction of $\xi = \sum_{n=0}^{\infty} 10^{-n!}$.
\end{enumerate}
$\Box$ \\



\emph{Proof (Exercise 2.3).}
Let $E$ be a subset of Liouville numbers as
$$E = \left\{ \sum_{n=0}^{\infty} \frac{a_n}{10^{n!}} : a_n \in \{4, 7\} \right\}.$$
$E$ is perfect. (The same argument of Exercise 2.17.)
Besides,
all numbers of $E$ are transcendental.
(Set $k_j = 10^{j!}$ and $h_j = 10^{j!} \sum_{n=0}^{j} \frac{a_n}{10^{n!}}$
and apply the same argument of Exercise 2.3.)
$\Box$ \\

\emph{Note.}
Or using Lemma to prove all numbers of $E$ are irrational. \\

\emph{Proof (Theorem 3.32).}
Let
$$E = \left\{ \sum_{n=1989}^{\infty} \frac{a_n}{n!} : a_n \in \{6, 4\} \right\}.$$
$E$ is perfect. (The same argument of Exercise 2.17.)
Besides,
all numbers of $E$ are irrational (The same argument of Theorem 3.32.)
$\Box$ \\

\emph{Proof (Non constructive existence proof).}
By Cantor-Bendixson theorem (Exercise 2.28),
\emph{it suffices to find a uncountable closed set in $\mathbb{R} - \mathbb{Q}$.}
\begin{enumerate}
\item[(1)]
Write $\mathbb{Q} = \{ r_1, r_2, \ldots \}$ since $\mathbb{Q}$ is countable.
Let $$I_n = B\left(r_n;\frac{1}{2^{n+1}}\right) \supseteq \{r_n\}$$
and
$$A = \bigcup_{n=1}^{\infty} I_n \supseteq \mathbb{Q}.$$
Hence $A$ is an open subset in $\mathbb{R}$.
\item[(2)]
Let $E = \mathbb{R} - A$.
By construction, $E$ is closed and $E \cap \mathbb{Q} = \varnothing$.
\item[(3)]
\emph{Show that $E$ is uncountable.}
\emph{It suffices to show that $m^{*}(E) > 0$.}
In fact, the outer measure of $U$ is
$$m^{*}(A) \leq \sum_{n=1}^{\infty} m^{*}(I_n) = \sum_{n=1}^{\infty} \frac{1}{2^n} = 1.$$
Thus,
$$m^{*}(E) \geq m^{*}(\mathbb{R}) - m^{*}(A) = \infty - 1 = \infty.$$
\end{enumerate}
Hence, the set of all condensation points of $E$ is our desired
nonempty perfect set in $\mathbb{R}-\mathbb{Q}$.
$\Box$ \\

\emph{Note.}
In fact, we can replace $\mathbb{Q}$
by the set of all real algebraic numbers (Exercise 2.2). \\\\



%%%%%%%%%%%%%%%%%%%%%%%%%%%%%%%%%%%%%%%%%%%%%%%%%%%%%%%%%%%%%%%%%%%%%%%%%%%%%%%%



\textbf{Exercise 2.19.}
\begin{enumerate}
\item[(a)]
\emph{If $A$ and $B$ are disjoint closed sets in some metric space $X$,
prove that they are separated.}
\item[(b)]
\emph{Prove the same for disjoint open sets.}
\item[(c)]
\emph{Fix $p \in X$, $\delta > 0$,
define $A$ to be the set of all $q \in X$ for which $d(p,q) < \delta$,
define $B$ similarly, with $>$ in place of $<$.
Prove that $A$ and $B$ are separated.}
\item[(d)]
\emph{Prove that every connected metric space with at least two points
is uncountable.  Hint: Use (c).} \\
\end{enumerate}

\emph{Proof of (a).}
Since
\begin{align*}
A \cap \overline{B}
&= A \cap B
  &\text{($B$ is closed)} \\
&= \varnothing,
  &\text{($A$ and $B$ are disjoint)} \\
\overline{A} \cap B
&= A \cap B
  &\text{($A$ is closed)} \\
&= \varnothing.
  &\text{($A$ and $B$ are disjoint)}
\end{align*}
$A$ and $B$ are separated.
$\Box$ \\

\emph{Proof of (b)(Theorem 2.27(c)).}
Note that $\widetilde{A}$ is a closed set containing $B$.
Since $\overline{B}$ is the smallest closed set containing $B$,
$\widetilde{A} \supseteq \overline{B}$ (Theorem 2.27(c)).
Hence $$A \cap \overline{B} \subseteq A \cap \widetilde{A} = \varnothing.$$
Similarly, $\overline{A} \cap B = \varnothing$.
Hence $A$ and $B$ are separated.
$\Box$ \\

\emph{Proof of (c).}
Since both
$$A = \{ q \in X : d(p,q) < \delta \} \text{ and } B = \{ q \in X : d(p,q) > \delta \}$$
are open in $X$, they are separated by (b).
$\Box$ \\

\emph{Proof of (d).}
Let $X$ be a connected metric space.
\begin{enumerate}
\item[(1)]
Let $p, q \in X$ with $p \neq q$.
Hence $d_X(p,q) = r > 0$ (Definition 2.15(a)).
\item[(2)]
Given any $\delta \in (0, r)$.
Define
$$A = \{ x \in X : d(p,x) < \delta \} \text{ and } B = \{ x \in X : d(p,x) > \delta \}.$$
$p \in A \neq \varnothing$ and $q \in B \neq \varnothing$.
\item[(3)]
If there were no $y_{\delta} \in X$ such that $d(p,y_{\delta}) = \delta$,
we can write $X = A \cup B$ as a union of two nonempty separated sets ((c)),
contrary to the connectedness of $X$.
\item[(4)]
Collect these $y$ as $E$.
Since $d$ is a function, there is a one-to-one map from $(0,r)$ to $E$
defined by $\delta \mapsto y_{\delta}$ in (3).
Since $(0,r)$ is uncountable, $X \supseteq E$ is uncountable.
\end{enumerate}
$\Box$ \\\\



%%%%%%%%%%%%%%%%%%%%%%%%%%%%%%%%%%%%%%%%%%%%%%%%%%%%%%%%%%%%%%%%%%%%%%%%%%%%%%%%



\textbf{Exercise 2.20.}
\emph{Are closures and interiors of connected sets always connected?
(Look at subsets of $\mathbb{R}^2$.)} \\

\emph{Proof.}
\begin{enumerate}
\item[(1)]
\emph{Interiors of connected sets are not always connected.}
Let $X = \mathbb{R}^2$ with the usual Euclidean metric be a metric space.
Take
$$E = B(89;1) \bigcup B(64;1) \bigcup \{ (x,0) \in \mathbb{R}^2 : 64 \leq x \leq 89 \}.$$
$E$ is connected and
$$E^{\circ} = B(89;1) \bigcup B(64;1)$$
is disconnected.
\item[(2)]
\emph{Closures of connected sets are always connected.}
\emph{It suffices to show that
$E$ is disconnected if $\overline{E}$ is disconnected.}
  \begin{enumerate}
  \item[(a)]
  Write $\overline{E} = A \cup B$ as a union of two nonempty separated sets.
  Here $A \neq \varnothing$, $B \neq \varnothing$,
  $A \cap \overline{B} = \varnothing$ and $\overline{A} \cap B = \varnothing$.
  \item[(b)]
  Write
  $$E = (A \cap E) \bigcup (B \cap E)$$
  and we will show that $E$ is disconnected.
  \item[(c)]
  \emph{Show that $A \cap E$ and $B \cap E$ are separated.}
  In fact,
  $$(A \cap E) \cap \overline{B \cap E}
  \subseteq A \cap \overline{B} = \varnothing,$$
  $$\overline{A \cap E} \cap (B \cap E)
  \subseteq \overline{A} \cap B = \varnothing.$$
  \item[(d)]
  \emph{Show that $A \cap E$ and $B \cap E$ are nonempty.}
  (Reductio ad absurdum)
  If $A \cap E = \varnothing$, then
  $$E = (A \cap E) \bigcup (B \cap E) = B \cap E \Longrightarrow E \subseteq B.$$
  So
  \begin{align*}
  A &= (A \cup B) \bigcap A
    &(A \subseteq A \cup B) \\
  &= \overline{E} \bigcap A \\
  &\subseteq \overline{B} \bigcap A
    &(E \subseteq B) \\
  &= \varnothing
  \end{align*}
  which contradicts $A \neq \varnothing$ in (a).
  Therefore, $A \cap E \neq \varnothing$.
  Similarly, $B \cap E \neq \varnothing$.
  \end{enumerate}
Hence, $E$ is disconnected if $\overline{E}$ is disconnected,
or closures of connected sets are always connected.
\end{enumerate}
$\Box$ \\\\



%%%%%%%%%%%%%%%%%%%%%%%%%%%%%%%%%%%%%%%%%%%%%%%%%%%%%%%%%%%%%%%%%%%%%%%%%%%%%%%%



\textbf{Exercise 2.21.}
\emph{Let $A$ and $B$ be separated subsets of some $\mathbb{R}^k$,
suppose $\mathbf{a} \in A$, $\mathbf{b} \in B$, and define
$$\mathbf{p}(t) = (1-t)\mathbf{a} + t\mathbf{b}$$
for $t \in \mathbb{R}^1$.
Put $A_0 = \mathbf{p}^{-1}(A)$,
$B_0 = \mathbf{p}^{-1}(B)$.
[Thus $t \in A_0$ if and only if $\mathbf{p}(t) \in A$.]}
\begin{enumerate}
\item[(a)]
\emph{Prove that $A_0$ and $B_0$ are separated subsets of $\mathbb{R}^1$.}
\item[(b)]
\emph{Prove that there exists $t_0 \in (0,1)$ such that
$\mathbf{p}(t_0) \notin A \bigcup B$.}
\item[(c)]
\emph{Prove that every convex subset of $\mathbb{R}^k$ is connected.} \\
\end{enumerate}

\emph{Proof of (a).}
\begin{enumerate}
\item[(1)]
Note that
  \begin{enumerate}
  \item[(a)]
  $\mathbf{a} \neq \mathbf{b}$ or $|\mathbf{a} - \mathbf{b}| > 0$
  since $A \cap B = \varnothing$.
  \item[(b)]
  $|\mathbf{p}(t) - \mathbf{p}(s)| = |t-s||\mathbf{a} - \mathbf{b}|$
  by a direct calculation.
  \item[(c)]
  $\mathbf{p}(t) = \mathbf{p}(s)$ if and only if $t = s$ by (a)(b).
  \end{enumerate}
\item[(2)]
\emph{Show that $A_0 \cap \overline{B_0} = \varnothing$.}
(Reductio ad absurdum)
If there were $t \in A_0 \cap \overline{B_0}$,
then $t \in A_0$ and $t$ is a limit point of $B_0$.
  \begin{enumerate}
  \item[(a)]
  $t \in A_0$ implies that $\mathbf{p}(t) \in A$.
  \item[(b)]
  \emph{Show that $t$ is a limit point of $B_0$ $\Longrightarrow$
  $\mathbf{p}(t)$ is a limit point of $B$.}
  Given any $\varepsilon > 0$,
  there is $s \in B_0$ such that
  $$|t-s| < \frac{\varepsilon}{|\mathbf{a} - \mathbf{b}|}
  \qquad \text{with} \qquad s \neq t$$
  since $t$ is a limit point of $B_0$.
  So by (1),
  $$|\mathbf{p}(t) - \mathbf{p}(s)| = |t-s||\mathbf{a} - \mathbf{b}|
  < \varepsilon.$$
  Here $\mathbf{p}(s) \in B$ and $\mathbf{p}(s) \neq \mathbf{p}(t)$.
  So $\mathbf{p}(t)$ is a limit point of $B$.
  \end{enumerate}
  By (a)(b), $\mathbf{p}(t) \in A \cap \overline{B} = \varnothing$,
  contrary to the assumption that $A$ and $B$ are separated.
\item[(3)]
\emph{Show that $\overline{A_0} \cap B_0 = \varnothing$.}
Similar to (2).
\end{enumerate}
By (2)(3), $A_0$ and $B_0$ are separated.
$\Box$ \\

\emph{Proof of (b).}
(Reductio ad absurdum)
If $\mathbf{p}(t)$ were in $A \bigcup B$ for all $t \in (0,1)$,
we will \emph{show that $[0,1]$ is separated by $A_0 \cap [0,1]$ and $B_0 \cap [0,1]$
to get a contradiction.}
\begin{enumerate}
\item[(1)]
  $\mathbf{p}(t)$ were in $A \bigcup B$ for all $t \in [0,1]$
  since
  $\mathbf{p}(0) = \mathbf{a} \in A \bigcup B$
  and
  $\mathbf{p}(1) = \mathbf{b} \in A \bigcup B$.
  Therefore,
  $$[0,1]
  \subseteq \mathbf{p}^{-1}(A \cup B)
  = \mathbf{p}^{-1}(A) \cup \mathbf{p}^{-1}(B)
  = A_0 \cup B_0.$$
\item[(2)]
  Let $A_1 = A_0 \cap [0,1]$ and $B_1 = B_0 \cap [0,1]$.
  So $[0,1] = A_1 \bigcup B_1$.
\item[(3)]
  \emph{Show that $A_1 \neq \varnothing$ and $B_1 \neq \varnothing$.}
  \begin{align*}
  \mathbf{p}(0) \in A
  &\Longleftrightarrow 0 \in \mathbf{p}^{-1}(A) = A_0 \\
  &\Longleftrightarrow 0 \in A_0 \text{ and } 0 \in [0,1] \\
  &\Longleftrightarrow 0 \in A_0 \cap [0,1] = A_1.
  \end{align*}
  Similarly, $1 \in B_1$. \\

  \emph{Note.} That's why we consider $[0,1]$ instead of $(0,1)$.
\item[(4)]
  \emph{Show that $A_1 \cap \overline{B_1} = \varnothing$
  and $\overline{A_1} \cap B_1 = \varnothing$.}
  Since $A_1 \subseteq A_0$ and $B_1 \subseteq B_0$,
  $A_1 \cap \overline{B_1} \subseteq A_0 \cap \overline{B_0} = \varnothing$
  or $A_1 \cap \overline{B_1} = \varnothing$.
  Similarly, $\overline{A_1} \cap B_1 = \varnothing$.
\end{enumerate}
By (2)(3)(4), $[0,1]$ is separated, contrary to the connectedness of $[0,1]$
(Theorem 2.47).
$\Box$ \\

\emph{Proof of (c).}
\begin{enumerate}
\item[(1)]
Let $E$ be a convex subset of $\mathbb{R}^k$.
Recall
$$\mathbf{p}(t) = (1-t)\mathbf{a} + t\mathbf{b} \in E$$
whenever $\mathbf{a}, \mathbf{b} \in E$ and $t \in (0,1)$.
\item[(2)]
(Reductio ad absurdum)
If $E$ were separated by $A$ and $B$, pick $\mathbf{a} \in A \subseteq E$
and $\mathbf{b} \in B \subseteq E$.
\item[(3)]
By (b), there exists $t_0 \in (0,1)$ such that $\mathbf{p}(t_0) \not\in A \cup B = E$,
contrary to the convexity of $E$.
\end{enumerate}
$\Box$ \\\\



%%%%%%%%%%%%%%%%%%%%%%%%%%%%%%%%%%%%%%%%%%%%%%%%%%%%%%%%%%%%%%%%%%%%%%%%%%%%%%%%



\textbf{Exercise 2.22.}
\emph{A metric space is called separable if it contains a countable dense subset.
Show that $\mathbb{R}^k$ is separable.
(Hint: Consider the set of points which have only rational coordinates.)} \\

\emph{Proof.}
Let $E$ be the set of points which have only rational coordinates.
\begin{enumerate}
\item[(1)]
\emph{Show that $E$ is countable.} $\mathbb{Q}$ is countable
and thus $E = \mathbb{Q}^k$ is countable (Theorem 2.13).
\item[(2)]
\emph{Show that $E$ is dense.}
Given any $\mathbf{p} = (p_1, \ldots, p_k) \in \mathbb{R}^k$.
We want to show that $\mathbf{p}$ is a limit point of $E$.
  \begin{enumerate}
  \item[(a)]
  Given any open neighborhood $B(\mathbf{p};r)$ of $\mathbf{p}$, $r > 0$.
  \item[(b)]
  Since $\mathbb{Q}$ is dense in $\mathbb{R}$ (Theorem 1.20),
  every coordinate of $\mathbf{p}$ is a limit point of $\mathbb{Q}$.
  In particular, for every $i = 1, 2, \ldots, k$,
  the open neighborhood $B\left(p_i, \frac{r}{\sqrt{k}}\right)$ of $p_i$
  contains a point $q_i \neq p_i$ and $q_i \in \mathbb{Q}$.
  \item[(c)]
  Collect all $q_i$ in (b)
  and define $\mathbf{q} = (q_1, \ldots, q_k) \in \mathbb{Q}^k = E$.
  By construction $\mathbf{q} \neq \mathbf{p}$ and
  \begin{align*}
    |\mathbf{p} - \mathbf{q}|
    &= \sqrt{(p_1 - q_1)^2 + \cdots + (p_k - q_k)^2} \\
    &< \sqrt{\left(\frac{r}{\sqrt{k}}\right)^2 + \cdots
      + \left(\frac{r}{\sqrt{k}}\right)^2} \\
    &= \sqrt{k \cdot \frac{r^2}{k}} \\
    &= r
  \end{align*}
  or $\mathbf{q} \in B(\mathbf{p};r)$.
  \end{enumerate}
  By (a)(b)(c), $E$ is dense in $\mathbb{R}^k$.
\end{enumerate}
By (1)(2), $\mathbb{R}^k$ is separable.
$\Box$ \\\\



%%%%%%%%%%%%%%%%%%%%%%%%%%%%%%%%%%%%%%%%%%%%%%%%%%%%%%%%%%%%%%%%%%%%%%%%%%%%%%%%



\textbf{Exercise 2.23.}
\emph{A collection $\{V_\alpha\}$ of open subsets of $X$ is
said to be a base for $X$ if the following is true:
For every $x \in X$ and every open set $G \subseteq X$ such that $x \in G$,
we have $x \in V_\alpha \subseteq G$ for some $\alpha$.
In other words, every open set in $X$ is the union of a subcollection of $\{V_\alpha\}$.} \\

\emph{Prove that every separable metric space has a countable base.
(Hint: Take all neighborhoods with rational radius and center
in some countable dense subset of $X$.)} \\

\emph{Note.}
$\mathbb{R}^k$ has a countable base (Exercise 2.22). \\

\emph{Proof (Hint).}
Let $X$ be a separable metric space,
and $E$ be a countable dense subset of $X$.
Let $\mathscr{B}$ be a collection of
all neighborhoods with rational radius and center in $E$.
\begin{enumerate}
\item[(1)]
$\mathscr{B}$ is countable (Theorem 2.12).
\item[(2)]
\emph{$\mathscr{B}$ is a base for $X$.}
Similar to Exercise 2.9(a).
Given any $p \in X$ and every open set $G \subseteq X$ such that
$p \in G$.
Since $p$ is in an open set $G$,
there exists an open neighborhood $B(p;r)$ of $p$ such that $B(p;r) \subseteq G$.
\item[(3)]
Let $r_0$ be rational such that $0 < r_0 < \frac{r}{2}$ (Theorem 1.20(b)).
Since $E$ is dense in $X$, there is $q \in E$ such that $d_X(p,q) < r_0$.
For such $r_0 \in \mathbb{Q}$ we pick an open neighborhood $B(q;r_0)$ of $q$.
Clearly, $B(q;r_0) \in \mathscr{B}$.
\item[(4)]
$p \in B(q;r_0)$ since $d_X(p,q) < r_0$.
\item[(5)]
\emph{Show that $B(q;r_0) \subseteq B(p;r) \subseteq G$.}
For any $z \in B(q;r_0)$, $d_X(z,p) \leq d_X(z,q)+ d_X(q,p) < r_0 + r_0
< \frac{r}{2} + \frac{r}{2} = r$.
That is, $z \in B(p;r)$.
\end{enumerate}
By (3)(4)(5), (2) is established.
By (1)(2), $\mathscr{B}$ is a countable base for $X$.
$\Box$ \\



\textbf{Supplement.}
\begin{enumerate}
\item[(1)]
In topology, a second-countable space, also called a completely separable space,
is a topological space whose topology has a countable base.
\item[(2)]
Every second-countable space is separable.
\item[(3)]
The reverse implication of (2) does not hold in general.
However, for metric spaces
the properties of being second-countable and separable are equivalent.
\item[(4)]
\emph{Show that every second-countable metric space $X$ is separable.}
  \begin{enumerate}
  \item[(a)]
  Let $\mathscr{B} = \{ B_n : n \in \mathbb{Z}^+ \}$ be a countable base of $X$.
  \item[(b)]
  For every $B_n \in \mathscr{B}$, pick any point $p_n$ of $B_n$
  and collect them as $$E = \{ p_n : p_n \in B_n \text{ for } n \in \mathbb{Z}^+ \}.$$
  \item[(c)]
  $E$ is countable.
  \item[(d)]
  \emph{Show that $E$ is dense.}
  Given any $x \in X$.
  For any open neighborhood $B(x)$ of $x$,
  $B(x)$ is a union of subcollection of $\mathscr{B}$.
  That is,
  there is always a point in $E$ by the construction of $E$.
  \end{enumerate}
  $\Box$ \\\\
\end{enumerate}



%%%%%%%%%%%%%%%%%%%%%%%%%%%%%%%%%%%%%%%%%%%%%%%%%%%%%%%%%%%%%%%%%%%%%%%%%%%%%%%%



\textbf{Exercise 2.24.}
\emph{Let $X$ be a metric space in which every infinite subset has a limit point.
Prove that $X$ is separable.} \\

\emph{(Hint: Fix $\delta>0$, and pick $x_1 \in X$.
Having chosen $x_1, \ldots, x_j \in X$, choose $x_{j+1}$, if possible,
so that $d(x_i,x_{j+1}) \geq \delta$ for $i=1,\ldots,j$.
Show that this process must stop after finite number of steps,
and that $X$ can therefore be covered by finite many neighborhoods of radius $\delta$.
Take $\delta = \frac{1}{n}$ $(n=1,2,3,\ldots)$
and consider the centers of the corresponding neighborhoods.) } \\

\emph{Note.}
The reverse implication does not hold (Exercise 2.10). \\

\emph{Proof (Hint).}
\begin{enumerate}
\item[(1)]
Fix $\delta>0$, and pick $x_1 \in X$.
\emph{Show that every limit point compact metric space $X$ is totally bounded.}
  \begin{enumerate}
  \item[(a)]
  Having chosen $x_1, \ldots, x_j \in X$, choose $x_{j+1}$, if possible,
  so that $d(x_i,x_{j+1}) \geq \delta$ for $i=1,\ldots,j$.
  Let $E_\delta$ be the set of these $x_i$.
  \item[(b)]
  \emph{Show that this process must stop after finite number of steps,
  and that $X$ can therefore be covered by finite many neighborhoods of radius $\delta$.}
  (Reductio ad absurdum)
    \begin{enumerate}
    \item[(i)]
    If not, $E_\delta$ is an infinite subset of $X$.
    By assumption there is a limit point of $E_\delta$, say $p \in X$.
    \item[(ii)]
    In particular, an open neighborhood $B\left(p;\frac{\delta}{64}\right)$ of $p$
    contains a point $x_n \in E_\delta$ with $p \neq x_n$.
    \item[(iii)]
    The neighborhood $B\left(p;\frac{\delta}{64}\right)$ contains no other point
    $x_m \in E_\delta$ with $m \neq n$.
    If so,
    $$d_X(x_n,x_m) \leq d_X(x_n,p)+d_X(p,x_m)
    < \frac{\delta}{64} + \frac{\delta}{64} < \delta,$$
	contrary to the construction of $E_\delta$.
    \item[(iv)]
    Note that $p \not\in E_\delta$ as a corollary to (iii).
    \item[(v)]
    So another open neighborhood $B\left(p;r\right)$ of $p$ with $r = d_X(p,x_n) > 0$
    contains no points $x_m \in E_\delta$ with $p \neq x_m$,
    contrary to the assumption that $p$ is a limit point of $E_\delta$.
    \end{enumerate}
  \end{enumerate}
\item[(2)]
\emph{Show that every totally bounded metric space $X$ is separable.}
Take $\delta = \frac{1}{n}$ $(n=1,2,3,\ldots)$ in (1),
and union all $E_{\frac{1}{n}}$ as
$$E = \bigcup_{n=1}^{\infty} E_{\frac{1}{n}} \subseteq X.$$
\emph{Show that $E$ is a countable dense subset of $X$.}
  \begin{enumerate}
  \item[(a)]
  \emph{Show that $E$ is countable.}
  Since $E$ is the countable union of finite set $E_{\frac{1}{n}}$,
  $E$ is countable (Theorem 2.12).
  \item[(b)]
  \emph{Show that $E$ is dense in $X$.}
  Given any $p \in X$.
  \emph{It suffices to show that given any open neighborhood $B(p;r)$ of $p \in X-E$,
  there exists $q \in E$ such that $q \in B(p;r)$.}
  Pick any $n \in \mathbb{Z}^+$ such that $\frac{1}{n} < r$ (Theorem 1.20(a)).
  By the construction of $E_{\frac{1}{n}}$,
  there is $q  \in E_{\frac{1}{n}}$ such that $p \in B\left(q;\frac{1}{n}\right)$,
  or $d_X(p,q) < \frac{1}{n} < r$, or $q \in  B(p;r)$.
  \end{enumerate}
\end{enumerate}
$\Box$ \\



\textbf{Supplement.}
\begin{enumerate}
\item[(1)]
A topological space $X$ is said to be limit point compact or weakly countably compact if every infinite subset of $X$ has a limit point in $X$.
\item[(2)]
In a metric space, limit point compactness, compactness, and sequential compactness are
all equivalent.
For general topological spaces, however, these three notions of compactness are not equivalent.
\item[(3)]
A metric space $X$ is totally bounded if and only if for every real number $\delta >0$,
there exists a finite collection of open balls in $X$ of radius $\delta$
whose union contains $X$. \\\\
\end{enumerate}



%%%%%%%%%%%%%%%%%%%%%%%%%%%%%%%%%%%%%%%%%%%%%%%%%%%%%%%%%%%%%%%%%%%%%%%%%%%%%%%



\textbf{Exercise 2.25.}
\emph{Prove that every compact metric space $K$ has a countable base,
and that $K$ is therefore separable.
(Hint: For every positive integer $n$,
there are finitely many neighborhood of radius $\frac{1}{n}$ whose union
covers $K$.)} \\

\emph{Proof (Exercise 2.24(a)).}
\begin{enumerate}
\item[(1)]
\emph{Show that every compact metric space $K$ is limit point compact.}
Given any subset $E \subseteq K$.
It suffices to show that if $E$ has no limit point, then $E$ must be finite.
  \begin{enumerate}
  \item[(a)]
  Since $E$ has no limit point, $E$ is closed.
  \item[(b)]
  For any point $p \in E$. Since $p$ is not a limit point,
  there is an open neighborhood $B(p)$ such that $B(p)$ contains no point other than $p$.
  \item[(c)]
  Similar to the proof of Theorem 2.35,
  let
  $$\mathscr{F} = \{ B(p) : p \in E \text{ with } B(p) \cap E = \{p\} \}
  \bigcup \widetilde{E}.$$
  Hence $\mathscr{F}$ is an open covering of $K$.
  \item[(d)]
  Since $K$ is compact by assumption,
  there is an finitely subcovering $\mathscr{F}'$ of $K$.
  Since $\widetilde{E}$ does not intersect $E$,
  each $B(p) \in \mathscr{F}'$
  contains only one point of $E$ and so $E$ is finite.
  \end{enumerate}
\item[(2)]
Since $K$ is limit point compact, $K$ is separable (Exercise 2.24).
\end{enumerate}
$\Box$ \\

\emph{Proof (Exercise 2.24(b)).}
\begin{enumerate}
\item[(1)]
\emph{Show that every compact metric space $K$ is totally bounded.}
Given any real number $\delta > 0$,
define an open covering $\mathscr{F}$ of $K$ by
$$\mathscr{F} = \{ B(p;\delta) : p \in K \}.$$
Since $K$ is compact,
there exists a finite subcovering $\mathscr{F}'$ of $K$.
$\mathscr{F}'$ is our desired finite collection of open balls in $X$
of radius $\delta$ whose union contains $X$.
\item[(2)]
Since $K$ is totally bounded, $K$ is separable (Exercise 2.24).
\end{enumerate}
$\Box$ \\

\emph{Proof (Hint).}
\begin{enumerate}
\item[(1)]
Given any positive integer $n > 0$,
define an open covering $\mathscr{F}_n$ of $K$ by
$$\mathscr{F}_n = \left\{ B\left(p;\frac{1}{n}\right) : p \in K \right\}.$$
Since $K$ is compact,
there exists a finite subcovering $\mathscr{G}_n$ of $K$.
\item[(2)]
\emph{Show that every compact metric space $K$ is second-countable.}
  \begin{enumerate}
  \item[(a)]
  Define $$\mathscr{B} = \bigcup_{n \geq 1} \mathscr{G}_n$$ be a collection.
  Since $\mathscr{B}$ is a countable union of finite set $\mathscr{G}_n$,
  $\mathscr{B}$ is countable.
  Hence it suffices to show that
  for every $p \in K$ and every open set $G \subseteq K$ such that $p \in G$,
  there is $B \in \mathscr{B}$ such that $x \in B \subseteq G$.
  \item[(b)]
  Since $G$ is open,
  there is an open neighborhood $B(p;r)$ of $p$ such that $B(p;r) \subseteq G$.
  \item[(c)]
  For such $r > 0$, there is $n \in \mathbb{Z}^+$
  with $0 < \frac{1}{n} < \frac{r}{2}$ (Theorem 1.20(a)).
  So $p$ is in some $B\left(q;\frac{1}{n}\right) \in \mathscr{G}_n \subseteq \mathscr{B}$
  since $\mathscr{G}_n$ is a subcovering of $K$.
  \item[(d)]
  \emph{Show that $B\left(q;\frac{1}{n}\right) \subseteq B(p;r) \subseteq G$.}
  For any $z \in B(q;\frac{1}{n})$,
  $$d_K(z,p) \leq d_K(z,q)+ d_K(q,p) < \frac{1}{n} + \frac{1}{n}
  < \frac{r}{2} + \frac{r}{2} = r.$$
  That is, $z \in B(p;r)$, or $B\left(q;\frac{1}{n}\right) \subseteq B(p;r) \subseteq G$.
  \end{enumerate}
  By (a)(b)(c)(d), $K$ is second-countable.
\item[(3)]
\emph{Show that every second-countable metric space is separable.}
Supplement (4) to Exercise 2.23.
\end{enumerate}
$\Box$ \\\\



%%%%%%%%%%%%%%%%%%%%%%%%%%%%%%%%%%%%%%%%%%%%%%%%%%%%%%%%%%%%%%%%%%%%%%%%%%%%%%%%



\textbf{Exercise 2.26.}
\emph{Let $X$ be a metric space in which every infinite subsets has a limit point.
Prove that $X$ is compact.} \\

\emph{By Exercises 2.23 and 2.24, $X$ has a countable base.
It follows that every open cover of $X$ has a countable subcovering $\{G_n\}$,
$n = 1, 2, 3, \ldots$.
If no finite subcollection of $\{G_n\}$ covers $X$,
then the complement $F_n$ of $G_1 \cup \cdots \cup G_n$
is nonempty for each $n$, but $\bigcap F_n$ is empty.
If $E$ is a set contains a point from each $F_n$,
consider a limit point of $E$, and obtain a contradiction.} \\

\emph{Note.} In every metric space, we have
\begin{align*}
\{ \text{compact} \}
&\Longleftrightarrow
\{ \text{limit point compact} \} \\
&\Longleftrightarrow
\{ \text{complete and totally bounded} \} \\
&\Longrightarrow
\{ \text{totally bounded} \} \\
&\Longrightarrow
\{ \text{separable} \} \\
&\Longleftrightarrow
\{ \text{second-countable} \} \\
&\Longleftrightarrow
\{ \text{Lindelof} \}. \\
\end{align*}

\emph{Proof (Hint).}
\begin{enumerate}
\item[(1)]
Since $X$ is limit point compact, $X$ is separable (Exercise 2.24).
Since $X$ is separable, $X$ is second-countable (Exercise 2.23).
\item[(2)]
\emph{Show that $X$ is Lindelof if $X$ is second-countable.}
Let $X$ be a second-countable metric space.
Let $\mathscr{B} = \{ B_n \}$ be a countable base of $X$.
  Given any open covering $\mathscr{F}$ of $X$.
  \begin{enumerate}
  \item[(a)]
  Iterate each $B_n \in \mathscr{B}$, pick one $G_n \in \mathscr{F}$ containing $B_n$,
  and collect them as
  $$\mathscr{G} = \{ G_n : G_n \supseteq B_n \text{ for } n \in \mathbb{Z}^+ \}.$$
  ($G_n$ might be duplicated.)
  \item[(b)]
  $\mathscr{G}$ is a countable subset of $\mathscr{F}$.
  \item[(c)]
  $\mathscr{G}$ covers $X$
  since $\mathscr{B}$ is a countable base of $X$.
  \end{enumerate}
\item[(3)]
Hence,
given any open covering $\mathscr{F}$ of $X$,
there is a countable subcovering $\mathscr{G} = \{ G_n \}$ of $X$.
(Reductio ad absurdum)
If there were no finite subcovering of $\mathscr{G}$,
then the complement $F_n$ of $G_1 \cup \cdots \cup G_n$
is nonempty for each $n$, but $\cap F_n$ is empty.
\item[(4)]
Let $E$ bet a set contains a point from each $F_n$.
$E$ is infinite and thus $E$ has a limit point, say $p$.
$p \in G_n$ for some $n$ since $\mathscr{G} = \{ G_n \}$ is an open covering of $X$.
Since $G_n$ is open, there is an open neighborhood $B(p)$ of $p$ such that $B(p) \subseteq G_n$.
By the construction of $F_n$,
$$B(p) \cap F_m = \varnothing$$ whenever $m \geq n$,
contrary to the assumption that $p$ is a limit point of $E$.
\end{enumerate}
Hence, $X$ is compact if $X$ is limit point compact.
$\Box$ \\



\textbf{Supplement.}
\begin{enumerate}
\item[(1)]
Lindelof space is a topological space in which every open covering has a countable subcovering.
\item[(2)]
\emph{Show that $X$ is second-countable if $X$ is Lindelof.}
Same as the Proof (Hint) of Exercise 2.25 except
changing the word ``compact'' to ``Lindelof'' and ``finite'' to ``countable.''
$\Box$
\item[(3)]In every metric space, we have
\begin{align*}
\{ \text{compact} \}
&\Longleftrightarrow
\{ \text{limit point compact} \}
\Longleftrightarrow
\{ \text{sequentially compact} \}. \\
\end{align*}
\end{enumerate}



%%%%%%%%%%%%%%%%%%%%%%%%%%%%%%%%%%%%%%%%%%%%%%%%%%%%%%%%%%%%%%%%%%%%%%%%%%%%%%%%



\textbf{Exercise 2.27.}
\emph{
Define a point $p$ in a metric space $X$ to be a {\it condensation
point} of a set $E \subseteq X$ if every neighborhood of $p$ contains
uncountably many points of $E$.} \\

\emph{Suppose $E \subseteq \mathbb{R}^k$, $E$ is uncountable,
and let $P$ be the set of all condensation points of $E$.
Prove that $P$ is perfect and that at most countably many points of $E$ are not in $P$.
In other words, show that $\widetilde{P} \cap E$ is at most countable.} \\

\emph{(Hint: Let $\{V_n\}$ be a countable base of $\mathbb{R}^k$,
let $W$ be the union of those $V_n$ for which $E \cap V_n$ is at most countable,
and show that $P = \widetilde{W}$.)} \\

\emph{Note.}
The statement is also true for separable metric space. \\

\emph{Proof.}
\begin{enumerate}
\item[(1)]
Let $\{V_n\}$ be a countable base of $\mathbb{R}^k$ (Exercise 2.22 and 2.23).
Let $W$ be the union of those $V_n$ for which $E \cap V_n$ is at most countable.
\item[(2)]
\emph{Show that $P = \widetilde{W}$.}
  \begin{enumerate}
  \item[(a)]
  $(P \subseteq \widetilde{W})$
  Given any $x \in P$.
  \begin{align*}
  x \in P
  \Longrightarrow&
  \text{$x$ is a condensation points of $E$} \\
  \Longrightarrow&
  \text{$\forall \: V_n \ni x, \exists B(x) \subseteq V_n$ such that
    $E \cap B(x)$ is uncountable} \\
  \Longrightarrow&
  \text{$E \cap V_n$ is uncountable} \\
  \Longrightarrow&
  x \not\in W.
  \end{align*}
  \item[(b)]
  $(P \supseteq \widetilde{W})$
  Given any $x \in \widetilde{W}$.
  Let $P(V_n)$ be the proposition that $E \cap V_n$ is at most countable.
  \begin{align*}
  x \in \widetilde{W}
  \Longrightarrow&
  x \not\in W = \bigcup_{P(V_n)} V_n \\
  \Longrightarrow&
  \text{$x \not\in V_n$ for which $E \cap V_n$ is at most countable} \\
  \Longrightarrow&
  \text{$\forall$ $B(x)$ of $x$, $x \in V_m \subseteq B(x)$ for some $V_m$}
    &\text{($\{V_n\}$: base of $X$)} \\
  \Longrightarrow&
  \text{$E \cap V_m$ is uncountable} \\
  \Longrightarrow&
  \text{$E \cap B(x) \supseteq E \cap V_m$ is uncountable} \\
  \Longrightarrow&
  \text{$x$ is a condensation point of $E$} \\
  \Longrightarrow&
  x \in P.
  \end{align*}
  \end{enumerate}
\item[(3)]
\emph{Show that $P$ is closed.}
$P$ is the complement of an open subset $W$.
\item[(4)]
\emph{Show that $P \subseteq P'$.}
(Reductio ad absurdum)
  \begin{enumerate}
  \item[(a)]
  If there were an isolated point $x \in P$, then
  there exists an open neighborhood $B(x)$ of $x$ such that $B(x) \cap P = \{x\}$.
  \item[(b)]
  Since $x$ is a condensation point of $E$,
  there are uncountably many points of $E$ in $B(x)$,
  and such points $y$ are not a condensation points of $E$ except $y = x$.
  \item[(c)]
  Given any point $y \in E \cap B(x)$ with $y \neq x$.
  Since $y$ is not a condensation point,
  there exists a neighborhood $B(y)$ of $y$ such that $B(y) \cap E$ is at most countable.
  Since $\{V_n\}$ is a base,
  for each $B(y)$ there exists $V_{n(y)}$ such that $y \in V_{n(y)} \subseteq B(y)$.
  Hence $$V_{n(y)} \cap E \subseteq B(y) \cap E$$
  is at most countable.
  \item[(d)]
  Hence,
  \begin{align*}
  E \cap B(x) - \{x\}
  &\subseteq \bigcup_{y \in E \cap B(x) - \{x\}} V_{n(y)}  \\
  &= \bigcup_{n(y)} V_{n(y)}
  \end{align*}
  is a countable union of at most countable sets,
  which is countable.
  Hence $E \cap B(x) - \{x\}$ or $E \cap B(x)$ is countable,
  contrary to the assumption that $E \cap B(x)$ is uncountable.
  \end{enumerate}
\item[(5)]
\emph{Show that $E \cap \widetilde{P}$ is at most countable.}
$$E \cap \widetilde{P}
= E \bigcap \left(\bigcup_{P(V_n)} V_n\right)
= \bigcup_{P(V_n)}(E \cap V_n)$$
is at most countable.
\end{enumerate}

$\Box$ \\\\



%%%%%%%%%%%%%%%%%%%%%%%%%%%%%%%%%%%%%%%%%%%%%%%%%%%%%%%%%%%%%%%%%%%%%%%%%%%%%%%%



\textbf{Exercise 2.28.}
\emph{Prove that every closed set in a separable metric space is
the union of a (possible empty) perfect set and a set which is at most
countable.
(Corollary: Every countable closed set in $\mathbb{R}^k$ has isolated points.)
(Hint: Use Exercise 2.27.) } \\

\emph{Proof (Exercise 2.27).}
Let $E$ be a closed set in a separable metric space.
\begin{enumerate}
\item[(1)]
$E$ contains all limit points of $E$,
especially contains all condensation points of $E$.
So we can write $$E = P \cup (E-P)$$
where $P$ is the set of all condensation points of $E$.
\item[(2)]
By Exercise 2.27, $P$ is perfect and
$E-P = E \cap \widetilde{P}$ is at most countable.
\end{enumerate}
$\Box$ \\

\textbf{Cantor-Bendixson theorem.}
\begin{enumerate}
\item[(1)]
Closed sets of a Polish space $X$ have the perfect set property in a particularly strong form:
any closed subset of $X$ may be written uniquely as
the disjoint union of a perfect set and a countable set.
\item[(2)]
A Polish space is a separable completely metrizable topological space;
that is, a space homeomorphic to a complete metric space
that has a countable dense subset. \\\\
\end{enumerate}



%%%%%%%%%%%%%%%%%%%%%%%%%%%%%%%%%%%%%%%%%%%%%%%%%%%%%%%%%%%%%%%%%%%%%%%%%%%%%%%%



\textbf{Exercise 2.29.}
\emph{Prove that every open set in $\mathbb{R}^1$ is the union of
an at most countable collection of disjoint segments.
(Hint: Use Exercise 2.22.)} \\

\emph{Proof.}
Let $E$ be an open subset of $\mathbb{R}^1$.
\begin{enumerate}
\item[(1)]
For each $x \in E$, let $I_x$ denote the largest open interval
containing $x$ and contained in $E$.
More precisely, since $E$ is open,
$x$ is contained in some small (non-trivial) interval,
and therefore if
$$a_x = \inf\{ a < x : (a,x) \subseteq E \} \text{ and }
b_x = \sup\{ b > x : (x,b) \subseteq E \}$$
we must have $a_x < x < b_x$ (with possibly infinite values for $a_x$ and $b_x$).
\item[(2)]
Let $I_x = (a_x, b_x)$, then by construction we have $x \in I_x$
as well as $I_x \subseteq E$.
Hence
$$E = \bigcup_{I_x \in \mathscr{F}} I_x,$$
where $\mathscr{F} = \{I_x\}_{x \in E}$.
\item[(3)]
Suppose that two intervals $I_x$ and $I_y$ intersect.
Then their union (which is also an open interval)
is contained in $E$ and contains $x$ (and $y$).
Since $I_x$ is maximal, $I_x \cup I_y \subseteq I_x$,
and similarly $I_x \cup I_y \subseteq I_y$.
This can happen only if $I_x = I_y$.
\item[(4)]
Therefore, any two distinct intervals in $\mathscr{F}$ must be disjoint.
Hence $\mathscr{F}$ is countable since each open interval $I_x \in \mathscr{F}$
contains a rational number.
\end{enumerate}
$\Box$ \\\\



%%%%%%%%%%%%%%%%%%%%%%%%%%%%%%%%%%%%%%%%%%%%%%%%%%%%%%%%%%%%%%%%%%%%%%%%%%%%%%%%



\textbf{Exercise 2.30.}
\emph{Imitate the proof of Theorem 2.43 to obtain the following result:}
\begin{quote}
\emph{If $\mathbb{R}^k = \bigcup_{n=1}^{\infty} F_n$,
where each $F_n$ is a closed subset of $\mathbb{R}^k$,
then at least one $F_n$ has a nonempty interior.} \\

\emph{Equivalent statement: If $G_n$ is a dense open subset of $\mathbb{R}^k$,
for $n = 1,2,3,\ldots$, then $\bigcap_{n=1}^{\infty} G_n$ is not empty
(in fact, it is dense in $\mathbb{R}^k$).}
\end{quote}
\emph{(This is a special case of Baire's theorem; see Exercise 3.22 for the general case.)} \\

\textbf{Baire category theorem.}
\emph{
If $G_n$ is a dense open subset of $\mathbb{R}^k$,
for $n = 1,2,\ldots$, then $$\bigcap_{n=1}^{\infty} G_n$$
is dense in $\mathbb{R}^k$.} \\

\emph{Proof of Baire category theorem.}
Given any open set $G_0$ in $\mathbb{R}^k$,
will show that $$\bigcap_{n=0}^{\infty} G_n \neq \varnothing.$$
\begin{enumerate}
\item[(1)]
Since $G_1$ is dense, $G_0 \cap G_1$ is nonempty.
Take any one point $\mathbf{x}_1$ in the open set $G_0 \cap G_1$,
then there exists an open neighborhood
$$V_1
= \{ \mathbf{y} \in \mathbb{R}^k : |\mathbf{y} - \mathbf{x}_1| < r_1 \}$$
of $\mathbf{x}_1$
such that
$$\overline{V_1}
= \{ \mathbf{y} \in \mathbb{R}^k : |\mathbf{y} - \mathbf{x}_1| \leq r_1 \}
\subseteq G_0 \cap G_1.$$
\item[(2)]
Suppose $V_n$ has been constructed,
take any one point $\mathbf{x}_{n+1}$ in the open set $V_n \cap G_{n+1}$,
then there exists an open neighborhood
$$V_{n+1}
= \{ \mathbf{y} \in \mathbb{R}^k : |\mathbf{y} - \mathbf{x}_{n+1}| < r_{n+1} \}$$
of $\mathbf{x}_{n+1}$ with $r_{n+1}$
such that
$$\overline{V_{n+1}}
= \{ \mathbf{y} \in \mathbb{R}^k : |\mathbf{y} - \mathbf{x}_{n+1}| \leq r_{n+1} \}
\subseteq V_n \cap G_{n+1}.$$
\item[(3)]
Note that
  \begin{enumerate}
  \item[(a)]
  each $\overline{V_n}$ is nonempty (containing $\mathbf{x}_n$) and compact.
  \item[(b)]
  $\overline{V_1} \supseteq \overline{V_2} \supseteq \cdots$
  (since
  $\overline{V_{n+1}} \subseteq V_n \cap G_{n+1} \subseteq V_n \subseteq \overline{V_n}$).
  \end{enumerate}
By Corollary to Theorem 2.36,
$$\bigcap_{n=1}^{\infty} \overline{V_n} \neq \varnothing.$$
\item[(4)]
Pick $\mathbf{x} \in \bigcap_{n=1}^{\infty} \overline{V_n}$.
Hence
\begin{align*}
\mathbf{x} \in \bigcap_{n=1}^{\infty} \overline{V_n}
&\Longleftrightarrow
\mathbf{x} \in \overline{V_n} \text{ for all } n=1,2,3,\ldots \\
&\Longrightarrow
\mathbf{x} \in \overline{V_1} \subseteq G_0 \cap G_1 \text{ and }
\mathbf{x} \in \overline{V_{n+1}} \subseteq V_n \cap G_{n+1} \subseteq G_{n+1} \\
&\Longrightarrow
\mathbf{x} \in G_0 \cap G_1 \cap \cdots = \bigcap_{n=0}^{\infty} G_n \\
&\Longrightarrow
\bigcap_{n=0}^{\infty} G_n \neq \varnothing.
\end{align*}
\end{enumerate}
$\Box$ \\\\



% No exercise left.

%%%%%%%%%%%%%%%%%%%%%%%%%%%%%%%%%%%%%%%%%%%%%%%%%%%%%%%%%%%%%%%%%%%%%%%%%%%%%%%%
%%%%%%%%%%%%%%%%%%%%%%%%%%%%%%%%%%%%%%%%%%%%%%%%%%%%%%%%%%%%%%%%%%%%%%%%%%%%%%%%



\end{document}