\documentclass{article}
\usepackage{amsfonts}
\usepackage{amsmath}
\usepackage{amssymb}
\usepackage{hyperref}
\usepackage{mathrsfs}
\parindent=0pt

\def\upint{\mathchoice%
    {\mkern13mu\overline{\vphantom{\intop}\mkern7mu}\mkern-20mu}%
    {\mkern7mu\overline{\vphantom{\intop}\mkern7mu}\mkern-14mu}%
    {\mkern7mu\overline{\vphantom{\intop}\mkern7mu}\mkern-14mu}%
    {\mkern7mu\overline{\vphantom{\intop}\mkern7mu}\mkern-14mu}%
  \int}
\def\lowint{\mkern3mu\underline{\vphantom{\intop}\mkern7mu}\mkern-10mu\int}

\begin{document}

\textbf{\Large Chapter 2: Basic Topology} \\\\



\emph{Author: Meng-Gen Tsai} \\
\emph{Email: plover@gmail.com} \\\\



\textbf{Exercise 2.1.}
\emph{Prove that the empty set is a subset of every set.} \\

\emph{Proof.}
By Definitions 1.3,
\begin{enumerate}
\item[(1)]
The set which contains no element will be called the \textbf{empty set},
\item[(2)]
If $A$ and $B$ are sets, and if every element of $A$ is an element of $B$,
we say that $A$ is a \textbf{subset} of $B$,
\end{enumerate}
every element of the empty set (there are none) belongs to every set.
That is, the empty set is a subset of every set.
$\Box$ \\\\



\textbf{Exercise 2.2.}
\emph{A complex number $z$ is said to be algebraic if there are integers
$a_0, ..., a_n$, not all zero, such that
$$a_0 z^n + a_1 z^{n-1} + \cdots + a_{n-1} z + a_n = 0.$$
Prove that the set of all algebraic numbers is countable.
(Hint: For every positive integer $N$ there are only finitely many equations with
$$n + |a_0| + |a_1| + \cdots + |a_n| = N.$$}

Might assume $a_0 \neq 0$. \\

For example, all rational numbers are algebraic
since $p = \frac{\alpha}{\beta}$ (where $\alpha, \beta \in \mathbb{Z}$)
is a root of $\beta z - \alpha = 0$. \\

Besides, $z = \sqrt{2} + \sqrt{3}$ is algebraic since $z^4 - 10z^2 + 1 = 0$.
In fact, $z = \pm\sqrt{2} + \pm\sqrt{3}$ are also algebraic since
$z^4 - 10z^2 + 1 =
(z - \sqrt{2} - \sqrt{3})(z + \sqrt{2} - \sqrt{3})
(z - \sqrt{2} + \sqrt{3})(z + \sqrt{2} + \sqrt{3})$. \\

\textbf{Lemma.}
\emph{The set of all polynomials over $\mathbb{Z}$ is countable implies that
the set of algebraic numbers is countable.} \\

\emph{Proof of Lemma.}
By definition, we write the set of algebraic numbers as
$$S = \bigcup_{f(x) \in \mathbb{Z}[x]} \{ z \in \mathbb{C} : f(z) = 0 \}.$$
Since each polynomial of degree $n$ has at most $n$ roots,
$\{ z \in \mathbb{C} : f(z) = 0 \}$ is finite for each given $f(x) \in \mathbb{Z}[x]$.
So $S$ is a countable union (by assumption) of finite sets, and hence countable.
($S$ is infinite since
every integer $\alpha$ is a root of $f(z) = z - \alpha$.)
$\Box$ \\

Thus, it suffices to show that
\emph{the set of all polynomials over $\mathbb{Z}$ is countable.} \\

\emph{Proof (Hint).}
For every positive integer $N$ there are only finitely many equations with
$n + |a_0| + |a_1| + \cdots + |a_n| = N.$
Write
$$P_N = \{ f(x) \in \mathbb{Z}[x] : n + |a_0| + |a_1| + \cdots + |a_n| = N \}$$
where $f(x) = a_0 z^n + a_1 z^{n-1} + \cdots + a_{n-1} z + a_n$ with $a_0 \neq 0$,
and
$$P = \bigcup_{N = 1}^{\infty} P_N.$$
$P$ is the set of all polynomials over $\mathbb{Z}$. \\

Each $P_N$ is finite for given $N$
(since the equation $n + |a_0| + |a_1| + \cdots + |a_n| = N$
has finitely many solutions
$(n, a_0, a_1, ..., a_n) \in \mathbb{Z}^{n+2}$).
So $P$ is a countable union of finite sets, and hence countable.
($P$ is infinite since
$\mathbb{Z}$ is a subring of $\mathbb{Z}[x]$.)
$\Box$ \\

\emph{Proof (Theorem 2.13).}
\begin{enumerate}
\item[(1)]
\emph{$\mathbb{Z}^N$ is countable for any integer $N > 0$.}
\item[(2)]
\emph{The set of all polynomials over $\mathbb{Z}$ is countable.}
Let
$$P_n = \{ f \in \mathbb{Z}[x] : \deg f = n \},$$
and
$$P = \bigcup_{n = 1}^{\infty} P_n = \mathbb{Z}[x].$$

\emph{Claim: $P_n$ is countable.}
Define a 1-1 correspondence $\varphi_n: P_n \rightarrow \mathbb{Z}^{n+1}$ by
$$\varphi_n(a_0 z^n + a_1 z^{n-1} + \cdots + a_n)
= (a_0, a_1, ..., a_{n-1}, a_n).$$
By (1), $P_n$ is countable.
Now $P$ is a countable union of countable sets,
and hence countable by Theorem 2.12.
\end{enumerate}
$\Box$ \\

\emph{Proof (Unique factorization theorem).}
\begin{enumerate}
\item[(1)]
\emph{The set of prime numbers is countable.}
Write all primes in the ascending order as $p_1, p_2, ..., p_n, ...$
where $p_1 = 2, p_2 = 3, ..., p_{10001} = 104743, ...$
(See \href{https://projecteuler.net/problem=7}{ProjectEuler 7: 10001st prime}.
Use sieve of Eratosthenes to get $p_{10001}$.)
\item[(2)]
\emph{The set of all polynomials over $\mathbb{Z}$ is countable.}
Let
$$P_n = \{ f \in \mathbb{Z}[x] : \deg f = n \},$$
and
$$P = \bigcup_{n = 1}^{\infty} P_n = \mathbb{Z}[x].$$

\emph{Claim: $P_n$ is countable.}
Define a map $\varphi_n: P_n \rightarrow \mathbb{Z}^+$ by
$$\varphi_n(a_0 z^n + a_1 z^{n-1} + \cdots + a_n)
= p_1^{\psi(a_0)} p_2^{\psi(a_1)} \cdots p_{n+1}^{\psi(a_n)},$$
where $\psi$ is a 1-1 correspondence from $\mathbb{Z}$ to $\mathbb{Z}^+$ (Example 2.5).
By the unique factorization theorem, $\varphi_n$ is 1-1.
(In fact, $\varphi_n$ is a 1-1 correspondence from $\mathbb{Z}^{n+1}$ to $\mathbb{Z}^+$.)
So $P_n$ is countable.
Now $P$ is a countable union of countable sets,
and hence countable by Theorem 2.12.
\end{enumerate}
$\Box$ \\\\



\textbf{Exercise 2.10.}
\emph{Let $X$ be an infinite set. For $p \in X$ and $q \in X$, define
\begin{equation*}
  d(p, q) =
    \begin{cases}
      1 & (\text{if } p \neq q) \\
      0 & (\text{if } p = q).
    \end{cases}
\end{equation*}
Prove that this is a metric.
Which subsets of the resulting metric space are open?
Which are closed?
Which are compact?} \\

(The statement holds even if $X$ is finite.)
We called $d$ the discrete metric,
and the corresponding topology on $X$ induces the discrete topology.
Conversely, if $X$ has the discrete topology, $X$ is always metrizable by the discrete metric. \\

\emph{Proof.}
\begin{enumerate}
\item[(1)] \emph{$d(p, q)$ is a metric.}
\begin{enumerate}
\item[(a)] \emph{$d(p, q) > 0$ if $p \neq q$; $d(p, p) = 0$.} Trivial.
\item[(b)] \emph{$d(p, q) = d(q, p)$.} Trivial.
\item[(c)] \emph{$d(p, q) \leq d(p, r) + d(r, q)$ for any $r \in X$.}
If $p = q$, nothing to do. If $p \neq q$, $r \neq p$ or $r \neq q$ for any $r \in X$.
(Assume not true, $r = p$ and $r = q$ implies that $p = q$ which is a contradiction.)
In any cases $d(p, r) + d(r, q) \geq 1 = d(p, q)$.
\end{enumerate}
\item[(2)] \emph{Every subset is open.}
Let $E$ be any subset of $X$.
Then every point $p \in E$ is an interior point of $E$.
In fact, we can pick one neighborhood $N_{\frac{1}{2}}(p)$ of $p$
containing only one point $p \in E$ or $N_{\frac{1}{2}}(p) = \{ p \}$,
and such neighborhood $N_{\frac{1}{2}}(p)$ is a subset of $E$.
So every subset of $X$ is open.
\item[(3)] \emph{Every subset is closed.}
Since every subset is open, every subset is closed by Theorem 2.23. \\

\textbf{Supplement.}
Might use Definition 2.18 (d) to prove directly since there are no limit points in $X$
if we consider one neighborhood $N_{\frac{1}{2}}(p)$ of $p$.
Therefore, every subset is closed.
Again we apply Theorem 2.23 to get that every subset is open
without using Definition 2.18 (f).
\item[(4)] \emph{A subset is compact iff it is finite.}
\begin{enumerate}
\item[(a)]
\emph{Any finite subset is compact.}
Say $E = \{p_1, p_2, ..., p_k\}$, and $\{ G_{\alpha} \}$ be an open covering of $E$.
From $\{ G_{\alpha} \}$
we pick
$G_{\alpha_1}$ containing $p_1$,
$G_{\alpha_2}$ containing $p_2$, ..., and
$G_{\alpha_k}$ containing $p_k$.
This process can be done in the finitely many steps.
Therefore,
$$\{ G_{\alpha_1}, G_{\alpha_2}, ..., G_{\alpha_k} \}$$
is a finite subcovering of $\{ G_{\alpha} \}$ covering $E$.
\item[(b)]
\emph{Any infinite subset is not compact.}
Take a collection
$$\mathscr{G} = \left\{ G_p = \{ p \} \right\}$$
of open subsets where $p$ runs all points in $E$.
Clearly, $\{ G_p \}$ is an open covering.
Assume
$$\mathscr{G}' = \left\{ G_{p_1}, G_{p_2}, ..., G_{p_k} \right\}$$
is any finite subcovering of $\mathscr{G}$.
Since $E$ is infinite, there exist a point $p \in E$ such that
$p \neq p_1$,
$p \neq p_2$, ...,
$p \neq p_k$.
Therefore, $\mathscr{G}'$ does not cover $p$,
or $\mathscr{G}$ does not contains any finite subcovering $\mathscr{G}'$.
\end{enumerate}
\end{enumerate}
$\Box$ \\

Notice that every subset is bounded.
Therefore, every subset is closed and bounded,
but only finite subset is compact, i.e.,
Heine-Borel theorem is not true in the infinite discrete topology. \\\\


\textbf{Exercise 2.12.}
\emph{Let $K \subseteq \mathbb{R}^1$
consist of $0$ and the numbers $\frac{1}{n}$, for $n = 1, 2, 3, ...$.
Prove that $K$ is compact directly from the definition
(without using the Heine-Borel theorem).} \\

\emph{Proof.}
Let $\{ G_{\alpha} \}$ be an open covering of $K$.
There is an open set $G_0 \in \{ G_{\alpha} \}$ containing $0$.
So there exists a neighborhood $N_r(0)$ of $0$ such that $N_r(0) \subseteq G_0$.
So $N_r(0)$ contains all points $q = \frac{1}{n}$ of $K$ whenever $n > \frac{1}{r}$.
To construct a finite subcovering of $\{ G_{\alpha} \}$,
we need to pick finitely many open sets from $\{ G_{\alpha} \}$
to cover the remaining points $q = \frac{1}{n}$ where $n = 1, 2, ..., \left[\frac{1}{r}\right]$,
say $G_1$ contains $q = \frac{1}{1}$, $G_2$ contains $q = \frac{1}{2}$, ...,
$G_{\left[\frac{1}{r}\right]}$ contains $q = \frac{1}{\left[\frac{1}{r}\right]}$.
(Might be duplicated.)
Hence,
$$\left\{ G_0, G_1, G_2, ..., G_{\left[\frac{1}{r}\right]} \right\}$$
is a finite subcovering of $\{ G_{\alpha} \}$ covering $K$.
$\Box$ \\

\emph{Proof (Heine-Borel theorem).}
\begin{enumerate}
\item[(1)]
$K$ is closed. In fact, the only limit point of $K$ is $0$, which is in $K$.
\begin{enumerate}
\item[(a)]
\emph{$p = 0$ is a limit point.}
Given $r > 0$.
There always exists $n \in \mathbb{Z}^{+}$ such that $r > \frac{1}{n}$.
So any neighborhood $N_r(0)$ of $p = 0$
contains at least one point $q = \frac{1}{n} \neq 0$ in $K$.
\item[(b)]
\emph{$p < 0$ is not a limit point.}
Pick a neighborhood $N_{r}(p)$ of $p$ where $r = |p| > 0$.
Then $N_{r}(p) \cap K = \varnothing$.
\item[(c)]
\emph{$p > 0$ is not a limit point.}
There always exists $m \in \mathbb{Z}^+$ such that $p > \frac{1}{n}$ whenever $n \geq m$.
Pick a neighborhood $N_{r}(p)$ of $p$ where $r = p - \frac{1}{m} > 0$.
Then $N_{r}(p)$ does not have all points $q = \frac{1}{n} \in K$ whenever $n \geq m$.
By Theorem 2.20, $p$ cannot be a limit point of $K$.
\end{enumerate}
\item[(2)]
$K$ is bounded. There is a real number $M = 2$ and a point $q = 0 \in \mathbb{R}^1$
such that $|p - q| = |p| < 2$ for all $p \in K$.
\end{enumerate}
By Heine-Borel theorem, $K$ is compact in $\mathbb{R}^1$.
$\Box$ \\\\



\textbf{Exercise 2.14.}
\emph{Give an example of an open cover of the segment $(0, 1)$
which has no finite subcover.} \\

\emph{Proof.}
In $\mathbb{R}^1$, take a collection
$$\mathscr{G} = \left\{ G_n = \left(\frac{1}{n}, 1\right) \right\}$$
of open subsets where $n \in \mathbb{Z}^+$.
\begin{enumerate}
\item[(1)]
\emph{$\mathscr{G}$ is an open covering of $(0, 1) \subseteq \mathbb{R}^1$.}
Actually, given $x \in (0, 1)$, there exists an positive integer $n$ such that $x > \frac{1}{n}$.
That is, $x \in \left(\frac{1}{n}, 1\right) = G_n$.
\item[(2)]
\emph{There is no finite subcovering of $\mathscr{G}$.}
Assume $$\mathscr{G}' = \left\{ G_{n_1}, G_{n_2}, ..., G_{n_k} \right\}$$
is any finite subcovering of $\mathscr{G}$ where $n_1 < n_2 < ... < n_k$.
Take $x \in \left(0, \frac{1}{n_k}\right) \neq \varnothing$,
$x = \frac{1}{2 n_k}$ for example.
Then $x \not\in G_{n_1}$, $x \not\in G_{n_1}$, ..., $x \not\in G_{n_k}$,
which contradicts that $\mathscr{G}'$ is a finite subcovering of $\mathscr{G}$ covering $(0, 1)$.
\end{enumerate}
$\Box$ \\\\



\end{document}