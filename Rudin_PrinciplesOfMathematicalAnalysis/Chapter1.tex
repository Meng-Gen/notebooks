\documentclass{article}
\usepackage{amsfonts}
\usepackage{amsmath}
\usepackage{amssymb}
\usepackage{hyperref}
\usepackage[none]{hyphenat}
\usepackage{mathrsfs}
\usepackage{physics}
\parindent=0pt

\def\upint{\mathchoice%
    {\mkern13mu\overline{\vphantom{\intop}\mkern7mu}\mkern-20mu}%
    {\mkern7mu\overline{\vphantom{\intop}\mkern7mu}\mkern-14mu}%
    {\mkern7mu\overline{\vphantom{\intop}\mkern7mu}\mkern-14mu}%
    {\mkern7mu\overline{\vphantom{\intop}\mkern7mu}\mkern-14mu}%
  \int}
\def\lowint{\mkern3mu\underline{\vphantom{\intop}\mkern7mu}\mkern-10mu\int}

\begin{document}



\textbf{\Large Chapter 1: The Real and Complex Number Systems} \\\\



\emph{Author: Meng-Gen Tsai} \\
\emph{Email: plover@gmail.com} \\\\



Unless the contrary is explicitly stated, all numbers that are mentioned in these exercise
are understood to be real. \\\\



%%%%%%%%%%%%%%%%%%%%%%%%%%%%%%%%%%%%%%%%%%%%%%%%%%%%%%%%%%%%%%%%%%%%%%%%%%%%%%%%



\textbf{Exercise 1.1.}
\emph{If $r$ is a rational ($r \neq 0$) and $x$ is irrational,
prove that $r + x$ and $rx$ are irrational.} \\

\emph{Proof.}
Assume $r + x \in \mathbb{Q}$.
$\mathbb{Q}$ is a field, then $-r \in \mathbb{Q}$ for any $r \in \mathbb{Q}$.
So $(-r) + (r + x) = (-r + r) + x = 0 + x = x \in \mathbb{Q}$, a contradiction. \\

Similarly, assume $rx \in \mathbb{Q}$. $r \in \mathbb{Q}$ with $r \neq 0$ implies that
there exists an element $1/r \in \mathbb{Q}$ such that $r \cdot (1/r) = 1$.
So $(1/r) \cdot (rx) = ((1/r) \cdot r) \cdot x = 1 \cdot x = x \in \mathbb{Q}$, a contradiction.
$\Box$ \\\\



%%%%%%%%%%%%%%%%%%%%%%%%%%%%%%%%%%%%%%%%%%%%%%%%%%%%%%%%%%%%%%%%%%%%%%%%%%%%%%%%



\textbf{Exercise 1.2.}
\emph{Prove that there is no rational number whose square is $12$.} \\

Apply the argument in Example 1.1.
Again we can examine this situation a little more closely.
Let $A$ be the set of all positive rational $p$ such that $p^2 < 12$ and
let $B$ be the set of all positive rational $p$ such that $p^2 > 12$.
We might show that
\emph{$A$ contains no largest number and
$B$ contains no largest number} again. \\

In fact, we can associate with each rational $p > 0$ the number
$$q = p - \frac{p^2 - 12}{p + 12} = \frac{12p + 12}{p + 12}.$$
Then
$$q^2 - 12 = \frac{132(p^2 - 12)}{(p + 12)^2}.$$
If $p \in A$ then $p^2 - 12 < 0$, $q > p$ and $q^2 < 12$. Thus $q \in A$.
If $p \in B$ then $p^2 - 12 > 0$, $0 < q < p$ and $q^2 > 12$. Thus $q \in B$. \\

\emph{Proof (Example 1.1).}
We now show that the equation
$$p^2 = 12$$
is not satisfied by any rational $p$.
If there were such a $p \in \mathbb{Q}$,
we could write $p = \frac{m}{n}$ where $m, n \in \mathbb{Z}$ are \emph{relatively prime}.
Let us assume this is done. Then $p^2 = 12$ implies
$$m^2 = 12 n^2.$$
This shows that $3 \mid m^2$. Hence $3 \mid m$ (since $3$ is a prime in $\mathbb{Z}$),
and so $m^2$ is divisible by $9$.
It follows that $12n^2$ is divisible by $9$,
so that $4n^2$ is divisible by $3$,
so that $n^2$ is divisible by $3$,
which implies that $3 \mid n$.
That is, both $m$ and $n$ have a common factor $3 > 1$,
contrary to our choice of $m$ and $n$.
Hence $p^2 = 12$ is impossible for rational $p$.
$\Box$ \\\\



%%%%%%%%%%%%%%%%%%%%%%%%%%%%%%%%%%%%%%%%%%%%%%%%%%%%%%%%%%%%%%%%%%%%%%%%%%%%%%%%



\textbf{Exercise 1.3.}
\emph{Prove Proposition 1.15.} \\

\textbf{Proposition 1.15.}
\emph{The axioms for multiplication imply the following statements.}
\begin{enumerate}
\item[(a)]
\emph{If $x \neq 0$ and $xy=xz$ then $y = z$.}
\item[(b)]
\emph{If $x \neq 0$ and $xy=x$ then $y = 1$.}
\item[(c)]
\emph{If $x \neq 0$ and $xy=1$ then $y = 1/x$.}
\item[(d)]
\emph{If $x \neq 0$ then $1(1/x) = x$.} \\
\end{enumerate}

\emph{Proof of (a).}
By the axioms for multiplication,
\begin{align*}
xy = xz, x \neq 0
&\Longrightarrow
\exists 1/x \in F, (1/x) \cdot (xy) = (1/x) \cdot (xz)
  &\text{(M5)} \\
&\Longrightarrow
((1/x)x)y = ((1/x)x)z
  &\text{(M3)} \\
&\Longrightarrow
(x(1/x))y = (x(1/x))z
  &\text{(M2)} \\
&\Longrightarrow
1y = 1z \\
&\Longrightarrow
y = z.
  &\text{(M4)}
\end{align*}
$\Box$ \\

\emph{Proof of (b).}
Let $z = 1$ in (a) and note that $x1 = 1x = x$ ((M2)(M4)).
$\Box$ \\

\emph{Proof of (c).}
Let $z = 1/x$ in (a) and note that $x(1/x) = 1$ ((M5)).
$\Box$ \\

\emph{Proof of (d).}
Since $x(1/x) = (1/x)x = 1$ ((M2)), by (c), $x = 1/(1/x)$.
$\Box$ \\\\



%%%%%%%%%%%%%%%%%%%%%%%%%%%%%%%%%%%%%%%%%%%%%%%%%%%%%%%%%%%%%%%%%%%%%%%%%%%%%%%%



\textbf{Exercise 1.4.}
\emph{Let $E$ be a nonempty subset of an ordered set;
suppose $\alpha$ is a lower bound of $E$ and
$\beta$ is an upper bound of $E$.
Prove that $\alpha \leq \beta$.} \\

\emph{Proof.}
\begin{enumerate}
\item[(1)]
Since $E \neq \varnothing$, there is $y \in E$.
\item[(2)]
By the definition of the upper bound,
$x \leq \beta$ for every $x \in E$.
In particular, $y \leq \beta$.
\item[(3)]
Similarly, $y \geq \alpha$.
\item[(4)]
By (2)(3), $\alpha \leq y \leq \beta$ for some $y \in E$.
In particular, $\alpha \leq \beta$ (Definition 1.5(ii)).
\end{enumerate}
$\Box$ \\\\



%%%%%%%%%%%%%%%%%%%%%%%%%%%%%%%%%%%%%%%%%%%%%%%%%%%%%%%%%%%%%%%%%%%%%%%%%%%%%%%%



\textbf{Exercise 1.5.}
\emph{Let $A$ be a nonempty set of real numbers which is bounded below.
Let $-A$ be the set of all numbers $-x$, where $x \in A$.
Prove that $$\inf A = -\sup(-A).$$}

\emph{Proof.}
Let $\alpha = \inf A$ and $\beta = \sup(-A)$.
\begin{enumerate}
\item[(1)]
\begin{align*}
x \geq \alpha \:\: \forall x \in A
&\Longrightarrow
-x \leq -\alpha \:\: \forall -x \in -A \\
&\Longrightarrow
-\alpha \text{ is an upper bound of $-A$} \\
&\Longrightarrow
\beta \leq -\alpha \\
&\Longrightarrow
\alpha \leq -\beta
\end{align*}
\item[(2)]
\begin{align*}
-x \leq \beta \:\: \forall -x \in -A
&\Longrightarrow
x \geq -\beta \:\: \forall x \in A \\
&\Longrightarrow
-\beta \text{ is a lower bound of $A$} \\
&\Longrightarrow
\alpha \geq -\beta
\end{align*}
\end{enumerate}
By (1)(2), $\alpha = -\beta$, or $\inf A = -\sup(-A)$.
$\Box$ \\\\



%%%%%%%%%%%%%%%%%%%%%%%%%%%%%%%%%%%%%%%%%%%%%%%%%%%%%%%%%%%%%%%%%%%%%%%%%%%%%%%%



\textbf{Exercise 1.6.}
\emph{Fix $b > 1$.}
\begin{enumerate}
\item[(a)]
\emph{If $m,n,p,q$ are integers, $n>0$, $q>0$, and $r=m/n=p/q$, prove that
$$(b^m)^{1/n} = (b^p)^{1/q}.$$
Hence it makes sense to define $b^r = (b^m)^{1/n}$.}
\item[(b)]
\emph{Prove that $b^{r+s} = b^r b^s$ if $r$ and $s$ are rational.}
\item[(c)]
\emph{If $x$ is real, define $B(x)$ to be the set of all numbers $b^t$,
where $t$ is rational and $t \leq x$.
Prove that
$$b^r = \sup B(r)$$
where $r$ is rational.
Hence it makes sense to define
$$b^x = \sup B(x)$$
for every real $x$.}
\item[(d)]
\emph{Prove that $b^{x+y} = b^x b^y$ for all real $x$ and $y$.} \\
\end{enumerate}

\emph{Proof of (a).}
\begin{enumerate}
\item[(1)]
Define $k = mq = np \in \mathbb{Z}$ (since $r = m/n = p/q$).
Notice that $nq > 0$ (since $n>0$ and $q>0$).
So there is one and only one $y \in \mathbb{R}$ such that $$y^{nq} = b^k$$
where $b^k$ is defined in $\mathbb{R}$ (Theorem 1.21).
\item[(2)]
\emph{Show that $y = (b^m)^{1/n}$ and $y = (b^p)^{1/q}$
are solutions of $y^{nq} = b^k$.}
In fact,
\begin{align*}
((b^m)^{1/n})^{nq} &= (b^m)^q = b^{mq} = b^k, \\
((b^p)^{1/q})^{nq} &= (b^p)^n = b^{pn} = b^k.
\end{align*}
\item[(3)]
By (1)(2), the uniqueness of $y$ shows that $(b^m)^{1/n} = (b^p)^{1/q}$,
or the map $r \mapsto b^r$ is well-defined for $r \in \mathbb{Q}$.
\end{enumerate}
$\Box$ \\

\emph{Proof of (b).}
Write $r=m/n$ and $s=p/q$ where $m,n,p,q$ are integers with $n>0$, $q>0$.
\begin{align*}
b^{r+s}
&= b^{\frac{mq+np}{nq}} \\
&= (b^{mq} \cdot b^{np})^{\frac{1}{nq}}
  &\text{($mq+np \in \mathbb{Z}$)} \\
&= (b^{mq})^{\frac{1}{nq}} \cdot (b^{np})^{\frac{1}{nq}}
  &\text{(Corollary to Theorem 1.21)} \\
&= b^{\frac{mq}{nq}} \cdot b^{\frac{np}{nq}} \\
&= b^{\frac{m}{n}} \cdot b^{\frac{n}{n}}
  &\text{((a))} \\
&= b^r \cdot b^s.
\end{align*}
$\Box$ \\

\emph{Proof of (c).}
\begin{enumerate}
\item[(1)]
Given any $r \in \mathbb{Q}^+$, $b^r > 1$ since $b > 1$ is given.
\item[(2)]
Given any $r, s \in \mathbb{Q}$, $b^r > b^s$ whenever $r > s$.
In fact,
\begin{align*}
b^r
&= b^{r-s} b^s
  &\text{((b))} \\
&> 1 \cdot b^s
  &\text{((1))} \\
&= b^s.
\end{align*}
\item[(3)]
Given any $r \in \mathbb{Q}$, $b^t \leq b^r$ for any $t \in \mathbb{Q}$
whenever $t \leq r$.
So $\sup B(r) \leq b^r$.
Conversely, since $r \in B(r)$, $b^r \leq \sup B(r)$.
So $b^r = \sup B(r)$.
\item[(4)]
Given any $x \in \mathbb{R}$.
We can always find $r, s \in \mathbb{Q}$ such that $r < x < s$.
Therefore, $r \in B(x)$ and $B(s)$ is an upper bound of $B(x)$.
So there is a least upper bound $\sup B(x)$ for $B(x)$, i.e.,
$b^r = \sup B(r)$ is well-defined.
\end{enumerate}
$\Box$ \\

\textbf{Lemma.}
\emph{If $x$ is real, define $B'(x)$ to be the set of all numbers $b^t$,
where $t$ is rational and $t < x$.
Prove that $\sup B'(x) = \sup B(x)$ for all $x \in \mathbb{R}$.} \\

\emph{Proof of Lemma (Reductio ad absurdum).}
It suffices to show that $\sup B'(r) = \sup B(r) = b^r$ for all $r \in \mathbb{Q}$.
(The case $x \in \mathbb{R}-\mathbb{Q}$ is nothing to do.)
Clearly, $\sup B'(r) \leq b^r$.
If $\alpha = \sup B'(r) < b^r$, then for $\frac{b^r}{\alpha} > 1$ there is
$n > (b-1)/\left(\frac{b^r}{\alpha} - 1\right)$ such that
$$b^{\frac{1}{n}} < \frac{b^r}{\alpha}$$
(Exercise 1.7(c)).
So
$\alpha < b^{r - \frac{1}{n}}$.
Therefore, $b^{r - \frac{1}{n}} \in B'(r)$ since $r - \frac{1}{n} \in \mathbb{Q}$,
or we find an element in $B'(r)$ such that is greater than $\alpha$,
contrary to the maximality of $\alpha$.
$\Box$ \\

\emph{Proof of (d).}
Apply Lemma to use $B(x)$ or $B'(x)$ interchangeably.
\begin{enumerate}
\item[(1)]
\emph{Show that $$\sup B'(x+y) \leq \sup B'(x)\sup B'(y).$$ } \\
Given any $b^t \in B'(x+y)$ such that $t < x+y$.
There are rational numbers $r, s$ such that $r < x$, $s < y$ and $t=r+s$.
(Rewrite $t < x+y$ as $t-y < x$. So there is a rational number $r$ such that $t-y < r < x$.
Let $s = t-r < y$.)
(Here we use $B'(x+y)$ instead of $B(x+y)$ to ensure the existence of $r$ and $s$.
That is, if $0 = -\sqrt{2} + \sqrt{2}$, we cannot find rational numbers
$r \leq -\sqrt{2}$ and $s \leq \sqrt{2}$ such that $r + s = 0$.)
Therefore,
$$b^t = b^{r+s} = b^r b^s \leq \sup B'(x) \sup B'(y)$$
(by (b)). Take supremum, $\sup B'(x+y) \leq \sup B'(x) \sup B'(y)$.
\item[(2)]
\emph{Show that $$\sup B'(x+y) \geq \sup B'(x)\sup B'(y).$$ } \\
Given any $b^r \in B'(x)$, $b^s \in B'(y)$. $r < x$ and $s < y$.
So $b^r b^s = b^{r+s} \in B'(x+y)$ (by (b)).
So $b^r b^s \leq \sup B'(x+y)$.
So
$$b^r \leq \frac{\sup B'(x+y)}{b^s}$$
since $b^s > 0$ for any $s \in \mathbb{Q}$.
Here $\frac{\sup B'(x+y)}{b^s}$ is an upper bound for $B'(x)$.
So
$$\sup B'(x) \leq \frac{\sup B'(x+y)}{b^s},$$
or $b^s \leq \frac{\sup B'(x+y)}{B'(x)}$.
Use the same argument again,
$$\sup B'(y) \leq \frac{\sup B'(x+y)}{\sup B'(x)}$$
or $\sup B'(x) \sup B'(y) \leq \sup B'(x+y)$.
\end{enumerate}
By (1)(2), $\sup B'(x) \sup B'(y) = \sup B'(x+y)$ or $b^x b^y = b^{x+y}$.
$\Box$ \\\\



%%%%%%%%%%%%%%%%%%%%%%%%%%%%%%%%%%%%%%%%%%%%%%%%%%%%%%%%%%%%%%%%%%%%%%%%%%%%%%%%



\textbf{Exercise 1.8.}
\emph{Prove that no order can be defined in the complex field that turns it
into an ordered field.
(Hint: $-1$ is a square.)} \\

\emph{Proof (Reductio ad absurdum).}
If $\mathbb{C}$ were an ordered field, consider the complex number $i = \sqrt{-1}$.

\begin{enumerate}
\item[(1)]
$i \neq 0$.
If $i$ were $0$, then $i \cdot i = 0 \cdot i$ or $-1 = 0$,
or $1 = 0$, contrary to $1 > 0$ (Proposition 1.18).
\item[(2)]
Since $i \neq 0$, we have $i^2 > 0$ (Proposition 1.18).
So $-1 > 0$, or $1 < 0$, contrary to the fact $1 > 0$ (Proposition 1.18).
\end{enumerate}
$\Box$ \\

\textbf{Supplement ($x^2 > 0$ if $x \neq 0$).}
\emph{Show that the only automorphism of $\mathbb{R}$ is the identity.
(Hint: If $\sigma$ is an automorphism, show that $\sigma|_{\mathbb{Q}} = \text{id}$,
and if $a > 0$, then $\sigma(a) > 0$).} \\

It is an interesting fact that there are infinitely many automorphisms of $\mathbb{C}$,
even thought $[\mathbb{C}:\mathbb{R}] = 2$.
Why is this fact not a contradiction to this problem? \\\\



%%%%%%%%%%%%%%%%%%%%%%%%%%%%%%%%%%%%%%%%%%%%%%%%%%%%%%%%%%%%%%%%%%%%%%%%%%%%%%%%



\textbf{Exercise 1.11.}
\emph{If $z$ is a complex number, prove that there exists an $r \geq 0$
and a complex number $w$ with $|w| = 1$ such that $z = rw$.
Are $w$ and $r$ always uniquely determined by $z$?} \\

To decide $r$ and $w$ in the relation $z = rw$, it is natural to take
absolute values on the both sides. That is, $|z| = r|w| = r$. \\

\emph{Proof.}
Let $r = |z| \geq 0$.
\begin{enumerate}
\item[(1)]
$r \neq 0$.
Define $w = \frac{z}{r} \in \mathbb{C}$. $|w| = \frac{|z|}{r} = 1$.
In this case $w$ and $r$ are uniquely determined.
\item[(2)]
$r = 0$ (or $z = 0$).
Define $w = e^{ix} = \cos x + i \sin x$ for any $x \in \mathbb{R}$.
$|w| = 1$.
Here $r$ is uniquely determined
but $w$ is not uniquely determined.
\end{enumerate}
$\Box$ \\\\



%%%%%%%%%%%%%%%%%%%%%%%%%%%%%%%%%%%%%%%%%%%%%%%%%%%%%%%%%%%%%%%%%%%%%%%%%%%%%%%%



\textbf{Exercise 1.12.}
\emph{If $z_1, \ldots, z_n$ are complex, prove that
$$|z_1 + z_2 + \cdots + z_n| \leq |z_1| + |z_2| + \cdots + |z_n|.$$}

\emph{Proof.}
Use mathematical induction on $n$. $n = 2$ is established by Theorem 1.33 (e).
Suppose the inequality holds on $n = k$, then $n = k + 1$ we again apply Theorem 1.33 (e)
to get the result, say
\begin{align*}
|z_1 + z_2 + \cdots + z_k + z_{k+1}|
&\leq |z_1 + z_2 + \cdots + z_k| + |z_{k+1}| \\
&\leq |z_1| + |z_2| + \cdots + |z_k| + |z_{k+1}|
\end{align*}
$\Box$ \\

\textbf{Supplement.}
\emph{If $\mathbf{x}_1, \ldots, \mathbf{x_n} \in \mathbb{R}^k$, then
$$|\mathbf{x_1} + \mathbf{x_2} + \cdots + \mathbf{x_n}|
\leq |\mathbf{x_1}| + |\mathbf{x_2}| + \cdots + |\mathbf{x_n}|.$$}

Here we might use Theorem 1.37 (e) to prove it.
Since the norm $|\cdot|$ on $\mathbb{C}$ is the same as the norm on $\mathbb{R}^2$,
we might prove this supplement first
and then set $k = 2$ on $\mathbb{R}^k = \mathbb{R}^2$
to give another proof of Exercise 1.12. \\\\



%%%%%%%%%%%%%%%%%%%%%%%%%%%%%%%%%%%%%%%%%%%%%%%%%%%%%%%%%%%%%%%%%%%%%%%%%%%%%%%%



\textbf{Exercise 1.13.}
\emph{If $x, y$ are complex, prove that
$$\abs{ \abs{x} - \abs{y} } \leq \abs{x-y}.$$}

We can show $f(x) = |x|$ is uniformly continuous in $\mathbb{R}$ by using this inequality. \\

\emph{Proof (Exercise 1.12).}
Since
\begin{align*}
|y| &\leq |x| + |y-x| = |x| + |x-y| \\
|x| &\leq |y| + |x-y|,
\end{align*}
we have
$$-|x-y| \leq |x| - |y| \leq |x-y|,$$
or
$$\abs{|x| - |y|} \leq |x-y|.$$
$\Box$ \\\\



%%%%%%%%%%%%%%%%%%%%%%%%%%%%%%%%%%%%%%%%%%%%%%%%%%%%%%%%%%%%%%%%%%%%%%%%%%%%%%%%



\textbf{Exercise 1.14.}
\emph{If $z$ is a complex number such that $|z|=1$, that is, such that $z\overline{z}=1$,
compute
$$|1+z|^2+|1-z|^2.$$}

\emph{Proof ($|z|^2 = z\overline{z}$).}
\begin{align*}
|1+z|^2 &= (1+z)\overline{(1+z)} = (1+z)(1+\overline{z}) = 1+z+\overline{z}+z\overline{z} \\
|1-z|^2 &= (1-z)\overline{(1-z)} = (1+z)(1-\overline{z}) = 1-z-\overline{z}+z\overline{z} \\
|1+z|^2+|1-z|^2 &= 2+2z\overline{z} = 2+2 = 4.
\end{align*}
$\Box$ \\

\emph{Proof (Exercise 1.17).}
Regard $\mathbb{C}$ as $\mathbb{R}^2$.
Then put $\mathbf{x} = 1, \mathbf{y} = z$ in the parallelogram law (Exercise 1.17)
to get $$|1+z|^2+|1-z|^2 = 2|1|^2 + 2|z|^2 = 4.$$
$\Box$ \\\\



%%%%%%%%%%%%%%%%%%%%%%%%%%%%%%%%%%%%%%%%%%%%%%%%%%%%%%%%%%%%%%%%%%%%%%%%%%%%%%%%



\textbf{Exercise 1.15.}
\emph{Under what conditions does equality hold in the Schwarz inequality?} \\

\textbf{Theorem 1.35 (Schwarz inequality).}
\emph{If $a_1, \ldots, a_n$ and $b_1, \ldots, b_n$ are complex numbers, then
$$\abs{\sum_{j=1}^n a_j \overline{b_j}}^2
\leq \sum_{j=1}^n |a_j|^2 \sum_{j=1}^n|b_j|^2.$$}

In fact, the Lagrange's identity for complex numbers shows
$$\abs{ \sum_{k=1}^{n} a_k \overline{b_k} }^2
= \sum_{k=1}^{n} \abs{a_k}^2 \sum_{k=1}^{n} \abs{b_k}^2
- \sum_{1 \leq k < j \leq n}
\abs{ a_k b_j - a_j b_k }^2.$$ \\

In general, the Binet-Cauchy identity shows
\begin{align*}
&\sum_{1 \leq k < j \leq n}
(a_k b_j - a_j b_k)(A_k B_j - A_j B_k) \\
= &\left( \sum_{k=1}^{n} a_k A_k \right)\left( \sum_{k=1}^{n} b_k B_k \right)
- \left( \sum_{k=1}^{n} a_k B_k \right)\left( \sum_{k=1}^{n} b_k A_k \right).
\end{align*} \\

\emph{Proof of Binet-Cauchy identity.}
\begin{align*}
&\sum_{1 \leq k < j \leq n}
(a_k b_j - a_j b_k)(A_k B_j - A_j B_k) \\
= &\sum_{1 \leq k < j \leq n}
(a_k b_j A_k B_j + a_j b_k A_j B_k)
- \sum_{1 \leq k < j \leq n}
(a_k b_j A_j B_k - a_j b_k A_k B_j) \\
= &\sum_{1 \leq k < j \leq n}
(a_k A_k b_j B_j + a_j A_j b_k B_k)
- \sum_{1 \leq k < j \leq n}
(a_k B_k b_j A_j + a_j B_j b_k A_k) \\
= &\sum_{1 \leq k \neq j \leq n} a_k A_k b_j B_j
 - \sum_{1 \leq k \neq j \leq n} a_k B_k b_j A_j \\
= &\sum_{1 \leq k, j \leq n} a_k A_k b_j B_j
 - \sum_{1 \leq k, j \leq n} a_k B_k b_j A_j \\
  & \text{(since $a_k A_k b_j B_j - a_k B_k b_j A_j = 0$ as $k = j$)} \\
= &\left( \sum_{k=1}^{n} a_k A_k \right)\left( \sum_{j=1}^{n} b_j B_j \right)
- \left( \sum_{k=1}^{n} a_k B_k \right)\left( \sum_{j=1}^{n} b_j A_j \right) \\
= &\left( \sum_{k=1}^{n} a_k A_k \right)\left( \sum_{k=1}^{n} b_k B_k \right)
- \left( \sum_{k=1}^{n} a_k B_k \right)\left( \sum_{k=1}^{n} b_k A_k \right).
\end{align*}
$\Box$ \\

\emph{Proof of Lagrange's identity.}
Put $(a_k, b_k, A_k, B_k) \mapsto (a_k, b_k, \overline{a_k}, \overline{b_k})$
in the Binet-Cauchy identity.
$\Box$ \\

\emph{Proof of Schwarz inequality (Lagrange's identity).}
Notice the term $$\sum_{1 \leq k < j \leq n} \abs{ a_k b_j - a_j b_k }^2 \geq 0.$$
$\Box$ \\

Write $\mathbf{a} = (a_1, \ldots, a_n)$ and $\mathbf{b} = (b_1, \ldots, b_n)$
as two vectors in the vector space $\mathbb{C}^n$ over $\mathbb{C}$.
Back to the exercise now. \\

\emph{Proof (Lagrange's identity).}
$\sum_{1 \leq k < j \leq n} \abs{ a_k b_j - a_j b_k }^2 = 0$
$\Longleftrightarrow$
$a_k b_j = a_j b_k$ for any $1 \leq k < j \leq n$.
The equality holds in the Schwarz inequality
$\Longleftrightarrow$
$\mathbf{a}$ and $\mathbf{b}$ are linearly dependent.
$\Box$ \\

\emph{Proof (Theorem 1.35).}
The equality holds in the Schwarz inequality.
$\Longleftrightarrow$
$B = 0$ or
the term $\sum |B a_j - C b_j|^2$ in the proof of Theorem 1.35 is $0$.
$\Longleftrightarrow$
$\mathbf{b} = \mathbf{0}$ or $\mathbf{a} = c\mathbf{b}$ for some $c \in \mathbb{C}$.
$\Longleftrightarrow$
$\mathbf{a}$ and $\mathbf{b}$ are linearly dependent.
$\Box$ \\\\



%%%%%%%%%%%%%%%%%%%%%%%%%%%%%%%%%%%%%%%%%%%%%%%%%%%%%%%%%%%%%%%%%%%%%%%%%%%%%%%%



\textbf{Exercise 1.17.}
\emph{Prove that
$$|\mathbf{x}+\mathbf{y}|^2 + |\mathbf{x}-\mathbf{y}|^2
= 2|\mathbf{x}|^2 + 2|\mathbf{y}|^2$$
if $\mathbf{x} \in \mathbb{R}^k$ and $\mathbf{y} \in \mathbb{R}^k$.
Interpret this geometrically,
as a statement about parallelograms.} \\

\emph{Proof.}
\begin{align*}
&|\mathbf{x}+\mathbf{y}|^2 + |\mathbf{x}-\mathbf{y}|^2 \\
=& (\mathbf{x}+\mathbf{y})\cdot(\mathbf{x}+\mathbf{y})
  + (\mathbf{x}-\mathbf{y})\cdot(\mathbf{x}-\mathbf{y}) \\
=& (\mathbf{x}\cdot\mathbf{x} + 2\mathbf{x}\cdot\mathbf{y} + \mathbf{y}\cdot\mathbf{y})
  + (\mathbf{x}\cdot\mathbf{x} - 2\mathbf{x}\cdot\mathbf{y} + \mathbf{y}\cdot\mathbf{y}) \\
=& 2\mathbf{x}\cdot\mathbf{x} + 2\mathbf{y}\cdot\mathbf{y} \\
=& 2|\mathbf{x}|^2 + 2|\mathbf{y}|^2.
\end{align*}

Interpret this geometrically,
the sum of the squares of the lengths of the four sides of a parallelogram
equals the sum of the squares of the lengths of the two diagonals. \\

If the parallelogram is a rectangle, the two diagonals are of equal lengths,
so that the statement reduces to the Pythagorean theorem.
$\Box$ \\\\



%%%%%%%%%%%%%%%%%%%%%%%%%%%%%%%%%%%%%%%%%%%%%%%%%%%%%%%%%%%%%%%%%%%%%%%%%%%%%%%%



\textbf{Exercise 1.18.}
\emph{If $k \geq 2$ and $\mathbf{x} \in \mathbb{R}^k$,
prove that there exists $\mathbf{y} \in \mathbb{R}^k$ such that
$\mathbf{y} \neq 0$ but $\mathbf{x} \cdot \mathbf{y} = 0$.
Is this also true if $k = 1$?} \\

\emph{Proof.}
\begin{enumerate}
\item[(1)]
There are only two possible cases.
  \begin{enumerate}
  \item[(a)]
  \emph{$\exists \: i$ such that $x_i = 0$.}
  Let $\mathbf{y} = (0, \ldots, 0, 1, 0, \ldots, 0) \neq 0$
  whose entries are all $0$ except for a $1$ in the $i$-th position.
  So $\mathbf{x} \cdot \mathbf{y} = 0 + \ldots + 0 = 0$.
  \item[(b)]
  \emph{$\forall \: i, x_i \neq 0$.}
  Since $k \geq 2$, we can define
  $\mathbf{y} = (x_2, -x_1, 0, \ldots, 0) \neq 0.$
  So $\mathbf{x} \cdot \mathbf{y} = x_1 x_2 + x_2 (-x_1) + 0 + \ldots + 0 = 0$.
  \end{enumerate}
\item[(2)]
It is not true for $k = 1$ since $\mathbb{R}^1 = \mathbb{R}$ is a field. \\
\end{enumerate}
$\Box$ \\\\



%%%%%%%%%%%%%%%%%%%%%%%%%%%%%%%%%%%%%%%%%%%%%%%%%%%%%%%%%%%%%%%%%%%%%%%%%%%%%%%%



\end{document}