\documentclass{article}
\usepackage{amsfonts}
\usepackage{amsmath}
\usepackage{amssymb}
\usepackage{hyperref}
\usepackage{mathrsfs}
\parindent=0pt

\def\upint{\mathchoice%
    {\mkern13mu\overline{\vphantom{\intop}\mkern7mu}\mkern-20mu}%
    {\mkern7mu\overline{\vphantom{\intop}\mkern7mu}\mkern-14mu}%
    {\mkern7mu\overline{\vphantom{\intop}\mkern7mu}\mkern-14mu}%
    {\mkern7mu\overline{\vphantom{\intop}\mkern7mu}\mkern-14mu}%
  \int}
\def\lowint{\mkern3mu\underline{\vphantom{\intop}\mkern7mu}\mkern-10mu\int}

\begin{document}

\textbf{\Large Chapter 1: The Real and Complex Number Systems} \\\\



\emph{Author: Meng-Gen Tsai} \\
\emph{Email: plover@gmail.com} \\\\



Unless the contrary is explicitly stated, all numbers that are mentioned in these exercise
are understood to be real. \\\\



\textbf{Exercise 1.1.}
\emph{If $r$ is a rational ($r \neq 0$) and $x$ is irrational,
prove that $r + x$ and $rx$ are irrational.} \\

\emph{Proof.}
Assume $r + x \in \mathbb{Q}$.
$\mathbb{Q}$ is a field, then $-r \in \mathbb{Q}$ for any $r \in \mathbb{Q}$.
So $(-r) + (r + x) = (-r + r) + x = 0 + x = x \in \mathbb{Q}$, a contradiction. \\

Similarly, assume $rx \in \mathbb{Q}$. $r \in \mathbb{Q}$ with $r \neq 0$ implies that
there exists an element $1/r \in \mathbb{Q}$ such that $r \cdot (1/r) = 1$.
So $(1/r) \cdot (rx) = ((1/r) \cdot r) \cdot x = 1 \cdot x = x \in \mathbb{Q}$, a contradiction.
$\Box$ \\\\



\textbf{Exercise 1.12.}
\emph{If $z_1, ..., z_n$ are complex, prove that
$$|z_1 + z_2 + \cdots + z_n| \leq |z_1| + |z_2| + \cdots + |z_n|.$$}

\emph{Proof.}
Use mathematical induction on $n$. $n = 2$ is established by Theorem 1.33 (e).
Suppose the inequality holds on $n = k$, then $n = k + 1$ we again apply Theorem 1.33 (e)
to get the result, say
\begin{align*}
|z_1 + z_2 + \cdots + z_k + z_{k+1}|
&\leq |z_1 + z_2 + \cdots + z_k| + |z_{k+1}| \\
&\leq |z_1| + |z_2| + \cdots + |z_k| + |z_{k+1}|
\end{align*}
$\Box$ \\

\textbf{Corollary.}
\emph{If $\mathbf{x}_1, ..., \mathbf{x_n} \in \mathbb{R}^k$, then
$$|\mathbf{x_1} + \mathbf{x_2} + \cdots + \mathbf{x_n}|
\leq |\mathbf{x_1}| + |\mathbf{x_2}| + \cdots + |\mathbf{x_n}|.$$}

Here we might use Theorem 1.37 (e) to prove it.
Since the norm $|\cdot|$ on $\mathbb{C}$ is the same as the norm on $\mathbb{R}^2$,
we might prove this corollary first and set $k = 2$ on $\mathbb{R}^k = \mathbb{R}^2$.



\end{document}