\documentclass{article}
\usepackage{amsfonts}
\usepackage{amsmath}
\usepackage{amssymb}
\usepackage{hyperref}
\usepackage[none]{hyphenat}
\usepackage{mathrsfs}
\parindent=0pt

\def\upint{\mathchoice%
    {\mkern13mu\overline{\vphantom{\intop}\mkern7mu}\mkern-20mu}%
    {\mkern7mu\overline{\vphantom{\intop}\mkern7mu}\mkern-14mu}%
    {\mkern7mu\overline{\vphantom{\intop}\mkern7mu}\mkern-14mu}%
    {\mkern7mu\overline{\vphantom{\intop}\mkern7mu}\mkern-14mu}%
  \int}
\def\lowint{\mkern3mu\underline{\vphantom{\intop}\mkern7mu}\mkern-10mu\int}

\begin{document}

\textbf{\Large Chapter 1: A Special Case of Fermat's Conjecture} \\\\



\emph{Author: Meng-Gen Tsai} \\
\emph{Email: plover@gmail.com} \\\\



% https://www.math.uci.edu/~ndonalds/math180b/6gaussian.pdf
% https://www.maths.nottingham.ac.uk/plp/pmzcw/download/fnt_chap5.pdf
% https://www.math.nagoya-u.ac.jp/~larsh/teaching/F2013_PM/lecture4.pdf



%%%%%%%%%%%%%%%%%%%%%%%%%%%%%%%%%%%%%%%%%%%%%%%%%%%%%%%%%%%%%%%%%%%%%%%%%%%%%%%%



\emph{Exercise 1.1-1.9: Define $N: \mathbb{Z}[i] \rightarrow \mathbb{Z}$ by
$N(a+bi) = a^2 + b^2$.} \\\\



\textbf{Exercise 1.1.}
\emph{Verify that for all $\alpha, \beta \in \mathbb{Z}[i]$,
$N(\alpha\beta) = N(\alpha)N(\beta)$,
either by direct computation or using the fact that
$N(a+bi) = (a+bi)(a-bi)$.
Conclude that if $\alpha \mid \gamma$ in $\mathbb{Z}[i]$,
then $N(\alpha) \mid N(\gamma)$ in $\mathbb{Z}$.} \\

\emph{Proof.}
\begin{enumerate}
\item[(1)]
\emph{Direct computation.}
Write $\alpha = a+bi, \beta=c+di$ where $a, b, c, d \in \mathbb{Z}$.
Thus,
\begin{align*}
N(\alpha\beta)
&= N((a+bi)(c+di)) \\
&= N((ac-bd) + (ad+bc)i) \\
&= (ac-bd)^2 + (ad+bc)^2 \\
&= (a^2 c^2 - 2abcd + b^2 d^2) + (a^2 d^2 + 2abcd + b^2 c^2) \\
&= a^2 c^2 + b^2 d^2 + a^2 d^2 + b^2 c^2, \\
N(\alpha)N(\beta)
&= N(a+bi) N(c+di) \\
&= (a^2 + b^2)(c^2 + d^2) \\
&= a^2 c^2 + b^2 d^2 + a^2 d^2 + b^2 c^2.
\end{align*}
Therefore, $N(\alpha\beta) = N(\alpha)N(\beta)$.
(Note that we also get the identity
$(a^2 + b^2)(c^2 + d^2) = (ac-bd)^2 + (ad+bc)^2$.)
\item[(2)]
\emph{Using the fact that $N(a+bi) = (a+bi)(a-bi)$,}
or $N(\alpha) = \alpha \overline{\alpha}$
for any $\alpha \in \mathbb{Z}[i]$.
Thus,
\begin{align*}
N(\alpha\beta)
&= \alpha\beta\overline{\alpha\beta} \\
&= \alpha\beta\overline{\alpha}\overline{\beta} \\
&= \alpha\overline{\alpha}\beta\overline{\beta} \\
&= N(\alpha)N(\beta).
\end{align*}
\item[(3)]
\emph{Show that if $\alpha \mid \gamma$ in $\mathbb{Z}[i]$,
then $N(\alpha) \mid N(\gamma)$ in $\mathbb{Z}$.}
Write $\gamma = \alpha \beta$ for some $\beta \in \mathbb{Z}[i]$.
So $N(\gamma) = N(\alpha) N(\beta) \in \mathbb{Z}$,
or $N(\alpha) \mid N(\gamma)$ in $\mathbb{Z}$.
\end{enumerate}
$\Box$ \\\\



%%%%%%%%%%%%%%%%%%%%%%%%%%%%%%%%%%%%%%%%%%%%%%%%%%%%%%%%%%%%%%%%%%%%%%%%%%%%%%%%



\textbf{Exercise 1.2.}
\emph{Let $\alpha \in \mathbb{Z}[i]$.
Show that $\alpha$ is a unit iff $N(\alpha) = 1$.
Conclude that the only unit are $\pm 1$ and $\pm i$.} \\

\emph{Proof.}
\begin{enumerate}
\item[(1)]
\emph{$(\Longrightarrow)$}
Since $\alpha$ is a unit, there is $\beta \in \mathbb{Z}[i]$ such that
$\alpha \beta = 1$.
By Exercise 1.1, $N(\alpha \beta) = N(1)$, or $N(\alpha) N(\beta) = 1$.
Since the image of $N$ is nonnegative integers, $N(\alpha) = 1$.
\item[(2)]
\emph{$(\Longleftarrow)$}
By Exercise 1.1, $N(\alpha) = \alpha \overline{\alpha}$,
or $1 = \alpha \overline{\alpha}$ since $N(\alpha) = 1$.
That is, $\overline{\alpha} \in \mathbb{Z}[i]$ is
the inverse of $\alpha \in \mathbb{Z}[i]$.
(Or by (1), we solve the equation $N(\alpha) = a^2 + b^2 = 1$,
and show that all four solutions ($\pm 1$ and $\pm i$) are unit.)
\end{enumerate}
Conclusion: a unit $\alpha = a+bi$ of $\mathbb{Z}[i]$
is satisfying the equation $N(\alpha) = a^2 + b^2 = 1$ by (1)(2).
That is, the only unit of $\mathbb{Z}[i]$ are $\pm 1$ and $\pm i$.
$\Box$ \\\\



%%%%%%%%%%%%%%%%%%%%%%%%%%%%%%%%%%%%%%%%%%%%%%%%%%%%%%%%%%%%%%%%%%%%%%%%%%%%%%%%



\textbf{Exercise 1.3.}
\emph{Let $\alpha \in \mathbb{Z}[i]$.
Show that if $N(\alpha)$ is a prime in $\mathbb{Z}$ then
$\alpha$ is irreducible in $\mathbb{Z}[i]$.
Show that the same conclusion holds
if $N(\alpha) = p^2$, where $p$ is a prime in $\mathbb{Z}$,
$p \equiv 3 \pmod{4}$.} \\

\emph{Proof.}
\begin{enumerate}
\item[(1)]
\emph{Show that if $N(\alpha)$ is a prime in $\mathbb{Z}$ then
$\alpha$ is irreducible in $\mathbb{Z}[i]$.}
Write $\alpha = \beta\gamma$.
Then $N(\alpha) = N(\beta)N(\gamma)$ is a prime in $\mathbb{Z}$.
Since each integer prime is irreducible, $N(\beta) = 1$ or $N(\gamma) = 1$.
So that $\beta$ is unit or $\gamma$ is unit by Exercise 1.2.
Hence, $\alpha$ is irreducible.
\item[(2)]
\emph{Show that $\alpha$ is irreducible in $\mathbb{Z}[i]$
if $N(\alpha) = p^2$, where $p$ is a prime in $\mathbb{Z}$,
$p \equiv 3 \pmod{4}$.}
Assume $\alpha = \beta\gamma$ were not irreducible.
Similar to (1), $N(\alpha) = N(\beta)N(\gamma) = p^2$.
Since $\beta$ and $\gamma$ are proper factors of $\alpha$,
$$N(\beta) = N(\gamma) = p.$$
Since any square $a^2 \equiv 0, 1 \pmod{4}$,
any $N(a+bi) = a^2+b^2 \equiv 0, 1, 2 \pmod{4}$.
Especially, $N(\beta) \equiv 0, 1, 2 \pmod{4}$,
contrary to $N(\beta) = p \equiv 3 \pmod{4}$
by the assumption.
Therefore, $\alpha$ is irreducible in $\mathbb{Z}[i]$.
\end{enumerate}
$\Box$ \\

\textbf{Supplement.}
\begin{enumerate}
\item[(1)]
The prime $2$ is reducible in $\mathbb{Z}[i]$ (Exercise 1.4).
\item[(2)]
Every prime $p \equiv 1 \pmod{4}$ is reducible in $\mathbb{Z}[i]$ (Exercise 1.8). \\\\
\end{enumerate}



%%%%%%%%%%%%%%%%%%%%%%%%%%%%%%%%%%%%%%%%%%%%%%%%%%%%%%%%%%%%%%%%%%%%%%%%%%%%%%%%



\textbf{Exercise 1.4.}
\emph{Show that $1-i$ is irreducible in $\mathbb{Z}$
and that $2 = u(1-i)^2$ for some unit $u$.} \\

\emph{Proof.}
\begin{enumerate}
\item[(1)]
\emph{$1-i$ is irreducible.}
Since $N(1-i) = 2$ is a prime in $\mathbb{Z}$,
$1-i$ is irreducible by Problem 1.3.
\item[(2)]
$2 = i(1-i)^2$ where $i$ is unit in $\mathbb{Z}$.
\end{enumerate}
$\Box$ \\\\



%%%%%%%%%%%%%%%%%%%%%%%%%%%%%%%%%%%%%%%%%%%%%%%%%%%%%%%%%%%%%%%%%%%%%%%%%%%%%%%%



\textbf{Exercise 1.5.}
\emph{Notice that
$(2+i)(2-i) = 5 = (1+2i)(1-2i)$.
How is this consistent with unique factorization?} \\

\emph{Proof.}
Since $2+i = i(1-2i)$ and $2-i = (-i)(1+2i)$,
the factorization is unique up to order and multiplication of primes by units.
$\Box$ \\\\



%%%%%%%%%%%%%%%%%%%%%%%%%%%%%%%%%%%%%%%%%%%%%%%%%%%%%%%%%%%%%%%%%%%%%%%%%%%%%%%%



\textbf{Exercise 1.6.}
\emph{Show that every nonzero, non-unit Gaussian integer $\alpha$
is a product of irreducible elements, by induction on $N(\alpha)$.} \\

\emph{Proof.}
Induction on $N(\alpha)$.
\begin{enumerate}
\item[(1)]
\emph{$n = 2$}.
Given $\alpha \in \mathbb{Z}[i]$ with $N(\alpha) = 2$.
Since $N(\alpha) = 2$ is a prime in $\mathbb{Z}$,
$\alpha$ is irreducible (Exercise 1.3).
\item[(2)]
\emph{Suppose the result holds for $n \leq k$.}
Given $\alpha \in \mathbb{Z}[i]$ with $N(\alpha) = k+1$.
There are only two possible cases.
\begin{enumerate}
\item[(a)]
\emph{$\alpha$ is irreducible.}
Nothing to do.
\item[(b)]
\emph{$\alpha$ is reducible.}
Write $\alpha = \beta\gamma$ where neither factor is unit.
Since $N(\alpha) = N(\beta)N(\gamma)$ and neither factor is unit,
$$2 \leq N(\beta), N(\gamma) \leq k.$$
By the induction hypothesis, each factor of $\alpha$ ($\beta$ and $\gamma$)
is a product of irreducible elements.
So that $\alpha$ again is a product of irreducible elements.
\end{enumerate}
In any cases,
$\alpha$ is a product of irreducible elements.
\end{enumerate}
By induction, the result is established.
$\Box$ \\\\



%%%%%%%%%%%%%%%%%%%%%%%%%%%%%%%%%%%%%%%%%%%%%%%%%%%%%%%%%%%%%%%%%%%%%%%%%%%%%%%%



\textbf{Exercise 1.7.}
\emph{Show that $\mathbb{Z}[i]$ is a principal ideal domain (PID); i.e.,
every ideal $I$ is principal.
(As shown in Appendix 1, this implies that $\mathbb{Z}[i]$ is a UFD.)} \\

\emph{Suggestion: Take $\alpha \in I-\{0\}$ such that $N(\alpha)$ is minimized,
and consider the multiplies $\gamma\alpha$, $\gamma \in \mathbb{Z}[i]$;
show that these are the vertices of an infinite family of squares which
fill up the complex plane.
(For example,
one of the squares has vertices $0$, $\alpha$, $i\alpha$, and $(1+i)\alpha$;
all others are translates of this one.)
Obviously $I$ contains all $\gamma\alpha$;
show by a geometric argument that if $I$ contains anything else then
minimality of $N(\alpha)$ would be contradicted.} \\

\emph{Proof (without geometric intuition).}
Define $N$ on $\mathbb{Q}[i]$ by $N(a+bi) = a^2+b^2$ where
$a+bi \in \mathbb{Q}[i]$ as usual.
\begin{enumerate}
\item[(1)]
\emph{Show that $\mathbb{Z}[i]$ is a Euclidean domain.}
Given $\alpha = a+bi \in \mathbb{Z}[i]$ and
$\gamma = c+di \in \mathbb{Z}[i]$ with $\gamma \neq 0$.
It suffices to show there exist $\delta$ and $\rho$ such that the identity
$\alpha = \gamma\delta + \rho$ holds and
either $\rho = 0$ or $N(\rho) < N(\gamma)$.
\begin{enumerate}
\item[(a)]
\emph{Pick $\delta \in \mathbb{Z}[i].$
(Intuition: Pick the `integer part' of $\frac{\alpha}{\gamma}$
as we did in integer numbers.)}
Write $\frac{\alpha}{\gamma} = r+si \in \mathbb{Q}[i]$.
Then we pick $\delta = m+ni \in \mathbb{Z}[i]$ such that
$|r-m| \leq \frac{1}{2}$ and $|s-n| \leq \frac{1}{2}$.
Therefore,
\begin{align*}
N\left( \frac{\alpha}{\gamma} - \delta \right)
&= (r-m)^2 + (s-n)^2 \\
&\leq \frac{1}{4} + \frac{1}{4} \\
&= \frac{1}{2}.
\end{align*}
\item[(b)]
\emph{Pick $\rho \in \mathbb{Z}[i]$.}
Clearly we can pick $\rho = \alpha - \gamma\delta \in \mathbb{Z}[i]$.
Therefore, $\rho = 0$ or
\begin{align*}
N(\rho)
&= N(\alpha - \gamma\delta) \\
&= N\left( \gamma\left( \frac{\alpha}{\gamma} - \delta \right) \right) \\
&= N(\gamma) N\left( \frac{\alpha}{\gamma} - \delta \right) \\
&\leq \frac{1}{2} N(\gamma) \\
&< N(\gamma).
\end{align*}
\end{enumerate}
\item[(2)]
\emph{Show that every Euclidean domain $R$ is a PID.}
Given any ideal $I$ of $R$.
Take $\alpha \in I-\{0\}$ such that $N(\alpha)$ is minimized.
\begin{enumerate}
  \item[(a)]
  $R\alpha \subseteq I$ clearly.
  \item[(b)]
  Conversely, for any $\beta \in I$, there are $\delta, \rho \in R$
  such that $\beta = \alpha \delta + \rho$, where either $\rho = 0$
  or $N(\rho) < N(\alpha)$.
  Since $\rho = \beta - \alpha \delta \in I$,
  we cannot have $N(\rho) < N(\alpha)$ by the minimality of $N(\alpha)$.
  Therefore, $\rho = 0$ and $\beta = \alpha \delta \in R\alpha$,
  or $R\alpha \supseteq I$.
\end{enumerate}
\end{enumerate}
By (1)(2),
$\mathbb{Z}[i]$ is a PID.
$\Box$ \\\\



%%%%%%%%%%%%%%%%%%%%%%%%%%%%%%%%%%%%%%%%%%%%%%%%%%%%%%%%%%%%%%%%%%%%%%%%%%%%%%%%



\textbf{Exercise 1.8.}
\emph{We will use the unique factorization in $\mathbb{Z}[i]$ to prove that
every prime $p \equiv 1 \pmod{4}$ is a sum of two squares.
\begin{enumerate}
  \item[(a)]
  Use the fact that the multiplicative group $(\mathbb{Z}/p\mathbb{Z})^{\times}$
  of integers mod $p$ is cyclic to show that
  if $p \equiv 1 \pmod{4}$ then $n^2 \equiv -1 \pmod{p}$ for some $n \in \mathbb{Z}$.
  \item[(b)]
  Prove that $p$ cannot be irreducible in $\mathbb{Z}[i]$.
  (Hint: $p \mid n^2+1 = (n+i)(n-i)$.)
  \item[(c)]
  Prove that $p$ is a sum of two squares.
  (Hint: (b) shows that $p = (a+bi)(c+di)$ with neither factor a unit.
  Take norms.) \\
\end{enumerate}}

\emph{Proof of (a).}
Since the multiplicative group $(\mathbb{Z}/p\mathbb{Z})^{\times}$
of integers mod $p$ is cyclic,
$(\mathbb{Z}/p\mathbb{Z})^{\times}$ is generated by (a primitive root)
$g \in \mathbb{Z}/p\mathbb{Z}$.
$g^{p-1} = 1$,
or
$$(g^{\frac{p-1}{2}} - 1)(g^{\frac{p-1}{2}} + 1) = 0$$
since $p$ is odd.
Since $\mathbb{Z}/p\mathbb{Z}$ is an integral domain,
$g^{\frac{p-1}{2}} - 1 = 0$ or $g^{\frac{p-1}{2}} + 1 = 0$.
$g$ cannot satisfy $g^{\frac{p-1}{2}} - 1 = 0$ since
$g$ is a generator of $(\mathbb{Z}/p\mathbb{Z})^{\times}$.
So,
$$g^{\frac{p-1}{2}} + 1 = 0.$$
Let $n = g^{\frac{p-1}{4}} \in \mathbb{Z}$ since $p \equiv 1 \pmod{4}$.
So $n^2+1 = 0 \pmod{p}$.
$\Box$ \\

\emph{Proof of (b).}
Since $n^2 + 1 \equiv 0 \pmod{p}$ by (a),
$p \mid n^2+1 = (n+i)(n-i)$.
If $p$ were irreducible in $\mathbb{Z}[i]$,
$p \mid (n+i)$ or $p \mid (n-i)$ by using the unique factorization in $\mathbb{Z}[i]$.
Hence
$$\frac{n+i}{p} = \frac{n}{p} + \frac{1}{p}i \not\in \mathbb{Z}[i],
\frac{n-i}{p} = \frac{n}{p} - \frac{1}{p}i \not\in \mathbb{Z}[i],$$
contrary to the assumption.
Therefore, $p$ is reducible in $\mathbb{Z}[i]$.
$\Box$ \\

\emph{Proof of (c).}
Since $p$ is reducible in $\mathbb{Z}[i]$ by (b),
write $p = (a+bi)(c+di)$ with neither factor a unit.
Take norms,
$$p^2 = N(p) = N(a+bi)N(c+di).$$
Since neither factor of $p$ is unit,
$N(a+bi) = p$,
or $a^2+b^2 = p$,
or $p$ is a sum of two squares.
$\Box$ \\\\



%%%%%%%%%%%%%%%%%%%%%%%%%%%%%%%%%%%%%%%%%%%%%%%%%%%%%%%%%%%%%%%%%%%%%%%%%%%%%%%%



\textbf{Exercise 1.9.}
\emph{Describe all irreducible elements in $\mathbb{Z}[i]$.} \\

\emph{Notice that $\alpha$ is irreducible if and only if $\overline{\alpha}$ is irreducible.}
(Write $\alpha = \beta\gamma$,
then $\overline{\alpha} = \overline{\beta}\overline{\gamma}$.
Besides, $\overline{\overline{\alpha}} = \alpha$.) \\

\emph{Proof.}
\emph{Show that all irreducible elements in $\mathbb{Z}[i]$ (up to units) are
\begin{enumerate}
  \item[(1)]
  $1+i$.
  \item[(2)]
  $\pi = a+bi$ for each integer prime $p \equiv 1 \pmod{4}$
  with $p = a^2+b^2$.
  \item[(3)]
  $p$ for each integer prime $p \equiv 3 \pmod{4}$.
\end{enumerate}}

Let $\alpha$ be any irreducible element in $\mathbb{Z}[i]$.
Consider $N(\alpha) = \alpha \overline{\alpha}$.
$N(\alpha) \neq 1$ since $\alpha$ is not unit.
By the unique factorization theorem in $\mathbb{Z}$,
$N(\alpha) \in \mathbb{Z}$ is a product of primes in $\mathbb{Z}$. \\

There are three possible cases.
\begin{enumerate}
  \item[(a)]
  \emph{$2 \mid N(\alpha)$.}
  Write $(1+i)(1-i) \mid \alpha \overline{\alpha}$ in $\mathbb{Z}[i]$.
  Notice that $1+i$, $1-i$, $\alpha$ and $\overline{\alpha}$ are all irreducible (Exercise 1.4).
  By the unique factorization theorem in $\mathbb{Z}[i]$,
  $\alpha = 1+i$ (up to units).
  \item[(b)]
  \emph{$p \mid N(\alpha)$ for some prime $p \equiv 3 \pmod{4}$.}
  Write $p \mid \alpha \overline{\alpha}$ in $\mathbb{Z}[i]$.
  Notice that $p$, $\alpha$ and $\overline{\alpha}$ are all irreducible (Exercise 1.3).
  By the unique factorization theorem in $\mathbb{Z}[i]$,
  $\alpha = p$ (up to units) or $\overline{\alpha} = p$ (up to units).
  So in any cases $\alpha = p$ (up to units). (Note that $\overline{p} = p$.)
  \item[(c)]
  \emph{$p \mid N(\alpha)$ for some prime $p \equiv 1 \pmod{4}$.}
  For such $p$, there is an irreducible $\pi \in \mathbb{Z}[i]$
  satisfying $p = \pi \overline{\pi}$ (Exercise 1.8).
  Now we write $\pi \overline{\pi} \mid \alpha \overline{\alpha}$ in $\mathbb{Z}[i]$.
  Notice that $\pi$, $\overline{\pi}$, $\alpha$ and $\overline{\alpha}$ are all irreducible.
  By the unique factorization theorem in $\mathbb{Z}[i]$,
  $\alpha = \pi$ or $\alpha = \overline{\pi}$.
  In any cases, $\alpha=a+bi$ for integer prime $p \equiv 1 \pmod{4}$
  with $p = a^2+b^2$.
\end{enumerate}
$\Box$ \\\\



%%%%%%%%%%%%%%%%%%%%%%%%%%%%%%%%%%%%%%%%%%%%%%%%%%%%%%%%%%%%%%%%%%%%%%%%%%%%%%%%



\emph{Exercise 1.10 - 1.14: Let
$\omega
= e^{\frac{2\pi i}{3}}
= -\frac{1}{2}+\frac{\sqrt{3}}{2}i$.
Define $N: \mathbb{Z}[\omega] \to \mathbb{Z}$ by
$N(a+b\omega) = a^2-ab+b^2$. } \\\\



\textbf{Exercise 1.10.}
\emph{Show that if $a+b\omega$ is written in the form $u+vi$
where $u$ and $v$ are real, then $N(a+b\omega)=u^2+v^2$.}

\emph{Proof.}
By $\omega = -\frac{1}{2}+\frac{\sqrt{3}}{2}i$,
write
$$a+b\omega
= \left( a - \frac{1}{2}b \right) + \left( \frac{\sqrt{3}}{2}b \right) i.$$
Here $u = a - \frac{1}{2}b \in \mathbb{R}$ and
$v = \frac{\sqrt{3}}{2}b \in \mathbb{R}$.
Hence $u^2+v^2 = (a - \frac{1}{2}b)^2 + (\frac{\sqrt{3}}{2}b)^2 = a^2 - ab + b^2
= N(a+b\omega)$.
$\Box$ \\\\



%%%%%%%%%%%%%%%%%%%%%%%%%%%%%%%%%%%%%%%%%%%%%%%%%%%%%%%%%%%%%%%%%%%%%%%%%%%%%%%%



\textbf{Exercise 1.11.}
\emph{Show that for all $\alpha, \beta \in \mathbb{Z}[\omega]$,
$N(\alpha\beta) = N(\alpha)N(\beta)$,
either by direct computation or by using Exercise 1.10.
Conclude that if $\alpha \mid \gamma$ in $\mathbb{Z}[\omega]$,
then $N(\alpha) \mid N(\gamma)$ in $\mathbb{Z}$.} \\

\emph{Proof.}
\begin{enumerate}
\item[(1)]
\emph{Direct computation.}
Note that $1+\omega+\omega^2 = 0$ or $\omega^2 = -1-\omega$.
Write $\alpha = a+b\omega, \beta=c+d\omega$ where $a, b, c, d \in \mathbb{Z}$.
Thus,
\begin{align*}
N(\alpha\beta)
&= N((a+b\omega)(c+d\omega)) \\
&= N(ac + (ad+bc)\omega + bd\omega^2) \\
&= N(ac + (ad+bc)\omega + bd(-1-\omega)) \\
&= N((ac-bd) + (ad+bc-bd)\omega) \\
&= (ac-bd)^2 - (ac-bd)(ad+bc-bd) + (ad+bc-bd)^2 \\
&= (a^2 - ab + b^2)(c^2 - cd + d^2), \\
N(\alpha)N(\beta)
&= N(a+b\omega) N(c+d\omega) \\
&= (a^2 - ab + b^2)(c^2 - cd + d^2).
\end{align*}
\item[(2)]
\emph{Exercise 1.10.}
The result is established by Exercise 1.10 and Exercise 1.1.
\item[(3)]
\emph{Using the fact that $N(a+b\omega) = (a+b\omega)\overline{(a+b\omega)}$.}
Similar to the argument of Exercise 1.1.
\item[(4)]
\emph{Show that if $\alpha \mid \gamma$ in $\mathbb{Z}[\omega]$,
then $N(\alpha) \mid N(\gamma)$ in $\mathbb{Z}$.}
Similar to the argument of Exercise 1.1.
\end{enumerate}
$\Box$ \\\\



%%%%%%%%%%%%%%%%%%%%%%%%%%%%%%%%%%%%%%%%%%%%%%%%%%%%%%%%%%%%%%%%%%%%%%%%%%%%%%%%



\textbf{Exercise 1.12.}
\emph{Let $\alpha \in \mathbb{Z}[\omega]$.
Show that $\alpha$ is a unit iff $N(\alpha) = 1$,
and find all units in $\mathbb{Z}[\omega]$.
(There are six of them.)} \\

\emph{Proof.}
\begin{enumerate}
\item[(1)]
\emph{$(\Longrightarrow)$}
Since $\alpha$ is a unit, there is $\beta \in \mathbb{Z}[\omega]$ such that
$\alpha \beta = 1$.
By Exercise 1.11, $N(\alpha \beta) = N(1)$, or $N(\alpha) N(\beta) = 1$.
Since the image of $N$ is nonnegative integers, $N(\alpha) = 1$.
\item[(2)]
\emph{$(\Longleftarrow)$}
By Exercise 1.10, $N(\alpha) = \alpha \overline{\alpha}$,
or $1 = \alpha \overline{\alpha}$ since $N(\alpha) = 1$.
That is, $\overline{\alpha} \in \mathbb{Z}[\omega]$ is
the inverse of $\alpha \in \mathbb{Z}[\omega]$.
\item[(3)]
By (1), we solve the equation $N(\alpha) = a^2 - ab + b^2 = 1$,
or $4 = (2a-b)^2 + 3b^2$.
There are 2 possible cases.
  \begin{enumerate}
  \item[(a)]
  $2a-b = \pm 1$, $b = \pm 1$.
  \item[(b)]
  $2a-b = \pm 2$, $b = \pm 0$.
  \end{enumerate}
  Solve these 6 pairs of equations yields the result
  $\pm 1, \pm \omega, \pm \omega^2$.
\end{enumerate}
$\Box$ \\\\



%%%%%%%%%%%%%%%%%%%%%%%%%%%%%%%%%%%%%%%%%%%%%%%%%%%%%%%%%%%%%%%%%%%%%%%%%%%%%%%%



\textbf{Exercise 1.13.}
\emph{Show that $1-\omega$ is irreducible in $\mathbb{Z}[\omega]$,
and that $3 = u(1-\omega)^2$ for some unit $u$. } \\

$3$ is not irreducible in $\mathbb{Z}[\omega]$. \\

\emph{Proof.}
\begin{enumerate}
\item[(1)]
$N(1-\omega) = 3$ is an integer prime.
Similar to the argument in Exercise 1.3,
$1-\omega$ is irreducible in $\mathbb{Z}[\omega]$.
\item[(2)]
Note that $1+\omega+\omega^2=0$.
So
$(1-\omega)^2 = 1-2\omega+\omega^2 = 3(-\omega)$, or
$(-\omega^2)(1-\omega)^2 = 3$.
By Exercise 1.12, $-\omega^2$ is unit.
Hence $3 = u(1-\omega)^2$ for some unit $u = -\omega^2$.
\end{enumerate}
$\Box$ \\\\



%%%%%%%%%%%%%%%%%%%%%%%%%%%%%%%%%%%%%%%%%%%%%%%%%%%%%%%%%%%%%%%%%%%%%%%%%%%%%%%%



\textbf{Exercise 1.14.}
\emph{Modify Exercise 1.7 to show that $\mathbb{Z}[\omega]$ is a PID, hence a UFD.
Here the squares are replaced by parallelograms;
one of them has vertices $0, \alpha, \omega\alpha, (\omega+1)\alpha$,
and all others are translates of this one.
Use Exercise 1.10 for the geometric argument at the end. } \\

Similar to Exercise 1.7. \\

\emph{Proof (without geometric intuition).}
Define $N$ on $\mathbb{Q}[\omega]$ by $N(a+b\omega) = a^2-ab+b^2$ where
$a+b\omega \in \mathbb{Q}[\omega]$ as usual.
\begin{enumerate}
\item[(1)]
\emph{Show that $\mathbb{Z}[\omega]$ is a Euclidean domain.}
Given $\alpha = a+b\omega \in \mathbb{Z}[\omega]$ and
$\gamma = c+d\omega \in \mathbb{Z}[\omega]$ with $\gamma \neq 0$.
It suffices to show there exist $\delta$ and $\rho$ such that the identity
$\alpha = \gamma\delta + \rho$ holds and
either $\rho = 0$ or $N(\rho) < N(\gamma)$.
\begin{enumerate}
\item[(a)]
\emph{Pick $\delta \in \mathbb{Z}[\omega].$
(Intuition: Pick the `integer part' of $\frac{\alpha}{\gamma}$
as we did in integer numbers.)}
Write $\frac{\alpha}{\gamma} = r+s\omega \in \mathbb{Q}[\omega]$.
Then we pick $\delta = m+n\omega \in \mathbb{Z}[\omega]$ such that
$|r-m| \leq \frac{1}{2}$ and $|s-n| \leq \frac{1}{2}$.
Therefore,
\begin{align*}
N\left( \frac{\alpha}{\gamma} - \delta \right)
&\leq |r-m|^2 + |r-m||s-n| + |s-n|^2 \\
&\leq \frac{1}{4} + \frac{1}{4} + \frac{1}{4} \\
&= \frac{3}{4}.
\end{align*}
\item[(b)]
\emph{Pick $\rho \in \mathbb{Z}[\omega]$.}
Clearly we can pick $\rho = \alpha - \gamma\delta \in \mathbb{Z}[\omega]$.
Therefore, $\rho = 0$ or
\begin{align*}
N(\rho)
&= N(\alpha - \gamma\delta) \\
&= N\left( \gamma\left( \frac{\alpha}{\gamma} - \delta \right) \right) \\
&= N(\gamma) N\left( \frac{\alpha}{\gamma} - \delta \right) \\
&\leq \frac{3}{4} N(\gamma) \\
&< N(\gamma).
\end{align*}
\end{enumerate}
\item[(2)]
\emph{Show that every Euclidean domain $R$ is a PID.}
Given any ideal $I$ of $R$.
Take $\alpha \in I-\{0\}$ such that $N(\alpha)$ is minimized.
\begin{enumerate}
  \item[(a)]
  $R\alpha \subseteq I$ clearly.
  \item[(b)]
  Conversely, for any $\beta \in I$, there are $\delta, \rho \in R$
  such that $\beta = \alpha \delta + \rho$, where either $\rho = 0$
  or $N(\rho) < N(\alpha)$.
  Since $\rho = \beta - \alpha \delta \in I$,
  we cannot have $N(\rho) < N(\alpha)$ by the minimality of $N(\alpha)$.
  Therefore, $\rho = 0$ and $\beta = \alpha \delta \in R\alpha$,
  or $R\alpha \supseteq I$.
\end{enumerate}
\end{enumerate}
By (1)(2),
$\mathbb{Z}[i]$ is a PID.
$\Box$ \\\\



%%%%%%%%%%%%%%%%%%%%%%%%%%%%%%%%%%%%%%%%%%%%%%%%%%%%%%%%%%%%%%%%%%%%%%%%%%%%%%%%



\textbf{Exercise 1.15.}
\emph{Here is a proof of Fermat's conjecture for $n = 4$:
If $x^4 + y^4 = z^4$ has a solution in positive integers,
then so does $x^4 + y^4 = w^2$.
Let $x,y,w$ be a solution with smallest possible $w$.
Then $x^2, y^2, w$ is a primitive Pythagorean triple.
Assuming (without loss of generality) that $x$ is odd,
we can write
$$x^2 = m^2-n^2, y^2 = 2mn, w = m^2+n^2$$
with $m$ and $n$ are relatively prime positive integers, not both odd. }
\begin{enumerate}
\item[(a)]
\emph{Show that
$$x=r^2-s^2, n=2rs, m=r^2+s^2$$
with $r$ and $s$ are relatively prime positive integers, not both odd. }
\item[(b)]
\emph{Show that $r,s$ and $m$ are pairwise relatively prime.
Using $y^2 = 4rsm$, conclude that $r$, $s$ and $m$ are all squares, say
$a^2$, $b^2$ and $c^2$. }
\item[(c)]
\emph{Show that $a^4+b^4=c^2$, and that this contradicts minimality of $w$. } \\
\end{enumerate}

\emph{Proof of (a).}
Write $x^2+n^2=m^2$ by moving $n^2$ of $x^2 = m^2-n^2$ to the left side.
Notice that $x$ is odd, and thus
$x=r^2-s^2, n=2rs, m=r^2+s^2$
with $r$ and $s$ are relatively prime positive integers, not both odd.
$\Box$ \\

\emph{Proof of (b).}
\begin{enumerate}
\item[(1)]
It suffices to show that $(r,m) = 1$.
By assumption, $(r,s)=1$.
So
$(r,s) = 1 \Rightarrow (r,s^2) = 1 \Rightarrow (r,r^2+s^2) = 1$
and note that $m=r^2+s^2$ to get the result.
\item[(2)]
$y^2 = 2mn = 2m(2rs) = 4rsm$ by (a).
Since $r,s$ and $m$ are pairwise relatively prime,
$r,s$ and $m$ are all squares.
\end{enumerate}
$\Box$ \\

\emph{Proof of (c).}
By (b), $r=a^2$, $s=b^2$, $m=c^2$.
By (a), $m=r^2+s^2$, or $c^2 = (a^2)^2 + (b^2)^2 = a^4 + b^4$.
However, $w = m^2+n^2 > m^2 > m = c^2 > c$, contrary to the minimality of $w$.
$\Box$ \\\\



%%%%%%%%%%%%%%%%%%%%%%%%%%%%%%%%%%%%%%%%%%%%%%%%%%%%%%%%%%%%%%%%%%%%%%%%%%%%%%%%



\emph{Exercise 1.16-1.28: Let $p$ be an odd prime,
$\omega = e^{\frac{2\pi i}{p}}$.} \\\\



\textbf{Exercise 1.16.}
\emph{Show that
$$(1-\omega)(1-\omega^2) \cdots (1-\omega^{p-1}) = p$$
by considering equation
$t^p - 1 = (t-1)(t-\omega)(t-\omega^2) \cdots (t-\omega^{p-1})$.} \\

\emph{Proof.}
Note that
$t^p - 1 = (t-1)(t^{p-1} + t^{p-2} + \cdots + t + 1)$.
Cancel out $t-1$ of Equation (2),
$$t^{p-1} + t^{p-2} + \cdots + t + 1 = (t-\omega)(t-\omega^2) \cdots (t-\omega^{p-1}).$$
Put $t = 1$ to get
$p = (1-\omega)(1-\omega^2) \cdots (1-\omega^{p-1})$.
$\Box$ \\\\



%%%%%%%%%%%%%%%%%%%%%%%%%%%%%%%%%%%%%%%%%%%%%%%%%%%%%%%%%%%%%%%%%%%%%%%%%%%%%%%%



\textbf{Exercise 1.17.}
\emph{Let $x^p + y^p = z^p$.
Suppose that $\mathbb{Z}[\omega]$ is a UFD and $\pi \mid x + y\omega$,
and $\pi$ is a prime in $\mathbb{Z}[\omega]$.
Show that $\pi$ does not divide any of the other factors on the left side of
$$(x+y)(x+y\omega)(x+y\omega^2) \cdots (x+y\omega^{p-1}) = z^p$$
by showing that if it did, then $\pi$ would divide both $z$ and $yp$
(Hint: Use Exercise 1.16);
but $z$ and $yp$ are relatively prime (assuming $p$ divides none of $x, y, z$),
hence $zm + ypn = 1$ for some $m, n \in \mathbb{Z}$.
How is this a contradiction? } \\

\emph{Proof.}
Write $$z = u {\pi}_1^{e_1} \cdots {\pi}_m^{e_m}$$ where
$u$ is unit and $\pi_k$ $(1 \leq k \leq m)$ are distinct primes in $\mathbb{Z}[\omega]$ and
$e_k \in \mathbb{Z}^+$ $(1 \leq k \leq m)$.
Since $\mathbb{Z}[\omega]$ is a UFD by assumption,
the factorization of $z$ is unique up to order and units.
\begin{enumerate}
\item[(1)]
\emph{Show that $\pi \mid z$.}
Since $\pi \mid x + y\omega$, $\pi \mid z^p$.
The factorization of $z^p$ is
$$z^p = u^p {\pi}_1^{pe_1} \cdots {\pi}_m^{pe_m}.$$
$u^p$ is unit, and $\pi | z^p$ implies that $\pi = \pi_k$ for some $k$,
that is, $\pi \mid z$.
\item[(2)]
\emph{Show that $\pi \mid yp$ if $\pi$ were divide any of
the other factors on the left side of
$(x+y)(x+y\omega)(x+y\omega^2) \cdots (x+y\omega^{p-1}) = z^p$.}
Say $\pi \mid x+y\omega^k$ for some $k \neq 1$.
So that $\pi \mid ((x+y\omega) - (x+y\omega^k))$,
or $\pi \mid y(\omega - \omega^k)$.
Since $k \neq 1$, there are two possible cases.
  \begin{enumerate}
  \item[(a)]
  $k > 1$. $\pi \mid y\omega(1 - \omega^{k-1})$.
  By Exercise 1.16, $\pi \mid y\omega p$, or $\pi \mid yp$ since $\omega$ is unit.
  ($\omega^{p-1}$ is the inverse of $\omega$ since $\omega \cdot \omega^{p-1} = 1$.)
  \item[(b)]
  $k = 0$. $\pi \mid y(\omega - 1)$, or $\pi \mid y(1 - \omega)$.
  By Exercise 1.16, $\pi \mid yp$.
  \end{enumerate}
In any case, $\pi \mid yp$.
\item[(3)]
Note that $z$ and $yp$ are integers, and they are relatively prime by the assumption
that $p$ divides none of $x, y, z$.
Therefore, on $\mathbb{Z}$ we have $zm + ypn = 1$ for some $m, n \in \mathbb{Z}$.
\item[(4)]
$zm + ypn = 1$ is also true in $\mathbb{Z}[\omega]$.
Therefore, by (1)(2) we have $\pi \mid (zm + ypn)$ or $\pi \mid 1$,
or $\pi$ is unit, contrary to the primality of $\pi$.
\end{enumerate}
$\Box$ \\\\



%%%%%%%%%%%%%%%%%%%%%%%%%%%%%%%%%%%%%%%%%%%%%%%%%%%%%%%%%%%%%%%%%%%%%%%%%%%%%%%%



\textbf{Exercise 1.18.}
\emph{Use Exercise 1.17 to show that if $\mathbb{Z}[\omega]$ is a UFD then
$x+y\omega = u\alpha^p$, $\alpha \in \mathbb{Z}[\omega]$,
$u$ a unit in $\mathbb{Z}[\omega]$. } \\

\emph{Proof.}
\begin{enumerate}
\item[(1)]
Write $z = u {\pi}_1^{e_1} \cdots {\pi}_m^{e_m}$ as Exercise 1.17.
So
$$z^p = u^p {\pi}_1^{pe_1} \cdots {\pi}_m^{pe_m}.$$
\item[(2)]
Factorize $x + y\omega = v q_1^{f_1} \cdots q_n^{f_n}$,
where $v$ is unit and all $q_h$ $(1 \leq h \leq n)$ are distinct primes in $\mathbb{Z}[\omega]$
and $f_h \in \mathbb{Z}^+$.
Since $\mathbb{Z}[\omega]$ is a UFD,
for every $q_h \mid x + y\omega$, there is some $k(h)$ such that $q_h = \pi_{k(h)}$
and also $q_h^{f_h} = \pi_{k(h)}^{pe_{k(h)}}$ or $f_h = pe_{k(h)}$.
\item[(3)]
Hence,
$$x + y\omega = v \left( \pi_{k(1)}^{e_{k(1)}} \cdots \pi_{k(n)}^{e_{k(n)}} \right)^p,$$
where $\alpha = \pi_{k(1)}^{e_{k(1)}} \cdots \pi_{k(n)}^{e_{k(n)}} \in \mathbb{Z}[\omega]$
and $v$ is unit.
\end{enumerate}
$\Box$ \\\\



%%%%%%%%%%%%%%%%%%%%%%%%%%%%%%%%%%%%%%%%%%%%%%%%%%%%%%%%%%%%%%%%%%%%%%%%%%%%%%%%



\textbf{Exercise 1.19.}
\emph{Dropping the assumption that $\mathbb{Z}[\omega]$ is a UFD but
using the fact that ideals factor uniquely (up to order) into prime ideals,
show that the principal ideal $(x+y\omega)$ has no prime ideal factor in common with
any of the other principal ideals on the left side of the equation
$$(x+y)(x+y\omega)(x+y\omega^2) \cdots (x+y\omega^{p-1}) = (z)^p$$
in which all factors are interpreted as principal ideals.
(Hint: Modify the proof of Exercise 1.17 appropriately,
using the fact that if $A$ is an ideal dividing another ideal $B$,
then $A \supseteq B$.)} \\

\emph{Proof.}
Write $$(z) = {\pi}_1^{e_1} \cdots {\pi}_m^{e_m}$$ where
$\pi_k$ $(1 \leq k \leq m)$ are distinct prime ideals of $\mathbb{Z}[\omega]$ and
$e_k \in \mathbb{Z}^+$ $(1 \leq k \leq m)$.
By assumption, the factorization of $z$ is unique up to order.
\begin{enumerate}
\item[(1)]
\emph{Show that $\pi \mid (z)$.}
Since $\pi \mid (x + y\omega)$, $\pi \mid (z)^p$.
The factorization of $(z)^p$ is
$$(z)^p = {\pi}_1^{pe_1} \cdots {\pi}_m^{pe_m}.$$
$\pi | (z)^p$ implies that $\pi = \pi_k$ for some $k$,
that is, $\pi \mid (z)$.
\item[(2)]
\emph{Show that $\pi \mid (yp)$ if $\pi$ were divide any of
the other factors on the left side of
$(x+y)(x+y\omega)(x+y\omega^2) \cdots (x+y\omega^{p-1}) = (z)^p$.}
Say $\pi \mid (x+y\omega^k)$ for some $k \neq 1$.
So that $x+y\omega \in \pi$ and $x+y\omega^k \in \pi$,
or $y(\omega - \omega^k) \in \pi$.
Since $k \neq 1$, there are two possible cases.
  \begin{enumerate}
  \item[(a)]
  $k > 1$. $y\omega(1 - \omega^{k-1}) \in \pi$.
  By Exercise 1.16, $y\omega p \in \pi$, or $yp \in \pi$ since $\omega$ is unit.
  ($\omega^{p-1}$ is the inverse of $\omega$ since $\omega \cdot \omega^{p-1} = 1$.)
  \item[(b)]
  $k = 0$. $y(\omega - 1) \in \pi$, or $y(1 - \omega) \in \pi$.
  By Exercise 1.16, $yp \in \pi$.
  \end{enumerate}
In any case, $yp \in \pi$, or $\pi \mid (yp)$.
\item[(3)]
Note that $z$ and $yp$ are integers, and they are relatively prime by the assumption
that $p$ divides none of $x, y, z$.
Therefore, on $\mathbb{Z}$ we have $zm + ypn = 1$ for some $m, n \in \mathbb{Z}$.
\item[(4)]
$zm + ypn = 1$ is also true in $\mathbb{Z}[\omega]$.
Therefore, by (1)(2) we have $z \in \pi$ and $yp \in \pi$.
So $zm + ypn \in \pi$ since $\pi$ is an ideal.
So $1 \in \pi$ or $\pi = (1)$, contrary to the primality of $\pi$.
\end{enumerate}
$\Box$ \\\\



%%%%%%%%%%%%%%%%%%%%%%%%%%%%%%%%%%%%%%%%%%%%%%%%%%%%%%%%%%%%%%%%%%%%%%%%%%%%%%%%



\textbf{Exercise 1.20.}
\emph{Use Exercise 1.19 to show that
$(x+y\omega) = I^p$ for some ideal $I$. } \\

\emph{Proof.}
\begin{enumerate}
\item[(1)]
Write $(z) = {\pi}_1^{e_1} \cdots {\pi}_m^{e_m}$ as Exercise 1.17.
So
$$(z)^p = {\pi}_1^{pe_1} \cdots {\pi}_m^{pe_m}.$$
\item[(2)]
Factorize $(x + y\omega) = q_1^{f_1} \cdots q_n^{f_n}$,
where every $q_h$ $(1 \leq h \leq n)$ are distinct prime ideals of $\mathbb{Z}[\omega]$
and $f_h \in \mathbb{Z}^+$.
By assumption that $\mathbb{Z}[\omega]$ is a Dedekind domain,
for every $q_h \mid (x + y\omega)$, there is some $k(h)$ such that $q_h = \pi_{k(h)}$
and also $q_h^{f_h} = \pi_{k(h)}^{pe_{k(h)}}$ or $f_h = pe_{k(h)}$.
\item[(3)]
Hence,
$$(x + y\omega) = \left( \pi_{k(1)}^{e_{k(1)}} \cdots \pi_{k(n)}^{e_{k(n)}} \right)^p,$$
where $I = \pi_{k(1)}^{e_{k(1)}} \cdots \pi_{k(n)}^{e_{k(n)}}$
is an ideal of $\mathbb{Z}[\omega]$.
\end{enumerate}
$\Box$ \\\\



%%%%%%%%%%%%%%%%%%%%%%%%%%%%%%%%%%%%%%%%%%%%%%%%%%%%%%%%%%%%%%%%%%%%%%%%%%%%%%%%



\textbf{Exercise 1.21.}
\emph{Show that every number of $\mathbb{Q}[\omega]$ is uniquely representable in the form
$$a_0 + a_1\omega + a_2\omega^2 + \cdots + a_{p-2}\omega^{p-2},
a_i \in \mathbb{Q} \:\: \forall \: i$$
by show that $\omega$ is a root of the polynomial
$$f(t) = t^{p-1} + t^{p-2} + \cdots + t + 1$$
and that $f(t)$ is irreducible over $\mathbb{Q}$.
(Hint: It is enough to show that $f(t+1)$ is irreducible,
which can be established by Eisenstein's criterion.
It helps to notice that $f(t+1) = \frac{(t+1)^p-1}{t}$.) } \\

\emph{Proof.}
\begin{enumerate}
\item[(1)]
\emph{Given any number $\alpha \in \mathbb{Q}[\omega]$.
Show that}
$$\alpha = a_0 + a_1\omega + a_2\omega^2 + \cdots + a_{p-2}\omega^{p-2},
a_i \in \mathbb{Q} \:\: \forall \: i.$$
Since $\omega^p = 1$, we can write
$$\alpha = a_0' + a_1'\omega + a_2'\omega^2 + \cdots + a_{p-2}'\omega^{p-2} + a_{p-1}'\omega^{p-1},
a_i \in \mathbb{Q} \:\: \forall \: i.$$
Note that $\omega^{p-1} + \omega^{p-2} + \cdots + \omega + 1 = 0$,
and thus we can replace $\omega^{p-1}$
by $- \omega^{p-2} - \cdots - \omega - 1$.
\item[(2)]
\emph{Show that $\omega$ is a root of the polynomial
$f(t) = t^{p-1} + t^{p-2} + \cdots + t + 1$.}
$f(\omega) = \omega^{p-1} + \omega^{p-2} + \cdots + \omega + 1 = 0$.
\item[(3)]
\emph{Show that $f(t)$ is irreducible over $\mathbb{Q}$.}
It suffices to show that $f(t+1)$ is irreducible over $\mathbb{Q}$.
Write $(t-1)f(t) = t^p - 1$. So
\begin{align*}
tf(t+1)
&= (t+1)^p - 1
  &\text{(Put $t \mapsto t+1$)} \\
&= \left( \sum_{k=0}^{p}{p \choose k}t^k \right) - 1
  &\text{(Binomial theorem)} \\
&= \sum_{k=1}^{p}{p \choose k}t^k, \\
f(t+1)
&= \sum_{k=1}^{p}{p \choose k}t^{k-1} \\
&= t^{p-1} + p t^{p-2} + \cdots + \frac{p(p-1)}{2} t + p.
\end{align*}
By Eisenstein's criterion, $f(t+1)$ is irreducible over $\mathbb{Q}$.
\item[(4)]
To show the uniqueness, it suffices to show that the relation
$$0 = a_0 + a_1\omega + a_2\omega^2 + \cdots + a_{p-2}\omega^{p-2}$$
implies all $a_i = 0$.
Say $g(t) = a_0 + a_1 t + a_2 t^2 + \cdots + a_{p-2} t^{p-2} \in \mathbb{Q}[t]$.
Clearly $g(\omega) = 0$. By the minimality of $f(t)$, $g(t)$ is identical zero,
or all $a_i = 0$.
\end{enumerate}
$\Box$ \\\\



%%%%%%%%%%%%%%%%%%%%%%%%%%%%%%%%%%%%%%%%%%%%%%%%%%%%%%%%%%%%%%%%%%%%%%%%%%%%%%%%



\textbf{Exercise 1.22.}
\emph{Use Exercise 1.21 to show that if
$\alpha \in \mathbb{Z}[\omega]$ and $p \mid \alpha$,
then (writing $\alpha = a_0 + a_1 \omega + \cdots + a_{p-2} \omega^{p-2}$,
$a_i \in \mathbb{Z}$) all $a_i$ are divisible by $p$. } \\

\emph{Proof.}
Since $p \mid \alpha$, there is $\beta \in \mathbb{Z}[\omega]$
such that $\alpha = p\beta$.
Write
\begin{align*}
\alpha &= a_0 + a_1 \omega + \cdots + a_{p-2} \omega^{p-2}, \\
\beta &= b_0 + b_1 \omega + \cdots + b_{p-2} \omega^{p-2},
\end{align*}
where $a_i, b_j \in \mathbb{Z}$.
By $\alpha = p\beta$ and Exercise 1.21, $a_i = pb_i$ for every $1 \leq i \leq p-2$.
So all $a_i$ are divisible by $p$.
$\Box$ \\\\



%%%%%%%%%%%%%%%%%%%%%%%%%%%%%%%%%%%%%%%%%%%%%%%%%%%%%%%%%%%%%%%%%%%%%%%%%%%%%%%%



\emph{Define congruence mod $p$ for $\beta, \gamma \in \mathbb{Z}[\omega]$
as follows:
$$\beta \equiv \gamma \pmod{p}
\text{ iff }
\beta - \gamma = \delta p \text{ for some } \delta \in \mathbb{Z}[\omega].$$
(Equivalently, this is congruence mod the principal ideal $p\mathbb{Z}[\omega]$.} \\\\



\textbf{Exercise 1.23.}
\emph{Show that if $\beta \equiv \gamma \pmod{p}$,
then $\overline{\beta} \equiv \overline{\gamma} \pmod{p}$
where the bar denotes complex conjugation. } \\

\emph{Proof.}
\begin{enumerate}
\item[(1)]
\emph{Show that $\overline{\delta} \in \mathbb{Z}[\omega]$ for any
$\delta \in \mathbb{Z}[\omega]$. }
Write $$\delta = a_0 + a_1 \omega + \cdots + a_{p-1} \omega^{p-1}$$
where $a_0, \ldots, a_{p-1} \in \mathbb{Z}$.
Take the complex conjugation to get
\begin{align*}
\overline{\delta}
&= \overline{a_0} + \overline{a_1} \cdot \overline{\omega} + \cdots
  + \overline{a_{p-1}} \cdot \overline{\omega}^{p-1} \\
&= a_0 + a_1 \overline{\omega} + \cdots + a_{p-1} \overline{\omega}^{p-1}
  &\text{(Every $a_k \in \mathbb{Z}$)} \\
&= a_0 + a_1 \omega^{p-1} + \cdots + a_{p-1} \omega \in \mathbb{Z}[\omega].
  &\text{($\omega^p = 1$)}
\end{align*}
\item[(2)]
\begin{align*}
&\beta \equiv \gamma \pmod{p} \\
\Longleftrightarrow&
\beta - \gamma = \delta p \text{ for some } \delta \in \mathbb{Z}[\omega] \\
\Longleftrightarrow&
\overline{\beta} - \overline{\gamma}
= \overline{\delta} p \text{ for some } \delta \in \mathbb{Z}[\omega]
  &\text{(Complex conjugation)} \\
\Longleftrightarrow&
\overline{\beta} - \overline{\gamma}
= \delta' p \text{ for some } \delta' \in \mathbb{Z}[\omega]
  &\text{((1))} \\
\Longleftrightarrow&
\overline{\beta} \equiv \overline{\gamma} \pmod{p}
\end{align*}
\end{enumerate}
$\Box$ \\\\



%%%%%%%%%%%%%%%%%%%%%%%%%%%%%%%%%%%%%%%%%%%%%%%%%%%%%%%%%%%%%%%%%%%%%%%%%%%%%%%%



\textbf{Exercise 1.24.}
\emph{Show that
$(\beta+\gamma)^p \equiv \beta^p + \gamma^p \pmod{p}$
and generalize this to sums of arbitrarily many terms by induction. } \\

\emph{Proof.}
\begin{enumerate}
\item[(1)]
Binomial theorem gives us
$$
(\beta+\gamma)^p
= \sum_{k=0}^{p}{p \choose k}\beta^k\gamma^{p-k}
= \beta^p+\gamma^p + \sum_{k=1}^{p-1}{p \choose k}\beta^k\gamma^{p-k}.$$
\item[(2)]
Note that every binomial coefficient ${p \choose k}$ is divided by $p$ in $\mathbb{Z}$
for $1 \leq k \leq p-1$.
Also, every term $\beta^k\gamma^{p-k}$ is in $\mathbb{Z}[\omega]$.
So $(\beta+\gamma)^p - \beta^p - \gamma^p = \delta p$
for some $\delta \in \mathbb{Z}[\omega]$.
Hence the result holds.
\item[(3)]
\emph{In general,}
$$\left( \sum_{k=1}^{n} \alpha_k \right)^p
\equiv \sum_{k=1}^{n} \alpha_k^p \pmod{p}.$$
Induction by
$(\alpha_1+\alpha_2)^p \equiv \alpha_1^p + \alpha_2^p \pmod{p}$
and
$\left( \sum_{k=1}^{n+1} \alpha_k \right)^p
\equiv \left( \sum_{k=1}^{n} \alpha_k \right)^p + \alpha_{n+1}^p
\equiv \left( \sum_{k=1}^{n} \alpha_k^p \right) + \alpha_{n+1}^p
\equiv \sum_{k=1}^{n+1} \alpha_k^p \pmod{p}.$
\end{enumerate}
$\Box$ \\\\



%%%%%%%%%%%%%%%%%%%%%%%%%%%%%%%%%%%%%%%%%%%%%%%%%%%%%%%%%%%%%%%%%%%%%%%%%%%%%%%%



\textbf{Exercise 1.25.}
\emph{Show that for all $\alpha \in \mathbb{Z}[\omega]$,
$\alpha^p$ is congruent $\pmod{p}$ to some $a \in \mathbb{Z}$.
(Hint: Write $\alpha$ in terms of $\omega$ and use Exercise 1.24.) } \\

\emph{Proof (Hint).}
Write $$\alpha = a_0 + a_1 \omega + \cdots + a_{p-1} \omega^{p-1}$$
where $a_0, \ldots, a_{p-1} \in \mathbb{Z}$.
By Exercise 1.24,
\begin{align*}
\alpha^p
&\equiv a_0^p + (a_1 \omega)^p + \cdots + (a_{p-1} \omega^{p-1})^p \\
&\equiv a_0^p + a_1^p \omega^p + \cdots + a_{p-1}^p (\omega^{p-1})^p \\
&\equiv a_0^p + a_1^p \omega^p + \cdots + a_{p-1}^p (\omega^p)^{p-1} \\
&\equiv a_0^p + a_1^p + \cdots + a_{p-1}^p.
  &\text{($\omega^p = 1$)}
\end{align*}
Here $a_0^p + a_1^p + \cdots + a_{p-1}^p \in \mathbb{Z}$,
and thus $\alpha^p$ is congruent $\pmod{p}$ to some integer.
$\Box$ \\\\



%%%%%%%%%%%%%%%%%%%%%%%%%%%%%%%%%%%%%%%%%%%%%%%%%%%%%%%%%%%%%%%%%%%%%%%%%%%%%%%%



\emph{Exercise 1.26-1.28: Now assume $p \geq 5$.
We will show that if $x+y\omega = u\alpha^p \pmod{p}$,
$\alpha \in \mathbb{Z}[\omega]$,
$u$ a unit in $\mathbb{Z}[\omega]$,
$x$ and $y$ integers not divisible by $p$,
then $x \equiv y \pmod{p}$.
For this we will need the following result, proved by Kummer,
on the units of $\mathbb{Z}[\omega]$:} \\

\emph{Lemma: If $u$ is a unit in $\mathbb{Z}[\omega]$ and
$\overline{u}$ is its complex conjugate, then $u/\overline{u}$
is a power of $\omega$.
(For the proof, see Exercise 2.12.)} \\\\



%%%%%%%%%%%%%%%%%%%%%%%%%%%%%%%%%%%%%%%%%%%%%%%%%%%%%%%%%%%%%%%%%%%%%%%%%%%%%%%%



\textbf{Exercise 1.26.}
\emph{Show that $x + y\omega \equiv u \alpha^p \pmod{p}$ implies
$$x + y\omega \equiv (x + y\omega^{-1}) \omega^k \pmod{p}$$
for some $k \in \mathbb{Z}$.
(Use the Lemma on units and Exercise 1.23 and 1.25.
Note that $\overline{\omega} = \omega^{-1}$.) } \\

\emph{Proof (Hint).}
\begin{align*}
&x + y\omega \equiv u \alpha^p \pmod{p} \\
\Longrightarrow&
x + y\omega \equiv ua \pmod{p} \text{ for some } a \in \mathbb{Z}
  &\text{(Exercise 1.25)} \\
\Longrightarrow&
\overline{x + y\omega} \equiv \overline{u a} \pmod{p}
  &\text{(Exercise 1.23)} \\
\Longrightarrow&
x + y\overline{\omega} \equiv \overline{u} a \pmod{p} \\
\Longrightarrow&
x + y \omega^{-1} \equiv \overline{u} a \pmod{p}
  &\text{($\overline{\omega} = \omega^{-1}$)} \\
\Longrightarrow&
x + y \omega^{-1} \equiv u \omega^{-k} a \pmod{p} \text{ for some } k \in \mathbb{Z}
  &\text{(Lemma)} \\
\Longrightarrow&
ua \equiv (x + y \omega^{-1})\omega^{k} \pmod{p} \\
\Longrightarrow&
x + y\omega \equiv (x + y \omega^{-1})\omega^{k} \pmod{p}.
\end{align*}
$\Box$ \\\\



%%%%%%%%%%%%%%%%%%%%%%%%%%%%%%%%%%%%%%%%%%%%%%%%%%%%%%%%%%%%%%%%%%%%%%%%%%%%%%%%



\textbf{Exercise 1.27.}
\emph{Use Exercise 1.22 to show that a contradiction results unless $k \equiv 1 \pmod{p}$.
(Recall that $p \nmid xy$, $p \geq 5$, and
$\omega^{p-1} + \omega^{p-2} + \cdots + \omega + 1 = 0$.) } \\

\emph{Proof.}
Exercise 1.26 shows
$$x + y\omega \equiv (x + y\omega^{-1}) \omega^k \pmod{p}.$$
Multiply $\omega$ on the both sides to get
$x\omega + y\omega^2 \equiv y\omega^{k} + x\omega^{k+1} \pmod{p}$, or
$$p \mid (x\omega + y\omega^2 - y\omega^{k} - x\omega^{k+1}).$$
If $k$ were satisfying $k \not\equiv 1 \pmod{p}$, then by Exercise 1.22 and $p \geq 5$
we have $p \mid x$ or $p \mid y$, contrary to the assumption that
$x$ and $y$ are integers not divisible by $p$.
$\Box$ \\\\



%%%%%%%%%%%%%%%%%%%%%%%%%%%%%%%%%%%%%%%%%%%%%%%%%%%%%%%%%%%%%%%%%%%%%%%%%%%%%%%%



\textbf{Exercise 1.28.}
\emph{Finally, show $x \equiv y \pmod{p}$.} \\

\emph{Proof.}
In the argument of Exercise 1.27 we have
$$p \mid ((x-y)\omega + (y-x)\omega^2)$$
by replacing $k = 1$.
By Exercise 1.22 and $p \geq 5$, $x - y$ is divisible by $p$,
or $x \equiv y \pmod{p}$ as integers.
$\Box$ \\\\



%%%%%%%%%%%%%%%%%%%%%%%%%%%%%%%%%%%%%%%%%%%%%%%%%%%%%%%%%%%%%%%%%%%%%%%%%%%%%%%%



\textbf{Exercise 1.29.}
\emph{Let $\omega = \exp(\frac{2\pi i}{23})$.
Verify that the product
$$(1+\omega^2+\omega^4+\omega^5+\omega^6+\omega^{10}+\omega^{11})
(1+\omega+\omega^5+\omega^6+\omega^7+\omega^9+\omega^{11})$$
is divisible by $2$ in $\mathbb{Z}[\omega]$,
although neither factor is.
It can be shown (Exercise 3.17) that $2$ is an irreducible element in $\mathbb{Z}[\omega]$;
it follows that $\mathbb{Z}[\omega]$ cannot be a UFD. } \\

\emph{Proof.}
Note that $\sum_{k=0}^{22} \omega^k = 0$.
So
\begin{align*}
&(1+\omega^2+\omega^4+\omega^5+\omega^6+\omega^{10}+\omega^{11})
(1+\omega+\omega^5+\omega^6+\omega^7+\omega^9+\omega^{11}) \\
=& 2(\omega^5+\omega^6+\omega^7+\omega^9+\omega^{10}+3\omega^{11}
+\omega^{12}+\omega^{13}+\omega^{15}+\omega^{16}+\omega^{17})
\end{align*}
is divisible by $2$ in $\mathbb{Z}[\omega]$,
although neither factor is.
$\Box$ \\\\



%%%%%%%%%%%%%%%%%%%%%%%%%%%%%%%%%%%%%%%%%%%%%%%%%%%%%%%%%%%%%%%%%%%%%%%%%%%%%%%%



\emph{Exercise 1.30-1.32: $R$ is an integral domain
(commutative ring with $1$ and no zero divisors).} \\\\



\textbf{Exercise 1.30.}
\emph{Show that two ideals in $R$ are isomorphic as $R$-modules
iff they are in the same ideal class.} \\

\emph{Proof.}
Given any two ideals $A, B$ in an commutative integral domain $R$.
\begin{enumerate}
\item[(1)]
$(\Longrightarrow)$
Let $\varphi: A \to B$ be an $R$-module isomorphism.
Given any nonzero $\alpha \in A$, we have
\begin{align*}
\varphi(\alpha)A
&= \{ \varphi(\alpha)a : a \in A \} \\
&= \{ \varphi(\alpha a) : a \in A \}
  &\text{($\varphi$ is a homomorphism)} \\
&= \{ \alpha\varphi(a) : a \in A \}
  &\text{($\varphi$ is a homomorphism)} \\
&= \{ \alpha b : b \in B \}
  &\text{($\varphi$ is an isomorphism)} \\
&= \alpha B.
\end{align*}
Notice that $\varphi(\alpha) \neq 0$ since $\alpha \neq 0$ and $\varphi$ is injective.
Therefore, $A \sim B$.
\item[(2)]
$(\Longleftarrow)$
Given $A \sim B$, there are nonzero $\alpha, \beta \in R$ such that $\alpha A = \beta B$.
Define a map $\varphi: A \to B$ by $\varphi(a) = b$ if $\alpha a = \beta b$.
  \begin{enumerate}
  \item[(a)]
  \emph{$\varphi$ is well-defined.}
    \begin{enumerate}
    \item[(i)]
    \emph{Existence of $b$.}
    Since $\alpha a \in \alpha A = \beta B$, there is $b \in B$ such that $\alpha a = \beta b$.
    \item[(ii)]
    \emph{Uniqueness of $b$.}
    If $\alpha a = \beta b_1 = \beta b_2$, $\beta(b_1-b_2) = 0$.
    Since $R$ is an integral domain and $\beta \neq 0$, $b_1-b_2 = 0$ or $b_1 = b_2$.
    \end{enumerate}
  \item[(b)]
  \emph{$\varphi$ is an $R$-module homomorphism.}
    \begin{enumerate}
    \item[(i)]
    \emph{Show that $\varphi(a_1+a_2) = \varphi(a_1) + \varphi(a_2)$.}
    Write $\varphi(a_1) = b_1$ and $\varphi(a_2) = b_2$.
    \begin{align*}
    &\varphi(a_1) = b_1 \text{ and } \varphi(a_2) = b_2 \\
    \Longrightarrow&
    \alpha a_1 = \beta b_1 \text{ and } \alpha a_2 = \beta b_2
      &\text{(Definition of $\varphi$)} \\
    \Longrightarrow&
    \alpha a_1 + \alpha a_2 = \beta b_1 + \beta b_2
      &\text{(Add together)} \\
    \Longrightarrow&
    \alpha (a_1 + a_2) = \beta (b_1 + b_2) \\
    \Longrightarrow&
    \varphi(a_1+a_2) = b_1+b_2 = \varphi(a_1) + \varphi(a_2).
      &\text{(Definition of $\varphi$)}
    \end{align*}
    \item[(ii)]
    \emph{Show that $\varphi(ra) = r\varphi(a)$.}
    Write $\varphi(a) = b$.
    \begin{align*}
    \varphi(a) = b
    \Longrightarrow&
    \alpha a = \beta b
      &\text{(Definition of $\varphi$)} \\
    \Longrightarrow&
    r \alpha a = r \beta b
      &\text{(Multiply $r$)} \\
    \Longrightarrow&
    \alpha(ra) = \beta (rb)
      &\text{($R$ is commutative)} \\
    \Longrightarrow&
    \varphi(ra) = rb = r \varphi(a).
      &\text{(Definition of $\varphi$)}
    \end{align*}
    \end{enumerate}
  \item[(c)]
  \emph{$\varphi$ is injective.}
  Given $\varphi(a) = 0$. Then $\alpha a = \beta b = \beta 0 = 0$.
  Since $R$ is an integral domain and $\alpha \neq 0$, $a = 0$.
  \item[(d)]
  \emph{$\varphi$ is surjective.}
  Given any $b \in B$. $\beta b \in \beta B = \alpha A$.
  There is $a \in A$ such that $\beta b = \alpha a$.
  Such $a$ satisfies $\varphi(a) = b$.
  \end{enumerate}
  Therefore, $\varphi: A \to B$ is an $R$-module isomorphism.
\end{enumerate}
$\Box$ \\\\



%%%%%%%%%%%%%%%%%%%%%%%%%%%%%%%%%%%%%%%%%%%%%%%%%%%%%%%%%%%%%%%%%%%%%%%%%%%%%%%%



\textbf{Exercise 1.31.}
\emph{Show that if $A$ is an ideal in $R$ and if $\alpha A$ is principal
for some nonzero $\alpha \in R$, then $A$ is principal.
Conclude that the principal ideals form an ideal class.} \\

\emph{Proof.}
\begin{enumerate}
\item[(1)]
Write $\alpha A = (b)$ for some $b \in \alpha A$.
That is, there is $a \in A$ such that
$$b = \alpha a.$$
\item[(2)]
\emph{Show that $A = (a)$ is principal.}
$(a) \subseteq A$ holds trivially since $a \in A$ and $A$ is an ideal.
Given any $x \in A$. $\alpha x \in \alpha A = (b)$, and thus
there is $y \in R$ such that $\alpha x = b y$.
Replace $b$ by $b = \alpha a$ to get $\alpha x = \alpha a y$ or
$$\alpha (x - ay) = 0.$$
Since $\alpha \neq 0$ and $R$ is an integral domain,
$x - ay = 0$ or $x = ay \in (a)$ or $A \subseteq (a)$.
Hence $A = (a)$ is principal.
\item[(3)]
\emph{Show that the principal ideals form an ideal class.}
Given any $A = (a) \neq 0$ and $B = (b) \neq 0$,
we have $bA = aB = (ab)$ for $a, b \in R$ or $A \sim B$.
\end{enumerate}
$\Box$ \\\\



%%%%%%%%%%%%%%%%%%%%%%%%%%%%%%%%%%%%%%%%%%%%%%%%%%%%%%%%%%%%%%%%%%%%%%%%%%%%%%%%



\textbf{Exercise 1.32.}
\emph{Show that the ideal classes in $R$ form a group iff for every ideal $A$
there is an ideal $B$ such that $AB$ is principal. } \\

\emph{Note.}
The Picard group of the spectrum of a Dedekind domain is its ideal class group. \\

\emph{Proof.}
Let $[A]$ be the ideal class representing by a nonzero ideal $A$ of $R$.
Let
$$\text{Pic}(R) = \{ [A] : A \text{ is an ideal of } R \}$$
be the set of all ideal classes.
Define the operation $\cdot : \text{Pic}(R) \times \text{Pic}(R) \to \text{Pic}(R)$
by $[A] \cdot [B] \mapsto [AB]$.

\begin{enumerate}
\item[(1)]
\emph{(Closure) Show that the operation $[A] \cdot [B] \mapsto [AB]$ is well-defined. }
Trivial due to the definition of the ideal class.
Note that $[A] \cdot [B] = [B] \cdot [A]$ by the commutativity of $R$.
\item[(2)]
\emph{(Associativity) Show that $([A] \cdot [B]) \cdot [C] = [A] \cdot ([B] \cdot [C])$. }
Trivial due to the definition of the ideal class.
\item[(3)]
\emph{(Identity element) Show that the non-zero principal ideals form the ideal class $[1]$. }
Exercise 1.30 and note that $(1)$ is principal too.
\item[(4)]
\emph{Show that the set $\text{Pic}(R)$ forms an (abelian) group with $[1]$
as the identity element if and only if every $[A]$ has an inverse in $\text{Pic}(R)$.}
By (1)(2)(3), the set $\text{Pic}(R)$ forms an (abelian) group iff
every element has an inverse element. The conclusion is established.
\end{enumerate}
$\Box$ \\\\



% No exercises left.

%%%%%%%%%%%%%%%%%%%%%%%%%%%%%%%%%%%%%%%%%%%%%%%%%%%%%%%%%%%%%%%%%%%%%%%%%%%%%%%%
%%%%%%%%%%%%%%%%%%%%%%%%%%%%%%%%%%%%%%%%%%%%%%%%%%%%%%%%%%%%%%%%%%%%%%%%%%%%%%%%



\end{document}