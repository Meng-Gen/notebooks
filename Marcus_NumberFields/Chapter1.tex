\documentclass{article}
\usepackage{amsfonts}
\usepackage{amsmath}
\usepackage{amssymb}
\usepackage{hyperref}
\usepackage[none]{hyphenat}
\usepackage{mathrsfs}
\parindent=0pt

\def\upint{\mathchoice%
    {\mkern13mu\overline{\vphantom{\intop}\mkern7mu}\mkern-20mu}%
    {\mkern7mu\overline{\vphantom{\intop}\mkern7mu}\mkern-14mu}%
    {\mkern7mu\overline{\vphantom{\intop}\mkern7mu}\mkern-14mu}%
    {\mkern7mu\overline{\vphantom{\intop}\mkern7mu}\mkern-14mu}%
  \int}
\def\lowint{\mkern3mu\underline{\vphantom{\intop}\mkern7mu}\mkern-10mu\int}

\begin{document}

\textbf{\Large Chapter 1: A Special Case of Fermat's Conjecture} \\\\



\emph{Author: Meng-Gen Tsai} \\
\emph{Email: plover@gmail.com} \\\\



% https://www.math.uci.edu/~ndonalds/math180b/6gaussian.pdf
% https://www.maths.nottingham.ac.uk/plp/pmzcw/download/fnt_chap5.pdf



\emph{Exercise 1.1-1.9: Define $N: \mathbb{Z}[i] \rightarrow \mathbb{Z}$ by
$N(a+bi) = a^2 + b^2$.} \\\\



\textbf{Exercise 1.1.}
\emph{Verify that for all $\alpha, \beta \in \mathbb{Z}[i]$,
$N(\alpha\beta) = N(\alpha)N(\beta)$,
either by direct computation or using the fact that
$N(a+bi) = (a+bi)(a-bi)$.
Conclude that if $\alpha \mid \gamma$ in $\mathbb{Z}[i]$,
then $N(\alpha) \mid N(\gamma)$ in $\mathbb{Z}$.} \\

\emph{Proof.}
\begin{enumerate}
\item[(1)]
\emph{Direct computation.}
Write $\alpha = a+bi, \beta=c+di$ where $a, b, c, d \in \mathbb{Z}$.
Thus,
\begin{align*}
N(\alpha\beta)
&= N((a+bi)(c+di)) \\
&= N((ac-bd) + (ad+bc)i) \\
&= (ac-bd)^2 + (ad+bc)^2 \\
&= (a^2 c^2 - 2abcd + b^2 d^2) + (a^2 d^2 + 2abcd + b^2 c^2) \\
&= a^2 c^2 + b^2 d^2 + a^2 d^2 + b^2 c^2, \\
N(\alpha)N(\beta)
&= N(a+bi) N(c+di) \\
&= (a^2 + b^2)(c^2 + d^2) \\
&= a^2 c^2 + b^2 d^2 + a^2 d^2 + b^2 c^2.
\end{align*}
Therefore, $N(\alpha\beta) = N(\alpha)N(\beta)$.
(Note that we also get the identity
$(a^2 + b^2)(c^2 + d^2) = (ac-bd)^2 + (ad+bc)^2$.)
\item[(2)]
\emph{Using the fact that $N(a+bi) = (a+bi)(a-bi)$,}
or $N(\alpha) = \alpha \overline{\alpha}$
for any $\alpha \in \mathbb{Z}[i]$.
Thus,
\begin{align*}
N(\alpha\beta)
&= \alpha\beta\overline{\alpha\beta} \\
&= \alpha\beta\overline{\alpha}\overline{\beta} \\
&= \alpha\overline{\alpha}\beta\overline{\beta} \\
&= N(\alpha)N(\beta).
\end{align*}
\item[(3)]
\emph{Show that if $\alpha \mid \gamma$ in $\mathbb{Z}[i]$,
then $N(\alpha) \mid N(\gamma)$ in $\mathbb{Z}$.}
Write $\gamma = \alpha \beta$ for some $\beta \in \mathbb{Z}[i]$.
So $N(\gamma) = N(\alpha) N(\beta) \in \mathbb{Z}$,
or $N(\alpha) \mid N(\gamma)$ in $\mathbb{Z}$.
\end{enumerate}
$\Box$ \\\\



\textbf{Exercise 1.2.}
\emph{Let $\alpha \in \mathbb{Z}[i]$.
Show that $\alpha$ is a unit iff $N(\alpha) = 1$.
Conclude that the only unit are $\pm 1$ and $\pm i$.} \\

\emph{Proof.}
\begin{enumerate}
\item[(1)]
\emph{$(\Longrightarrow)$}
Since $\alpha$ is a unit, there is $\beta \in \mathbb{Z}[i]$ such that
$\alpha \beta = 1$.
By Exercise 1.1, $N(\alpha \beta) = N(1)$, or $N(\alpha) N(\beta) = 1$.
Since the image of $N$ is nonnegative integers, $N(\alpha) = 1$.
\item[(2)]
\emph{$(\Longleftarrow)$}
By Exercise 1.1, $N(\alpha) = \alpha \overline{\alpha}$,
or $1 = \alpha \overline{\alpha}$ since $N(\alpha) = 1$.
That is, $\overline{\alpha} \in \mathbb{Z}[i]$ is
the inverse of $\alpha \in \mathbb{Z}[i]$.
(Or by (1), we solve the equation $N(\alpha) = a^2 + b^2 = 1$,
and show that all four solutions ($\pm 1$ and $\pm i$) are unit.)
\end{enumerate}
Conclusion: a unit $\alpha = a+bi$ of $\mathbb{Z}[i]$
is satisfying the equation $N(\alpha) = a^2 + b^2 = 1$ by (1)(2).
That is, the only unit of $\mathbb{Z}[i]$ are $\pm 1$ and $\pm i$.
$\Box$ \\\\



\textbf{Exercise 1.3.}
\emph{Let $\alpha \in \mathbb{Z}[i]$.
Show that if $N(\alpha)$ is a prime in $\mathbb{Z}$ then
$\alpha$ is irreducible in $\mathbb{Z}[i]$.
Show that the same conclusion holds
if $N(\alpha) = p^2$, where $p$ is a prime in $\mathbb{Z}$,
$p \equiv 3 \pmod{4}$.} \\

\emph{Proof.}
\begin{enumerate}
\item[(1)]
\emph{Show that if $N(\alpha)$ is a prime in $\mathbb{Z}$ then
$\alpha$ is irreducible in $\mathbb{Z}[i]$.}
Write $\alpha = \beta\gamma$.
Then $N(\alpha) = N(\beta)N(\gamma)$ is a prime in $\mathbb{Z}$.
Since each integer prime is irreducible, $N(\beta) = 1$ or $N(\gamma) = 1$.
So that $\beta$ is unit or $\gamma$ is unit by Exercise 1.2.
Hence, $\alpha$ is irreducible.
\item[(2)]
\emph{Show that $\alpha$ is irreducible in $\mathbb{Z}[i]$
if $N(\alpha) = p^2$, where $p$ is a prime in $\mathbb{Z}$,
$p \equiv 3 \pmod{4}$.}
Assume $\alpha = \beta\gamma$ were not irreducible.
Similar to (1), $N(\alpha) = N(\beta)N(\gamma) = p^2$.
Since $\beta$ and $\gamma$ are proper factors of $\alpha$,
$$N(\beta) = N(\gamma) = p.$$
Since any square $a^2 \equiv 0, 1 \pmod{4}$,
any $N(a+bi) = a^2+b^2 \equiv 0, 1, 2 \pmod{4}$.
Especially, $N(\beta) \equiv 0, 1, 2 \pmod{4}$,
contrary to $N(\beta) = p \equiv 3 \pmod{4}$
by the assumption.
Therefore, $\alpha$ is irreducible in $\mathbb{Z}[i]$.
\end{enumerate}
$\Box$ \\

\textbf{Supplement.}
\begin{enumerate}
\item[(1)]
The prime $2$ is reducible in $\mathbb{Z}[i]$ (Exercise 1.4).
\item[(2)]
Every prime $p \equiv 1 \pmod{4}$ is reducible in $\mathbb{Z}[i]$ (Exercise 1.8). \\\\
\end{enumerate}



\textbf{Exercise 1.4.}
\emph{Show that $1-i$ is irreducible in $\mathbb{Z}$
and that $2 = u(1-i)^2$ for some unit $u$.} \\

\emph{Proof.}
\begin{enumerate}
\item[(1)]
\emph{$1-i$ is irreducible.}
Since $N(1-i) = 2$ is a prime in $\mathbb{Z}$,
$1-i$ is irreducible by Problem 1.3.
\item[(2)]
$2 = i(1-i)^2$ where $i$ is unit in $\mathbb{Z}$.
\end{enumerate}
$\Box$ \\\\



\textbf{Exercise 1.5.}
\emph{Notice that
$(2+i)(2-i) = 5 = (1+2i)(1-2i)$.
How is this consistent with unique factorization?} \\

\emph{Proof.}
Since $2+i = i(1-2i)$ and $2-i = (-i)(1+2i)$,
the factorization is unique up to order and multiplication of primes by units.
$\Box$ \\\\



\textbf{Exercise 1.6.}
\emph{Show that every nonzero, non-unit Gaussian integer $\alpha$
is a product of irreducible elements, by induction on $N(\alpha)$.} \\

\emph{Proof.}
Induction on $N(\alpha)$.
\begin{enumerate}
\item[(1)]
\emph{$n = 2$}.
Given $\alpha \in \mathbb{Z}[i]$ with $N(\alpha) = 2$.
Since $N(\alpha) = 2$ is a prime in $\mathbb{Z}$,
$\alpha$ is irreducible (Exercise 1.3).
\item[(2)]
\emph{Suppose the result holds for $n \leq k$.}
Given $\alpha \in \mathbb{Z}[i]$ with $N(\alpha) = k+1$.
There are only two possible cases.
\begin{enumerate}
\item[(a)]
\emph{$\alpha$ is irreducible.}
Nothing to do.
\item[(b)]
\emph{$\alpha$ is reducible.}
Write $\alpha = \beta\gamma$ where neither factor is unit.
Since $N(\alpha) = N(\beta)N(\gamma)$ and neither factor is unit,
$$2 \leq N(\beta), N(\gamma) \leq k.$$
By the induction hypothesis, each factor of $\alpha$ ($\beta$ and $\gamma$)
is a product of irreducible elements.
So that $\alpha$ again is a product of irreducible elements.
\end{enumerate}
In any cases,
$\alpha$ is a product of irreducible elements.
\end{enumerate}
By induction, the result is established.
$\Box$ \\\\



\textbf{Exercise 1.7.}
\emph{Show that $\mathbb{Z}[i]$ is a principal ideal domain (PID); i.e.,
every ideal $I$ is principal.
(As shown in Appendix 1, this implies that $\mathbb{Z}[i]$ is a UFD.)} \\

\emph{Suggestion: Take $\alpha \in I-\{0\}$ such that $N(\alpha)$ is minimized,
and consider the multiplies $\gamma\alpha$, $\gamma \in \mathbb{Z}[i]$;
show that these are the vertices of an infinite family of squares which
fill up the complex plane.
(For example,
one of the squares has vertices $0$, $\alpha$, $i\alpha$, and $(1+i)\alpha$;
all others are translates of this one.)
Obviously $I$ contains all $\gamma\alpha$;
show by a geometric argument that if $I$ contains anything else then
minimality of $N(\alpha)$ would be contradicted.} \\

\emph{Proof (without geometric intuition).}
Define $N$ on $\mathbb{Q}[i]$ by $N(a+bi) = a^2+b^2$ where
$a+bi \in \mathbb{Q}[i]$ as usual.
\begin{enumerate}
\item[(1)]
\emph{Show that $\mathbb{Z}[i]$ is a Euclidean domain.}
Given $\alpha = a+bi \in \mathbb{Z}[i]$ and
$\gamma = c+di \in \mathbb{Z}[i]$ with $\gamma \neq 0$.
It suffices to show there exist $\delta$ and $\rho$ such that the identity
$\alpha = \gamma\delta + \rho$ holds and
either $\rho = 0$ or $N(\rho) < N(\gamma)$.
\begin{enumerate}
\item[(a)]
\emph{Pick $\delta \in \mathbb{Z}[i].$
(Intuition: Pick the `integer part' of $\frac{\alpha}{\gamma}$
as we did in integer numbers.)}
Write $\frac{\alpha}{\gamma} = r+si \in \mathbb{Q}[i]$.
Then we pick $\delta = m+ni \in \mathbb{Z}[i]$ such that
$|r-m| \leq \frac{1}{2}$ and $|s-n| \leq \frac{1}{2}$.
Therefore,
\begin{align*}
N\left( \frac{\alpha}{\gamma} - \delta \right)
&= (r-m)^2 + (s-n)^2 \\
&\leq \frac{1}{4} + \frac{1}{4} \\
&= \frac{1}{2}.
\end{align*}
\item[(b)]
\emph{Pick $\rho \in \mathbb{Z}[i]$.}
Clearly we can pick $\rho = \alpha - \gamma\delta \in \mathbb{Z}[i]$.
Therefore, $\rho = 0$ or
\begin{align*}
N(\rho)
&= N(\alpha - \gamma\delta) \\
&= N\left( \gamma\left( \frac{\alpha}{\gamma} - \delta \right) \right) \\
&= N(\gamma) N\left( \frac{\alpha}{\gamma} - \delta \right) \\
&\leq \frac{1}{2} N(\gamma) \\
&< N(\gamma).
\end{align*}
\end{enumerate}
\item[(2)]
\emph{Show that every Euclidean domain $R$ is a PID.}
Given any ideal $I$ of $R$.
Take $\alpha \in I-\{0\}$ such that $N(\alpha)$ is minimized.
\begin{enumerate}
  \item[(a)]
  $R\alpha \subseteq I$ clearly.
  \item[(b)]
  Conversely, for any $\beta \in I$, there are $\delta, \rho \in R$
  such that $\beta = \alpha \delta + \rho$, where either $\rho = 0$
  or $N(\rho) < N(\alpha)$.
  Since $\rho = \beta - \alpha \delta \in I$,
  we cannot have $N(\rho) < N(\alpha)$ by the minimality of $N(\alpha)$.
  Therefore, $\rho = 0$ and $\beta = \alpha \delta \in R\alpha$,
  or $R\alpha \supseteq I$.
\end{enumerate}
\end{enumerate}
By (1)(2),
$\mathbb{Z}[i]$ is a PID.
$\Box$ \\\\



\textbf{Exercise 1.8.}
\emph{We will use the unique factorization in $\mathbb{Z}[i]$ to prove that
every prime $p \equiv 1 \pmod{4}$ is a sum of two squares.
\begin{enumerate}
  \item[(a)]
  Use the fact that the multiplicative group $(\mathbb{Z}/p\mathbb{Z})^{\times}$
  of integers mod $p$ is cyclic to show that
  if $p \equiv 1 \pmod{4}$ then $n^2 \equiv -1 \pmod{p}$ for some $n \in \mathbb{Z}$.
  \item[(b)]
  Prove that $p$ cannot be irreducible in $\mathbb{Z}[i]$.
  (Hint: $p \mid n^2+1 = (n+i)(n-i)$.)
  \item[(c)]
  Prove that $p$ is a sum of two squares.
  (Hint: (b) shows that $p = (a+bi)(c+di)$ with neither factor a unit.
  Take norms.) \\
\end{enumerate}}

\emph{Proof of (a).}
Since the multiplicative group $(\mathbb{Z}/p\mathbb{Z})^{\times}$
of integers mod $p$ is cyclic,
$(\mathbb{Z}/p\mathbb{Z})^{\times}$ is generated by (a primitive root)
$g \in \mathbb{Z}/p\mathbb{Z}$.
$g^{p-1} = 1$,
or
$$(g^{\frac{p-1}{2}} - 1)(g^{\frac{p-1}{2}} + 1) = 0$$
since $p$ is odd.
Since $\mathbb{Z}/p\mathbb{Z}$ is an integral domain,
$g^{\frac{p-1}{2}} - 1 = 0$ or $g^{\frac{p-1}{2}} + 1 = 0$.
$g$ cannot satisfy $g^{\frac{p-1}{2}} - 1 = 0$ since
$g$ is a generator of $(\mathbb{Z}/p\mathbb{Z})^{\times}$.
So,
$$g^{\frac{p-1}{2}} + 1 = 0.$$
Let $n = g^{\frac{p-1}{4}} \in \mathbb{Z}$ since $p \equiv 1 \pmod{4}$.
So $n^2+1 = 0 \pmod{p}$.
$\Box$ \\

\emph{Proof of (b).}
Since $n^2 + 1 \equiv 0 \pmod{p}$ by (a),
$p \mid n^2+1 = (n+i)(n-i)$.
If $p$ were irreducible in $\mathbb{Z}[i]$,
$p \mid (n+i)$ or $p \mid (n-i)$ by using the unique factorization in $\mathbb{Z}[i]$.
Hence
$$\frac{n+i}{p} = \frac{n}{p} + \frac{1}{p}i \not\in \mathbb{Z}[i],
\frac{n-i}{p} = \frac{n}{p} - \frac{1}{p}i \not\in \mathbb{Z}[i],$$
contrary to the assumption.
Therefore, $p$ is reducible in $\mathbb{Z}[i]$.
$\Box$ \\

\emph{Proof of (c).}
Since $p$ is reducible in $\mathbb{Z}[i]$ by (b),
write $p = (a+bi)(c+di)$ with neither factor a unit.
Take norms,
$$p^2 = N(p) = N(a+bi)N(c+di).$$
Since neither factor of $p$ is unit,
$N(a+bi) = p$,
or $a^2+b^2 = p$,
or $p$ is a sum of two squares.
$\Box$ \\\\



\textbf{Exercise 1.9.}
\emph{Describe all irreducible elements in $\mathbb{Z}[i]$.} \\

\emph{Notice that $\alpha$ is irreducible if and only if $\overline{\alpha}$ is irreducible.}
(Write $\alpha = \beta\gamma$,
then $\overline{\alpha} = \overline{\beta}\overline{\gamma}$.
Besides, $\overline{\overline{\alpha}} = \alpha$.) \\

\emph{Proof.}
\emph{Show that all irreducible elements in $\mathbb{Z}[i]$ (up to units) are
\begin{enumerate}
  \item[(1)]
  $1+i$.
  \item[(2)]
  $\pi = a+bi$ for each integer prime $p \equiv 1 \pmod{4}$
  with $p = a^2+b^2$.
  \item[(3)]
  $p$ for each integer prime $p \equiv 3 \pmod{4}$.
\end{enumerate}}

Let $\alpha$ be any irreducible element in $\mathbb{Z}[i]$.
Consider $N(\alpha) = \alpha \overline{\alpha}$.
$N(\alpha) \neq 1$ since $\alpha$ is not unit.
By the unique factorization theorem in $\mathbb{Z}$,
$N(\alpha) \in \mathbb{Z}$ is a product of primes in $\mathbb{Z}$. \\

There are three possible cases.
\begin{enumerate}
  \item[(a)]
  \emph{$2 \mid N(\alpha)$.}
  Write $(1+i)(1-i) \mid \alpha \overline{\alpha}$ in $\mathbb{Z}[i]$.
  Notice that $1+i$, $1-i$, $\alpha$ and $\overline{\alpha}$ are all irreducible (Exercise 1.4).
  By the unique factorization theorem in $\mathbb{Z}[i]$,
  $\alpha = 1+i$ (up to units).
  \item[(b)]
  \emph{$p \mid N(\alpha)$ for some prime $p \equiv 3 \pmod{4}$.}
  Write $p \mid \alpha \overline{\alpha}$ in $\mathbb{Z}[i]$.
  Notice that $p$, $\alpha$ and $\overline{\alpha}$ are all irreducible (Exercise 1.3).
  By the unique factorization theorem in $\mathbb{Z}[i]$,
  $\alpha = p$ (up to units) or $\overline{\alpha} = p$ (up to units).
  So in any cases $\alpha = p$ (up to units). (Note that $\overline{p} = p$.)
  \item[(c)]
  \emph{$p \mid N(\alpha)$ for some prime $p \equiv 1 \pmod{4}$.}
  For such $p$, there is an irreducible $\pi \in \mathbb{Z}[i]$
  satisfying $p = \pi \overline{\pi}$ (Exercise 1.8).
  Now we write $\pi \overline{\pi} \mid \alpha \overline{\alpha}$ in $\mathbb{Z}[i]$.
  Notice that $\pi$, $\overline{\pi}$, $\alpha$ and $\overline{\alpha}$ are all irreducible.
  By the unique factorization theorem in $\mathbb{Z}[i]$,
  $\alpha = \pi$ or $\alpha = \overline{\pi}$.
  In any cases, $\alpha=a+bi$ for integer prime $p \equiv 1 \pmod{4}$
  with $p = a^2+b^2$.
\end{enumerate}
$\Box$ \\\\



%%%%%%%%%%%%%%%%%%%%%%%%%%%%%%%%%%%%%%%%%%%%%%%%%%%%%%%%%%%%%%%%%%%%%%%%%%%%%%%%



\emph{Exercise 1.16-1.28: Let $p$ be an odd prime,
$\omega = e^{\frac{2\pi i}{p}}$.} \\\\



\textbf{Exercise 1.16.}
\emph{Show that
$$(1-\omega)(1-\omega^2) \cdots (1-\omega^{p-1}) = p$$
by considering equation (2).} \\

\textbf{Equation (2).}
$t^p - 1 = (t-1)(t-\omega)(t-\omega^2) \cdots (t-\omega^{p-1})$. \\

\emph{Proof.}
Note that
$t^p - 1 = (t-1)(t^{p-1} + t^{p-2} + \cdots + t + 1)$.
Cancel out $t-1$ of Equation (2),
$$t^{p-1} + t^{p-2} + \cdots + t + 1 = (t-\omega)(t-\omega^2) \cdots (t-\omega^{p-1}).$$
Put $t = 1$ to get
$p = (1-\omega)(1-\omega^2) \cdots (1-\omega^{p-1})$.
$\Box$ \\\\



%%%%%%%%%%%%%%%%%%%%%%%%%%%%%%%%%%%%%%%%%%%%%%%%%%%%%%%%%%%%%%%%%%%%%%%%%%%%%%%%



\emph{Exercise 1.30-1.32: $R$ is an integral domain
(commutative ring with $1$ and no zero divisors).} \\\\



\textbf{Exercise 1.30.}
\emph{Show that two ideals in $R$ are isomorphic as $R$-modules
iff they are in the same ideal class.} \\


\end{document}