\documentclass{article}
\usepackage{amsfonts}
\usepackage{amsmath}
\usepackage{amssymb}
\usepackage{hyperref}
\usepackage[none]{hyphenat}
\usepackage{mathrsfs}
\parindent=0pt

\def\upint{\mathchoice%
    {\mkern13mu\overline{\vphantom{\intop}\mkern7mu}\mkern-20mu}%
    {\mkern7mu\overline{\vphantom{\intop}\mkern7mu}\mkern-14mu}%
    {\mkern7mu\overline{\vphantom{\intop}\mkern7mu}\mkern-14mu}%
    {\mkern7mu\overline{\vphantom{\intop}\mkern7mu}\mkern-14mu}%
  \int}
\def\lowint{\mkern3mu\underline{\vphantom{\intop}\mkern7mu}\mkern-10mu\int}

\begin{document}

\textbf{\Large Chapter 2: Number Fields and Number Rings} \\\\



\emph{Author: Meng-Gen Tsai} \\
\emph{Email: plover@gmail.com} \\\\



%%%%%%%%%%%%%%%%%%%%%%%%%%%%%%%%%%%%%%%%%%%%%%%%%%%%%%%%%%%%%%%%%%%%%%%%%%%%%%%%



\textbf{Exercise 2.28.}
\emph{Let $f(x) = x^3+ax+b$, $a$ and $b \in \mathbb{Z}$,
and assume $f$ is irreducible over $\mathbb{Q}$.
Let $\alpha$ be a root of $f$.}
\begin{enumerate}
\item[(a)]
\emph{Show that $f'(\alpha) = -\frac{2a\alpha+3b}{\alpha}$.}
\item[(b)]
\emph{Show that $2a\alpha+3b$ is a root of
$$\left( \frac{x-3b}{2a} \right)^3 + a\left( \frac{x-3b}{2a} \right) + b.$$
Use this to find $N_{\mathbb{Q}}^{\mathbb{Q}[\alpha]} (2a\alpha+3b)$.}
\item[(c)]
\emph{Show that $\text{disc}(\alpha) = -(4a^3+27b^2)$.}
\item[(d)]
\emph{Suppose $\alpha^3=\alpha+1$.
Prove that $\{1,\alpha,\alpha^2\}$ is an integral basis for $\mathbb{A}\cap\mathbb{Q}[\alpha]$.
(See Exercise 2.27(e).)
Do the same if $\alpha^3+\alpha=1$.} \\
\end{enumerate}

\emph{Proof of (a).}
\begin{enumerate}
\item[(1)]
\emph{Show that $\alpha \neq 0$.}
If $\alpha$ were $0$, then $f(\alpha) = f(0) = b$.
So $f(x) = x^3+ax = x(x^2+a)$ is reducible, contrary to the irreducibility of $f$.
\item[(2)]
Since $\alpha$ be a root of $f$,
$f(\alpha) = 0$,
or $\alpha^3 + a\alpha + b = 0$,
or $\alpha^3 = -a\alpha-b$.
\item[(3)]
\begin{align*}
f'(x) = 3x^2 + a
&\Longrightarrow
f'(\alpha) = 3\alpha^2 + a \\
&\Longleftrightarrow
\alpha f'(\alpha) = 3\alpha^3 + a
  &(\alpha \neq 0) \\
&\Longleftrightarrow
\alpha f'(\alpha) = 3(-a\alpha-b) + a\alpha
  &(\alpha^3 = -a\alpha-b) \\
&\Longleftrightarrow
\alpha f'(\alpha) = -2a\alpha-3b.
\end{align*}
So $f'(\alpha) = -\frac{2a\alpha+3b}{\alpha}$.
\end{enumerate}
$\Box$ \\

\emph{Proof of (b).}
\begin{enumerate}
\item[(1)]
Since $\alpha^3 + a\alpha + b = 0$,
$$\left( \frac{(2a\alpha+3b)-3b}{2a} \right)^3
  + a\left( \frac{(2a\alpha+3b)-3b}{2a} \right) + b = 0.$$
That is, $2a\alpha+3b$ is a root of
$\left( \frac{x-3b}{2a} \right)^3 + a\left( \frac{x-3b}{2a} \right) + b.$
\item[(2)]
$N_{\mathbb{Q}}^{\mathbb{Q}[\alpha]}(2a\alpha+3b)$ is the product of three roots of
$\left( \frac{x-3b}{2a} \right)^3 + a\left( \frac{x-3b}{2a} \right) + b$.
Hence,
\begin{align*}
N_{\mathbb{Q}}^{\mathbb{Q}[\alpha]}(2a\alpha+3b)
&= (2a)^3\left[ \left(\frac{-3b}{2a}\right)^3 + a \cdot \frac{-3b}{2a} + b \right] \\
&= 8a^3\left[ \frac{-27b^3}{8a^3} - \frac{b}{2} \right] \\
&= -27b^3-4a^3b.
\end{align*}
\end{enumerate}
$\Box$ \\

\emph{Proof of (c).}
\begin{align*}
\text{disc}(\alpha)
&= (-1)^{\frac{n(n-1)}{2}} N_{\mathbb{Q}}^{\mathbb{Q}[\alpha]}(f'(\alpha))
  &\text{(Theorem 2.8)} \\
&= - N_{\mathbb{Q}}^{\mathbb{Q}[\alpha]}\left( -\frac{2a\alpha+3b}{\alpha} \right)
  &\text{($n=3$ and (a))} \\
&= \frac{N_{\mathbb{Q}}^{\mathbb{Q}[\alpha]}(2a\alpha+3b)}
  {N_{\mathbb{Q}}^{\mathbb{Q}[\alpha]}(\alpha)} \\
&= \frac{-27b^3-4a^3b}{b}
  &\text{((b))} \\
&= -4a^3-27b^2.
\end{align*}
$\Box$ \\

\emph{Proof of (d).}
\begin{enumerate}
\item[(1)]
$\alpha^3 = \alpha + 1$. $\alpha^3 - \alpha - 1 = 0$.
So $\text{disc}(\alpha) = -23$ (by (c)).
Since $\text{disc}(\alpha)$ is squarefree,
$\{1,\alpha,\alpha^2\}$ forms an integral basis for
$\mathbb{A}\cap\mathbb{Q}[\alpha]$ (Exercise 2.27(e)).
\item[(2)]
$\alpha^3 + \alpha = 1$. $\alpha^3 + \alpha - 1 = 0$.
So $\text{disc}(\alpha) = -31$ (by (c)).
The rest is similar to (1).
\end{enumerate}
$\Box$ \\\\


%%%%%%%%%%%%%%%%%%%%%%%%%%%%%%%%%%%%%%%%%%%%%%%%%%%%%%%%%%%%%%%%%%%%%%%%%%%%%%%%
%%%%%%%%%%%%%%%%%%%%%%%%%%%%%%%%%%%%%%%%%%%%%%%%%%%%%%%%%%%%%%%%%%%%%%%%%%%%%%%%



\end{document}