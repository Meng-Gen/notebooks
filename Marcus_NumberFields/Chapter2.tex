\documentclass{article}
\usepackage{amsfonts}
\usepackage{amsmath}
\usepackage{amssymb}
\usepackage{hyperref}
\usepackage[none]{hyphenat}
\usepackage{mathrsfs}
\parindent=0pt

\def\upint{\mathchoice%
    {\mkern13mu\overline{\vphantom{\intop}\mkern7mu}\mkern-20mu}%
    {\mkern7mu\overline{\vphantom{\intop}\mkern7mu}\mkern-14mu}%
    {\mkern7mu\overline{\vphantom{\intop}\mkern7mu}\mkern-14mu}%
    {\mkern7mu\overline{\vphantom{\intop}\mkern7mu}\mkern-14mu}%
  \int}
\def\lowint{\mkern3mu\underline{\vphantom{\intop}\mkern7mu}\mkern-10mu\int}

\begin{document}

\textbf{\Large Chapter 2: Number Fields and Number Rings} \\\\



\emph{Author: Meng-Gen Tsai} \\
\emph{Email: plover@gmail.com} \\\\



%%%%%%%%%%%%%%%%%%%%%%%%%%%%%%%%%%%%%%%%%%%%%%%%%%%%%%%%%%%%%%%%%%%%%%%%%%%%%%%%



\textbf{Exercise 2.1.}
\begin{enumerate}
\item[(a)]
\emph{Show that every number field of degree $2$ over $\mathbb{Q}$
is one of the quadratic fields $\mathbb{Q}[\sqrt{m}]$, $m \in \mathbb{Z}$.}
\item[(b)]
\emph{Show that the fields $\mathbb{Q}[\sqrt{m}]$, $m$ squarefree,
are pairwise distinct.
(Hint: Consider the equation $\sqrt{m} = a+b\sqrt{n}$);
use this to show that they are in fact pairwise non-isomorphic.} \\
\end{enumerate}

\emph{Proof of (a).}
Let $f(x) = ax^2+bx+c$, $a, b, c \in \mathbb{Z}$ ($a \neq 0$)
and assume $f$ is irreducible over $\mathbb{Q}$.
Let $\alpha$ be a root of $f(x)$.
So
$$\alpha = \frac{-b \pm \sqrt{m}}{2a}$$
where $m = b^2-4ac \in \mathbb{Z}$.
Therefore,
$$\mathbb{Q}[\alpha]
= \mathbb{Q}\left[ \frac{-b \pm \sqrt{m}}{2a} \right]
= \mathbb{Q}[\sqrt{m}].$$
$\Box$ \\

\emph{Proof of (b).}
\emph{Show that $\mathbb{Q}[\sqrt{m}]$ and $\mathbb{Q}[\sqrt{n}]$
are not isomorphic as fields if $m$ and $n$ are squarefree and $m \neq n$.}
Reductio ad absurdum.
\begin{enumerate}
\item[(1)]
If $\varphi: \mathbb{Q}[\sqrt{m}] \to \mathbb{Q}[\sqrt{n}]$ were an isomorphism
as fields, then $\varphi$ is an identity map on $\mathbb{Q}$, and
\begin{align*}
&\varphi(\sqrt{m}) = a + b\sqrt{n} \text{ for some } a, b \in \mathbb{Q} \\
\Longrightarrow&
\varphi(\sqrt{m})\varphi(\sqrt{m}) = (a + b\sqrt{n})^2 \\
\Longrightarrow&
\varphi(\sqrt{m} \sqrt{m}) = (a + b\sqrt{n})^2 \\
\Longrightarrow&
\varphi(m) = a^2 + nb^2 + 2ab\sqrt{n} \\
\Longrightarrow&
m = a^2 + nb^2 + 2ab\sqrt{n}.
\end{align*}
If $2ab \neq 0$, then $\sqrt{n} = \frac{m-a^2-nb^2}{2ab} \in \mathbb{Q}$,
contrary to the assumption that $n$ is squarefree.
Hence $2ab = 0$.
\item[(2)]
$a = 0$.
Write $b = \frac{r}{s} \in \mathbb{Q}$ where $r, s \in \mathbb{Z}$ and $(r, s) = 1$.
So $$ms^2 = nr^2.$$
Hence
\begin{align*}
b \neq 0
&\Longrightarrow
s^2 > 0 \text{ and } r^2 > 0 \\
&\Longrightarrow
\text{$m$ and $n$ have the same sign} \\
&\Longrightarrow
\text{($\exists$ prime $p \mid m$, $p \nmid n$) or
($\exists$ prime $q \mid n$, $q \nmid m$) since $m \neq n$}.
\end{align*}
  \begin{enumerate}
  \item[(a)]
  \emph{There is a prime $p \mid m$ but $p \nmid n$.}
  \begin{align*}
  p \mid m
  &\Longrightarrow
  \text{Write $m = pm_1$ for some $m_1 \in \mathbb{Z}$} \\
  &\Longrightarrow
  (pm_1)s^2 = nr^2
    &(ms^2 = nr^2) \\
  &\Longrightarrow
  p \mid nr^2 \\
  &\Longrightarrow
  p \mid r^2
    &(\text{$p \nmid n$ by assumption}) \\
  &\Longrightarrow
  p \mid r
    &(\text{$p$ is a prime}) \\
  &\Longrightarrow
  \text{Write $r = pr_1$ for some $r_1 \in \mathbb{Z}$} \\
  &\Longrightarrow
  (pm_1)s^2 = n(pr_1)^2
    &(ms^2 = nr^2) \\
  &\Longrightarrow
  m_1s^2 = npr_1^2 \\
  &\Longrightarrow
  p \mid m_1s^2 \\
  &\Longrightarrow
  p \mid m_1
    &(\text{$(r,s)=1$ and $p \mid r$}) \\
  &\Longrightarrow
  \text{Write $m_1 = pm_2$ for some $r_2 \in \mathbb{Z}$} \\
  &\Longrightarrow
  m = p^2m_2,
  \end{align*}
  contrary to the assumption that $m$ is squarefree.
  \item[(b)]
  \emph{There is a prime $q \mid n$ but $q \nmid m$.}
  Similar to (a).
  \end{enumerate}
\item[(3)]
$b = 0$.
$m = a^2$.
Write $a = \frac{r}{s} \in \mathbb{Q}$ where $r, s \in \mathbb{Z}$ and $(r, s) = 1$.
Hence $ms^2 = r^2$.
Similar to the argument in (2).
\item[(4)]
By (2)(3), no such isomorphism $\varphi$, that is,
$\mathbb{Q}[\sqrt{m}]$ and $\mathbb{Q}[\sqrt{n}]$
are not isomorphic as fields.
\end{enumerate}
$\Box$ \\



\textbf{Supplement (Isomorphic as vector spaces).}
\emph{Show that $\mathbb{Q}[\sqrt{m}]$ and $\mathbb{Q}[\sqrt{n}]$
are isomorphic as $\mathbb{Q}$-vector spaces.} \\

\emph{Proof.}
$[\mathbb{Q}[\sqrt{m}]:\mathbb{Q}] = [\mathbb{Q}[\sqrt{n}]:\mathbb{Q}] = 2$.
There is a natural map $\varphi: \mathbb{Q}[\sqrt{m}] \to \mathbb{Q}[\sqrt{n}]$
defined by $\varphi(a + b\sqrt{m}) = a + b\sqrt{n}$.
Clearly $\varphi$ is well-defined, linear, injective and surjective.
$\Box$ \\\\



%%%%%%%%%%%%%%%%%%%%%%%%%%%%%%%%%%%%%%%%%%%%%%%%%%%%%%%%%%%%%%%%%%%%%%%%%%%%%%%%



\textbf{Exercise 2.2.}
\emph{Let $I$ be the ideal generated by $2$ and $1 + \sqrt{-3}$ in the ring
$\mathbb{Z}[\sqrt{-3}] = \{ a+b\sqrt{-3} : a,b \in \mathbb{Z} \}$.
Show that $I \neq (2)$ but $I^2 = 2I$.
Conclude that ideals in $\mathbb{Z}[\sqrt{-3}]$ do not factor uniquely
into prime ideals.
Show moreover that $I$ is the unique prime ideal containing $(2)$
and conclude that $(2)$ is not a product of prime ideals.} \\

\emph{Proof.}
\begin{enumerate}
\item[(1)]
\emph{Show that $I \neq (2)$.}
  \begin{enumerate}
  \item[(a)]
  \emph{Show that $I \supseteq (2)$.}
  $2 \in (2,1+\sqrt{-3}) = I$.
  \item[(b)]
  \emph{Show that $I \not\subseteq (2)$.}
  Consider $1+\sqrt{-3} \in I$.
  (Reductio ad absurdum)
  If $1+\sqrt{-3}$ were in $(2)$, then there exists $a+b\sqrt{-3}$ such that
    $$1+\sqrt{-3} = 2(a+b\sqrt{-3}) = 2a+2b\sqrt{-3}.$$
  Thus, $a = \frac{1}{2}$ and $b = \frac{1}{2}$, which is absurd.
  \end{enumerate}
\item[(2)]
\emph{Show that $I^2 = 2I$.}
  \begin{enumerate}
  \item[(a)]
  \emph{Show that $I^2 \supseteq 2I$.}
  Since $2 \in (2,1+\sqrt{-3}) = I$, $2I \subseteq I^2$.
  \item[(b)]
  \emph{Show that $I^2 \subseteq 2I$.}
  All elements of $I^2$ are generated by
  $$2 \cdot 2, 2(1+\sqrt{-3}) \text{ and } (1+\sqrt{-3})^2.$$
  Clearly, $2 \cdot 2, 2(1+\sqrt{-3}) \in 2I$.
  Besides,
  $$(1+\sqrt{-3})^2 = -2 + 2\sqrt{-3} = 2(-(2) + (1+\sqrt{-3})) \in 2I.$$
  Hence $I^2 \subseteq 2I$.
  \end{enumerate}

\end{enumerate}
$\Box$ \\\\



%%%%%%%%%%%%%%%%%%%%%%%%%%%%%%%%%%%%%%%%%%%%%%%%%%%%%%%%%%%%%%%%%%%%%%%%%%%%%%%%



\textbf{Exercise 2.4.}
\emph{Suppose $a_0, \ldots, a_{n-1}$ are algebraic integers and
$\alpha$ is a complex number satisfying
$$\alpha^n + a_{n-1} \alpha^{n-1} + \cdots + a_1 \alpha + a_0 = 0.$$
Show that the ring
$\mathbb{Z}[a_0, \ldots, a_{n-1}, \alpha]$ has a finitely generated additive group.
(Hint: Consider the products $a_0^{m_0} a_1^{m_1} \cdots a_{n-1}^{m_{n-1}} \alpha^m$
and show that only finitely many values of the exponents are needed.)
Conclude that $\alpha$ is an algebraic integer.} \\

\emph{Proof.}
Let $V = \mathbb{Z}[a_0, \ldots, a_{n-1}, \alpha]$.
Let $n_k$ be the degree of the algebraic integer $a_k$ where $0 \leq k \leq n-1$.
\begin{enumerate}
\item[(1)]
\emph{Show that $V$ is finitely generated as an additive subgroup of $\mathbb{C}$.}
\emph{It suffices to show that $V$ is generated by
$$a_0^{m_0} a_1^{m_1} \cdots a_{n-1}^{m_{n-1}} \alpha^m$$
where $0 \leq m_k < n_k$ and $0 \leq m < n$.}
Given any $x \in V$,
$x$ is a finite sum of the product
$a_0^{m_0} a_1^{m_1} \cdots a_{n-1}^{m_{n-1}} \alpha^m$
with $m_k \geq 0$ and $m \geq 0$. \\

If $m \geq n$,
replace $\alpha^m$
by
\begin{align*}
\alpha^m
=& \alpha^{m-n} \alpha^{n} \\
=& \alpha^{m-n} (-a_{n-1} \alpha^{n-1} - \cdots - a_1 \alpha - a_0) \\
=& -a_{n-1} \alpha^{m-1} - \cdots - a_1 \alpha^{m-n+1} - a_0 \alpha^{m-n}.
\end{align*}
Repeat this process to reduce the degree of $\alpha^m$ less than $n$.
Therefore, we can write $x$ as a finite sum of the product
$a_0^{m_0'} a_1^{m_1'} \cdots a_{n-1}^{m_{n-1}'} \alpha^{m'}$
with $m_k' \geq 0$ and $0 \leq m' < n$. \\

Once the degree of $\alpha^m$ is reduced,
continue to reduce the degree of each $a_k^{m_k'}$
without affecting other $a_h$ ($h \neq k$) and $\alpha$.
Now replace $a_k^{m_k'}$
by
$$a_k^{m_k'} = \sum_{i = 0}^{n_k-1} b_{k,i} a_k^{i}$$
where $b_{k,i} \in \mathbb{Z}$.
Therefore, we can write $x$ as a finite sum of the product
$a_0^{m_0''} a_1^{m_1''} \cdots a_{n-1}^{m_{n-1}''} \alpha^{m'}$
with $0 \leq m_k'' < n_k$ and $0 \leq m' < n$.
\item[(4)]
\emph{Show that $\alpha$ is an algebraic integer.}
Since $\alpha \in V$, $\alpha V \subseteq V$.
Thus $\alpha$ is an algebraic integer (Theorem 2.2).
\end{enumerate}
$\Box$ \\\\



%%%%%%%%%%%%%%%%%%%%%%%%%%%%%%%%%%%%%%%%%%%%%%%%%%%%%%%%%%%%%%%%%%%%%%%%%%%%%%%%



\textbf{Exercise 2.5.}
\emph{Show that if $f$ is any polynomials over $\mathbb{Z}/p\mathbb{Z}$ ($p$ a prime)
then $f(x^p) = (f(x))^p$.
(Suggestion: Use induction on the number of terms.)} \\

\emph{Proof.}
\begin{enumerate}
\item[(1)]
\emph{Let $${p \choose k} = \frac{p!}{k!(p-k)!}$$
be a binomial coefficient.
If $1 \leq k \leq p-1$, show that $p$ divides ${p \choose k}$.}
  \begin{enumerate}
  \item[(a)]
    If $1 \leq k \leq p-1$, then $p \nmid k!$ and $p \nmid (p-k)!$
    since $p$ is a prime.
  \item[(b)]
    Write $a = \frac{p!}{k!(p-k)!} \in \mathbb{Z}$.
    Hence,
    \begin{align*}
    a = \frac{p!}{k!(p-k)!}
    &\Longleftrightarrow
    p! = ak!(p-k)! \\
    &\Longrightarrow
    p \mid p! \text{ or } p \mid ak!(p-k)! \\
    &\Longrightarrow
    p \mid a \:\:\:\: \text{by (a).}
    \end{align*}
    Hence $p$ divides ${p \choose k}$ if $1 \leq k \leq p-1$.
  \end{enumerate}
\item[(2)]
Note that $a^p = a \in \mathbb{Z}/p\mathbb{Z}$ for all $a \in \mathbb{Z}/p\mathbb{Z}$.
\item[(3)]
Write
$$f(x) = a_n x^n + a_{n-1} x^{n-1} + \cdots + a_1 x + a_0 \in \mathbb{Z}/p\mathbb{Z}[x].$$
Induction on $n$.
  \begin{enumerate}
  \item[(a)]
    $n = 0$. So $f(x) = a_0$, and thus $f(x)^p = a_0^p = a_0$ by (2).
  \item[(b)]
    $n = 1$. By $f(x) = a_1 x + a_0$,
    \begin{align*}
    f(x)^p
    &= (a_1 x + a_0)^p \\
    &= a_1^p x^p
      + \sum_{k=1}^{p-1} {p \choose k} (a_1 x)^k a_0^{p-k}
      + a_0^p
      &\text{(Binomial theorem)} \\
    &= a_1^p x^p + a_0^p
      &\text{((1))} \\
    &= a_1 x^p + a_0
      &\text{((2))} \\
    &= f(x^p).
    \end{align*}
  \item[(c)]
  If the statement holds for $n-1$, then
    \begin{align*}
    f(x)^p
    &= (a_n x^n + a_{n-1} x^{n-1} + \cdots + a_1 x + a_0)^p \\
    &= [a_n x^n + (a_{n-1} x^{n-1} + \cdots + a_1 x + a_0)]^p \\
    &= (a_n x^n)^p + (a_{n-1} x^{n-1} + \cdots + a_1 x + a_0)^p
      &\text{(Same as (b))} \\
    &= a_n (x^p)^n+ (a_{n-1} x^{n-1} + \cdots + a_1 x + a_0)^p
      &\text{((2))} \\
    &= a_n (x^p)^n + a_{n-1} (x^p)^{n-1} + \cdots + a_1 x^p + a_0
      &\text{(Induction hypothesis)} \\
    &= f(x^p).
    \end{align*}
  The inductive step is established.
  \end{enumerate}
  By induction, $f(x)^p = f(x^p)$ holds for any $n \geq 0$.
\end{enumerate}
$\Box$ \\\\



%%%%%%%%%%%%%%%%%%%%%%%%%%%%%%%%%%%%%%%%%%%%%%%%%%%%%%%%%%%%%%%%%%%%%%%%%%%%%%%%



\textbf{Exercise 2.6.}
\emph{Show that if $f$ and $g$ are polynomials over a field $K$ and
$f^2 \mid g$ in $K[x]$, then $f \mid g'$.
(Hint: Write $g = f^2 h$ and differentiate.)} \\

\emph{Proof (Hint).}
Since $f^2 \mid g$ in $K[x]$, there exists $h \in K[x]$ such $g = f^2 h$.
Differentiate to get $g' = 2f f' h + f^2 h' = f(2f'h + fh')$,
or $f \mid g'$ in $K[x]$.
$\Box$ \\\\



%%%%%%%%%%%%%%%%%%%%%%%%%%%%%%%%%%%%%%%%%%%%%%%%%%%%%%%%%%%%%%%%%%%%%%%%%%%%%%%%



\textbf{Exercise 2.14.}
\emph{Show that $1+\sqrt{2}$ is a unit in $\mathbb{Z}[\sqrt{2}]$.
Use the powers of $1+\sqrt{2}$ to generate infinitely many solutions
to the diophantine equation $a^2 - 2b^2 = \pm 1$.
(It will be shown in Chapter 5 that all units in $\mathbb{Z}[\sqrt{2}]$
are of the form $\pm(1+\sqrt{2})^k$, $k \in \mathbb{Z}$.)} \\

Might assume to find nonnegative solutions to the Pell's equation $a^2 - 2b^2 = \pm 1$. \\

\emph{Proof.}
\begin{enumerate}
\item[(1)]
\emph{Show that $1+\sqrt{2}$ is a unit in $\mathbb{Z}[\sqrt{2}]$.}
There is $-1+\sqrt{2} \in \mathbb{Z}[\sqrt{2}]$
such that $$(1+\sqrt{2})(-1+\sqrt{2}) = 1 \in \mathbb{Z}[\sqrt{2}].$$
Hence $1+\sqrt{2}$ is a unit.
\item[(2)]
\emph{$N(a+b\sqrt{2}) = |a^2 - 2b^2|$ is a norm on $\mathbb{Z}[\sqrt{2}]$.}
To prove this, use the same argument as Exercise 1.1 and note that
$$N(a+b\sqrt{2}) = |(a+b\sqrt{2})(a-b\sqrt{2})|.$$
\item[(3)]
By (1)(2),
all $(1+\sqrt{2})^k$ with $k \geq 0$
are distinct solutions to the diophantine equation $a^2 - 2b^2 = \pm 1$.
Explicitly, let
\begin{align*}
(a_0,b_0) &= (1,0), \\
(a_1,b_1) &= (1,1), \\
(a_2,b_2) &= (3,2), \\
(a_3,b_3) &= (7,5), \\
&\cdots \\
(a_k,b_k) &= (a_{k-1}+2b_{k-1},a_{k-1}+b_{k-1}), \\
&\cdots
\end{align*}
Note that all $(a_k,b_k)$ are distinct and satisfying $a_k^2 - 2b_k^2 = \pm 1$.
Hence we get infinitely many solutions to the Pell's equation $a^2 - 2b^2 = \pm 1$.
\end{enumerate}

\emph{Note.}
Suppose that all units in $\mathbb{Z}[\sqrt{2}]$
are of the form $\pm(1+\sqrt{2})^k$, $k \in \mathbb{Z}$.
Note that $(1+\sqrt{2})^k = (-1+\sqrt{2})^{-k}$.
Thus we can find all nonnegative solutions to the Pell's equation $a^2 - 2b^2 = \pm 1$
are exactly the same as (3).
$\Box$ \\\\



%%%%%%%%%%%%%%%%%%%%%%%%%%%%%%%%%%%%%%%%%%%%%%%%%%%%%%%%%%%%%%%%%%%%%%%%%%%%%%%%



\textbf{Exercise 2.15.}
\begin{enumerate}
\item[(a)]
\emph{Show that $\mathbb{Z}[\sqrt{-5}]$ contains no element whose norm is $2$ or $3$.}
\item[(b)]
\emph{Verify that $2 \cdot 3 = (1+\sqrt{-5})(1-\sqrt{-5})$ is an example
of non-unique factorization in the number ring $\mathbb{Z}[\sqrt{-5}]$.} \\
\end{enumerate}

\emph{Proof of (a).}
Since $N(a+b\sqrt{-5}) = a^2 + 5b^2 \equiv a^2 \equiv 0,1,4 \pmod{5}$,
there is no element whose norm is $2$ or $3$.
$\Box$ \\

\emph{Proof of (b).}
\begin{enumerate}
\item[(1)]
\emph{Show that $2 \cdot 3 = (1+\sqrt{-5})(1-\sqrt{-5})$.}
$$2 \cdot 3 = 6 \text{ and } (1+\sqrt{-5})(1-\sqrt{-5}) = 6.$$
\item[(2)]
\emph{Show that $2$ is irreducible.}
Suppose $2 = \alpha\beta$ where $\alpha, \beta \in \mathbb{Z}[\sqrt{-5}]$.
Take norm to get
\begin{align*}
  N(2) = N(\alpha)N(\beta)
  \Longrightarrow&
  4 = N(\alpha)N(\beta) \\
  \Longrightarrow&
  N(\alpha) = 1 \text{ or } N(\beta) = 1
    &\text{((1))} \\
  \Longrightarrow&
  \text{$\alpha$ is unit or $\beta$ is unit}.
\end{align*}
\item[(3)]
\emph{Show that $3$ is irreducible.}
Similar to (2).
\item[(4)]
\emph{Show that $1 \pm \sqrt{-5}$ is irreducible.}
Since $N(1 \pm \sqrt{-5}) = 2$ is prime, $1+\sqrt{-5}$ is irreducible.
\end{enumerate}
Hence $6$ has a non-unique factorization
in the number ring $\mathbb{Z}[\sqrt{-5}]$.
$\Box$ \\\\



%%%%%%%%%%%%%%%%%%%%%%%%%%%%%%%%%%%%%%%%%%%%%%%%%%%%%%%%%%%%%%%%%%%%%%%%%%%%%%%%



\textbf{Exercise 2.28.}
\emph{Let $f(x) = x^3+ax+b$, $a$ and $b \in \mathbb{Z}$,
and assume $f$ is irreducible over $\mathbb{Q}$.
Let $\alpha$ be a root of $f$.}
\begin{enumerate}
\item[(a)]
\emph{Show that $f'(\alpha) = -\frac{2a\alpha+3b}{\alpha}$.}
\item[(b)]
\emph{Show that $2a\alpha+3b$ is a root of
$$\left( \frac{x-3b}{2a} \right)^3 + a\left( \frac{x-3b}{2a} \right) + b.$$
Use this to find $N_{\mathbb{Q}}^{\mathbb{Q}[\alpha]} (2a\alpha+3b)$.}
\item[(c)]
\emph{Show that $\text{disc}(\alpha) = -(4a^3+27b^2)$.}
\item[(d)]
\emph{Suppose $\alpha^3=\alpha+1$.
Prove that $\{1,\alpha,\alpha^2\}$ is an integral basis for $\mathbb{A}\cap\mathbb{Q}[\alpha]$.
(See Exercise 2.27(e).)
Do the same if $\alpha^3+\alpha=1$.} \\
\end{enumerate}

\emph{Proof of (a).}
\begin{enumerate}
\item[(1)]
\emph{Show that $\alpha \neq 0$.}
If $\alpha$ were $0$, then $f(\alpha) = f(0) = b$.
So $f(x) = x^3+ax = x(x^2+a)$ is reducible, contrary to the irreducibility of $f$.
\item[(2)]
Since $\alpha$ be a root of $f$,
$f(\alpha) = 0$,
or $\alpha^3 + a\alpha + b = 0$,
or $\alpha^3 = -a\alpha-b$.
\item[(3)]
\begin{align*}
f'(x) = 3x^2 + a
&\Longrightarrow
f'(\alpha) = 3\alpha^2 + a \\
&\Longleftrightarrow
\alpha f'(\alpha) = 3\alpha^3 + a\alpha
  &(\alpha \neq 0) \\
&\Longleftrightarrow
\alpha f'(\alpha) = 3(-a\alpha-b) + a\alpha
  &(\alpha^3 = -a\alpha-b) \\
&\Longleftrightarrow
\alpha f'(\alpha) = -2a\alpha-3b.
\end{align*}
So $f'(\alpha) = -\frac{2a\alpha+3b}{\alpha}$.
\end{enumerate}
$\Box$ \\

\emph{Proof of (b).}
\begin{enumerate}
\item[(1)]
Since $\alpha^3 + a\alpha + b = 0$,
$$\left( \frac{(2a\alpha+3b)-3b}{2a} \right)^3
  + a\left( \frac{(2a\alpha+3b)-3b}{2a} \right) + b = 0.$$
That is, $2a\alpha+3b$ is a root of
$\left( \frac{x-3b}{2a} \right)^3 + a\left( \frac{x-3b}{2a} \right) + b$.
\item[(2)]
$N_{\mathbb{Q}}^{\mathbb{Q}[\alpha]}(2a\alpha+3b)$ is the product of three roots of
$\left( \frac{x-3b}{2a} \right)^3 + a\left( \frac{x-3b}{2a} \right) + b$.
Hence,
\begin{align*}
N_{\mathbb{Q}}^{\mathbb{Q}[\alpha]}(2a\alpha+3b)
&= (2a)^3\left[ \left(\frac{-3b}{2a}\right)^3 + a \cdot \frac{-3b}{2a} + b \right] \\
&= 8a^3\left[ \frac{-27b^3}{8a^3} - \frac{b}{2} \right] \\
&= -27b^3-4a^3b.
\end{align*}
\end{enumerate}
$\Box$ \\

\emph{Proof of (c).}
\begin{align*}
\text{disc}(\alpha)
&= (-1)^{\frac{n(n-1)}{2}} N_{\mathbb{Q}}^{\mathbb{Q}[\alpha]}(f'(\alpha))
  &\text{(Theorem 2.8)} \\
&= - N_{\mathbb{Q}}^{\mathbb{Q}[\alpha]}\left( -\frac{2a\alpha+3b}{\alpha} \right)
  &\text{($n=3$ and (a))} \\
&= \frac{N_{\mathbb{Q}}^{\mathbb{Q}[\alpha]}(2a\alpha+3b)}
  {N_{\mathbb{Q}}^{\mathbb{Q}[\alpha]}(\alpha)} \\
&= \frac{-27b^3-4a^3b}{b}
  &\text{((b))} \\
&= -27b^2-4a^3.
\end{align*}
$\Box$ \\

\emph{Proof of (d).}
\begin{enumerate}
\item[(1)]
  \begin{enumerate}
  \item[(a)]
  $\alpha^3 = \alpha + 1$, or $\alpha^3 - \alpha - 1 = 0$.
  \item[(b)]
  $f(x) = x^3 - x - 1$ is irreducible over $\mathbb{Q}$
  since $f(x)$ is irreducible over $\mathbb{Z}/3\mathbb{Z}$.
  \item[(c)]
  $\text{disc}(\alpha) = -23$ (by (c)).
  \item[(d)]
  Since $\text{disc}(\alpha)$ is squarefree,
  the result is established (Exercise 2.27(e)).
  \end{enumerate}
\item[(2)]
  \begin{enumerate}
  \item[(a)]
  $\alpha^3 + \alpha = 1$, or $\alpha^3 + \alpha - 1 = 0$.
  \item[(b)]
  $f(x) = x^3 + x - 1$ is irreducible over $\mathbb{Q}$
  since $f(x)$ is irreducible over $\mathbb{Z}/2\mathbb{Z}$.
  \item[(c)]
  $\text{disc}(\alpha) = -31$ (by (c)).
  \item[(d)]
  Since $\text{disc}(\alpha)$ is squarefree,
  the result is established (Exercise 2.27(e)).
 \end{enumerate}
\end{enumerate}
$\Box$ \\\\



%%%%%%%%%%%%%%%%%%%%%%%%%%%%%%%%%%%%%%%%%%%%%%%%%%%%%%%%%%%%%%%%%%%%%%%%%%%%%%%%



\textbf{Exercise 2.43.}
\emph{Let $f(x) = x^5+ax+b$, $a$ and $b \in \mathbb{Z}$,
and assume $f$ is irreducible over $\mathbb{Q}$.
Let $\alpha$ be a root of $f$.}
\begin{enumerate}
\item[(a)]
\emph{Show that $\text{disc}(\alpha) = 4^4 a^5 + 5^4 b^4$. (Suggestion: See Exercise 2.28.)}
\item[(b)]
\emph{Suppose $\alpha^5=\alpha+1$.
Prove that $\mathbb{A}\cap\mathbb{Q}[\alpha] = \mathbb{Z}[\alpha]$.
($x^5 - x - 1$ is irreducible over $\mathbb{Q}$;
this can be shown by reducing $\pmod{3}$.)}
\item[(c)]
...
\item[(d)]
... \\
\end{enumerate}

\emph{Proof of (a)(Exercise 2.28).}
\begin{enumerate}
\item[(1)]
\emph{Show that $f'(\alpha) = -\frac{4a\alpha+5b}{\alpha}$.}
  \begin{enumerate}
  \item[(a)]
  \emph{Show that $\alpha \neq 0$.}
  If $\alpha$ were $0$, then $f(\alpha) = f(0) = b$.
  So $f(x) = x^5+ax = x(x^4+a)$ is reducible, contrary to the irreducibility of $f$.
  \item[(b)]
  Since $\alpha$ be a root of $f$,
  $f(\alpha) = 0$,
  or $\alpha^5 + a\alpha + b = 0$,
  or $\alpha^5 = -a\alpha-b$.
  \item[(c)]
  \begin{align*}
  f'(x) = 5x^4 + a
  &\Longrightarrow
  f'(\alpha) = 5\alpha^4 + a \\
  &\Longleftrightarrow
  \alpha f'(\alpha) = 5\alpha^5 + a\alpha
    &(\alpha \neq 0) \\
  &\Longleftrightarrow
  \alpha f'(\alpha) = 5(-a\alpha-b) + a\alpha
    &(\alpha^5 = -a\alpha-b) \\
  &\Longleftrightarrow
  \alpha f'(\alpha) = -4a\alpha-5b.
  \end{align*}
  So $f'(\alpha) = -\frac{4a\alpha+5b}{\alpha}$.
  \end{enumerate}
\item[(2)]
\emph{Show that $4a\alpha+5b$ is a root of
$$\left( \frac{x-5b}{4a} \right)^5 + a\left( \frac{x-5b}{4a} \right) + b.$$
Use this to show that
$N_{\mathbb{Q}}^{\mathbb{Q}[\alpha]}(4a\alpha+5b) = -4^4a^5b-5^5b^5$.}
  \begin{enumerate}
  \item[(a)]
  Since $\alpha^5 + a\alpha + b = 0$,
  $$\left( \frac{(4a\alpha+5b)-5b}{4a} \right)^5
    + a\left( \frac{(4a\alpha+5b)-5b}{4a} \right) + b = 0.$$
  That is, $4a\alpha+5b$ is a root of
  $\left( \frac{x-5b}{4a} \right)^5 + a\left( \frac{x-5b}{4a} \right) + b$.
  \item[(b)]
  $N_{\mathbb{Q}}^{\mathbb{Q}[\alpha]}(4a\alpha+5b)$ is the product of $5$ roots of
  $\left( \frac{x-5b}{4a} \right)^5 + a\left( \frac{x-5b}{4a} \right) + b$.
  Hence,
  \begin{align*}
  N_{\mathbb{Q}}^{\mathbb{Q}[\alpha]}(4a\alpha+5b)
  &= (4a)^5\left[ \left(\frac{-5b}{4a}\right)^5 + a \cdot \frac{-5b}{4a} + b \right] \\
  &= 4^5a^5\left[ \frac{-5^5b^5}{4^5a^5} - \frac{b}{4} \right] \\
  &= -5^5b^5-4^4a^5b.
  \end{align*}
  \end{enumerate}
\item[(3)]
\emph{Show that $\text{disc}(\alpha) = 4^4 a^5 + 5^4 b^4$.}
\begin{align*}
\text{disc}(\alpha)
&= (-1)^{\frac{n(n-1)}{2}} N_{\mathbb{Q}}^{\mathbb{Q}[\alpha]}(f'(\alpha))
  &\text{(Theorem 2.8)} \\
&= N_{\mathbb{Q}}^{\mathbb{Q}[\alpha]}\left( -\frac{4a\alpha+5b}{\alpha} \right)
  &\text{($n=5$ and (1))} \\
&= -\frac{N_{\mathbb{Q}}^{\mathbb{Q}[\alpha]}(4a\alpha+5b)}
  {N_{\mathbb{Q}}^{\mathbb{Q}[\alpha]}(\alpha)} \\
&= - \frac{-4^4a^5b-5^5b^5}{b}
  &\text{((2))} \\
&= 4^4 a^5 + 5^4 b^4.
\end{align*}
\end{enumerate}
$\Box$ \\

\emph{Proof of (b)(Exercise 2.28).}
\begin{enumerate}
\item[(1)]
$\alpha^5 = \alpha + 1$, or $\alpha^5 - \alpha - 1 = 0$.
\item[(2)]
$f(x) = x^5 - x - 1$ is irreducible over $\mathbb{Q}$
since $f(x)$ is irreducible over $\mathbb{Z}/3\mathbb{Z}$.
\item[(3)]
$\text{disc}(\alpha) = 881$ (by (a)).
\item[(4)]
Since $\text{disc}(\alpha)$ is squarefree (a prime number),
the result is established (Exercise 2.27(e)).
\end{enumerate}
$\Box$ \\\\



%%%%%%%%%%%%%%%%%%%%%%%%%%%%%%%%%%%%%%%%%%%%%%%%%%%%%%%%%%%%%%%%%%%%%%%%%%%%%%%%



\textbf{Exercise 2.44.}
\emph{Let $f(x) = x^5+ax^4+b$, $a$ and $b \in \mathbb{Z}$,
and assume $f$ is irreducible over $\mathbb{Q}$.
Let $\alpha$ be a root of $f$ and
let $d_1, d_2, d_3$ and $d_4$ be as in Theorem 2.13. }
\begin{enumerate}
\item[(a)]
\emph{Show that $\text{disc}(\alpha) = b^3(4^4 a^5 + 5^5 b)$.}
\item[(b)]
...
\item[(c)]
...
\item[(d)]
... \\
\end{enumerate}

\emph{Proof of (a).}
TODO.
$\Box$ \\\\



%%%%%%%%%%%%%%%%%%%%%%%%%%%%%%%%%%%%%%%%%%%%%%%%%%%%%%%%%%%%%%%%%%%%%%%%%%%%%%%%



\textbf{Exercise 2.45.}
\emph{Obtain a formula for $\text{disc}(\alpha)$ if $\alpha$ is a root of
an irreducible polynomial $x^n + ax + b$ over $\mathbb{Q}$.
Do the same for $x^n + ax^{n-1}+b$.} \\

Assume that $n \geq 2$. \\

\emph{Proof of $x^n + ax + b$ (Exercise 2.28).}
\begin{enumerate}
\item[(1)]
\emph{Show that $f'(\alpha) = -\frac{(n-1)a\alpha+nb}{\alpha}$.}
  \begin{enumerate}
  \item[(a)]
  \emph{Show that $\alpha \neq 0$.}
  If $\alpha$ were $0$, then $f(\alpha) = f(0) = b$.
  So $f(x) = x^n+ax = x(x^{n-1}+a)$ is reducible, contrary to the irreducibility of $f$.
  \item[(b)]
  Since $\alpha$ be a root of $f$,
  $f(\alpha) = 0$,
  or $\alpha^n + a\alpha + b = 0$,
  or $\alpha^n = -a\alpha-b$.
  \item[(c)]
  \begin{align*}
  f'(x) = nx^{n-1} + a
  &\Longrightarrow
  f'(\alpha) = n\alpha^{n-1} + a \\
  &\Longleftrightarrow
  \alpha f'(\alpha) = n\alpha^n + a\alpha
    &(\alpha \neq 0) \\
  &\Longleftrightarrow
  \alpha f'(\alpha) = n(-a\alpha-b) + a\alpha
    &(\alpha^n = -a\alpha-b) \\
  &\Longleftrightarrow
  \alpha f'(\alpha) = -(n-1)a\alpha-nb.
  \end{align*}
  So $f'(\alpha) = -\frac{(n-1)a\alpha+nb}{\alpha}$.
  \end{enumerate}
\item[(2)]
\emph{
Let $\beta = (n-1)a\alpha+nb$.
Show that $\beta$ is a root of
$$\left( \frac{x-nb}{(n-1)a} \right)^n + a\left( \frac{x-nb}{(n-1)a} \right) + b.$$
Use this to show that
$$N_{\mathbb{Q}}^{\mathbb{Q}[\alpha]}(\beta) = -(n-1)^{n-1}a^nb+(-1)^n n^n b^n.$$}
  \begin{enumerate}
  \item[(a)]
  Since $\alpha^n + a\alpha + b = 0$,
  $$\left( \frac{\beta-nb}{(n-1)a} \right)^n
    + a\left( \frac{\beta-nb}{(n-1)a} \right) + b = 0.$$
  That is, $\beta$ is a root of
  $\left( \frac{x-nb}{(n-1)a} \right)^n + a\left( \frac{x-nb}{(n-1)a} \right) + b$.
  \item[(b)]
  $N_{\mathbb{Q}}^{\mathbb{Q}[\alpha]}(\beta)$ is the product of $n$ roots of
  $\left( \frac{x-nb}{(n-1)a} \right)^n + a\left( \frac{x-nb}{(n-1)a} \right) + b$.
  Hence,
  \begin{align*}
  N_{\mathbb{Q}}^{\mathbb{Q}[\alpha]}(\beta)
  &= ((n-1)a)^n\left[ \left(\frac{-nb}{(n-1)a}\right)^n
    + a \cdot \frac{-nb}{(n-1)a} + b \right] \\
  &= (n-1)^n a^n\left[ \frac{(-1)^n n^n b^n}{(n-1)^n a^n} - \frac{b}{n-1} \right] \\
  &= (-1)^n n^n b^n - (n-1)^{n-1} a^n b.
  \end{align*}
  \end{enumerate}
\item[(3)]
\emph{Show that $\text{disc}(\alpha) = (-1)^{\frac{(n-1)(n-2)}{2}} (n-1)^{n-1}a^n
  + (-1)^{\frac{n(n-1)}{2}} n^n b^{n-1}$.}
\begin{align*}
\text{disc}(\alpha)
&= (-1)^{\frac{n(n-1)}{2}} N_{\mathbb{Q}}^{\mathbb{Q}[\alpha]}(f'(\alpha))
  &\text{(Theorem 2.8)} \\
&= (-1)^{\frac{n(n-1)}{2}} N_{\mathbb{Q}}^{\mathbb{Q}[\alpha]}
  \left( -\frac{(n-1)a\alpha+nb}{\alpha} \right)
  &\text{((1))} \\
&= (-1)^{\frac{n(n-1)}{2}}(-1)^n
  \frac{N_{\mathbb{Q}}^{\mathbb{Q}[\alpha]}((n-1)a\alpha+nb)}
  {N_{\mathbb{Q}}^{\mathbb{Q}[\alpha]}(\alpha)} \\
&= (-1)^{\frac{n(n-1)}{2}}(-1)^n \frac{-(n-1)^{n-1}a^nb+(-1)^n n^n b^n}{b}
  &\text{((2))} \\
&= (-1)^{\frac{(n-1)(n-2)}{2}} (n-1)^{n-1}a^n
  + (-1)^{\frac{n(n-1)}{2}} n^n b^{n-1}.
\end{align*}
\end{enumerate}
$\Box$ \\

\emph{Proof of $x^n + ax^{n-1} + b$.}
TODO.
$\Box$ \\\\



%%%%%%%%%%%%%%%%%%%%%%%%%%%%%%%%%%%%%%%%%%%%%%%%%%%%%%%%%%%%%%%%%%%%%%%%%%%%%%%%
%%%%%%%%%%%%%%%%%%%%%%%%%%%%%%%%%%%%%%%%%%%%%%%%%%%%%%%%%%%%%%%%%%%%%%%%%%%%%%%%



\end{document}