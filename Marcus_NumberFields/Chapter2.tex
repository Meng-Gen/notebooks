\documentclass{article}
\usepackage{amsfonts}
\usepackage{amsmath}
\usepackage{amssymb}
\usepackage{hyperref}
\usepackage[none]{hyphenat}
\usepackage{mathrsfs}
\parindent=0pt

\def\upint{\mathchoice%
    {\mkern13mu\overline{\vphantom{\intop}\mkern7mu}\mkern-20mu}%
    {\mkern7mu\overline{\vphantom{\intop}\mkern7mu}\mkern-14mu}%
    {\mkern7mu\overline{\vphantom{\intop}\mkern7mu}\mkern-14mu}%
    {\mkern7mu\overline{\vphantom{\intop}\mkern7mu}\mkern-14mu}%
  \int}
\def\lowint{\mkern3mu\underline{\vphantom{\intop}\mkern7mu}\mkern-10mu\int}

\begin{document}

\textbf{\Large Chapter 2: Number Fields and Number Rings} \\\\



\emph{Author: Meng-Gen Tsai} \\
\emph{Email: plover@gmail.com} \\\\



%%%%%%%%%%%%%%%%%%%%%%%%%%%%%%%%%%%%%%%%%%%%%%%%%%%%%%%%%%%%%%%%%%%%%%%%%%%%%%%%



\textbf{Exercise 2.28.}
\emph{Let $f(x) = x^3+ax+b$, $a$ and $b \in \mathbb{Z}$,
and assume $f$ is irreducible over $\mathbb{Q}$.
Let $\alpha$ be a root of $f$.}
\begin{enumerate}
\item[(a)]
\emph{Show that $f'(\alpha) = -\frac{2a\alpha+3b}{\alpha}$.}
\item[(b)]
\emph{Show that $2a\alpha+3b$ is a root of
$$\left( \frac{x-3b}{2a} \right)^3 + a\left( \frac{x-3b}{2a} \right) + b.$$
Use this to find $N_{\mathbb{Q}}^{\mathbb{Q}[\alpha]} (2a\alpha+3b)$.}
\item[(c)]
\emph{Show that $\text{disc}(\alpha) = -(4a^3+27b^2)$.}
\item[(d)]
\emph{Suppose $\alpha^3=\alpha+1$.
Prove that $\{1,\alpha,\alpha^2\}$ is an integral basis for $\mathbb{A}\cap\mathbb{Q}[\alpha]$.
(See Exercise 2.27(e).)
Do the same if $\alpha^3+\alpha=1$.} \\
\end{enumerate}

\emph{Proof of (a).}
\begin{enumerate}
\item[(1)]
\emph{Show that $\alpha \neq 0$.}
If $\alpha$ were $0$, then $f(\alpha) = f(0) = b$.
So $f(x) = x^3+ax = x(x^2+a)$ is reducible, contrary to the irreducibility of $f$.
\item[(2)]
Since $\alpha$ be a root of $f$,
$f(\alpha) = 0$,
or $\alpha^3 + a\alpha + b = 0$,
or $\alpha^3 = -a\alpha-b$.
\item[(3)]
\begin{align*}
f'(x) = 3x^2 + a
&\Longrightarrow
f'(\alpha) = 3\alpha^2 + a \\
&\Longleftrightarrow
\alpha f'(\alpha) = 3\alpha^3 + a\alpha
  &(\alpha \neq 0) \\
&\Longleftrightarrow
\alpha f'(\alpha) = 3(-a\alpha-b) + a\alpha
  &(\alpha^3 = -a\alpha-b) \\
&\Longleftrightarrow
\alpha f'(\alpha) = -2a\alpha-3b.
\end{align*}
So $f'(\alpha) = -\frac{2a\alpha+3b}{\alpha}$.
\end{enumerate}
$\Box$ \\

\emph{Proof of (b).}
\begin{enumerate}
\item[(1)]
Since $\alpha^3 + a\alpha + b = 0$,
$$\left( \frac{(2a\alpha+3b)-3b}{2a} \right)^3
  + a\left( \frac{(2a\alpha+3b)-3b}{2a} \right) + b = 0.$$
That is, $2a\alpha+3b$ is a root of
$\left( \frac{x-3b}{2a} \right)^3 + a\left( \frac{x-3b}{2a} \right) + b$.
\item[(2)]
$N_{\mathbb{Q}}^{\mathbb{Q}[\alpha]}(2a\alpha+3b)$ is the product of three roots of
$\left( \frac{x-3b}{2a} \right)^3 + a\left( \frac{x-3b}{2a} \right) + b$.
Hence,
\begin{align*}
N_{\mathbb{Q}}^{\mathbb{Q}[\alpha]}(2a\alpha+3b)
&= (2a)^3\left[ \left(\frac{-3b}{2a}\right)^3 + a \cdot \frac{-3b}{2a} + b \right] \\
&= 8a^3\left[ \frac{-27b^3}{8a^3} - \frac{b}{2} \right] \\
&= -27b^3-4a^3b.
\end{align*}
\end{enumerate}
$\Box$ \\

\emph{Proof of (c).}
\begin{align*}
\text{disc}(\alpha)
&= (-1)^{\frac{n(n-1)}{2}} N_{\mathbb{Q}}^{\mathbb{Q}[\alpha]}(f'(\alpha))
  &\text{(Theorem 2.8)} \\
&= - N_{\mathbb{Q}}^{\mathbb{Q}[\alpha]}\left( -\frac{2a\alpha+3b}{\alpha} \right)
  &\text{($n=3$ and (a))} \\
&= \frac{N_{\mathbb{Q}}^{\mathbb{Q}[\alpha]}(2a\alpha+3b)}
  {N_{\mathbb{Q}}^{\mathbb{Q}[\alpha]}(\alpha)} \\
&= \frac{-27b^3-4a^3b}{b}
  &\text{((b))} \\
&= -27b^2-4a^3.
\end{align*}
$\Box$ \\

\emph{Proof of (d).}
\begin{enumerate}
\item[(1)]
  \begin{enumerate}
  \item[(a)]
  $\alpha^3 = \alpha + 1$, or $\alpha^3 - \alpha - 1 = 0$.
  \item[(b)]
  $f(x) = x^3 - x - 1$ is irreducible over $\mathbb{Q}$
  since $f(x)$ is irreducible over $\mathbb{Z}/3\mathbb{Z}$.
  \item[(c)]
  $\text{disc}(\alpha) = -23$ (by (c)).
  \item[(d)]
  Since $\text{disc}(\alpha)$ is squarefree,
  the result is established (Exercise 2.27(e)).
  \end{enumerate}
\item[(2)]
  \begin{enumerate}
  \item[(a)]
  $\alpha^3 + \alpha = 1$, or $\alpha^3 + \alpha - 1 = 0$.
  \item[(b)]
  $f(x) = x^3 + x - 1$ is irreducible over $\mathbb{Q}$
  since $f(x)$ is irreducible over $\mathbb{Z}/2\mathbb{Z}$.
  \item[(c)]
  $\text{disc}(\alpha) = -31$ (by (c)).
  \item[(d)]
  Since $\text{disc}(\alpha)$ is squarefree,
  the result is established (Exercise 2.27(e)).
 \end{enumerate}
\end{enumerate}
$\Box$ \\\\



%%%%%%%%%%%%%%%%%%%%%%%%%%%%%%%%%%%%%%%%%%%%%%%%%%%%%%%%%%%%%%%%%%%%%%%%%%%%%%%%



\textbf{Exercise 2.43.}
\emph{Let $f(x) = x^5+ax+b$, $a$ and $b \in \mathbb{Z}$,
and assume $f$ is irreducible over $\mathbb{Q}$.
Let $\alpha$ be a root of $f$.}
\begin{enumerate}
\item[(a)]
\emph{Show that $\text{disc}(\alpha) = 4^4 a^5 + 5^4 b^4$. (Suggestion: See Exercise 2.28.)}
\item[(b)]
\emph{Suppose $\alpha^5=\alpha+1$.
Prove that $\mathbb{A}\cap\mathbb{Q}[\alpha] = \mathbb{Z}[\alpha]$.
($x^5 - x - 1$ is irreducible over $\mathbb{Q}$;
this can be shown by reducing $\pmod{3}$.)}
\item[(c)]
...
\item[(d)]
... \\
\end{enumerate}

\emph{Proof of (a)(Exercise 2.28).}
\begin{enumerate}
\item[(1)]
\emph{Show that $f'(\alpha) = -\frac{4a\alpha+5b}{\alpha}$.}
  \begin{enumerate}
  \item[(a)]
  \emph{Show that $\alpha \neq 0$.}
  If $\alpha$ were $0$, then $f(\alpha) = f(0) = b$.
  So $f(x) = x^5+ax = x(x^4+a)$ is reducible, contrary to the irreducibility of $f$.
  \item[(b)]
  Since $\alpha$ be a root of $f$,
  $f(\alpha) = 0$,
  or $\alpha^5 + a\alpha + b = 0$,
  or $\alpha^5 = -a\alpha-b$.
  \item[(c)]
  \begin{align*}
  f'(x) = 5x^4 + a
  &\Longrightarrow
  f'(\alpha) = 5\alpha^4 + a \\
  &\Longleftrightarrow
  \alpha f'(\alpha) = 5\alpha^5 + a\alpha
    &(\alpha \neq 0) \\
  &\Longleftrightarrow
  \alpha f'(\alpha) = 5(-a\alpha-b) + a\alpha
    &(\alpha^5 = -a\alpha-b) \\
  &\Longleftrightarrow
  \alpha f'(\alpha) = -4a\alpha-5b.
  \end{align*}
  So $f'(\alpha) = -\frac{4a\alpha+5b}{\alpha}$.
  \end{enumerate}
\item[(2)]
\emph{Show that $4a\alpha+5b$ is a root of
$$\left( \frac{x-5b}{4a} \right)^5 + a\left( \frac{x-5b}{4a} \right) + b.$$
Use this to show that
$N_{\mathbb{Q}}^{\mathbb{Q}[\alpha]}(4a\alpha+5b) = -4^4a^5b-5^5b^5$.}
  \begin{enumerate}
  \item[(a)]
  Since $\alpha^5 + a\alpha + b = 0$,
  $$\left( \frac{(4a\alpha+5b)-5b}{4a} \right)^5
    + a\left( \frac{(4a\alpha+5b)-5b}{4a} \right) + b = 0.$$
  That is, $4a\alpha+5b$ is a root of
  $\left( \frac{x-5b}{4a} \right)^5 + a\left( \frac{x-5b}{4a} \right) + b$.
  \item[(b)]
  $N_{\mathbb{Q}}^{\mathbb{Q}[\alpha]}(4a\alpha+5b)$ is the product of $5$ roots of
  $\left( \frac{x-5b}{4a} \right)^5 + a\left( \frac{x-5b}{4a} \right) + b$.
  Hence,
  \begin{align*}
  N_{\mathbb{Q}}^{\mathbb{Q}[\alpha]}(4a\alpha+5b)
  &= (4a)^5\left[ \left(\frac{-5b}{4a}\right)^5 + a \cdot \frac{-5b}{4a} + b \right] \\
  &= 4^5a^5\left[ \frac{-5^5b^5}{4^5a^5} - \frac{b}{4} \right] \\
  &= -5^5b^5-4^4a^5b.
  \end{align*}
  \end{enumerate}
\item[(3)]
\emph{Show that $\text{disc}(\alpha) = 4^4 a^5 + 5^4 b^4$.}
\begin{align*}
\text{disc}(\alpha)
&= (-1)^{\frac{n(n-1)}{2}} N_{\mathbb{Q}}^{\mathbb{Q}[\alpha]}(f'(\alpha))
  &\text{(Theorem 2.8)} \\
&= N_{\mathbb{Q}}^{\mathbb{Q}[\alpha]}\left( -\frac{4a\alpha+5b}{\alpha} \right)
  &\text{($n=5$ and (1))} \\
&= -\frac{N_{\mathbb{Q}}^{\mathbb{Q}[\alpha]}(4a\alpha+5b)}
  {N_{\mathbb{Q}}^{\mathbb{Q}[\alpha]}(\alpha)} \\
&= - \frac{-4^4a^5b-5^5b^5}{b}
  &\text{((2))} \\
&= 4^4 a^5 + 5^4 b^4.
\end{align*}
\end{enumerate}
$\Box$ \\

\emph{Proof of (b)(Exercise 2.28).}
\begin{enumerate}
\item[(1)]
$\alpha^5 = \alpha + 1$, or $\alpha^5 - \alpha - 1 = 0$.
\item[(2)]
$f(x) = x^5 - x - 1$ is irreducible over $\mathbb{Q}$
since $f(x)$ is irreducible over $\mathbb{Z}/3\mathbb{Z}$.
\item[(3)]
$\text{disc}(\alpha) = 881$ (by (a)).
\item[(4)]
Since $\text{disc}(\alpha)$ is squarefree (a prime number),
the result is established (Exercise 2.27(e)).
\end{enumerate}
$\Box$ \\\\



%%%%%%%%%%%%%%%%%%%%%%%%%%%%%%%%%%%%%%%%%%%%%%%%%%%%%%%%%%%%%%%%%%%%%%%%%%%%%%%%



\textbf{Exercise 2.44.}
\emph{Let $f(x) = x^5+ax^4+b$, $a$ and $b \in \mathbb{Z}$,
and assume $f$ is irreducible over $\mathbb{Q}$.
Let $\alpha$ be a root of $f$ and
let $d_1, d_2, d_3$ and $d_4$ be as in Theorem 2.13. }
\begin{enumerate}
\item[(a)]
\emph{Show that $\text{disc}(\alpha) = b^3(4^4 a^5 + 5^5 b)$.}
\item[(b)]
...
\item[(c)]
...
\item[(d)]
... \\
\end{enumerate}

\emph{Proof of (a)(Exercise 2.28).}
...
$\Box$ \\\\



%%%%%%%%%%%%%%%%%%%%%%%%%%%%%%%%%%%%%%%%%%%%%%%%%%%%%%%%%%%%%%%%%%%%%%%%%%%%%%%%



\textbf{Exercise 2.45.}
\emph{Obtain a formula for $\text{disc}(\alpha)$ if $\alpha$ is a root of
an irreducible polynomial $x^n + ax + b$ over $\mathbb{Q}$.
Do the same for $x^n + ax^{n-1}+b$.} \\

Assume that $n \geq 2$. \\

\emph{Proof of $x^n + ax + b$ (Exercise 2.28).}
\begin{enumerate}
\item[(1)]
\emph{Show that $f'(\alpha) = -\frac{(n-1)a\alpha+nb}{\alpha}$.}
  \begin{enumerate}
  \item[(a)]
  \emph{Show that $\alpha \neq 0$.}
  If $\alpha$ were $0$, then $f(\alpha) = f(0) = b$.
  So $f(x) = x^n+ax = x(x^{n-1}+a)$ is reducible, contrary to the irreducibility of $f$.
  \item[(b)]
  Since $\alpha$ be a root of $f$,
  $f(\alpha) = 0$,
  or $\alpha^n + a\alpha + b = 0$,
  or $\alpha^n = -a\alpha-b$.
  \item[(c)]
  \begin{align*}
  f'(x) = nx^{n-1} + a
  &\Longrightarrow
  f'(\alpha) = n\alpha^{n-1} + a \\
  &\Longleftrightarrow
  \alpha f'(\alpha) = n\alpha^n + a\alpha
    &(\alpha \neq 0) \\
  &\Longleftrightarrow
  \alpha f'(\alpha) = n(-a\alpha-b) + a\alpha
    &(\alpha^n = -a\alpha-b) \\
  &\Longleftrightarrow
  \alpha f'(\alpha) = -(n-1)a\alpha-nb.
  \end{align*}
  So $f'(\alpha) = -\frac{(n-1)a\alpha+nb}{\alpha}$.
  \end{enumerate}
\item[(2)]
\emph{
Let $\beta = (n-1)a\alpha+nb$.
Show that $\beta$ is a root of
$$\left( \frac{x-nb}{(n-1)a} \right)^n + a\left( \frac{x-nb}{(n-1)a} \right) + b.$$
Use this to show that
$$N_{\mathbb{Q}}^{\mathbb{Q}[\alpha]}(\beta) = -(n-1)^{n-1}a^nb+(-1)^n n^n b^n.$$}
  \begin{enumerate}
  \item[(a)]
  Since $\alpha^n + a\alpha + b = 0$,
  $$\left( \frac{\beta-nb}{(n-1)a} \right)^n
    + a\left( \frac{\beta-nb}{(n-1)a} \right) + b = 0.$$
  That is, $\beta$ is a root of
  $\left( \frac{x-nb}{(n-1)a} \right)^n + a\left( \frac{x-nb}{(n-1)a} \right) + b$.
  \item[(b)]
  $N_{\mathbb{Q}}^{\mathbb{Q}[\alpha]}(\beta)$ is the product of $n$ roots of
  $\left( \frac{x-nb}{(n-1)a} \right)^n + a\left( \frac{x-nb}{(n-1)a} \right) + b$.
  Hence,
  \begin{align*}
  N_{\mathbb{Q}}^{\mathbb{Q}[\alpha]}(\beta)
  &= ((n-1)a)^n\left[ \left(\frac{-nb}{(n-1)a}\right)^n
    + a \cdot \frac{-nb}{(n-1)a} + b \right] \\
  &= (n-1)^n a^n\left[ \frac{(-1)^n n^n b^n}{(n-1)^n a^n} - \frac{b}{n-1} \right] \\
  &= (-1)^n n^n b^n - (n-1)^{n-1} a^n b.
  \end{align*}
  \end{enumerate}
\item[(3)]
\emph{Show that $\text{disc}(\alpha) = (-1)^{\frac{(n-1)(n-2)}{2}} (n-1)^{n-1}a^n
  + (-1)^{\frac{n(n-1)}{2}} n^n b^{n-1}$.}
\begin{align*}
\text{disc}(\alpha)
&= (-1)^{\frac{n(n-1)}{2}} N_{\mathbb{Q}}^{\mathbb{Q}[\alpha]}(f'(\alpha))
  &\text{(Theorem 2.8)} \\
&= (-1)^{\frac{n(n-1)}{2}} N_{\mathbb{Q}}^{\mathbb{Q}[\alpha]}
  \left( -\frac{(n-1)a\alpha+nb}{\alpha} \right)
  &\text{((1))} \\
&= (-1)^{\frac{n(n-1)}{2}}(-1)^n
  \frac{N_{\mathbb{Q}}^{\mathbb{Q}[\alpha]}((n-1)a\alpha+nb)}
  {N_{\mathbb{Q}}^{\mathbb{Q}[\alpha]}(\alpha)} \\
&= (-1)^{\frac{n(n-1)}{2}}(-1)^n \frac{-(n-1)^{n-1}a^nb+(-1)^n n^n b^n}{b}
  &\text{((2))} \\
&= (-1)^{\frac{(n-1)(n-2)}{2}} (n-1)^{n-1}a^n
  + (-1)^{\frac{n(n-1)}{2}} n^n b^{n-1}.
\end{align*}
\end{enumerate}
$\Box$ \\

\emph{Proof of $x^n + ax^{n-1} + b$ (Exercise 2.28).}
...
$\Box$ \\\\



%%%%%%%%%%%%%%%%%%%%%%%%%%%%%%%%%%%%%%%%%%%%%%%%%%%%%%%%%%%%%%%%%%%%%%%%%%%%%%%%
%%%%%%%%%%%%%%%%%%%%%%%%%%%%%%%%%%%%%%%%%%%%%%%%%%%%%%%%%%%%%%%%%%%%%%%%%%%%%%%%



\end{document}