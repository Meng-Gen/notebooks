\documentclass{article}
\usepackage{amsfonts}
\usepackage{amsmath}
\usepackage{amssymb}
\usepackage{centernot}
\usepackage{hyperref}
\usepackage[none]{hyphenat}
\usepackage{mathrsfs}
\usepackage{mathtools}
\usepackage{physics}
\usepackage{tikz-cd}
\parindent=0pt



\title{\textbf{Solutions to the book: \\ \emph{Marcus, Number Fields}}}
\author{Meng-Gen Tsai \\ plover@gmail.com}



\begin{document}
\maketitle
\tableofcontents



%%%%%%%%%%%%%%%%%%%%%%%%%%%%%%%%%%%%%%%%%%%%%%%%%%%%%%%%%%%%%%%%%%%%%%%%%%%%%%%%
%%%%%%%%%%%%%%%%%%%%%%%%%%%%%%%%%%%%%%%%%%%%%%%%%%%%%%%%%%%%%%%%%%%%%%%%%%%%%%%%



% Reference:
% https://www.math.uci.edu/~ndonalds/math180b/6gaussian.pdf
% https://www.maths.nottingham.ac.uk/plp/pmzcw/download/fnt_chap5.pdf
% https://www.math.nagoya-u.ac.jp/~larsh/teaching/F2013_PM/lecture4.pdf
% https://jrsijsling.eu/notes/ant-notes.pdf



%%%%%%%%%%%%%%%%%%%%%%%%%%%%%%%%%%%%%%%%%%%%%%%%%%%%%%%%%%%%%%%%%%%%%%%%%%%%%%%%
%%%%%%%%%%%%%%%%%%%%%%%%%%%%%%%%%%%%%%%%%%%%%%%%%%%%%%%%%%%%%%%%%%%%%%%%%%%%%%%%



\newpage
\section*{Chapter 1: A Special Case of Fermat's Conjecture \\}
\addcontentsline{toc}{section}{Chapter 1: A Special Case of Fermat's Conjecture}



\emph{Exercise 1.1-1.9: Define $N: \mathbb{Z}[i] \rightarrow \mathbb{Z}$ by
$N(a+bi) = a^2 + b^2$.} \\\\



\subsubsection*{Exercise 1.1.}
\addcontentsline{toc}{subsubsection}{Exercise 1.1.}
\emph{Verify that for all $\alpha, \beta \in \mathbb{Z}[i]$,
$N(\alpha\beta) = N(\alpha)N(\beta)$,
either by direct computation or using the fact that
$N(a+bi) = (a+bi)(a-bi)$.
Conclude that if $\alpha \mid \gamma$ in $\mathbb{Z}[i]$,
then $N(\alpha) \mid N(\gamma)$ in $\mathbb{Z}$.} \\

\emph{Proof.}
\begin{enumerate}
\item[(1)]
\emph{Direct computation.}
Write $\alpha = a+bi, \beta=c+di$ where $a, b, c, d \in \mathbb{Z}$.
Thus,
\begin{align*}
N(\alpha\beta)
&= N((a+bi)(c+di)) \\
&= N((ac-bd) + (ad+bc)i) \\
&= (ac-bd)^2 + (ad+bc)^2 \\
&= (a^2 c^2 - 2abcd + b^2 d^2) + (a^2 d^2 + 2abcd + b^2 c^2) \\
&= a^2 c^2 + b^2 d^2 + a^2 d^2 + b^2 c^2, \\
N(\alpha)N(\beta)
&= N(a+bi) N(c+di) \\
&= (a^2 + b^2)(c^2 + d^2) \\
&= a^2 c^2 + b^2 d^2 + a^2 d^2 + b^2 c^2.
\end{align*}
Therefore, $N(\alpha\beta) = N(\alpha)N(\beta)$.
(Note that we also get the identity
$(a^2 + b^2)(c^2 + d^2) = (ac-bd)^2 + (ad+bc)^2$.)
\item[(2)]
\emph{Using the fact that $N(a+bi) = (a+bi)(a-bi)$,}
or $N(\alpha) = \alpha \overline{\alpha}$
for any $\alpha \in \mathbb{Z}[i]$.
Thus,
\begin{align*}
N(\alpha\beta)
&= \alpha\beta\overline{\alpha\beta} \\
&= \alpha\beta\overline{\alpha}\overline{\beta} \\
&= \alpha\overline{\alpha}\beta\overline{\beta} \\
&= N(\alpha)N(\beta).
\end{align*}
\item[(3)]
\emph{Show that if $\alpha \mid \gamma$ in $\mathbb{Z}[i]$,
then $N(\alpha) \mid N(\gamma)$ in $\mathbb{Z}$.}
Write $\gamma = \alpha \beta$ for some $\beta \in \mathbb{Z}[i]$.
So $N(\gamma) = N(\alpha) N(\beta) \in \mathbb{Z}$,
or $N(\alpha) \mid N(\gamma)$ in $\mathbb{Z}$.
\end{enumerate}
$\Box$ \\\\



%%%%%%%%%%%%%%%%%%%%%%%%%%%%%%%%%%%%%%%%%%%%%%%%%%%%%%%%%%%%%%%%%%%%%%%%%%%%%%%%



\subsubsection*{Exercise 1.2.}
\addcontentsline{toc}{subsubsection}{Exercise 1.2.}
\emph{Let $\alpha \in \mathbb{Z}[i]$.
Show that $\alpha$ is a unit iff $N(\alpha) = 1$.
Conclude that the only unit are $\pm 1$ and $\pm i$.} \\

\emph{Proof.}
\begin{enumerate}
\item[(1)]
\emph{$(\Longrightarrow)$}
Since $\alpha$ is a unit, there is $\beta \in \mathbb{Z}[i]$ such that
$\alpha \beta = 1$.
By Exercise 1.1, $N(\alpha \beta) = N(1)$, or $N(\alpha) N(\beta) = 1$.
Since the image of $N$ is nonnegative integers, $N(\alpha) = 1$.
\item[(2)]
\emph{$(\Longleftarrow)$}
By Exercise 1.1, $N(\alpha) = \alpha \overline{\alpha}$,
or $1 = \alpha \overline{\alpha}$ since $N(\alpha) = 1$.
That is, $\overline{\alpha} \in \mathbb{Z}[i]$ is
the inverse of $\alpha \in \mathbb{Z}[i]$.
(Or by (1), we solve the equation $N(\alpha) = a^2 + b^2 = 1$,
and show that all four solutions ($\pm 1$ and $\pm i$) are unit.)
\end{enumerate}
Conclusion: a unit $\alpha = a+bi$ of $\mathbb{Z}[i]$
is satisfying the equation $N(\alpha) = a^2 + b^2 = 1$ by (1)(2).
That is, the only unit of $\mathbb{Z}[i]$ are $\pm 1$ and $\pm i$.
$\Box$ \\\\



%%%%%%%%%%%%%%%%%%%%%%%%%%%%%%%%%%%%%%%%%%%%%%%%%%%%%%%%%%%%%%%%%%%%%%%%%%%%%%%%



\subsubsection*{Exercise 1.3.}
\addcontentsline{toc}{subsubsection}{Exercise 1.3.}
\emph{Let $\alpha \in \mathbb{Z}[i]$.
Show that if $N(\alpha)$ is a prime in $\mathbb{Z}$ then
$\alpha$ is irreducible in $\mathbb{Z}[i]$.
Show that the same conclusion holds
if $N(\alpha) = p^2$, where $p$ is a prime in $\mathbb{Z}$,
$p \equiv 3 \pmod{4}$.} \\

\emph{Proof.}
\begin{enumerate}
\item[(1)]
\emph{Show that if $N(\alpha)$ is a prime in $\mathbb{Z}$ then
$\alpha$ is irreducible in $\mathbb{Z}[i]$.}
Write $\alpha = \beta\gamma$.
Then $N(\alpha) = N(\beta)N(\gamma)$ is a prime in $\mathbb{Z}$.
Since each integer prime is irreducible, $N(\beta) = 1$ or $N(\gamma) = 1$.
So that $\beta$ is unit or $\gamma$ is unit by Exercise 1.2.
Hence, $\alpha$ is irreducible.
\item[(2)]
\emph{Show that $\alpha$ is irreducible in $\mathbb{Z}[i]$
if $N(\alpha) = p^2$, where $p$ is a prime in $\mathbb{Z}$,
$p \equiv 3 \pmod{4}$.}
Assume $\alpha = \beta\gamma$ were not irreducible.
Similar to (1), $N(\alpha) = N(\beta)N(\gamma) = p^2$.
Since $\beta$ and $\gamma$ are proper factors of $\alpha$,
$$N(\beta) = N(\gamma) = p.$$
Since any square $a^2 \equiv 0, 1 \pmod{4}$,
any $N(a+bi) = a^2+b^2 \equiv 0, 1, 2 \pmod{4}$.
Especially, $N(\beta) \equiv 0, 1, 2 \pmod{4}$,
contrary to $N(\beta) = p \equiv 3 \pmod{4}$
by the assumption.
Therefore, $\alpha$ is irreducible in $\mathbb{Z}[i]$.
\end{enumerate}
$\Box$ \\



\subsubsection*{Supplement.}
\addcontentsline{toc}{subsubsection}{Supplement.}
\begin{enumerate}
\item[(1)]
The prime $2$ is reducible in $\mathbb{Z}[i]$ (Exercise 1.4).
\item[(2)]
Every prime $p \equiv 1 \pmod{4}$ is reducible in $\mathbb{Z}[i]$ (Exercise 1.8). \\\\
\end{enumerate}



%%%%%%%%%%%%%%%%%%%%%%%%%%%%%%%%%%%%%%%%%%%%%%%%%%%%%%%%%%%%%%%%%%%%%%%%%%%%%%%%



\subsubsection*{Exercise 1.4.}
\addcontentsline{toc}{subsubsection}{Exercise 1.4.}
\emph{Show that $1-i$ is irreducible in $\mathbb{Z}$
and that $2 = u(1-i)^2$ for some unit $u$.} \\

\emph{Proof.}
\begin{enumerate}
\item[(1)]
\emph{$1-i$ is irreducible.}
Since $N(1-i) = 2$ is a prime in $\mathbb{Z}$,
$1-i$ is irreducible by Problem 1.3.
\item[(2)]
$2 = i(1-i)^2$ where $i$ is unit in $\mathbb{Z}$.
\end{enumerate}
$\Box$ \\\\



%%%%%%%%%%%%%%%%%%%%%%%%%%%%%%%%%%%%%%%%%%%%%%%%%%%%%%%%%%%%%%%%%%%%%%%%%%%%%%%%



\subsubsection*{Exercise 1.5.}
\addcontentsline{toc}{subsubsection}{Exercise 1.5.}
\emph{Notice that
$(2+i)(2-i) = 5 = (1+2i)(1-2i)$.
How is this consistent with unique factorization?} \\

\emph{Proof.}
Since $2+i = i(1-2i)$ and $2-i = (-i)(1+2i)$,
the factorization is unique up to order and multiplication of primes by units.
$\Box$ \\\\



%%%%%%%%%%%%%%%%%%%%%%%%%%%%%%%%%%%%%%%%%%%%%%%%%%%%%%%%%%%%%%%%%%%%%%%%%%%%%%%%



\subsubsection*{Exercise 1.6.}
\addcontentsline{toc}{subsubsection}{Exercise 1.6.}
\emph{Show that every nonzero, non-unit Gaussian integer $\alpha$
is a product of irreducible elements, by induction on $N(\alpha)$.} \\

\emph{Proof.}
Induction on $N(\alpha)$.
\begin{enumerate}
\item[(1)]
\emph{$n = 2$}.
Given $\alpha \in \mathbb{Z}[i]$ with $N(\alpha) = 2$.
Since $N(\alpha) = 2$ is a prime in $\mathbb{Z}$,
$\alpha$ is irreducible (Exercise 1.3).
\item[(2)]
\emph{Suppose the result holds for $n \leq k$.}
Given $\alpha \in \mathbb{Z}[i]$ with $N(\alpha) = k+1$.
There are only two possible cases.
\begin{enumerate}
\item[(a)]
\emph{$\alpha$ is irreducible.}
Nothing to do.
\item[(b)]
\emph{$\alpha$ is reducible.}
Write $\alpha = \beta\gamma$ where neither factor is unit.
Since $N(\alpha) = N(\beta)N(\gamma)$ and neither factor is unit,
$$2 \leq N(\beta), N(\gamma) \leq k.$$
By the induction hypothesis, each factor of $\alpha$ ($\beta$ and $\gamma$)
is a product of irreducible elements.
So that $\alpha$ again is a product of irreducible elements.
\end{enumerate}
In any cases,
$\alpha$ is a product of irreducible elements.
\end{enumerate}
By induction, the result is established.
$\Box$ \\\\



%%%%%%%%%%%%%%%%%%%%%%%%%%%%%%%%%%%%%%%%%%%%%%%%%%%%%%%%%%%%%%%%%%%%%%%%%%%%%%%%



\subsubsection*{Exercise 1.7.}
\addcontentsline{toc}{subsubsection}{Exercise 1.7.}
\emph{Show that $\mathbb{Z}[i]$ is a principal ideal domain (PID); i.e.,
every ideal $I$ is principal.
(As shown in Appendix 1, this implies that $\mathbb{Z}[i]$ is a UFD.)} \\

\emph{Suggestion: Take $\alpha \in I \setminus \{0\}$ such that $N(\alpha)$ is minimized,
and consider the multiplies $\gamma\alpha$, $\gamma \in \mathbb{Z}[i]$;
show that these are the vertices of an infinite family of squares which
fill up the complex plane.
(For example,
one of the squares has vertices $0$, $\alpha$, $i\alpha$, and $(1+i)\alpha$;
all others are translates of this one.)
Obviously $I$ contains all $\gamma\alpha$;
show by a geometric argument that if $I$ contains anything else then
minimality of $N(\alpha)$ would be contradicted.} \\

\emph{Proof (without geometric intuition).}
Define $N$ on $\mathbb{Q}[i]$ by $N(a+bi) = a^2+b^2$ where
$a+bi \in \mathbb{Q}[i]$ as usual.
\begin{enumerate}
\item[(1)]
\emph{Show that $\mathbb{Z}[i]$ is a Euclidean domain.}
Given $\alpha = a+bi \in \mathbb{Z}[i]$ and
$\gamma = c+di \in \mathbb{Z}[i]$ with $\gamma \neq 0$.
It suffices to show there exist $\delta$ and $\rho$ such that the identity
$\alpha = \gamma\delta + \rho$ holds and
either $\rho = 0$ or $N(\rho) < N(\gamma)$.
\begin{enumerate}
\item[(a)]
\emph{Pick $\delta \in \mathbb{Z}[i].$
(Intuition: Pick the `integer part' of $\frac{\alpha}{\gamma}$
as we did in integer numbers.)}
Write $\frac{\alpha}{\gamma} = r+si \in \mathbb{Q}[i]$.
Then we pick $\delta = m+ni \in \mathbb{Z}[i]$ such that
$|r-m| \leq \frac{1}{2}$ and $|s-n| \leq \frac{1}{2}$.
Therefore,
\begin{align*}
N\left( \frac{\alpha}{\gamma} - \delta \right)
&= (r-m)^2 + (s-n)^2 \\
&\leq \frac{1}{4} + \frac{1}{4} \\
&= \frac{1}{2}.
\end{align*}
\item[(b)]
\emph{Pick $\rho \in \mathbb{Z}[i]$.}
Clearly we can pick $\rho = \alpha - \gamma\delta \in \mathbb{Z}[i]$.
Therefore, $\rho = 0$ or
\begin{align*}
N(\rho)
&= N(\alpha - \gamma\delta) \\
&= N\left( \gamma\left( \frac{\alpha}{\gamma} - \delta \right) \right) \\
&= N(\gamma) N\left( \frac{\alpha}{\gamma} - \delta \right) \\
&\leq \frac{1}{2} N(\gamma) \\
&< N(\gamma).
\end{align*}
\end{enumerate}
\item[(2)]
\emph{Show that every Euclidean domain $R$ is a PID.}
Given any ideal $I$ of $R$.
Take $\alpha \in I \setminus \{0\}$ such that $N(\alpha)$ is minimized.
\begin{enumerate}
  \item[(a)]
  $R\alpha \subseteq I$ clearly.
  \item[(b)]
  Conversely, for any $\beta \in I$, there are $\delta, \rho \in R$
  such that $\beta = \alpha \delta + \rho$, where either $\rho = 0$
  or $N(\rho) < N(\alpha)$.
  Since $\rho = \beta - \alpha \delta \in I$,
  we cannot have $N(\rho) < N(\alpha)$ by the minimality of $N(\alpha)$.
  Therefore, $\rho = 0$ and $\beta = \alpha \delta \in R\alpha$,
  or $R\alpha \supseteq I$.
\end{enumerate}
\end{enumerate}
By (1)(2),
$\mathbb{Z}[i]$ is a PID.
$\Box$ \\\\



%%%%%%%%%%%%%%%%%%%%%%%%%%%%%%%%%%%%%%%%%%%%%%%%%%%%%%%%%%%%%%%%%%%%%%%%%%%%%%%%



\subsubsection*{Exercise 1.8.}
\addcontentsline{toc}{subsubsection}{Exercise 1.8.}
\emph{We will use the unique factorization in $\mathbb{Z}[i]$ to prove that
every prime $p \equiv 1 \pmod{4}$ is a sum of two squares.
\begin{enumerate}
  \item[(a)]
  Use the fact that the multiplicative group $(\mathbb{Z}/p\mathbb{Z})^{\times}$
  of integers mod $p$ is cyclic to show that
  if $p \equiv 1 \pmod{4}$ then $n^2 \equiv -1 \pmod{p}$ for some $n \in \mathbb{Z}$.
  \item[(b)]
  Prove that $p$ cannot be irreducible in $\mathbb{Z}[i]$.
  (Hint: $p \mid n^2+1 = (n+i)(n-i)$.)
  \item[(c)]
  Prove that $p$ is a sum of two squares.
  (Hint: (b) shows that $p = (a+bi)(c+di)$ with neither factor a unit.
  Take norms.) \\
\end{enumerate}}

\emph{Proof of (a).}
Since the multiplicative group $(\mathbb{Z}/p\mathbb{Z})^{\times}$
of integers mod $p$ is cyclic,
$(\mathbb{Z}/p\mathbb{Z})^{\times}$ is generated by (a primitive root)
$g \in \mathbb{Z}/p\mathbb{Z}$.
$g^{p-1} = 1$,
or
$$(g^{\frac{p-1}{2}} - 1)(g^{\frac{p-1}{2}} + 1) = 0$$
since $p$ is odd.
Since $\mathbb{Z}/p\mathbb{Z}$ is an integral domain,
$g^{\frac{p-1}{2}} - 1 = 0$ or $g^{\frac{p-1}{2}} + 1 = 0$.
$g$ cannot satisfy $g^{\frac{p-1}{2}} - 1 = 0$ since
$g$ is a generator of $(\mathbb{Z}/p\mathbb{Z})^{\times}$.
So,
$$g^{\frac{p-1}{2}} + 1 = 0.$$
Let $n = g^{\frac{p-1}{4}} \in \mathbb{Z}$ since $p \equiv 1 \pmod{4}$.
So $n^2+1 = 0 \pmod{p}$.
$\Box$ \\

\emph{Proof of (b).}
Since $n^2 + 1 \equiv 0 \pmod{p}$ by (a),
$p \mid n^2+1 = (n+i)(n-i)$.
If $p$ were irreducible in $\mathbb{Z}[i]$,
$p \mid (n+i)$ or $p \mid (n-i)$ by using the unique factorization in $\mathbb{Z}[i]$.
Hence
$$\frac{n+i}{p} = \frac{n}{p} + \frac{1}{p}i \not\in \mathbb{Z}[i],
\frac{n-i}{p} = \frac{n}{p} - \frac{1}{p}i \not\in \mathbb{Z}[i],$$
contrary to the assumption.
Therefore, $p$ is reducible in $\mathbb{Z}[i]$.
$\Box$ \\

\emph{Proof of (c).}
Since $p$ is reducible in $\mathbb{Z}[i]$ by (b),
write $p = (a+bi)(c+di)$ with neither factor a unit.
Take norms,
$$p^2 = N(p) = N(a+bi)N(c+di).$$
Since neither factor of $p$ is unit,
$N(a+bi) = p$,
or $a^2+b^2 = p$,
or $p$ is a sum of two squares.
$\Box$ \\\\



%%%%%%%%%%%%%%%%%%%%%%%%%%%%%%%%%%%%%%%%%%%%%%%%%%%%%%%%%%%%%%%%%%%%%%%%%%%%%%%%



\subsubsection*{Exercise 1.9.}
\addcontentsline{toc}{subsubsection}{Exercise 1.9.}
\emph{Describe all irreducible elements in $\mathbb{Z}[i]$.} \\

\emph{Notice that $\alpha$ is irreducible if and only if $\overline{\alpha}$ is irreducible.}
(Write $\alpha = \beta\gamma$,
then $\overline{\alpha} = \overline{\beta}\overline{\gamma}$.
Besides, $\overline{\overline{\alpha}} = \alpha$.) \\

\emph{Proof.}
\emph{Show that all irreducible elements in $\mathbb{Z}[i]$ (up to units) are
\begin{enumerate}
  \item[(1)]
  $1+i$.
  \item[(2)]
  $\pi = a+bi$ for each integer prime $p \equiv 1 \pmod{4}$
  with $p = a^2+b^2$.
  \item[(3)]
  $p$ for each integer prime $p \equiv 3 \pmod{4}$.
\end{enumerate}}

Let $\alpha$ be any irreducible element in $\mathbb{Z}[i]$.
Consider $N(\alpha) = \alpha \overline{\alpha}$.
$N(\alpha) \neq 1$ since $\alpha$ is not unit.
By the unique factorization theorem in $\mathbb{Z}$,
$N(\alpha) \in \mathbb{Z}$ is a product of primes in $\mathbb{Z}$. \\

There are three possible cases.
\begin{enumerate}
  \item[(a)]
  \emph{$2 \mid N(\alpha)$.}
  Write $(1+i)(1-i) \mid \alpha \overline{\alpha}$ in $\mathbb{Z}[i]$.
  Notice that $1+i$, $1-i$, $\alpha$ and $\overline{\alpha}$ are all irreducible (Exercise 1.4).
  By the unique factorization theorem in $\mathbb{Z}[i]$,
  $\alpha = 1+i$ (up to units).
  \item[(b)]
  \emph{$p \mid N(\alpha)$ for some prime $p \equiv 3 \pmod{4}$.}
  Write $p \mid \alpha \overline{\alpha}$ in $\mathbb{Z}[i]$.
  Notice that $p$, $\alpha$ and $\overline{\alpha}$ are all irreducible (Exercise 1.3).
  By the unique factorization theorem in $\mathbb{Z}[i]$,
  $\alpha = p$ (up to units) or $\overline{\alpha} = p$ (up to units).
  So in any cases $\alpha = p$ (up to units). (Note that $\overline{p} = p$.)
  \item[(c)]
  \emph{$p \mid N(\alpha)$ for some prime $p \equiv 1 \pmod{4}$.}
  For such $p$, there is an irreducible $\pi \in \mathbb{Z}[i]$
  satisfying $p = \pi \overline{\pi}$ (Exercise 1.8).
  Now we write $\pi \overline{\pi} \mid \alpha \overline{\alpha}$ in $\mathbb{Z}[i]$.
  Notice that $\pi$, $\overline{\pi}$, $\alpha$ and $\overline{\alpha}$ are all irreducible.
  By the unique factorization theorem in $\mathbb{Z}[i]$,
  $\alpha = \pi$ or $\alpha = \overline{\pi}$.
  In any cases, $\alpha=a+bi$ for integer prime $p \equiv 1 \pmod{4}$
  with $p = a^2+b^2$.
\end{enumerate}
$\Box$ \\\\



%%%%%%%%%%%%%%%%%%%%%%%%%%%%%%%%%%%%%%%%%%%%%%%%%%%%%%%%%%%%%%%%%%%%%%%%%%%%%%%%



\emph{Exercise 1.10 - 1.14: Let
$\omega
= e^{\frac{2\pi i}{3}}
= -\frac{1}{2}+\frac{\sqrt{3}}{2}i$.
Define $N: \mathbb{Z}[\omega] \to \mathbb{Z}$ by
$N(a+b\omega) = a^2-ab+b^2$. } \\\\



\subsubsection*{Exercise 1.10.}
\addcontentsline{toc}{subsubsection}{Exercise 1.10.}
\emph{Show that if $a+b\omega$ is written in the form $u+vi$
where $u$ and $v$ are real, then $N(a+b\omega)=u^2+v^2$.}

\emph{Proof.}
By $\omega = -\frac{1}{2}+\frac{\sqrt{3}}{2}i$,
write
$$a+b\omega
= \left( a - \frac{1}{2}b \right) + \left( \frac{\sqrt{3}}{2}b \right) i.$$
Here $u = a - \frac{1}{2}b \in \mathbb{R}$ and
$v = \frac{\sqrt{3}}{2}b \in \mathbb{R}$.
Hence $u^2+v^2 = (a - \frac{1}{2}b)^2 + (\frac{\sqrt{3}}{2}b)^2 = a^2 - ab + b^2
= N(a+b\omega)$.
$\Box$ \\\\



%%%%%%%%%%%%%%%%%%%%%%%%%%%%%%%%%%%%%%%%%%%%%%%%%%%%%%%%%%%%%%%%%%%%%%%%%%%%%%%%



\subsubsection*{Exercise 1.11.}
\addcontentsline{toc}{subsubsection}{Exercise 1.11.}
\emph{Show that for all $\alpha, \beta \in \mathbb{Z}[\omega]$,
$N(\alpha\beta) = N(\alpha)N(\beta)$,
either by direct computation or by using Exercise 1.10.
Conclude that if $\alpha \mid \gamma$ in $\mathbb{Z}[\omega]$,
then $N(\alpha) \mid N(\gamma)$ in $\mathbb{Z}$.} \\

\emph{Proof.}
\begin{enumerate}
\item[(1)]
\emph{Direct computation.}
Note that $1+\omega+\omega^2 = 0$ or $\omega^2 = -1-\omega$.
Write $\alpha = a+b\omega, \beta=c+d\omega$ where $a, b, c, d \in \mathbb{Z}$.
Thus,
\begin{align*}
N(\alpha\beta)
&= N((a+b\omega)(c+d\omega)) \\
&= N(ac + (ad+bc)\omega + bd\omega^2) \\
&= N(ac + (ad+bc)\omega + bd(-1-\omega)) \\
&= N((ac-bd) + (ad+bc-bd)\omega) \\
&= (ac-bd)^2 - (ac-bd)(ad+bc-bd) + (ad+bc-bd)^2 \\
&= (a^2 - ab + b^2)(c^2 - cd + d^2), \\
N(\alpha)N(\beta)
&= N(a+b\omega) N(c+d\omega) \\
&= (a^2 - ab + b^2)(c^2 - cd + d^2).
\end{align*}
\item[(2)]
\emph{Exercise 1.10.}
The result is established by Exercise 1.10 and Exercise 1.1.
\item[(3)]
\emph{Using the fact that $N(a+b\omega) = (a+b\omega)\overline{(a+b\omega)}$.}
Similar to the argument of Exercise 1.1.
\item[(4)]
\emph{Show that if $\alpha \mid \gamma$ in $\mathbb{Z}[\omega]$,
then $N(\alpha) \mid N(\gamma)$ in $\mathbb{Z}$.}
Similar to the argument of Exercise 1.1.
\end{enumerate}
$\Box$ \\\\



%%%%%%%%%%%%%%%%%%%%%%%%%%%%%%%%%%%%%%%%%%%%%%%%%%%%%%%%%%%%%%%%%%%%%%%%%%%%%%%%



\subsubsection*{Exercise 1.12.}
\addcontentsline{toc}{subsubsection}{Exercise 1.12.}
\emph{Let $\alpha \in \mathbb{Z}[\omega]$.
Show that $\alpha$ is a unit iff $N(\alpha) = 1$,
and find all units in $\mathbb{Z}[\omega]$.
(There are six of them.)} \\

\emph{Proof.}
\begin{enumerate}
\item[(1)]
$(\Longrightarrow)$
Since $\alpha$ is a unit, there is $\beta \in \mathbb{Z}[\omega]$ such that
$\alpha \beta = 1$.
By Exercise 1.11, $N(\alpha \beta) = N(1)$, or $N(\alpha) N(\beta) = 1$.
Since the image of $N$ is nonnegative integers, $N(\alpha) = 1$.
\item[(2)]
$(\Longleftarrow)$
By Exercise 1.10, $N(\alpha) = \alpha \overline{\alpha}$,
or $1 = \alpha \overline{\alpha}$ since $N(\alpha) = 1$.
That is, $\overline{\alpha} \in \mathbb{Z}[\omega]$ is
the inverse of $\alpha \in \mathbb{Z}[\omega]$.
\item[(3)]
By (1), we solve the equation $N(\alpha) = a^2 - ab + b^2 = 1$,
or $4 = (2a-b)^2 + 3b^2$.
There are 2 possible cases.
  \begin{enumerate}
  \item[(a)]
  $2a-b = \pm 1$, $b = \pm 1$.
  \item[(b)]
  $2a-b = \pm 2$, $b = \pm 0$.
  \end{enumerate}
  Solve these 6 pairs of equations yields the result
  $\pm 1, \pm \omega, \pm \omega^2$.
\end{enumerate}
$\Box$ \\\\



%%%%%%%%%%%%%%%%%%%%%%%%%%%%%%%%%%%%%%%%%%%%%%%%%%%%%%%%%%%%%%%%%%%%%%%%%%%%%%%%



\subsubsection*{Exercise 1.13.}
\addcontentsline{toc}{subsubsection}{Exercise 1.13.}
\emph{Show that $1-\omega$ is irreducible in $\mathbb{Z}[\omega]$,
and that $3 = u(1-\omega)^2$ for some unit $u$. } \\

$3$ is not irreducible in $\mathbb{Z}[\omega]$. \\

\emph{Proof.}
\begin{enumerate}
\item[(1)]
$N(1-\omega) = 3$ is an integer prime.
Similar to the argument in Exercise 1.3,
$1-\omega$ is irreducible in $\mathbb{Z}[\omega]$.
\item[(2)]
Note that $1+\omega+\omega^2=0$.
So
$(1-\omega)^2 = 1-2\omega+\omega^2 = 3(-\omega)$, or
$(-\omega^2)(1-\omega)^2 = 3$.
By Exercise 1.12, $-\omega^2$ is unit.
Hence $3 = u(1-\omega)^2$ for some unit $u = -\omega^2$.
\end{enumerate}
$\Box$ \\\\



%%%%%%%%%%%%%%%%%%%%%%%%%%%%%%%%%%%%%%%%%%%%%%%%%%%%%%%%%%%%%%%%%%%%%%%%%%%%%%%%



\subsubsection*{Exercise 1.14.}
\addcontentsline{toc}{subsubsection}{Exercise 1.14.}
\emph{Modify Exercise 1.7 to show that $\mathbb{Z}[\omega]$ is a PID, hence a UFD.
Here the squares are replaced by parallelograms;
one of them has vertices $0, \alpha, \omega\alpha, (\omega+1)\alpha$,
and all others are translates of this one.
Use Exercise 1.10 for the geometric argument at the end. } \\

Similar to Exercise 1.7. \\

\emph{Proof (without geometric intuition).}
Define $N$ on $\mathbb{Q}[\omega]$ by $N(a+b\omega) = a^2-ab+b^2$ where
$a+b\omega \in \mathbb{Q}[\omega]$ as usual.
\begin{enumerate}
\item[(1)]
\emph{Show that $\mathbb{Z}[\omega]$ is a Euclidean domain.}
Given $\alpha = a+b\omega \in \mathbb{Z}[\omega]$ and
$\gamma = c+d\omega \in \mathbb{Z}[\omega]$ with $\gamma \neq 0$.
It suffices to show there exist $\delta$ and $\rho$ such that the identity
$\alpha = \gamma\delta + \rho$ holds and
either $\rho = 0$ or $N(\rho) < N(\gamma)$.
\begin{enumerate}
\item[(a)]
\emph{Pick $\delta \in \mathbb{Z}[\omega].$
(Intuition: Pick the `integer part' of $\frac{\alpha}{\gamma}$
as we did in integer numbers.)}
Write $\frac{\alpha}{\gamma} = r+s\omega \in \mathbb{Q}[\omega]$.
Then we pick $\delta = m+n\omega \in \mathbb{Z}[\omega]$ such that
$|r-m| \leq \frac{1}{2}$ and $|s-n| \leq \frac{1}{2}$.
Therefore,
\begin{align*}
N\left( \frac{\alpha}{\gamma} - \delta \right)
&\leq |r-m|^2 + |r-m||s-n| + |s-n|^2 \\
&\leq \frac{1}{4} + \frac{1}{4} + \frac{1}{4} \\
&= \frac{3}{4}.
\end{align*}
\item[(b)]
\emph{Pick $\rho \in \mathbb{Z}[\omega]$.}
Clearly we can pick $\rho = \alpha - \gamma\delta \in \mathbb{Z}[\omega]$.
Therefore, $\rho = 0$ or
\begin{align*}
N(\rho)
&= N(\alpha - \gamma\delta) \\
&= N\left( \gamma\left( \frac{\alpha}{\gamma} - \delta \right) \right) \\
&= N(\gamma) N\left( \frac{\alpha}{\gamma} - \delta \right) \\
&\leq \frac{3}{4} N(\gamma) \\
&< N(\gamma).
\end{align*}
\end{enumerate}
\item[(2)]
\emph{Show that every Euclidean domain $R$ is a PID.}
Given any ideal $I$ of $R$.
Take $\alpha \in I \setminus \{0\}$ such that $N(\alpha)$ is minimized.
\begin{enumerate}
  \item[(a)]
  $R\alpha \subseteq I$ clearly.
  \item[(b)]
  Conversely, for any $\beta \in I$, there are $\delta, \rho \in R$
  such that $\beta = \alpha \delta + \rho$, where either $\rho = 0$
  or $N(\rho) < N(\alpha)$.
  Since $\rho = \beta - \alpha \delta \in I$,
  we cannot have $N(\rho) < N(\alpha)$ by the minimality of $N(\alpha)$.
  Therefore, $\rho = 0$ and $\beta = \alpha \delta \in R\alpha$,
  or $R\alpha \supseteq I$.
\end{enumerate}
\end{enumerate}
By (1)(2),
$\mathbb{Z}[\omega]$ is a PID.
$\Box$ \\\\



%%%%%%%%%%%%%%%%%%%%%%%%%%%%%%%%%%%%%%%%%%%%%%%%%%%%%%%%%%%%%%%%%%%%%%%%%%%%%%%%



\subsubsection*{Exercise 1.15.}
\addcontentsline{toc}{subsubsection}{Exercise 1.15.}
\emph{Here is a proof of Fermat's conjecture for $n = 4$:
If $x^4 + y^4 = z^4$ has a solution in positive integers,
then so does $x^4 + y^4 = w^2$.
Let $x,y,w$ be a solution with smallest possible $w$.
Then $x^2, y^2, w$ is a primitive Pythagorean triple.
Assuming (without loss of generality) that $x$ is odd,
we can write
$$x^2 = m^2-n^2, y^2 = 2mn, w = m^2+n^2$$
with $m$ and $n$ are relatively prime positive integers, not both odd. }
\begin{enumerate}
\item[(a)]
\emph{Show that
$$x=r^2-s^2, n=2rs, m=r^2+s^2$$
with $r$ and $s$ are relatively prime positive integers, not both odd. }
\item[(b)]
\emph{Show that $r,s$ and $m$ are pairwise relatively prime.
Using $y^2 = 4rsm$, conclude that $r$, $s$ and $m$ are all squares, say
$a^2$, $b^2$ and $c^2$. }
\item[(c)]
\emph{Show that $a^4+b^4=c^2$, and that this contradicts minimality of $w$. } \\
\end{enumerate}

\emph{Proof of (a).}
Write $x^2+n^2=m^2$ by moving $n^2$ of $x^2 = m^2-n^2$ to the left side.
Notice that $x$ is odd, and thus
$x=r^2-s^2, n=2rs, m=r^2+s^2$
with $r$ and $s$ are relatively prime positive integers, not both odd.
$\Box$ \\

\emph{Proof of (b).}
\begin{enumerate}
\item[(1)]
It suffices to show that $(r,m) = 1$.
By assumption, $(r,s)=1$.
So
$(r,s) = 1 \Rightarrow (r,s^2) = 1 \Rightarrow (r,r^2+s^2) = 1$
and note that $m=r^2+s^2$ to get the result.
\item[(2)]
$y^2 = 2mn = 2m(2rs) = 4rsm$ by (a).
Since $r,s$ and $m$ are pairwise relatively prime,
$r,s$ and $m$ are all squares.
\end{enumerate}
$\Box$ \\

\emph{Proof of (c).}
By (b), $r=a^2$, $s=b^2$, $m=c^2$.
By (a), $m=r^2+s^2$, or $c^2 = (a^2)^2 + (b^2)^2 = a^4 + b^4$.
However, $w = m^2+n^2 > m^2 > m = c^2 > c$, contrary to the minimality of $w$.
$\Box$ \\\\



%%%%%%%%%%%%%%%%%%%%%%%%%%%%%%%%%%%%%%%%%%%%%%%%%%%%%%%%%%%%%%%%%%%%%%%%%%%%%%%%



\emph{Exercise 1.16-1.28: Let $p$ be an odd prime,
$\omega = e^{\frac{2\pi i}{p}}$.} \\\\



\subsubsection*{Exercise 1.16.}
\addcontentsline{toc}{subsubsection}{Exercise 1.16.}
\emph{Show that
$$(1-\omega)(1-\omega^2) \cdots (1-\omega^{p-1}) = p$$
by considering equation
$t^p - 1 = (t-1)(t-\omega)(t-\omega^2) \cdots (t-\omega^{p-1})$.} \\

\emph{Proof.}
Note that
$t^p - 1 = (t-1)(t^{p-1} + t^{p-2} + \cdots + t + 1)$.
Cancel out $t-1$ of Equation (2),
$$t^{p-1} + t^{p-2} + \cdots + t + 1 = (t-\omega)(t-\omega^2) \cdots (t-\omega^{p-1}).$$
Put $t = 1$ to get
$p = (1-\omega)(1-\omega^2) \cdots (1-\omega^{p-1})$.
$\Box$ \\\\



%%%%%%%%%%%%%%%%%%%%%%%%%%%%%%%%%%%%%%%%%%%%%%%%%%%%%%%%%%%%%%%%%%%%%%%%%%%%%%%%



\subsubsection*{Exercise 1.17.}
\addcontentsline{toc}{subsubsection}{Exercise 1.17.}
\emph{Let $x^p + y^p = z^p$.
Suppose that $\mathbb{Z}[\omega]$ is a UFD and $\pi \mid x + y\omega$,
and $\pi$ is a prime in $\mathbb{Z}[\omega]$.
Show that $\pi$ does not divide any of the other factors on the left side of
$$(x+y)(x+y\omega)(x+y\omega^2) \cdots (x+y\omega^{p-1}) = z^p$$
by showing that if it did, then $\pi$ would divide both $z$ and $yp$
(Hint: Use Exercise 1.16);
but $z$ and $yp$ are relatively prime (assuming $p$ divides none of $x, y, z$),
hence $zm + ypn = 1$ for some $m, n \in \mathbb{Z}$.
How is this a contradiction? } \\

\emph{Proof.}
Write $$z = u {\pi}_1^{e_1} \cdots {\pi}_m^{e_m}$$ where
$u$ is unit and $\pi_k$ $(1 \leq k \leq m)$ are distinct primes in $\mathbb{Z}[\omega]$ and
$e_k \in \mathbb{Z}^+$ $(1 \leq k \leq m)$.
Since $\mathbb{Z}[\omega]$ is a UFD by assumption,
the factorization of $z$ is unique up to order and units.
\begin{enumerate}
\item[(1)]
\emph{Show that $\pi \mid z$.}
Since $\pi \mid x + y\omega$, $\pi \mid z^p$.
The factorization of $z^p$ is
$$z^p = u^p {\pi}_1^{pe_1} \cdots {\pi}_m^{pe_m}.$$
$u^p$ is unit, and $\pi | z^p$ implies that $\pi = \pi_k$ for some $k$,
that is, $\pi \mid z$.
\item[(2)]
\emph{Show that $\pi \mid yp$ if $\pi$ were divide any of
the other factors on the left side of
$(x+y)(x+y\omega)(x+y\omega^2) \cdots (x+y\omega^{p-1}) = z^p$.}
Say $\pi \mid x+y\omega^k$ for some $k \neq 1$.
So that $\pi \mid ((x+y\omega) - (x+y\omega^k))$,
or $\pi \mid y(\omega - \omega^k)$.
Since $k \neq 1$, there are two possible cases.
  \begin{enumerate}
  \item[(a)]
  $k > 1$. $\pi \mid y\omega(1 - \omega^{k-1})$.
  By Exercise 1.16, $\pi \mid y\omega p$, or $\pi \mid yp$ since $\omega$ is unit.
  ($\omega^{p-1}$ is the inverse of $\omega$ since $\omega \cdot \omega^{p-1} = 1$.)
  \item[(b)]
  $k = 0$. $\pi \mid y(\omega - 1)$, or $\pi \mid y(1 - \omega)$.
  By Exercise 1.16, $\pi \mid yp$.
  \end{enumerate}
In any case, $\pi \mid yp$.
\item[(3)]
Note that $z$ and $yp$ are integers, and they are relatively prime by the assumption
that $p$ divides none of $x, y, z$.
Therefore, on $\mathbb{Z}$ we have $zm + ypn = 1$ for some $m, n \in \mathbb{Z}$.
\item[(4)]
$zm + ypn = 1$ is also true in $\mathbb{Z}[\omega]$.
Therefore, by (1)(2) we have $\pi \mid (zm + ypn)$ or $\pi \mid 1$,
or $\pi$ is unit, contrary to the primality of $\pi$.
\end{enumerate}
$\Box$ \\\\



%%%%%%%%%%%%%%%%%%%%%%%%%%%%%%%%%%%%%%%%%%%%%%%%%%%%%%%%%%%%%%%%%%%%%%%%%%%%%%%%



\subsubsection*{Exercise 1.18.}
\addcontentsline{toc}{subsubsection}{Exercise 1.18.}
\emph{Use Exercise 1.17 to show that if $\mathbb{Z}[\omega]$ is a UFD then
$x+y\omega = u\alpha^p$, $\alpha \in \mathbb{Z}[\omega]$,
$u$ a unit in $\mathbb{Z}[\omega]$. } \\

\emph{Proof.}
\begin{enumerate}
\item[(1)]
Write $z = u {\pi}_1^{e_1} \cdots {\pi}_m^{e_m}$ as Exercise 1.17.
So
$$z^p = u^p {\pi}_1^{pe_1} \cdots {\pi}_m^{pe_m}.$$
\item[(2)]
Factorize $x + y\omega = v q_1^{f_1} \cdots q_n^{f_n}$,
where $v$ is unit and all $q_h$ $(1 \leq h \leq n)$ are distinct primes in $\mathbb{Z}[\omega]$
and $f_h \in \mathbb{Z}^+$.
Since $\mathbb{Z}[\omega]$ is a UFD,
for every $q_h \mid x + y\omega$, there is some $k(h)$ such that $q_h = \pi_{k(h)}$
and also $q_h^{f_h} = \pi_{k(h)}^{pe_{k(h)}}$ or $f_h = pe_{k(h)}$.
\item[(3)]
Hence,
$$x + y\omega = v \left( \pi_{k(1)}^{e_{k(1)}} \cdots \pi_{k(n)}^{e_{k(n)}} \right)^p,$$
where $\alpha = \pi_{k(1)}^{e_{k(1)}} \cdots \pi_{k(n)}^{e_{k(n)}} \in \mathbb{Z}[\omega]$
and $v$ is unit.
\end{enumerate}
$\Box$ \\\\



%%%%%%%%%%%%%%%%%%%%%%%%%%%%%%%%%%%%%%%%%%%%%%%%%%%%%%%%%%%%%%%%%%%%%%%%%%%%%%%%



\subsubsection*{Exercise 1.19.}
\addcontentsline{toc}{subsubsection}{Exercise 1.19.}
\emph{Dropping the assumption that $\mathbb{Z}[\omega]$ is a UFD but
using the fact that ideals factor uniquely (up to order) into prime ideals,
show that the principal ideal $(x+y\omega)$ has no prime ideal factor in common with
any of the other principal ideals on the left side of the equation
$$(x+y)(x+y\omega)(x+y\omega^2) \cdots (x+y\omega^{p-1}) = (z)^p$$
in which all factors are interpreted as principal ideals.
(Hint: Modify the proof of Exercise 1.17 appropriately,
using the fact that if $A$ is an ideal dividing another ideal $B$,
then $A \supseteq B$.)} \\

\emph{Proof.}
Write $$(z) = {\pi}_1^{e_1} \cdots {\pi}_m^{e_m}$$ where
$\pi_k$ $(1 \leq k \leq m)$ are distinct prime ideals of $\mathbb{Z}[\omega]$ and
$e_k \in \mathbb{Z}^+$ $(1 \leq k \leq m)$.
By assumption that $\mathbb{Z}[\omega]$ is a Dedekind domain,
the factorization of $z$ is unique up to order.
\begin{enumerate}
\item[(1)]
\emph{Show that $\pi \mid (z)$.}
Since $\pi \mid (x + y\omega)$, $\pi \mid (z)^p$.
The factorization of $(z)^p$ is
$$(z)^p = {\pi}_1^{pe_1} \cdots {\pi}_m^{pe_m}.$$
$\pi | (z)^p$ implies that $\pi = \pi_k$ for some $k$,
that is, $\pi \mid (z)$.
\item[(2)]
\emph{Show that $\pi \mid (yp)$ if $\pi$ were divide any of
the other factors on the left side of
$(x+y)(x+y\omega)(x+y\omega^2) \cdots (x+y\omega^{p-1}) = (z)^p$.}
Say $\pi \mid (x+y\omega^k)$ for some $k \neq 1$.
So that $x+y\omega \in \pi$ and $x+y\omega^k \in \pi$,
or $y(\omega - \omega^k) \in \pi$.
Since $k \neq 1$, there are two possible cases.
  \begin{enumerate}
  \item[(a)]
  $k > 1$. $y\omega(1 - \omega^{k-1}) \in \pi$.
  By Exercise 1.16, $y\omega p \in \pi$, or $yp \in \pi$ since $\omega$ is unit.
  ($\omega^{p-1}$ is the inverse of $\omega$ since $\omega \cdot \omega^{p-1} = 1$.)
  \item[(b)]
  $k = 0$. $y(\omega - 1) \in \pi$, or $y(1 - \omega) \in \pi$.
  By Exercise 1.16, $yp \in \pi$.
  \end{enumerate}
In any case, $yp \in \pi$, or $\pi \mid (yp)$.
\item[(3)]
Note that $z$ and $yp$ are integers, and they are relatively prime by the assumption
that $p$ divides none of $x, y, z$.
Therefore, on $\mathbb{Z}$ we have $zm + ypn = 1$ for some $m, n \in \mathbb{Z}$.
\item[(4)]
$zm + ypn = 1$ is also true in $\mathbb{Z}[\omega]$.
Therefore, by (1)(2) we have $z \in \pi$ and $yp \in \pi$.
So $zm + ypn \in \pi$ since $\pi$ is an ideal.
So $1 \in \pi$ or $\pi = (1)$, contrary to the primality of $\pi$.
\end{enumerate}
$\Box$ \\\\



%%%%%%%%%%%%%%%%%%%%%%%%%%%%%%%%%%%%%%%%%%%%%%%%%%%%%%%%%%%%%%%%%%%%%%%%%%%%%%%%



\subsubsection*{Exercise 1.20.}
\addcontentsline{toc}{subsubsection}{Exercise 1.20.}
\emph{Use Exercise 1.19 to show that
$(x+y\omega) = I^p$ for some ideal $I$. } \\

\emph{Proof.}
\begin{enumerate}
\item[(1)]
Write $(z) = {\pi}_1^{e_1} \cdots {\pi}_m^{e_m}$ as Exercise 1.17.
So
$$(z)^p = {\pi}_1^{pe_1} \cdots {\pi}_m^{pe_m}.$$
\item[(2)]
Factorize $(x + y\omega) = q_1^{f_1} \cdots q_n^{f_n}$,
where every $q_h$ $(1 \leq h \leq n)$ are distinct prime ideals of $\mathbb{Z}[\omega]$
and $f_h \in \mathbb{Z}^+$.
By assumption that $\mathbb{Z}[\omega]$ is a Dedekind domain,
for every $q_h \mid (x + y\omega)$, there is some $k(h)$ such that $q_h = \pi_{k(h)}$
and also $q_h^{f_h} = \pi_{k(h)}^{pe_{k(h)}}$ or $f_h = pe_{k(h)}$.
\item[(3)]
Hence,
$$(x + y\omega) = \left( \pi_{k(1)}^{e_{k(1)}} \cdots \pi_{k(n)}^{e_{k(n)}} \right)^p,$$
where $I = \pi_{k(1)}^{e_{k(1)}} \cdots \pi_{k(n)}^{e_{k(n)}}$
is an ideal of $\mathbb{Z}[\omega]$.
\end{enumerate}
$\Box$ \\\\



%%%%%%%%%%%%%%%%%%%%%%%%%%%%%%%%%%%%%%%%%%%%%%%%%%%%%%%%%%%%%%%%%%%%%%%%%%%%%%%%



\subsubsection*{Exercise 1.21.}
\addcontentsline{toc}{subsubsection}{Exercise 1.21.}
\emph{Show that every number of $\mathbb{Q}[\omega]$ is uniquely representable in the form
$$a_0 + a_1\omega + a_2\omega^2 + \cdots + a_{p-2}\omega^{p-2},
a_i \in \mathbb{Q} \:\: \forall \: i$$
by show that $\omega$ is a root of the polynomial
$$f(t) = t^{p-1} + t^{p-2} + \cdots + t + 1$$
and that $f(t)$ is irreducible over $\mathbb{Q}$.
(Hint: It is enough to show that $f(t+1)$ is irreducible,
which can be established by Eisenstein's criterion.
It helps to notice that $f(t+1) = \frac{(t+1)^p-1}{t}$.) } \\

\emph{Proof.}
\begin{enumerate}
\item[(1)]
\emph{Given any number $\alpha \in \mathbb{Q}[\omega]$.
Show that}
$$\alpha = a_0 + a_1\omega + a_2\omega^2 + \cdots + a_{p-2}\omega^{p-2},
a_i \in \mathbb{Q} \:\: \forall \: i.$$
Since $\omega^p = 1$, we can write
$$\alpha = a_0' + a_1'\omega + a_2'\omega^2 + \cdots + a_{p-2}'\omega^{p-2} + a_{p-1}'\omega^{p-1},
a_i \in \mathbb{Q} \:\: \forall \: i.$$
Note that $\omega^{p-1} + \omega^{p-2} + \cdots + \omega + 1 = 0$,
and thus we can replace $\omega^{p-1}$
by $- \omega^{p-2} - \cdots - \omega - 1$.
\item[(2)]
\emph{Show that $\omega$ is a root of the polynomial
$f(t) = t^{p-1} + t^{p-2} + \cdots + t + 1$.}
$f(\omega) = \omega^{p-1} + \omega^{p-2} + \cdots + \omega + 1 = 0$.
\item[(3)]
\emph{Show that $f(t)$ is irreducible over $\mathbb{Q}$.}
It suffices to show that $f(t+1)$ is irreducible over $\mathbb{Q}$.
Write $(t-1)f(t) = t^p - 1$. So
\begin{align*}
tf(t+1)
&= (t+1)^p - 1
  &\text{(Put $t \mapsto t+1$)} \\
&= \left( \sum_{k=0}^{p}{p \choose k}t^k \right) - 1
  &\text{(Binomial theorem)} \\
&= \sum_{k=1}^{p}{p \choose k}t^k, \\
f(t+1)
&= \sum_{k=1}^{p}{p \choose k}t^{k-1} \\
&= t^{p-1} + p t^{p-2} + \cdots + \frac{p(p-1)}{2} t + p.
\end{align*}
By Eisenstein's criterion, $f(t+1)$ is irreducible over $\mathbb{Q}$.
\item[(4)]
To show the uniqueness, it suffices to show that the relation
$$0 = a_0 + a_1\omega + a_2\omega^2 + \cdots + a_{p-2}\omega^{p-2}$$
implies all $a_i = 0$.
Say $g(t) = a_0 + a_1 t + a_2 t^2 + \cdots + a_{p-2} t^{p-2} \in \mathbb{Q}[t]$.
Clearly $g(\omega) = 0$. By the minimality of $f(t)$, $g(t)$ is identical zero,
or all $a_i = 0$.
\end{enumerate}
$\Box$ \\\\



%%%%%%%%%%%%%%%%%%%%%%%%%%%%%%%%%%%%%%%%%%%%%%%%%%%%%%%%%%%%%%%%%%%%%%%%%%%%%%%%



\subsubsection*{Exercise 1.22.}
\addcontentsline{toc}{subsubsection}{Exercise 1.22.}
\emph{Use Exercise 1.21 to show that if
$\alpha \in \mathbb{Z}[\omega]$ and $p \mid \alpha$,
then (writing $\alpha = a_0 + a_1 \omega + \cdots + a_{p-2} \omega^{p-2}$,
$a_i \in \mathbb{Z}$) all $a_i$ are divisible by $p$. } \\

\emph{Proof.}
Since $p \mid \alpha$, there is $\beta \in \mathbb{Z}[\omega]$
such that $\alpha = p\beta$.
Write
\begin{align*}
\alpha &= a_0 + a_1 \omega + \cdots + a_{p-2} \omega^{p-2}, \\
\beta &= b_0 + b_1 \omega + \cdots + b_{p-2} \omega^{p-2},
\end{align*}
where $a_i, b_j \in \mathbb{Z}$.
By $\alpha = p\beta$ and Exercise 1.21, $a_i = pb_i$ for every $1 \leq i \leq p-2$.
So all $a_i$ are divisible by $p$.
$\Box$ \\\\



%%%%%%%%%%%%%%%%%%%%%%%%%%%%%%%%%%%%%%%%%%%%%%%%%%%%%%%%%%%%%%%%%%%%%%%%%%%%%%%%



\emph{Define congruence mod $p$ for $\beta, \gamma \in \mathbb{Z}[\omega]$
as follows:
$$\beta \equiv \gamma \pmod{p}
\text{ iff }
\beta - \gamma = \delta p \text{ for some } \delta \in \mathbb{Z}[\omega].$$
(Equivalently, this is congruence mod the principal ideal $p\mathbb{Z}[\omega]$.} \\\\



\subsubsection*{Exercise 1.23.}
\addcontentsline{toc}{subsubsection}{Exercise 1.23.}
\emph{Show that if $\beta \equiv \gamma \pmod{p}$,
then $\overline{\beta} \equiv \overline{\gamma} \pmod{p}$
where the bar denotes complex conjugation. } \\

\emph{Proof.}
\begin{enumerate}
\item[(1)]
\emph{Show that $\overline{\delta} \in \mathbb{Z}[\omega]$ for any
$\delta \in \mathbb{Z}[\omega]$. }
Write $$\delta = a_0 + a_1 \omega + \cdots + a_{p-1} \omega^{p-1}$$
where $a_0, \ldots, a_{p-1} \in \mathbb{Z}$.
Take the complex conjugation to get
\begin{align*}
\overline{\delta}
&= \overline{a_0} + \overline{a_1} \cdot \overline{\omega} + \cdots
  + \overline{a_{p-1}} \cdot \overline{\omega}^{p-1} \\
&= a_0 + a_1 \overline{\omega} + \cdots + a_{p-1} \overline{\omega}^{p-1}
  &\text{(Every $a_k \in \mathbb{Z}$)} \\
&= a_0 + a_1 \omega^{p-1} + \cdots + a_{p-1} \omega \in \mathbb{Z}[\omega].
  &\text{($\omega^p = 1$)}
\end{align*}
\item[(2)]
\begin{align*}
&\beta \equiv \gamma \pmod{p} \\
\Longleftrightarrow&
\beta - \gamma = \delta p \text{ for some } \delta \in \mathbb{Z}[\omega] \\
\Longleftrightarrow&
\overline{\beta} - \overline{\gamma}
= \overline{\delta} p \text{ for some } \delta \in \mathbb{Z}[\omega]
  &\text{(Complex conjugation)} \\
\Longleftrightarrow&
\overline{\beta} - \overline{\gamma}
= \delta' p \text{ for some } \delta' \in \mathbb{Z}[\omega]
  &\text{((1))} \\
\Longleftrightarrow&
\overline{\beta} \equiv \overline{\gamma} \pmod{p}
\end{align*}
\end{enumerate}
$\Box$ \\\\



%%%%%%%%%%%%%%%%%%%%%%%%%%%%%%%%%%%%%%%%%%%%%%%%%%%%%%%%%%%%%%%%%%%%%%%%%%%%%%%%



\subsubsection*{Exercise 1.24.}
\addcontentsline{toc}{subsubsection}{Exercise 1.24.}
\emph{Show that
$(\beta+\gamma)^p \equiv \beta^p + \gamma^p \pmod{p}$
and generalize this to sums of arbitrarily many terms by induction. } \\

\emph{Proof.}
\begin{enumerate}
\item[(1)]
Binomial theorem gives us
$$
(\beta+\gamma)^p
= \sum_{k=0}^{p}{p \choose k}\beta^k\gamma^{p-k}
= \beta^p+\gamma^p + \sum_{k=1}^{p-1}{p \choose k}\beta^k\gamma^{p-k}.$$
\item[(2)]
Note that every binomial coefficient ${p \choose k}$ is divided by $p$ in $\mathbb{Z}$
for $1 \leq k \leq p-1$.
Also, every term $\beta^k\gamma^{p-k}$ is in $\mathbb{Z}[\omega]$.
So $(\beta+\gamma)^p - \beta^p - \gamma^p = \delta p$
for some $\delta \in \mathbb{Z}[\omega]$.
Hence the result holds.
\item[(3)]
\emph{In general,}
$$\left( \sum_{k=1}^{n} \alpha_k \right)^p
\equiv \sum_{k=1}^{n} \alpha_k^p \pmod{p}.$$
Induction by
$(\alpha_1+\alpha_2)^p \equiv \alpha_1^p + \alpha_2^p \pmod{p}$
and
$\left( \sum_{k=1}^{n+1} \alpha_k \right)^p
\equiv \left( \sum_{k=1}^{n} \alpha_k \right)^p + \alpha_{n+1}^p
\equiv \left( \sum_{k=1}^{n} \alpha_k^p \right) + \alpha_{n+1}^p
\equiv \sum_{k=1}^{n+1} \alpha_k^p \pmod{p}.$
\end{enumerate}
$\Box$ \\\\



%%%%%%%%%%%%%%%%%%%%%%%%%%%%%%%%%%%%%%%%%%%%%%%%%%%%%%%%%%%%%%%%%%%%%%%%%%%%%%%%



\subsubsection*{Exercise 1.25.}
\addcontentsline{toc}{subsubsection}{Exercise 1.25.}
\emph{Show that for all $\alpha \in \mathbb{Z}[\omega]$,
$\alpha^p$ is congruent $\pmod{p}$ to some $a \in \mathbb{Z}$.
(Hint: Write $\alpha$ in terms of $\omega$ and use Exercise 1.24.) } \\

\emph{Proof (Hint).}
Write $$\alpha = a_0 + a_1 \omega + \cdots + a_{p-1} \omega^{p-1}$$
where $a_0, \ldots, a_{p-1} \in \mathbb{Z}$.
By Exercise 1.24,
\begin{align*}
\alpha^p
&\equiv a_0^p + (a_1 \omega)^p + \cdots + (a_{p-1} \omega^{p-1})^p \\
&\equiv a_0^p + a_1^p \omega^p + \cdots + a_{p-1}^p (\omega^{p-1})^p \\
&\equiv a_0^p + a_1^p \omega^p + \cdots + a_{p-1}^p (\omega^p)^{p-1} \\
&\equiv a_0^p + a_1^p + \cdots + a_{p-1}^p.
  &\text{($\omega^p = 1$)}
\end{align*}
Here $a_0^p + a_1^p + \cdots + a_{p-1}^p \in \mathbb{Z}$,
and thus $\alpha^p$ is congruent $\pmod{p}$ to some integer.
$\Box$ \\\\



%%%%%%%%%%%%%%%%%%%%%%%%%%%%%%%%%%%%%%%%%%%%%%%%%%%%%%%%%%%%%%%%%%%%%%%%%%%%%%%%



\emph{Exercise 1.26-1.28: Now assume $p \geq 5$.
We will show that if $x+y\omega = u\alpha^p \pmod{p}$,
$\alpha \in \mathbb{Z}[\omega]$,
$u$ a unit in $\mathbb{Z}[\omega]$,
$x$ and $y$ integers not divisible by $p$,
then $x \equiv y \pmod{p}$.
For this we will need the following result, proved by Kummer,
on the units of $\mathbb{Z}[\omega]$:} \\

\emph{Lemma: If $u$ is a unit in $\mathbb{Z}[\omega]$ and
$\overline{u}$ is its complex conjugate, then $u/\overline{u}$
is a power of $\omega$.
(For the proof, see Exercise 2.12.)} \\\\



%%%%%%%%%%%%%%%%%%%%%%%%%%%%%%%%%%%%%%%%%%%%%%%%%%%%%%%%%%%%%%%%%%%%%%%%%%%%%%%%



\subsubsection*{Exercise 1.26.}
\addcontentsline{toc}{subsubsection}{Exercise 1.26.}
\emph{Show that $x + y\omega \equiv u \alpha^p \pmod{p}$ implies
$$x + y\omega \equiv (x + y\omega^{-1}) \omega^k \pmod{p}$$
for some $k \in \mathbb{Z}$.
(Use the Lemma on units and Exercise 1.23 and 1.25.
Note that $\overline{\omega} = \omega^{-1}$.) } \\

\emph{Proof (Hint).}
\begin{align*}
&x + y\omega \equiv u \alpha^p \pmod{p} \\
\Longrightarrow&
x + y\omega \equiv ua \pmod{p} \text{ for some } a \in \mathbb{Z}
  &\text{(Exercise 1.25)} \\
\Longrightarrow&
\overline{x + y\omega} \equiv \overline{u a} \pmod{p}
  &\text{(Exercise 1.23)} \\
\Longrightarrow&
x + y\overline{\omega} \equiv \overline{u} a \pmod{p} \\
\Longrightarrow&
x + y \omega^{-1} \equiv \overline{u} a \pmod{p}
  &\text{($\overline{\omega} = \omega^{-1}$)} \\
\Longrightarrow&
x + y \omega^{-1} \equiv u \omega^{-k} a \pmod{p} \text{ for some } k \in \mathbb{Z}
  &\text{(Lemma)} \\
\Longrightarrow&
ua \equiv (x + y \omega^{-1})\omega^{k} \pmod{p} \\
\Longrightarrow&
x + y\omega \equiv (x + y \omega^{-1})\omega^{k} \pmod{p}.
\end{align*}
$\Box$ \\\\



%%%%%%%%%%%%%%%%%%%%%%%%%%%%%%%%%%%%%%%%%%%%%%%%%%%%%%%%%%%%%%%%%%%%%%%%%%%%%%%%



\subsubsection*{Exercise 1.27.}
\addcontentsline{toc}{subsubsection}{Exercise 1.27.}
\emph{Use Exercise 1.22 to show that a contradiction results unless $k \equiv 1 \pmod{p}$.
(Recall that $p \nmid xy$, $p \geq 5$, and
$\omega^{p-1} + \omega^{p-2} + \cdots + \omega + 1 = 0$.) } \\

\emph{Proof.}
Exercise 1.26 shows
$$x + y\omega \equiv (x + y\omega^{-1}) \omega^k \pmod{p}.$$
Multiply $\omega$ on the both sides to get
$x\omega + y\omega^2 \equiv y\omega^{k} + x\omega^{k+1} \pmod{p}$, or
$$p \mid (x\omega + y\omega^2 - y\omega^{k} - x\omega^{k+1}).$$
If $k$ were satisfying $k \not\equiv 1 \pmod{p}$, then by Exercise 1.22 and $p \geq 5$
we have $p \mid x$ or $p \mid y$, contrary to the assumption that
$x$ and $y$ are integers not divisible by $p$.
$\Box$ \\\\



%%%%%%%%%%%%%%%%%%%%%%%%%%%%%%%%%%%%%%%%%%%%%%%%%%%%%%%%%%%%%%%%%%%%%%%%%%%%%%%%



\subsubsection*{Exercise 1.28.}
\addcontentsline{toc}{subsubsection}{Exercise 1.28.}
\emph{Finally, show $x \equiv y \pmod{p}$.} \\

\emph{Proof.}
In the argument of Exercise 1.27 we have
$$p \mid ((x-y)\omega + (y-x)\omega^2)$$
by replacing $k = 1$.
By Exercise 1.22 and $p \geq 5$, $x - y$ is divisible by $p$,
or $x \equiv y \pmod{p}$ as integers.
$\Box$ \\\\



%%%%%%%%%%%%%%%%%%%%%%%%%%%%%%%%%%%%%%%%%%%%%%%%%%%%%%%%%%%%%%%%%%%%%%%%%%%%%%%%



\subsubsection*{Exercise 1.29.}
\addcontentsline{toc}{subsubsection}{Exercise 1.29.}
\emph{Let $\omega = \exp(\frac{2\pi i}{23})$.
Verify that the product
$$(1+\omega^2+\omega^4+\omega^5+\omega^6+\omega^{10}+\omega^{11})
(1+\omega+\omega^5+\omega^6+\omega^7+\omega^9+\omega^{11})$$
is divisible by $2$ in $\mathbb{Z}[\omega]$,
although neither factor is.
It can be shown (Exercise 3.17) that $2$ is an irreducible element in $\mathbb{Z}[\omega]$;
it follows that $\mathbb{Z}[\omega]$ cannot be a UFD. } \\

\emph{Proof.}
Note that $\sum_{k=0}^{22} \omega^k = 0$.
So
\begin{align*}
&(1+\omega^2+\omega^4+\omega^5+\omega^6+\omega^{10}+\omega^{11})
(1+\omega+\omega^5+\omega^6+\omega^7+\omega^9+\omega^{11}) \\
=& 2(\omega^5+\omega^6+\omega^7+\omega^9+\omega^{10}+3\omega^{11}
+\omega^{12}+\omega^{13}+\omega^{15}+\omega^{16}+\omega^{17})
\end{align*}
is divisible by $2$ in $\mathbb{Z}[\omega]$,
although neither factor is.
$\Box$ \\\\



%%%%%%%%%%%%%%%%%%%%%%%%%%%%%%%%%%%%%%%%%%%%%%%%%%%%%%%%%%%%%%%%%%%%%%%%%%%%%%%%



\emph{Exercise 1.30-1.32: $R$ is an integral domain
(commutative ring with $1$ and no zero divisors).} \\\\



\subsubsection*{Exercise 1.30.}
\addcontentsline{toc}{subsubsection}{Exercise 1.30.}
\emph{Show that two ideals in $R$ are isomorphic as $R$-modules
iff they are in the same ideal class.} \\

\emph{Proof.}
Given any two ideals $A, B$ in an commutative integral domain $R$.
\begin{enumerate}
\item[(1)]
$(\Longrightarrow)$
Let $\varphi: A \to B$ be an $R$-module isomorphism.
Given any nonzero $\alpha \in A$, we have
\begin{align*}
\varphi(\alpha)A
&= \{ \varphi(\alpha)a : a \in A \} \\
&= \{ \varphi(\alpha a) : a \in A \}
  &\text{($\varphi$ is a homomorphism)} \\
&= \{ \alpha\varphi(a) : a \in A \}
  &\text{($\varphi$ is a homomorphism)} \\
&= \{ \alpha b : b \in B \}
  &\text{($\varphi$ is an isomorphism)} \\
&= \alpha B.
\end{align*}
Notice that $\varphi(\alpha) \neq 0$ since $\alpha \neq 0$ and $\varphi$ is injective.
Therefore, $A \sim B$.
\item[(2)]
$(\Longleftarrow)$
Given $A \sim B$, there are nonzero $\alpha, \beta \in R$ such that $\alpha A = \beta B$.
Define a map $\varphi: A \to B$ by $\varphi(a) = b$ if $\alpha a = \beta b$.
  \begin{enumerate}
  \item[(a)]
  \emph{$\varphi$ is well-defined.}
    \begin{enumerate}
    \item[(i)]
    \emph{Existence of $b$.}
    Since $\alpha a \in \alpha A = \beta B$, there is $b \in B$ such that $\alpha a = \beta b$.
    \item[(ii)]
    \emph{Uniqueness of $b$.}
    If $\alpha a = \beta b_1 = \beta b_2$, $\beta(b_1-b_2) = 0$.
    Since $R$ is an integral domain and $\beta \neq 0$, $b_1-b_2 = 0$ or $b_1 = b_2$.
    \end{enumerate}
  \item[(b)]
  \emph{$\varphi$ is an $R$-module homomorphism.}
    \begin{enumerate}
    \item[(i)]
    \emph{Show that $\varphi(a_1+a_2) = \varphi(a_1) + \varphi(a_2)$.}
    Write $\varphi(a_1) = b_1$ and $\varphi(a_2) = b_2$.
    \begin{align*}
    &\varphi(a_1) = b_1 \text{ and } \varphi(a_2) = b_2 \\
    \Longrightarrow&
    \alpha a_1 = \beta b_1 \text{ and } \alpha a_2 = \beta b_2
      &\text{(Definition of $\varphi$)} \\
    \Longrightarrow&
    \alpha a_1 + \alpha a_2 = \beta b_1 + \beta b_2
      &\text{(Add together)} \\
    \Longrightarrow&
    \alpha (a_1 + a_2) = \beta (b_1 + b_2) \\
    \Longrightarrow&
    \varphi(a_1+a_2) = b_1+b_2 = \varphi(a_1) + \varphi(a_2).
      &\text{(Definition of $\varphi$)}
    \end{align*}
    \item[(ii)]
    \emph{Show that $\varphi(ra) = r\varphi(a)$.}
    Write $\varphi(a) = b$.
    \begin{align*}
    \varphi(a) = b
    \Longrightarrow&
    \alpha a = \beta b
      &\text{(Definition of $\varphi$)} \\
    \Longrightarrow&
    r \alpha a = r \beta b
      &\text{(Multiply $r$)} \\
    \Longrightarrow&
    \alpha(ra) = \beta (rb)
      &\text{($R$ is commutative)} \\
    \Longrightarrow&
    \varphi(ra) = rb = r \varphi(a).
      &\text{(Definition of $\varphi$)}
    \end{align*}
    \end{enumerate}
  \item[(c)]
  \emph{$\varphi$ is injective.}
  Given $\varphi(a) = 0$. Then $\alpha a = \beta b = \beta 0 = 0$.
  Since $R$ is an integral domain and $\alpha \neq 0$, $a = 0$.
  \item[(d)]
  \emph{$\varphi$ is surjective.}
  Given any $b \in B$. $\beta b \in \beta B = \alpha A$.
  There is $a \in A$ such that $\beta b = \alpha a$.
  Such $a$ satisfies $\varphi(a) = b$.
  \end{enumerate}
  Therefore, $\varphi: A \to B$ is an $R$-module isomorphism.
\end{enumerate}
$\Box$ \\\\



%%%%%%%%%%%%%%%%%%%%%%%%%%%%%%%%%%%%%%%%%%%%%%%%%%%%%%%%%%%%%%%%%%%%%%%%%%%%%%%%


\subsubsection*{Exercise 1.31.}
\addcontentsline{toc}{subsubsection}{Exercise 1.31.}
\emph{Show that if $A$ is an ideal in $R$ and if $\alpha A$ is principal
for some nonzero $\alpha \in R$, then $A$ is principal.
Conclude that the principal ideals form an ideal class.} \\

\emph{Proof.}
\begin{enumerate}
\item[(1)]
Write $\alpha A = (b)$ for some $b \in \alpha A$.
That is, there is $a \in A$ such that
$$b = \alpha a.$$
\item[(2)]
\emph{Show that $A = (a)$ is principal.}
$(a) \subseteq A$ holds trivially since $a \in A$ and $A$ is an ideal.
Given any $x \in A$. $\alpha x \in \alpha A = (b)$, and thus
there is $y \in R$ such that $\alpha x = b y$.
Replace $b$ by $b = \alpha a$ to get $\alpha x = \alpha a y$ or
$$\alpha (x - ay) = 0.$$
Since $\alpha \neq 0$ and $R$ is an integral domain,
$x - ay = 0$ or $x = ay \in (a)$ or $A \subseteq (a)$.
Hence $A = (a)$ is principal.
\item[(3)]
\emph{Show that the principal ideals form an ideal class.}
Given any $A = (a) \neq 0$ and $B = (b) \neq 0$,
we have $bA = aB = (ab)$ for $a, b \in R$ or $A \sim B$.
\end{enumerate}
$\Box$ \\\\



%%%%%%%%%%%%%%%%%%%%%%%%%%%%%%%%%%%%%%%%%%%%%%%%%%%%%%%%%%%%%%%%%%%%%%%%%%%%%%%%



\subsubsection*{Exercise 1.32.}
\addcontentsline{toc}{subsubsection}{Exercise 1.32.}
\emph{Show that the ideal classes in $R$ form a group iff for every ideal $A$
there is an ideal $B$ such that $AB$ is principal. } \\

\emph{Note.}
The Picard group of the spectrum of a Dedekind domain is its ideal class group. \\

\emph{Proof.}
Let $[A]$ be the ideal class representing by a nonzero ideal $A$ of $R$.
Let
$$\text{Pic}(R) = \{ [A] : A \text{ is an ideal of } R \}$$
be the set of all ideal classes.
Define the operation $\cdot : \text{Pic}(R) \times \text{Pic}(R) \to \text{Pic}(R)$
by $[A] \cdot [B] \mapsto [AB]$.

\begin{enumerate}
\item[(1)]
\emph{(Closure) Show that the operation $[A] \cdot [B] \mapsto [AB]$ is well-defined. }
Trivial due to the definition of the ideal class.
Note that $[A] \cdot [B] = [B] \cdot [A]$ by the commutativity of $R$.
\item[(2)]
\emph{(Associativity) Show that $([A] \cdot [B]) \cdot [C] = [A] \cdot ([B] \cdot [C])$. }
Trivial due to the definition of the ideal class.
\item[(3)]
\emph{(Identity element) Show that the non-zero principal ideals form the ideal class $[1]$. }
Exercise 1.30 and note that $(1)$ is principal too.
\item[(4)]
\emph{Show that the set $\text{Pic}(R)$ forms an (abelian) group with $[1]$
as the identity element if and only if every $[A]$ has an inverse in $\text{Pic}(R)$.}
By (1)(2)(3), the set $\text{Pic}(R)$ forms an (abelian) group iff
every element has an inverse element. The conclusion is established.
\end{enumerate}
$\Box$ \\\\



%%%%%%%%%%%%%%%%%%%%%%%%%%%%%%%%%%%%%%%%%%%%%%%%%%%%%%%%%%%%%%%%%%%%%%%%%%%%%%%%
%%%%%%%%%%%%%%%%%%%%%%%%%%%%%%%%%%%%%%%%%%%%%%%%%%%%%%%%%%%%%%%%%%%%%%%%%%%%%%%%
%%%%%%%%%%%%%%%%%%%%%%%%%%%%%%%%%%%%%%%%%%%%%%%%%%%%%%%%%%%%%%%%%%%%%%%%%%%%%%%%
%%%%%%%%%%%%%%%%%%%%%%%%%%%%%%%%%%%%%%%%%%%%%%%%%%%%%%%%%%%%%%%%%%%%%%%%%%%%%%%%



\newpage
\section*{Chapter 2: Number Fields and Number Rings \\}
\addcontentsline{toc}{section}{Chapter 2: Number Fields and Number Rings}



\subsubsection*{Exercise 2.1.}
\addcontentsline{toc}{subsubsection}{Exercise 2.1.}
\begin{enumerate}
\item[(a)]
\emph{Show that every number field of degree $2$ over $\mathbb{Q}$
is one of the quadratic fields $\mathbb{Q}[\sqrt{m}]$, $m \in \mathbb{Z}$.}
\item[(b)]
\emph{Show that the fields $\mathbb{Q}[\sqrt{m}]$, $m$ squarefree,
are pairwise distinct.
(Hint: Consider the equation $\sqrt{m} = a+b\sqrt{n}$);
use this to show that they are in fact pairwise non-isomorphic.} \\
\end{enumerate}

\emph{Proof of (a).}
Let $f(x) = ax^2+bx+c$, $a, b, c \in \mathbb{Z}$ ($a \neq 0$)
and assume $f$ is irreducible over $\mathbb{Q}$.
Let $\alpha$ be a root of $f(x)$.
So
$$\alpha = \frac{-b \pm \sqrt{m}}{2a}$$
where $m = b^2-4ac \in \mathbb{Z}$.
Therefore,
$$\mathbb{Q}[\alpha]
= \mathbb{Q}\left[ \frac{-b \pm \sqrt{m}}{2a} \right]
= \mathbb{Q}[\sqrt{m}].$$
$\Box$ \\

\emph{Proof of (b).}
\emph{Show that $\mathbb{Q}[\sqrt{m}]$ and $\mathbb{Q}[\sqrt{n}]$
are not isomorphic as fields if $m$ and $n$ are squarefree and $m \neq n$.}
Reductio ad absurdum.
\begin{enumerate}
\item[(1)]
If $\varphi: \mathbb{Q}[\sqrt{m}] \to \mathbb{Q}[\sqrt{n}]$ were an isomorphism
as fields, then $\varphi$ is an identity map on $\mathbb{Q}$, and
\begin{align*}
&\varphi(\sqrt{m}) = a + b\sqrt{n} \text{ for some } a, b \in \mathbb{Q} \\
\Longrightarrow&
\varphi(\sqrt{m})\varphi(\sqrt{m}) = (a + b\sqrt{n})^2 \\
\Longrightarrow&
\varphi(\sqrt{m} \sqrt{m}) = (a + b\sqrt{n})^2 \\
\Longrightarrow&
\varphi(m) = a^2 + nb^2 + 2ab\sqrt{n} \\
\Longrightarrow&
m = a^2 + nb^2 + 2ab\sqrt{n}.
\end{align*}
If $2ab \neq 0$, then $\sqrt{n} = \frac{m-a^2-nb^2}{2ab} \in \mathbb{Q}$,
contrary to the assumption that $n$ is squarefree.
Hence $2ab = 0$.
\item[(2)]
$a = 0$.
Write $b = \frac{r}{s} \in \mathbb{Q}$ where $r, s \in \mathbb{Z}$ and $(r, s) = 1$.
So $$ms^2 = nr^2.$$
Hence
\begin{align*}
b \neq 0
&\Longrightarrow
s^2 > 0 \text{ and } r^2 > 0 \\
&\Longrightarrow
\text{$m$ and $n$ have the same sign} \\
&\Longrightarrow
\text{($\exists$ prime $p \mid m$, $p \nmid n$) or
($\exists$ prime $q \mid n$, $q \nmid m$) since $m \neq n$}.
\end{align*}
  \begin{enumerate}
  \item[(a)]
  \emph{There is a prime $p \mid m$ but $p \nmid n$.}
  \begin{align*}
  p \mid m
  &\Longrightarrow
  \text{Write $m = pm_1$ for some $m_1 \in \mathbb{Z}$} \\
  &\Longrightarrow
  (pm_1)s^2 = nr^2
    &(ms^2 = nr^2) \\
  &\Longrightarrow
  p \mid nr^2 \\
  &\Longrightarrow
  p \mid r^2
    &(\text{$p \nmid n$ by assumption}) \\
  &\Longrightarrow
  p \mid r
    &(\text{$p$ is a prime}) \\
  &\Longrightarrow
  \text{Write $r = pr_1$ for some $r_1 \in \mathbb{Z}$} \\
  &\Longrightarrow
  (pm_1)s^2 = n(pr_1)^2
    &(ms^2 = nr^2) \\
  &\Longrightarrow
  m_1s^2 = npr_1^2 \\
  &\Longrightarrow
  p \mid m_1s^2 \\
  &\Longrightarrow
  p \mid m_1
    &(\text{$(r,s)=1$ and $p \mid r$}) \\
  &\Longrightarrow
  \text{Write $m_1 = pm_2$ for some $r_2 \in \mathbb{Z}$} \\
  &\Longrightarrow
  m = p^2m_2,
  \end{align*}
  contrary to the assumption that $m$ is squarefree.
  \item[(b)]
  \emph{There is a prime $q \mid n$ but $q \nmid m$.}
  Similar to (a).
  \end{enumerate}
\item[(3)]
$b = 0$.
$m = a^2$.
Write $a = \frac{r}{s} \in \mathbb{Q}$ where $r, s \in \mathbb{Z}$ and $(r, s) = 1$.
Hence $ms^2 = r^2$.
Similar to the argument in (2).
\item[(4)]
By (2)(3), no such isomorphism $\varphi$, that is,
$\mathbb{Q}[\sqrt{m}]$ and $\mathbb{Q}[\sqrt{n}]$
are not isomorphic as fields.
\end{enumerate}
$\Box$ \\



\subsubsection*{Supplement. (Isomorphic as vector spaces)}
\addcontentsline{toc}{subsubsection}{Supplement. (Isomorphic as vector spaces)}
\emph{Show that $\mathbb{Q}[\sqrt{m}]$ and $\mathbb{Q}[\sqrt{n}]$
are isomorphic as $\mathbb{Q}$-vector spaces.} \\



\emph{Proof.}
$[\mathbb{Q}[\sqrt{m}]:\mathbb{Q}] = [\mathbb{Q}[\sqrt{n}]:\mathbb{Q}] = 2$.
There is a natural map $\varphi: \mathbb{Q}[\sqrt{m}] \to \mathbb{Q}[\sqrt{n}]$
defined by $\varphi(a + b\sqrt{m}) = a + b\sqrt{n}$.
Clearly $\varphi$ is well-defined, linear, injective and surjective.
$\Box$ \\\\



%%%%%%%%%%%%%%%%%%%%%%%%%%%%%%%%%%%%%%%%%%%%%%%%%%%%%%%%%%%%%%%%%%%%%%%%%%%%%%%%



\subsubsection*{Exercise 2.2.}
\addcontentsline{toc}{subsubsection}{Exercise 2.2.}
\emph{Let $I$ be the ideal generated by $2$ and $1 + \sqrt{-3}$ in the ring
$\mathbb{Z}[\sqrt{-3}] = \{ a+b\sqrt{-3} : a,b \in \mathbb{Z} \}$.
Show that $I \neq (2)$ but $I^2 = 2I$.
Conclude that ideals in $\mathbb{Z}[\sqrt{-3}]$ do not factor uniquely
into prime ideals.
Show moreover that $I$ is the unique prime ideal containing $(2)$
and conclude that $(2)$ is not a product of prime ideals.} \\

\emph{Proof.}
\begin{enumerate}
\item[(1)]
  \emph{Show that $I \neq (2)$.}
    \begin{enumerate}
    \item[(a)]
      \emph{Show that $I \supseteq (2)$.}
      $2 \in (2,1+\sqrt{-3}) = I$.
      
    \item[(b)]
      \emph{Show that $I \not\subseteq (2)$.}
      Consider $1+\sqrt{-3} \in I$.
      (Reductio ad absurdum)
      If $1+\sqrt{-3}$ were in $(2)$, then there exists $a+b\sqrt{-3}$ such that
        $$1+\sqrt{-3} = 2(a+b\sqrt{-3}) = 2a+2b\sqrt{-3}.$$
      Thus, $a = \frac{1}{2}$ and $b = \frac{1}{2}$, which is absurd.
    \end{enumerate}

\item[(2)]
\emph{Show that $I^2 = 2I$.}
  \begin{enumerate}
  \item[(a)]
    \emph{Show that $I^2 \supseteq 2I$.}
    Since $2 \in (2,1+\sqrt{-3}) = I$, $2I \subseteq I^2$.
  
  \item[(b)]
    \emph{Show that $I^2 \subseteq 2I$.}
    All elements of $I^2$ are generated by
    $$2 \cdot 2, 2(1+\sqrt{-3}) \text{ and } (1+\sqrt{-3})^2.$$
    Clearly, $2 \cdot 2, 2(1+\sqrt{-3}) \in 2I$.
    Besides,
    $$(1+\sqrt{-3})^2 = -2 + 2\sqrt{-3} = 2(-(2) + (1+\sqrt{-3})) \in 2I.$$
    Hence $I^2 \subseteq 2I$.
  \end{enumerate}

\item[(3)]
\emph{Show that ideals in $\mathbb{Z}[\sqrt{-3}]$ do not factor uniquely
into prime ideals.}
  It is followed by $I^2 = 2I$ and $I \neq (2)$.

\item[(4)]
\emph{Show that $I$ is the unique prime ideal containing $(2)$.}
  \begin{enumerate}
  \item[(a)]
    \emph{Show that $I = (2, 1+\sqrt{-3})$ is a prime ideal containing $(2)$.}
    Note that
    \[
      \mathbb{Z}[\sqrt{-3}]/(2)
      = (\mathbb{Z}/2\mathbb{Z})[\sqrt{-3}]
      = \{ 0, 1, \sqrt{-3}, 1+\sqrt{-3} \}
    \]
    and
    \[
      I/(2) = (1+\sqrt{-3})
    \]
    is an ideal of $\mathbb{Z}[\sqrt{-3}]/(2)$.
    So
    \[
      \mathbb{Z}[\sqrt{-3}]/I
      = (\mathbb{Z}[\sqrt{-3}]/(2)) / (I/(2))
      = \{ 0,1 \}
      = \mathbb{Z}/2\mathbb{Z}
    \]
    is an integral domain.
    Hence $I$ is a prime ideal containing $(2)$.

  \item[(b)]
    Suppose $I'$ is a prime ideal containing $(2)$.
    Similar to part (a),
    \begin{align*}
      \mathbb{Z}[\sqrt{-3}]/I'
      &= (\mathbb{Z}[\sqrt{-3}]/(2)) / (I'/(2)) \\
      &= \{ 0, 1, \sqrt{-3}, 1+\sqrt{-3} \} / (I'/(2))
    \end{align*}
    must be an integral domain.

  \item[(c)]
    Since $\{ 0, 1, \sqrt{-3}, 1+\sqrt{-3} \}$ is not an integral domain,
    $I'/(2) \neq (0)$ or $I' \neq (2)$.
    Also, $I'/(2) \neq \{ 0, 1, \sqrt{-3}, 1+\sqrt{-3} \}$ implies that
    $I'/(2) \neq (1) = (\sqrt{-3})$.
    Therefore we must have $I'/(2) = (1+\sqrt{-3})$.
    Here the existence is guaranteed by part (a).
  \end{enumerate}

\item[(5)]
\emph{Show that $(2)$ is not a product of prime ideals.}
  (Reductio ad absurdum)
  Suppose $(2)$ were a product of prime ideals.
  By part (4), we might write $(2) = I^n$ for some positive integer $n$.
  Since $I \neq (2)$ and $I^2 = 2I$,
  \[
    (2) = (2) I^{n-1} \subseteq (2) I.
  \]
  for some $n \geq 2$.

\item[(6)]
  Take $2 \in (2) \subseteq (2) I$.
  Write
  \[
    2
    = 2 a_1 + \cdots + 2 a_k
    = 2 \underbrace{(a_1 + \cdots + a_k)}_{:= a \in I}
  \]
  where $a_1, \ldots, a_k \in I$.
  We take the norm of the both sides to get $N(a) = 1$. $a$ is a unit in $\mathbb{Z}[\sqrt{-3}]$.
  $I = \mathbb{Z}[\sqrt{-3}]$, which is absurd.
  Therefore $(2)$ is not a product of prime ideals.
\end{enumerate}
$\Box$ \\\\



%%%%%%%%%%%%%%%%%%%%%%%%%%%%%%%%%%%%%%%%%%%%%%%%%%%%%%%%%%%%%%%%%%%%%%%%%%%%%%%%



\subsubsection*{Exercise 2.3.}
\addcontentsline{toc}{subsubsection}{Exercise 2.3.}
\emph{Complete the proof of Corollary 2, Theorem 2.1.} \\

Corollary 2: Let $m$ be a squarefree integer.
The set of algebraic integers in the quadratic field $\mathbb{Q}[\sqrt{m}]$ is
\begin{align*}
    \{a + b\sqrt{m} : a, b \in \mathbb{Z} \}
        & \text{ if $m \equiv 2, 3 \pmod 4$,} \\
    \left\{ \frac{a+b\sqrt{m}}{2} : a, b \in \mathbb{Z}, a \equiv b \pmod 2\right\}
        & \text{ if $m \equiv 1 \pmod 4$.} \\
\end{align*}



\emph{Proof.}
\begin{enumerate}
\item[(1)]
  Let $\alpha = r + s\sqrt{m}$, $r, s \in \mathbb{Q}$.
  If $s \neq 0$, then the monic irreducible polynomial over $\mathbb{Q}$ having $\alpha$ as a root is
  \[
    x^2 - 2rx + r^2 - ms^2.
  \]
  Thus $\alpha$ is an algebraic integer iff $2r$ and $r^2 - ms^2$ are both integers.

\item[(2)]
  Hence $4(r^2 - ms^2) = (2r)^2 - m(2s)^2 \in \mathbb{Z}$.
  $m(2s)^2 \in \mathbb{Z}$ since $2r \in \mathbb{Z}$.
  Hence $2s \in \mathbb{Z}$ since $m$ is squarefree.
  Let $a = 2r, b = 2s \in \mathbb{Z}$.
  Then $a^2 - mb^2 = 4(r^2 - ms^2) \equiv 0 \pmod 4$.
  Note that a square $\equiv 0, 1 \pmod 4$ and thus we consider the following two cases.

\item[(3)]
  If $m \equiv 1 \pmod 4$, then
  \begin{align*}
    &\:a^2 - mb^2 \equiv a^2 - b^2 \pmod 4 \\
    \Longrightarrow &\:
    \text{$a$ and $b$ has the same parity} \\
    \Longrightarrow &\:
    \alpha = r + s\sqrt{m} = \frac{a + b\sqrt{m}}{2}, a, b \in \mathbb{Z}, a \equiv b \pmod 2.
  \end{align*}

\item[(4)]
  If $m \equiv 2, 3 \pmod 4$, then
  \begin{align*}
    &\:a^2 - mb^2 \equiv a^2+2b^2 \text{ or } a^2+b^2 \pmod 4 \\
    \Longrightarrow &\:
    \text{both $a$ and $b$ are even} \\
    \Longrightarrow &\:
    \text{both $r$ and $s$ are rational integers} \\
    \Longrightarrow &\:
    \alpha = r + s\sqrt{m}, r, s \in \mathbb{Z}.
  \end{align*}
\end{enumerate}
$\Box$ \\\\



\subsubsection*{Supplement.}
\addcontentsline{toc}{subsubsection}{Supplement.}
(Exercise I.2.4 in [J\"urgen Neukirch, \emph{Algebraic Number Theory}].)
\emph{Let $D$ be a squarefree rational integer $\neq 0, 1$ and
$d$ the discriminant of the quadratic number field $K = \mathbb{Q}(\sqrt{D})$.
Show that
\begin{equation*}
  d = \begin{cases}
    D  & \text{if $D \equiv 1 \pmod 4$,}\\
    4D & \text{if $D \equiv 2, 3 \pmod 4$.}
  \end{cases}
\end{equation*}
and that an integral basis of $K$ is given by $\{1, \sqrt{D}\}$ in the second case,
by $\left\{ 1, \frac{1+\sqrt{D}}{2} \right\}$ in the first case,
and by $\left\{ 1, \frac{d+\sqrt{d}}{2} \right\}$ in both case.} \\



\emph{Proof.}
\begin{enumerate}
\item[(1)]
  The Galois group of $K|\mathbb{Q}$ has two elements, the identity
  and an automorphism sending $\sqrt{D}$ to $-\sqrt{D}$.

\item[(2)]
  Note that $\alpha \in \mathcal{O}_K$
  iff $\mathrm{Tr}_{K|\mathbb{Q}}(\alpha), N_{K|\mathbb{Q}}(\alpha) \in \mathbb{Z}$
  (by noting that the equation
  $x^2 - \mathrm{Tr}_{K|\mathbb{Q}}(\alpha) x + N_{K|\mathbb{Q}}(\alpha) = 0$ has
  a root $x = \alpha$).
  So given $\alpha = x + y \sqrt{D} \in \mathcal{O}_K$, we have
  \begin{align*}
    \mathrm{Tr}_{K|\mathbb{Q}}(\alpha) &= 2x \in \mathbb{Z}, \\
    N_{K|\mathbb{Q}}(\alpha) &= x^2 - D y^2 \in \mathbb{Z}.
  \end{align*}

\item[(3)]
  So $4(x^2 - D y^2) = (2x)^2 - D(2y)^2 \in \mathbb{Z}$.
  So $D(2y)^2 \in \mathbb{Z}$ since $2x \in \mathbb{Z}$.
  So $2y \in \mathbb{Z}$ since $D$ is squarefree $\neq 0, 1$.
  Let $r = 2x, s = 2y$.
  Then $r^2 - Ds^2 = 4(x^2 - D y^2) \equiv 0 \pmod 4$.
  Note that a square $\equiv 0, 1 \pmod 4$ and thus we consider the following two cases.

\item[(4)]
  If $D \equiv 1 \pmod 4$, then
  \begin{align*}
    &\:r^2 - Ds^2 \equiv r^2 - s^2 \pmod 4 \\
    \Longrightarrow &\:
    \text{$r$ and $s$ has the same parity} \\
    \Longrightarrow &\:
    \mathcal{O}_K = \left\{ \frac{r+s\sqrt{D}}{2} : r \equiv s \pmod 2 \right\} \\
    \Longrightarrow &\:
    \mathcal{O}_K = \left\{ \frac{r-s}{2} + s \cdot \frac{1+\sqrt{D}}{2}
      : r \equiv s \pmod 2 \right\} \\
    \Longrightarrow &\:
    \mathcal{O}_K = \mathbb{Z} + \mathbb{Z} \frac{1+\sqrt{D}}{2}.
  \end{align*}
  So $\left\{ 1, \frac{1+\sqrt{D}}{2} \right\}$ is an integral basis of $K$.
  Hence
  \[
    d
    =
    \begin{vmatrix}
      1 & \frac{1+\sqrt{D}}{2} \\
      1 & \frac{1-\sqrt{D}}{2}
    \end{vmatrix}^2
    = D.
  \]

\item[(5)]
  If $D \equiv 2, 3 \pmod 4$, then
  \begin{align*}
    &\:r^2 - Ds^2 \equiv r^2+2s^2 \text{ or } r^2+s^2 \pmod 4 \\
    \Longrightarrow &\:
    \text{both $r$ and $s$ are even} \\
    \Longrightarrow &\:
    \text{both $x$ and $y$ are rational integers} \\
    \Longrightarrow &\:
    \mathcal{O}_K = \mathbb{Z} + \mathbb{Z}\sqrt{D}.
  \end{align*}
  So $\{ 1, \sqrt{D} \}$ is an integral basis of $K$.
  Hence
  \[
    d
    =
    \begin{vmatrix}
      1 & \sqrt{D} \\
      1 & -\sqrt{D}
    \end{vmatrix}^2
    = 4D.
  \]

\item[(6)]
  By (4)(5),
  $\left\{ 1, \frac{d+\sqrt{d}}{2} \right\}$ is an integral basis of $K$
  for any case.
\end{enumerate}
$\Box$ \\\\



%%%%%%%%%%%%%%%%%%%%%%%%%%%%%%%%%%%%%%%%%%%%%%%%%%%%%%%%%%%%%%%%%%%%%%%%%%%%%%%%



\subsubsection*{Exercise 2.4.}
\addcontentsline{toc}{subsubsection}{Exercise 2.4.}
\emph{Suppose $a_0, \ldots, a_{n-1}$ are algebraic integers and
$\alpha$ is a complex number satisfying
$$\alpha^n + a_{n-1} \alpha^{n-1} + \cdots + a_1 \alpha + a_0 = 0.$$
Show that the ring
$\mathbb{Z}[a_0, \ldots, a_{n-1}, \alpha]$ has a finitely generated additive group.
(Hint: Consider the products $a_0^{m_0} a_1^{m_1} \cdots a_{n-1}^{m_{n-1}} \alpha^m$
and show that only finitely many values of the exponents are needed.)
Conclude that $\alpha$ is an algebraic integer.} \\

\emph{Proof.}
Let $V = \mathbb{Z}[a_0, \ldots, a_{n-1}, \alpha]$.
Let $n_k$ be the degree of the algebraic integer $a_k$ where $0 \leq k \leq n-1$.
\begin{enumerate}
\item[(1)]
\emph{Show that $V$ is finitely generated as an additive subgroup of $\mathbb{C}$.}
\emph{It suffices to show that $V$ is generated by
$$a_0^{m_0} a_1^{m_1} \cdots a_{n-1}^{m_{n-1}} \alpha^m$$
where $0 \leq m_k < n_k$ and $0 \leq m < n$.}
Given any $x \in V$,
$x$ is a finite sum of the product
$a_0^{m_0} a_1^{m_1} \cdots a_{n-1}^{m_{n-1}} \alpha^m$
with $m_k \geq 0$ and $m \geq 0$. \\

If $m \geq n$,
replace $\alpha^m$
by
\begin{align*}
\alpha^m
=& \alpha^{m-n} \alpha^{n} \\
=& \alpha^{m-n} (-a_{n-1} \alpha^{n-1} - \cdots - a_1 \alpha - a_0) \\
=& -a_{n-1} \alpha^{m-1} - \cdots - a_1 \alpha^{m-n+1} - a_0 \alpha^{m-n}.
\end{align*}
Repeat this process to reduce the degree of $\alpha^m$ less than $n$.
Therefore, we can write $x$ as a finite sum of the product
$a_0^{m_0'} a_1^{m_1'} \cdots a_{n-1}^{m_{n-1}'} \alpha^{m'}$
with $m_k' \geq 0$ and $0 \leq m' < n$. \\

Once the degree of $\alpha^m$ is reduced,
continue to reduce the degree of each $a_k^{m_k'}$
without affecting other $a_h$ ($h \neq k$) and $\alpha$.
Now replace $a_k^{m_k'}$
by
$$a_k^{m_k'} = \sum_{i = 0}^{n_k-1} b_{k,i} a_k^{i}$$
where $b_{k,i} \in \mathbb{Z}$.
Therefore, we can write $x$ as a finite sum of the product
$a_0^{m_0''} a_1^{m_1''} \cdots a_{n-1}^{m_{n-1}''} \alpha^{m'}$
with $0 \leq m_k'' < n_k$ and $0 \leq m' < n$.
\item[(4)]
\emph{Show that $\alpha$ is an algebraic integer.}
Since $\alpha \in V$, $\alpha V \subseteq V$.
Thus $\alpha$ is an algebraic integer (Theorem 2.2).
\end{enumerate}
$\Box$ \\\\



%%%%%%%%%%%%%%%%%%%%%%%%%%%%%%%%%%%%%%%%%%%%%%%%%%%%%%%%%%%%%%%%%%%%%%%%%%%%%%%%



\subsubsection*{Exercise 2.5.}
\addcontentsline{toc}{subsubsection}{Exercise 2.5.}
\emph{Show that if $f$ is any polynomials over $\mathbb{Z}/p\mathbb{Z}$ ($p$ a prime)
then $f(x^p) = (f(x))^p$.
(Suggestion: Use induction on the number of terms.)} \\

\emph{Proof.}
\begin{enumerate}
\item[(1)]
\emph{Let $${p \choose k} = \frac{p!}{k!(p-k)!}$$
be a binomial coefficient.
If $1 \leq k \leq p-1$, show that $p$ divides ${p \choose k}$.}
  \begin{enumerate}
  \item[(a)]
    If $1 \leq k \leq p-1$, then $p \nmid k!$ and $p \nmid (p-k)!$
    since $p$ is a prime.
  \item[(b)]
    Write $a = \frac{p!}{k!(p-k)!} \in \mathbb{Z}$.
    Hence,
    \begin{align*}
    a = \frac{p!}{k!(p-k)!}
    &\Longleftrightarrow
    p! = ak!(p-k)! \\
    &\Longrightarrow
    p \mid p! \text{ or } p \mid ak!(p-k)! \\
    &\Longrightarrow
    p \mid a \:\:\:\: \text{by (a).}
    \end{align*}
    Hence $p$ divides ${p \choose k}$ if $1 \leq k \leq p-1$.
  \end{enumerate}
\item[(2)]
Note that $a^p = a \in \mathbb{Z}/p\mathbb{Z}$ for all $a \in \mathbb{Z}/p\mathbb{Z}$.
\item[(3)]
Write
$$f(x) = a_n x^n + a_{n-1} x^{n-1} + \cdots + a_1 x + a_0 \in \mathbb{Z}/p\mathbb{Z}[x].$$
Induction on $n$.
  \begin{enumerate}
  \item[(a)]
    $n = 0$. So $f(x) = a_0$, and thus $f(x)^p = a_0^p = a_0$ by (2).
  \item[(b)]
    $n = 1$. By $f(x) = a_1 x + a_0$,
    \begin{align*}
    f(x)^p
    &= (a_1 x + a_0)^p \\
    &= a_1^p x^p
      + \sum_{k=1}^{p-1} {p \choose k} (a_1 x)^k a_0^{p-k}
      + a_0^p
      &\text{(Binomial theorem)} \\
    &= a_1^p x^p + a_0^p
      &\text{((1))} \\
    &= a_1 x^p + a_0
      &\text{((2))} \\
    &= f(x^p).
    \end{align*}
  \item[(c)]
  If the statement holds for $n-1$, then
    \begin{align*}
    f(x)^p
    &= (a_n x^n + a_{n-1} x^{n-1} + \cdots + a_1 x + a_0)^p \\
    &= [a_n x^n + (a_{n-1} x^{n-1} + \cdots + a_1 x + a_0)]^p \\
    &= (a_n x^n)^p + (a_{n-1} x^{n-1} + \cdots + a_1 x + a_0)^p
      &\text{(Same as (b))} \\
    &= a_n (x^p)^n+ (a_{n-1} x^{n-1} + \cdots + a_1 x + a_0)^p
      &\text{((2))} \\
    &= a_n (x^p)^n + a_{n-1} (x^p)^{n-1} + \cdots + a_1 x^p + a_0
      &\text{(Induction hypothesis)} \\
    &= f(x^p).
    \end{align*}
  The inductive step is established.
  \end{enumerate}
  By induction, $f(x)^p = f(x^p)$ holds for any $n \geq 0$.
\end{enumerate}
$\Box$ \\\\



%%%%%%%%%%%%%%%%%%%%%%%%%%%%%%%%%%%%%%%%%%%%%%%%%%%%%%%%%%%%%%%%%%%%%%%%%%%%%%%%



\subsubsection*{Exercise 2.6.}
\addcontentsline{toc}{subsubsection}{Exercise 2.6.}
\emph{Show that if $f$ and $g$ are polynomials over a field $K$ and
$f^2 \mid g$ in $K[x]$, then $f \mid g'$.
(Hint: Write $g = f^2 h$ and differentiate.)} \\

\emph{Proof (Hint).}
Since $f^2 \mid g$ in $K[x]$, there exists $h \in K[x]$ such $g = f^2 h$.
Differentiate to get $g' = 2f f' h + f^2 h' = f(2f'h + fh')$,
or $f \mid g'$ in $K[x]$.
$\Box$ \\\\



%%%%%%%%%%%%%%%%%%%%%%%%%%%%%%%%%%%%%%%%%%%%%%%%%%%%%%%%%%%%%%%%%%%%%%%%%%%%%%%%



\subsubsection*{Exercise 2.7.}
\addcontentsline{toc}{subsubsection}{Exercise 2.7.}
\emph{Complete the proof of Corollary 2, Theorem 2.3.} \\



Corollary 2: The galois group of $\mathbb{Q}[\omega]$ over $\mathbb{Q}$ is isomorphic to
the multiplicative group of integer $\pmod m$
\[
  (\mathbb{Z}/m\mathbb{Z})^{*} = \{ k : 1 \leq k \leq m, (k,m) = 1\}.
\]
For each $k \in (\mathbb{Z}/m\mathbb{Z})^{*}$,
the corresponding automorphism in the galois group sends $\omega$ to $\omega^k$
(and hence $g(\omega) \to g(\omega^k)$ for each $g \in \mathbb{Z}[x]$). \\



\emph{Proof.}
\begin{enumerate}
\item[(1)]
  An automorphism of $\mathbb{Q}[\omega]$ is uniquely determined by the image of $\omega$,
  and Theorem 2.3 shows that $\omega$ can be sent to any of the $\omega^k$, $(k,m) = 1$.
  (Clearly it can't be sent anywhere else.)
  This established the one-to-one correspondence between the galois group and
  the multiplicative group of integer $\pmod m$, say
  \[
    \alpha: \mathrm{Gal}(\mathbb{Q}[\omega]/\mathbb{Q})
    \to (\mathbb{Z}/m\mathbb{Z})^{*}.
  \]

\item[(2)]
  The composition of automorphisms corresponds to multiplication $\pmod m$ in the natural way.
  That is, if $\sigma, \tau \in \mathrm{Gal}(\mathbb{Q}[\omega]/\mathbb{Q})$
  with $\sigma(\omega) = \omega^k$ and $\tau(\omega) = \omega^h$,
  then
  \[
    (\sigma\tau)(\omega) = \sigma(\omega^h) = \omega^{kh}
    \xrightarrow{\alpha} kh.
  \]
  Hence $\alpha$ is a group homomorphism.
\end{enumerate}
$\Box$ \\\\



%%%%%%%%%%%%%%%%%%%%%%%%%%%%%%%%%%%%%%%%%%%%%%%%%%%%%%%%%%%%%%%%%%%%%%%%%%%%%%%%



\subsubsection*{Exercise 2.8.}
\addcontentsline{toc}{subsubsection}{Exercise 2.8.}
\begin{enumerate}
\item[(a)]
  \emph{Let $\omega = e^{\frac{2\pi i}{p}}$, $p$ an odd prime.
  Show that $\mathbb{Q}[\omega]$ contains $\sqrt{p}$ if $p \equiv 1 \pmod{4}$,
  and $\sqrt{-p}$ if $p \equiv 3 \pmod{4}$.
  (Hint: Recall that we have shown that $\mathrm{disc}(\omega) = \pm p^{p-2}$ with $+$ holding
  iff $p \equiv 1 \pmod{4}$.)
  Express $\sqrt{-3}$ and $\sqrt{5}$ as polynomials in the appropriate $\omega$.}

\item[(b)]
  \emph{Show that the eighth cyclotomic field contains $\sqrt{2}$.}

\item[(c)]
  \emph{Show that every quadratic field is contained in a cyclotomic field:
  In fact, $\mathbb{Q}[\sqrt{m}]$ is contained in the $d$-th cyclotomic field,
  where $d = \mathrm{disc}(\mathcal{O}_{\mathbb{Q}[\sqrt{m}]})$.
  (More generally, Kronecker and Weber proved that every abelian extension of $\mathbb{Q}$
  (normal with abelian Galois group) is contained in a cyclotomic field.
  See the Chapter 4 exercises.
  Hilbert and others investigated the abelian extensions of an arbitrary number field;
  their results are known as \textbf{class field theory}, which will be discussed in Chapter 8.)} \\
\end{enumerate}



\emph{Proof of (a).}
\begin{enumerate}
\item[(1)]
  Recall that we have shown that
  \[
    \mathrm{disc}(\omega)
    = \prod_{1 \leq r < s \leq p}(\omega_r - \omega_s)^2
    = (-1)^{\frac{p-1}{2}} p^{p-2}
    = (-1)^{\frac{p-1}{2}} p \cdot p^{p-3}
  \]
  where $\omega_1 = \omega, \ldots, \omega_p$ are the conjugates of $\omega$ over $\mathbb{Q}$.
  Hence
  \[
    \prod_{1 \leq r < s \leq p}(\omega_r - \omega_s)
    = \pm \sqrt{(-1)^{\frac{p-1}{2}} p} \cdot p^{\frac{p-3}{2}}
   \in \mathbb{Q}[\omega].
  \]
  Note that $p^{\frac{p-3}{2}} \in \mathbb{Q}$ as $p \geq 3$ is odd
  and $\pm$ is unrelated as $\mathbb{Q}[\omega]$ is a field.
  Therefore
  \[
    \sqrt{(-1)^{\frac{p-1}{2}} p} \in \mathbb{Q}[\omega].
  \]

\item[(2)]
  \emph{Express $\sqrt{-3}$ as polynomials in the appropriate $\omega$.}
  Take $\omega = e^{\frac{2\pi i}{3}}$.
  A direct computing shows that
  \begin{align*}
    \prod_{1 \leq r < s \leq 3}(\omega_r - \omega_s)
    &= \prod_{1 \leq r < s \leq 3}(\omega^r - \omega^s) \\
    &= (1-\omega)(1-\omega^2)(\omega-\omega^2) \\
    &= 3(-\omega^2 + \omega) \\
    &= 3 \sqrt{-3}.
  \end{align*}
  Hence $\sqrt{-3} = -\omega^2 + \omega$.

\item[(3)]
  \emph{Express $\sqrt{5}$ as polynomials in the appropriate $\omega$.}
  Take $\omega = e^{\frac{2\pi i}{5}}$.
  A direct computing shows that
  \begin{align*}
    \prod_{1 \leq r < s \leq 5}(\omega_r - \omega_s)
    &= \prod_{1 \leq r < s \leq 5}(\omega^r - \omega^s) \\
    &= 3(\omega - \omega^2) \\
    &= -25(\omega^4 - \omega^3 - \omega^2 + \omega) \\
    &= -25 \sqrt{5}.
  \end{align*}
  Hence $\sqrt{5} = \omega^4 - \omega^3 - \omega^2 + \omega$.

\item[(4)]
  (Another proof)
  The quadratic Gauss sum shows that
  \[
    \sum_{n=0}^{p-1} e^{\frac{2\pi i n^2}{p}} = \sqrt{(-1)^{\frac{p-1}{2}} p}.
  \]
  So $\sqrt{-3} = 2\omega_3 + 1$ and $\sqrt{5} = 2\omega^4_5 + 2\omega_5 + 1$.
\end{enumerate}
$\Box$ \\



\emph{Proof of (b).}
\begin{enumerate}
\item[(1)]
  A root of eighth unity is $\omega = \frac{\sqrt{2}}{2} + \frac{\sqrt{2}}{2} i$.

\item[(2)]
  Hence
  \[
    \omega + \omega^{-1}
    = \left( \frac{\sqrt{2}}{2} + \frac{\sqrt{-2}}{2} \right)
        + \left( \frac{\sqrt{2}}{2} - \frac{\sqrt{-2}}{2} \right)
    = \sqrt{2} \in \mathbb{Q}[\omega].
  \]
\end{enumerate} 
$\Box$ \\



\emph{Proof of (c).}
\begin{enumerate}
\item[(1)]
  Note that $\mathbb{Q}[\omega_a, \omega_b] = \mathbb{Q}[\omega_{ab}]$ if
  $a,b \in \mathbb{Z}$ are relatively prime.
  Might assume that $m$ is squarefree since $\mathbb{Q}[\sqrt{ab^2}] = \mathbb{Q}[\sqrt{a}]$.
  Consider the following four cases.

\item[(2)]
  Suppose $m > 0$ and $2 \nmid m$.
  Write
  \[
    m = p_1 \cdots p_r \cdot q_1 \cdots q_s
  \]
  as a product of distinct primes where $p_i \equiv 1 \pmod 4$ and $q_j \equiv 3 \pmod 4$.
  Part (a) shows that
  \[
    \sqrt{p_1}, \ldots, \sqrt{p_r}, \sqrt{-q_1}, \cdots, \sqrt{-q_s}
    \in \mathbb{Q}[\omega_{p_1}, \ldots, \omega_{p_r}, \omega_{q_1}, \ldots, \omega_{q_s}].
  \]
  So $\sqrt{(-1)^s m} \in \mathbb{Q}[\omega_{m}]$.
  If $s$ is even, then $\sqrt{m} \in \mathbb{Q}[\omega_{m}]$ or
  $\mathbb{Q}[\sqrt{m}] \subseteq \mathbb{Q}[\omega_{m}]$.
  If $s$ is odd, then $\sqrt{m} \in \mathbb{Q}[\omega_{m}, \omega_{4}] = \mathbb{Q}[\omega_{4m}]$
  (since $\sqrt{-1} \in \mathbb{Q}[\omega_{4}]$).
  In any case,
  $\mathbb{Q}[\sqrt{m}]$ is contained in the $d$-th cyclotomic field,
  where $d = \mathrm{disc}(\mathcal{O}_{\mathbb{Q}[\sqrt{m}]})$.
  (See Supplement to Exercise 2.3.)

\item[(3)]
  Suppose $m < 0$ and $2 \nmid m$.
  Similar to (2).

\item[(4)]
  Suppose $m > 0$ and $2 \mid m$.
  Write
  \[
    m = 2 \cdot p_1 \cdots p_r \cdot q_1 \cdots q_s
  \]
  as a product of distinct primes where $p_i \equiv 1 \pmod 4$ and $q_j \equiv 3 \pmod 4$.
  Parts (a)(b) show that
  \[
    \sqrt{2}, \sqrt{p_1}, \ldots, \sqrt{p_r}, \sqrt{-q_1}, \cdots, \sqrt{-q_s}
    \in \mathbb{Q}[\omega_8, \omega_{p_1}, \ldots, \omega_{p_r}, \omega_{q_1}, \ldots, \omega_{q_s}].
  \]
  So $\sqrt{(-1)^s m} \in \mathbb{Q}[\omega_{4m}]$.
  Note that $\sqrt{(-1)^s} \in \mathbb{Q}[\omega_{4}] \subseteq \mathbb{Q}[\omega_{4m}]$.
  Hence $\sqrt{m} \in \mathbb{Q}[\omega_{4m}]$ is contained in the $d$-th cyclotomic field,
  where $d = 4m = \mathrm{disc}(\mathcal{O}_{\mathbb{Q}[\sqrt{m}]})$.

\item[(5)]
  Suppose $m < 0$ and $2 \mid m$. Same as (4).
\end{enumerate}
$\Box$ \\\\



%%%%%%%%%%%%%%%%%%%%%%%%%%%%%%%%%%%%%%%%%%%%%%%%%%%%%%%%%%%%%%%%%%%%%%%%%%%%%%%%



\subsubsection*{Exercise 2.9.}
\addcontentsline{toc}{subsubsection}{Exercise 2.9.}
\emph{With notation as in the proof of Corollary 3, Theorem 2.3,
show that there exist integers $u$ and $v$ such that
$e^{\frac{2\pi i}{r}} = \omega^u \theta^v$.} \\



\emph{Proof.}
\begin{enumerate}
\item[(1)]
  Recall $\omega = e^{\frac{2\pi i}{m}}$, $\theta = e^{\frac{2\pi i}{k}}$
  and $r$ is the least common multiple of $k$ and $m$.

\item[(2)]
  As $r$ is the least common multiple of $k$ and $m$,
  there exist coprime integers $a$ and $b$ such that $r = am = bk$.
  As $(a,b) = 1$, there exist integers $u$ and $v$ such that $au + bv = 1$.

\item[(3)]
  Hence,
  \begin{align*}
    \omega^u \theta^v
    &= e^{\frac{2\pi i u}{m}} \cdot e^{\frac{2\pi i v}{k}} \\
    &= e^{\frac{2\pi i au}{r}} \cdot e^{\frac{2\pi i bv}{r}} \\
    &= e^{\frac{2\pi i (au+bv)}{r}} \\
    &= e^{\frac{2\pi i}{r}}.
  \end{align*}
\end{enumerate}
$\Box$ \\\\



%%%%%%%%%%%%%%%%%%%%%%%%%%%%%%%%%%%%%%%%%%%%%%%%%%%%%%%%%%%%%%%%%%%%%%%%%%%%%%%%



\subsubsection*{Exercise 2.10.}
\addcontentsline{toc}{subsubsection}{Exercise 2.10.}
\emph{Complete the proof of Corollary 3 to Theorem 2.3,
by showing if $m$ is even, $m \mid r$, and $\varphi(r) \leq \varphi(m)$, then $r = m$.} \\

\emph{Proof.}
\begin{enumerate}
\item[(1)]
Since $m$ is even,
write the unique factorization of $m$ as
$$m = p_1^{\alpha_1} p_2^{\alpha_2} \cdots p_k^{\alpha_k}$$
where $p_1 = 2$, all $\alpha_i \geq 1$ $(1 \leq i \leq k)$,
and all $p_i$ $(1 \leq i \leq k)$
are distinct prime numbers.
\item[(2)]
Since $m \mid r$, write $r = mm_1$ for some $m_1 \in \mathbb{Z}$.
Thus we can write the unique factorization of $r$ as
$$m = p_1^{\beta_1} p_2^{\beta_2} \cdots p_k^{\beta_k}
q_1^{\gamma_1} \cdots q_h^{\gamma_h}$$
where all $\beta_i \geq \alpha_i \geq 1$ $(1 \leq i \leq k)$
and all $p_i$ $(1 \leq i \leq k)$ and $q_j$ $(1 \leq j \leq h)$
are distinct prime numbers.
Here $h$ might be zero if $m_1 = 1$, and all $q_j \mid m_1$ but $q_j \nmid m$.
\item[(3)]
Thus,
\begin{align*}
\varphi(m)
&= m
    \left( 1 - \frac{1}{2} \right)
    \cdots
    \left( 1 - \frac{1}{p_k} \right) \\
\varphi(r)
&= mm_1
    \left( 1 - \frac{1}{2} \right)
    \cdots
    \left( 1 - \frac{1}{p_k} \right)
    \left( 1 - \frac{1}{q_1} \right)
    \cdots
    \left( 1 - \frac{1}{q_h} \right) \\
&= \varphi(m) m_1
    \left( 1 - \frac{1}{q_1} \right)
    \cdots
    \left( 1 - \frac{1}{q_h} \right) \\
&\geq \varphi(m) (q_1 \cdots q_h)
    \left( 1 - \frac{1}{q_1} \right)
    \cdots
    \left( 1 - \frac{1}{q_h} \right) \\
&\geq \varphi(m)(q_1-1) \cdots (q_h-1).
\end{align*}
\item[(4)]
Since all $q_j \neq 2$ $(1 \leq j \leq h)$,
$q_j - 1 > 1$.
Hence by (3) and assumption that $\varphi(r) \leq \varphi(m)$,
$h = 0$ or $m_1 = 1$ or $r = m$.
\end{enumerate}
$\Box$ \\\\



%%%%%%%%%%%%%%%%%%%%%%%%%%%%%%%%%%%%%%%%%%%%%%%%%%%%%%%%%%%%%%%%%%%%%%%%%%%%%%%%



\subsubsection*{Exercise 2.11.}
\addcontentsline{toc}{subsubsection}{Exercise 2.11.}
\begin{enumerate}
\item[(a)]
\emph{Suppose all roots of a monic polynomial $f \in \mathbb{Q}[x]$
has absolute value $1$.
Show that the coefficient of $x^r$ has absolute value $\leq {n \choose r}$,
where $n$ is the degree of $f$ and ${n \choose r}$ is the binomial coefficient.}
\item[(b)]
\emph{Show that there are only finitely many algebraic integers $\alpha$
of fixed degree $n$, all of whose conjugates (including $\alpha$) have absolute value $1$.
(Note: If you don't use Theorem 2.1, your proof is probably wrong.)}
\item[(c)]
\emph{Show that $\alpha$ must be a root of $1$.
(Show that its powers are restricted to a finite set.)} \\
\end{enumerate}

\emph{Proof of (a).}
\begin{enumerate}
\item[(1)]
Write
$f(x) = (x-\alpha_1) \cdots (x-\alpha_n)$ where $\alpha_i \in \mathbb{C}$, $|\alpha_i| = 1$
for $i = 1, 2, \ldots, n$.
\item[(2)]
So $$f(x) = x^n - s_1 x^{n-1} + s_2 x^{n-2} + \cdots + (-1)s_n$$
where
$$s_r = \sum_{1 \leq j_1 < \cdots < j_r \leq n} \alpha_{j_1} \cdots \alpha_{j_r} \in \mathbb{C}.$$
Let $c_r = (-1)^r s_{n-r}$ be the coefficient of $x^r$.
\item[(3)]
\begin{align*}
|c_r|
&= |(-1)^r s_{n-r}| \\
&= \abs{\sum_{1 \leq j_1 < \cdots < j_{n-r} \leq n} \alpha_{j_1} \cdots \alpha_{j_{n-r}}} \\
&\leq \sum_{1 \leq j_1 < \cdots < j_{n-r} \leq n} \abs{\alpha_{j_1} \cdots \alpha_{j_{n-r}}} \\
&= \sum_{1 \leq j_1 < \cdots < j_{n-r} \leq n} |\alpha_{j_1}| \cdots |\alpha_{j_{n-r}}| \\
&= \sum_{1 \leq j_1 < \cdots < j_{n-r} \leq n} 1 \\
&= {n \choose n-r} \\
&= {n \choose r}.
\end{align*}
\end{enumerate}
$\Box$ \\

\emph{Proof of (b).}
\begin{enumerate}
\item[(1)]
Let $f$ be an irreducible monic polynomial over $\mathbb{Z}$ of degree $n$
such that $f(\alpha) = 0$.
So $f$ is irreducible over $\mathbb{Q}$ (Theorem 2.1),
and thus all the conjugates of $\alpha$ (including $\alpha$) are roots of $f$.
\item[(2)]
By (a), all the coefficient of $x^r$ has absolute value $\leq {n \choose r}$.
Since all the coefficient of $x^r$ are integers,
there are finitely many irreducible monic polynomials $f \in \mathbb{Z}[x]$
such that $f(\alpha) = 0$ with $|\alpha| = 1$.
\item[(3)]
For each such $f$, there are only finitely many roots.
Therefore, there are only finitely many such algebraic integers $\alpha$.
\end{enumerate}
$\Box$ \\



\emph{Proof of (c).}
\begin{enumerate}
\item[(1)]
If $\alpha_1, \ldots, \alpha_n$
are the roots of $f$ of degree $n$ over $\mathbb{Q}$,
then for every $r \in \mathbb{Z}^+$,
$\alpha_1^r, \ldots, \alpha_n^r$
are all the roots of some monic polynomial $f_r$ of degree $n$ over $\mathbb{Q}$
(Fundamental theorem of symmetric polynomials).
\item[(2)]
Now we consider the powers of $\alpha$.
All the powers of $\alpha$ ($\alpha^r$) are algebraic integers (Theorem 2.2), and of
degree at most $n$.
(Let $g \in \mathbb{Z}[x]$ be the minimal polynomial of $\alpha^r$ over $\mathbb{Q}$.
By (1), $f_r(\alpha^r) = 0$, and thus $g \mid f_r$.
Hence $\deg(g) \leq \deg(f_r) = n$.)
\item[(3)]
By (b), the powers of $\alpha$ are restricted to a finite set,
say $\alpha^r = \alpha^s$ for some $s > r \geq 1$.
So $\alpha^{s-r} = 1$ with $s-r \geq 1$.
That is, $\alpha$ is a root of unity.
\end{enumerate}
$\Box$ \\\\



%%%%%%%%%%%%%%%%%%%%%%%%%%%%%%%%%%%%%%%%%%%%%%%%%%%%%%%%%%%%%%%%%%%%%%%%%%%%%%%%



\subsubsection*{Exercise 2.12. (Kummer's Lemma)}
\addcontentsline{toc}{subsubsection}{Exercise 2.12. (Kummer's Lemma)}
\emph{Now we can prove Kummer's lemma on units in the $p$-th cyclotomic field,
as stated before Exercise 1.26:
Let $\omega = e^{\frac{2\pi i}{p}}$, $p$ an odd prime,
and suppose $u$ is a unit in $\mathbb{Z}[\omega]$.}
\begin{enumerate}
\item[(a)]
\emph{Show that $u/\overline{u}$ is a root of $1$.
(Use Exercise 2.11(c) above and observe that complex conjugation is a member of
the Galois group of $\mathbb{Z}[\omega]$ over $\mathbb{Q}$.)
Conclude that $u/\overline{u} = \pm \omega^k$ for some $k$.}
\item[(b)]
\emph{Show that the $+$ sign holds: Assuming $u/\overline{u} = -\omega^k$,
we have $u^p = -\overline{u^p}$;
show that this implies that $u^p$ is divisible by $p$ in $\mathbb{Z}[\omega]$.
(Use Exercise 1.23 and 1.25)
But this is impossible since $u^p$ is a unit.} \\
\end{enumerate}

\emph{Proof of (a).}
Write $\alpha = u/\overline{u}$. Then
\begin{align*}
|\alpha| = 1
&\Longrightarrow
\text{$\alpha$ is a root of unity}
  &\text{(Exercise 2.11)} \\
&\Longrightarrow
\text{$\alpha$ is a $2p$-th root of unity}
  &\text{(Corollary 3 to Theorem 2.3)} \\
&\Longrightarrow
\text{$\alpha = \pm \omega^k$ for some $k \in \mathbb{Z}$}
\end{align*}
$\Box$ \\

\emph{Proof of (b).}
(Reductio ad absurdum)
Assume that $u/\overline{u} = -\omega^k$,
then
\begin{align*}
u/\overline{u} = -\omega^k
&\Longrightarrow
(u/\overline{u})^p = (-\omega^k)^p \\
&\Longrightarrow
u^p/\overline{u}^p = (-1)^p \omega^{pk} = -1
  &\text{($p$ is odd)} \\
&\Longrightarrow
u^p = -\overline{u}^p = -\overline{u^p}
\end{align*}
By Exercise 1.25, $u^p \equiv a \pmod{p}$ for some $a \in \mathbb{Z}$.
By Exercise 1.23, $\overline{u^p} \equiv \overline{a} \equiv a \pmod{p}$.
Thus
\begin{align*}
u^p = -\overline{u^p}
&\Longrightarrow
a \equiv -a \pmod{p} \\
&\Longrightarrow
2a \equiv 0 \pmod{p} \\
&\Longrightarrow
a \equiv 0 \pmod{p}
  &\text{($p$ is odd)}
\end{align*}
or $u^p \equiv 0 \pmod{p}$, contradicts the assumption that $u$ is a unit.
Hence $u/\overline{u} = \omega^k$ for some $k$.
$\Box$ \\\\



%%%%%%%%%%%%%%%%%%%%%%%%%%%%%%%%%%%%%%%%%%%%%%%%%%%%%%%%%%%%%%%%%%%%%%%%%%%%%%%%



\subsubsection*{Exercise 2.13.}
\addcontentsline{toc}{subsubsection}{Exercise 2.13.}
\emph{Show that $1$ and $-1$ are the only units in the ring
$\mathcal{O}_{\mathbb{Q}[\sqrt{m}]}$, $m$ squarefree, $m < 0$,
$m \neq -1, -3$.
What if $m = -1$ or $-3$?} \\

\emph{Proof.}
\begin{enumerate}
\item[(1)]
  Let $K = \mathbb{Q}[\sqrt{m}]$.
  Define a norm $N$ on $K$
  by $$N(a+b\sqrt{m}) = (a+b\sqrt{m})(a-b\sqrt{m}) = a^2 + |m|b^2.$$

\item[(2)]
  Corollary 2 to Theorem 2.1 shows that
  \begin{equation*}
    \mathcal{O}_K =
      \begin{cases}
        \{ a+b\sqrt{m} : a,b \in \mathbb{Z} \}
          & (m \equiv 2,3 \pmod{4}), \\
        \left\{ \frac{a+b\sqrt{m}}{2} : a,b \in \mathbb{Z}, a \equiv b \pmod{2} \right\}
          & (m \equiv 1 \pmod{4}).
      \end{cases}
  \end{equation*}
  Clearly, $N$ maps $\mathcal{O}_K$ to nonnegative integers.
  That is, $u$ is a unit in $\mathcal{O}_K$ if and only if $N(u) = 1$
  (by the fact that $N(u) = u \overline{u}$).

\item[(3)]
  If $m \equiv 2,3 \pmod{4}$ and $u = a + b\sqrt{m} \in \mathcal{O}_K$ is a unit
  ($a, b \in \mathbb{Z}$),
  then $$N(u) = 1 = a^2 + |m|b^2.$$
    \begin{enumerate}
    \item[(a)]
      \emph{$m = -1$ or $|m| = 1$.}
      $1 = a^2 + b^2$ or $(a,b) = (\pm 1,0), (0, \pm 1)$.
      Hence all units in $\mathcal{O}_K$ are
      $$\pm 1, \pm \sqrt{-1}.$$

    \item[(b)]
      \emph{$m < -1$ or $|m| > 1$.}
      $1 = a^2 + |m|b^2$ implies that $b^2 = 0$.
      Hence all units in $\mathcal{O}_K$ are $\pm 1$.
    \end{enumerate}

\item[(4)]
  If $m \equiv 1 \pmod{4}$ and $u = \frac{a + b\sqrt{m}}{2} \in \mathcal{O}_K$ is a unit
  ($a, b \in \mathbb{Z}, a \equiv b \pmod{2}$),
  then $N(u) = 1 = (\frac{a}{2})^2 + |m|(\frac{b}{2})^2$ or
  $$4 = a^2 + |m|b^2.$$
    \begin{enumerate}
    \item[(a)]
      \emph{$m = -3$ or $|m| = 3$.}
      $4 = a^2 + 3b^2$ or $(a,b) = (\pm 2,0), (\pm 1, \pm 1)$.
      Hence all units in $\mathcal{O}_K$ are
      $$\pm 1, \frac{\pm 1 \pm \sqrt{-3}}{2}.$$
    
    \item[(b)]
      \emph{$m < -3$ or $|m| > 3$.}
      $4 = a^2 + |m|b^2$ implies that $b^2 = 0$.
      Hence all units in $\mathcal{O}_K$ are $\pm 1$.
    \end{enumerate}

\item[(5)]
  By (3)(4), all units in $\mathcal{O}_K$ are
    \begin{equation*}
      \begin{cases}
        \pm 1                                & (m \neq -1, -3), \\
        \pm 1, \pm \sqrt{-1}                 & (m = -1), \\
        \pm 1, \frac{\pm 1 \pm \sqrt{-3}}{2} & (m = -3).
      \end{cases}
    \end{equation*}
\end{enumerate}
$\Box$ \\\\



%%%%%%%%%%%%%%%%%%%%%%%%%%%%%%%%%%%%%%%%%%%%%%%%%%%%%%%%%%%%%%%%%%%%%%%%%%%%%%%%



\subsubsection*{Exercise 2.14.}
\addcontentsline{toc}{subsubsection}{Exercise 2.14.}
\emph{Show that $1+\sqrt{2}$ is a unit in $\mathbb{Z}[\sqrt{2}]$.
Use the powers of $1+\sqrt{2}$ to generate infinitely many solutions
to the diophantine equation $a^2 - 2b^2 = \pm 1$.
(It will be shown in Chapter 5 that all units in $\mathbb{Z}[\sqrt{2}]$
are of the form $\pm(1+\sqrt{2})^k$, $k \in \mathbb{Z}$.)} \\

Might assume to find nonnegative solutions to the Pell's equation $a^2 - 2b^2 = \pm 1$. \\



\emph{Proof.}
\begin{enumerate}
\item[(1)]
\emph{Show that $1+\sqrt{2}$ is a unit in $\mathbb{Z}[\sqrt{2}]$.}
There is $-1+\sqrt{2} \in \mathbb{Z}[\sqrt{2}]$
such that $$(1+\sqrt{2})(-1+\sqrt{2}) = 1 \in \mathbb{Z}[\sqrt{2}].$$
Hence $1+\sqrt{2}$ is a unit.
\item[(2)]
\emph{$N(a+b\sqrt{2}) = |a^2 - 2b^2|$ is a norm on $\mathbb{Z}[\sqrt{2}]$.}
To prove this, use the same argument as Exercise 1.1 and note that
$$N(a+b\sqrt{2}) = |(a+b\sqrt{2})(a-b\sqrt{2})|.$$
\item[(3)]
By (1)(2),
all $(1+\sqrt{2})^k$ with $k \geq 0$
are distinct solutions to the diophantine equation $a^2 - 2b^2 = \pm 1$.
Explicitly, let
\begin{align*}
  (a_0,b_0) &= (1,0), \\
  (a_1,b_1) &= (1,1), \\
  (a_2,b_2) &= (3,2), \\
  (a_3,b_3) &= (7,5), \\
  &\cdots \\
  (a_k,b_k) &= (a_{k-1}+2b_{k-1},a_{k-1}+b_{k-1}), \\
  &\cdots
\end{align*}
Note that all $(a_k,b_k)$ are distinct and satisfying $a_k^2 - 2b_k^2 = \pm 1$.
Hence we get infinitely many solutions to the Pell's equation $a^2 - 2b^2 = \pm 1$.
\end{enumerate}

\emph{Note.}
Suppose that all units in $\mathbb{Z}[\sqrt{2}]$
are of the form $\pm(1+\sqrt{2})^k$, $k \in \mathbb{Z}$.
Note that $(1+\sqrt{2})^k = (-1+\sqrt{2})^{-k}$.
Thus we can find all nonnegative solutions to the Pell's equation $a^2 - 2b^2 = \pm 1$
are exactly the same as (3).
$\Box$ \\



\subsubsection*{Supplement. (Exercise I.1.6 in J\"urgen Neukirch, \emph{Algebraic Number Theory})}
\addcontentsline{toc}{subsubsection}{Supplement.
(Exercise I.1.6 in J\"urgen Neukirch, \emph{Algebraic Number Theory})}
\emph{Show that the ring $\mathbb{Z}[\sqrt{d}] = \mathbb{Z} + \mathbb{Z}\sqrt{d}$,
for any squarefree rational integer $d > 1$, has infinitely many units.} \\



\emph{Proof.}
The proof is quoted from Proposition 17.5.2 in the book:
Ireland and Rosen, \emph{A Classical Introduction to Modern Number Theory, 2nd Ed.}
\begin{enumerate}
\item[(1)]
  Define the norm of $z = x + y\sqrt{d} \in \mathbb{Z}[\sqrt{d}]$ by
  $N(z) = z \overline{z}$ or
  \[
    N(x + y\sqrt{d})
    = \underbrace{(x + y\sqrt{d})}_{= z} \underbrace{(x - y\sqrt{d})}_{:= \overline{z}}
    = x^2 - dy^2.
  \]
  Note that a norm is multiplicative.
  Similar to Exercise I.1.1,
  $\alpha \in \mathbb{Z}[\sqrt{d}]$ is a unit if and only if $N(\alpha) = \pm 1$.

\item[(2)]
  To show $\mathbb{Z}[\sqrt{d}]$ has infinitely many units, it suffices to show
  the equation $x^2 - dy^2 = 1$ has infinitely many $(x,y)$ solutions.

\item[(3)]
  \emph{If $\xi$ is irrational then there are infinitely many rational numbers
  $\frac{x}{y}$, $(x,y) = 1$ such that $\abs{\frac{x}{y}-\xi} < \frac{1}{y^2}$.}
  It is followed by the pigeonhole principle.

\item[(4)]
  \emph{If $d$ is a positive squarefree integer then there is a constant $M := 2\sqrt{d}+1$
  such that $\abs{x^2 - dy^2} < M$ has infinitely many solutions over $\mathbb{Z}$.}
  Write $x^2 - dy^2 = (x + y\sqrt{d})(x - y\sqrt{d})$.
  By part (3), there exist infinitely many pairs of relatively prime integers $(x,y)$, $y > 0$
  satisfying $\abs{x - y\sqrt{d}} < \frac{1}{y}$.
  Hence
  \begin{align*}
    \abs{x^2 - dy^2}
    =& \: \abs{x + y\sqrt{d}}\abs{x - y\sqrt{d}} \\
    \leq & \: \left(\abs{x - y\sqrt{d}} + 2 y\sqrt{d} \right) \abs{x - y\sqrt{d}} \\
    \leq & \: 2 \sqrt{d} + 1.
  \end{align*}

\item[(5)]
  By part (4), there is an integer $m$ such that $x^2 - dy^2 = m$
  for infinitely many solutions over $\mathbb{Z}$.
  Here $m \neq 0$.
  We might assume $x, y > 0$ and $x$ components of solutions are distinct.

\item[(6)]
  The pigeonhole principle shows that there are two distinct solutions $(x_1,y_1)$, $(x_2,y_2)$
  with $x_1 \neq x_2$ such that
  \[
    x_1 \equiv x_2 \pmod{|m|},
    \qquad 
    y_1 \equiv y_2 \pmod{|m|}.
  \]
  Let $\alpha = x_1 - y_1\sqrt{d}$, $\beta = x_2 + y_2\sqrt{d}$ and $\gamma = \alpha\beta$.
  Hence
  \begin{align*}
    \gamma
    &= (x_1 - y_1\sqrt{d})(x_2 + y_2\sqrt{d}) \\
    &= \underbrace{(x_1 x_2 - d y_1 y_2)}_{\equiv 0 \pmod{|m|}}
        + \underbrace{(x_1 y_2 - x_2 y_1)}_{\equiv 0 \pmod{|m|}}\sqrt{d} \\
    &:= m(u + v\sqrt{d})
  \end{align*}
  for some $u + v\sqrt{d} \in \mathbb{Z}[\sqrt{d}]$.
  Taking norms of $\gamma = \alpha\beta$ gives $N(\gamma) = N(\alpha)N(\beta)$ or
  \[
    m^2 (u + v\sqrt{d}) = m^2.
  \]
  Hence $u + v\sqrt{d} = 1$.
  By construction of $x_1, x_2$, $v \neq 0$.
  Therefore the equation $x^2 - dy^2 = 1$ has one solution with $x, y > 0$.

\item[(7)]
  By part (6), we might take a unit $\varepsilon = x + y\sqrt{d} \in \mathbb{Z}[\sqrt{d}]$
  with $x, y > 0$.
  Note that $\varepsilon \geq 1 + \sqrt{d} > 1$ (over the ordered field $\mathbb{R}$).
  Hence there are infinitely many units
  \[
    \varepsilon, \varepsilon^2, \varepsilon^3, \ldots
  \]
  in $\mathbb{Z}[\sqrt{d}]$.
\end{enumerate}
$\Box$ \\



\emph{Note.}
  \emph{Furthermore, show that there is a unit $\varepsilon$ such that every unit has the form
  $\pm \varepsilon^n$, $n \in \mathbb{Z}$.} \\



\emph{Proof.}
\begin{enumerate}
\item[(1)]
  By the well-ordering principle,
  there is a unit $\varepsilon = x_1 + y_1\sqrt{d} \in \mathbb{Z}[\sqrt{d}]$ such that
  $x_1, y_1 > 0$ and $(x_1,y_1)$ is the smallest solution of $x^2 - dy^2 = \pm 1$ with $x, y > 0$.

\item[(2)]
  Now given any unit $\varepsilon' = x + y\sqrt{d}$, $x, y > 0$,
  it suffices to show that there is a positive integer $n$ such that $\varepsilon' = \varepsilon^n$.

\item[(3)]
  (Reductio ad absurdum)
  If not, there were a positive integer $n$ such that $\varepsilon^n < \varepsilon' < \varepsilon^{n+1}$.
  Hence $1 < \varepsilon^{-n} \varepsilon' < \varepsilon$.
  Say $\varepsilon^{-n} \varepsilon' := x' + y'\sqrt{d}$.
  As $\varepsilon^{-n} \varepsilon' > 1 > 0$, the inverse is satisfying $x' - y'\sqrt{d} > 0$.
  Hence $x' > 0$.

\item[(4)]
  As the inverse is satisfying $x' - y'\sqrt{d} < 1$, $y' \geq 0$.
  Note that $y' \neq 0$ (since $\varepsilon > 1$).
  Hence the existence of $\varepsilon^{-n} \varepsilon'$ contradicts the minimality of $\varepsilon$.

\item[(5)]
  Now suppose a unit $\varepsilon' = x + y\sqrt{d}$ is of the form $x > 0$, $y < 0$.
  Then $\varepsilon'^{-1} = x - y\sqrt{d} = \varepsilon^n$ for some positive integer $n$ by (2)(3)(4).
  Hence $\varepsilon' = \varepsilon^{-n}$ for some positive integer $n$.
  Other two cases of $\varepsilon' = x + y\sqrt{d}$ are similar.
  Therefore, every unit has the form $\pm \varepsilon^n$, $n \in \mathbb{Z}$.
\end{enumerate}
$\Box$ \\



\subsubsection*{Supplement. (Exercise I.1.7 in J\"urgen Neukirch, \emph{Algebraic Number Theory})}
\addcontentsline{toc}{subsubsection}{Supplement.
(Exercise I.1.7 in J\"urgen Neukirch, \emph{Algebraic Number Theory})}
\emph{Show that the ring $\mathbb{Z}[\sqrt{2}] = \mathbb{Z} + \mathbb{Z}\sqrt{2}$ is euclidean.
Show furthermore that its units are given by $\pm(1+\sqrt{2})^n$, $n \in \mathbb{Z}$,
and determine its prime elements.} \\



\emph{Proof.}
\begin{enumerate}
\item[(1)]
  \emph{Show that $\mathbb{Z}[\sqrt{2}]$ is euclidean with respect to the function
  $N: \mathbb{Z}[\sqrt{2}] \to \mathbb{N} \cup \{0\}$, $\alpha \mapsto \alpha \overline{\alpha}$.}
  For $\alpha, \beta \neq 0 \in \mathbb{Z}[\sqrt{2}]$,
  one has to find $\gamma, \rho \in \mathbb{Z}[\sqrt{2}]$
  such that
  \[
    \alpha = \gamma\beta + \rho,
    \qquad
    N(\rho) < N(\beta).
  \]

\item[(2)]
  Extend the norm function $N$ to $\mathbb{Q}[\sqrt{2}]$.
  Write
  \[
    \frac{\alpha}{\beta} = x + y\sqrt{2} \in \mathbb{Q}[\sqrt{2}].
  \]
  Take $\gamma = u + v\sqrt{2} \in \mathbb{Z}[\sqrt{2}]$
  such that $u, v$ are satisfying $\abs{u - x} \leq \frac{1}{2}$, $\abs{v - y} \leq \frac{1}{2}$.
  Now take $\rho = \alpha - \gamma\beta$.

\item[(3)]
  Hence,
  \[
    N\left(\frac{\alpha}{\beta} - \gamma\right)
    = (u-x)^2 + 2 (v-y)^2
    \leq \left(\frac{1}{2}\right)^{2} + 2 \cdot \left(\frac{1}{2}\right)^{2}
    < 1
  \]
  and thus
  \[
    N(\rho)
    = N(\alpha - \gamma\beta)
    = N(\beta) N\left(\frac{\alpha}{\beta} - \gamma\right)
    < N(\beta).
  \]

\item[(4)]
  \emph{Show that its units are given by $\pm(1+\sqrt{2})^n$, $n \in \mathbb{Z}$.}
  $\varepsilon = 1 + \sqrt{2} \in \mathbb{Z}[\sqrt{2}]$ is a unit such that
  $(1,1)$ is the smallest solution of $x^2 - 2y^2 = \pm 1$ with $x, y > 0$.
  By the note in Exercise I.1.6,
  all units are given by $\pm(1+\sqrt{2})^n$, $n \in \mathbb{Z}$.

\item[(5)]
  \emph{For all prime numbers $p \neq 2$, one has
  $p = a^2 - 2b^2$ ($a, b \in \mathbb{Z}$) if and only if $p \equiv 1, 7 \pmod 8$.}
  Similar to the proof of Proposition I.1.1,
  it suffices to show that a prime number $p \equiv 1, 7 \pmod 8$ of $\mathbb{Z}$
  does not remain a prime element in the ring $\mathbb{Z}[\sqrt{2}]$.
  (Reductio ad absurdum)
  Note that the congruence
  \[
    2 \equiv x^2 \pmod p
  \]
  admits a solution (by the law of quadratic reciprocity).
  Thus we have $p \mid x^2 - 2 = (x + \sqrt{2})(x - \sqrt{2})$.
  Hence $\frac{x}{p} \pm \frac{\sqrt{2}}{p} \in \mathbb{Z}[\sqrt{2}]$, which is absurd.

\item[(6)]
  \emph{The prime element $\pi$ of $\mathbb{Z}[\sqrt{2}]$, up to associated elements,
  are given as follows.}
  \begin{enumerate}
  \item[(i)]
    $\pi = \sqrt{2}$,

  \item[(ii)]
    \emph{$\pi = a + \sqrt{2}b$ with $a^2 - 2b^2 = p$, $p \equiv 1, 7 \pmod 8$},

  \item[(iii)]
    $\pi = p$, $p \equiv 3, 5 \pmod 8$.
  \end{enumerate}
  \emph{Here, $p$ denotes a prime number of $\mathbb{Z}$.}
  The proof is exactly the same as Theorem I.1.4.
\end{enumerate}
$\Box$ \\\\



%%%%%%%%%%%%%%%%%%%%%%%%%%%%%%%%%%%%%%%%%%%%%%%%%%%%%%%%%%%%%%%%%%%%%%%%%%%%%%%%



\subsubsection*{Exercise 2.15.}
\addcontentsline{toc}{subsubsection}{Exercise 2.15.}
\begin{enumerate}
\item[(a)]
\emph{Show that $\mathbb{Z}[\sqrt{-5}]$ contains no element whose norm is $2$ or $3$.}
\item[(b)]
\emph{Verify that $2 \cdot 3 = (1+\sqrt{-5})(1-\sqrt{-5})$ is an example
of non-unique factorization in the number ring $\mathbb{Z}[\sqrt{-5}]$.} \\
\end{enumerate}

\emph{Proof of (a).}
Since $N(a+b\sqrt{-5}) = a^2 + 5b^2 \equiv a^2 \equiv 0,1,4 \pmod{5}$,
there is no element whose norm is $2$ or $3$.
$\Box$ \\

\emph{Proof of (b).}
\begin{enumerate}
\item[(1)]
  \emph{Show that $2 \cdot 3 = (1+\sqrt{-5})(1-\sqrt{-5})$.}
  $$2 \cdot 3 = 6 \text{ and } (1+\sqrt{-5})(1-\sqrt{-5}) = 6.$$

\item[(2)]
  \emph{Show that $2$ is irreducible.}
  Suppose $2 = \alpha\beta$ where $\alpha, \beta \in \mathbb{Z}[\sqrt{-5}]$.
  Take norm to get
  \begin{align*}
    N(2) = N(\alpha)N(\beta)
    &\Longrightarrow
    4 = N(\alpha)N(\beta) \\
    &\Longrightarrow
    N(\alpha) = 1 \text{ or } N(\beta) = 1
      &\text{((1))} \\
    &\Longrightarrow
    \text{$\alpha$ or $\beta$ is unit}.
  \end{align*}

\item[(3)]
  \emph{Show that $3$ is irreducible.}
  Similar to (2).

\item[(4)]
  \emph{Show that $1 \pm \sqrt{-5}$ is irreducible.}
  Since $N(1 \pm \sqrt{-5}) = 2$ is prime, $1+\sqrt{-5}$ is irreducible.
\end{enumerate}
Hence $6$ has a non-unique factorization
in the number ring $\mathbb{Z}[\sqrt{-5}]$.
$\Box$ \\\\


%%%%%%%%%%%%%%%%%%%%%%%%%%%%%%%%%%%%%%%%%%%%%%%%%%%%%%%%%%%%%%%%%%%%%%%%%%%%%%%%



\subsubsection*{Exercise 2.16.}
\addcontentsline{toc}{subsubsection}{Exercise 2.16.}
\emph{Set $\alpha = \sqrt[4]{2}$.
Use the trace $T = T^{\mathbb{Q}[\alpha]}$ to show that $\sqrt{3} \not\in \mathbb{Q}[\alpha]$.
(Hint: Write $\sqrt{3} = a + b\alpha + c\alpha^2 + d\alpha^3$ and successively show that
$a = 0$; $b = 0$ (what is $T\left( \frac{\sqrt{3}}{\alpha} \right)$?); $c = 0$;
and finally obtain a contradiction.)} \\



\emph{Proof.}
\begin{enumerate}
\item[(1)]
  Let $K = \mathbb{Q}[\alpha]$.
  (Reductio ad absurdum)
  If $\sqrt{3} \in K$, then we can write $\sqrt{3} = a + b\alpha + c\alpha^2 + d\alpha^3$
  for some integers $a$, $b$, $c$ and $d$.

\item[(2)]
  Note that $K = \mathbb{Q}[\alpha] = \mathbb{Q}[\sqrt{3}]$ by assumption.
  Hence
  \begin{align*}
    & \: T^{\mathbb{Q}[\sqrt{3}]}(\sqrt{3})
        = T^{\mathbb{Q}[\alpha]}(a + b\alpha + c\alpha^2 + d\alpha^3) \\
    \Longrightarrow & \:
        0 = 4a \\
    \Longrightarrow & \:
        a = 0.
  \end{align*}
  So $\sqrt{3} = b\alpha + c\alpha^2 + d\alpha^3$.

\item[(3)]
  $\sqrt{3} = b\alpha + c\alpha^2 + d\alpha^3$ implies that
  \[
    \underbrace{\sqrt{3}\alpha^3}_{= \sqrt[4]{72}} = 2b + 2c\alpha + 2d\alpha^2.
  \]
  Since $\mathbb{Q}[\sqrt[4]{72}] \subseteq K$
  and $[\mathbb{Q}[\sqrt[4]{72}]:\mathbb{Q}] = [\mathbb{Q}[\sqrt[4]{2}]:\mathbb{Q}] = 4$,
  $K = \mathbb{Q}[\sqrt[4]{72}]$.
  Hence
  \begin{align*}
    & \: T^{\mathbb{Q}[\sqrt[4]{72}]}(\sqrt[4]{72})
        = T^{\mathbb{Q}[\alpha]}(2b + 2c\alpha + 2d\alpha^2) \\
    \Longrightarrow & \:
        0 = 8b \\
    \Longrightarrow & \:
        b = 0.
  \end{align*}
  So $\sqrt{3} = c\alpha^2 + d\alpha^3$.

\item[(4)]
  Similar to (3).
  $\sqrt{3} = c\alpha^2 + d\alpha^3$ implies that
  \[
    \underbrace{\sqrt{3}\alpha^2}_{= \sqrt{6}} = 2c + 2d\alpha.
  \]
  Since $\mathbb{Q}[\sqrt{6}] \subseteq K$
  and $[\mathbb{Q}[\sqrt{6}]:\mathbb{Q}] = [\mathbb{Q}[\sqrt{3}]:\mathbb{Q}] = 2$,
  $K = \mathbb{Q}[\sqrt{6}]$.
  Hence
  \begin{align*}
    T^{\mathbb{Q}[\sqrt{6}]}(\sqrt{6})
        = T^{\mathbb{Q}[\alpha]}(2c + 2d\alpha)
    \Longrightarrow
        0 = 8c
    \Longrightarrow
        c = 0.
  \end{align*}
  So $\sqrt{3} = d\alpha^3$.

\item[(5)]
  Similar to (3)(4), $d = 0$ and thus $\sqrt{3} = 0$, which is absurd.
\end{enumerate}
$\Box$ \\\\



%%%%%%%%%%%%%%%%%%%%%%%%%%%%%%%%%%%%%%%%%%%%%%%%%%%%%%%%%%%%%%%%%%%%%%%%%%%%%%%%


%%%%%%%%%%%%%%%%%%%%%%%%%%%%%%%%%%%%%%%%%%%%%%%%%%%%%%%%%%%%%%%%%%%%%%%%%%%%%%%%


%%%%%%%%%%%%%%%%%%%%%%%%%%%%%%%%%%%%%%%%%%%%%%%%%%%%%%%%%%%%%%%%%%%%%%%%%%%%%%%%



\subsubsection*{Exercise 2.19. (Vandermonde determinant)}
\addcontentsline{toc}{subsubsection}{Exercise 2.19. (Vandermonde determinant)}
\emph{Let $R$ be a commutative ring and fix elements $a_1, a_2, \ldots \in R$.
We will prove by induction that the Vandermonde determinant
\[
  \begin{vmatrix}
    1 & a_1 & \cdots & a_1^{n-1} \\
    \vdots & \vdots & \ddots & \vdots \\
    1 & a_n & \cdots & a_n^{n-1}
  \end{vmatrix}
\]
is equal to the product $\prod_{1 \leq r < s \leq n}(a_s - a_r)$.
Assuming that the result holds for some $n$, consider the determinant
\[
  \begin{vmatrix}
    1 & a_1 & \cdots & a_1^{n} \\
    \vdots & \vdots & \ddots & \vdots \\
    1 & a_n & \cdots & a_n^{n} \\
    1 & a_{n+1} & \cdots & a_{n+1}^{n}
  \end{vmatrix}.
\]
Show that this is equal to
\[
  \begin{vmatrix}
    1 & a_1 & \cdots & f(a_1) \\
    \vdots & \vdots & \ddots & \vdots \\
    1 & a_n & \cdots & f(a_n) \\
    1 & a_{n+1} & \cdots & f(a_{n+1})
  \end{vmatrix}
\]
for any monic polynomial $f$ over $R$ of degree $n$.
Then choose $f$ cleverly so that the determinant is easily calculated.} \\



\emph{Proof.}
\begin{enumerate}
\item[(1)]
  Let
  \[
    V_n =
    \begin{pmatrix}
      1 & a_1 & \cdots & a_1^{n-1} \\
      \vdots & \vdots & \ddots & \vdots \\
      1 & a_n & \cdots & a_n^{n-1}
    \end{pmatrix}
  \]
  be the Vandermonde matrix.
  We will apply the induction to show that $\det(V_n) = \prod_{1 \leq r < s \leq n}(a_s - a_r)$.

\item[(2)]
  Nothing to do for $n = 1, 2$.
  Now Assuming that the result holds for some $n$, consider the determinant
  \[
    \det(V_{n+1}) = \begin{vmatrix}
      1 & a_1 & \cdots & a_1^{n} \\
      \vdots & \vdots & \ddots & \vdots \\
      1 & a_n & \cdots & a_n^{n} \\
      1 & a_{n+1} & \cdots & a_{n+1}^{n}
    \end{vmatrix}.
  \]

\item[(3)]
  \emph{Show that
  \[
    \det(V_{n+1}) = \begin{vmatrix}
      1 & a_1 & \cdots & f(a_1) \\
      \vdots & \vdots & \ddots & \vdots \\
      1 & a_n & \cdots & f(a_n) \\
      1 & a_{n+1} & \cdots & f(a_{n+1})
    \end{vmatrix}
  \]
  for any monic polynomial $f$ over $R$ of degree $n$.}
  Note that $\det(V_{n+1})$ is unchanged by adding a multiple of one column of $V_{n+1}$
  to another column of $V_{n+1}$.
  In particular, we add a multiple of the $i$-th column of $V_{n+1}$ to
  the last column of $V_{n+1}$ for $i = 1, 2, \ldots, n$.
  Then we obtain the equation
  \[
    \det(V_{n+1}) = \begin{vmatrix}
      1 & a_1 & \cdots & f(a_1) \\
      \vdots & \vdots & \ddots & \vdots \\
      1 & a_n & \cdots & f(a_n) \\
      1 & a_{n+1} & \cdots & f(a_{n+1})
    \end{vmatrix}.
  \]

\item[(4)]
  In particular,
  we take
  \[
    f(x) = (x - a_1)(x - a_2) \cdots (x - a_n).
  \]
  Therefore
  \begin{align*}
    \det(V_{n+1})
    &= \begin{vmatrix}
      1 & a_1 & \cdots & 0 \\
      \vdots & \vdots & \ddots & \vdots \\
      1 & a_n & \cdots & 0 \\
      1 & a_{n+1} & \cdots & \prod_{1 \leq r \leq n}(a_{n+1} - a_r)
    \end{vmatrix} \\
    &= (-1)^{(n+1)+(n+1)}\prod_{1 \leq r \leq n}(a_{n+1} - a_r)
    \begin{vmatrix}
      1 & a_1 & \cdots & a_1^{n-1} \\
      \vdots & \vdots & \ddots & \vdots \\
      1 & a_n & \cdots & a_n^{n-1}
    \end{vmatrix} \\
    &= \prod_{1 \leq r \leq n}(a_{n+1} - a_r)
        \prod_{1 \leq r < s \leq n}(a_s - a_r) \\
    &= \prod_{1 \leq r < s \leq n+1}(a_s - a_r).
  \end{align*}
  By induction, the result is established.
\end{enumerate}
$\Box$ \\\\



%%%%%%%%%%%%%%%%%%%%%%%%%%%%%%%%%%%%%%%%%%%%%%%%%%%%%%%%%%%%%%%%%%%%%%%%%%%%%%%%



\subsubsection*{Exercise 2.20.}
\addcontentsline{toc}{subsubsection}{Exercise 2.20.}
\emph{Let $f$ be a monic irreducible polynomial over a number field $K$ and
let $\alpha$ be one of its roots in $\mathbb{C}$.
Show that $f'(\alpha) = \prod_{\beta \neq \alpha}(\alpha - \beta)$
with the product taken over all roots $\beta-\alpha$.
(Hint: Write $f(x) = (x-\alpha)g(x)$.)} \\



\emph{Proof.}
\begin{enumerate}
\item[(1)]
  Note that $f$ has no repeated roots in $\mathbb{C}$ by the irreducibility of $f$.
  So we can write
  \[
    f(x) = (x-\alpha)g(x) = (x-\alpha)\prod_{\beta \neq \alpha}(x - \beta).
  \]

\item[(2)]
  So
  \[
    f'(x) = g(x) + (x-\alpha)g'(x)
  \]
  by the Leibniz rule.
  Take $x = \alpha$ to get
  \[
    f'(\alpha) = g(\alpha) = \prod_{\beta \neq \alpha}(\alpha - \beta).
  \]
\end{enumerate}
$\Box$ \\\\



%%%%%%%%%%%%%%%%%%%%%%%%%%%%%%%%%%%%%%%%%%%%%%%%%%%%%%%%%%%%%%%%%%%%%%%%%%%%%%%%


%%%%%%%%%%%%%%%%%%%%%%%%%%%%%%%%%%%%%%%%%%%%%%%%%%%%%%%%%%%%%%%%%%%%%%%%%%%%%%%%


%%%%%%%%%%%%%%%%%%%%%%%%%%%%%%%%%%%%%%%%%%%%%%%%%%%%%%%%%%%%%%%%%%%%%%%%%%%%%%%%


%%%%%%%%%%%%%%%%%%%%%%%%%%%%%%%%%%%%%%%%%%%%%%%%%%%%%%%%%%%%%%%%%%%%%%%%%%%%%%%%



\subsubsection*{Exercise 2.28.}
\addcontentsline{toc}{subsubsection}{Exercise 2.28.}
\emph{Let $f(x) = x^3+ax+b$, $a$ and $b \in \mathbb{Z}$,
and assume $f$ is irreducible over $\mathbb{Q}$.
Let $\alpha$ be a root of $f$.}
\begin{enumerate}
\item[(a)]
  \emph{Show that $f'(\alpha) = -\frac{2a\alpha+3b}{\alpha}$.}

\item[(b)]
  \emph{Show that $2a\alpha+3b$ is a root of
  $$\left( \frac{x-3b}{2a} \right)^3 + a\left( \frac{x-3b}{2a} \right) + b.$$
  Use this to find $N_{\mathbb{Q}}^{\mathbb{Q}[\alpha]} (2a\alpha+3b)$.}

\item[(c)]
  \emph{Show that $\textrm{disc}(\alpha) = -(4a^3+27b^2)$.}

\item[(d)]
  \emph{Suppose $\alpha^3=\alpha+1$.
  Prove that $\{1,\alpha,\alpha^2\}$ is an integral basis for $\mathcal{O}_{\mathbb{Q}[\alpha]}$.
  (See Exercise 2.27(e).)
  Do the same if $\alpha^3+\alpha=1$.} \\
\end{enumerate}



\emph{Proof of (a).}
\begin{enumerate}
\item[(1)]
  \emph{Show that $\alpha \neq 0$.}
  If $\alpha$ were $0$, then $f(\alpha) = f(0) = b$.
  So $f(x) = x^3+ax = x(x^2+a)$ is reducible, contrary to the irreducibility of $f$.

\item[(2)]
  Since $\alpha$ be a root of $f$,
  $f(\alpha) = 0$,
  or $\alpha^3 + a\alpha + b = 0$,
  or $\alpha^3 = -a\alpha-b$.

\item[(3)]
  \begin{align*}
    f'(x) = 3x^2 + a
    &\Longrightarrow
    f'(\alpha) = 3\alpha^2 + a \\
    &\Longleftrightarrow
    \alpha f'(\alpha) = 3\alpha^3 + a\alpha
      &(\alpha \neq 0) \\
    &\Longleftrightarrow
    \alpha f'(\alpha) = 3(-a\alpha-b) + a\alpha
      &(\alpha^3 = -a\alpha-b) \\
    &\Longleftrightarrow
    \alpha f'(\alpha) = -2a\alpha-3b.
  \end{align*}
  So $f'(\alpha) = -\frac{2a\alpha+3b}{\alpha}$.
\end{enumerate}
$\Box$ \\



\emph{Proof of (b).}
\begin{enumerate}
\item[(1)]
  Since $\alpha^3 + a\alpha + b = 0$,
  \[
    \left( \frac{(2a\alpha+3b)-3b}{2a} \right)^3
    + a\left( \frac{(2a\alpha+3b)-3b}{2a} \right) + b = 0.
  \]
  That is, $2a\alpha+3b$ is a root of
  $\left( \frac{x-3b}{2a} \right)^3 + a\left( \frac{x-3b}{2a} \right) + b$.

\item[(2)]
  $N_{\mathbb{Q}}^{\mathbb{Q}[\alpha]}(2a\alpha+3b)$ is the product of three roots of
  $\left( \frac{x-3b}{2a} \right)^3 + a\left( \frac{x-3b}{2a} \right) + b$.
  Hence,
  \begin{align*}
    N_{\mathbb{Q}}^{\mathbb{Q}[\alpha]}(2a\alpha+3b)
    &= (2a)^3\left[ \left(\frac{-3b}{2a}\right)^3 + a \cdot \frac{-3b}{2a} + b \right] \\
    &= 8a^3\left[ \frac{-27b^3}{8a^3} - \frac{b}{2} \right] \\
    &= -27b^3-4a^3b.
  \end{align*}
\end{enumerate}
$\Box$ \\



\emph{Proof of (c).}
\begin{align*}
  \textrm{disc}(\alpha)
  &= (-1)^{\frac{n(n-1)}{2}} N_{\mathbb{Q}}^{\mathbb{Q}[\alpha]}(f'(\alpha))
    &\text{(Theorem 2.8)} \\
  &= - N_{\mathbb{Q}}^{\mathbb{Q}[\alpha]}\left( -\frac{2a\alpha+3b}{\alpha} \right)
    &\text{($n=3$ and (a))} \\
  &= \frac{N_{\mathbb{Q}}^{\mathbb{Q}[\alpha]}(2a\alpha+3b)}
    {N_{\mathbb{Q}}^{\mathbb{Q}[\alpha]}(\alpha)} \\
  &= \frac{-27b^3-4a^3b}{b}
    &\text{((b))} \\
  &= -27b^2-4a^3.
\end{align*}
$\Box$ \\



\emph{Proof of (d).}
\begin{enumerate}
\item[(1)]
  Write $\alpha^3 = \alpha + 1$ as $\alpha^3 - \alpha - 1 = 0$.
  Note that $f(x) = x^3 - x - 1$ is irreducible over $\mathbb{Q}$
  since $f(x)$ is irreducible over $\mathbb{Z}/3\mathbb{Z}$.
  So $\textrm{disc}(\alpha) = -23$ (by (c)).
  Since $\textrm{disc}(\alpha)$ is squarefree,
  the result is established (Exercise 2.27(e)).

\item[(2)]
  Similar to (1).
  Write $\alpha^3 + \alpha = 1$ as $\alpha^3 + \alpha - 1 = 0$.
  Note that $f(x) = x^3 + x - 1$ is irreducible over $\mathbb{Q}$
  since $f(x)$ is irreducible over $\mathbb{Z}/2\mathbb{Z}$.
  So $\textrm{disc}(\alpha) = -31$ (by (c)).
  Since $\textrm{disc}(\alpha)$ is squarefree,
  the result is established (Exercise 2.27(e)).
\end{enumerate}
$\Box$ \\\\



%%%%%%%%%%%%%%%%%%%%%%%%%%%%%%%%%%%%%%%%%%%%%%%%%%%%%%%%%%%%%%%%%%%%%%%%%%%%%%%%



\subsubsection*{Exercise 2.43.}
\addcontentsline{toc}{subsubsection}{Exercise 2.43.}
\emph{Let $f(x) = x^5+ax+b$, $a$ and $b \in \mathbb{Z}$,
and assume $f$ is irreducible over $\mathbb{Q}$.
Let $\alpha$ be a root of $f$.}
\begin{enumerate}
\item[(a)]
  \emph{Show that $\textrm{disc}(\alpha) = 4^4 a^5 + 5^4 b^4$. (Suggestion: See Exercise 2.28.)}

\item[(b)]
  \emph{Suppose $\alpha^5=\alpha+1$.
  Prove that $\mathcal{O}_{\mathbb{Q}[\alpha]} = \mathbb{Z}[\alpha]$.
  ($x^5 - x - 1$ is irreducible over $\mathbb{Q}$;
  this can be shown by reducing $\pmod{3}$.)} \\
\end{enumerate}



\emph{Proof of (a)(Exercise 2.28).}
\begin{enumerate}
\item[(1)]
\emph{Show that $f'(\alpha) = -\frac{4a\alpha+5b}{\alpha}$.}
  \begin{enumerate}
  \item[(a)]
    \emph{Show that $\alpha \neq 0$.}
    If $\alpha$ were $0$, then $f(\alpha) = f(0) = b$.
    So $f(x) = x^5+ax = x(x^4+a)$ is reducible, contrary to the irreducibility of $f$.

  \item[(b)]
    Since $\alpha$ be a root of $f$,
    $f(\alpha) = 0$,
    or $\alpha^5 + a\alpha + b = 0$,
    or $\alpha^5 = -a\alpha-b$.

  \item[(c)]
    \begin{align*}
      f'(x) = 5x^4 + a
      &\Longrightarrow
      f'(\alpha) = 5\alpha^4 + a \\
      &\Longleftrightarrow
      \alpha f'(\alpha) = 5\alpha^5 + a\alpha
        &(\alpha \neq 0) \\
      &\Longleftrightarrow
      \alpha f'(\alpha) = 5(-a\alpha-b) + a\alpha
        &(\alpha^5 = -a\alpha-b) \\
      &\Longleftrightarrow
      \alpha f'(\alpha) = -4a\alpha-5b.
    \end{align*}
    So $f'(\alpha) = -\frac{4a\alpha+5b}{\alpha}$.
  \end{enumerate}

\item[(2)]
  \emph{Show that $4a\alpha+5b$ is a root of
  $$\left( \frac{x-5b}{4a} \right)^5 + a\left( \frac{x-5b}{4a} \right) + b.$$
  Use this to show that
  $N_{\mathbb{Q}}^{\mathbb{Q}[\alpha]}(4a\alpha+5b) = -4^4a^5b-5^5b^5$.}
  \begin{enumerate}
  \item[(a)]
    Since $\alpha^5 + a\alpha + b = 0$,
    \[
      \left( \frac{(4a\alpha+5b)-5b}{4a} \right)^5
      + a\left( \frac{(4a\alpha+5b)-5b}{4a} \right) + b = 0.
    \]
    That is, $4a\alpha+5b$ is a root of
    $\left( \frac{x-5b}{4a} \right)^5 + a\left( \frac{x-5b}{4a} \right) + b$.

  \item[(b)]
    $N_{\mathbb{Q}}^{\mathbb{Q}[\alpha]}(4a\alpha+5b)$ is the product of $5$ roots of
    $\left( \frac{x-5b}{4a} \right)^5 + a\left( \frac{x-5b}{4a} \right) + b$.
    Hence,
    \begin{align*}
      N_{\mathbb{Q}}^{\mathbb{Q}[\alpha]}(4a\alpha+5b)
      &= (4a)^5\left[ \left(\frac{-5b}{4a}\right)^5 + a \cdot \frac{-5b}{4a} + b \right] \\
      &= 4^5a^5\left[ \frac{-5^5b^5}{4^5a^5} - \frac{b}{4} \right] \\
      &= -5^5b^5-4^4a^5b.
    \end{align*}
  \end{enumerate}

\item[(3)]
  \emph{Show that $\textrm{disc}(\alpha) = 4^4 a^5 + 5^4 b^4$.}
  \begin{align*}
    \textrm{disc}(\alpha)
    &= (-1)^{\frac{n(n-1)}{2}} N_{\mathbb{Q}}^{\mathbb{Q}[\alpha]}(f'(\alpha))
      &\text{(Theorem 2.8)} \\
    &= N_{\mathbb{Q}}^{\mathbb{Q}[\alpha]}\left( -\frac{4a\alpha+5b}{\alpha} \right)
      &\text{($n=5$ and (1))} \\
    &= -\frac{N_{\mathbb{Q}}^{\mathbb{Q}[\alpha]}(4a\alpha+5b)}
      {N_{\mathbb{Q}}^{\mathbb{Q}[\alpha]}(\alpha)} \\
    &= - \frac{-4^4a^5b-5^5b^5}{b}
      &\text{((2))} \\
    &= 4^4 a^5 + 5^4 b^4.
  \end{align*}
\end{enumerate}
$\Box$ \\



\emph{Proof of (b)(Exercise 2.28).}
  Write $\alpha^5 = \alpha + 1$ as $\alpha^5 - \alpha - 1 = 0$.
  Note that $f(x) = x^5 - x - 1$ is irreducible over $\mathbb{Q}$
  since $f(x)$ is irreducible over $\mathbb{Z}/3\mathbb{Z}$.
  So $\textrm{disc}(\alpha) = 881$ (by (a)).
  Since $\textrm{disc}(\alpha)$ is squarefree (a prime number),
  the result is established (Exercise 2.27(e)).
$\Box$ \\\\



%%%%%%%%%%%%%%%%%%%%%%%%%%%%%%%%%%%%%%%%%%%%%%%%%%%%%%%%%%%%%%%%%%%%%%%%%%%%%%%%



\subsubsection*{Exercise 2.45.}
\addcontentsline{toc}{subsubsection}{Exercise 2.45.}
\emph{Obtain a formula for $\textrm{disc}(\alpha)$ if $\alpha$ is a root of
an irreducible polynomial $x^n + ax + b$ over $\mathbb{Q}$.
Do the same for $x^n + ax^{n-1}+b$.} \\

Assume that $n \geq 2$. \\



\emph{Proof of $x^n + ax + b$ (Exercise 2.28).}
\begin{enumerate}
\item[(1)]
  \emph{Show that $f'(\alpha) = -\frac{(n-1)a\alpha+nb}{\alpha}$.}
  \begin{enumerate}
    \item[(a)]
      \emph{Show that $\alpha \neq 0$.}
      If $\alpha$ were $0$, then $f(\alpha) = f(0) = b$.
      So $f(x) = x^n+ax = x(x^{n-1}+a)$ is reducible, contrary to the irreducibility of $f$.

    \item[(b)]
      Since $\alpha$ be a root of $f$,
      $f(\alpha) = 0$,
      or $\alpha^n + a\alpha + b = 0$,
      or $\alpha^n = -a\alpha-b$.

    \item[(c)]
      \begin{align*}
        f'(x) = nx^{n-1} + a
        &\Longrightarrow
        f'(\alpha) = n\alpha^{n-1} + a \\
        &\Longleftrightarrow
        \alpha f'(\alpha) = n\alpha^n + a\alpha
          &(\alpha \neq 0) \\
        &\Longleftrightarrow
        \alpha f'(\alpha) = n(-a\alpha-b) + a\alpha
          &(\alpha^n = -a\alpha-b) \\
        &\Longleftrightarrow
        \alpha f'(\alpha) = -(n-1)a\alpha-nb.
        \end{align*}
      So $f'(\alpha) = -\frac{(n-1)a\alpha+nb}{\alpha}$.
  \end{enumerate}

\item[(2)]
  \emph{Let $\beta = (n-1)a\alpha+nb$.
  Show that $\beta$ is a root of
  \[
    \left( \frac{x-nb}{(n-1)a} \right)^n + a\left( \frac{x-nb}{(n-1)a} \right) + b.
  \]
  Use this to show that}
  \[
    N_{\mathbb{Q}}^{\mathbb{Q}[\alpha]}(\beta) = -(n-1)^{n-1}a^nb+(-1)^n n^n b^n.
  \]
  \begin{enumerate}
  \item[(a)]
    Since $\alpha^n + a\alpha + b = 0$,
    \[
      \left( \frac{\beta-nb}{(n-1)a} \right)^n
      + a\left( \frac{\beta-nb}{(n-1)a} \right) + b = 0.
    \]
    That is, $\beta$ is a root of
    $\left( \frac{x-nb}{(n-1)a} \right)^n + a\left( \frac{x-nb}{(n-1)a} \right) + b$.

  \item[(b)]
    $N_{\mathbb{Q}}^{\mathbb{Q}[\alpha]}(\beta)$ is the product of $n$ roots of
    $\left( \frac{x-nb}{(n-1)a} \right)^n + a\left( \frac{x-nb}{(n-1)a} \right) + b$.
    Hence,
    \begin{align*}
      N_{\mathbb{Q}}^{\mathbb{Q}[\alpha]}(\beta)
      &= ((n-1)a)^n\left[ \left(\frac{-nb}{(n-1)a}\right)^n
        + a \cdot \frac{-nb}{(n-1)a} + b \right] \\
      &= (n-1)^n a^n\left[ \frac{(-1)^n n^n b^n}{(n-1)^n a^n} - \frac{b}{n-1} \right] \\
      &= (-1)^n n^n b^n - (n-1)^{n-1} a^n b.
    \end{align*}
  \end{enumerate}

\item[(3)]
  \emph{Show that $\textrm{disc}(\alpha) = (-1)^{\frac{(n-1)(n-2)}{2}} (n-1)^{n-1}a^n
    + (-1)^{\frac{n(n-1)}{2}} n^n b^{n-1}$.}
  \begin{align*}
    \textrm{disc}(\alpha)
    &= (-1)^{\frac{n(n-1)}{2}} N_{\mathbb{Q}}^{\mathbb{Q}[\alpha]}(f'(\alpha))
      &\text{(Theorem 2.8)} \\
    &= (-1)^{\frac{n(n-1)}{2}} N_{\mathbb{Q}}^{\mathbb{Q}[\alpha]}
      \left( -\frac{(n-1)a\alpha+nb}{\alpha} \right)
      &\text{((1))} \\
    &= (-1)^{\frac{n(n-1)}{2}}(-1)^n
      \frac{N_{\mathbb{Q}}^{\mathbb{Q}[\alpha]}((n-1)a\alpha+nb)}
      {N_{\mathbb{Q}}^{\mathbb{Q}[\alpha]}(\alpha)} \\
    &= (-1)^{\frac{n(n-1)}{2}}(-1)^n \frac{-(n-1)^{n-1}a^nb+(-1)^n n^n b^n}{b}
      &\text{((2))} \\
    &= (-1)^{\frac{(n-1)(n-2)}{2}} (n-1)^{n-1}a^n
      + (-1)^{\frac{n(n-1)}{2}} n^n b^{n-1}.
  \end{align*}
\end{enumerate}
$\Box$ \\\\



%%%%%%%%%%%%%%%%%%%%%%%%%%%%%%%%%%%%%%%%%%%%%%%%%%%%%%%%%%%%%%%%%%%%%%%%%%%%%%%%
%%%%%%%%%%%%%%%%%%%%%%%%%%%%%%%%%%%%%%%%%%%%%%%%%%%%%%%%%%%%%%%%%%%%%%%%%%%%%%%%
%%%%%%%%%%%%%%%%%%%%%%%%%%%%%%%%%%%%%%%%%%%%%%%%%%%%%%%%%%%%%%%%%%%%%%%%%%%%%%%%
%%%%%%%%%%%%%%%%%%%%%%%%%%%%%%%%%%%%%%%%%%%%%%%%%%%%%%%%%%%%%%%%%%%%%%%%%%%%%%%%



\end{document}