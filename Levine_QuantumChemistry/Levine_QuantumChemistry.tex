\documentclass{article}
\usepackage{amsfonts}
\usepackage{amsmath}
\usepackage{amssymb}
\usepackage{centernot}
\usepackage{hyperref}
\usepackage[none]{hyphenat}
\usepackage{mathrsfs}
\usepackage{mathtools}
\usepackage{physics}
\usepackage{tikz-cd}
\parindent=0pt



\title{\textbf{Solutions to the book: \\ \emph{Ira N. Levine, Quantum Chemistry, 5th edition}}}
\author{Meng-Gen Tsai \\ plover@gmail.com}



\begin{document}
\maketitle
\tableofcontents



%%%%%%%%%%%%%%%%%%%%%%%%%%%%%%%%%%%%%%%%%%%%%%%%%%%%%%%%%%%%%%%%%%%%%%%%%%%%%%%%
%%%%%%%%%%%%%%%%%%%%%%%%%%%%%%%%%%%%%%%%%%%%%%%%%%%%%%%%%%%%%%%%%%%%%%%%%%%%%%%%



% Reference:



%%%%%%%%%%%%%%%%%%%%%%%%%%%%%%%%%%%%%%%%%%%%%%%%%%%%%%%%%%%%%%%%%%%%%%%%%%%%%%%%
%%%%%%%%%%%%%%%%%%%%%%%%%%%%%%%%%%%%%%%%%%%%%%%%%%%%%%%%%%%%%%%%%%%%%%%%%%%%%%%%



\newpage
\section*{Chapter 1: The Schr\"odinger Equation \\}
\addcontentsline{toc}{section}{Chapter 1: The Schr\"odinger Equation}



\subsubsection*{Problem 1.1.}
\addcontentsline{toc}{subsubsection}{Problem 1.1.}
\begin{enumerate}
\item[(a)]
\emph{Calculate the energy of one photon of infrared radiation whose wavelength
is $1064$ nm.}
\item[(b)]
\emph{An Nd:YAG laser emits a pulse of $1064$-nm radiation of average power
$5 \times 10^6 W$ and duration $2 \times 10^{-8}s$.
Find the number of photons emitted in this pulse. (Recall that $1W = 1J/s$.)}
\end{enumerate}

\emph{Solution of (a).}
\begin{align*}
E_{\text{photon}}
&= h \nu
  &\text{(Equation (1.1))} \\
&= \frac{hc}{\lambda}
  &\text{(Equation (1.3))} \\
&= \frac{(6.626 \times 10^{-34}Js)(2.998 \times 10^8 m/s)}{1064 \times 10^{-9} m} \\
&= 1.867 \times 10^{-19} J.
\end{align*}
$\Box$ \\

\emph{Solution of (b).}
Total energy in one pulse is $E = (5 \times 10^6 W)(2 \times 10^{-8}s) = 0.1J$.
By (a), the energy of one photon $E_{\text{photon}} = 1.867 \times 10^{-19} J$.
So the number of photons is
$$n = \frac{E}{E_{\text{photon}}}
= \frac{0.1J}{1.867 \times 10^{-19} J}
= 5 \times 10^{17}.$$
$\Box$ \\\\



%%%%%%%%%%%%%%%%%%%%%%%%%%%%%%%%%%%%%%%%%%%%%%%%%%%%%%%%%%%%%%%%%%%%%%%%%%%%%%%%
%%%%%%%%%%%%%%%%%%%%%%%%%%%%%%%%%%%%%%%%%%%%%%%%%%%%%%%%%%%%%%%%%%%%%%%%%%%%%%%%



\newpage
\section*{Chapter 3: Operators \\}
\addcontentsline{toc}{section}{Chapter 3: Operators}



\subsubsection*{Problem 3.6.}
\addcontentsline{toc}{subsubsection}{Problem 3.6.}
\emph{Prove that $\hat{A} + \hat{B} = \hat{B} + \hat{A}$. } \\

Two operators $\hat{A}$ and $\hat{B}$ are said to be equal if
$\hat{A}f = \hat{B}f$ for all functions $f$. \\

\emph{Proof.}
$$(\hat{A} + \hat{B})f = \hat{A}f + \hat{B}f
= \hat{B}f + \hat{A}f = (\hat{B} + \hat{A})f$$
holds for any function $f$.
By definition,
$\hat{A} + \hat{B} = \hat{B} + \hat{A}$.
$\Box$ \\\\



%%%%%%%%%%%%%%%%%%%%%%%%%%%%%%%%%%%%%%%%%%%%%%%%%%%%%%%%%%%%%%%%%%%%%%%%%%%%%%%%



\subsubsection*{Problem 3.7.}
\addcontentsline{toc}{subsubsection}{Problem 3.7.}
\emph{Let $\hat{D} = d/dx$.
Verify that $(\hat{D}+x)(\hat{D}-x) = \hat{D}^2-x^2-1$. } \\

\emph{Proof.}
\begin{align*}
((\hat{D}+x)(\hat{D}-x))f
&= (\hat{D}+x)((\hat{D}-x)f) \\
&= (\hat{D}+x)(f'-xf) \\
&= (f'-xf)' + x(f'-xf) \\
&= (f''-f-xf') + (xf'-x^2f) \\
&= f''-f-x^2f \\
&= (\hat{D}^2-x^2-1)f
\end{align*}
holds for any function $f$.
By definition,
$(\hat{D}+x)(\hat{D}-x) = \hat{D}^2-x^2-1$.
$\Box$ \\\\



%%%%%%%%%%%%%%%%%%%%%%%%%%%%%%%%%%%%%%%%%%%%%%%%%%%%%%%%%%%%%%%%%%%%%%%%%%%%%%%%



\subsubsection*{Problem 3.27.}
\addcontentsline{toc}{subsubsection}{Problem 3.27.}
\emph{Evaluate the commutators}
\begin{enumerate}
\item[(a)]
$[\hat{x}, \hat{p}_x]$;
\item[(b)]
$[\hat{x}, \hat{p}_x^2]$;
\item[(c)]
$[\hat{x}, \hat{p}_y]$;
\item[(d)]
$[\hat{x}, \hat{V}(x,y,z)]$;
\item[(e)]
\emph{$[\hat{x}, \hat{H}]$, where the Hamiltonian operator is
$$\hat{H} = -\frac{\hbar}{2m}\left(
\frac{\partial^2}{\partial x^2} + \frac{\partial^2}{\partial y^2} + \frac{\partial^2}{\partial z^2}
\right) + V(x,y,z);$$}
\item[(f)]
$[\hat{x}\hat{y}\hat{z}, \hat{p}_x^2]$. \\
\end{enumerate}

\emph{Proof of (a).}
\begin{align*}
[\hat{x}, \hat{p}_x]f
&= (\hat{x}\hat{p}_x - \hat{p}_x \hat{x})f \\
&= (\hat{x}\hat{p}_x)f - (\hat{p}_x \hat{x})f \\
&= (\hat{x})\left( \frac{\hbar}{i}\frac{\partial f}{\partial x} \right)
  - (\hat{p}_x)(xf) \\
&= x \frac{\hbar}{i}\frac{\partial f}{\partial x}
  - \frac{\hbar}{i}\left( f + x \frac{\partial f}{\partial x} \right) \\
&= - \frac{\hbar}{i} f
\end{align*}
holds for any function $f$.
By definition,
$[\hat{x}, \hat{p}_x] = - \frac{\hbar}{i}$.
$\Box$ \\

\emph{Proof of (b).}
\begin{align*}
[\hat{x}, \hat{p}_x^2]f
&= (\hat{x}\hat{p}_x^2 - \hat{p}_x^2 \hat{x})f \\
&= (\hat{x}\hat{p}_x^2)f - (\hat{p}_x^2 \hat{x})f \\
&= (\hat{x} \hat{p}_x)\left( \frac{\hbar}{i}\frac{\partial f}{\partial x} \right)
  - (\hat{p}_x^2)(xf) \\
&= (\hat{x})\left( \frac{\hbar}{i}\frac{\hbar}{i} \frac{\partial^2 f}{\partial x^2} \right)
  - (\hat{p}_x) \frac{\hbar}{i} \left( f + x \frac{\partial f}{\partial x} \right) \\
&= x \left( \frac{\hbar}{i}\frac{\hbar}{i} \frac{\partial^2 f}{\partial x^2} \right)
  - \frac{\hbar}{i} \frac{\hbar}{i} \left(
    \frac{\partial f}{\partial x} + \frac{\partial f}{\partial x}
  + x \frac{\partial^2 f}{\partial x^2} \right) \\
&= - \frac{\hbar}{i} \frac{\hbar}{i} \cdot 2 \frac{\partial f}{\partial x} \\
&= \left( 2 \hbar \frac{\partial}{\partial x} \right) f
\end{align*}
holds for any function $f$.
By definition,
$[\hat{x}, \hat{p}_x^2] = 2 \hbar \frac{\partial}{\partial x}$.
$\Box$ \\

\emph{Proof of (c).}
\begin{align*}
[\hat{x}, \hat{p}_y]f
&= (\hat{x}\hat{p}_y - \hat{p}_y \hat{x})f \\
&= (\hat{x}\hat{p}_y)f - (\hat{p}_y \hat{x})f \\
&= (\hat{x})\left( \frac{\hbar}{i}\frac{\partial f}{\partial y} \right)
  - (\hat{p}_y)(xf) \\
&= x \frac{\hbar}{i}\frac{\partial f}{\partial x}
  - \frac{\hbar}{i} \cdot x \frac{\partial f}{\partial y} \\
&= 0
\end{align*}
holds for any function $f$.
By definition,
$[\hat{x}, \hat{p}_y] = 0$.
$\Box$ \\

\emph{Proof of (d).}
\begin{align*}
[\hat{x}, \hat{V}(x,y,z)]f
&= (\hat{x}\hat{V}(x,y,z) - \hat{V}(x,y,z) \hat{x})f \\
&= (\hat{x}\hat{V}(x,y,z))f - (\hat{V}(x,y,z) \hat{x})f \\
&= \hat{x}(V(x,y,z)f) - \hat{V}(x,y,z)(xf) \\
&= x V(x,y,z)f - V(x,y,z) xf \\
&= 0
\end{align*}
holds for any function $f$.
By definition,
$[\hat{x}, \hat{V}(x,y,z)] = 0$.
$\Box$ \\

\emph{Proof of (e).}
\begin{enumerate}
\item[(1)]
Given any function $f$,
\begin{align*}
\frac{\partial^2}{\partial x^2}(fx)
&= \frac{\partial}{\partial x}\left( x\frac{\partial f}{\partial x} + f \right) \\
&= \frac{\partial f}{\partial x} + x\frac{\partial^2 f}{\partial x^2}
  + \frac{\partial f}{\partial x} \\
&= 2\frac{\partial f}{\partial x} + x\frac{\partial^2 f}{\partial x^2}, \\
\frac{\partial^2}{\partial y^2}(fx)
&= x \frac{\partial^2 f}{\partial y^2}, \\
\frac{\partial^2}{\partial z^2}(fx)
&= x \frac{\partial^2 f}{\partial z^2}.
\end{align*}
\item[(2)]
\begin{align*}
(\hat{H}\hat{x})f
&= (\hat{H})(xf) \\
&= -\frac{\hbar}{2m}
    \left(
      \frac{\partial^2}{\partial x^2}(xf)
      + \frac{\partial^2}{\partial y^2}(xf)
      + \frac{\partial^2}{\partial z^2}(xf)
    \right) + V(x,y,z)xf \\
&= -\frac{\hbar}{2m}
    \left(
      2 \frac{\partial f}{\partial x} + x\frac{\partial^2 f}{\partial x^2}
      + x \frac{\partial^2 f}{\partial y^2}
      + x \frac{\partial^2 f}{\partial z^2}
    \right) + V(x,y,z)xf \\
&= -\frac{\hbar}{m} \frac{\partial f}{\partial x} + (\hat{x}\hat{H})f
\end{align*}
\item[(3)]
\begin{align*}
[\hat{x}, \hat{H}]f
&= (\hat{x}\hat{H} - \hat{H} \hat{x})f \\
&= (\hat{x}\hat{H})f - (\hat{H}\hat{x})f \\
&= \frac{\hbar}{m} \frac{\partial f}{\partial x}
\end{align*}
holds for any function $f$.
By definition,
$[\hat{x}, \hat{H}] = \frac{\hbar}{m} \frac{\partial}{\partial x}$.
\end{enumerate}
$\Box$ \\

\emph{Proof of (f).}
Similar to (b),
$[\hat{x}\hat{y}\hat{z}, \hat{p}_x^2] = 2 \hbar yz \frac{\partial}{\partial x}$.
$\Box$ \\\\



%%%%%%%%%%%%%%%%%%%%%%%%%%%%%%%%%%%%%%%%%%%%%%%%%%%%%%%%%%%%%%%%%%%%%%%%%%%%%%%%



\subsubsection*{Problem 3.33.}
\addcontentsline{toc}{subsubsection}{Problem 3.33.}
\emph{Prove the multiple-integral identity
\[
  \int_{-\infty}^{\infty}\int_{-\infty}^{\infty}\int_{-\infty}^{\infty}
    F(x)G(y)H(z)dx dy dz
  = \int_{-\infty}^{\infty} F(x) dx
    \int_{-\infty}^{\infty} G(y) dy
    \int_{-\infty}^{\infty} H(z) dz.
\]}

\emph{Proof.}
Write $\int = \int_{-\infty}^{\infty}$.
\begin{align*}
  &\int\int\int
    F(x)G(y)H(z)dx dy dz \\
  =&
  \int\int \left( \int F(x)\underbrace{G(y)H(z)}_{\text{constant w.r.t $x$}}dx \right) dy dz \\
  =&
  \int\int G(y)H(z) \underbrace{\int F(x)dx}_{\text{constant w.r.t $y$ and $z$}} dy dz \\
  =&
  \int F(x)dx \int\int G(y)H(z) dy dz \\
  =& \int F(x) dx\int G(y) dy\int H(z) dz.
    &\text{(Similar arguments)}
\end{align*}
$\Box$ \\\\



%%%%%%%%%%%%%%%%%%%%%%%%%%%%%%%%%%%%%%%%%%%%%%%%%%%%%%%%%%%%%%%%%%%%%%%%%%%%%%%%
%%%%%%%%%%%%%%%%%%%%%%%%%%%%%%%%%%%%%%%%%%%%%%%%%%%%%%%%%%%%%%%%%%%%%%%%%%%%%%%%



\end{document}