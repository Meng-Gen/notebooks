\documentclass{article}
\usepackage{amsfonts}
\usepackage{amsmath}
\usepackage{amssymb}
\usepackage{hyperref}
\usepackage[none]{hyphenat}
\usepackage{mathrsfs}
\usepackage{physics}
\parindent=0pt

\def\upint{\mathchoice%
    {\mkern13mu\overline{\vphantom{\intop}\mkern7mu}\mkern-20mu}%
    {\mkern7mu\overline{\vphantom{\intop}\mkern7mu}\mkern-14mu}%
    {\mkern7mu\overline{\vphantom{\intop}\mkern7mu}\mkern-14mu}%
    {\mkern7mu\overline{\vphantom{\intop}\mkern7mu}\mkern-14mu}%
  \int}
\def\lowint{\mkern3mu\underline{\vphantom{\intop}\mkern7mu}\mkern-10mu\int}

\begin{document}



\textbf{\Large Chapter 1: The Schr\"odinger Equation} \\\\



\emph{Author: Meng-Gen Tsai} \\
\emph{Email: plover@gmail.com} \\\\



%%%%%%%%%%%%%%%%%%%%%%%%%%%%%%%%%%%%%%%%%%%%%%%%%%%%%%%%%%%%%%%%%%%%%%%%%%%%%%%%



\textbf{Problem 1.1.}
\begin{enumerate}
\item[(a)]
\emph{Calculate the energy of one photon of infrared radiation whose wavelength
is $1064$ nm.}
\item[(b)]
\emph{An Nd:YAG laser emits a pulse of $1064$-nm radiation of average power
$5 \times 10^6 W$ and duration $2 \times 10^{-8}s$.
Find the number of photons emitted in this pulse. (Recall that $1W = 1J/s$.)}
\end{enumerate}

\emph{Solution of (a).}
\begin{align*}
E_{\text{photon}}
&= h \nu
  &\text{(Equation (1.1))} \\
&= \frac{hc}{\lambda}
  &\text{(Equation (1.3))} \\
&= \frac{(6.626 \times 10^{-34}Js)(2.998 \times 10^8 m/s)}{1064 \times 10^{-9} m} \\
&= 1.867 \times 10^{-19} J.
\end{align*}
$\Box$ \\

\emph{Solution of (b).}
Total energy in one pulse is $E = (5 \times 10^6 W)(2 \times 10^{-8}s) = 0.1J$.
By (a), the energy of one photon $E_{\text{photon}} = 1.867 \times 10^{-19} J$.
So the number of photons is
$$n = \frac{E}{E_{\text{photon}}}
= \frac{0.1J}{1.867 \times 10^{-19} J}
= 5 \times 10^{17}.$$
$\Box$ \\\\



%%%%%%%%%%%%%%%%%%%%%%%%%%%%%%%%%%%%%%%%%%%%%%%%%%%%%%%%%%%%%%%%%%%%%%%%%%%%%%%%
%%%%%%%%%%%%%%%%%%%%%%%%%%%%%%%%%%%%%%%%%%%%%%%%%%%%%%%%%%%%%%%%%%%%%%%%%%%%%%%%



\end{document}