\documentclass{article}
\usepackage{amsfonts}
\usepackage{amsmath}
\usepackage{amssymb}
\usepackage{hyperref}
\usepackage{mathrsfs}
\parindent=0pt

\def\upint{\mathchoice%
    {\mkern13mu\overline{\vphantom{\intop}\mkern7mu}\mkern-20mu}%
    {\mkern7mu\overline{\vphantom{\intop}\mkern7mu}\mkern-14mu}%
    {\mkern7mu\overline{\vphantom{\intop}\mkern7mu}\mkern-14mu}%
    {\mkern7mu\overline{\vphantom{\intop}\mkern7mu}\mkern-14mu}%
  \int}
\def\lowint{\mkern3mu\underline{\vphantom{\intop}\mkern7mu}\mkern-10mu\int}

\begin{document}

\textbf{\Large Chapter 1: Rings and Ideals} \\\\



\emph{Author: Meng-Gen Tsai} \\
\emph{Email: plover@gmail.com} \\\\



\textbf{Exercise 1.1}
\emph{Let $x$ be a nilpotent element of $A$.
Show that $1+x$ is a unit of $A$.
Deduce that the sum of a nilpotent element and a unit is a unit.} \\

\emph{Proof.}
\begin{enumerate}
\item[(1)]
Suppose $x^m = 0$ for some odd integer $m \geq 0$.
Then
$$1 = 1+x^m = (1+x)(1-x+x^2-\cdots+(-1)^{m-1}x^{m-1}),$$
or $1+x$ is a unit.
\item[(2)]
If $u$ is any unit and $x$ is any nilpotent,
$u + x= u \cdot (1 + u^{-1}x)$ is a product of two units
(using that $u^{-1}x$ is nilpotent and applying (1))
and hence a unit again.
\end{enumerate}
$\Box$ \\\\



\end{document}