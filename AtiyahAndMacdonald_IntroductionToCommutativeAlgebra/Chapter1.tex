\documentclass{article}
\usepackage{amsfonts}
\usepackage{amsmath}
\usepackage{amssymb}
\usepackage{hyperref}
\usepackage{mathrsfs}

\parindent=0pt

\def\upint{\mathchoice%
    {\mkern13mu\overline{\vphantom{\intop}\mkern7mu}\mkern-20mu}%
    {\mkern7mu\overline{\vphantom{\intop}\mkern7mu}\mkern-14mu}%
    {\mkern7mu\overline{\vphantom{\intop}\mkern7mu}\mkern-14mu}%
    {\mkern7mu\overline{\vphantom{\intop}\mkern7mu}\mkern-14mu}%
  \int}
\def\lowint{\mkern3mu\underline{\vphantom{\intop}\mkern7mu}\mkern-10mu\int}

\begin{document}

\textbf{\Large Chapter 1: Rings and Ideals} \\\\



\emph{Author: Meng-Gen Tsai} \\
\emph{Email: plover@gmail.com} \\\\



% https://spaces.ac.cn/usr/uploads/2017/07/4208763092.pdf
% https://www.math.arizona.edu/~cais/371Page/homework/371s1.pdf
% https://math.mit.edu/~jhirsh/notes2_zariski.pdf
% http://yyao.gsucreate.org/math831/831-2.pdf



\textbf{Exercise 1.1.}
\emph{Let $x$ be a nilpotent element of $A$.
Show that $1+x$ is a unit of $A$.
Deduce that the sum of a nilpotent element and a unit is a unit.} \\

\emph{Proof.}
\begin{enumerate}
\item[(1)]
Suppose $x^m = 0$ for some odd integer $m \geq 0$.
Then
$$1 = 1+x^m = (1+x)(1-x+x^2-\cdots+(-1)^{m-1}x^{m-1}),$$
or $1+x$ is a unit.
\item[(2)]
If $u$ is any unit and $x$ is any nilpotent,
$u + x= u \cdot (1 + u^{-1}x)$ is a product of two units
(using that $u^{-1}x$ is nilpotent and applying (1))
and hence a unit again.
\end{enumerate}
$\Box$ \\

\emph{Proof (Proposition 1.9).}
\begin{enumerate}
\item[(1)]
\emph{The nilradical is a subset of the Jacobson radical.}
\begin{enumerate}
\item[(a)]
The nilradical $\mathfrak{N}$ of $A$ is the intersection of all the prime ideals of $A$
by Proposition 1.8.
\item[(b)]
The Jacobson radical $\mathfrak{J}$ of $A$ is the intersection of all the maximal ideals of $A$
by definition.
\end{enumerate}
\item[(2)]
By Proposition 1.9,
$x \in \mathfrak{J}$ if and only if
$1-xy$ is a unit in $A$ for all $y \in A$.
So $1+x = 1 - (-x) \cdot 1$ is a unit in $A$
since $x$ is a nilpotent and $\mathfrak{J}$ is an ideal.
\end{enumerate}
$\Box$ \\\\



\textbf{Exercise 1.2.}
\emph{Let $A$ be a ring and
let $A[x]$ be the ring of polynomials in an indeterminate $x$,
with coefficients in $A$.
Let $f = a_0 + a_1 x + \cdots + a_n x^n \in A[x]$.
Prove that}
\begin{enumerate}
\item[(i)]
\emph{$f$ is a unit in $A[x]$ if and only if
$a_0$ is a unit in $A$ and
$a_1, ..., a_n$ are nilpotent.
(Hint: If $b_0 + b_1 x + \cdots + b_m x^m$ is the inverse of $f$,
prove by induction on $r$ that $a_n^{r+1} b_{m-r} = 0$.
Hence show that $a_n$ is nilpotent, and then use Exercise 1.1.)}
\item[(ii)]
\emph{$f$ is nilpotent if and only if
$a_0, a_1, ..., a_n$ are nilpotent.}
\item[(iii)]
\emph{$f$ is a zero-divisor if and only if
there exists $a \neq 0$ such that $af = 0$.
(Hint: Choose a polynomial $g = b_0 + b_1 x + \cdots + b_m x^m$
of least degree $m$ such that $fg = 0$.
Then $a_n b_m = 0$, hence $a_n g = 0$
(because $a_n g$ annihilates $f$ and has degree $< m$).
Now show by induction that $a_{n-r}g = 0$ $(0 \leq r \leq n)$.)}
\item[(iv)]
\emph{$f$ is said to be primitive if $(a_0, a_1, ..., a_n) = (1)$.
Prove that if $f, g \in A[x]$, then $fg$ is primitive if and only if
$f$ and $g$ are primitive.} \\
\end{enumerate}

\emph{Proof of (i).}
\begin{enumerate}
\item[(1)]
$(\Longleftarrow)$ holds by Exercise 1.1.
\item[(2)]
$(\Longrightarrow)$
There exists the inverse $g$ of $f$, say $g = b_0 + b_1 x + \cdots + b_m x^m$
satisfying
$1 = fg$.
Clearly, $1 = a_0 b_0$, or $a_0$ is a unit in $A$.
Also,
\begin{align*}
0
&= a_n b_m, \\
0
&= a_n b_{m-1} + a_{n-1} b_m, \\
0
&= a_n b_{m-2} + a_{n-1} b_{m-1} + a_{n-2} b_m, \\
& \cdots
\end{align*}
A direct computing shows that
\begin{align*}
0
&= a_n^{1} b_m, \\
0
&= a_n (a_n b_{m-1} + a_{n-1} b_m) \\
&= a_n^{2} b_{m-1} + a_{n-1} a_n b_m \\
&= a_n^{2} b_{m-1}, \\
0
&= a_n^{2} (a_n b_{m-2} + a_{n-1} b_{m-1} + a_{n-2} b_m) \\
&= a_n^{3} b_{m-2} + a_{n-1} a_n^{2} b_{m-1} + a_{n-2} a_n^{2} b_m \\
&= a_n^{3} b_{m-2}, \\
& \cdots
\end{align*}
So we might have $a_n^{r+1} b_{m-r} = 0$ for $r = 0, 1, 2, ..., m$.
\item[(3)]
\emph{Show that $a_n^{r+1} b_{m-r} = 0$ for $r = 0, 1, 2, ..., m$
by induction on $r$.}
\begin{enumerate}
\item[(a)]
As $r = 0$, $a_n b_m = 0$ by comparing the coefficient of $fg = 1$ at $x^{n+m}$.
\item[(b)]
For any $r > 0$, comparing the coefficient of $fg = 1$ at $x^{n+m-r}$,
$$0 = a_n b_{m-r} + a_{n-1} b_{m-r+1} + \cdots + a_{n-r} b_m.$$
Multiplying by $a_n^r$ on the both sides,
\begin{align*}
0
&= a_n^{r+1} b_{m-r} + a_{n-1} a_n^{r} b_{m-r+1} + \cdots + a_{n-r} a_n^{r} b_m \\
&= a_n^{r+1} b_{m-r}.
\end{align*}
by the induction hypothesis.
\end{enumerate}
\item[(4)]
\emph{$a_n$ is a nilpotent.}
Putting $r = m$ in $a_n^{r+1} b_{m-r} = 0$ and get $a_n^{m+1} b_0 = 0$.
Notice that $b_0$ is a unit, $a_n^{m+1} = 0$, or $a_n$ is a nilpotent.
\item[(5)]
Consider $f - a_n x^n = a_0 + a_1 x + \cdots + a_{n-1} x^{n-1}$,
a polynomial $\in A[x]$ of degree $n-1$.
Note that $f$ is a unit and $a_n x^n$ is a nilpotent.
By Exercise 1.1, $f - a_n x^n$ is a unit too.
Applying the (2)(3)(4) again, $a_{n-1}$ is a nilpotent as $n-1 > 0$,
that is, applying descending induction on $n$ then yields the desired property.
\end{enumerate}
$\Box$ \\

\emph{Proof of (ii).}
\begin{enumerate}
\item[(1)]
$(\Longleftarrow)$ holds since the nilradical of any ring is an ideal.
\item[(2)]
$(\Longrightarrow)$ $f^N = 0$ for some $N > 0$.
So $0 = f^N = a_n^N x^{nN} + \cdots + a_0^N$.
Comparing the coefficient in the leading term $x^{nN}$ leads to $a_n^N = 0$,
or $a_n$ is a nilpotent.
\item[(3)]
Consider $f - a_n x^n = a_0 + a_1 x + \cdots + a_{n-1} x^{n-1}$,
a polynomial $\in A[x]$ of degree $n-1$.
Note that $f$ and $a_n x^n$ are nilpotent.
$f - a_n x^n$ is a nilpotent too.
Similar to step (5) in the proof of (i),
applying descending induction on $n$ then yields the desired property.
\end{enumerate}
$\Box$ \\

\emph{Proof of (iii).}
\begin{enumerate}
\item[(1)]
$(\Longleftarrow)$ holds trivially.
\item[(2)]
$(\Longrightarrow)$
Pick a polynomial $g = b_0 + b_1 x + \cdots + b_m x^m$
of least degree $m$ such that $fg = 0$.
Especially, $a_n b_m = 0$.
\item[(3)]
Consider
\begin{align*}
a_n g
&= a_n b_0 + \cdots + a_n b_{m-1} x^{m-1} + a_n b_m x^m \\
&= a_n b_0 + \cdots + a_n b_{m-1} x^{m-1}
\end{align*}
(since $a_n b_m = 0$).
$a_n g$ is a polynomial over $A$ of having degree strictly less than $m$.
Notice that $f \cdot (a_n g) = a_n \cdot (fg)= 0$.
By minimality of $m$, $a_n g = 0$.
\item[(4)]
Induction on the degree $n$ of $f$.
\begin{enumerate}
\item[(a)]
As $n = 0$, $f = a_0$. There exists $b_m \neq 0$ such that $b_m f = b_m a_0 = 0$ by (2).
\item[(b)]
For any zero-divisor $f$ of degree $n$,
there is a polynomial $g = b_0 + b_1 x + \cdots + b_m x^m$
of least degree $m$ such that $fg = 0$. By (2)(3),
\begin{align*}
(f - a_n x^n) \cdot g
&= fg - a_n x^n g \\
&= 0 - 0 \\
&= 0.
\end{align*}
That is, $f - a_n x^n$ is a zero-divisor of degree $n-1$.
By the induction hypothesis,
there exists $b_m \neq 0$ such that $b_{m}(f - a_n x^n) = 0$.
So $b_m f = b_{m}(f - a_n x^n) + b_m a_n x^n = 0 + 0 = 0$.
\item[(c)]
By (a)(b), $(\Longrightarrow)$ holds by mathematical induction.
\end{enumerate}
\end{enumerate}
$\Box$ \\

\emph{Proof of (iv).}
Note that
\begin{enumerate}
\item[(1)]
$f \notin \mathfrak{m}[x]$ for any maximal ideal $\mathfrak{m}$ of $A$
if and only if $f$ is primitive.
\item[(2)]
For any maximal ideal $\mathfrak{m}$ of $A$,
$A/\mathfrak{m}$ is a field (or an integral domain).
\item[(3)]
$A[x]$ is an integral domain if $A$ is an integral domain.
\item[(4)]
$A[x]/\mathfrak{m}[x] \cong (A/\mathfrak{m})[x]$ as a ring isomorphism.
\end{enumerate}
Hence,
\begin{align*}
f, g:\text{ primitive}
&\Longleftrightarrow
f, g \notin \mathfrak{m}[x] \text{ for any maximal ideal } \mathfrak{m} \\
&\Longleftrightarrow
f, g \neq 0 \text{ in } (A/\mathfrak{m})[x] \text{ for any maximal ideal } \mathfrak{m} \\
&\Longleftrightarrow
fg \neq 0 \text{ in } (A/\mathfrak{m})[x] \text{ for any maximal ideal } \mathfrak{m} \\
&\Longleftrightarrow
fg \notin \mathfrak{m}[x] \text{ for any maximal ideal } \mathfrak{m} \\
&\Longleftrightarrow
fg:\text{ primitive}.
\end{align*}
$\Box$ \\\\



\textbf{Exercise 1.4.}
\emph{In the ring $A[x]$, the Jacobson radical is equal to the nilradical.} \\

\emph{Proof.}
\begin{enumerate}
\item[(1)]
The nilradical $\mathfrak{N}$ is a subset of the Jacobson radical $\mathfrak{J}$.
It suffices to show that $\mathfrak{J} \subseteq \mathfrak{N}$.
\item[(2)]
Given any $f \in \mathfrak{J}$. By Proposition 1.9,
$f \in \mathfrak{J}$ if and only if
$1 - fy$ is a unit in $A[x]$ for all $y \in A[x]$.
Especially, pick $y = x \in A[x]$ and then $1 - xf$ is a unit in $A[x]$.
\item[(3)]
By Exercise 1.2 (i), all coefficients of $f$ are nilpotent.
By Exercise 1.2 (ii), $f$ is nilpotent, or $f \in \mathfrak{N}$.
\end{enumerate}
$\Box$ \\\\


\textbf{Exercise 1.7.}
\emph{Let $A$ be a ring in which every element satisfies
$x^n = x$ for some $n > 1$ (depending on $x$).
Show that every prime ideal in $A$ is maximal.} \\

\emph{Proof.}
It suffices to show that
\emph{for any prime ideal $\mathfrak{p}$ in $A$, $A/\mathfrak{p}$ is a field.}
\begin{enumerate}
\item[(1)]
Take any $0 \neq \overline{x} \in A/\mathfrak{p}$,
which is represented by $x \in A-\mathfrak{p}$.
By assumption there exists $n \geq 2$ such that $x^n = x$.
So $\overline{x}^n = \overline{x}$ or $\overline{x}(\overline{x}^{n-1} - 1) = 0$.
\item[(2)]
Since $\mathfrak{p}$ is prime, $A/\mathfrak{p}$ is a integral domain.
That is, $\overline{x} = 0$ (impossible) or $\overline{x}^{n-1} - 1 = 0$.
Write $\overline{x} \cdot \overline{x}^{n-2} = 1$ in $A/\mathfrak{p}$.
So $\overline{x}^{n-2}$ is an inverse of $\overline{x} \neq 0$ in $A/\mathfrak{p}$,
which implies that $A/\mathfrak{p}$ is a field (since $\overline{x}$ is arbitrary).
\item[(3)]
$A/\mathfrak{p}$ is a field if and only if $\mathfrak{p}$ is maximal.
\end{enumerate}
$\Box$ \\\\



\textbf{\large The prime spectrum of a ring} \\\\



\textbf{Exercise 1.15.}
\emph{Let $A$ be a ring and let $X$ be the set of all prime ideals of $A$.
For each subset $E$ of $A$,
let $V(E)$ denote the set of all prime ideals of $A$ which contain $E$.
Prove that}
\begin{enumerate}
\item[(i)]
\emph{if $\mathfrak{a}$ is the ideal generated by $E$,
then $V(E) = V(\mathfrak{a}) = V(r(\mathfrak{a}))$.}
\item[(ii)]
\emph{$V(0) = X$, $V(1) = \varnothing$.}
\item[(iii)]
\emph{if $(E_i)_{i \in I}$ is any family of subsets of $A$,
then
$$V\left( \bigcup_{i \in I}E_i \right) = \bigcap_{i \in I} V(E_i).$$}
\item[(iv)]
\emph{$V(\mathfrak{a} \cap \mathfrak{b})
= V(\mathfrak{a} \mathfrak{b})
= V(\mathfrak{a}) \cup V(\mathfrak{b})$
for any ideals $\mathfrak{a}$, $\mathfrak{b}$ of $A$.}
\end{enumerate}

\emph{The results show that the sets $V(E)$ satisfy
the axioms for closed sets in a topological space.
The resulting topology is called the Zariski topology.
The topological space $X$ is called the prime spectrum of $A$,
and is written $\text{Spec}(A)$.} \\

Note that if $E_1 \subseteq E_2$,
then $V(E_1) \supseteq V(E_2)$. \\

\emph{Proof of (i).}
\begin{enumerate}
\item[(1)]
\emph{Show that $V(E) = V(\mathfrak{a})$.}
  \begin{enumerate}
  \item[(a)]
  \emph{Show that $V(E) \subseteq V(\mathfrak{a})$.}
  Given any $\mathfrak{p} \in V(E)$, $\mathfrak{p} \supseteq E$.
  For any $a \in \mathfrak{a}$,
  since $\mathfrak{a}$ is generated by $E$,
  we can write $a$ as a finite sum
  $a = \sum \alpha \beta$ where $\alpha \in A$ and $\beta \in E$.
  Since $E \subseteq \mathfrak{p}$, all $\beta \in \mathfrak{p}$.
  Since $\mathfrak{p}$ is an ideal,
  $a = \sum \alpha \beta \in \mathfrak{p}$.
  That is, $\mathfrak{p} \supseteq \mathfrak{a}$,
  or $\mathfrak{p} \in V(\mathfrak{a})$.
  \item[(b)]
  \emph{$V(E) \supseteq V(\mathfrak{a})$} since $\mathfrak{a} \supseteq E$.
  \end{enumerate}
\item[(2)]
\emph{Show that $V(\mathfrak{a}) = V(r(\mathfrak{a}))$.}
  \begin{enumerate}
  \item[(a)]
  \emph{Show that $V(\mathfrak{a}) \subseteq V(r(\mathfrak{a}))$.}
  Given any $\mathfrak{p} \in V(\mathfrak{a})$,
  \begin{align*}
  \mathfrak{p} \in V(\mathfrak{a})
  &\Longrightarrow \mathfrak{p} \supseteq \mathfrak{a} \\
  &\Longrightarrow \mathfrak{p} \supseteq \text{the intersection of the primes ideals }
  \mathfrak{p} \supseteq \mathfrak{a} \\
  &\Longrightarrow \mathfrak{p} \supseteq r(\mathfrak{a}) \text{ (by Proposition 1.14)}\\
  &\Longrightarrow \mathfrak{p} \in V(r(\mathfrak{a})).
  \end{align*}
  \item[(b)]
  \emph{$V(\mathfrak{a}) \supseteq V(r(\mathfrak{a}))$}
  since $r(\mathfrak{a}) \supseteq \mathfrak{a}$.
  \end{enumerate}
\end{enumerate}
$\Box$ \\

\emph{Proof of (ii).}
\begin{enumerate}
\item[(1)]
\emph{$V(1) = \varnothing$} since no prime ideal contains $1$ by definition.
\item[(2)]
\emph{$V(0) = X$} since $0$ is in every ideal (especially in every prime ideal).
\end{enumerate}
$\Box$ \\

\emph{Proof of (iii).}
\begin{align*}
\mathfrak{p} \in V\left( \bigcup_{i \in I}E_i \right)
&\Longleftrightarrow \mathfrak{p} \supseteq \bigcup_{i \in I}E_i \\
&\Longleftrightarrow \mathfrak{p} \supseteq E_i \text{ for all } i \in I \\
&\Longleftrightarrow \mathfrak{p} \in V(E_i) \text{ for all } i \in I \\
&\Longleftrightarrow \mathfrak{p} \in \bigcap_{i \in I} V(E_i).
\end{align*}
$\Box$ \\

\textbf{Lemma.}
\emph{For any $\mathfrak{p} \supseteq \mathfrak{a} \mathfrak{b}$,
$\mathfrak{p} \supseteq \mathfrak{a}$ or $\mathfrak{p} \supseteq \mathfrak{b}$.} \\

\emph{Proof of Lemma.}
\begin{enumerate}
\item[(1)] If $\mathfrak{p} \supseteq \mathfrak{a}$. We are done.
\item[(2)] If $\mathfrak{p} \not\supseteq \mathfrak{a}$,
there exists $a \in \mathfrak{a} - \mathfrak{p}$.
So for any $b \in \mathfrak{b}$, $b \in \mathfrak{p}$
since $ab \in \mathfrak{ab} \subseteq \mathfrak{p}$ and $\mathfrak{p}$ is a prime ideal,
that is, $\mathfrak{p} \supseteq \mathfrak{b}$.
\end{enumerate}
By (1)(2), $\mathfrak{p} \supseteq \mathfrak{a}$ or $\mathfrak{p} \supseteq \mathfrak{b}$.
$\Box$ \\

\emph{Proof of (iv).}
\begin{enumerate}
\item[(1)]
\emph{Show that $V(\mathfrak{a} \cap \mathfrak{b}) = V(\mathfrak{a} \mathfrak{b})$.}
  \begin{enumerate}
  \item[(a)]
  \emph{$V(\mathfrak{a} \cap \mathfrak{b}) \subseteq V(\mathfrak{a} \mathfrak{b})$}
  since $\mathfrak{a} \mathfrak{b} \subseteq \mathfrak{a} \cap \mathfrak{b}$.
  \item[(b)]
  \emph{Show that $V(\mathfrak{a} \cap \mathfrak{b}) \supseteq V(\mathfrak{a} \mathfrak{b})$.}
  Given any $\mathfrak{p} \in V(\mathfrak{a} \mathfrak{b})$,
  $\mathfrak{p} \supseteq \mathfrak{a} \mathfrak{b}$.
  By Lemma, $\mathfrak{p} \supseteq \mathfrak{a}$ or $\mathfrak{p} \supseteq \mathfrak{b}$.
  Notice that $\mathfrak{a} \supseteq \mathfrak{a \cap b}$
  and $\mathfrak{b} \supseteq \mathfrak{a \cap b}$.
  In any case, $\mathfrak{p} \supseteq \mathfrak{a \cap b}$,
  $\mathfrak{p} \in V(\mathfrak{a} \cap \mathfrak{b})$.
  \end{enumerate}
\item[(2)]
\emph{Show that $V(\mathfrak{a} \mathfrak{b}) = V(\mathfrak{a}) \cup V(\mathfrak{b})$.}
  \begin{enumerate}
  \item[(a)]
  \emph{Show that $V(\mathfrak{a} \mathfrak{b}) \subseteq V(\mathfrak{a}) \cup V(\mathfrak{b})$.}
  Given any $\mathfrak{p} \in V(\mathfrak{a} \mathfrak{b})$,
  $\mathfrak{p} \supseteq \mathfrak{a} \mathfrak{b}$.
  By Lemma,
  $\mathfrak{p} \supseteq \mathfrak{a}$ or $\mathfrak{p} \supseteq \mathfrak{b}$,
  $\mathfrak{p} \in V(\mathfrak{a})$ or $\mathfrak{p} \in V(\mathfrak{b})$,
  $\mathfrak{p} \in V(\mathfrak{a}) \cup V(\mathfrak{b})$.
  \item[(b)]
  \emph{Show that $V(\mathfrak{a} \mathfrak{b}) \supseteq V(\mathfrak{a}) \cup V(\mathfrak{b})$.}
  Given any $\mathfrak{p} \in V(\mathfrak{a}) \cup V(\mathfrak{b})$,
  $\mathfrak{p} \in V(\mathfrak{a})$ or $\mathfrak{p} \in V(\mathfrak{b})$,
  $\mathfrak{p} \supseteq \mathfrak{a}$ or $\mathfrak{p} \supseteq \mathfrak{b}$.
  Notice that $\mathfrak{a} \supseteq \mathfrak{ab}$
  and $\mathfrak{b} \supseteq \mathfrak{ab}$.
  In any cases, $\mathfrak{p} \supseteq \mathfrak{ab}$,
  or $\mathfrak{p} \in V(\mathfrak{ab}$).
  \end{enumerate}
\end{enumerate}
$\Box$ \\\\



\textbf{Exercise 1.17.}
\emph{For each $f \in A$,
let $X_f$ denote the complement of $V(f)$ in $X = \text{Spec}(A)$.
The sets $X_f$ are open.
Show that they form a basis of open sets for the Zariski topology, and that}
\begin{enumerate}
\item[(i)]
\emph{$X_f \cap X_g = X_{fg}$.}
\item[(ii)]
\emph{$X_f = \varnothing \Longleftrightarrow f$ is nilpotent.}
\item[(iii)]
\emph{$X_f = X \Longleftrightarrow f$ is a unit.}
\item[(iv)]
\emph{$X_f = X_g \Longleftrightarrow r((f)) = r((g))$.}
\item[(v)]
\emph{$X$ is quasi-compact (compact), that is,
every open covering of $X$ has a finite subcovering.}
\item[(vi)]
\emph{More generally, each $X_f$ is quasi-compact.}
\item[(vii)]
\emph{An open subset of $X$ is quasi-compact if and only if
it is a finite union of sets $X_f$.}
\end{enumerate}

\emph{The sets $X_f$ are called basic open sets of
$X = \text{Spec}(A)$.} \\

\emph{(Hint: To prove (v), remark that it is enough to consider a covering of $X$
by basic open sets $X_{f_i} (i \in I)$.
Show that the $f_i$ generate the unit ideal and hence that
there is an equation of the form
$$1 = \sum_{i \in J} g_i f_i \:\:\:\: (g_i \in A)$$
where $J$ is some finite subset of $I$.
Then the $X_{f_i} (i \in J)$ cover $X$.)} \\

\emph{Proof of basis.}
It is equivalent to Exercise 1.15 (iii).
Given any open set $O$ in $X$.
Write $O = X - V(\mathfrak{a})$ for some ideal $\mathfrak{a}$ of $A$.
Since
$$V(\mathfrak{a})
= V \left( \bigcup_{f \in \mathfrak{a}} (f) \right)
= \bigcap_{f \in \mathfrak{a}} V(f),$$
we have
$$O
= X - V(\mathfrak{a})
= X - \bigcap_{f \in \mathfrak{a}} V(f)
= \bigcup_{f \in \mathfrak{a}} (X - V(f))
= \bigcup_{f \in \mathfrak{a}} X_f,$$
or any open set is a union of basic open sets.
$\Box$ \\

\emph{Proof of (i).}
$X_f \cap X_g = X_{fg} \Longleftrightarrow V(f) \cup V(g) = V(fg)$ holds by Exercise 1.15 (iv).
$\Box$ \\

\emph{Proof of (ii).}
\begin{align*}
X_f = \varnothing
&\Longleftrightarrow V(f) = X \\
&\Longleftrightarrow f \in \mathfrak{p}
\text{ for all prime ideal $\mathfrak{p}$ of $A$} \\
&\Longleftrightarrow f \in \mathfrak{N},
\text{the nilradical of $A$ (Proposition 1.8)} \\
&\Longleftrightarrow f \text{ is nilpotent (Proposition 1.7)}
\end{align*}
$\Box$ \\

\emph{Proof of (ii) (Using (iv)).}
\begin{align*}
X_f = \varnothing
&\Longleftrightarrow X_f = X_0
  &\text{(Exercise 15(ii))} \\
&\Longleftrightarrow r(f) = r(0)
  &\text{((iv))} \\
&\Longleftrightarrow f \in r(f) = r(0)
  & \\
&\Longleftrightarrow f^m = 0 \text{ for some $m > 0$}
  & \\
&\Longleftrightarrow f \text{ is nilpotent}
  &
\end{align*}
$\Box$ \\

\emph{Proof of (iii).}
\begin{align*}
X_f = X
&\Longleftrightarrow V(f) = \varnothing \\
&\Longleftrightarrow f \not\in \mathfrak{p}
\text{ for all prime ideal $\mathfrak{p}$ of $A$} \\
&\Longleftrightarrow f \text{ is unit (Corollary 1.5)}
\end{align*}
$\Box$ \\

\emph{Proof of (iii) (Using (iv)).}
\begin{align*}
X_f = X
&\Longleftrightarrow X_f = X_1
  &\text{(Exercise 15(ii))} \\
&\Longleftrightarrow r(f) = r(1)
  &\text{((iv))} \\
&\Longleftrightarrow f \in r(f) = r(1)
  & \\
&\Longleftrightarrow f^m = 1 \text{ for some $m > 0$}
  & \\
&\Longleftrightarrow f \text{ is unit}
  &
\end{align*}
$\Box$ \\

\emph{Proof of (iv).}
\begin{enumerate}
\item[(1)]
\emph{Show that
$X_f \subseteq X_g \Longleftrightarrow r((f)) \subseteq r((g))$.}
Actually,
\begin{align*}
X_f \subseteq X_g
&\Longrightarrow V(f) \supseteq V(g) \\
&\Longrightarrow \{ \mathfrak{p} \in \text{Spec}(A) : \mathfrak{p} \supseteq (f) \}
  \supseteq \{ \mathfrak{p} \in \text{Spec}(A) : \mathfrak{p} \supseteq (g) \} \\
&\Longrightarrow \bigcap_{(f) \subseteq \mathfrak{p} \in \text{Spec}(A)} \mathfrak{p}
  \subseteq \bigcap_{(g) \subseteq \mathfrak{p} \in \text{Spec}(A)} \mathfrak{p}  \\
&\overset{1.14}{\Longrightarrow} r(f) \subseteq r(g) \\
&\Longrightarrow V(r(f)) \supseteq V(r(g)) \\
&\Longrightarrow V(f) \supseteq V(g) \\
&\Longrightarrow X_f \subseteq X_g.
\end{align*}
\item[(2)]
By (1),
\begin{align*}
X_f \subseteq X_g &\Longleftrightarrow r((f)) \subseteq r((g)), \\
X_f \supseteq X_g &\Longleftrightarrow r((f)) \supseteq r((g)).
\end{align*}
Hence,
$$X_f = X_g \Longleftrightarrow r((f)) = r((g)).$$
\end{enumerate}
$\Box$ \\

\emph{Proof of (v).}
By (iii), it is a special case of (vi) as $X_f = X_1 = X$.
$\Box$ \\

\emph{Proof of (vi).}
Notice that it is enough to consider a covering of $X_f$
by basic open sets $X_{f_i} (i \in I)$.
\begin{enumerate}
\item[(1)]
Since $X_f$ is covered by $X_{f_i} (i \in I)$,
\begin{align*}
X_f = \bigcup_{i \in I} X_{f_i}
&\Longrightarrow X - V(f) = \bigcup_{i \in I} X - V(f_i) \\
&\Longrightarrow V(f) = \bigcap_{i \in I} V(f_i) \\
&\Longrightarrow V(f) = V\left( \sum_{i \in I} f_i \right) \\
&\Longrightarrow r(f) = r\left( \sum_{i \in I} f_i \right).
\end{align*}
Hence, $f \in r(f) = r\left( \sum_{i \in I} f_i \right)$ can be expressed as
$$f^m = \sum_{j \in J} g_j f_j$$
where $J$ is a finite subset of $I$ and $g_j \in A$.
That is, $f^m \in \sum_{j \in J} f_j$.
\item[(2)]
\emph{Show that $V\left( \sum_{j \in J} f_j \right) = V(f)$.}
\begin{enumerate}
  \item[(a)]
  $(\subseteq)$ For any prime ideal $\mathfrak{p} \supseteq \sum_{j \in J} f_j$,
  $f^m \in \mathfrak{p}$ or $f \in \mathfrak{p}$ (since $\mathfrak{p}$ is prime).
  So $\mathfrak{p} \supseteq (f)$,
  or $V\left( \sum_{j \in J} f_j \right) \subseteq V(f)$.
  \item[(b)]
  $(\supseteq)$
  $$\sum_{j \in J} f_j \subseteq \sum_{i \in I} f_i
  \Longrightarrow
  V\left( \sum_{j \in J} f_j \right) \supseteq V\left( \sum_{i \in I} f_i \right) = V(f).$$
\end{enumerate}
\item[(3)]
Therefore, $X_f$ is covered by finite subcovering $\{X_{f_j}\} (j \in J)$.
\end{enumerate}
$\Box$ \\

\emph{Proof of (vii).}
\begin{enumerate}
\item[(1)]
($\Longrightarrow$)
Given an open subset $O$.
Since $X_f$ form a basis of open sets,
$$O = \bigcup_{f \in \mathfrak{a}} X_f \text{ for some ideal $\mathfrak{a}$ of $A$}$$
Especially, $\{X_f\}_{f \in \mathfrak{a}}$ is an open covering of $O$.
Since $O$ is quasi-compact, there exists a finite subcovering $\{X_f\}_{f \in J}$ of $O$,
where $J$ is a finite subset of $\mathfrak{a}$ (as a set).
That is,
$O = \bigcup_{f \in J} X_f$ is a finite union of sets $X_f$.
\item[(2)]
($\Longleftarrow$)
Since $X_f$ is quasi-compact, any finite union of quasi-compact sets is quasi-compact again.
\end{enumerate}
$\Box$ \\\\


\end{document}