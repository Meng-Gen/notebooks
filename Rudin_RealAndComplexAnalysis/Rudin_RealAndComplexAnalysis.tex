\documentclass{article}
\usepackage{amsfonts}
\usepackage{amsmath}
\usepackage{amssymb}
\usepackage{centernot}
\usepackage{hyperref}
\usepackage[none]{hyphenat}
\usepackage{mathrsfs}
\usepackage{mathtools}
\usepackage{physics}
\usepackage{tikz-cd}
\parindent=0pt



\title{\textbf{Solutions to the book: \\ \emph{Rudin, Real and Complex Analysis, 2nd edition}}}
\author{Meng-Gen Tsai \\ plover@gmail.com}



\begin{document}
\maketitle
\tableofcontents



%%%%%%%%%%%%%%%%%%%%%%%%%%%%%%%%%%%%%%%%%%%%%%%%%%%%%%%%%%%%%%%%%%%%%%%%%%%%%%%%
%%%%%%%%%%%%%%%%%%%%%%%%%%%%%%%%%%%%%%%%%%%%%%%%%%%%%%%%%%%%%%%%%%%%%%%%%%%%%%%%



% Reference:



%%%%%%%%%%%%%%%%%%%%%%%%%%%%%%%%%%%%%%%%%%%%%%%%%%%%%%%%%%%%%%%%%%%%%%%%%%%%%%%%
%%%%%%%%%%%%%%%%%%%%%%%%%%%%%%%%%%%%%%%%%%%%%%%%%%%%%%%%%%%%%%%%%%%%%%%%%%%%%%%%



\newpage
\section*{Chapter 3: $L^p$-Spaces \\}
\addcontentsline{toc}{section}{Chapter 3: $L^p$-Spaces}



\subsubsection*{Exercise 3.3.}
\addcontentsline{toc}{subsubsection}{Exercise 3.3.}
\emph{Assume that $\varphi$ is a continuous real function on $(a,b)$ such that
\[
  \varphi\left(\frac{x+y}{2}\right)
  \leq
  \frac{1}{2} \varphi(x) + \frac{1}{2} \varphi(y)
\]
for all $x$ and $y \in (a,b)$.
Prove that $\varphi$ is convex.
(The conclusion does not follow if continuity is omitted from the hypotheses.)} \\



\emph{Proof.}
\begin{enumerate}
\item[(1)]
  \emph{Show that
  $$\varphi\left( \frac{x_1 + \cdots + x_n}{n} \right)
  \leq \frac{\varphi(x_1) + \cdots + \varphi(x_n)}{n}$$
  whenever $a < x_i < b$ $(1 \leq i \leq n)$.}
  Apply Cauchy induction and use the same argument in proving the AM-GM inequality.
  As $n = 1, 2$, the inequality holds by assumption.
  Suppose $n = 2^k$ $(k \geq 1)$ the inequality holds.
  As $n = 2^{k+1}$,
  \begin{align*}
    &\varphi\left( \frac{x_1 + \cdots + x_{2^{k+1}}}{2^{k+1}} \right) \\
    =& \varphi\left( \frac{1}{2} \left(\frac{x_1 + \cdots + x_{2^k}}{2^k}
      + \frac{x_{2^k+1} + \cdots + x_{2^{k+1}}}{2^k}\right) \right) \\
    \leq& \frac{1}{2}
      \left(
        \varphi\left(\frac{x_1 + \cdots + x_{2^k}}{2^k} \right)
        + \varphi\left(\frac{x_{2^k+1} + \cdots + x_{2^{k+1}}}{2^k} \right)
      \right) \\
    \leq& \frac{1}{2}
      \left(
        \frac{\varphi(x_1) + \cdots + \varphi(x_{2^k})}{2^k}
        + \frac{\varphi(x_{2^k+1}) + \cdots + \varphi(x_{2^{k+1}})}{2^k}
      \right) \\
    =& \frac{\varphi(x_1) + \cdots + \varphi(x_{2^k})
      + \varphi(x_{2^k+1}) + \cdots + \varphi(x_{2^{k+1}})}{2^{k+1}} \\
    =& \frac{\varphi(x_1) + \cdots + \varphi(x_{2^{k+1}})}{2^{k+1}}.
  \end{align*}
  As $n$ is not a power of $2$,
  then it is certainly less than some natural power of $2$, say $n < 2^m$ for some $m$.
  Let
  $$x_{n+1} = \cdots = x_{2^m} = \frac{x_1 + \cdots + x_n}{n} = \alpha.$$
  Then by the induction hypothesis,
  \begin{align*}
    \varphi(\alpha)
    &= \varphi\left( \frac{x_1 + \cdots + x_n + \alpha + \cdots + \alpha}{2^m} \right) \\
    &\leq \frac{\varphi(x_1) + \cdots + \varphi(x_n) + \varphi(\alpha) + \cdots + \varphi(\alpha)}{2^m} \\
    &\leq \frac{\varphi(x_1) + \cdots + \varphi(x_n) + (2^m - n)\varphi(\alpha)}{2^m}, \\
    2^m \varphi(\alpha)
    &\leq \varphi(x_1) + \cdots + \varphi(x_n) + (2^m - n)\varphi(\alpha), \\
    n \varphi(\alpha)
    &\leq \varphi(x_1) + \cdots + \varphi(x_n),
  \end{align*}
  or $\varphi\left( \frac{1}{n} (x_1 + \cdots + x_n) \right)
  \leq \frac{1}{n}(\varphi(x_1) + \cdots \varphi(x_n))$.

\item[(2)]
  Hence,
  $$\varphi(\lambda x + (1 - \lambda) y) \leq \lambda \varphi(x) + (1 - \lambda) \varphi(y)$$
  for any rational $\lambda$ in $(0,1)$.
  (Given any positive integers $p < q$,
  put $n = q$,
  $x_1 = \cdots = x_p = x$
  and $x_{p+1} = \cdots = x_n = y$ in (1).)
  \item[(3)]
  Given any real $\lambda \in (0,1)$,
  there is a sequence of rational numbers $\{r_n\} \subseteq (0, 1)$
  such that $r_n \to \lambda$.
  By (2),
  $$\varphi(r_n x + (1 - r_n) y) \leq r_n \varphi(x) + (1 - r_n) \varphi(y)$$
  for any rational $r_n$ in $(0,1)$.
  Taking limit on the both sides and using the continuity of $f$,
  we have
  $$\varphi(\lambda x + (1 - \lambda) y) \leq \lambda \varphi(x) + (1 - \lambda) \varphi(y).$$
\end{enumerate}
$\Box$ \\



\emph{Proof (Reductio ad absurdum).}
If $\varphi$ were not convex,
then there is a subinterval $[c,d] \subseteq (a,b)$
such that
$$\frac{\varphi(d)-\varphi(c)}{d-c} < \frac{\varphi(x_0)-\varphi(c)}{x_0-c}$$
for some $x_0 \in [c, d]$.
Let
$$\psi(x) = \varphi(x)-\varphi(c) - \frac{\varphi(d)-\varphi(c)}{d-c}(x - c)$$
for $x \in [c,d]$.
Therefore,
\begin{enumerate}
\item[(1)]
  $\psi(x)$ is continuous and midpoint convex.

\item[(2)]
  $\psi(c) = \psi(d) = 0$.

\item[(3)]
  Let $M = \sup\{\psi(x) : x \in [c,d]\}$.
  $\infty > M > 0$ due to the continuity of $\psi$ and the existence of $x_0$.
  And let $\xi = \inf \{ x \in [c,d] : \psi(x) = M \}$.
  By the continuity of $g$, $\psi(\xi) = M$.
  $\xi \in (c,d)$ by (2).

\item[(4)]
  Since $(c,d)$ is open, there is $h > 0$ such that $(\xi-h,\xi+h) \subseteq (c,d)$.
  By the minimality of $\xi$ and $M$, $\psi(\xi-h) < \psi(\xi)$ and $\psi(\xi+h) \leq \psi(\xi)$.
\end{enumerate}
Therefore,
\begin{align*}
\psi(\xi-h) + \psi(\xi+h)
&< 2 \psi(\xi), \\
\frac{\psi(\xi-h) + \psi(\xi+h)}{2}
&< \psi(h) \\
&= \psi\left( \frac{(\xi-h) + (\xi+h)}{2} \right),
\end{align*}
contrary to the midpoint convexity of $\psi$.
$\Box$ \\\\



%%%%%%%%%%%%%%%%%%%%%%%%%%%%%%%%%%%%%%%%%%%%%%%%%%%%%%%%%%%%%%%%%%%%%%%%%%%%%%%%
%%%%%%%%%%%%%%%%%%%%%%%%%%%%%%%%%%%%%%%%%%%%%%%%%%%%%%%%%%%%%%%%%%%%%%%%%%%%%%%%



\end{document}