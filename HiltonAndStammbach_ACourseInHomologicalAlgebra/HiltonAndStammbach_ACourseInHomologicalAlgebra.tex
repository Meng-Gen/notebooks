\documentclass{article}
\usepackage{amsfonts}
\usepackage{amsmath}
\usepackage{amssymb}
\usepackage{centernot}
\usepackage{hyperref}
\usepackage[none]{hyphenat}
\usepackage{mathrsfs}
\usepackage{mathtools}
\usepackage{physics}
\usepackage{tikz-cd}
\parindent=0pt



\title{\textbf{Notes on the book: \\ \emph{P.J. Hilton and U. Stammbach, A Course in Homological Algebra}}}
\author{Meng-Gen Tsai \\ plover@gmail.com}



\begin{document}
\maketitle
\tableofcontents



%%%%%%%%%%%%%%%%%%%%%%%%%%%%%%%%%%%%%%%%%%%%%%%%%%%%%%%%%%%%%%%%%%%%%%%%%%%%%%%%
%%%%%%%%%%%%%%%%%%%%%%%%%%%%%%%%%%%%%%%%%%%%%%%%%%%%%%%%%%%%%%%%%%%%%%%%%%%%%%%%
%%%%%%%%%%%%%%%%%%%%%%%%%%%%%%%%%%%%%%%%%%%%%%%%%%%%%%%%%%%%%%%%%%%%%%%%%%%%%%%%
%%%%%%%%%%%%%%%%%%%%%%%%%%%%%%%%%%%%%%%%%%%%%%%%%%%%%%%%%%%%%%%%%%%%%%%%%%%%%%%%



% Reference:



%%%%%%%%%%%%%%%%%%%%%%%%%%%%%%%%%%%%%%%%%%%%%%%%%%%%%%%%%%%%%%%%%%%%%%%%%%%%%%%%
%%%%%%%%%%%%%%%%%%%%%%%%%%%%%%%%%%%%%%%%%%%%%%%%%%%%%%%%%%%%%%%%%%%%%%%%%%%%%%%%
%%%%%%%%%%%%%%%%%%%%%%%%%%%%%%%%%%%%%%%%%%%%%%%%%%%%%%%%%%%%%%%%%%%%%%%%%%%%%%%%
%%%%%%%%%%%%%%%%%%%%%%%%%%%%%%%%%%%%%%%%%%%%%%%%%%%%%%%%%%%%%%%%%%%%%%%%%%%%%%%%



\newpage
\section*{Chapter I: Modules \\}
\addcontentsline{toc}{section}{Chapter I: Modules}



\subsection*{\S 1. Modules \\}
\addcontentsline{toc}{subsection}{\S 1. Modules}



\subsubsection*{Exercise 1.1. (Diagram chasing)}
\addcontentsline{toc}{subsubsection}{Exercise 1.1. (Diagram chasing)}
\emph{Complete the proof of Lemma 1.1.
Show moreover that $\alpha$ is surjective (resp. injective) if
$\alpha'$, $\alpha''$ are surjective (resp. injective).} \\



\emph{Lemma 1.1.}
Let $0 \to A' \to A \to A'' \to 0$ and $0 \to B' \to B \to B'' \to 0$ be two short exact sequences.
Suppose that in the commutative diagram
\begin{center}
  \begin{tikzcd}
    0 \arrow[r]
      & A' \arrow[r, "\mu"]\arrow[d, "\alpha'"]
      & A \arrow[r, "\varepsilon"]\arrow[d, "\alpha"]
      & A'' \arrow[r]\arrow[d, "\alpha''"]
      & 0 \\
    0 \arrow[r] & B' \arrow[r, "\mu'"] & B \arrow[r, "\varepsilon'"] & B'' \arrow[r] & 0
  \end{tikzcd}
\end{center}
any two of the three homomorphisms $\alpha'$, $\alpha$, $\alpha''$ are isomorphisms.
Then the third is an isomorphism, too. \\



\emph{Proof (Diagram chasing).}
\begin{enumerate}
\item[(1)]
  \emph{Show that $\alpha$ is surjective if $\alpha'$, $\alpha''$ are surjective.}
  \begin{enumerate}
  \item[(a)]
    Take any $b \in B$, it suffices to find $a \in A$ such that $\alpha a = b$.

  \item[(b)]
    Consider the commutative diagram
    \begin{center}
      \begin{tikzcd}
        A \arrow[r, two heads, "\varepsilon"]\arrow[d, "\alpha"] & A'' \arrow[d, "\alpha''"] \\
        B \arrow[r, "\varepsilon'"] & B''
      \end{tikzcd}
    \end{center}
    $\varepsilon' b \in B'$.
    By the surjectivity of $\alpha''$, $\exists \: a'' \in A''$ such that $\alpha'' a'' = \varepsilon' b$.
    By the surjectivity of $\varepsilon$, $\exists \: \overline{a} \in A$
    such that $\varepsilon \overline{a} = a''$.
    Hence
    \begin{align*}
      \varepsilon'(b - \alpha \overline{a})
      &= \varepsilon' b - \varepsilon' \alpha \overline{a} \\
      &= \varepsilon' b - \alpha'' \varepsilon \overline{a}
        &(\text{The diagram commutes}) \\
      &= \varepsilon' b - \alpha'' a'' \\
      &= \varepsilon' b - \varepsilon' b \\
      &= 0.
    \end{align*}

  \item[(c)]
    Consider the short exact sequence
    \begin{center}
      \begin{tikzcd}
        0 \arrow[r]
          & B' \arrow[r, "\mu'"]
          & B \arrow[r, "\varepsilon'"]
          & B'' \arrow[r]
          & 0
      \end{tikzcd}
    \end{center}
    As $\varepsilon'(b - \alpha \overline{a}) = 0$,
    $\exists \: b' \in B'$ such that $\mu' b' = b - \alpha \overline{a}$.

  \item[(d)]
    Consider the commutative diagram
    \begin{center}
      \begin{tikzcd}
        A' \arrow[r, "\mu"]\arrow[d, "\alpha'"] & A \arrow[d, "\alpha"] \\
        B' \arrow[r, "\mu'"] & B
      \end{tikzcd}
    \end{center}
    By the surjectivity of $\alpha'$,
    $\exists \: a' \in A'$ such that $\alpha' a' = b'$.
    Hence
    \begin{align*}
      \alpha(\mu a' + \overline{a})
      &= \alpha\mu a' + \alpha\overline{a} \\
      &= \mu'\alpha'a' + \alpha\overline{a}
        &(\text{The diagram commutes}) \\
      &= \mu'b' + \alpha\overline{a} \\
      &= (b - \alpha\overline{a}) + \alpha\overline{a} \\
      &= b.
    \end{align*}
    Therefore, there exists $a := \mu a' + \overline{a}$ such that $\alpha a = b$.
  \end{enumerate}

\item[(2)]
  \emph{Show that $\alpha$ is injective if $\alpha'$, $\alpha''$ are injective.}
  \begin{enumerate}
  \item[(a)]
    It suffices to show that $\ker\alpha = 0$.
    Take $a \in \ker\alpha$. ($\alpha(a) = \alpha a = 0$.)

  \item[(b)]
    Consider the commutative diagram
    \begin{center}
      \begin{tikzcd}
        A \arrow[r, "\varepsilon"]\arrow[d, "\alpha"] & A'' \arrow[d, "\alpha''"] \\
        B \arrow[r, "\varepsilon'"] & B''
      \end{tikzcd}
    \end{center}
    we have $0 = \varepsilon'\alpha a = \alpha''\varepsilon a$.
    By the injectivity of $\alpha''$, $\varepsilon a = 0$.

  \item[(c)]
    Consider the short exact sequence
    \begin{center}
      \begin{tikzcd}
        0 \arrow[r]
          & A' \arrow[r, "\mu"]
          & A \arrow[r, "\varepsilon"]
          & A'' \arrow[r]
          & 0
      \end{tikzcd}
    \end{center}
    As $\varepsilon a = 0$, $\exists \: a' \in A'$ such that
    $\mu a' = a$.

  \item[(d)]
    Consider the commutative diagram
    \begin{center}
      \begin{tikzcd}
        A' \arrow[r, "\mu"]\arrow[d, "\alpha'"] & A \arrow[d, "\alpha"] \\
        B' \arrow[r, tail, "\mu'"] & B
      \end{tikzcd}
    \end{center}
    $0 = \alpha a = \alpha\mu a' = \mu'\alpha' a'$.
    By the injectivity of $\mu'\alpha'$, $a' = 0$.
    Therefore, $a = \mu a' = 0$.
  \end{enumerate}

\item[(3)]
  \emph{Suppose $\alpha$ is surjective. Show that $\alpha''$ is surjective.}
  \begin{enumerate}
  \item[(a)]
    Take any $b'' \in B''$, it suffices to find $a'' \in A''$ such that $\alpha'' a'' = b''$.

  \item[(b)]
    Consider the commutative diagram
    \begin{center}
      \begin{tikzcd}
        A \arrow[r, "\varepsilon"]\arrow[d, "\alpha"] & A'' \arrow[d, "\alpha''"] \\
        B \arrow[r, two heads, "\varepsilon'"] & B''
      \end{tikzcd}
    \end{center}
    By the surjectivity of $\varepsilon'$,
    $\exists \: b \in B$ such that $\varepsilon' b = b''$.
    By the surjectivity of $\alpha$,
    $\exists \: a \in A$ such that $\alpha a = b$.
    Take $a'' := \varepsilon a \in A''$.
    Hence
    \begin{align*}
      \alpha'' a''
      &= \alpha'' \varepsilon a \\
      &= \varepsilon' \alpha a
        &(\text{The diagram commutes}) \\
      &= \varepsilon' b \\
      &= b''.
    \end{align*}
  \end{enumerate}

\item[(4)]
  \emph{Suppose $\alpha'$ is surjective and $\alpha$ is injective.
  Show that $\alpha''$ is injective.}
  \begin{enumerate}
  \item[(a)]
    It suffices to show that $\ker\alpha'' = 0$.
    Take $a'' \in \ker\alpha''$. ($\alpha''(a'') = \alpha'' a'' = 0$.)

  \item[(b)]
    Consider the commutative diagram
    \begin{center}
      \begin{tikzcd}
        A \arrow[r, two heads, "\varepsilon"]\arrow[d, "\alpha"] & A'' \arrow[d, "\alpha''"] \\
        B \arrow[r, "\varepsilon'"] & B''
      \end{tikzcd}
    \end{center}
    By the surjectivity of $\varepsilon$,
    $\exists \: a \in A$ such that $\varepsilon a = a''$.
    So
    \begin{align*}
      0
      &= \alpha'' a'' \\
      &= \alpha'' \varepsilon a \\
      &= \varepsilon' \alpha a.
        &(\text{The diagram commutes})
    \end{align*}

  \item[(c)]
    Consider the short exact sequence
    \begin{center}
      \begin{tikzcd}
        0 \arrow[r]
          & B' \arrow[r, "\mu'"]
          & B \arrow[r, "\varepsilon'"]
          & B'' \arrow[r]
          & 0
      \end{tikzcd}
    \end{center}
    As $\varepsilon'(\alpha a) = 0$,
    $\exists \: b' \in B'$ such that $\mu' b' = \alpha a$.

  \item[(d)]
    Consider the commutative diagram
    \begin{center}
      \begin{tikzcd}
        A' \arrow[r, "\mu"]\arrow[d, "\alpha'"] & A \arrow[d, "\alpha"] \\
        B' \arrow[r, "\mu'"] & B
      \end{tikzcd}
    \end{center}
    By surjectivity of $\alpha'$,
    $\exists \: a' \in A'$ such that $\alpha' a' = b'$.
    So
    \begin{align*}
      \alpha a
      &= \mu' b' \\
      &= \mu' \alpha' a' \\
      &= \alpha \mu a'.
        &(\text{The diagram commutes})
    \end{align*}
    By the injectivity of $\alpha$, $a = \mu a'$.
    Hence
    \[
      a''
      = \varepsilon a
      = \varepsilon\mu a'
      = 0.
    \]
    Therefore $\ker\alpha'' = 0$.
  \end{enumerate}

\item[(5)]
  By (3)(4), $\alpha''$ is an isomorphism
  if both $\alpha'$ and $\alpha$ are isomorphisms.

\item[(6)]
  \emph{Suppose $\alpha$ is surjective and $\alpha''$ is injective.
  Show that $\alpha'$ is surjective.}
  \begin{enumerate}
  \item[(a)]
    Take any $b' \in B'$, it suffices to find $a' \in A'$ such that $\alpha' a' = b'$.
    Let $b := \mu' b' \in B$ and note that $\varepsilon' b = 0$ by the exactness of
    \[
      0 \to B' \to B \to B'' \to 0.
    \]

  \item[(b)]
    Consider the commutative diagram
    \begin{center}
      \begin{tikzcd}
        A \arrow[r, "\varepsilon"]\arrow[d, "\alpha"] & A'' \arrow[d, "\alpha''"] \\
        B \arrow[r, "\varepsilon'"] & B''
      \end{tikzcd}
    \end{center}
    By the surjectivity of $\alpha$,
    $\exists \: a \in A$ such that $\alpha a = b$.
    So
    \begin{align*}
      0
      &= \varepsilon' b \\
      &= \varepsilon' \alpha a \\
      &= \alpha'' \varepsilon a.
        &(\text{The diagram commutes})
    \end{align*}
    By the injectivity of $\alpha''$, $\varepsilon a = 0$.

  \item[(c)]
    Consider the short exact sequence
    \begin{center}
      \begin{tikzcd}
        0 \arrow[r]
          & A' \arrow[r, "\mu"]
          & A \arrow[r, "\varepsilon"]
          & A'' \arrow[r]
          & 0
      \end{tikzcd}
    \end{center}
    As $\varepsilon a = 0$, $\exists \: a' \in A'$ such that $\mu a' = a$.

  \item[(d)]
    Consider the commutative diagram
    \begin{center}
      \begin{tikzcd}
        A' \arrow[r, "\mu"]\arrow[d, "\alpha'"] & A \arrow[d, "\alpha"] \\
        B' \arrow[r, tail, "\mu'"] & B
      \end{tikzcd}
    \end{center}
    Note that
    \begin{align*}
      \mu'(\alpha'a')
      &= \mu' \alpha' a' \\
      &= \alpha \mu a'
        &(\text{The diagram commutes})
      &= \alpha a \\
      &= b \\
      &= \mu' b'.
    \end{align*}
    By the injectivity of $\mu'$, $b' = \alpha' a'$ for some $a' \in A'$.
  \end{enumerate}

\item[(7)]
  \emph{Suppose $\alpha$ is injective.
  Show that $\alpha'$ is injective.}
  \begin{enumerate}
  \item[(a)]
    It suffices to show that $\ker\alpha' = 0$.
    Take $a' \in \ker\alpha'$. ($\alpha'(a') = \alpha' a' = 0$.)

  \item[(b)]
    Consider the commutative diagram
    \begin{center}
      \begin{tikzcd}
        A' \arrow[r, tail, "\mu"]\arrow[d, "\alpha'"] & A \arrow[d, "\alpha"] \\
        B' \arrow[r, "\mu'"] & B
      \end{tikzcd}
    \end{center}
    Note that
    \begin{align*}
      0
      &= \mu' 0 \\
      &= \mu'\alpha' a' \\
      &= \alpha\mu a'.
        &(\text{The diagram commutes})
    \end{align*}
    The injectivity of $\alpha\mu$ shows that $a' = 0$.
  \end{enumerate}

\item[(8)]
  By (6)(7), $\alpha'$ is an isomorphism
  if both $\alpha$ and $\alpha''$ are isomorphisms.
\end{enumerate}
$\Box$ \\\\



%%%%%%%%%%%%%%%%%%%%%%%%%%%%%%%%%%%%%%%%%%%%%%%%%%%%%%%%%%%%%%%%%%%%%%%%%%%%%%%%



\subsubsection*{Exercise 1.2. (Five lemma)}
\addcontentsline{toc}{subsubsection}{Exercise 1.2. (Five lemma)}
\emph{Show that, given a commutative diagram}
\begin{center}
  \begin{tikzcd}
    \cdots \arrow[r]
        & A_1 \arrow[r]\arrow[d, "\varphi_1"]
        & A_2 \arrow[r]\arrow[d, "\varphi_2"]
        & A_3 \arrow[r]\arrow[d, "\varphi_3"]
        & A_4 \arrow[r]\arrow[d, "\varphi_4"]
        & A_5 \arrow[r]\arrow[d, "\varphi_5"]
        & \cdots \\
    \cdots \arrow[r]
        & B_1 \arrow[r]
        & B_2 \arrow[r]
        & B_3 \arrow[r]
        & B_4 \arrow[r]
        & B_5 \arrow[r]
        & \cdots
  \end{tikzcd}
\end{center}
\emph{with exact rows, in which $\varphi_1$, $\varphi_2$, $\varphi_4$, $\varphi_5$ are isomorphisms,
then $\varphi_3$ is also an isomorphism.
Can we weaken the hypotheses in a reasonable way?} \\



One reasonable hypotheses:
\begin{enumerate}
\item[(a)]
  \emph{If $\varphi_1$ is surjective and $\varphi_2, \varphi_4$ is injective,
  then $\varphi_3$ is injective.}

\item[(b)]
  \emph{If $\varphi_5$ is injective and $\varphi_2, \varphi_4$ is surjective,
  then $\varphi_3$ is surjective.} \\
\end{enumerate}



\emph{Proof of (a).}
\begin{enumerate}
\item[(1)]
  Write
  \begin{center}
    \begin{tikzcd}
      \cdots \arrow[r]
          & A_1 \arrow[r, "\alpha_1"]\arrow[d, "\varphi_1"]
          & A_2 \arrow[r, "\alpha_2"]\arrow[d, "\varphi_2"]
          & A_3 \arrow[r, "\alpha_3"]\arrow[d, "\varphi_3"]
          & A_4 \arrow[r, "\alpha_4"]\arrow[d, "\varphi_4"]
          & A_5 \arrow[r]\arrow[d, "\varphi_5"]
          & \cdots \\
      \cdots \arrow[r]
          & B_1 \arrow[r, "\beta_1"]
          & B_2 \arrow[r, "\beta_2"]
          & B_3 \arrow[r, "\beta_3"]
          & B_4 \arrow[r, "\beta_4"]
          & B_5 \arrow[r]
          & \cdots
    \end{tikzcd}
  \end{center}
  Take $a \in \ker(\varphi_3)$ and then we need to show $a = 0$.

\item[(2)]
  The commutative diagram
  \begin{center}
    \begin{tikzcd}
      A_3 \arrow[r, "\alpha_3"]\arrow[d, "\varphi_3"] & A_4 \arrow[d, tail, "\varphi_4"] \\
      B_3 \arrow[r, "\beta_3"] & B_4
    \end{tikzcd}
  \end{center}
  implies that $0 = \beta_3 0 = \beta_3 \varphi_3 a = \varphi_4 \alpha_3 a$.
  The injectivity of $\varphi_4$ implies that $\alpha_3 a = 0$.

\item[(3)]
  The exact sequence
  \begin{center}
    \begin{tikzcd}
      \cdots \arrow[r]
          & A_2 \arrow[r, "\alpha_2"]
          & A_3 \arrow[r, "\alpha_3"]
          & A_4 \arrow[r]
          & \cdots
    \end{tikzcd}
  \end{center}
  shows that $a \in \ker(\alpha_3) = \mathrm{im}(\alpha_2)$.
  So there exists $a_2 \in A_2$ such that $\alpha_2 a_2 = a$.

\item[(4)]
  The commutative diagram
  \begin{center}
    \begin{tikzcd}
      A_2 \arrow[r, "\alpha_2"]\arrow[d, "\varphi_2"] & A_3 \arrow[d, "\varphi_3"] \\
      B_2 \arrow[r, "\beta_2"] & B_3
    \end{tikzcd}
  \end{center}
  implies that $0 = \varphi_3 a = \varphi_3 \alpha_2 a_2 = \beta_2 \varphi_2 a_2$.

\item[(5)]
  The exact sequence
  \begin{center}
    \begin{tikzcd}
      \cdots \arrow[r]
          & B_1 \arrow[r, "\beta_1"]
          & B_2 \arrow[r, "\beta_2"]
          & B_3 \arrow[r]
          & \cdots
    \end{tikzcd}
  \end{center}
  shows that $\varphi_2 a_2 \in \ker(\beta_2) = \mathrm{im}(\beta_1)$.
  So there exists $b_1 \in B_1$ such that $\varphi_2 a_2 = \beta_1 b_1$.

\item[(6)]
  Consider the commutative diagram
  \begin{center}
    \begin{tikzcd}
      A_1 \arrow[r, "\alpha_1"]\arrow[d, two heads, "\varphi_1"] & A_2 \arrow[d, tail, "\varphi_2"] \\
      B_1 \arrow[r, "\beta_1"] & B_2
    \end{tikzcd}
  \end{center}
  The surjectivity of $\varphi_i$ implies that
  $\exists \: a_1 \in A_1$ such that $\varphi_1 a_1 = b_1$.
  Hence the commutative diagram implies that
  $\varphi_2(\alpha_1 a_1) = \varphi_2\alpha_1 a_1 = \beta_1 \varphi_1 a_1 = \beta_1 b_1 = \varphi_2 a_2$.
  The injectivity of $\varphi_2$ implies that $\alpha_1 a_1 = a_2$.

\item[(7)]
  The exact sequence
  \begin{center}
    \begin{tikzcd}
      \cdots \arrow[r]
          & A_1 \arrow[r, "\alpha_1"]
          & A_2 \arrow[r, "\alpha_2"]
          & A_3 \arrow[r]
          & \cdots
    \end{tikzcd}
  \end{center}
  shows that $a = \alpha_2 a_2 = \alpha_2 \alpha_1 a_1 = 0$.
  Therefore $\varphi_3$ is injective.
\end{enumerate}
$\Box$ \\



\emph{Proof of (b).}
\begin{enumerate}
\item[(1)]
  Take any $b \in B_3$, it suffices to find $a \in A$ such that $\varphi_3 a = b$.

\item[(2)]
  Let $b_4 := \beta_3 b \in B_4$.
  The exact sequence
  \begin{center}
    \begin{tikzcd}
      \cdots \arrow[r]
          & B_3 \arrow[r, "\beta_3"]
          & B_4 \arrow[r, "\beta_4"]
          & B_5 \arrow[r]
          & \cdots
    \end{tikzcd}
  \end{center}
  shows that $\beta_4 b_4 = \beta_4(\beta_3 b) = 0$.

\item[(3)]
  Look at the commutative diagram
  \begin{center}
    \begin{tikzcd}
      A_4 \arrow[r, "\alpha_4"]\arrow[d, two heads, "\varphi_4"] & A_5 \arrow[d, tail, "\varphi_5"] \\
      B_4 \arrow[r, "\beta_4"] & B_5
    \end{tikzcd}
  \end{center}
  By the surjectivity of $\varphi_4$,
  $\exists \: a_4 \in A_4$ such that $\varphi_4 a_4 = b_4$.
  So the commutative diagram says that
  $0 = \beta_4 b_4 = \beta_4 \varphi_4 a_4 = \varphi_5 \alpha_4 a_4$.
  By the injectivity of $\varphi_5$, $\alpha_4 a_4 = 0$.

\item[(4)]
  The exact sequence
  \begin{center}
    \begin{tikzcd}
      \cdots \arrow[r]
          & A_3 \arrow[r, "\alpha_3"]
          & A_4 \arrow[r, "\alpha_4"]
          & A_5 \arrow[r]
          & \cdots
    \end{tikzcd}
  \end{center}
  shows that $a_4 \in \ker(\alpha_4) = \mathrm{im}(\alpha_3)$.
  So there exists $a_3 \in A_3$ such that $\alpha_3 a_3 = a_4$.

\item[(5)]
  Let $\overline{b} = b - \varphi_3 a_3 \in B_3$.
  The commutative diagram
  \begin{center}
    \begin{tikzcd}
      A_3 \arrow[r, "\alpha_3"]\arrow[d, "\varphi_3"] & A_4 \arrow[d, "\varphi_4"] \\
      B_3 \arrow[r, "\beta_3"] & B_4
    \end{tikzcd}
  \end{center}
  implies that
  $\beta_3 \overline{b}
  = \beta_3 b - \beta_3 \varphi_3 a_3
  = \beta_3 b - \varphi_4 \alpha_3 a_3
  = \beta_3 b - \varphi_4 a_4
  = \beta_3 b - b_4
  = \beta_3 b - \beta_3 b = 0$.
  So $\overline{b} \in \ker(\beta_3)$.

\item[(6)]
  The exact sequence
  \begin{center}
    \begin{tikzcd}
      \cdots \arrow[r]
          & B_2 \arrow[r, "\beta_2"]
          & B_3 \arrow[r, "\beta_3"]
          & B_4 \arrow[r]
          & \cdots
    \end{tikzcd}
  \end{center}
  shows that $\overline{b} \in \ker(\beta_3) = \mathrm{im}(\beta_2)$.
  Hence $\exists \: b_2 \in B_2$ such that $\overline{b} = \beta_2 b_2$.

\item[(7)]
  Look at the commutative diagram
  \begin{center}
    \begin{tikzcd}
      A_2 \arrow[r, "\alpha_2"]\arrow[d, two heads, "\varphi_2"] & A_3 \arrow[d, "\varphi_3"] \\
      B_2 \arrow[r, "\beta_2"] & B_3
    \end{tikzcd}
  \end{center}
  The surjectivity of $\varphi_2$ implies that
  $\exists \: a_2 \in A_2$ such that $b_2 = \varphi_2 a_2$.
  Let $a := \alpha_2 a_2 + a_3$.
  Hence
  \begin{align*}
    \varphi_3(a)
    &= \varphi_3\alpha_2 a_2 + \varphi_3 a_3 \\
    &= \beta_2\varphi_2 a_2 + \varphi_3 a_3
      &(\text{The diagram commutes}) \\
    &= \beta_2 b_2 + \varphi_3 a_3 \\
    &= \overline{b} + \varphi_3 a_3 \\
    &= (b - \varphi_3 a_3) + \varphi_3 a_3 \\
    &= b.
  \end{align*}
\end{enumerate}
$\Box$ \\\\



%%%%%%%%%%%%%%%%%%%%%%%%%%%%%%%%%%%%%%%%%%%%%%%%%%%%%%%%%%%%%%%%%%%%%%%%%%%%%%%%



\subsubsection*{Exercise 1.3.}
\addcontentsline{toc}{subsubsection}{Exercise 1.3.}
\emph{Give examples of short exact sequences of abelian groups
\begin{align*}
  & 0 \to A' \to A \to A'' \to 0, \\
  & 0 \to B' \to B \to B'' \to 0
\end{align*}
such that}
\begin{enumerate}
\item[(i)]
  \emph{$A' \cong B'$, $A \cong B$, $A'' \not\cong B''$.}

\item[(ii)]
  \emph{$A' \cong B'$, $A \not\cong B$, $A'' \cong B''$.}

\item[(iii)]
  \emph{$A' \not\cong B'$, $A \cong B$, $A'' \cong B''$.} \\
\end{enumerate}



\emph{Proof of (i).}
  Define
  \begin{center}
    \begin{tikzcd}
      0 \arrow[r]
          & \mathbb{Z}/(2) \arrow[r, "\iota_1"]\arrow[d, equal]
          & \mathbb{Z}/(2) \times \mathbb{Z}/(3) \times \mathbb{Z}/(5) \arrow[r, "\pi_2"]\arrow[d, equal]
          & \mathbb{Z}/(3) \arrow[r]
          & 0 \\
      0 \arrow[r]
          & \mathbb{Z}/(2) \arrow[r, "\iota_1"]
          & \mathbb{Z}/(2) \times \mathbb{Z}/(3) \times \mathbb{Z}/(5) \arrow[r, "\pi_3"]
          & \mathbb{Z}/(5) \arrow[r]
          & 0
    \end{tikzcd}
  \end{center}
  where $\iota_1: a \mapsto (a, 0, 0)$,
  $\pi_2: (a_1, a_2, a_3) \mapsto a_2$ and $\pi_3: (a_1, a_2, a_3) \mapsto a_3$.
$\Box$ \\



\emph{Proof of (ii).}
  Define
  \begin{center}
    \begin{tikzcd}
      0 \arrow[r]
          & \mathbb{Z}/(2) \arrow[r, "\iota_1"]\arrow[d, equal]
          & \mathbb{Z}/(2) \times \mathbb{Z}/(3) \times \mathbb{Z}/(5) \arrow[r, "\pi_2"]
          & \mathbb{Z}/(3) \arrow[r]\arrow[d, equal]
          & 0 \\
      0 \arrow[r]
          & \mathbb{Z}/(2) \arrow[r, "\iota_1"]
          & \mathbb{Z}/(2) \times \mathbb{Z}/(3) \times \mathbb{Z}/(7) \arrow[r, "\pi_2"]
          & \mathbb{Z}/(3) \arrow[r]
          & 0
    \end{tikzcd}
  \end{center}
  where $\iota_1: a \mapsto (a, 0, 0)$ and $\pi_2: (a_1, a_2, a_3) \mapsto a_2$.
$\Box$ \\



\emph{Proof of (iii).}
  Define
  \begin{center}
    \begin{tikzcd}
      0 \arrow[r]
          & \mathbb{Z}/(2) \arrow[r, "\iota_1"]
          & \mathbb{Z}/(2) \times \mathbb{Z}/(3) \times \mathbb{Z}/(5) \arrow[r, "\pi_3"]\arrow[d, equal]
          & \mathbb{Z}/(5) \arrow[r]\arrow[d, equal]
          & 0 \\
      0 \arrow[r]
          & \mathbb{Z}/(3) \arrow[r, "\iota_2"]
          & \mathbb{Z}/(2) \times \mathbb{Z}/(3) \times \mathbb{Z}/(5) \arrow[r, "\pi_3"]
          & \mathbb{Z}/(5) \arrow[r]
          & 0
    \end{tikzcd}
  \end{center}
  where $\iota_1: a \mapsto (a, 0, 0)$, $\iota_2: a \mapsto (0, a, 0)$
  and $\pi_3: (a_1, a_2, a_3) \mapsto a_3$.
$\Box$ \\\\



%%%%%%%%%%%%%%%%%%%%%%%%%%%%%%%%%%%%%%%%%%%%%%%%%%%%%%%%%%%%%%%%%%%%%%%%%%%%%%%%
%%%%%%%%%%%%%%%%%%%%%%%%%%%%%%%%%%%%%%%%%%%%%%%%%%%%%%%%%%%%%%%%%%%%%%%%%%%%%%%%
%%%%%%%%%%%%%%%%%%%%%%%%%%%%%%%%%%%%%%%%%%%%%%%%%%%%%%%%%%%%%%%%%%%%%%%%%%%%%%%%
%%%%%%%%%%%%%%%%%%%%%%%%%%%%%%%%%%%%%%%%%%%%%%%%%%%%%%%%%%%%%%%%%%%%%%%%%%%%%%%%



\end{document}