\documentclass{article}
\usepackage{amsfonts}
\usepackage{amsmath}
\usepackage{amssymb}
\usepackage{hyperref}
\usepackage[none]{hyphenat}
\usepackage{mathrsfs}
\usepackage{physics}
\parindent=0pt

\def\upint{\mathchoice%
    {\mkern13mu\overline{\vphantom{\intop}\mkern7mu}\mkern-20mu}%
    {\mkern7mu\overline{\vphantom{\intop}\mkern7mu}\mkern-14mu}%
    {\mkern7mu\overline{\vphantom{\intop}\mkern7mu}\mkern-14mu}%
    {\mkern7mu\overline{\vphantom{\intop}\mkern7mu}\mkern-14mu}%
  \int}
\def\lowint{\mkern3mu\underline{\vphantom{\intop}\mkern7mu}\mkern-10mu\int}

\begin{document}



\textbf{\Large Chapter 1: Roots of Commutative Algebra} \\\\



\emph{Author: Meng-Gen Tsai} \\
\emph{Email: plover@gmail.com} \\\\



% https://kconrad.math.uconn.edu/blurbs/linmultialg/noetherianmod.pdf
% http://www.math.unl.edu/~tmarley1/math902/homework/902hw2soln.pdf



%%%%%%%%%%%%%%%%%%%%%%%%%%%%%%%%%%%%%%%%%%%%%%%%%%%%%%%%%%%%%%%%%%%%%%%%%%%%%%%%
%%%%%%%%%%%%%%%%%%%%%%%%%%%%%%%%%%%%%%%%%%%%%%%%%%%%%%%%%%%%%%%%%%%%%%%%%%%%%%%%



\textbf{\large Noetherian Rings and Modules} \\\\



\textbf{Exercise 1.1.}
\emph{Prove that the following conditions on a module $M$ over a commutative ring $R$
are equivalent (the fourth is Hilbert's original formulation;
the first and the third are the ones most often used).
The case $M = R$ is the case of ideals.}
\begin{enumerate}
\item[(1)]
\emph{$M$ is Noetherian (that is, every submodule of $M$ is finitely generated).}
\item[(2)]
\emph{Every ascending chain of submodules of $M$ terminates
(``ascending chain condition'').}
\item[(3)]
\emph{Every set of submodules of $M$ contains elements maximal under inclusion.}
\item[(4)]
\emph{Given any sequence of elements $f_1, f_2, \ldots \in M$,
there is a number $m$ such that for each $n > m$ there is an expression
$f_n = \sum_{i=1}^{m} a_i f_i$ with $a_i \in R$.} \\
\end{enumerate}

\emph{Idea.}
$(1) \Rightarrow (2) \Rightarrow (4) \Rightarrow (3) \Rightarrow (1)$. \\

\emph{Proof of $(1) \Rightarrow (2)$.}
Given any ascending chain of submodules $N_1 \subseteq N_2 \subseteq \cdots$,
let $$N = \bigcup_{i=1}^{\infty} N_i.$$
\begin{enumerate}
\item[(a)]
\emph{$N$ is a submodule.}
By the ascending chain condition, each pair of elements in $N$ are in a common $N_m$.
\item[(b)]
$N$ is finitely generated by assumption.
By the ascending chain condition again, all generators of $N$ are in a common $N_m$.
So $N = N_m$ for some $m$.
\item[(c)]
Since $N_m = N \supseteq N_n$ whenever $n \geq m$,
$N_m = N_{m+1} = \cdots$.
\end{enumerate}
$\Box$ \\

\emph{Proof of $(2) \Rightarrow (4)$.}
Let $N_k$ be generated by $f_1, f_2, \ldots, f_k$.
\begin{enumerate}
\item[(a)]
$N_1 \subseteq N_2 \subseteq \cdots$ is an ascending chain of submodules of $M$.
\item[(b)]
By assumption there is a number $m$ such that $N_m = N_{m+1} = \cdots$.
\item[(c)]
Given any $n \geq m$, $f_n \in N_n = N_m$.
So we can write $f_n = \sum_{i=1}^{m} a_i f_i$ with $a_i \in R$
since $N_m$ is generated by $f_1, f_2, \ldots, f_m$.
\end{enumerate}
$\Box$ \\

\emph{Proof of $(4) \Rightarrow (3)$.}
\emph{It suffices to show that $\neg (3) \Rightarrow \neg (4)$.}
There exists a nonempty collection
$\Sigma$ of submodules of $M$ containing no maximal element under inclusion.
\begin{enumerate}
\item[(a)]
Start with any submodule $N_1$ in $\Sigma$,
and recursively pick submodule $N_2, N_3, \ldots$ such that
$N_1 \subsetneq N_2 \subsetneq N_3 \subsetneq \cdots$.
\item[(b)]
Pick $f_1 \in N_1$ and $f_i \in N_i - N_{i-1} \neq \varnothing$ for $i \geq 2$.
The sequence of elements $f_1, f_2, \ldots \in M$ is what we want.
\end{enumerate}
$\Box$ \\

\emph{Proof of $(3) \Rightarrow (1)$.}
\emph{Show that $N$ is finitely generated if $N$ is any submodule of $M$.}
Let $\Sigma$ be the set of all finitely generated submodules of $N$.
\begin{enumerate}
\item[(a)]
$\Sigma \neq \varnothing$ since $0$ is a finitely generated submodules of $N$.
\item[(b)]
By assumption, there exists a maximal element $N_0$ of $\Sigma$.
$N_0$ is finitely generated.
\item[(c)]
(Reductio ad absurdum)
If $N_0$ were not equal to $N$, there is $x \in N - N_0$.
Clearly the submodule $N_0 + xR$ of $N$ is finitely generated and
$N_0 + xR \supsetneq N_0$, contrary to the maximality of $N_0$.
\end{enumerate}
$\Box$ \\

\emph{Proof of $(2) \Rightarrow (3)$.}
It is the part (a) of the proof of $(4) \Rightarrow (3)$.
$\Box$ \\

\emph{Proof of $(3) \Rightarrow (2)$.}
Given any ascending chain of submodules $N_1 \subseteq N_2 \subseteq \cdots$.
The set $$\Sigma = \{N_i\}_{i \geq 1}$$ has a maximal element, say $N_m$.
Hence $N_m = N_{m+1} = \cdots$ by the maximality of $N_m$.
$\Box$ \\

\textbf{Remark.}
In general, let $\Sigma$ be a set partially ordered by a relation $\leq$.
Then the following conditions on $\Sigma$ are equivalent:
\begin{enumerate}
\item[(1)]
Every increasing sequence $x_1 \leq x_2 \leq \cdots \in \Sigma$ is stationary.
\item[(2)]
Every non-empty subset of $\Sigma$ has a maximal element. \\\\
\end{enumerate}



%%%%%%%%%%%%%%%%%%%%%%%%%%%%%%%%%%%%%%%%%%%%%%%%%%%%%%%%%%%%%%%%%%%%%%%%%%%%%%%%



\textbf{Exercise 1.2 (Emmy Noether).}
\emph{Prove that if $R$ is Noetherian,
and $I \subsetneq R$ is an ideal,
then among the primes of $R$ containing $I$ there are only finitely many
that are minimal with respect to inclusion
(these are usually called the \textbf{minimal primes of $I$},
or the \textbf{primes minimal over $I$}) as follows:
Assuming that the proposition fails, the Noetherian hypothesis guarantees
the existence of an ideal $I$ maximal among ideals in $R$ for which it fails.
Show that $I$ cannot be prime,
so we can find elements $f$ and $g$ in $R$, not in $I$, such that $fg \in I$.
Now show that every prime minimal over $I$ is minimal over
one of the larger ideals $(I,f)$ and $(I,g)$.} \\

\emph{Note.}
With Hilbert's basis theorem and the Nullstellensatz (see Exercise 1.9),
Exercise 1.2 gives one of the fundamental finiteness theorems of algebraic geometry:
An algebraic set can have only finitely many irreducible components.
Originally the result was proved by difficult inductive arguments and elimination theory.
For a further discussion of the significance of this reslt
see the beginning of Chapter 3, and particularly example 2 there.
The result of this exercise is strengthened in Theorem 3.1. \\\\



\textbf{Lemma.}
\emph{For any $\mathfrak{p} \supseteq \mathfrak{a} \mathfrak{b}$,
$\mathfrak{p} \supseteq \mathfrak{a}$ or $\mathfrak{p} \supseteq \mathfrak{b}$.} \\

\emph{Proof of Lemma.}
\begin{enumerate}
\item[(1)] If $\mathfrak{p} \supseteq \mathfrak{a}$. We are done.
\item[(2)] If $\mathfrak{p} \not\supseteq \mathfrak{a}$,
there exists $a \in \mathfrak{a} - \mathfrak{p}$.
So for any $b \in \mathfrak{b}$, $b \in \mathfrak{p}$
since $ab \in \mathfrak{ab} \subseteq \mathfrak{p}$ and $\mathfrak{p}$ is a prime ideal,
that is, $\mathfrak{p} \supseteq \mathfrak{b}$.
\end{enumerate}
By (1)(2), $\mathfrak{p} \supseteq \mathfrak{a}$ or $\mathfrak{p} \supseteq \mathfrak{b}$.
$\Box$ \\\\



\emph{Proof.}
(Reductio ad absurdum)
\begin{enumerate}
\item[(1)]
Assuming that the proposition fails, the Noetherian hypothesis of $R$
guarantees the existence of an ideal $I$ maximal among ideals in $R$ for which it fails.
\item[(2)]
\emph{Show that $I$ cannot be prime.}
(Reductio ad absurdum)
If $I$ were prime,
then there were only one minimal prime $I$ itself, which is absurd.
\item[(3)]
Therefore, there exist elements $f, g \in R$ such that $fg \in I$
but $f \not\in I$ and $g \not\in I$.
$(I,f) \supsetneq I$, $(I,g) \supsetneq I$ and $(I,f)(I,g) \subseteq I$.
\item[(4)]
By Lemma, any prime containing $I$ must contain either $(I,f)$ or $(I,g)$.
In particular, any prime minimal over $I$ is minimal over either $(I,f)$ or $(I,g)$.
However, by the choice of $I$,
both $(I,f)$ and $(I,g)$ have only finitely many minimal primes,
which is absurd.
\end{enumerate}
$\Box$ \\\\



%%%%%%%%%%%%%%%%%%%%%%%%%%%%%%%%%%%%%%%%%%%%%%%%%%%%%%%%%%%%%%%%%%%%%%%%%%%%%%%%



\textbf{Exercise 1.3.}
\emph{Let $M'$ be a submodule of $M$.
Show that $M$ is Noetherian iff both $M'$ and $M/M'$ are Noetherian.} \\

\emph{Proof.}
\begin{enumerate}
\item[(1)]
$(\Longrightarrow)$
  \begin{enumerate}
  \item[(a)]
  \emph{Show that $M'$ is Noetherian if $M$ is Noetherian.}
  This is an immediate consequence of the definition of a Noetherian module
  since a submodule of a submodule is a submodule.
  \item[(b)]
  \emph{Show that $M/M'$ is Noetherian if $M$ is Noetherian.}
  Every submodule of $M/M'$ has the form $M''/M'$
  where $M''$ is a submodule of $M$ with $M' \subseteq M'' \subseteq M$.
  Since $M$ is Noetherian, $M''$ is finitely generated,
  and the reduction of those generators mod $M'$ will generate $M''/M'$
  as a finitely generated module.
  \end{enumerate}
\item[(2)]
$(\Longleftarrow)$
  \begin{enumerate}
  \item[(a)]
  Given any submodule $M''$ of $M$.
  Then the image of $M''$ in $M/M'$ is finitely generated and
  $M'' \cap M'$ is finitely generated too.
  \item[(b)]
  Say $x_1, \ldots, x_k \in M''$ generate the image of $M''$ in $M/M'$ and
  say $y_1, \ldots, y_h \in M''$ generate $M'' \cap M'$.
  \item[(c)]
  Given any $x \in M''$, we have
  \begin{align*}
  &x \equiv r_1 x_1 + \cdots + r_k x_k \pmod{M'} \text{ for some } r_i \in R \\
  \Longrightarrow&
  x - \sum_{i=1}^{k} r_i x_k \equiv 0 \pmod{M'} \\
  \Longrightarrow&
  x - \sum_{i=1}^{k} r_i x_k \in M' \\
  \Longrightarrow&
  x - \sum_{i=1}^{k} r_i x_k \in M'' \cap M' \\
  \Longrightarrow&
  x - \sum_{i=1}^{k} r_i x_k = \sum_{j=1}^{h} s_j y_j \text{ for some } s_j \in R \\
  \Longrightarrow&
  x = \sum_{i=1}^{k} r_i x_k + \sum_{j=1}^{h} s_j y_j \\
  \Longrightarrow&
  \text{$x$ is generated by $x_1, \ldots, x_k, y_1, \ldots, y_h$}
  \end{align*}
  \end{enumerate}
  Hence $M''$ is finitely generated for any submodule $M''$ of $M$,
  that is, $M$ is Noetherian.
\end{enumerate}
$\Box$ \\\\



%%%%%%%%%%%%%%%%%%%%%%%%%%%%%%%%%%%%%%%%%%%%%%%%%%%%%%%%%%%%%%%%%%%%%%%%%%%%%%%%



\textbf{\large Algebra and Geometry} \\\\



\textbf{Exercise 1.8 (A formal Nullstellensatz).}
\emph{Let $\mathcal{X}$ and $\mathcal{J}$ be partially ordered sets,
and suppose that $I: \mathcal{X} \to \mathcal{J}$ and $Z: \mathcal{J} \to \mathcal{X}$
are functions such that}
\begin{quote}
  \begin{enumerate}
  \item[(i)]
  \emph{$I$ and $Z$ reverse the order in the sense that $x \leq y \in \mathcal{X}$
  implies $I(x) \geq I(y)$, and $i \leq j \in \mathcal{J}$ implies $Z(i) \geq Z(j)$.}
  \item[(ii)]
  \emph{$ZI$ and $IZ$ are increasing functions, in the sense that
  $x \in \mathcal{X}$ implies $ZI(x) \geq x$,
  and $i \in \mathcal{J}$ implies $IZ(i) \geq i$.}
  \end{enumerate}
\end{quote}
\begin{enumerate}
\item[(a)]
  \emph{Show that $I$ and $Z$ establish a one-to-one correspondence between
  the subsets $I(\mathcal{X}) \subseteq \mathcal{J}$ and $Z(\mathcal{J}) \subseteq \mathcal{X}$.}
\item[(b)]
  \emph{Let $k$ be a field.
  Call an ideal $I \subseteq k[x_1,\ldots,x_n]$ \textbf{formally radical}
  if it is of the form $I(X)$ for some set $X \subseteq k^n$.
  Use part (a) to prove that there is a one-to-one correspondence between
  formally radical ideals and algebraic subsets of $k^n$.
  (Hilbert's Nullstellensatz identifies the formally radical ideals
  with the ordinary radical ideals when $k$ is algebraically closed.)} \\
\end{enumerate}

\emph{Proof of (a).}
\begin{enumerate}
  \item[(1)]
  \emph{It suffices to show that $IZ$ is the identity map on $I(\mathcal{X})$
  and $ZI$ is the identity map on $Z(\mathcal{J})$.}
  By symmetry, it suffices to show the first statement.
  \item[(2)]
  Given any $y \in I(\mathcal{X})$, there exists $x \in \mathcal{X}$ such that $y = I(x)$.
  Take $IZ$ on the both sides, we have $IZ(y) \geq y$ by (ii).
  Hence $IZI(x) \geq I(x)$.
  \item[(3)]
  Besides, $ZI(x) \geq x$ by (ii). Take $I$ on the both sides,
  we have $I(x) \geq IZI(x)$ by (i).
  Since $\mathcal{J}$ is a partially ordered set,
  $I(x) = IZI(x)$
  or $y = IZ(y)$ for all $y \in I(\mathcal{X})$,
  or $IZ$ is the identity map on $I(\mathcal{X})$.
\end{enumerate}
$\Box$ \\

\emph{Proof of (b).}
\begin{enumerate}
  \item[(1)]
  Let
  \begin{align*}
    \mathcal{X} &= \{ \text{subsets } X \subseteq k^n \}, \\
    \mathcal{J} &= \{ \text{ideals } \mathfrak{a} \subseteq k[x_1,\ldots,x_n] \}.
  \end{align*}
  Define the partially order of $\mathcal{X}$ or $\mathcal{J}$ by the set inclusion.
  \item[(2)]
  Let $I: \mathcal{X} \to \mathcal{J}$ defined by
  \[
    I(X)
    = \{ f \in k[x_1,\ldots,x_n] : f(a_1,\ldots,a_n) = 0
      \: \forall \: (a_1, \ldots, a_n) \in X \}
  \]
  and $Z: \mathcal{J} \to \mathcal{X}$ defined by
  \[
    Z(\mathfrak{a})
    = \{(a_1, \ldots, a_n) \in k^n : f(a_1,\ldots,a_n) = 0
      \: \forall \: f \in \mathfrak{a} \}.
  \]
  \item[(3)]
  It is clear that
    \begin{enumerate}
    \item[(a)]
      $I(X) \supseteq I(Y)$ if $Y \supseteq X$.
    \item[(b)]
      $Z(\mathfrak{a}) \supseteq Z(\mathfrak{b})$ if $\mathfrak{b} \supseteq \mathfrak{a}$.
    \item[(c)]
      $ZI(X) \supseteq X$ and $IZ(\mathfrak{a}) \supseteq \mathfrak{a}$.
    \end{enumerate}
  \item[(4)]
  By (a),
  there a one-to-one correspondence between
  the subsets $I(\mathcal{X}) \subseteq \mathcal{J}$ and $Z(\mathcal{J}) \subseteq \mathcal{X}$,
  or there a one-to-one correspondence between
  formally radical ideals and algebraic subsets of $k^n$.
\end{enumerate}
$\Box$ \\\\



%%%%%%%%%%%%%%%%%%%%%%%%%%%%%%%%%%%%%%%%%%%%%%%%%%%%%%%%%%%%%%%%%%%%%%%%%%%%%%%%



\textbf{Exercise 1.9.}
\emph{Let $S = k[x_1,\ldots,x_r]$, with $k$ an algebraically closed field.
Show that under the correspondence of radical ideals in $S$ and
algebraic subsets of $\mathbb{A}^r$,
the primes ideals correspond to the algebraic sets that
cannot be written as a proper union of smaller algebraic sets.} \\

\emph{Proof.}
Let $I(X)$ be a prime ideal where $X$ is some subset of $k^n$.
\begin{enumerate}
  \item[(1)]
  (Reductio ad absurdum)
  If $X$ were a proper union of smaller algebraic sets, write $X = X_1 \cup X_2$
  where $X_1 \subsetneq X$ and $X_2 \subsetneq X$.
  \item[(2)]
  Therefore, $I(X_1) \supsetneq I(X)$ and $I(X_2) \supsetneq I(X)$ (Exercise 1.8).
  Now we can take $f \in I(X_1)-I(X)$ and $g \in I(X_2)-I(X)$.
  \item[(3)]
  By the definition of $I$,
  \begin{align*}
    f(a_1,\ldots,a_n) = 0 \: \forall \: (a_1, \ldots, a_n) \in X_1, \\
    g(a_1,\ldots,a_n) = 0 \: \forall \: (a_1, \ldots, a_n) \in X_2,
  \end{align*}
  or
  \[
    f(a_1, \ldots, a_n)
    g(a_1, \ldots, a_n) = 0 \: \forall \: (a_1, \ldots, a_n) \in X_1 \cup X_2 = X,
  \]
  or $fg \in I(X)$.
  \item[(4)]
  Since $I(X)$ is prime, $f \in I(X)$ or $g \in I(X)$,
  which is absurd.
\end{enumerate}
$\Box$ \\\\



%%%%%%%%%%%%%%%%%%%%%%%%%%%%%%%%%%%%%%%%%%%%%%%%%%%%%%%%%%%%%%%%%%%%%%%%%%%%%%%%



\textbf{Exercise 1.12.}
\emph{Find equations for a parabola meeting a circle just once
in the complex plane, represented by Figure 1.5 (see the textbook).} \\

\emph{Proof.}
\begin{equation*}
  \begin{cases}
    x^2 + (y-1)^2 - 1 = 0 \\
    x^2 - 2y = 0
  \end{cases}
\end{equation*}
meets at $(0,0)$.
$\Box$ \\\\



%%%%%%%%%%%%%%%%%%%%%%%%%%%%%%%%%%%%%%%%%%%%%%%%%%%%%%%%%%%%%%%%%%%%%%%%%%%%%%%%
%%%%%%%%%%%%%%%%%%%%%%%%%%%%%%%%%%%%%%%%%%%%%%%%%%%%%%%%%%%%%%%%%%%%%%%%%%%%%%%%



\end{document}