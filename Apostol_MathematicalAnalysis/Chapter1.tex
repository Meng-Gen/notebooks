\documentclass{article}
\usepackage{amsfonts}
\usepackage{amsmath}
\usepackage{amssymb}
\usepackage{hyperref}
\usepackage{mathrsfs}
\parindent=0pt

\def\upint{\mathchoice%
    {\mkern13mu\overline{\vphantom{\intop}\mkern7mu}\mkern-20mu}%
    {\mkern7mu\overline{\vphantom{\intop}\mkern7mu}\mkern-14mu}%
    {\mkern7mu\overline{\vphantom{\intop}\mkern7mu}\mkern-14mu}%
    {\mkern7mu\overline{\vphantom{\intop}\mkern7mu}\mkern-14mu}%
  \int}
\def\lowint{\mkern3mu\underline{\vphantom{\intop}\mkern7mu}\mkern-10mu\int}

\begin{document}

\textbf{\Large Chapter 1: The Real And Complex Number Systems} \\\\



\emph{Author: Meng-Gen Tsai} \\
\emph{Email: plover@gmail.com} \\\\



\textbf{\large Integers} \\\\



\textbf{Exercise 1.1}
\emph{Prove that there is no largest prime. (A proof was known to Euclid.)} \\

There are many proofs of this result. We provide some of them. \\

\emph{Proof (Due to Euclid).}
If
$p_1, p_2, ..., p_t$ were all primes, then
write $$n = p_1 p_2 \cdots p_t + 1$$
and there were a prime number $p$ dividing $n$.
\begin{enumerate}
\item[(1)]
$p$ can not be any of $p_i (1 \leq i \leq t)$,
otherwise $p$ would divide the difference $n - p_1 p_2 \cdots p_t = 1$.
\item[(2)]
This prime $p$ is another prime $\neq p_i$ for $1 \leq i \leq t$,
which is absurd.
\end{enumerate}
$\Box$ \\

\emph{Proof (Unique factorization theorem).}
Given $N$.
\begin{enumerate}
\item[(1)]
\emph{Show that $\sum_{n \leq N} \frac{1}{n}
\leq \prod_{p \leq N} \left( 1 - \frac{1}{p} \right)^{-1}$.} \\
By the unique factorization theorem on $n \leq N$,
$$\sum_{n \leq N} \frac{1}{n}
\leq \prod_{p \leq N} \left( 1 + \frac{1}{p} + \frac{1}{p^2} + \cdots \right)
= \prod_{p \leq N} \left( 1 - \frac{1}{p} \right)^{-1}.$$
\item[(2)]
By (1) and the fact that $\sum \frac{1}{n}$ diverges,
there are infinitely many primes.
\end{enumerate}
$\Box$ \\

\emph{Proof (Due to Eckford Cohen).}
\begin{enumerate}
\item[(1)]
\emph{$\text{ord}_p n!
= \left[\frac{n}{p}\right] + \left[\frac{n}{p^2}\right] + \left[\frac{n}{p^3}\right] + \cdots$.}
For any $k = 1, 2, ..., n$, we can express $k$ as $k = p^s t$
where $s = \text{ord}_p k$ is a non-negative integer and $(t, p) = 1$.
There are $[\frac{n}{p^a}]$ numbers such that $p^a \mid k$ for $a = 1, 2, ...$.
Therefore, there are $$\left[\frac{n}{p^a}\right] - \left[\frac{n}{p^{a+1}}\right]$$
numbers such that $\text{ord}_p k = a$ for $a = 1, 2, ...$. Hence,
\begin{align*}
\text{ord}_p n!
&= \left( \left[\frac{n}{p}\right] - \left[\frac{n}{p^2}\right] \right)
 + 2 \left( \left[\frac{n}{p^2}\right] - \left[\frac{n}{p^3}\right] \right)
 + 3 \left( \left[\frac{n}{p^3}\right] - \left[\frac{n}{p^4}\right] \right) + \cdots \\
&= \left[\frac{n}{p}\right] + \left[\frac{n}{p^2}\right] + \left[\frac{n}{p^3}\right] + \cdots.
\end{align*}
\item[(2)]
\emph{$\text{ord}_p n! \leq \frac{n}{p - 1}$ and that
$n!^{\frac{1}{n}} \leq \prod_{p|n!}p^{\frac{1}{p - 1}}$.}
\begin{align*}
\text{ord}_p n!
&= \left[\frac{n}{p}\right] + \left[\frac{n}{p^2}\right] + \left[\frac{n}{p^3}\right] + \cdots \\
&\leq \frac{n}{p} + \frac{n}{p^2} + \frac{n}{p^3} + \cdots \\
&= \frac{\frac{n}{p}}{1 - \frac{1}{p}} \\
&= \frac{n}{p - 1}.
\end{align*}
Thus,
$$n!
= \prod_{p|n!} p^{\text{ord}_p n!}
\leq \prod_{p|n!} p^{\frac{n}{p - 1}}
= \left( \prod_{p|n!} p^{\frac{1}{p - 1}} \right)^n, $$
or
$$n!^{\frac{1}{n}} \leq \prod_{p|n!}p^{\frac{1}{p - 1}}.$$
\item[(3)]
\emph{$(n!)^2 \geq n^n$.}
Write
$(n!)^2 = \prod_{k=1}^n k \prod_{k=1}^n (n + 1 - k) = \prod_{k=1}^n k(n + 1 - k)$,
and $n^n = \prod_{k=1}^n n$.
It suffices to show that $k(n + 1 - k) \geq n$ for each $1 \leq k \leq n$.
Notice that $k(n + 1 - k) - n = (n - k)(k - 1) \geq 0$ for $1 \leq k \leq n$.
The inequality holds.
\item[(4)]
By (3)(4), $\prod_{p|n!}p^{\frac{1}{p - 1}} \geq \sqrt{n}$.
Assume that there are finitely many primes,
the value $\prod_{p|n!} p^{\frac{1}{p - 1}}$ is a finite number
whenever the value of $n$.
However, $\sqrt{n} \rightarrow \infty$ as $n \rightarrow \infty$,
which leads to a contradiction.
Hence there are infinitely many primes.
\end{enumerate}
$\Box$ \\

\emph{Proof (Formula for $\phi(n)$).}
If
$p_1, p_2, ..., p_t$ were all primes, then let
$n = p_1 p_2 \cdots p_t$ and all numbers between $2$ and $n$ are
NOT relatively prime to $n$.
Thus, $\phi(n) = 1$ by the definition of $\phi$.
By the formula for $\phi$,
\begin{align*}
\phi(n)
&= n
\left( 1 - \frac{1}{p_1} \right)
\left( 1 - \frac{1}{p_2} \right)
\cdots
\left( 1 - \frac{1}{p_t} \right) \\
1
&= (p_1 p_2 \cdots p_t)
\left( 1 - \frac{1}{p_1} \right)
\left( 1 - \frac{1}{p_2} \right)
\cdots
\left( 1 - \frac{1}{p_t} \right) \\
&= (p_1 - 1)(p_2 - 1) \cdots (p_t - 1) > 1,
\end{align*}
which is a contradiction (since $3$ is a prime).
Hence there are infinitely many primes.
$\Box$ \\\\


\textbf{Exercise 1.2}
\emph{If $n$ is a positive integer, prove the algebraic identity
$$a^n - b^n = (a - b) \sum_{k=0}^{n-1} a^k b^{n-1-k}.$$} \\

\emph{Proof.}
\begin{enumerate}
\item[(1)]
\begin{align*}
(a - b) \sum_{k=0}^{n-1} a^k b^{n-1-k}
&= a \sum_{k=0}^{n-1} a^k b^{n-1-k} - b \sum_{k=0}^{n-1} a^k b^{n-1-k} \\
&= \sum_{k=0}^{n-1} a^{k+1} b^{n-1-k} - \sum_{k=0}^{n-1} a^k b^{n-k}.
\end{align*}
\item[(2)] Arrange index in summation symbols.
\begin{align*}
\sum_{k=0}^{n-1} a^{k+1} b^{n-1-k}
&= \sum_{k=1}^{n} a^{k} b^{n-k}
= a^n + \sum_{k=1}^{n-1} a^{k} b^{n-k}, \\
\sum_{k=0}^{n-1} a^k b^{n-k}
&= b^n + \sum_{k=1}^{n-1} a^{k} b^{n-k}.
\end{align*}
\item[(3)]
By (1)(2),
\begin{align*}
(a - b) \sum_{k=0}^{n-1} a^k b^{n-1-k}
&= \left( a^n + \sum_{k=1}^{n-1} a^{k} b^{n-k} \right)
- \left( b^n + \sum_{k=1}^{n-1} a^{k} b^{n-k} \right) \\
&= a^n - b^n.
\end{align*}
\end{enumerate}
$\Box$ \\

\textbf{Supplement.} Some exercises without proof.
\begin{enumerate}
\item[(1)]
\emph{Let $x$ be a nilpotent element of $A$.
Show that $1+x$ is a unit of $A$.
Deduce that the sum of a nilpotent element and a unit is a unit.}
(Exercise 1.1 in Atiyah and Macdonald,
Introduction to Commutative Algebra.)

\item[(2)]
\emph{Prove that $1^k + 2^k + \cdots + (p-1)^k \equiv 0 \: (p)$
if $p - 1 \nmid k$ and $-1 (p)$ if $p - 1 \mid k$.}
(Exercise 4.11 in Kenneth Ireland and Michael Rosen,
A Classical Introduction to Modern Number Theory, Second Edition)

\item[(3)]
\emph{Use the existence of a primitive root to give another proof
of Wilson's theorem $(p - 1)! \equiv -1 \: (p)$.}
(Exercise 4.12 in Kenneth Ireland and Michael Rosen,
A Classical Introduction to Modern Number Theory, Second Edition)

\item[(4)]
\emph{Suppose $n$ and $F$ are integers and $n, F > 0$. Show that
$$B_n(Fx) = F^{n-1} \sum_{a=0}^{F-1} B_n \left(x + \frac{a}{F} \right).$$
where $B_n(x)$ are Bernoulli polynomials.}
(Exercise 15.19 in Kenneth Ireland and Michael Rosen,
A Classical Introduction to Modern Number Theory, Second Edition)
\end{enumerate}
$\Box$ \\\\



\textbf{Exercise 1.3}
\emph{If $2^n - 1$ is a prime, prove that $n$ is prime.
A prime of the form $2^p - 1$, where $p$ is prime, is called a Mersenne prime.} \\

It suffices to prove that:
\emph{If $a^n - 1$ is a prime, show that $a = 2$ and that $n$ is a prime.}
Primes of the form $2^p - 1$ are called Mersenne primes.
For example, $2^3 - 1 = 7$ and $2^5 - 1 = 31$.
It is not known if there are infinitely many Mersenne primes. \\

\emph{Proof.}
\begin{enumerate}
\item[(1)]
\emph{$n$ is a prime.}
Assume $n$ were not prime, say $n = rs$ for some $r, s > 1$.
By Exercise 1.2,
$a^{rs} - 1 = (a^s - 1)(\sum_{k=0}^{r-1} a^{sk})$.
$a^s - 1 = 1$ since $a^s - 1 < a^{rs} - 1$ and $a^{rs} - 1$ is a prime.
Hence $s=1$ and ($a=2$), which is absurd.
\item[(2)]
\emph{$a = 2$.}
If $a$ is odd, then $a^p - 1 > 2$ is even, which is not a prime.
If $a > 2$ is even,
$a^p - 1 = (a - 1)(\sum_{k=0}^{p-1} a^k)$.
Both $a - 1 > 1$ and $\sum_{k=0}^{p-1} a^k > 1$, which is absurd.
\end{enumerate}
By (1)(2), $a = 2$ and that $n$ is a prime if $a^n - 1$ is a prime.
$\Box$ \\\\



\textbf{\large Rational and irrational numbers} \\\\



\textbf{Exercise 1.11}
\emph{Given any real $x > 0$,
prove that there is an irrational number between $0$ and $x$.} \\

\emph{Proof.}
There are only two possible cases: $x$ is rational, or $x$ is irrational.
\begin{enumerate}
\item[(1)]
\emph{$x$ is rational.}
Pick $y = \frac{x}{\sqrt{89}} \in (0, x) \subseteq \mathbb{R}$. $y$ is irrational.
\item[(2)]
\emph{$x$ is irrational.}
Pick $y = \frac{x}{\sqrt{64}} \in (0, x) \subseteq \mathbb{R}$. $y$ is irrational.
\end{enumerate}
$\Box$ \\

\emph{Proof (Exercise 4.12).}
Pick
$$y
= \lim_{m \rightarrow \infty}[\lim_{n \rightarrow \infty} \cos^{2n}(m!\pi x)]
\cdot \frac{x}{\sqrt{89}}
+
(1 - \lim_{m \rightarrow \infty}[\lim_{n \rightarrow \infty} \cos^{2n}(m!\pi x)])
\cdot \frac{x}{\sqrt{64}}.$$
\begin{enumerate}
\item[(1)]
\emph{$x$ is rational.}
$y = \frac{x}{\sqrt{89}} \in (0, x) \subseteq \mathbb{R}$ is irrational.
\item[(2)]
\emph{$x$ is irrational.}
$y = \frac{x}{\sqrt{64}} \in (0, x) \subseteq \mathbb{R}$ is irrational.
\end{enumerate}
$\Box$ \\\\



\end{document}