\documentclass{article}
\usepackage{amsfonts}
\usepackage{amsmath}
\usepackage{amssymb}
\usepackage{hyperref}
\usepackage{mathrsfs}
\parindent=0pt

\def\upint{\mathchoice%
    {\mkern13mu\overline{\vphantom{\intop}\mkern7mu}\mkern-20mu}%
    {\mkern7mu\overline{\vphantom{\intop}\mkern7mu}\mkern-14mu}%
    {\mkern7mu\overline{\vphantom{\intop}\mkern7mu}\mkern-14mu}%
    {\mkern7mu\overline{\vphantom{\intop}\mkern7mu}\mkern-14mu}%
  \int}
\def\lowint{\mkern3mu\underline{\vphantom{\intop}\mkern7mu}\mkern-10mu\int}

\begin{document}

\textbf{\Large Chapter 1: The Real And Complex Number Systems} \\\\



\emph{Author: Meng-Gen Tsai} \\
\emph{Email: plover@gmail.com} \\\\



\textbf{Exercise 1.1}
\emph{Prove that there is no largest prime. (A proof was known to Euclid.)} \\

There are many proofs of this result. We provide some of them. \\

\emph{Proof (Due to Euclid).}
If
$p_1, p_2, ..., p_t$ were all primes, then
write $$n = p_1 p_2 \cdots p_t + 1$$
and there were a prime number $p$ dividing $n$.
\begin{enumerate}
\item[(1)]
$p$ can not be any of $p_i (1 \leq i \leq t)$,
otherwise $p$ would divide the difference $n - p_1 p_2 \cdots p_t = 1$.
\item[(2)]
This prime $p$ is another prime $\neq p_i$ for $1 \leq i \leq t$.
\end{enumerate}
By (2), there exists a prime $p$ other than all primes, which is absurd.
$\Box$ \\

\emph{Proof (Unique factorization theorem).}
Given $N$.
\begin{enumerate}
\item[(1)]
\emph{Show that $\sum_{n \leq N} \frac{1}{n}
\leq \prod_{p \leq N} \left( 1 - \frac{1}{p} \right)^{-1}$.} \\
By the unique factorization theorem on $n \leq N$,
$$\sum_{n \leq N} \frac{1}{n}
\leq \prod_{p \leq N} \left( 1 + \frac{1}{p} + \frac{1}{p^2} + \cdots \right)
= \prod_{p \leq N} \left( 1 - \frac{1}{p} \right)^{-1}.$$
\item[(2)]
By (1) and the fact that $\sum \frac{1}{n}$ diverges,
there are infinitely many primes.
\end{enumerate}
$\Box$ \\

\emph{Proof (Due to Eckford Cohen).}
\begin{enumerate}
\item[(1)]
\emph{$\text{ord}_p n!
= \left[\frac{n}{p}\right] + \left[\frac{n}{p^2}\right] + \left[\frac{n}{p^3}\right] + \cdots$.}
For any $k = 1, 2, ..., n$, we can express $k$ as $k = p^s t$
where $s = \text{ord}_p k$ is a non-negative integer and $(t, p) = 1$.
There are $[\frac{n}{p^a}]$ numbers such that $p^a \mid k$ for $a = 1, 2, ...$.
Therefore, there are $$\left[\frac{n}{p^a}\right] - \left[\frac{n}{p^{a+1}}\right]$$
numbers such that $\text{ord}_p k = a$ for $a = 1, 2, ...$. Hence,
\begin{align*}
\text{ord}_p n!
&= \left( \left[\frac{n}{p}\right] - \left[\frac{n}{p^2}\right] \right)
 + 2 \left( \left[\frac{n}{p^2}\right] - \left[\frac{n}{p^3}\right] \right)
 + 3 \left( \left[\frac{n}{p^3}\right] - \left[\frac{n}{p^4}\right] \right) + \cdots \\
&= \left[\frac{n}{p}\right] + \left[\frac{n}{p^2}\right] + \left[\frac{n}{p^3}\right] + \cdots.
\end{align*}
\item[(2)]
\emph{$\text{ord}_p n! \leq \frac{n}{p - 1}$ and that
$n!^{\frac{1}{n}} \leq \prod_{p|n!}p^{\frac{1}{p - 1}}$.}
\begin{align*}
\text{ord}_p n!
&= \left[\frac{n}{p}\right] + \left[\frac{n}{p^2}\right] + \left[\frac{n}{p^3}\right] + \cdots \\
&\leq \frac{n}{p} + \frac{n}{p^2} + \frac{n}{p^3} + \cdots \\
&= \frac{\frac{n}{p}}{1 - \frac{1}{p}} \\
&= \frac{n}{p - 1}.
\end{align*}
Thus,
$$n!
= \prod_{p|n!} p^{\text{ord}_p n!}
\leq \prod_{p|n!} p^{\frac{n}{p - 1}}
= \left( \prod_{p|n!} p^{\frac{1}{p - 1}} \right)^n, $$
or
$$n!^{\frac{1}{n}} \leq \prod_{p|n!}p^{\frac{1}{p - 1}}.$$
\item[(3)]
\emph{$(n!)^2 \geq n^n$.}
Write
$(n!)^2 = \prod_{k=1}^n k \prod_{k=1}^n (n + 1 - k) = \prod_{k=1}^n k(n + 1 - k)$,
and $n^n = \prod_{k=1}^n n$.
It suffices to show that $k(n + 1 - k) \geq n$ for each $1 \leq k \leq n$.
Notice that $k(n + 1 - k) - n = (n - k)(k - 1) \geq 0$ for $1 \leq k \leq n$.
The inequality holds.
\item[(4)]
By (3)(4), $\prod_{p|n!}p^{\frac{1}{p - 1}} \geq \sqrt{n}$.
Assume that there are finitely many primes,
the value $\prod_{p|n!} p^{\frac{1}{p - 1}}$ is a finite number
whenever the value of $n$.
However, $\sqrt{n} \rightarrow \infty$ as $n \rightarrow \infty$,
which leads to a contradiction.
Hence there are infinitely many primes.
\end{enumerate}
$\Box$ \\

\emph{Proof (Formula for $\phi(n)$).}
If
$p_1, p_2, ..., p_t$ were all primes, then let
$n = p_1 p_2 \cdots p_t$ and all numbers between $2$ and $n$ are
NOT relatively prime to $n$.
Thus, $\phi(n) = 1$ by the definition of $\phi$.
By the formula for $\phi$,
\begin{align*}
\phi(n)
&= n
\left( 1 - \frac{1}{p_1} \right)
\left( 1 - \frac{1}{p_2} \right)
\cdots
\left( 1 - \frac{1}{p_t} \right) \\
1
&= (p_1 p_2 \cdots p_t)
\left( 1 - \frac{1}{p_1} \right)
\left( 1 - \frac{1}{p_2} \right)
\cdots
\left( 1 - \frac{1}{p_t} \right) \\
&= (p_1 - 1)(p_2 - 1) \cdots (p_t - 1) > 1,
\end{align*}
which is a contradiction (since $3$ is a prime).
Hence there are infinitely many primes.
$\Box$ \\

\end{document}