\documentclass{article}
\usepackage{amsfonts}
\usepackage{amsmath}
\usepackage{amssymb}
\usepackage{hyperref}
\usepackage{mathrsfs}
\parindent=0pt

\def\upint{\mathchoice%
    {\mkern13mu\overline{\vphantom{\intop}\mkern7mu}\mkern-20mu}%
    {\mkern7mu\overline{\vphantom{\intop}\mkern7mu}\mkern-14mu}%
    {\mkern7mu\overline{\vphantom{\intop}\mkern7mu}\mkern-14mu}%
    {\mkern7mu\overline{\vphantom{\intop}\mkern7mu}\mkern-14mu}%
  \int}
\def\lowint{\mkern3mu\underline{\vphantom{\intop}\mkern7mu}\mkern-10mu\int}

\begin{document}

\textbf{\Large Chapter 2: Some Basic Notions of Set Theory} \\\\



\emph{Author: Meng-Gen Tsai} \\
\emph{Email: plover@gmail.com} \\\\



\textbf{Exercise 2.6.}
\emph{Let $f: S \rightarrow T$ be a function.
If $A$ and $B$ are arbitrary subsets of $S$, prove that
$$f(A \cup B) = f(A) \cap f(B) \text{ and }
f(A \cap B) \subseteq f(A) \cup f(B).$$
Generalize to arbitrary unions and intersections.} \\

\textbf{Generalization.}
Let $f: S \rightarrow T$ be a function.
If $\mathscr{F}$ is an arbitrary collection of sets, then
$$f\left( \bigcup_{A \in \mathscr{F}} A \right)
= \bigcap_{A \in \mathscr{F}} f(A) \text{ and }
f\left( \bigcap_{A \in \mathscr{F}} A \right)
\subseteq \bigcup_{A \in \mathscr{F}} f(A).$$ \\

\emph{Note.}
$f(A \cap B)$ might not be equal to $f(A) \cup f(B)$.
For example, let $f: \mathbb{R} \rightarrow \mathbb{R}$ defined by $f(x) = 0$.
Then for any nonempty disjoint subsets $A$ and $B$,
we have $\varnothing = f(A \cap B) \not\supseteq f(A) \cup f(B) = \{0\}$. \\

\emph{Proof.}
\begin{enumerate}
\item[(1)]
\begin{align*}
  \forall \: y \in f\left( \bigcup_{A \in \mathscr{F}} A \right)
  &\Longleftrightarrow
  \exists \: x \in \bigcup_{A \in \mathscr{F}} A \text{ such that } f(x) = y \\
  &\Longleftrightarrow
  \exists \: x \in A \text{ for some } A \in \mathscr{F} \text{ such that } f(x) = y \\
  &\Longleftrightarrow
  \exists \: A \in \mathscr{F} \text{ such that } y \in f(A) \\
  &\Longleftrightarrow
  \forall \: y \in \bigcap_{A \in \mathscr{F}} f(A) \\
\end{align*}
\item[(2)]
\begin{align*}
  \forall \: y \in f\left( \bigcap_{A \in \mathscr{F}} A \right)
  \Longleftrightarrow&
  \exists \: x \in \bigcap_{A \in \mathscr{F}} A \text{ such that } f(x) = y \\
  \Longleftrightarrow&
  \exists \: x \text{ in all } A \in \mathscr{F} \text{ such that } f(x) = y \\
  &\text{ ($x$ not depending on $A$) } \\
  \Longrightarrow&
  \forall \: A \in \mathscr{F}, \exists \: x \in A \text{ such that } f(x) = y \\
  &\text{ ($x$ depending on $A$) } \\
  \Longleftrightarrow&
  \forall \: A \in \mathscr{F}, y \in f(A) \\
  \Longleftrightarrow&
  \forall \: y \in \bigcup_{A \in \mathscr{F}} f(A).
\end{align*}
\end{enumerate}
$\Box$ \\\\



\textbf{Exercise 2.7.}
\emph{Let $f: S \rightarrow T$ be a function.
If $Y \subseteq T$,
we denote by $f^{-1}(Y)$ the largest subset of $S$ which $f$ maps into $Y$.
That is,
$$f^{-1}(Y) = \{ x : x \in S \text{ and } f(x) \in Y \}.$$
The set $f^{-1}(Y)$ is called the inverse image of $Y$ under $f$.
Prove the following for arbitrary subsets $X$ of $S$ and $Y$ of $T$.}
\begin{enumerate}
\item[(a)]
\emph{$X \subseteq f^{-1}[f(X)]$.}
\item[(b)]
\emph{$f[f^{-1}(Y)] \subseteq Y$.}
\item[(c)]
\emph{$f^{-1}(Y_1 \cup Y_2) = f^{-1}(Y_1) \cup f^{-1}(Y_2)$.}
\item[(d)]
\emph{$f^{-1}(Y_1 \cap Y_2) = f^{-1}(Y_1) \cap f^{-1}(Y_2)$.}
\item[(e)]
\emph{$f^{-1}(T - Y) = S - f^{-1}(Y)$.}
\item[(f)]
\emph{Generalize (c) and (d) to arbitrary unions and intersections.} \\
\end{enumerate}

\emph{Proof of (a).}
\begin{align*}
  \forall \: x \in X
  &\Longrightarrow
  f(x) \in f(X)
    & \\
  &\Longleftrightarrow
  x \in f^{-1}[f(X)].
    &\text{(Definition of the inverse image)}
\end{align*}
$\Box$ \\

\emph{Proof of (b).}
\begin{align*}
  \forall \: y \in f[f^{-1}(Y)]
  &\Longleftrightarrow
  \exists \: x \in f^{-1}(Y) \text{ such that } y = f(x) \\
  &\Longleftrightarrow
  \exists \: x, f(x) \in Y \text{ such that } y = f(x) \\
  &\Longrightarrow
  \exists \: x, y = f(x) \in Y.
\end{align*}
$\Box$ \\

\emph{Proof of (c).}
\emph{For an arbitrary collection $\mathscr{F}$ of subsets $Y$ of $T$,
show that
$$f^{-1}\left( \bigcup_{Y \in \mathscr{F}} Y \right)
= \bigcup_{Y \in \mathscr{F}} f^{-1}(Y).$$}
\begin{align*}
  \forall \: x \in f^{-1}\left( \bigcup_{Y \in \mathscr{F}} Y \right)
  &\Longleftrightarrow
  f(x) \in \bigcup_{Y \in \mathscr{F}} Y \\
  &\Longleftrightarrow
  f(x) \in Y \text{ for some } Y \in \mathscr{F} \\
  &\Longleftrightarrow
  x \in f^{-1}(Y) \text{ for some } Y \in \mathscr{F} \\
  &\Longleftrightarrow
  x \in \bigcup_{Y \in \mathscr{F}} f^{-1}(Y).
\end{align*}
$\Box$ \\

\emph{Proof of (d).}
Similar to (c).
\emph{For an arbitrary collection $\mathscr{F}$ of subsets $Y$ of $T$,
show that
$$f^{-1}\left( \bigcap_{Y \in \mathscr{F}} Y \right)
= \bigcap_{Y \in \mathscr{F}} f^{-1}(Y).$$}
\begin{align*}
  \forall \: x \in f^{-1}\left( \bigcap_{Y \in \mathscr{F}} Y \right)
  &\Longleftrightarrow
  f(x) \in \bigcap_{Y \in \mathscr{F}} Y \\
  &\Longleftrightarrow
  f(x) \in Y \text{ for all } Y \in \mathscr{F} \\
  &\Longleftrightarrow
  x \in f^{-1}(Y) \text{ for all } Y \in \mathscr{F} \\
  &\Longleftrightarrow
  x \in \bigcap_{Y \in \mathscr{F}} f^{-1}(Y).
\end{align*}
$\Box$ \\

\emph{Proof of (e).}
\begin{align*}
  \forall \: x \in f^{-1}(T - Y)
  &\Longleftrightarrow
  f(x) \in T - Y \\
  &\Longleftrightarrow
  f(x) \not\in Y \\
  &\Longleftrightarrow
  x \not\in f^{-1}(Y) \\
  &\Longleftrightarrow
  x \in S - f^{-1}(Y).
\end{align*}
$\Box$ \\

\emph{Proof of (f).}
Proved in (c)(d).
$\Box$ \\\\



\textbf{Exercise 2.15.}
\emph{A real number is called algebraic
if it is a root of an algebraic equation $f(x) = 0$,
where $a_0 + a_1 x + \cdots + a_n x^n = 0$ is a polynomial with integer coefficients.
Prove that the set of all polynomials with integer coefficients is countable
and deduce that the set of algebraic numbers is also countable.} \\

Might assume $a_n \neq 0$. \\

For example, all rational numbers are algebraic
since $p = \frac{\alpha}{\beta}$ (where $\alpha, \beta \in \mathbb{Z}$)
is a root of $\beta x - \alpha = 0$. \\

Besides, $x = \sqrt{2} + \sqrt{3}$ is algebraic since $x^4 - 10x^2 + 1 = 0$.
In fact, $x = \pm\sqrt{2} + \pm\sqrt{3}$ are also algebraic since
$x^4 - 10x^2 + 1 =
(x - \sqrt{2} - \sqrt{3})(x + \sqrt{2} - \sqrt{3})
(x - \sqrt{2} + \sqrt{3})(x + \sqrt{2} + \sqrt{3})$. \\

\textbf{Note.} \emph{Countable set} in the sense of Tom M. Apostol
is equivalent to \emph{at most countable set} in the sense of Walter Rudin. \\

\textbf{Lemma.}
\emph{The set of all polynomials over $\mathbb{Z}$ is countable implies that
the set of algebraic numbers is countable.} \\

\emph{Proof of Lemma.}
By definition, we write the set of algebraic numbers as
$$S = \bigcup_{f(x) \in \mathbb{Z}[x]} \{ \alpha \in \mathbb{R} : f(\alpha) = 0 \}.$$
Since each polynomial of degree $n$ has at most $n$ roots,
$\{ \alpha \in \mathbb{R} : f(\alpha) = 0 \}$ is finite (or countable)
for each given $f(x) \in \mathbb{Z}[x]$.
So $S$ is a countable union (by assumption) of countable sets, and hence countable
by Theorem 2.27.
$\Box$ \\

Now we show that
\emph{the set of all polynomials over $\mathbb{Z}$ is countable.} \\

\emph{Proof (Walter Rudin).}
For every positive integer $N$ there are only finitely many equations with
$n + |a_0| + |a_1| + \cdots + |a_n| = N.$
Write
$$P_N = \{ f(x) \in \mathbb{Z}[x] : n + |a_0| + |a_1| + \cdots + |a_n| = N \}$$
where $f(x) = a_0 + a_1 x + \cdots + a_n x^n$ with $a_n \neq 0$,
and
$$P = \bigcup_{N = 1}^{\infty} P_N.$$
$P$ is the set of all polynomials over $\mathbb{Z}$. \\

Each $P_N$ is finite (or countable) for given $N$
(since the equation $n + |a_0| + |a_1| + \cdots + |a_n| = N$
has finitely many solutions
$(n, a_0, a_1, ..., a_n) \in \mathbb{Z}^{n+2}$).
So $P$ is a countable union of countable sets, and hence countable
by Theorem 2.27.
$\Box$ \\

\emph{Proof (Theorem 2.18).}
\begin{enumerate}
\item[(1)]
\emph{$\mathbb{Z}^N$ is countable for any integer $N > 0$.}
Induction on $N$ and apply the same argument of Theorem 2.18.
\item[(2)]
\emph{The set of all polynomials over $\mathbb{Z}$ is countable.}
Let
$$P_n = \{ f \in \mathbb{Z}[x] : \deg f = n \},$$
and
$$P = \bigcup_{n = 1}^{\infty} P_n = \mathbb{Z}[x].$$

\emph{Claim: $P_n$ is countable.}
Define a one-to-one map $\varphi_n: P_n \rightarrow \mathbb{Z}^{n+1}$ by
$$\varphi_n(a_0 + a_1 x + \cdots + a_n x^n)
= (a_0, a_1, ..., a_n).$$
By (1) and Theorem 2.16, $P_n$ is countable.
Now $P$ is a countable union of countable sets,
and hence countable by Theorem 2.27.
\end{enumerate}
$\Box$ \\

\emph{Proof (Unique factorization theorem).}
\begin{enumerate}
\item[(1)]
\emph{The set of prime numbers is countable.}
Write all primes in the ascending order as $p_1, p_2, ..., p_n, ...$
where $p_1 = 2, p_2 = 3, ..., p_{10001} = 104743, ...$
(See \href{https://projecteuler.net/problem=7}{ProjectEuler 7: 10001st prime}.
Use sieve of Eratosthenes to get $p_{10001}$.)
\item[(2)]
\emph{The set of all polynomials over $\mathbb{Z}$ is countable.}
Let
$$P_n = \{ f \in \mathbb{Z}[x] : \deg f = n \},$$
and
$$P = \bigcup_{n = 1}^{\infty} P_n = \mathbb{Z}[x].$$

\emph{Claim: $P_n$ is countable.}
Define a map $\varphi_n: P_n \rightarrow \mathbb{Z}^+$ by
$$\varphi_n(a_0 + a_1 x + \cdots + a_n x^n)
= p_1^{\psi(a_0)} p_2^{\psi(a_1)} \cdots p_{n+1}^{\psi(a_n)},$$
where $\psi$ is a one-to-one correspondence from $\mathbb{Z}$ to $\mathbb{Z}^+$.
By the unique factorization theorem, $\varphi_n$ is one-to-one.
So $P_n$ is countable by Theorem 2.16.
Now $P$ is a countable union of countable sets,
and hence countable by Theorem 2.27.
\end{enumerate}
$\Box$ \\\\



\end{document}