\documentclass{article}
\usepackage{amsfonts}
\usepackage{amsmath}
\usepackage{amssymb}
\usepackage{centernot}
\usepackage{hyperref}
\usepackage[none]{hyphenat}
\usepackage{mathrsfs}
\usepackage{mathtools}
\usepackage{physics}
\usepackage{tikz-cd}
\parindent=0pt



\title{\textbf{Solutions to the book: \\\emph{Apostol, Mathematical Analysis, 2nd edition}}}
\author{Meng-Gen Tsai \\ plover@gmail.com}



\begin{document}
\maketitle
\tableofcontents



%%%%%%%%%%%%%%%%%%%%%%%%%%%%%%%%%%%%%%%%%%%%%%%%%%%%%%%%%%%%%%%%%%%%%%%%%%%%%%%%
%%%%%%%%%%%%%%%%%%%%%%%%%%%%%%%%%%%%%%%%%%%%%%%%%%%%%%%%%%%%%%%%%%%%%%%%%%%%%%%%



% Reference:



%%%%%%%%%%%%%%%%%%%%%%%%%%%%%%%%%%%%%%%%%%%%%%%%%%%%%%%%%%%%%%%%%%%%%%%%%%%%%%%%
%%%%%%%%%%%%%%%%%%%%%%%%%%%%%%%%%%%%%%%%%%%%%%%%%%%%%%%%%%%%%%%%%%%%%%%%%%%%%%%%
%%%%%%%%%%%%%%%%%%%%%%%%%%%%%%%%%%%%%%%%%%%%%%%%%%%%%%%%%%%%%%%%%%%%%%%%%%%%%%%%



\newpage
\section*{Chapter 1: The Real And Complex Number Systems \\}
\addcontentsline{toc}{section}{Chapter 1: The Real And Complex Number Systems}



\subsection*{Integers \\}
\addcontentsline{toc}{subsection}{Integers}



\subsubsection*{Exercise 1.1.}
\addcontentsline{toc}{subsubsection}{Exercise 1.1.}
\emph{Prove that there is no largest prime. (A proof was known to Euclid.)} \\

There are many proofs of this result. We provide some of them. \\

\emph{Proof (Due to Euclid).}
If
$p_1, p_2, \ldots, p_t$ were all primes, then
we consider $$n = p_1 p_2 \cdots p_t + 1.$$
Thus there is a prime number $p$ dividing $n$.
$p$ can not be any of $p_i$ for $1 \leq i \leq t$;
otherwise $p$ would divide the difference $n - p_1 p_2 \cdots p_t = 1$.
That is, $p \neq p_i$ for $1 \leq i \leq t$,
contrary to the assumption.
$\Box$ \\



\textbf{Supplement (Due to Euclid).}
\begin{enumerate}
\item[(1)]
  \emph{Show that $k[x]$, with $k$ a field,
  has infinitely many irreducible polynomials.}
  If
  $f_1, f_2, \ldots, f_t$ were all irreducible polynomials, then
  we consider $$g = f_1 f_2 \cdots f_t + 1 \in k[x].$$
  So there is an irreducible polynomial $f$ dividing $g$
  (since $\deg g = \deg f_1 + \deg f_2 + \cdots + \deg f_t \geq 1$).
  $f$ can not be any of $c_i f_i$ for $1 \leq i \leq t$ and $c_i \in k - \{0\}$;
  otherwise $f$ would divide the difference $g - f_1 f_2 \cdots f_t = 1$.
  That is, $f \neq c_i f_i$ for $1 \leq i \leq t$ and $c_i \in k - \{0\}$,
  contrary to the assumption.

\item[(2)]
  \emph{Show that any algebraically closed field is infinite.}
  Let $k$ be an algebraically closed field.
  If $a_1, \ldots, a_n$ were all elements in $k$, then
  we consider a monic polynomials
  \[
    F(X) = (X - a_1) \cdots (X - a_n) + 1 \in k[X].
  \]
  Since $k$ is algebraically closed,
  there is an element $a \in k$ such that $F(a) = 0$.
  By assumption, $a = a_i$ for some $1 \leq i \leq n$,
  and thus $F(a) = F(a_i) = 1$, contrary to the fact that
  a field is a commutative ring where $0 \neq 1$ and all nonzero elements are invertible.
\end{enumerate}
$\Box$ \\



\emph{Proof (Unique factorization theorem).}
Given $N$.
\begin{enumerate}
\item[(1)]
  \emph{Show that $\sum_{n \leq N} \frac{1}{n}
  \leq \prod_{p \leq N} \left( 1 - \frac{1}{p} \right)^{-1}$.} \\
  By the unique factorization theorem on $n \leq N$,
  \[
    \sum_{n \leq N} \frac{1}{n}
    \leq \prod_{p \leq N} \left( 1 + \frac{1}{p} + \frac{1}{p^2} + \cdots \right)
    = \prod_{p \leq N} \left( 1 - \frac{1}{p} \right)^{-1}.
  \]

\item[(2)]
  By (1) and the fact that $\sum \frac{1}{n}$ diverges,
  there are infinitely many primes.
\end{enumerate}
$\Box$ \\



\emph{Proof (Due to Eckford Cohen).}
\begin{enumerate}
\item[(1)]
  \emph{$\text{ord}_p n!
  = \left[\frac{n}{p}\right] + \left[\frac{n}{p^2}\right] + \left[\frac{n}{p^3}\right] + \cdots$.}
  For any $k = 1, 2, \ldots, n$, we can express $k$ as $k = p^s t$
  where $s = \text{ord}_p k$ is a non-negative integer and $(t, p) = 1$.
  There are $[\frac{n}{p^a}]$ numbers such that $p^a \mid k$ for $a = 1, 2, \ldots$.
  Therefore, there are $$\left[\frac{n}{p^a}\right] - \left[\frac{n}{p^{a+1}}\right]$$
  numbers such that $\text{ord}_p k = a$ for $a = 1, 2, \ldots$. Hence,
  \begin{align*}
  \text{ord}_p n!
    &= \left( \left[\frac{n}{p}\right] - \left[\frac{n}{p^2}\right] \right)
     + 2 \left( \left[\frac{n}{p^2}\right] - \left[\frac{n}{p^3}\right] \right)
     + 3 \left( \left[\frac{n}{p^3}\right] - \left[\frac{n}{p^4}\right] \right) + \cdots \\
    &= \left[\frac{n}{p}\right] + \left[\frac{n}{p^2}\right] + \left[\frac{n}{p^3}\right] + \cdots.
  \end{align*}

\item[(2)]
  \emph{$\text{ord}_p n! \leq \frac{n}{p - 1}$ and that
  $n!^{\frac{1}{n}} \leq \prod_{p|n!}p^{\frac{1}{p - 1}}$.}
  \begin{align*}
  \text{ord}_p n!
  &= \left[\frac{n}{p}\right] + \left[\frac{n}{p^2}\right] + \left[\frac{n}{p^3}\right] + \cdots \\
  &\leq \frac{n}{p} + \frac{n}{p^2} + \frac{n}{p^3} + \cdots \\
  &= \frac{\frac{n}{p}}{1 - \frac{1}{p}} \\
  &= \frac{n}{p - 1}.
  \end{align*}
  Thus,
  \[
    n!
    = \prod_{p|n!} p^{\text{ord}_p n!}
    \leq \prod_{p|n!} p^{\frac{n}{p - 1}}
    = \left( \prod_{p|n!} p^{\frac{1}{p - 1}} \right)^n,
  \]
  or
  \[
    n!^{\frac{1}{n}} \leq \prod_{p|n!}p^{\frac{1}{p - 1}}.
  \]

\item[(3)]
  \emph{$(n!)^2 \geq n^n$.}
  Write
  $(n!)^2 = \prod_{k=1}^n k \prod_{k=1}^n (n + 1 - k) = \prod_{k=1}^n k(n + 1 - k)$,
  and $n^n = \prod_{k=1}^n n$.
  It suffices to show that $k(n + 1 - k) \geq n$ for each $1 \leq k \leq n$.
  Notice that $k(n + 1 - k) - n = (n - k)(k - 1) \geq 0$ for $1 \leq k \leq n$.
  The inequality holds.

\item[(4)]
  By (3)(4), $\prod_{p|n!}p^{\frac{1}{p - 1}} \geq \sqrt{n}$.
  Assume that there are finitely many primes,
  the value $\prod_{p|n!} p^{\frac{1}{p - 1}}$ is a finite number
  whenever the value of $n$.
  However, $\sqrt{n} \rightarrow \infty$ as $n \rightarrow \infty$,
  which leads to a contradiction.
  Hence there are infinitely many primes.
\end{enumerate}
$\Box$ \\



\emph{Proof (Formula for $\phi(n)$).}
If
$p_1, p_2, \ldots, p_t$ were all primes, then let
$n = p_1 p_2 \cdots p_t$ and all numbers between $2$ and $n$ are
NOT relatively prime to $n$.
Thus, $\phi(n) = 1$ by the definition of $\phi$.
By the formula for $\phi$,
\begin{align*}
  \phi(n)
  &= n
  \left( 1 - \frac{1}{p_1} \right)
  \left( 1 - \frac{1}{p_2} \right)
  \cdots
  \left( 1 - \frac{1}{p_t} \right) \\
  1
  &= (p_1 p_2 \cdots p_t)
  \left( 1 - \frac{1}{p_1} \right)
  \left( 1 - \frac{1}{p_2} \right)
  \cdots
  \left( 1 - \frac{1}{p_t} \right) \\
  &= (p_1 - 1)(p_2 - 1) \cdots (p_t - 1) > 1,
\end{align*}
which is a contradiction (since $3$ is a prime).
Hence there are infinitely many primes.
$\Box$ \\\\



%%%%%%%%%%%%%%%%%%%%%%%%%%%%%%%%%%%%%%%%%%%%%%%%%%%%%%%%%%%%%%%%%%%%%%%%%%%%%%%%



\subsubsection*{Exercise 1.2.}
\addcontentsline{toc}{subsubsection}{Exercise 1.2.}
\emph{If $n$ is a positive integer, prove the algebraic identity
$$a^n - b^n = (a - b) \sum_{k=0}^{n-1} a^k b^{n-1-k}.$$} \\

\emph{Proof.}
\begin{enumerate}
\item[(1)]
\begin{align*}
(a - b) \sum_{k=0}^{n-1} a^k b^{n-1-k}
&= a \sum_{k=0}^{n-1} a^k b^{n-1-k} - b \sum_{k=0}^{n-1} a^k b^{n-1-k} \\
&= \sum_{k=0}^{n-1} a^{k+1} b^{n-1-k} - \sum_{k=0}^{n-1} a^k b^{n-k}.
\end{align*}
\item[(2)] Arrange summation index:
\begin{align*}
\sum_{k=0}^{n-1} a^{k+1} b^{n-1-k}
&= \sum_{k=1}^{n} a^{k} b^{n-k}
= a^n + \sum_{k=1}^{n-1} a^{k} b^{n-k}, \\
\sum_{k=0}^{n-1} a^k b^{n-k}
&= b^n + \sum_{k=1}^{n-1} a^{k} b^{n-k}.
\end{align*}
\item[(3)]
By (1)(2),
\begin{align*}
(a - b) \sum_{k=0}^{n-1} a^k b^{n-1-k}
&= \left( a^n + \sum_{k=1}^{n-1} a^{k} b^{n-k} \right)
- \left( b^n + \sum_{k=1}^{n-1} a^{k} b^{n-k} \right) \\
&= a^n - b^n.
\end{align*}
\end{enumerate}
$\Box$ \\

\textbf{Supplement.} Some exercises without proof.
\begin{enumerate}
\item[(1)]
\emph{Let $x$ be a nilpotent element of $A$.
Show that $1+x$ is a unit of $A$.
Deduce that the sum of a nilpotent element and a unit is a unit.}
(Exercise 1.1 in the textbook: \emph{Atiyah and Macdonald,
Introduction to Commutative Algebra})

\item[(2)]
\emph{Prove that $1^k + 2^k + \cdots + (p-1)^k \equiv 0 \: (p)$
if $p - 1 \nmid k$ and $-1 (p)$ if $p - 1 \mid k$.}
(Exercise 4.11 in the textbook: \emph{Kenneth Ireland and Michael Rosen,
A Classical Introduction to Modern Number Theory, 2nd edition})

\item[(3)]
\emph{Use the existence of a primitive root to give another proof
of Wilson's theorem $(p - 1)! \equiv -1 \: (p)$.}
(Exercise 4.12 in the textbook: \emph{Kenneth Ireland and Michael Rosen,
A Classical Introduction to Modern Number Theory, 2nd edition})

\item[(4)]
\emph{Suppose $n$ and $F$ are integers and $n, F > 0$. Show that
$$B_n(Fx) = F^{n-1} \sum_{a=0}^{F-1} B_n \left(x + \frac{a}{F} \right).$$
where $B_n(x)$ are Bernoulli polynomials.}
(Exercise 15.19 in the textbook: \emph{Kenneth Ireland and Michael Rosen,
A Classical Introduction to Modern Number Theory, 2nd edition})

\item[(5)]
Exercise 1.3.

\item[(6)]
Exercise 1.4.
\end{enumerate}
$\Box$ \\\\



%%%%%%%%%%%%%%%%%%%%%%%%%%%%%%%%%%%%%%%%%%%%%%%%%%%%%%%%%%%%%%%%%%%%%%%%%%%%%%%%


\subsubsection*{Exercise 1.3.}
\addcontentsline{toc}{subsubsection}{Exercise 1.3.}
\emph{If $2^n - 1$ is a prime, prove that $n$ is prime.
A prime of the form $2^p - 1$, where $p$ is prime, is called a Mersenne prime.} \\

It suffices to prove that:
\emph{If $a^n - 1$ is a prime, show that $a = 2$ and that $n$ is a prime.}
Primes of the form $2^p - 1$ are called Mersenne primes.
For example, $2^3 - 1 = 7$ and $2^5 - 1 = 31$.
It is not known if there are infinitely many Mersenne primes. \\

\emph{Proof.}
\begin{enumerate}
\item[(1)]
\emph{$n$ is a prime.}
Assume $n$ were not prime, say $n = rs$ for some $r, s > 1$.
By Exercise 1.2,
$a^{rs} - 1 = (a^s - 1)(\sum_{k=0}^{r-1} a^{sk})$.
$a^s - 1 = 1$ since $a^s - 1 < a^{rs} - 1$ and $a^{rs} - 1$ is a prime.
Hence $s=1$ and ($a=2$), which is absurd.
\item[(2)]
\emph{$a = 2$.}
If $a$ is odd, then $a^p - 1 > 2$ is even, which is not a prime.
If $a > 2$ is even,
$a^p - 1 = (a - 1)(\sum_{k=0}^{p-1} a^k)$.
Both $a - 1 > 1$ and $\sum_{k=0}^{p-1} a^k > 1$, which is absurd.
\end{enumerate}
By (1)(2), $a = 2$ and that $n$ is a prime if $a^n - 1$ is a prime.
$\Box$ \\\\



%%%%%%%%%%%%%%%%%%%%%%%%%%%%%%%%%%%%%%%%%%%%%%%%%%%%%%%%%%%%%%%%%%%%%%%%%%%%%%%%



\subsubsection*{Exercise 1.6.}
\addcontentsline{toc}{subsubsection}{Exercise 1.6.}
\emph{Prove that every nonempty set of positive integers contains a smallest member.
This is called the well-ordering principle.} \\

\emph{Proof.}
Use mathematical induction to establish
that the well-ordering principle.

\begin{enumerate}
\item[(1)]
Given a set $S$ of positive integers,
let $P(n)$ be the proposition
`If $m \in S$ for some $m \leq n$, then $S$ has a least element'.
Want to show $P(n)$ is true for all $n \in \mathbb{N}$.
\begin{enumerate}
\item[(a)]
$P(1)$ is true.
For $m \in S$ with $m \leq n = 1$,
or $m = 1$ by the minimality of $1 \in \mathbb{N}$,
$S$ has a least element $1$ ($m$ itself) in $\mathbb{N}$.
\item[(b)]
Suppose $P(n)$ is true.
If $n+1 \in S$, then there are only two possible cases.
  \begin{enumerate}
  \item[(i)]
  There is a positive integer $m \in S$ less than $n+1$.
  So $n \geq m \in S$.
  Since $P(n)$ is true, $S$ has a least element.
  \item[(ii)]
  There is no positive integer $m \in S$ less than $n+1$.
  In this case $n+1$ is the least element in $S$.
  \end{enumerate}
  In any cases (i)(ii), $S$ has a least element, or $P(n+1)$ is true.
\end{enumerate}
By mathematical induction, $P(n)$ is true for all $n \in \mathbb{N}$.
\item[(2)]
\emph{Show that the well-ordering principle holds.}
Let $T$ be a nonempty subset of $\mathbb{N}$,
so there exists a positive integer $k \in T$.
Notice that $P(k)$ is true by (1),
thus $T$ has a least element since $k \leq k$.
\end{enumerate}
$\Box$ \\

\textbf{Supplement.}
\emph{Show that the well-ordering principle implies
the principle of mathematical induction.} \\

\emph{Proof.}
Suppose that
\begin{enumerate}
\item[(1)]
$P(n)$ be a proposition defined for each $n \in \mathbb{N}$,
\item[(2)]
$P(1)$ is true,
\item[(3)]
$[P(n) \Rightarrow P(n+1)]$ is true.
\end{enumerate}

Consider the set
$$S = \{ n \in \mathbb{N} : P(n) \text{ is false} \} \subseteq \mathbb{N}.$$
Want to show
\emph{$S$ is empty, or the principle of mathematical induction holds.}
If $S$ were nonempty,
by the well-ordering principle $S$ has a smallest element $m$.
$m$ cannot be $1$ by (2).
Say $m > 1$.
Therefore, $m - 1 \in \mathbb{N}$
and $P(m-1)$ is true by the minimality of $m$.
By (3), $P((m-1)+1) = P(m)$ is true, which is absurd.
$\Box$ \\\\



%%%%%%%%%%%%%%%%%%%%%%%%%%%%%%%%%%%%%%%%%%%%%%%%%%%%%%%%%%%%%%%%%%%%%%%%%%%%%%%%
%%%%%%%%%%%%%%%%%%%%%%%%%%%%%%%%%%%%%%%%%%%%%%%%%%%%%%%%%%%%%%%%%%%%%%%%%%%%%%%%



\subsection*{Rational and irrational numbers \\}
\addcontentsline{toc}{subsection}{Rational and irrational numbers}



\subsubsection*{Exercise 1.11.}
\addcontentsline{toc}{subsubsection}{Exercise 1.11.}
\emph{Given any real $x > 0$,
prove that there is an irrational number between $0$ and $x$.} \\



\emph{Proof.}
There are only two possible cases: $x$ is rational, or $x$ is irrational.
\begin{enumerate}
\item[(1)]
\emph{$x$ is rational.}
Pick $y = \frac{x}{\sqrt{89}} \in (0, x) \subseteq \mathbb{R}$. $y$ is irrational.
\item[(2)]
\emph{$x$ is irrational.}
Pick $y = \frac{x}{\sqrt{64}} \in (0, x) \subseteq \mathbb{R}$. $y$ is irrational.
\end{enumerate}
$\Box$ \\

\emph{Proof (Exercise 4.12).}
Pick
$$y
= \lim_{m \rightarrow \infty}[\lim_{n \rightarrow \infty} \cos^{2n}(m!\pi x)]
\cdot \frac{x}{\sqrt{89}}
+
(1 - \lim_{m \rightarrow \infty}[\lim_{n \rightarrow \infty} \cos^{2n}(m!\pi x)])
\cdot \frac{x}{\sqrt{64}}.$$
\begin{enumerate}
\item[(1)]
\emph{$x$ is rational.}
$y = \frac{x}{\sqrt{89}} \in (0, x) \subseteq \mathbb{R}$ is irrational.
\item[(2)]
\emph{$x$ is irrational.}
$y = \frac{x}{\sqrt{64}} \in (0, x) \subseteq \mathbb{R}$ is irrational.
\end{enumerate}
$\Box$ \\\\



%%%%%%%%%%%%%%%%%%%%%%%%%%%%%%%%%%%%%%%%%%%%%%%%%%%%%%%%%%%%%%%%%%%%%%%%%%%%%%%%
%%%%%%%%%%%%%%%%%%%%%%%%%%%%%%%%%%%%%%%%%%%%%%%%%%%%%%%%%%%%%%%%%%%%%%%%%%%%%%%%



\subsection*{Upper bounds \\}
\addcontentsline{toc}{subsection}{Upper bounds}



%%%%%%%%%%%%%%%%%%%%%%%%%%%%%%%%%%%%%%%%%%%%%%%%%%%%%%%%%%%%%%%%%%%%%%%%%%%%%%%%
%%%%%%%%%%%%%%%%%%%%%%%%%%%%%%%%%%%%%%%%%%%%%%%%%%%%%%%%%%%%%%%%%%%%%%%%%%%%%%%%



\subsection*{Inequalities \\}
\addcontentsline{toc}{subsection}{Inequalities}



\subsubsection*{Exercise 1.23.}
\addcontentsline{toc}{subsubsection}{Exercise 1.23.}
\emph{Prove Lagrange's identity for real numbers:
$$\left( \sum_{k=1}^{n} a_k b_k \right)^2
= \left( \sum_{k=1}^{n} a_k \right)^2 \left( \sum_{k=1}^{n} b_k \right)^2
- \sum_{1 \leq k < j \leq n}
(a_k b_j - a_j b_k)^2.$$}

Note that this identity implies the Cauchy-Schwarz inequality. \\

\emph{Proof.}
Put $(a_k, b_k, A_k, B_k) \mapsto (a_k, b_k, a_k, b_k)$
in the following generalization (Binet-Cauchy identity).
$\Box$ \\

\textbf{Generalization (Binet-Cauchy identity).}
\begin{align*}
&\sum_{1 \leq k < j \leq n}
(a_k b_j - a_j b_k)(A_k B_j - A_j B_k) \\
= &\left( \sum_{k=1}^{n} a_k A_k \right)\left( \sum_{k=1}^{n} b_k B_k \right)
- \left( \sum_{k=1}^{n} a_k B_k \right)\left( \sum_{k=1}^{n} b_k A_k \right).
\end{align*}

\emph{Proof.}
\begin{align*}
&\sum_{1 \leq k < j \leq n}
(a_k b_j - a_j b_k)(A_k B_j - A_j B_k) \\
= &\sum_{1 \leq k < j \leq n}
(a_k b_j A_k B_j + a_j b_k A_j B_k)
- \sum_{1 \leq k < j \leq n}
(a_k b_j A_j B_k - a_j b_k A_k B_j) \\
= &\sum_{1 \leq k < j \leq n}
(a_k A_k b_j B_j + a_j A_j b_k B_k)
- \sum_{1 \leq k < j \leq n}
(a_k B_k b_j A_j + a_j B_j b_k A_k) \\
= &\sum_{1 \leq k \neq j \leq n} a_k A_k b_j B_j
 - \sum_{1 \leq k \neq j \leq n} a_k B_k b_j A_j \\
= &\sum_{1 \leq k, j \leq n} a_k A_k b_j B_j
 - \sum_{1 \leq k, j \leq n} a_k B_k b_j A_j \\
  & \text{(since $a_k A_k b_j B_j - a_k B_k b_j A_j = 0$ as $k = j$)} \\
= &\left( \sum_{k=1}^{n} a_k A_k \right)\left( \sum_{j=1}^{n} b_j B_j \right)
- \left( \sum_{k=1}^{n} a_k B_k \right)\left( \sum_{j=1}^{n} b_j A_j \right) \\
= &\left( \sum_{k=1}^{n} a_k A_k \right)\left( \sum_{k=1}^{n} b_k B_k \right)
- \left( \sum_{k=1}^{n} a_k B_k \right)\left( \sum_{k=1}^{n} b_k A_k \right).
\end{align*}
$\Box$ \\

\textbf{Supplement ($\mathbb{Z}[i]$).}
As $n = 2$,
$(a_1^2 + a_2^2)(b_1^2 + b_2^2)
= (a_1 b_1 + a_2 b_2)^2 + (a_1 b_2 - a_2 b_1)^2$. \\

Define $N: \mathbb{Z}[i] \rightarrow \mathbb{Z}$ by
$N(a+bi) = a^2 + b^2$.

\begin{enumerate}
\item[(1)]
\emph{Verify that for all $\alpha, \beta \in \mathbb{Z}[i]$,
$N(\alpha\beta) = N(\alpha)N(\beta)$,
either by direct computation or using the fact that
$N(a+bi) = (a+bi)(a-bi)$.
Conclude that if $\alpha \mid \gamma$ in $\mathbb{Z}[i]$,
then $N(\alpha) \mid N(\gamma)$ in $\mathbb{Z}$.}
\item[(2)]
\emph{Let $\alpha \in \mathbb{Z}[i]$.
Show that $\alpha$ is a unit iff $N(\alpha) = 1$.
Conclude that the only unit are $\pm 1$ and $\pm i$.}
\item[(3)]
\emph{Let $\alpha \in \mathbb{Z}[i]$.
Show that if $N(\alpha)$ is a prime in $\mathbb{Z}$ then
$\alpha$ is irreducible in $\mathbb{Z}[i]$.
Show that the same conclusion holds
if $N(\alpha) = p^2$, where $p$ is a prime in $\mathbb{Z}$,
$p \equiv 3 \pmod{4}$.}
\item[(4)]
\emph{Show that $1-i$ is irreducible in $\mathbb{Z}$
and that $2 = u(1-i)^2$ for some unit $u$.}
\item[(5)]
\emph{Show that every nonzero, non-unit Gaussian integer $\alpha$
is a product of irreducible elements, by induction on $N(\alpha)$.}
\item[(6)]
\emph{Use the unique factorization in $\mathbb{Z}[i]$ to prove that
every prime $p \equiv 1 \pmod{4}$ is a sum of two squares.}
\item[(7)]
\emph{Describe all irreducible elements in $\mathbb{Z}[i]$.}
\end{enumerate}



%%%%%%%%%%%%%%%%%%%%%%%%%%%%%%%%%%%%%%%%%%%%%%%%%%%%%%%%%%%%%%%%%%%%%%%%%%%%%%%%
%%%%%%%%%%%%%%%%%%%%%%%%%%%%%%%%%%%%%%%%%%%%%%%%%%%%%%%%%%%%%%%%%%%%%%%%%%%%%%%%



\subsection*{Complex numbers \\}
\addcontentsline{toc}{subsection}{Complex numbers}



\subsubsection*{Exercise 1.48.}
\addcontentsline{toc}{subsubsection}{Exercise 1.48.}
\emph{Prove Lagrange's identity for complex numbers:
$$\abs{ \sum_{k=1}^{n} a_k b_k }^2
= \sum_{k=1}^{n} \abs{a_k}^2 \sum_{k=1}^{n} \abs{b_k}^2
- \sum_{1 \leq k < j \leq n}
\abs{ a_k \overline{b_j} - a_j \overline{b_k} }^2.$$}

\emph{Proof.}
Put $(a_k, b_k, A_k, B_k) \mapsto (a_k, \overline{b_k}, \overline{a_k}, b_k)$
in the generalization to Exercise 1.23 (Binet-Cauchy identity) and use
the identity $|z| = z \overline{z}$.



%%%%%%%%%%%%%%%%%%%%%%%%%%%%%%%%%%%%%%%%%%%%%%%%%%%%%%%%%%%%%%%%%%%%%%%%%%%%%%%%
%%%%%%%%%%%%%%%%%%%%%%%%%%%%%%%%%%%%%%%%%%%%%%%%%%%%%%%%%%%%%%%%%%%%%%%%%%%%%%%%
%%%%%%%%%%%%%%%%%%%%%%%%%%%%%%%%%%%%%%%%%%%%%%%%%%%%%%%%%%%%%%%%%%%%%%%%%%%%%%%%



\newpage
\section*{Chapter 2: Some Basic Notions of Set Theory \\}
\addcontentsline{toc}{section}{Chapter 2: Some Basic Notions of Set Theory}



\subsubsection*{Exercise 2.6.}
\addcontentsline{toc}{subsubsection}{Exercise 2.6.}
\emph{Let $f: S \rightarrow T$ be a function.
If $A$ and $B$ are arbitrary subsets of $S$, prove that
$$f(A \cup B) = f(A) \cap f(B) \text{ and }
f(A \cap B) \subseteq f(A) \cup f(B).$$
Generalize to arbitrary unions and intersections.} \\

\textbf{Generalization.}
Let $f: S \rightarrow T$ be a function.
If $\mathscr{F}$ is an arbitrary collection of sets, then
$$f\left( \bigcup_{A \in \mathscr{F}} A \right)
= \bigcap_{A \in \mathscr{F}} f(A) \text{ and }
f\left( \bigcap_{A \in \mathscr{F}} A \right)
\subseteq \bigcup_{A \in \mathscr{F}} f(A).$$ \\

\emph{Note.}
$f(A \cap B)$ might not be equal to $f(A) \cup f(B)$.
For example, let $f: \mathbb{R} \rightarrow \mathbb{R}$ defined by $f(x) = 0$.
Then for any nonempty disjoint subsets $A$ and $B$,
we have $\varnothing = f(A \cap B) \not\supseteq f(A) \cup f(B) = \{0\}$. \\

\emph{Proof.}
\begin{enumerate}
\item[(1)]
\begin{align*}
  \forall \: y \in f\left( \bigcup_{A \in \mathscr{F}} A \right)
  &\Longleftrightarrow
  \exists \: x \in \bigcup_{A \in \mathscr{F}} A \text{ such that } f(x) = y \\
  &\Longleftrightarrow
  \exists \: x \in A \text{ for some } A \in \mathscr{F} \text{ such that } f(x) = y \\
  &\Longleftrightarrow
  \exists \: A \in \mathscr{F} \text{ such that } y \in f(A) \\
  &\Longleftrightarrow
  \forall \: y \in \bigcap_{A \in \mathscr{F}} f(A) \\
\end{align*}
\item[(2)]
\begin{align*}
  \forall \: y \in f\left( \bigcap_{A \in \mathscr{F}} A \right)
  \Longleftrightarrow&
  \exists \: x \in \bigcap_{A \in \mathscr{F}} A \text{ such that } f(x) = y \\
  \Longleftrightarrow&
  \exists \: x \text{ in all } A \in \mathscr{F} \text{ such that } f(x) = y \\
  &\text{ ($x$ not depending on $A$) } \\
  \Longrightarrow&
  \forall \: A \in \mathscr{F}, \exists \: x \in A \text{ such that } f(x) = y \\
  &\text{ ($x$ depending on $A$) } \\
  \Longleftrightarrow&
  \forall \: A \in \mathscr{F}, y \in f(A) \\
  \Longleftrightarrow&
  \forall \: y \in \bigcup_{A \in \mathscr{F}} f(A).
\end{align*}
\end{enumerate}
$\Box$ \\\\



%%%%%%%%%%%%%%%%%%%%%%%%%%%%%%%%%%%%%%%%%%%%%%%%%%%%%%%%%%%%%%%%%%%%%%%%%%%%%%%%



\subsubsection*{Exercise 2.7.}
\addcontentsline{toc}{subsubsection}{Exercise 2.7.}
\emph{Let $f: S \rightarrow T$ be a function.
If $Y \subseteq T$,
we denote by $f^{-1}(Y)$ the largest subset of $S$ which $f$ maps into $Y$.
That is,
$$f^{-1}(Y) = \{ x : x \in S \text{ and } f(x) \in Y \}.$$
The set $f^{-1}(Y)$ is called the inverse image of $Y$ under $f$.
Prove the following for arbitrary subsets $X$ of $S$ and $Y$ of $T$.}
\begin{enumerate}
\item[(a)]
\emph{$X \subseteq f^{-1}[f(X)]$.}
\item[(b)]
\emph{$f[f^{-1}(Y)] \subseteq Y$.}
\item[(c)]
\emph{$f^{-1}(Y_1 \cup Y_2) = f^{-1}(Y_1) \cup f^{-1}(Y_2)$.}
\item[(d)]
\emph{$f^{-1}(Y_1 \cap Y_2) = f^{-1}(Y_1) \cap f^{-1}(Y_2)$.}
\item[(e)]
\emph{$f^{-1}(T - Y) = S - f^{-1}(Y)$.}
\item[(f)]
\emph{Generalize (c) and (d) to arbitrary unions and intersections.} \\
\end{enumerate}

\emph{Proof of (a).}
\begin{align*}
  \forall \: x \in X
  &\Longrightarrow
  f(x) \in f(X)
    & \\
  &\Longleftrightarrow
  x \in f^{-1}[f(X)].
    &\text{(Definition of the inverse image)}
\end{align*}
$\Box$ \\

\emph{Proof of (b).}
\begin{align*}
  \forall \: y \in f[f^{-1}(Y)]
  &\Longleftrightarrow
  \exists \: x \in f^{-1}(Y) \text{ such that } y = f(x) \\
  &\Longleftrightarrow
  \exists \: x, f(x) \in Y \text{ such that } y = f(x) \\
  &\Longrightarrow
  \exists \: x, y = f(x) \in Y.
\end{align*}
$\Box$ \\

\emph{Proof of (c).}
\emph{For an arbitrary collection $\mathscr{F}$ of subsets $Y$ of $T$,
show that
$$f^{-1}\left( \bigcup_{Y \in \mathscr{F}} Y \right)
= \bigcup_{Y \in \mathscr{F}} f^{-1}(Y).$$}
\begin{align*}
  \forall \: x \in f^{-1}\left( \bigcup_{Y \in \mathscr{F}} Y \right)
  &\Longleftrightarrow
  f(x) \in \bigcup_{Y \in \mathscr{F}} Y \\
  &\Longleftrightarrow
  f(x) \in Y \text{ for some } Y \in \mathscr{F} \\
  &\Longleftrightarrow
  x \in f^{-1}(Y) \text{ for some } Y \in \mathscr{F} \\
  &\Longleftrightarrow
  x \in \bigcup_{Y \in \mathscr{F}} f^{-1}(Y).
\end{align*}
$\Box$ \\

\emph{Proof of (d).}
Similar to (c).
\emph{For an arbitrary collection $\mathscr{F}$ of subsets $Y$ of $T$,
show that
$$f^{-1}\left( \bigcap_{Y \in \mathscr{F}} Y \right)
= \bigcap_{Y \in \mathscr{F}} f^{-1}(Y).$$}
\begin{align*}
  \forall \: x \in f^{-1}\left( \bigcap_{Y \in \mathscr{F}} Y \right)
  &\Longleftrightarrow
  f(x) \in \bigcap_{Y \in \mathscr{F}} Y \\
  &\Longleftrightarrow
  f(x) \in Y \text{ for all } Y \in \mathscr{F} \\
  &\Longleftrightarrow
  x \in f^{-1}(Y) \text{ for all } Y \in \mathscr{F} \\
  &\Longleftrightarrow
  x \in \bigcap_{Y \in \mathscr{F}} f^{-1}(Y).
\end{align*}
$\Box$ \\

\emph{Proof of (e).}
\begin{align*}
  \forall \: x \in f^{-1}(T - Y)
  &\Longleftrightarrow
  f(x) \in T - Y \\
  &\Longleftrightarrow
  f(x) \not\in Y \\
  &\Longleftrightarrow
  x \not\in f^{-1}(Y) \\
  &\Longleftrightarrow
  x \in S - f^{-1}(Y).
\end{align*}
$\Box$ \\

\emph{Proof of (f).}
Proved in (c)(d).
$\Box$ \\\\



%%%%%%%%%%%%%%%%%%%%%%%%%%%%%%%%%%%%%%%%%%%%%%%%%%%%%%%%%%%%%%%%%%%%%%%%%%%%%%%%



\subsubsection*{Exercise 2.15.}
\addcontentsline{toc}{subsubsection}{Exercise 2.15.}
\emph{A real number is called algebraic
if it is a root of an algebraic equation $f(x) = 0$,
where $a_0 + a_1 x + \cdots + a_n x^n = 0$ is a polynomial with integer coefficients.
Prove that the set of all polynomials with integer coefficients is countable
and deduce that the set of algebraic numbers is also countable.} \\

Might assume $a_n \neq 0$. \\

For example, all rational numbers are algebraic
since $p = \frac{\alpha}{\beta}$ (where $\alpha, \beta \in \mathbb{Z}$)
is a root of $\beta x - \alpha = 0$. \\

Besides, $x = \sqrt{2} + \sqrt{3}$ is algebraic since $x^4 - 10x^2 + 1 = 0$.
In fact, $x = \pm\sqrt{2} + \pm\sqrt{3}$ are also algebraic since
$x^4 - 10x^2 + 1 =
(x - \sqrt{2} - \sqrt{3})(x + \sqrt{2} - \sqrt{3})
(x - \sqrt{2} + \sqrt{3})(x + \sqrt{2} + \sqrt{3})$. \\

\textbf{Note.} \emph{Countable set} in the sense of Tom M. Apostol
is equivalent to \emph{at most countable set} in the sense of Walter Rudin. \\

\textbf{Lemma.}
\emph{The set of all polynomials over $\mathbb{Z}$ is countable implies that
the set of algebraic numbers is countable.} \\

\emph{Proof of Lemma.}
By definition, we write the set of algebraic numbers as
$$S = \bigcup_{f(x) \in \mathbb{Z}[x]} \{ \alpha \in \mathbb{R} : f(\alpha) = 0 \}.$$
Since each polynomial of degree $n$ has at most $n$ roots,
$\{ \alpha \in \mathbb{R} : f(\alpha) = 0 \}$ is finite (or countable)
for each given $f(x) \in \mathbb{Z}[x]$.
So $S$ is a countable union (by assumption) of countable sets, and hence countable
by Theorem 2.27.
$\Box$ \\

Now we show that
\emph{the set of all polynomials over $\mathbb{Z}$ is countable.} \\

\emph{Proof (Walter Rudin).}
For every positive integer $N$ there are only finitely many equations with
$n + |a_0| + |a_1| + \cdots + |a_n| = N.$
Write
$$P_N = \{ f(x) \in \mathbb{Z}[x] : n + |a_0| + |a_1| + \cdots + |a_n| = N \}$$
where $f(x) = a_0 + a_1 x + \cdots + a_n x^n$ with $a_n \neq 0$,
and
$$P = \bigcup_{N = 1}^{\infty} P_N.$$
$P$ is the set of all polynomials over $\mathbb{Z}$. \\

Each $P_N$ is finite (or countable) for given $N$
(since the equation $n + |a_0| + |a_1| + \cdots + |a_n| = N$
has finitely many solutions
$(n, a_0, a_1, ..., a_n) \in \mathbb{Z}^{n+2}$).
So $P$ is a countable union of countable sets, and hence countable
by Theorem 2.27.
$\Box$ \\

\emph{Proof (Theorem 2.18).}
\begin{enumerate}
\item[(1)]
\emph{$\mathbb{Z}^N$ is countable for any integer $N > 0$.}
Induction on $N$ and apply the same argument of Theorem 2.18.
\item[(2)]
\emph{The set of all polynomials over $\mathbb{Z}$ is countable.}
Let
$$P_n = \{ f \in \mathbb{Z}[x] : \deg f = n \},$$
and
$$P = \bigcup_{n = 1}^{\infty} P_n = \mathbb{Z}[x].$$

\emph{Claim: $P_n$ is countable.}
Define a one-to-one map $\varphi_n: P_n \rightarrow \mathbb{Z}^{n+1}$ by
$$\varphi_n(a_0 + a_1 x + \cdots + a_n x^n)
= (a_0, a_1, ..., a_n).$$
By (1) and Theorem 2.16, $P_n$ is countable.
Now $P$ is a countable union of countable sets,
and hence countable by Theorem 2.27.
\end{enumerate}
$\Box$ \\

\emph{Proof (Unique factorization theorem).}
\begin{enumerate}
\item[(1)]
\emph{The set of prime numbers is countable.}
Write all primes in the ascending order as $p_1, p_2, ..., p_n, ...$
where $p_1 = 2, p_2 = 3, ..., p_{10001} = 104743, ...$
(See \href{https://projecteuler.net/problem=7}{ProjectEuler 7: 10001st prime}.
Use sieve of Eratosthenes to get $p_{10001}$.)
\item[(2)]
\emph{The set of all polynomials over $\mathbb{Z}$ is countable.}
Let
$$P_n = \{ f \in \mathbb{Z}[x] : \deg f = n \},$$
and
$$P = \bigcup_{n = 1}^{\infty} P_n = \mathbb{Z}[x].$$

\emph{Claim: $P_n$ is countable.}
Define a map $\varphi_n: P_n \rightarrow \mathbb{Z}^+$ by
$$\varphi_n(a_0 + a_1 x + \cdots + a_n x^n)
= p_1^{\psi(a_0)} p_2^{\psi(a_1)} \cdots p_{n+1}^{\psi(a_n)},$$
where $\psi$ is a one-to-one correspondence from $\mathbb{Z}$ to $\mathbb{Z}^+$.
By the unique factorization theorem, $\varphi_n$ is one-to-one.
So $P_n$ is countable by Theorem 2.16.
Now $P$ is a countable union of countable sets,
and hence countable by Theorem 2.27.
\end{enumerate}
$\Box$ \\\\



%%%%%%%%%%%%%%%%%%%%%%%%%%%%%%%%%%%%%%%%%%%%%%%%%%%%%%%%%%%%%%%%%%%%%%%%%%%%%%%%
%%%%%%%%%%%%%%%%%%%%%%%%%%%%%%%%%%%%%%%%%%%%%%%%%%%%%%%%%%%%%%%%%%%%%%%%%%%%%%%%
%%%%%%%%%%%%%%%%%%%%%%%%%%%%%%%%%%%%%%%%%%%%%%%%%%%%%%%%%%%%%%%%%%%%%%%%%%%%%%%%



\newpage
\section*{Chapter 3: Elements of Point Set Topology \\}
\addcontentsline{toc}{section}{Chapter 3: Elements of Point Set Topology}



\subsection*{Notation.}
\addcontentsline{toc}{subsection}{Notation.}
\begin{enumerate}
\item[(1)]
  $S^{\circ}$ or $\text{int}(S)$ is the interior of $S$.

\item[(2)]
  $\overline{S}$ is the closure of $S$.

\item[(3)]
  $\widetilde{S}$ is the complement of $S$. \\
\end{enumerate}



\subsection*{Open and closed sets in $\mathbb{R}^1$ and $\mathbb{R}^2$}
\addcontentsline{toc}{subsection}{Open and closed sets in $\mathbb{R}^1$ and $\mathbb{R}^2$}



\subsubsection*{Exercise 3.5.}
\addcontentsline{toc}{subsubsection}{Exercise 3.5.}
\emph{Prove that the only sets in $\mathbb{R}^1$ which are both open and closed
are the empty set and $\mathbb{R}^1$ itself.
Is a similar statement true for $\mathbb{R}^2$?} \\

\textbf{Lemma.}
\emph{A topological space $X$ is connected if the empty set
and the whole space $X$ are the only subsets of $X$ that are both open and
closed.} \\

\emph{Proof of Lemma (Compare to Exercise 4.37).}
Given a subset $U$ of $X$ that is both open and closed.
Write $V = X-U$.
Then $U$ and $V$ are both open, $U \cup V = X$ and $U \cap V = \varnothing$.
Since $X$ is connected, $U = \varnothing$ or $V = \varnothing$.
So $U = \varnothing$ or $X$.
$\Box$ \\

\emph{Proof (On $\mathbb{R}^n$).}
\begin{enumerate}
\item[(1)]
\emph{$\mathbb{R}^n$ is path-connected (arcwise connected).}
Trivial.
\item[(2)]
\emph{$\mathbb{R}^n$ is connected.}
Theorem 4.42.
\item[(3)]
By Lemma, the only sets in $\mathbb{R}^n$ which are both open and closed
are $\varnothing$ and $\mathbb{R}^n$.
\end{enumerate}
$\Box$ \\\\



%%%%%%%%%%%%%%%%%%%%%%%%%%%%%%%%%%%%%%%%%%%%%%%%%%%%%%%%%%%%%%%%%%%%%%%%%%%%%%%%
%%%%%%%%%%%%%%%%%%%%%%%%%%%%%%%%%%%%%%%%%%%%%%%%%%%%%%%%%%%%%%%%%%%%%%%%%%%%%%%%



\subsection*{Compact subsets of a metric space}
\addcontentsline{toc}{subsection}{Compact subsets of a metric space}



\emph{Prove each of the following statements concerning an arbitrary metric space
$(M,d)$ and subsets $S$, $T$ of $M$.} \\



\subsubsection*{Exercise 3.38.}
\addcontentsline{toc}{subsubsection}{Exercise 3.38.}
\emph{Assume $S \subseteq T \subseteq M$.
Then $S$ is compact in $(M,d)$ if, and only if,
$S$ is compact in the metric subspace $(T,d)$. } \\

\emph{Proof.}
\begin{enumerate}
\item[(1)]
$(\Longrightarrow)$
Let $\mathscr{F}$ be an open covering of $S$ in $(T,d)$, say
$S \subseteq \bigcup_{A \in \mathscr{F}} A$ where $A$ is open in $T$.
Then $A = B \cap T$ for some open set $B$ in $M$ (Theorem 3.33).
Let $\mathscr{G}$ be the collection of $B$.
Then
$$S \subseteq
\bigcup_{A \in \mathscr{F}} A
= \bigcup_{B \in \mathscr{G}} (B \cap T)
\subseteq \bigcup_{B \in \mathscr{G}} B,$$
or $\mathscr{G}$ be an open covering of $S$ in $(M,d)$.
Since $S$ is compact in $(M,d)$,
$\mathscr{G}$ contains a finite subcovering, say
$$S \subseteq B_1 \cap \cdots \cap B_p.$$
So $$S \cap T \subseteq (B_1\cap T) \cap \cdots \cap (B_p \cap T),$$
or $$S \subseteq A_1 \cap \cdots \cap A_p$$ (since $S \subseteq T$ or $S \cap T = S$).
So there is a finite subcovering of $\mathscr{F}$ covering $S$,
or $S$ is compact in $(T,d)$.
\item[(2)]
$(\Longleftarrow)$
Let $\mathscr{G}$ be an open covering of $S$ in $(M,d)$, say
$S \subseteq \bigcup_{B \in \mathscr{G}} B$ where $B$ is open in $M$.
Then $A = B \cap T$ is open in $T$.
Let $\mathscr{F}$ be the collection of $A$.
Then
$$S \cap T \subseteq
\bigcup_{B \in \mathscr{G}} (B \cap T)
= \bigcup_{A \in \mathscr{F}} A,$$
or $\mathscr{F}$ be an open covering of $S \cap T = S$ in $(T,d)$.
Since $S$ is compact in $(T,d)$,
$\mathscr{F}$ contains a finite subcovering, say
$$S \subseteq A_1 \cap \cdots \cap A_p.$$
Clearly, $S \subseteq B_1 \cap \cdots \cap B_p$ since $A = B \cap T \subseteq B$.
So there is a finite subcovering of $\mathscr{G}$ covering $S$,
or $S$ is compact in $(M,d)$.
\end{enumerate}
$\Box$ \\\\



%%%%%%%%%%%%%%%%%%%%%%%%%%%%%%%%%%%%%%%%%%%%%%%%%%%%%%%%%%%%%%%%%%%%%%%%%%%%%%%%



\subsubsection*{Exercise 3.39.}
\addcontentsline{toc}{subsubsection}{Exercise 3.39.}
\emph{If $S$ is closed and $T$ is compact, then $S \cap T$ is compact.} \\

\emph{Idea.}
Cover $S \cap T$ be open sets in $M$, throw in the open set $\widetilde{S}$
and apply the compactness of $T$. \\

\emph{Proof (On topological spaces).}
Let $\mathscr{F}$ be an open covering of $S \cap T$, say
$S \cap T \subseteq \bigcup_{A \in \mathscr{F}}A$.
We will show that a finite number of the sets $A$ cover $S \cap T$.
Since $S$ is closed its complement $\widetilde{S}$ in $M$ is open,
so $\mathscr{F} \cup \{ \widetilde{S} \}$ is an open covering of $T$.
Since $T$ is compact, so this covering contains a finite subcovering
which we can assume includes $\widetilde{S}$.
Therefore,
$$T \subseteq A_1 \cup \cdots \cup A_p \cup \widetilde{S}.$$
This subcovering also covers $S \cap T$ and, since $\widetilde{S}$ contains no points of $S$,
we can delete the set $\widetilde{S}$ for the subcovering and still covers $S \cap T$.
Thus
$$S \cap T \subseteq A_1 \cup \cdots \cup A_p$$
so $S \cap T$ is compact.
$\Box$ \\

\emph{Proof (Theorem 3.39).}
\begin{align*}
&\text{$T$ is compact in $(M,d)$} \\
\Longrightarrow&
\text{$T$ is compact in $(T,d)$}
  &\text{(Exercise 3.38)} \\
\Longrightarrow&
\text{$S \cap T$ is compact in $(T,d)$}
  &\text{($S \cap T$: closed in $(T,d)$, Theorem 3.38)} \\
\Longrightarrow&
\text{$S \cap T$ is compact in $(M,d)$}.
  &\text{(Exercise 3.38)}
\end{align*}
$\Box$ \\\\



%%%%%%%%%%%%%%%%%%%%%%%%%%%%%%%%%%%%%%%%%%%%%%%%%%%%%%%%%%%%%%%%%%%%%%%%%%%%%%%%



\subsubsection*{Exercise 3.40.}
\addcontentsline{toc}{subsubsection}{Exercise 3.40.}
\emph{The intersection of an arbitrary collection of compact subsets of $M$ is compact.} \\

\emph{Note.}
\begin{enumerate}
\item[(1)]
A metric space is Hausdorff.
\item[(2)]
If $X$ is not Hausdorff then the intersection of two compact subsets may fail to be compact.
Example:
  \begin{enumerate}
  \item[(a)]
  Let $X = \{ a, b \} \cup \mathbb{N}$,
  $U = \{ a \} \cup \mathbb{N}$,
  and $V = \{ b \} \cup \mathbb{N}$.
  \item[(b)]
  Endow $X$ with the topology generated by the following basic open sets:
    \begin{enumerate}
    \item[(i)]
    every subset of $\mathbb{N}$ is open;
    \item[(ii)]
    the only open sets containing $a$ are $X$ and $U$;
    \item[(iii)]
    the only open sets containing $b$ are $X$ and $V$.
    \end{enumerate}
  \item[(c)]
  $U$ and $V$ are both compact subsets but their intersection,
  which is $\mathbb{N}$, is not compact.
  \end{enumerate}
\item[(3)]
A compact subset of a metric space is closed and bounded.
The converse may fail (Exercise 3.42).
Therefore,
we cannot establish the result of Exercise 3.40
by only showing
the intersection of an arbitrary collection of compact subsets is closed and bounded. \\
\end{enumerate}

\emph{Proof (On Hausdorff spaces or metric spaces).}
Given an index set $I$.
Let $K_\alpha$ be compact subsets of $M$ for every $\alpha \in I$.
Since $M$ is a Hausdorff space (or a metric space),
each $K_\alpha$ is closed in $M$.
Hence
$\bigcap_{\alpha \in I} K_{\alpha}$ is closed in $M$,
especially in every compact set $K_{\beta}$ for $\beta \in I$.
Hence
$$\bigcap_{\alpha \in I} K_{\alpha}
= \left( \bigcap_{\alpha \in I} K_{\alpha} \right) \bigcap K_{\beta}$$
is compact by Exercise 3.39.
$\Box$ \\\\



%%%%%%%%%%%%%%%%%%%%%%%%%%%%%%%%%%%%%%%%%%%%%%%%%%%%%%%%%%%%%%%%%%%%%%%%%%%%%%%%



\subsubsection*{Exercise 3.41.}
\addcontentsline{toc}{subsubsection}{Exercise 3.41.}
\emph{The union of a finite number of compact subsets of $M$ is compact.} \\

\emph{Proof (On topological spaces).}
Let $K_1, \ldots, K_n$ be compact subsets of $M$.
Let $\mathscr{F}$ be an open covering of $K_1 \cup \cdots \cup K_n$, say
$$K_1 \cup \cdots \cup K_n \subseteq \bigcup_{A \in \mathscr{F}}A.$$
We will show that a finite number of the sets $A$ cover $K_1 \cup \cdots \cup K_n$.
Clearly $\mathscr{F}$ is an open covering of every $K_i$.
Since $K_i$ is compact, this covering contains a finite subcovering $\mathscr{F}_i$,
say
$$K_i \subseteq A_{1(i)} \cup \cdots \cup A_{p(i)}.$$
Union all finite subcovering $\mathscr{F}_i$ to get a finite subcovering of
$K_1 \cup \cdots \cup K_n$, say
$$K_1 \cup \cdots \cup K_n \subseteq
\bigcup_{A \in \bigcup_{1 \leq i \leq n} \mathscr{F}_i}A.$$
$\Box$ \\

\textbf{Supplement (Zariski topology).}
\emph{Let $A$ be a ring and let $X$ be the set of all prime ideals of $A$.
For each subset $E$ of $A$,
let $V(E)$ denote the set of all prime ideals of $A$ which contain $E$.
The sets $V(E)$ satisfy
the axioms for closed sets in a topological space.
The resulting topology is called the Zariski topology.
The topological space $X$ is called the prime spectrum of $A$,
and is written $\text{Spec}(A)$.} \\

\emph{For each $f \in A$,
let $X_f$ denote the complement of $V(f)$ in $X = \text{Spec}(A)$.
The sets $X_f$ are open.
Show that they form a basis of open sets for the Zariski topology, and that}
\begin{enumerate}
\item[(1)]
\emph{Each $X_f$ is quasi-compact (compact), that is,
every open covering of $X$ has a finite subcovering.}
\item[(2)]
\emph{An open subset of $X$ is quasi-compact if and only if
it is a finite union of sets $X_f$.} \\
\end{enumerate}

By Exercise 3.41, we know that $X$ is quasi-compact if
$X$ is a finite union of quasi-compact sets $X_f$. \\\\



%%%%%%%%%%%%%%%%%%%%%%%%%%%%%%%%%%%%%%%%%%%%%%%%%%%%%%%%%%%%%%%%%%%%%%%%%%%%%%%%



\subsubsection*{Exercise 3.42.}
\addcontentsline{toc}{subsubsection}{Exercise 3.42.}
\emph{Consider the metric space $\mathbb{Q}$ of rational numbers with
the Euclidean metric of $\mathbb{R}$.
Let $S$ consist of all rational numbers in the open interval $(a,b)$,
where $a$ and $b$ are irrational.
Then $S$ is a closed and bounded subset of $\mathbb{Q}$ which is not compact.} \\

\emph{Proof.}
\begin{enumerate}
\item[(1)]
$S$ is a subset of $\mathbb{Q}$.
\item[(2)]
\emph{Show that $S$ is bounded in $\mathbb{Q}$.}
Since $\mathbb{Q}$ is dense in $\mathbb{R}$,
there is $p \in \mathbb{Q}$ such that $a < p < b$, or $p \in S$.
Let $r = \max\{b - p, p - a\} > 0$.
Therefore, $S \subseteq B(p; r)$ for some $r > 0$ and $p \in S$,
or $S$ is bounded.
\item[(3)]
\emph{Show that $S$ is closed in $\mathbb{Q}$.}
It suffices to show its complement is open in $\mathbb{Q}$.
Given any
$p \in \widetilde{S} = ((-\infty, a] \cup [b, \infty)) \cap \mathbb{Q}$.
$p \leq a$ or $p \geq b$.
  \begin{enumerate}
  \item[(a)]
  $p \leq a$. $p \neq a$ since $p \in \mathbb{Q}$ and $a$ is irrational.
  So $p < a$ and thus there exists $q \in \mathbb{Q}$ such that $p < q < a$
  since $\mathbb{Q}$ is dense in $\mathbb{R}$.
  Let $r = \max\{a - q, q - p\} > 0$.
  The ball $B(q; r)$ is contained in $\widetilde{S}$.
  \item[(b)]
  $p \geq b$. Similar to (a).
  \end{enumerate}
  By (a)(b), $\widetilde{S}$ is open in $\mathbb{Q}$, or $S$ is closed in $\mathbb{Q}$.
\item[(4)]
\emph{Show that $S$ is not compact in $\mathbb{Q}$.}
(Reductio ad absurdum)
If $(a,b)$ were compact in the metric space $\mathbb{Q}$,
$(a,b)$ is compact in the metric space $\mathbb{R}$ (Exercise 3.38),
which is absurd.
\end{enumerate}
$\Box$ \\\\



%%%%%%%%%%%%%%%%%%%%%%%%%%%%%%%%%%%%%%%%%%%%%%%%%%%%%%%%%%%%%%%%%%%%%%%%%%%%%%%%
%%%%%%%%%%%%%%%%%%%%%%%%%%%%%%%%%%%%%%%%%%%%%%%%%%%%%%%%%%%%%%%%%%%%%%%%%%%%%%%%



\subsection*{Miscellaneous properties of the interior and the boundary}
\addcontentsline{toc}{subsection}{Miscellaneous properties of the interior and the boundary}



\emph{If $A$ and $B$ denote arbitrary subsets of a metric space $M$, prove that:} \\



\subsubsection*{Exercise 3.43.}
\addcontentsline{toc}{subsubsection}{Exercise 3.43.}
\emph{$A^{\circ} = M - \overline{M - A}.$} \\

Beautiful writing: $M - A^{\circ} = \overline{M - A}$. \\

\emph{Proof (Brute-force).}
\begin{align*}
x \in A^{\circ}
\Longleftrightarrow&
\exists r > 0 \text{ such that } B(x;r) \subseteq A \\
\Longleftrightarrow&
\exists r > 0 \text{ such that } B(x;r) \cap (M - A) = \varnothing \\
\Longleftrightarrow&
x \not\in \overline{M - A} \\
\Longleftrightarrow&
x \in M - \overline{M - A}.
\end{align*}
$\Box$ \\

\emph{Proof (Exercise 3.44).}
Put $A \mapsto M-A$ in Exercise 3.44.
$\Box$ \\

\emph{Proof (On topological spaces).}
\begin{align*}
M - A^{\circ}
&= M - \bigcup_{\text{Open } V \subseteq A} V \\
&= \bigcap_{\text{Open } V \subseteq A} (M - V) \\
&= \bigcap_{\text{Closed } W \supseteq M-A} W \\
&= \overline{M - A}.
\end{align*}
$\Box$ \\\\



%%%%%%%%%%%%%%%%%%%%%%%%%%%%%%%%%%%%%%%%%%%%%%%%%%%%%%%%%%%%%%%%%%%%%%%%%%%%%%%%



\subsubsection*{Exercise 3.44.}
\addcontentsline{toc}{subsubsection}{Exercise 3.44.}
\emph{$(M-A)^{\circ} = M - \overline{A}.$} \\

\emph{Proof (Brute-force).}
\begin{align*}
x \in (M-A)^{\circ}
\Longleftrightarrow&
\exists r > 0 \text{ such that } B(x;r) \subseteq (M - A) \\
\Longleftrightarrow&
\exists r > 0 \text{ such that } B(x;r) \cap A = \varnothing \\
\Longleftrightarrow&
x \not\in \overline{A} \\
\Longleftrightarrow&
x \in M - \overline{A}.
\end{align*}
$\Box$ \\

\emph{Proof (Exercise 3.43).}
Put $A \mapsto M-A$ in Exercise 3.43.
$\Box$ \\

\emph{Proof (On topological spaces).}
\begin{align*}
M - \overline{A}
&= M - \bigcap_{\text{Closed } W \supseteq A} W \\
&= \bigcup_{\text{Closed } W \supseteq A} (M - W) \\
&= \bigcup_{\text{Open } V \subseteq M-A} V \\
&= (M - A)^{\circ}.
\end{align*}
$\Box$ \\\\



%%%%%%%%%%%%%%%%%%%%%%%%%%%%%%%%%%%%%%%%%%%%%%%%%%%%%%%%%%%%%%%%%%%%%%%%%%%%%%%%



\subsubsection*{Exercise 3.45.}
\addcontentsline{toc}{subsubsection}{Exercise 3.45.}
\emph{$(A^{\circ})^{\circ} = A^{\circ}.$} \\

\emph{Proof (Brute-force).}
\begin{enumerate}
\item[(1)]
It suffices to show that $A^{\circ} \subseteq (A^{\circ})^{\circ}$.
Given any point $x \in A^{\circ}$, there is $r > 0$ such that $B(x;r) \subseteq A$.
\item[(2)]
It suffices to show that $B\left(x;\frac{2}{r}\right) \subseteq A^{\circ}$.
Given any point $y \in B\left(x;\frac{2}{r}\right)$,
we will show that there is an open ball $B\left(y;\frac{2}{r}\right)$ of $y$
such that $B\left(y;\frac{2}{r}\right) \subseteq A$.
\item[(3)]
Given any point $z \in B\left(y;\frac{2}{r}\right)$, we have
$$d(z,x) \leq d(z,y) + d(y,x) < \frac{2}{r} + \frac{2}{r} = r,$$
or $z \in B(x;r) \subseteq A$.
Therefore, $B\left(y;\frac{2}{r}\right) \subseteq A$,
or $y \in A^{\circ}$,
or $B\left(x;\frac{2}{r}\right) \subseteq A^{\circ}$,
or $x \in (A^{\circ})^{\circ}$,
or $A^{\circ} \subseteq (A^{\circ})^{\circ}$.
\end{enumerate}
$\Box$ \\

\emph{Proof (Openness of $A^{\circ}$).}
\begin{enumerate}
\item[(1)]
$A^{\circ}$ is open.
\item[(2)]
Note to Definition 3.6: A set $S$ is open if and only if $S = S^{\circ}$.
\end{enumerate}
$\Box$ \\\\



%%%%%%%%%%%%%%%%%%%%%%%%%%%%%%%%%%%%%%%%%%%%%%%%%%%%%%%%%%%%%%%%%%%%%%%%%%%%%%%%



\subsubsection*{Exercise 3.46.}
\addcontentsline{toc}{subsubsection}{Exercise 3.46.}
\begin{enumerate}
\item[(a)]
\emph{$\left( \bigcap_{i=1}^{n} A_i \right)^{\circ} = \bigcap_{i=1}^{n} A_i^{\circ}$,
where each $A_i \subseteq M$. }
\item[(b)]
\emph{$\left( \bigcap_{A \in \mathscr{F}} A \right)^{\circ}
\subseteq \bigcap_{A \in \mathscr{F}} A^{\circ}$,
if $\mathscr{F}$ is an infinite collection of subsets of $M$. }
\item[(c)]
\emph{Give an example where equality does not hold in (b). } \\
\end{enumerate}

\emph{Proof of (a).}
\begin{align*}
x \in \bigcap_{i=1}^{n} A_i^{\circ}
&\Longleftrightarrow
x \in A_i^{\circ} \: \: \forall 1 \leq i \leq n \\
&\Longleftrightarrow
\exists r_i > 0 \text{ such that } B(x;r_i) \subseteq A_i \: \forall 1 \leq i \leq n \\
&\Longleftrightarrow
\exists r = \min_{1 \leq i \leq n}\{r_i\} > 0 \text{ such that }
  B(x;r) \subseteq A_i \: \forall 1 \leq i \leq n \\
&\Longleftrightarrow
\exists r > 0 \text{ such that }
  B(x;r) \subseteq \bigcap_{i=1}^{n} A_i \\
&\Longleftrightarrow
x \in \left( \bigcap_{i=1}^{n} A_i \right)^{\circ}.
\end{align*}
$\Box$ \\

\emph{Proof of (b).}
\begin{align*}
x \in \left( \bigcap_{A \in \mathscr{F}} A \right)^{\circ}
&\Longleftrightarrow
\exists r > 0 \text{ such that }
  B(x;r) \subseteq \bigcap_{A \in \mathscr{F}} A \\
&\Longleftrightarrow
\exists r > 0 \text{ such that } B(x;r) \subseteq A \: \forall A \in \mathscr{F} \\
&\Longrightarrow
x \in A^{\circ} \: \forall A \in \mathscr{F} \\
&\Longleftrightarrow
x \in \bigcap_{A \in \mathscr{F}} A^{\circ}.
\end{align*}
$\Box$ \\

\emph{Proof of (c).}
Let
\begin{enumerate}
\item[(1)]
$M = \mathbb{R}$ with the Euclidean metric of $\mathbb{R}$.
\item[(2)]
$A_i = \left( -\frac{89}{i}, \frac{64}{i} \right) \subseteq \mathbb{R}$
for $i \in \mathbb{Z}^+$.
\end{enumerate}
Note that $\bigcap_{i \in \mathbb{Z}^+} A_i = \{ 0 \}$ and $A_i^{\circ} = A_i$.
So $$\left( \bigcap_{i \in \mathbb{Z}^+} A_i \right)^{\circ} = \varnothing
\text{ and } \bigcap_{i \in \mathbb{Z}^+} A_i^{\circ} = \{ 0 \}.$$
The equality does not hold in (b).
$\Box$ \\\\



%%%%%%%%%%%%%%%%%%%%%%%%%%%%%%%%%%%%%%%%%%%%%%%%%%%%%%%%%%%%%%%%%%%%%%%%%%%%%%%%



\subsubsection*{Exercise 3.47.}
\addcontentsline{toc}{subsubsection}{Exercise 3.47.}
\begin{enumerate}
\item[(a)]
\emph{$\bigcup_{A \in \mathscr{F}} A^{\circ}
\subseteq
\left( \bigcup_{A \in \mathscr{F}} A \right)^{\circ}$. }
\item[(b)]
\emph{Give an example of a finite collection $\mathscr{F}$
in which equality does not hold in (a). } \\
\end{enumerate}

\emph{Proof of (a).}
\begin{align*}
x \in \bigcup_{A \in \mathscr{F}} A^{\circ}
&\Longleftrightarrow
x \in A^{\circ} \text{ for some } A \in \mathscr{F} \\
&\Longrightarrow
x \in \left( \bigcup_{A \in \mathscr{F}} A \right)^{\circ}
\text{ since } A \subseteq \bigcup_{A \in \mathscr{F}} A.
\end{align*}
$\Box$ \\

\emph{Proof of (b).}
Exercise 3.50.
$\Box$ \\\\



%%%%%%%%%%%%%%%%%%%%%%%%%%%%%%%%%%%%%%%%%%%%%%%%%%%%%%%%%%%%%%%%%%%%%%%%%%%%%%%%



\subsubsection*{Exercise 3.48.}
\addcontentsline{toc}{subsubsection}{Exercise 3.48.}
\begin{enumerate}
\item[(a)]
\emph{$(\partial A)^{\circ} = \varnothing$ if $A$ is open or if $A$ is closed in $M$. }
\item[(b)]
\emph{Give an example in which $(\partial A)^{\circ} = M$. } \\
\end{enumerate}

\emph{Proof of (a).}
\begin{enumerate}
\item[(1)]
$A$ is open.
\begin{align*}
(\partial A)^{\circ}
&= (\overline{A} \cap \overline{M-A})^{\circ}
  &\text{(Exercise 3.51)} \\
&= (\overline{A})^{\circ} \cap (\overline{M-A})^{\circ}
  &\text{(Exercise 3.46(a))} \\
&= (\overline{A})^{\circ} \cap (M-A)^{\circ}
  &\text{($A$: open)} \\
&= (\overline{A})^{\circ} \cap (M - \overline{A})
  &\text{(Exercise 3.44)} \\
&\subseteq \overline{A} \cap (M - \overline{A})
  &(S^{\circ} \subseteq S) \\
&= \varnothing.
\end{align*}
\item[(2)]
$A$ is closed.
$\partial A
= \overline{A} \cap \overline{M-A}
= \partial (M - A)
= \varnothing$ (Exercise 3.51).
Or copy the above argument:
\begin{align*}
(\partial A)^{\circ}
&= (\overline{A} \cap \overline{M-A})^{\circ}
  &\text{(Exercise 3.51)} \\
&= (\overline{A})^{\circ} \cap (\overline{M-A})^{\circ}
  &\text{(Exercise 3.46(a))} \\
&= A^{\circ} \cap (\overline{M-A})^{\circ}
  &\text{($A$: closed)} \\
&= (M - (\overline{M-A})) \cap (\overline{M-A})^{\circ}
  &\text{(Exercise 3.43)} \\
&\subseteq (M - (\overline{M-A})) \cap \overline{M-A}
  &(S^{\circ} \subseteq S) \\
&= \varnothing.
\end{align*}

\end{enumerate}
$\Box$ \\

\emph{Proof of (b).}
Similar to Exercise 3.50. Let
\begin{enumerate}
\item[(1)]
$M = \mathbb{R}$ with the Euclidean metric of $\mathbb{R}$.
\item[(2)]
$A = \mathbb{Q} \subseteq \mathbb{R}$,
or $A = \widetilde{\mathbb{Q}} \subseteq \mathbb{R}$.
\end{enumerate}
$\Box$ \\\\



%%%%%%%%%%%%%%%%%%%%%%%%%%%%%%%%%%%%%%%%%%%%%%%%%%%%%%%%%%%%%%%%%%%%%%%%%%%%%%%%



\subsubsection*{Exercise 3.49.}
\addcontentsline{toc}{subsubsection}{Exercise 3.49.}
\emph{If $A^{\circ} = B^{\circ} = \varnothing$ and if $A$ is closed in $M$,
then $(A \cup B)^{\circ} = \varnothing$. } \\

\emph{Proof (Reductio ad absurdum).}
\begin{enumerate}
\item[(1)]
If $x \in (A \cup B)^{\circ}$, then there exists an open ball $B(x) \subseteq A \cup B$.
\item[(2)]
Since $x$ is not an interior point of $A$ ($A^{\circ} = \varnothing$),
the open ball $B(x) \not\subseteq A$, or $B(x) \cap \widetilde{A} \neq \varnothing$.
\item[(3)]
Notice that $A$ is closed.
Hence $\widetilde{A}$ is open and so is $B(x) \cap \widetilde{A}$.
Since $B(x) \cap \widetilde{A}$ is not empty,
some $y \in B(x) \cap \widetilde{A}$ and thus
there is another open ball $B(y) \subseteq B(x) \cap \widetilde{A}$.
\item[(4)]
\begin{align*}
B(y)
&\subseteq B(x) \cap \widetilde{A} \\
&\subseteq (A \cup B) \cap \widetilde{A} \\
&= (A \cap \widetilde{A}) \cup (B \cap \widetilde{A}) \\
&= \varnothing \cup (B \cap \widetilde{A}) \\
&= B \cap \widetilde{A} \\
&\subseteq B.
\end{align*}
So that $y$ is an interior point of $B$, contrary to $B^{\circ} = \varnothing$.
\end{enumerate}
Therefore, the result is established.
$\Box$ \\\\



%%%%%%%%%%%%%%%%%%%%%%%%%%%%%%%%%%%%%%%%%%%%%%%%%%%%%%%%%%%%%%%%%%%%%%%%%%%%%%%%



\subsubsection*{Exercise 3.50.}
\addcontentsline{toc}{subsubsection}{Exercise 3.50.}
\emph{Give an example in which $A^{\circ} = B^{\circ} = \varnothing$ but
$(A \cup B)^{\circ} = M$. } \\

\emph{Proof.}
Let
\begin{enumerate}
\item[(1)]
$M = \mathbb{R}$ with the Euclidean metric of $\mathbb{R}$.
\item[(2)]
$A = \mathbb{Q} \subseteq \mathbb{R}$.
\item[(3)]
$B = \widetilde{\mathbb{Q}} \subseteq \mathbb{R}$.
\end{enumerate}
$\Box$ \\\\



%%%%%%%%%%%%%%%%%%%%%%%%%%%%%%%%%%%%%%%%%%%%%%%%%%%%%%%%%%%%%%%%%%%%%%%%%%%%%%%%



\subsubsection*{Exercise 3.51.}
\addcontentsline{toc}{subsubsection}{Exercise 3.51.}
\emph{$\partial A = \overline{A} \cap \overline{M-A}$ and
$\partial A = \partial(M - A)$.} \\

Also, $\partial A = \overline{A} - A^{\circ}$ (Exercise 3.43)
and $\widetilde{\partial A} = A^{\circ} \cup \widetilde{A}^{\circ}$
(Exercise 3.43, 3.44). \\

\emph{Proof.}
Definition 3.40 and Definition 3.19 show $\partial A = \overline{A} \cap \overline{M-A}$.
Notice that $A = M - (M-A)$, and thus
$\partial A = \overline{A} \cap \overline{M-A} = \partial(M - A)$.
$\Box$ \\\\



%%%%%%%%%%%%%%%%%%%%%%%%%%%%%%%%%%%%%%%%%%%%%%%%%%%%%%%%%%%%%%%%%%%%%%%%%%%%%%%%



\subsubsection*{Exercise 3.52.}
\addcontentsline{toc}{subsubsection}{Exercise 3.52.}
\emph{If $\overline{A} \cap \overline{B} = \varnothing$,
then $\partial(A \cup B) = \partial A \cup \partial B$.} \\

The proof is all about this relation
$B(x;r) \cap \widetilde{A} \cap \widetilde{B} \neq \varnothing
\Longleftrightarrow
B(x;r) \cap \widetilde{A} \neq \varnothing.$ \\

\emph{Proof (Brute-force).}
\begin{enumerate}
\item[(1)]
$(\subseteq)$
Given any $x \in \partial(A \cup B)$.
For any $r > 0$,
\begin{align*}
&B(x;r) \cap (A \cup B) \neq \varnothing \text{ and }
B(x;r) \cap \widetilde{A \cup B} \neq \varnothing \\
\Longleftrightarrow&
(B(x;r) \cap A) \cup (B(x;r) \cap B) \neq \varnothing \text{ and }
B(x;r) \cap \widetilde{A} \cap \widetilde{B} \neq \varnothing \\
\Longleftrightarrow&
( B(x;r) \cap A \neq \varnothing
  \text{ or } B(x;r) \cap B \neq \varnothing )
\text{ and }
B(x;r) \cap \widetilde{A} \cap \widetilde{B} \neq \varnothing \\
\Longrightarrow&
( B(x;r) \cap A \neq \varnothing
  \text{ and } B(x;r) \cap \widetilde{A} \neq \varnothing)
\text{ or } \\
&( B(x;r) \cap B \neq \varnothing
  \text{ and } B(x;r) \cap \widetilde{B} \neq \varnothing) \\
\Longleftrightarrow&
x \in \partial A \text{ or } x \in \partial B \\
\Longleftrightarrow&
x \in \partial A \cup \partial B.
\end{align*}
\item[(2)]
$(\supseteq)$
Since $\overline{A} \cap \overline{B} = \varnothing$ and Exercise 3.51,
$\partial A \cap \partial B = \varnothing$
and $\partial A \cap B^{\circ} = \varnothing$
and $A^{\circ} \cap \partial B = \varnothing$.
Given any $x \in \partial A \cup \partial B$.
There are only two possible cases.
  \begin{enumerate}
  \item[(a)]
  \emph{$x \in \partial A$ and $x \not\in \partial B$.}
  \begin{align*}
  x \not\in \partial B
  \Longleftrightarrow&
  x \in M - \partial B \\
  \Longleftrightarrow&
  x \in M - ( \overline{B} \cap \overline{M-B})
    &\text{(Exercise 3.51)} \\
  \Longleftrightarrow&
  x \in (M - \overline{B}) \cup (M - \overline{M-B}) \\
  \Longleftrightarrow&
  x \in (M-B)^{\circ} \cup B^{\circ}
    &\text{(Exercise 3.43, 3.44)} \\
  \Longleftrightarrow&
  x \in (M-B)^{\circ} \text{ or } x \in B^{\circ} \\
  \Longrightarrow&
  x \in (M-B)^{\circ}
    &(\partial A \cap B^{\circ} = \varnothing)
  \end{align*}
  $x$ is an interior point of $\widetilde{B}$.
  Hence there exists $r_0 > 0$ such that $B(x;r_0) \subseteq \widetilde{B}$.
  Given any $r_0 > r > 0$, we have
  \begin{align*}
  x \in \partial A
  \Longleftrightarrow&
  B(x;r) \cap A \neq \varnothing
  \text{ and } B(x;r) \cap \widetilde{A} \neq \varnothing \\
  \Longrightarrow&
  B(x;r) \cap (A \cup B) \neq \varnothing \text{ and }\\
  &B(x;r) \cap \widetilde{A \cup B}
  = B(x;r) \cap \widetilde{A} \cap \widetilde{B}
  \neq \varnothing \\
  \Longleftrightarrow&
  x \in \partial(A \cup B).
  \end{align*}
  \item[(b)]
  \emph{$x \in \partial B$ and $x \not\in \partial A$.}
  Similar to (a).
  \end{enumerate}
\end{enumerate}
$\Box$ \\\\



%%%%%%%%%%%%%%%%%%%%%%%%%%%%%%%%%%%%%%%%%%%%%%%%%%%%%%%%%%%%%%%%%%%%%%%%%%%%%%%%
%%%%%%%%%%%%%%%%%%%%%%%%%%%%%%%%%%%%%%%%%%%%%%%%%%%%%%%%%%%%%%%%%%%%%%%%%%%%%%%%
%%%%%%%%%%%%%%%%%%%%%%%%%%%%%%%%%%%%%%%%%%%%%%%%%%%%%%%%%%%%%%%%%%%%%%%%%%%%%%%%



\newpage
\section*{Chapter 4: Limits and Continuity \\}
\addcontentsline{toc}{section}{Chapter 4: Limits and Continuity}



\subsection*{Continuity of real-valued functions \\}
\addcontentsline{toc}{subsection}{Continuity of real-valued functions}



\subsubsection*{Exercise 4.19.}
\addcontentsline{toc}{subsubsection}{Exercise 4.19.}
\emph{Let $f$ be continuous on $[a,b]$ and define $g$ as follows:
$g(a) = f(a)$ and, for $a < x \leq b$,
let $g(x)$ be the maximum value of $f$ in the subinterval $[a,x]$.
Show that $g$ is continuous on $[a,b]$.} \\

Indeed, $g(x) = \max_{a \leq t \leq x} f(t)$ for $x \in [a,b]$. \\

\emph{Proof.}
\begin{enumerate}
\item[(1)]
$f$ is continuous on $[a,b]$ at a point $p$ $\Longleftrightarrow$
Given any $\varepsilon' > 0$, there exists $\delta' > 0$ such that
$|f(x) - f(p)| < \varepsilon'$ whenever $|x-p| < \delta'$ (and $x \in [a,b]$).
We left $\varepsilon'$ and $\delta'$ undecided temporarily.
\item[(2)]
To estimate $g$ on
$$[p-\delta', p+\delta'] \cap [a,b],$$
we need to study the behavior of $f$ on $[a,p+\delta'] \cap [a,b]$
(by the definition of $g(x)$),
and then use the continuity of $f$ to establish the desired result.
\item[(3)]
Look at where $f$ takes the maximum value over on $[a,p+\delta'] \cap [a,b]$ at.
There are two possible cases (might overlapped):
  \begin{enumerate}
  \item[(a)]
  \emph{At a point in $[a,p-\delta'] \cap [a,b]$.}
  In this case $g$ is constant on $[p-\delta', p+\delta'] \cap [a,b]$,
  or $|g(x) - g(p)| = 0$.
  \item[(b)]
  \emph{At a point $q \in (p-\delta',p+\delta'] \cap [a,b]$.}
  For any $x \in [p-\delta', p+\delta'] \cap [a,b]$,
    \begin{enumerate}
    \item[(i)]
    $f(p) - \varepsilon' < g(x)$ by the maximality of $g$ on $[a,x]$.
    \item[(ii)]
    $g(x) \leq f(q) < f(p) + \varepsilon'$
    since $g$ is an increasing function and
    $f$ takes the maximum value over on $[a,p+\delta'] \cap [a,b]$ at
    $q \in (p-\delta',p+\delta'] \cap [a,b]$.
    \end{enumerate}
  By (i)(ii),
  $$f(p) - \varepsilon' < g(x) < f(p) + \varepsilon'$$
  for  any $x \in [p-\delta', p+\delta'] \cap [a,b]$ (especially $x = p$).
  Therefore,
  $$|g(x) - g(p)| < 2 \varepsilon'
  \text{ whenever } |x-p| < \delta' (\text{and } x \in [a,b]).$$

  \end{enumerate}
  By (a)(b), we have
  $|g(x) - g(p)| < 2 \varepsilon'
  \text{ whenever } |x-p| < \delta' (\text{and } x \in [a,b])$ in any cases.
\item[(4)]
Retake $\varepsilon' = \frac{\varepsilon}{2} > 0$ and $\delta = \delta' > 0$.
\end{enumerate}
$\Box$ \\\\



%%%%%%%%%%%%%%%%%%%%%%%%%%%%%%%%%%%%%%%%%%%%%%%%%%%%%%%%%%%%%%%%%%%%%%%%%%%%%%%%
%%%%%%%%%%%%%%%%%%%%%%%%%%%%%%%%%%%%%%%%%%%%%%%%%%%%%%%%%%%%%%%%%%%%%%%%%%%%%%%%



\subsection*{Continuity in metric spaces \\}
\addcontentsline{toc}{subsection}{Continuity in metric spaces}



In Exercise 4.29 through 4.33, we assume that $f: S \rightarrow T$ is a function
from one metric space $(S, d_S)$ to another $(T, d_T)$. \\



\subsubsection*{Exercise 4.29.}
\addcontentsline{toc}{subsubsection}{Exercise 4.29.}
\emph{Prove that $f$ is continuous on $S$ if and only if
$$f^{-1}(B^{\circ}) \subseteq (f^{-1}(B))^{\circ}
\qquad \text{for every subset $B$ of $T$}.$$}

Denote the interior of any set $S$ by $S^{\circ}$. \\

\emph{Proof (On topological spaces).}
\begin{enumerate}
\item[(1)]
$(\Longrightarrow)$
\begin{align*}
  \forall x \in f^{-1}(B^{\circ})
  &\Longrightarrow
  f(x) \in B^{\circ} \\
  &\Longrightarrow
  \exists \text{ open neighborhood } V \subseteq B^{\circ} \subseteq B \text{ containing } f(x) \\
  &\Longrightarrow
  x \in f^{-1}(V) \subseteq f^{-1}(B) \\
  &\Longrightarrow
  f^{-1}(V) \text{ is open in $S$ since $f$ is continuous} \\
  &\Longrightarrow
  f^{-1}(V)\text{ is open neighborhood} \subseteq f^{-1}(B) \text{ containing } x \\
  &\Longrightarrow
  x \in (f^{-1}(B))^{\circ}.
\end{align*}
\item[(2)]
$(\Longleftarrow)$
\emph{Given any open subset $V$ of $T$, need to show
$U = f^{-1}(V)$ is open in $S$.}
\begin{align*}
f^{-1}(V)
&= f^{-1}(V^{\circ})
  &\text{($V$ is open)} \\
&\subseteq (f^{-1}(V))^{\circ}
  &\text{(Assumption)}
\end{align*}
So $U \subseteq U^{\circ}$ or $U = U^{\circ}$ is open.
\end{enumerate}
$\Box$ \\\\



%%%%%%%%%%%%%%%%%%%%%%%%%%%%%%%%%%%%%%%%%%%%%%%%%%%%%%%%%%%%%%%%%%%%%%%%%%%%%%%%



\subsubsection*{Exercise 4.30.}
\addcontentsline{toc}{subsubsection}{Exercise 4.30.}
\emph{Prove that $f$ is continuous on $S$ if and only if
$$f(\overline{A}) \subseteq \overline{f(A)}
\qquad \text{for every subset $A$ of $S$}.$$}

Denote the closure of any set $S$ by $\overline{S}$. \\

\emph{Proof (On topological spaces).}
\begin{enumerate}
\item[(1)]
$(\Longrightarrow)$
Since $f$ is continuous and $\overline{f(A)}$ is closed,
$f^{-1}(\overline{f(A)})$ is closed.
Hence,
\begin{align*}
f^{-1}(\overline{f(A)})
&\supseteq f^{-1}(f(A))
  & \text{(Monotonicity of $f^{-1}$)} \\
&\supseteq A,
  & \text{(Exercise 2.7(a))} \\
\overline{A}
&\subseteq f^{-1}(\overline{f(A)}),
  & \text{(Monotonicity of closure)} \\
f(\overline{A})
&\subseteq f(f^{-1}(\overline{f(A)}))
  & \text{(Monotonicity of $f$)} \\
&\subseteq \overline{f(A)}.
  & \text{(Exercise 2.7(b))}
\end{align*}
\item[(2)]
$(\Longleftarrow)$
\emph{Given any closed subset $D$ of $T$, need to show
$C = f^{-1}(D)$ is closed in $S$.}
\begin{align*}
f(\overline{C})
&\subseteq \overline{f(C)}
  &\text{(Assumption)} \\
&= \overline{f(f^{-1}(D))}
  &(C = f^{-1}(D))\\
&\subseteq \overline{D}
  &\text{(Exercise 2.7(b))} \\
&= D,
  &\text{($D$ is closed)} \\
f^{-1}(f(\overline{C}))
&\subseteq f^{-1}(D),
  &\text{(Monotonicity of $f^{-1}$)} \\
\overline{C} \subseteq f^{-1}(f(\overline{C}))
&\subseteq f^{-1}(D) = C.
  &\text{(Exercise 2.7(a))}
\end{align*}
So $C \supseteq \overline{C}$ or $C = \overline{C}$ is closed.
\end{enumerate}
$\Box$ \\



\textbf{Supplement (Continuity).}
\emph{Let $f$ be a map from a topological space on $X$
to a topological space on $Y$.
Then, the following statements are equivalent:}
\begin{enumerate}
\item[(1)]
\emph{$f$ is continuous:
For each $x \in X$ and every neighborhood $V$ of $f(x)$,
there is a neighborhood $U$ of $x$ such that $f(U) \subseteq V$.}
\item[(2)]
\emph{For every open set $O$ in $Y$, the inverse image $f^{-1}(O)$
is open in $X$.}
\item[(3)]
\emph{For every closed set $C$ in $Y$, the inverse image $f^{-1}(C)$
is closed in $X$.}
\item[(4)]
\emph{$f(A)^{\circ} \subseteq f(A^{\circ})$ for every subset $A$ of $X$.}
\item[(5)]
\emph{$f^{-1}(B^{\circ}) \subseteq (f^{-1}(B))^{\circ}$ for every subset $B$ of $Y$.}
\item[(6)]
\emph{$f(\overline{A}) \subseteq \overline{f(A)}$ for every subset $A$ of $X$.}
\item[(7)]
\emph{$\overline{f^{-1}(B)} \subseteq f^{-1}(\overline{B})$ for every subset $B$ of $Y$.} \\\\
\end{enumerate}



%%%%%%%%%%%%%%%%%%%%%%%%%%%%%%%%%%%%%%%%%%%%%%%%%%%%%%%%%%%%%%%%%%%%%%%%%%%%%%%%



\subsubsection*{Exercise 4.33.}
\addcontentsline{toc}{subsubsection}{Exercise 4.33.}
\emph{Give an example of a continuous $f$ and a Cauchy sequence $\{x_n\}$
in some metric space $S$ for which $\{f(x_n)\}$ is not a Cauchy sequence in $T$.} \\

Compare with Exercise 4.54 to get some hints. \\

\emph{Proof.}
Let
$$S
= \left\{ 1, \frac{1}{2}, \frac{1}{3}, ... \right\}
= \left\{ \frac{1}{n} : n \in \mathbb{Z}^+ \right\}.$$
Define $f: S \rightarrow \mathbb{R}$ by $f\left(\frac{1}{n}\right) = (-1)^n$.
Then $f$ is continuous (but not uniformly continuous).
The sequence $\{x_n\} = \left\{ \frac{1}{n} \right\}$ in $S$ is a Cauchy sequence,
but the sequence $\{f(x_n)\} = \{(-1)^n\}$ is not a Cauchy sequence in $\mathbb{R}$.
$\Box$ \\\\



%%%%%%%%%%%%%%%%%%%%%%%%%%%%%%%%%%%%%%%%%%%%%%%%%%%%%%%%%%%%%%%%%%%%%%%%%%%%%%%%
%%%%%%%%%%%%%%%%%%%%%%%%%%%%%%%%%%%%%%%%%%%%%%%%%%%%%%%%%%%%%%%%%%%%%%%%%%%%%%%%



\subsection*{Uniform continuity \\}
\addcontentsline{toc}{subsection}{Uniform continuity}



\subsubsection*{Exercise 4.50.}
\addcontentsline{toc}{subsubsection}{Exercise 4.50.}
\emph{Prove that a function which is uniformly continuous on $S$
is also continuous on $S$.} \\

\emph{Proof.}
The proof is straightforward.
\begin{enumerate}
\item[(1)]
Suppose $f: S \rightarrow T$ is uniformly continuous on $S$.
Given any $\varepsilon > 0$, there is $\delta > 0$ such that
$d_T(f(x), f(y)) < \varepsilon$ whenever $d_S(x, y) < \delta$.
\item[(2)]
\emph{Show that $f$ is continuous at any point $p$ in $S$.}
Set $y = p$ in (1).
\end{enumerate}
$\Box$ \\\\



%%%%%%%%%%%%%%%%%%%%%%%%%%%%%%%%%%%%%%%%%%%%%%%%%%%%%%%%%%%%%%%%%%%%%%%%%%%%%%%%



\subsubsection*{Exercise 4.51.}
\addcontentsline{toc}{subsubsection}{Exercise 4.51.}
\emph{If $f(x) = x^2$ for $x \in \mathbb{R}$,
prove that $f$ is not uniformly continuous on $\mathbb{R}$}. \\

\emph{Proof.}
Prove by contradiction.
\begin{enumerate}
\item[(1)]
If $f$ were uniformly continuous on $\mathbb{R}$,
then for any $\varepsilon > 0$, there is $\delta > 0$ such that
$|f(x) - f(y)| < \varepsilon$ whenever $|x - y| < \delta$.
Here we pick $\varepsilon = 1 > 0$.
\item[(2)]
So
$$|f(x) - f(y)| = |x^2 - y^2| = |x+y||x-y| < 1$$
for any $|x-y| < \delta$.
In particular, we pick $x = \frac{1}{\delta}$
and $y = \frac{1}{\delta} + \frac{\delta}{2}$.
Now $|x-y| = \frac{\delta}{2} < \delta$, and thus $|f(x) - f(y)| = |x+y||x-y| < 1$
would be true.
However,
$$|f(x) - f(y)| = |x+y||x-y|
= \left(\frac{2}{\delta}+ \frac{\delta}{2}\right)
\left(\frac{\delta}{2}\right)
> \frac{2}{\delta} \cdot \frac{\delta}{2} = 1,$$
contrary to $|f(x) - f(y)| = |x+y||x-y| < 1$.
\end{enumerate}
$\Box$ \\\\



%%%%%%%%%%%%%%%%%%%%%%%%%%%%%%%%%%%%%%%%%%%%%%%%%%%%%%%%%%%%%%%%%%%%%%%%%%%%%%%%



\subsubsection*{Exercise 4.52.}
\addcontentsline{toc}{subsubsection}{Exercise 4.52.}
\emph{Assume that $f$ is uniformly continuous on a bounded set $S$ in $\mathbb{R}^n$.
Prove that $f$ must be bounded on $S$.} \\

The conclusion is false if boundedness of $S$ is omitted from the hypothesis.
For example, $f(x) = x$ on $\mathbb{R}$ is uniformly continuous on $\mathbb{R}$
but $f(\mathbb{R}) = \mathbb{R}$ is unbounded. \\

\emph{Proof (Brute-force).}
\begin{enumerate}
\item[(1)]
Since $f: S \rightarrow T$ is uniformly continuous,
given any $\varepsilon > 0$, there is $\delta > 0$ such that
$d_T(f(x), f(y)) < \varepsilon$ whenever $d_S(x, y) < \delta$.
In particular, pick $\varepsilon = 1$.
\item[(2)]
By the boundedness of $S$, there is $M > 0$ such that $\norm{x} < M$ for all $x \in S$.
In particular, each coordinate of $x \in \mathbb{R}^n$ is less than $M$.
\item[(3)]
For such $\delta > 0$, we construct a covering of $S \subseteq \mathbb{R}^n$.
Construct a special collection $\mathscr{C}$ of $n$-cells
$$I_{\mathbf{a}} =
  \left[ \frac{\delta}{2\sqrt{n}}a_1, \frac{\delta}{2\sqrt{n}}(a_1+1) \right]
  \times
  \cdots
  \times
  \left[ \frac{\delta}{2\sqrt{n}}a_n, \frac{\delta}{2\sqrt{n}}(a_n+1) \right]
$$
where $\mathbf{a} = (a_1, ..., a_n) \in \mathbb{Z}^n$ satisfying
$$\abs{a_i} < \frac{2\sqrt{n}M}{\delta} + 1 \:\: (1 \leq i \leq n).$$
By construction, $\mathscr{C}$ is a finite covering of $S$.
\item[(4)]
For every $n$-cell $I_{\mathbf{a}}$ of the collection $\mathscr{C}$,
pick a point $x_{\mathbf{a}} \in S \bigcap I_{\mathbf{a}}$ if possible.
This process will terminate eventually since $\mathscr{C}$ is a finite.
Collect these representative points as $\mathscr{D} = \{ x_{\mathbf{a}} \}$.
Notice that $\mathscr{D}$ is finite again.
\item[(5)]
Now for any point $x \in S$, $x$ lies in some $I_{\mathbf{a}}$
containing $x_\mathbf{a}$.
Both $x$ and $x_\mathbf{a}$ are in the same cell and their distance satisfies
$$\norm{x - x_\mathbf{a}}
\leq \sqrt{
\left(\frac{\delta}{2\sqrt{n}}\right)^2 +
\cdots +
\left(\frac{\delta}{2\sqrt{n}}\right)^2}
= \frac{\delta}{2}
< \delta$$
and thus by (1)
$$\norm{f(x) - f(x_\mathbf{a})} < 1,
\text{ or }
\norm{f(x)} <  1 + \norm{f(x_\mathbf{a})}.$$
\item[(6)]
Let
$$M = 1 + \max_{x_\mathbf{a} \in \mathscr{D}} \norm{f(x_\mathbf{a})}.$$
So given any $x \in S$, $\norm{f(x)} < M$.
\end{enumerate}
$\Box$ \\

\emph{Proof (Heine-Borel Theorem).}
Heine-Borel theorem provides the finiteness property to construct
the boundedness property of $f$.
\begin{enumerate}
\item[(1)]
\emph{Let $S$ be a bounded subset of a metric space $X$.
Show that the closure of $S$ in $X$ is also bounded in $X$.}
$S$ is bounded if $S \subseteq B_X(a;r)$ for some $r > 0$ and some $a \in X$.
(The ball $B_X(a;r)$ is defined to the set of all $x \in X$ such that
$d_X(x, a) < r$.)
Take the closure on the both sides,
$$\overline{S}
\subseteq \overline{B_X(a;r)}
= \{ x \in X : d_X(x, a) \leq r \}
\subseteq B_X(a;2r),$$
or $\overline{S}$ is bounded.
\item[(2)]
Since $f: S \rightarrow T$ is uniformly continuous,
given any $\varepsilon > 0$, there is $\delta > 0$ such that
$d_T(f(x), f(y)) < \varepsilon$ whenever $d_S(x, y) < \delta$.
In particular, pick $\varepsilon = 1$.
\item[(3)]
For such $\delta > 0$, we construct an open covering of $\overline{S} \subseteq \mathbb{R}^n$.
Pick a collection $\mathscr{C}$ of open balls
$B(a;\delta) \subseteq \mathbb{R}^n$
where $a$ runs over all elements of $S$.
$\mathscr{C}$ covers $\overline{S}$ (by the definition of accumulation points).
Since $\overline{S} $ is closed and bounded (by applying (1) on the boundedness of $S$),
$\overline{S}$ is compact
(Heine-Borel theorem on $\mathbb{R}^n$).
That is, there is a finite subcollection $\mathscr{C}'$ of $\mathscr{C}$
also covers $\overline{S}$, say
$$\mathscr{C}'
= \left\{B(a_1;\delta)), B(a_2;\delta), ..., B(a_m;\delta) \right\}.$$
\item[(4)]
Given any $x \in S \subseteq \overline{S}$,
there is some $a_i \in S$ $(1 \leq i \leq m)$ such that $x \in B(a_i;\delta)$.
In such ball, $d_S(x, a_i) < \delta$.
By (2), $\norm{f(x) - f(a_i)} < 1$,
or $\norm{f(x)} < 1 + \norm{f(a_i)}$.
Almost done.
Notice that $a_i$ depends on $x$,
and thus we might use finiteness of $\{ a_1, a_2, ..., a_m \}$
to remove dependence of $a_i$.
\item[(5)]
Let
$$M = 1 + \max_{1 \leq i \leq m}{\norm{f(a_i)}}.$$
So given any $x \in S$, $\norm{f(x)} < M$.
\end{enumerate}
$\Box$ \\

\textbf{Supplement.}
Exercise about considering the closure.
(Problem 3.5 in H. L. Royden, Real Analysis, 3rd Edition.)
\emph{Let $A = \mathbb{Q} \cap [0,1]$,
and let $\{ I_n\}$ be a finite collection of open intervals covering $A$.
Then $\sum l(I_n) \geq 1$.} \\

\emph{Proof.}
\begin{align*}
1
= m^{*}[0, 1]
= m^{*}\overline{A}
&\leq m^{*}\left( \overline{\bigcup I_n} \right)
= m^{*}\left( \bigcup \overline{I_n} \right) \\
&\leq \sum m^{*}(\overline{I_n})
= \sum l(\overline{I_n})
= \sum l(I_n).
\end{align*}
$\Box$ \\\\



%%%%%%%%%%%%%%%%%%%%%%%%%%%%%%%%%%%%%%%%%%%%%%%%%%%%%%%%%%%%%%%%%%%%%%%%%%%%%%%%



\subsubsection*{Exercise 4.54.}
\addcontentsline{toc}{subsubsection}{Exercise 4.54.}
\emph{Assume $f: S \rightarrow T$ is uniformly continuous on $S$,
where $S$ and $T$ are metric spaces.
If $\{x_n\}$ is any Cauchy sequence in $S$,
prove that $\{f(x_n)\}$ is a Cauchy sequence in $T$.
(Compare with Exercise 4.33.)} \\

Therefore, we need to find a continuous but not uniformly continuous function
to solve Exercise 4.33:
\emph{Give an example of a continuous $f$ and a Cauchy sequence $\{x_n\}$
in some metric space $S$ for which $\{f(x_n)\}$ is not a Cauchy sequence in $T$.} \\

\emph{Proof.}
The proof is straightforward.
\begin{enumerate}
\item[(1)]
Since $f: S \rightarrow T$ is uniformly continuous on $S$,
given any $\varepsilon > 0$, there is $\delta > 0$ such that
$d_T(f(x), f(y)) < \varepsilon$ whenever $d_S(x, y) < \delta$.
\item[(2)]
Since $\{x_n\}$ is any Cauchy sequence in $S$,
especially for such $\delta > 0$ in (1), there is an integer $N$ such that
$d_S(x_m, x_n) < \delta$ whenever $m \geq N$ and $n \geq N$.
So as $m \geq N$ and $n \geq N$, we have
$d_T(f(x_m), f(x_n)) < \varepsilon$ by (1),
or $\{f(x_n)\}$ itself is a Cauchy sequence in $T$.
\end{enumerate}
$\Box$ \\\\



%%%%%%%%%%%%%%%%%%%%%%%%%%%%%%%%%%%%%%%%%%%%%%%%%%%%%%%%%%%%%%%%%%%%%%%%%%%%%%%%
%%%%%%%%%%%%%%%%%%%%%%%%%%%%%%%%%%%%%%%%%%%%%%%%%%%%%%%%%%%%%%%%%%%%%%%%%%%%%%%%
%%%%%%%%%%%%%%%%%%%%%%%%%%%%%%%%%%%%%%%%%%%%%%%%%%%%%%%%%%%%%%%%%%%%%%%%%%%%%%%%



\end{document}