\documentclass{article}
\usepackage{amsfonts}
\usepackage{amsmath}
\usepackage{amssymb}
\usepackage{hyperref}
\usepackage[none]{hyphenat}
\usepackage{mathrsfs}
\usepackage{physics}
\parindent=0pt

\def\upint{\mathchoice%
    {\mkern13mu\overline{\vphantom{\intop}\mkern7mu}\mkern-20mu}%
    {\mkern7mu\overline{\vphantom{\intop}\mkern7mu}\mkern-14mu}%
    {\mkern7mu\overline{\vphantom{\intop}\mkern7mu}\mkern-14mu}%
    {\mkern7mu\overline{\vphantom{\intop}\mkern7mu}\mkern-14mu}%
  \int}
\def\lowint{\mkern3mu\underline{\vphantom{\intop}\mkern7mu}\mkern-10mu\int}

\begin{document}

\textbf{\Large Chapter 3: Elements of Point Set Topology} \\\\



\emph{Author: Meng-Gen Tsai} \\
\emph{Email: plover@gmail.com} \\\\



%%%%%%%%%%%%%%%%%%%%%%%%%%%%%%%%%%%%%%%%%%%%%%%%%%%%%%%%%%%%%%%%%%%%%%%%%%%%%%%%
%%%%%%%%%%%%%%%%%%%%%%%%%%%%%%%%%%%%%%%%%%%%%%%%%%%%%%%%%%%%%%%%%%%%%%%%%%%%%%%%



\textbf{\large Compact subsets of a metric space} \\\\

\emph{Prove each of the following statements concerning an arbitrary metric space
$(M,d)$ and subsets $S$, $T$ of $M$.} \\\\

\textbf{Exercise 3.39.}
\emph{If $S$ is closed and $T$ is compact, then $S \cap T$ is compact.} \\

\emph{Proof (On topological spaces).}
Let $\mathscr{F}$ be an open covering of $S \cap T$, say
$S \cap T \subseteq \bigcup_{A \in \mathscr{F}}A$.
We will show that a finite number of the sets $A$ cover $S \cap T$.
Since $S$ is closed its complement $\widetilde{S}$ in $M$ is open,
so $\mathscr{F} \cup \{ \widetilde{S} \}$ is an open covering of $T$.
Since $T$ is compact, so this covering contains a finite subcovering
which we can assume includes $\widetilde{S}$.
Therefore,
$$T \subseteq A_1 \cup \cdots \cup A_p \cup \widetilde{S}.$$
This subcovering also covers $S \cap T$ and, since $\widetilde{S}$ contains no points of $S$,
we can delete the set $\widetilde{S}$ for the subcovering and still covers $S \cap T$.
Thus
$$S \cap T \subseteq A_1 \cup \cdots \cup A_p$$
so $S \cap T$ is compact.
$\Box$ \\

\emph{Proof (Theorem 3.39).}
\begin{align*}
&\text{$T$ is compact in $(M,d)$} \\
\Longrightarrow&
\text{$T$ is compact in $(T,d)$}
  &\text{(Exercise 3.38)} \\
\Longrightarrow&
\text{$S \cap T$ is compact in $(T,d)$}
  &\text{($S \cap T$: closed in $(T,d)$, Theorem 3.38)} \\
\Longrightarrow&
\text{$S \cap T$ is compact in $(M,d)$}.
  &\text{(Exercise 3.38)}
\end{align*}
$\Box$ \\\\



%%%%%%%%%%%%%%%%%%%%%%%%%%%%%%%%%%%%%%%%%%%%%%%%%%%%%%%%%%%%%%%%%%%%%%%%%%%%%%%%



\textbf{Exercise 3.41.}
\emph{The union of a finite number of compact subsets of $M$ is compact.} \\

\emph{Proof (On topological spaces).}
Let $K_1, \ldots, K_n$ be compact subsets of $M$.
Let $\mathscr{F}$ be an open covering of $K_1 \cup \cdots \cup K_n$, say
$$K_1 \cup \cdots \cup K_n \subseteq \bigcup_{A \in \mathscr{F}}A.$$
We will show that a finite number of the sets $A$ cover $K_1 \cup \cdots \cup K_n$.
Clearly $\mathscr{F}$ is an open covering of every $K_i$.
Since $K_i$ is compact, this covering contains a finite subcovering $\mathscr{F}_i$,
say
$$K_i \subseteq A_{1(i)} \cup \cdots \cup A_{p(i)}.$$
Union all finite subcovering $\mathscr{F}_i$ to get a finite subcovering of
$K_1 \cup \cdots \cup K_n$, say
$$K_1 \cup \cdots \cup K_n \subseteq
\bigcup_{A \in \bigcup_{1 \leq i \leq n} \mathscr{F}_i}A.$$
$\Box$ \\

\textbf{Supplement (Zariski topology).}
\emph{Let $A$ be a ring and let $X$ be the set of all prime ideals of $A$.
For each subset $E$ of $A$,
let $V(E)$ denote the set of all prime ideals of $A$ which contain $E$.
The sets $V(E)$ satisfy
the axioms for closed sets in a topological space.
The resulting topology is called the Zariski topology.
The topological space $X$ is called the prime spectrum of $A$,
and is written $\text{Spec}(A)$.} \\

\emph{For each $f \in A$,
let $X_f$ denote the complement of $V(f)$ in $X = \text{Spec}(A)$.
The sets $X_f$ are open.
Show that they form a basis of open sets for the Zariski topology, and that}
\begin{enumerate}
\item[(1)]
\emph{Each $X_f$ is quasi-compact (compact), that is,
every open covering of $X$ has a finite subcovering.}
\item[(2)]
\emph{An open subset of $X$ is quasi-compact if and only if
it is a finite union of sets $X_f$.} \\
\end{enumerate}

By Exercise 3.41, we know that $X$ is quasi-compact if
$X$ is a finite union of quasi-compact sets $X_f$. \\\\



%%%%%%%%%%%%%%%%%%%%%%%%%%%%%%%%%%%%%%%%%%%%%%%%%%%%%%%%%%%%%%%%%%%%%%%%%%%%%%%%
%%%%%%%%%%%%%%%%%%%%%%%%%%%%%%%%%%%%%%%%%%%%%%%%%%%%%%%%%%%%%%%%%%%%%%%%%%%%%%%%



\end{document}