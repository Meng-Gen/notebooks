\documentclass{article}
\usepackage{amsfonts}
\usepackage{amsmath}
\usepackage{amssymb}
\usepackage{hyperref}
\usepackage[none]{hyphenat}
\usepackage{mathrsfs}
\usepackage{physics}
\parindent=0pt

\def\upint{\mathchoice%
    {\mkern13mu\overline{\vphantom{\intop}\mkern7mu}\mkern-20mu}%
    {\mkern7mu\overline{\vphantom{\intop}\mkern7mu}\mkern-14mu}%
    {\mkern7mu\overline{\vphantom{\intop}\mkern7mu}\mkern-14mu}%
    {\mkern7mu\overline{\vphantom{\intop}\mkern7mu}\mkern-14mu}%
  \int}
\def\lowint{\mkern3mu\underline{\vphantom{\intop}\mkern7mu}\mkern-10mu\int}

\begin{document}

\textbf{\Large Chapter 3: Elements of Point Set Topology} \\\\



\emph{Author: Meng-Gen Tsai} \\
\emph{Email: plover@gmail.com} \\\\



%%%%%%%%%%%%%%%%%%%%%%%%%%%%%%%%%%%%%%%%%%%%%%%%%%%%%%%%%%%%%%%%%%%%%%%%%%%%%%%%
%%%%%%%%%%%%%%%%%%%%%%%%%%%%%%%%%%%%%%%%%%%%%%%%%%%%%%%%%%%%%%%%%%%%%%%%%%%%%%%%



\textbf{Notation.}
\begin{enumerate}
\item[(1)]
$S^{\circ}$ or $\text{int}(S)$ is the interior of $S$.
\item[(2)]
$\overline{S}$ is the closure of $S$.
\item[(3)]
$\widetilde{S}$ is the complement of $S$. \\\\
\end{enumerate}



%%%%%%%%%%%%%%%%%%%%%%%%%%%%%%%%%%%%%%%%%%%%%%%%%%%%%%%%%%%%%%%%%%%%%%%%%%%%%%%%
%%%%%%%%%%%%%%%%%%%%%%%%%%%%%%%%%%%%%%%%%%%%%%%%%%%%%%%%%%%%%%%%%%%%%%%%%%%%%%%%



\textbf{\large Open and closed sets in $\mathbb{R}^1$ and $\mathbb{R}^2$} \\\\



\textbf{Exercise 3.5.}
\emph{Prove that the only sets in $\mathbb{R}^1$ which are both open and closed
are the empty set and $\mathbb{R}^1$ itself.
Is a similar statement true for $\mathbb{R}^2$?} \\

\textbf{Lemma.}
\emph{A topological space $X$ is connected if the empty set
and the whole space $X$ are the only subsets of $X$ that are both open and
closed.} \\

\emph{Proof of Lemma (Compare to Exercise 4.37).}
Given a subset $U$ of $X$ that is both open and closed.
Write $V = X-U$.
Then $U$ and $V$ are both open, $U \cup V = X$ and $U \cap V = \varnothing$.
Since $X$ is connected, $U = \varnothing$ or $V = \varnothing$,
or $U = \varnothing$ or $U = X$.
$\Box$ \\

\emph{Proof (On $\mathbb{R}^n$).}
\begin{enumerate}
\item[(1)]
\emph{$\mathbb{R}^n$ is path-connected (arcwise connected).}
Trivial.
\item[(2)]
\emph{$\mathbb{R}^n$ is connected.}
Theorem 4.42.
\item[(3)]
By Lemma, the only sets in $\mathbb{R}^n$ which are both open and closed
are $\varnothing$ and $\mathbb{R}^n$.
\end{enumerate}
$\Box$ \\\\



%%%%%%%%%%%%%%%%%%%%%%%%%%%%%%%%%%%%%%%%%%%%%%%%%%%%%%%%%%%%%%%%%%%%%%%%%%%%%%%%
%%%%%%%%%%%%%%%%%%%%%%%%%%%%%%%%%%%%%%%%%%%%%%%%%%%%%%%%%%%%%%%%%%%%%%%%%%%%%%%%



\textbf{\large Compact subsets of a metric space} \\\\

\emph{Prove each of the following statements concerning an arbitrary metric space
$(M,d)$ and subsets $S$, $T$ of $M$.} \\\\



\textbf{Exercise 3.38.}
\emph{Assume $S \subseteq T \subseteq M$.
Then $S$ is compact in $(M,d)$ if, and only if,
$S$ is compact in the metric subspace $(T,d)$. } \\

\emph{Proof.}
\begin{enumerate}
\item[(1)]
$(\Longrightarrow)$
Let $\mathscr{F}$ be an open covering of $S$ in $(T,d)$, say
$S \subseteq \bigcup_{A \in \mathscr{F}} A$ where $A$ is open in $T$.
Then $A = B \cap T$ for some open set $B$ in $M$ (Theorem 3.33).
Let $\mathscr{G}$ be the collection of $B$.
Then
$$S \subseteq
\bigcup_{A \in \mathscr{F}} A
= \bigcup_{B \in \mathscr{G}} (B \cap T)
\subseteq \bigcup_{B \in \mathscr{G}} B,$$
or $\mathscr{G}$ be an open covering of $S$ in $(M,d)$.
Since $S$ is compact in $(M,d)$,
$\mathscr{G}$ contains a finite subcovering, say
$$S \subseteq B_1 \cap \cdots \cap B_p.$$
So $$S \cap T \subseteq (B_1\cap T) \cap \cdots \cap (B_p \cap T),$$
or $$S \subseteq A_1 \cap \cdots \cap A_p$$ (since $S \subseteq T$ or $S \cap T = S$).
So there is a finite subcovering of $\mathscr{F}$ covering $S$,
or $S$ is compact in $(T,d)$.
\item[(2)]
$(\Longleftarrow)$
Let $\mathscr{G}$ be an open covering of $S$ in $(M,d)$, say
$S \subseteq \bigcup_{B \in \mathscr{G}} B$ where $B$ is open in $M$.
Then $A = B \cap T$ is open in $T$.
Let $\mathscr{F}$ be the collection of $A$.
Then
$$S \cap T \subseteq
\bigcup_{B \in \mathscr{G}} (B \cap T)
= \bigcup_{A \in \mathscr{F}} A,$$
or $\mathscr{F}$ be an open covering of $S \cap T = S$ in $(T,d)$.
Since $S$ is compact in $(T,d)$,
$\mathscr{F}$ contains a finite subcovering, say
$$S \subseteq A_1 \cap \cdots \cap A_p.$$
Clearly, $S \subseteq B_1 \cap \cdots \cap B_p$ since $A = B \cap T \subseteq B$.
So there is a finite subcovering of $\mathscr{G}$ covering $S$,
or $S$ is compact in $(M,d)$.
\end{enumerate}
$\Box$ \\\\



%%%%%%%%%%%%%%%%%%%%%%%%%%%%%%%%%%%%%%%%%%%%%%%%%%%%%%%%%%%%%%%%%%%%%%%%%%%%%%%%



\textbf{Exercise 3.39.}
\emph{If $S$ is closed and $T$ is compact, then $S \cap T$ is compact.} \\

\emph{Idea.}
Cover $S \cap T$ be open sets in $M$, throw in the open set $\widetilde{S}$
and apply the compactness of $T$. \\

\emph{Proof (On topological spaces).}
Let $\mathscr{F}$ be an open covering of $S \cap T$, say
$S \cap T \subseteq \bigcup_{A \in \mathscr{F}}A$.
We will show that a finite number of the sets $A$ cover $S \cap T$.
Since $S$ is closed its complement $\widetilde{S}$ in $M$ is open,
so $\mathscr{F} \cup \{ \widetilde{S} \}$ is an open covering of $T$.
Since $T$ is compact, so this covering contains a finite subcovering
which we can assume includes $\widetilde{S}$.
Therefore,
$$T \subseteq A_1 \cup \cdots \cup A_p \cup \widetilde{S}.$$
This subcovering also covers $S \cap T$ and, since $\widetilde{S}$ contains no points of $S$,
we can delete the set $\widetilde{S}$ for the subcovering and still covers $S \cap T$.
Thus
$$S \cap T \subseteq A_1 \cup \cdots \cup A_p$$
so $S \cap T$ is compact.
$\Box$ \\

\emph{Proof (Theorem 3.39).}
\begin{align*}
&\text{$T$ is compact in $(M,d)$} \\
\Longrightarrow&
\text{$T$ is compact in $(T,d)$}
  &\text{(Exercise 3.38)} \\
\Longrightarrow&
\text{$S \cap T$ is compact in $(T,d)$}
  &\text{($S \cap T$: closed in $(T,d)$, Theorem 3.38)} \\
\Longrightarrow&
\text{$S \cap T$ is compact in $(M,d)$}.
  &\text{(Exercise 3.38)}
\end{align*}
$\Box$ \\\\



%%%%%%%%%%%%%%%%%%%%%%%%%%%%%%%%%%%%%%%%%%%%%%%%%%%%%%%%%%%%%%%%%%%%%%%%%%%%%%%%



\textbf{Exercise 3.40.}
\emph{The intersection of an arbitrary collection of compact subsets of $M$ is compact.} \\

\emph{Note.}
\begin{enumerate}
\item[(1)]
A metric space is Hausdorff.
\item[(2)]
If $X$ is not Hausdorff then the intersection of two compact subsets may fail to be compact.
Example:
  \begin{enumerate}
  \item[(a)]
  Let $X = \{ a, b \} \cup \mathbb{N}$,
  $U = \{ a \} \cup \mathbb{N}$,
  and $V = \{ b \} \cup \mathbb{N}$.
  \item[(b)]
  Endow $X$ with the topology generated by the following basic open sets:
    \begin{enumerate}
    \item[(i)]
    every subset of $\mathbb{N}$ is open;
    \item[(ii)]
    the only open sets containing $a$ are $X$ and $U$;
    \item[(iii)]
    the only open sets containing $b$ are $X$ and $V$.
    \end{enumerate}
  \item[(c)]
  $U$ and $V$ are both compact subsets but their intersection,
  which is $\mathbb{N}$, is not compact.
  \end{enumerate}
\item[(3)]
A compact subset of a metric space is closed and bounded.
The converse may fail (Exercise 3.42).
Therefore,
we cannot establish the result of Exercise 3.40
by only showing
the intersection of an arbitrary collection of compact subsets is closed and bounded. \\
\end{enumerate}

\emph{Proof (On Hausdorff spaces or metric spaces).}
Given an index set $I$.
Let $K_\alpha$ be compact subsets of $M$ for every $\alpha \in I$.
Since $M$ is a Hausdorff space (or a metric space),
each $K_\alpha$ is closed in $M$.
Hence
$\bigcap_{\alpha \in I} K_{\alpha}$ is closed in $M$,
especially in every compact set $K_{\beta}$ for $\beta \in I$.
Hence
$$\bigcap_{\alpha \in I} K_{\alpha}
= \left( \bigcap_{\alpha \in I} K_{\alpha} \right) \bigcap K_{\beta}$$
is compact by Exercise 3.39.
$\Box$ \\\\



%%%%%%%%%%%%%%%%%%%%%%%%%%%%%%%%%%%%%%%%%%%%%%%%%%%%%%%%%%%%%%%%%%%%%%%%%%%%%%%%



\textbf{Exercise 3.41.}
\emph{The union of a finite number of compact subsets of $M$ is compact.} \\

\emph{Proof (On topological spaces).}
Let $K_1, \ldots, K_n$ be compact subsets of $M$.
Let $\mathscr{F}$ be an open covering of $K_1 \cup \cdots \cup K_n$, say
$$K_1 \cup \cdots \cup K_n \subseteq \bigcup_{A \in \mathscr{F}}A.$$
We will show that a finite number of the sets $A$ cover $K_1 \cup \cdots \cup K_n$.
Clearly $\mathscr{F}$ is an open covering of every $K_i$.
Since $K_i$ is compact, this covering contains a finite subcovering $\mathscr{F}_i$,
say
$$K_i \subseteq A_{1(i)} \cup \cdots \cup A_{p(i)}.$$
Union all finite subcovering $\mathscr{F}_i$ to get a finite subcovering of
$K_1 \cup \cdots \cup K_n$, say
$$K_1 \cup \cdots \cup K_n \subseteq
\bigcup_{A \in \bigcup_{1 \leq i \leq n} \mathscr{F}_i}A.$$
$\Box$ \\

\textbf{Supplement (Zariski topology).}
\emph{Let $A$ be a ring and let $X$ be the set of all prime ideals of $A$.
For each subset $E$ of $A$,
let $V(E)$ denote the set of all prime ideals of $A$ which contain $E$.
The sets $V(E)$ satisfy
the axioms for closed sets in a topological space.
The resulting topology is called the Zariski topology.
The topological space $X$ is called the prime spectrum of $A$,
and is written $\text{Spec}(A)$.} \\

\emph{For each $f \in A$,
let $X_f$ denote the complement of $V(f)$ in $X = \text{Spec}(A)$.
The sets $X_f$ are open.
Show that they form a basis of open sets for the Zariski topology, and that}
\begin{enumerate}
\item[(1)]
\emph{Each $X_f$ is quasi-compact (compact), that is,
every open covering of $X$ has a finite subcovering.}
\item[(2)]
\emph{An open subset of $X$ is quasi-compact if and only if
it is a finite union of sets $X_f$.} \\
\end{enumerate}

By Exercise 3.41, we know that $X$ is quasi-compact if
$X$ is a finite union of quasi-compact sets $X_f$. \\\\



%%%%%%%%%%%%%%%%%%%%%%%%%%%%%%%%%%%%%%%%%%%%%%%%%%%%%%%%%%%%%%%%%%%%%%%%%%%%%%%%



\textbf{Exercise 3.42.}
\emph{Consider the metric space $\mathbb{Q}$ of rational numbers with
the Euclidean metric of $\mathbb{R}$.
Let $S$ consist of all rational numbers in the open interval $(a,b)$,
where $a$ and $b$ are irrational.
Then $S$ is a closed and bounded subset of $\mathbb{Q}$ which is not compact.} \\

\emph{Proof.}
\begin{enumerate}
\item[(1)]
$S$ is a subset of $\mathbb{Q}$.
\item[(2)]
\emph{Show that $S$ is bounded in $\mathbb{Q}$.}
Since $\mathbb{Q}$ is dense in $\mathbb{R}$,
there is $p \in \mathbb{Q}$ such that $a < p < b$, or $p \in S$.
Let $r = \max\{b - p, p - a\} > 0$.
Therefore, $S \subseteq B(p; r)$ for some $r > 0$ and $p \in S$,
or $S$ is bounded.
\item[(3)]
\emph{Show that $S$ is closed in $\mathbb{Q}$.}
It suffices to show its complement is open in $\mathbb{Q}$.
Given any
$p \in \widetilde{S} = ((-\infty, a] \cup [b, \infty)) \cap \mathbb{Q}$.
$p \leq a$ or $p \geq b$.
  \begin{enumerate}
  \item[(a)]
  $p \leq a$. $p \neq a$ since $p \in \mathbb{Q}$ and $a$ is irrational.
  So $p < a$ and thus there exists $q \in \mathbb{Q}$ such that $p < q < a$
  since $\mathbb{Q}$ is dense in $\mathbb{R}$.
  Let $r = \max\{a - q, q - p\} > 0$.
  The ball $B(q; r)$ is contained in $\widetilde{S}$.
  \item[(b)]
  $p \geq b$. Similar to (a).
  \end{enumerate}
  By (a)(b), $\widetilde{S}$ is open in $\mathbb{Q}$, or $S$ is closed in $\mathbb{Q}$.
\item[(4)]
\emph{Show that $S$ is not compact in $\mathbb{Q}$.}
(Reductio ad absurdum)
If $(a,b)$ were compact in the metric space $\mathbb{Q}$,
$(a,b)$ is compact in the metric space $\mathbb{R}$ (Exercise 3.38),
which is absurd.
\end{enumerate}
$\Box$ \\\\



% No exercises left.

%%%%%%%%%%%%%%%%%%%%%%%%%%%%%%%%%%%%%%%%%%%%%%%%%%%%%%%%%%%%%%%%%%%%%%%%%%%%%%%%
%%%%%%%%%%%%%%%%%%%%%%%%%%%%%%%%%%%%%%%%%%%%%%%%%%%%%%%%%%%%%%%%%%%%%%%%%%%%%%%%



\textbf{\large Miscellaneous properties of the interior and the boundary} \\\\

\emph{If $A$ and $B$ denote arbitrary subsets of a metric space $M$, prove that:} \\\\



\textbf{Exercise 3.43.}
\emph{$A^{\circ} = M - \overline{M - A}.$} \\

Beautiful writing: $M - A^{\circ} = \overline{M - A}$. \\

\emph{Proof (Brute-force).}
\begin{align*}
x \in A^{\circ}
\Longleftrightarrow&
\exists r > 0 \text{ such that } B(x;r) \subseteq A \\
\Longleftrightarrow&
\exists r > 0 \text{ such that } B(x;r) \cap (M - A) = \varnothing \\
\Longleftrightarrow&
x \not\in \overline{M - A} \\
\Longleftrightarrow&
x \in M - \overline{M - A}.
\end{align*}
$\Box$ \\

\emph{Proof (Exercise 3.44).}
Put $A \mapsto M-A$ in Exercise 3.44.
$\Box$ \\

\emph{Proof (On topological spaces).}
\begin{align*}
M - A^{\circ}
&= M - \bigcup_{\text{Open } V \subseteq A} V \\
&= \bigcap_{\text{Open } V \subseteq A} (M - V) \\
&= \bigcap_{\text{Closed } W \supseteq M-A} W \\
&= \overline{M - A}.
\end{align*}
$\Box$ \\\\



%%%%%%%%%%%%%%%%%%%%%%%%%%%%%%%%%%%%%%%%%%%%%%%%%%%%%%%%%%%%%%%%%%%%%%%%%%%%%%%%



\textbf{Exercise 3.44.}
\emph{$(M-A)^{\circ} = M - \overline{A}.$} \\

\emph{Proof (Brute-force).}
\begin{align*}
x \in (M-A)^{\circ}
\Longleftrightarrow&
\exists r > 0 \text{ such that } B(x;r) \subseteq (M - A) \\
\Longleftrightarrow&
\exists r > 0 \text{ such that } B(x;r) \cap A = \varnothing \\
\Longleftrightarrow&
x \not\in \overline{A} \\
\Longleftrightarrow&
x \in M - \overline{A}.
\end{align*}
$\Box$ \\

\emph{Proof (Exercise 3.43).}
Put $A \mapsto M-A$ in Exercise 3.43.
$\Box$ \\

\emph{Proof (On topological spaces).}
\begin{align*}
M - \overline{A}
&= M - \bigcap_{\text{Closed } W \supseteq A} W \\
&= \bigcup_{\text{Closed } W \supseteq A} (M - W) \\
&= \bigcup_{\text{Open } V \subseteq M-A} V \\
&= (M - A)^{\circ}.
\end{align*}
$\Box$ \\\\



%%%%%%%%%%%%%%%%%%%%%%%%%%%%%%%%%%%%%%%%%%%%%%%%%%%%%%%%%%%%%%%%%%%%%%%%%%%%%%%%



\textbf{Exercise 3.45.}
\emph{$(A^{\circ})^{\circ} = A^{\circ}.$} \\

\emph{Proof (Brute-force).}
\begin{enumerate}
\item[(1)]
It suffices to show that $A^{\circ} \subseteq (A^{\circ})^{\circ}$.
Given any point $x \in A^{\circ}$, there is $r > 0$ such that $B(x;r) \subseteq A$.
\item[(2)]
It suffices to show that $B\left(x;\frac{2}{r}\right) \subseteq A^{\circ}$.
Given any point $y \in B\left(x;\frac{2}{r}\right)$,
we will show that there is an open ball $B\left(y;\frac{2}{r}\right)$ of $y$
such that $B\left(y;\frac{2}{r}\right) \subseteq A$.
\item[(3)]
Given any point $z \in B\left(y;\frac{2}{r}\right)$, we have
$$d(z,x) \leq d(z,y) + d(y,x) < \frac{2}{r} + \frac{2}{r} = r,$$
or $z \in B(x;r) \subseteq A$.
Therefore, $B\left(y;\frac{2}{r}\right) \subseteq A$,
or $y \in A^{\circ}$,
or $B\left(x;\frac{2}{r}\right) \subseteq A^{\circ}$,
or $x \in (A^{\circ})^{\circ}$,
or $A^{\circ} \subseteq (A^{\circ})^{\circ}$.
\end{enumerate}
$\Box$ \\

\emph{Proof (Openness of $A^{\circ}$).}
\begin{enumerate}
\item[(1)]
$A^{\circ}$ is open.
\item[(2)]
Note to Definition 3.6: A set $S$ is open if and only if $S = S^{\circ}$.
\end{enumerate}
$\Box$ \\\\



%%%%%%%%%%%%%%%%%%%%%%%%%%%%%%%%%%%%%%%%%%%%%%%%%%%%%%%%%%%%%%%%%%%%%%%%%%%%%%%%



\textbf{Exercise 3.46.}
\begin{enumerate}
\item[(a)]
\emph{$\left( \bigcap_{i=1}^{n} A_i \right)^{\circ} = \bigcap_{i=1}^{n} A_i^{\circ}$,
where each $A_i \subseteq M$. }
\item[(b)]
\emph{$\left( \bigcap_{A \in \mathscr{F}} A \right)^{\circ}
\subseteq \bigcap_{A \in \mathscr{F}} A^{\circ}$,
if $\mathscr{F}$ is an infinite collection of subsets of $M$. }
\item[(c)]
\emph{Give an example where equality does not hold in (b). } \\
\end{enumerate}

\emph{Proof of (a).}
\begin{align*}
x \in \bigcap_{i=1}^{n} A_i^{\circ}
&\Longleftrightarrow
x \in A_i^{\circ} \: \: \forall 1 \leq i \leq n \\
&\Longleftrightarrow
\exists r_i > 0 \text{ such that } B(x;r_i) \subseteq A_i \: \forall 1 \leq i \leq n \\
&\Longleftrightarrow
\exists r = \min_{1 \leq i \leq n}\{r_i\} > 0 \text{ such that }
  B(x;r) \subseteq A_i \: \forall 1 \leq i \leq n \\
&\Longleftrightarrow
\exists r > 0 \text{ such that }
  B(x;r) \subseteq \bigcap_{i=1}^{n} A_i \\
&\Longleftrightarrow
x \in \left( \bigcap_{i=1}^{n} A_i \right)^{\circ}.
\end{align*}
$\Box$ \\

\emph{Proof of (b).}
\begin{align*}
x \in \left( \bigcap_{A \in \mathscr{F}} A \right)^{\circ}
&\Longleftrightarrow
\exists r > 0 \text{ such that }
  B(x;r) \subseteq \bigcap_{A \in \mathscr{F}} A \\
&\Longleftrightarrow
\exists r > 0 \text{ such that } B(x;r) \subseteq A \: \forall A \in \mathscr{F} \\
&\Longrightarrow
x \in A^{\circ} \: \forall A \in \mathscr{F} \\
&\Longleftrightarrow
x \in \bigcap_{A \in \mathscr{F}} A^{\circ}.
\end{align*}
$\Box$ \\

\emph{Proof of (c).}
Let
\begin{enumerate}
\item[(1)]
$M = \mathbb{R}$ with the Euclidean metric of $\mathbb{R}$.
\item[(2)]
$A_i = \left( -\frac{89}{i}, \frac{64}{i} \right) \subseteq \mathbb{R}$
for $i \in \mathbb{Z}^+$.
\end{enumerate}
Note that $\bigcap_{i \in \mathbb{Z}^+} A_i = \{ 0 \}$ and $A_i^{\circ} = A_i$.
So $$\left( \bigcap_{i \in \mathbb{Z}^+} A_i \right)^{\circ} = \varnothing
\text{ and } \bigcap_{i \in \mathbb{Z}^+} A_i^{\circ} = \{ 0 \}.$$
The equality does not hold in (b).
$\Box$ \\\\



%%%%%%%%%%%%%%%%%%%%%%%%%%%%%%%%%%%%%%%%%%%%%%%%%%%%%%%%%%%%%%%%%%%%%%%%%%%%%%%%



\textbf{Exercise 3.47.}
\begin{enumerate}
\item[(a)]
\emph{$\bigcup_{A \in \mathscr{F}} A^{\circ}
\subseteq
\left( \bigcup_{A \in \mathscr{F}} A \right)^{\circ}$. }
\item[(b)]
\emph{Give an example of a finite collection $\mathscr{F}$
in which equality does not hold in (a). } \\
\end{enumerate}

\emph{Proof of (a).}
\begin{align*}
x \in \bigcup_{A \in \mathscr{F}} A^{\circ}
&\Longleftrightarrow
x \in A^{\circ} \text{ for some } A \in \mathscr{F} \\
&\Longrightarrow
x \in \left( \bigcup_{A \in \mathscr{F}} A \right)^{\circ}
\text{ since } A \subseteq \bigcup_{A \in \mathscr{F}} A.
\end{align*}
$\Box$ \\

\emph{Proof of (b).}
Exercise 3.50.
$\Box$ \\\\



%%%%%%%%%%%%%%%%%%%%%%%%%%%%%%%%%%%%%%%%%%%%%%%%%%%%%%%%%%%%%%%%%%%%%%%%%%%%%%%%



\textbf{Exercise 3.48.}
\begin{enumerate}
\item[(a)]
\emph{$(\partial A)^{\circ} = \varnothing$ if $A$ is open or if $A$ is closed in $M$. }
\item[(b)]
\emph{Give an example in which $(\partial A)^{\circ} = M$. } \\
\end{enumerate}

\emph{Proof of (a).}
\begin{enumerate}
\item[(1)]
$A$ is open.
\begin{align*}
(\partial A)^{\circ}
&= (\overline{A} \cap \overline{M-A})^{\circ}
  &\text{(Exercise 3.51)} \\
&= (\overline{A})^{\circ} \cap (\overline{M-A})^{\circ}
  &\text{(Exercise 3.46(a))} \\
&= (\overline{A})^{\circ} \cap (M-A)^{\circ}
  &\text{($A$: open)} \\
&= (\overline{A})^{\circ} \cap (M - \overline{A})
  &\text{(Exercise 3.44)} \\
&\subseteq \overline{A} \cap (M - \overline{A})
  &(S^{\circ} \subseteq S) \\
&= \varnothing.
\end{align*}
\item[(2)]
$A$ is closed.
$\partial A
= \overline{A} \cap \overline{M-A}
= \partial (M - A)
= \varnothing$ (Exercise 3.51).
Or copy the above argument:
\begin{align*}
(\partial A)^{\circ}
&= (\overline{A} \cap \overline{M-A})^{\circ}
  &\text{(Exercise 3.51)} \\
&= (\overline{A})^{\circ} \cap (\overline{M-A})^{\circ}
  &\text{(Exercise 3.46(a))} \\
&= A^{\circ} \cap (\overline{M-A})^{\circ}
  &\text{($A$: closed)} \\
&= (M - (\overline{M-A})) \cap (\overline{M-A})^{\circ}
  &\text{(Exercise 3.43)} \\
&\subseteq (M - (\overline{M-A})) \cap \overline{M-A}
  &(S^{\circ} \subseteq S) \\
&= \varnothing.
\end{align*}

\end{enumerate}
$\Box$ \\

\emph{Proof of (b).}
Similar to Exercise 3.50. Let
\begin{enumerate}
\item[(1)]
$M = \mathbb{R}$ with the Euclidean metric of $\mathbb{R}$.
\item[(2)]
$A = \mathbb{Q} \subseteq \mathbb{R}$,
or $A = \widetilde{\mathbb{Q}} \subseteq \mathbb{R}$.
\end{enumerate}
$\Box$ \\\\



%%%%%%%%%%%%%%%%%%%%%%%%%%%%%%%%%%%%%%%%%%%%%%%%%%%%%%%%%%%%%%%%%%%%%%%%%%%%%%%%



\textbf{Exercise 3.49.}
\emph{If $A^{\circ} = B^{\circ} = \varnothing$ and if $A$ is closed in $M$,
then $(A \cup B)^{\circ} = \varnothing$. } \\

\emph{Proof (Reductio ad absurdum).}
\begin{enumerate}
\item[(1)]
If $x \in (A \cup B)^{\circ}$, then there exists an open ball $B(x) \subseteq A \cup B$.
\item[(2)]
Since $x$ is not an interior point of $A$ ($A^{\circ} = \varnothing$),
the open ball $B(x) \not\subseteq A$, or $B(x) \cap \widetilde{A} \neq \varnothing$.
\item[(3)]
Notice that $A$ is closed.
Hence $\widetilde{A}$ is open and so is $B(x) \cap \widetilde{A}$.
Since $B(x) \cap \widetilde{A}$ is not empty,
some $y \in B(x) \cap \widetilde{A}$ and thus
there is another open ball $B(y) \subseteq B(x) \cap \widetilde{A}$.
\item[(4)]
\begin{align*}
B(y)
&\subseteq B(x) \cap \widetilde{A} \\
&\subseteq (A \cup B) \cap \widetilde{A} \\
&= (A \cap \widetilde{A}) \cup (B \cap \widetilde{A}) \\
&= \varnothing \cup (B \cap \widetilde{A}) \\
&= B \cap \widetilde{A} \\
&\subseteq B.
\end{align*}
So that $y$ is an interior point of $B$, contrary to $B^{\circ} = \varnothing$.
\end{enumerate}
Therefore, the result is established.
$\Box$ \\\\



%%%%%%%%%%%%%%%%%%%%%%%%%%%%%%%%%%%%%%%%%%%%%%%%%%%%%%%%%%%%%%%%%%%%%%%%%%%%%%%%



\textbf{Exercise 3.50.}
\emph{Give an example in which $A^{\circ} = B^{\circ} = \varnothing$ but
$(A \cup B)^{\circ} = M$. } \\

\emph{Proof.}
Let
\begin{enumerate}
\item[(1)]
$M = \mathbb{R}$ with the Euclidean metric of $\mathbb{R}$.
\item[(2)]
$A = \mathbb{Q} \subseteq \mathbb{R}$.
\item[(3)]
$B = \widetilde{\mathbb{Q}} \subseteq \mathbb{R}$.
\end{enumerate}
$\Box$ \\\\



%%%%%%%%%%%%%%%%%%%%%%%%%%%%%%%%%%%%%%%%%%%%%%%%%%%%%%%%%%%%%%%%%%%%%%%%%%%%%%%%



\textbf{Exercise 3.51.}
\emph{$\partial A = \overline{A} \cap \overline{M-A}$ and
$\partial A = \partial(M - A)$.} \\

Also, $\partial A = \overline{A} - A^{\circ}$ (Exercise 3.43)
and $\widetilde{\partial A} = A^{\circ} \cup \widetilde{A}^{\circ}$
(Exercise 3.43, 3.44). \\

\emph{Proof.}
Definition 3.40 and Definition 3.19 show $\partial A = \overline{A} \cap \overline{M-A}$.
Notice that $A = M - (M-A)$, and thus
$\partial A = \overline{A} \cap \overline{M-A} = \partial(M - A)$.
$\Box$ \\\\



%%%%%%%%%%%%%%%%%%%%%%%%%%%%%%%%%%%%%%%%%%%%%%%%%%%%%%%%%%%%%%%%%%%%%%%%%%%%%%%%


\textbf{Exercise 3.52.}
\emph{If $\overline{A} \cap \overline{B} = \varnothing$,
then $\partial(A \cup B) = \partial A \cup \partial B$.} \\

The proof is all about this relation
$B(x;r) \cap \widetilde{A} \cap \widetilde{B} \neq \varnothing
\Longleftrightarrow
B(x;r) \cap \widetilde{A} \neq \varnothing.$ \\

\emph{Proof (Brute-force).}
\begin{enumerate}
\item[(1)]
$(\subseteq)$
Given any $x \in \partial(A \cup B)$.
For any $r > 0$,
\begin{align*}
&B(x;r) \cap (A \cup B) \neq \varnothing \text{ and }
B(x;r) \cap \widetilde{A \cup B} \neq \varnothing \\
\Longleftrightarrow&
(B(x;r) \cap A) \cup (B(x;r) \cap B) \neq \varnothing \text{ and }
B(x;r) \cap \widetilde{A} \cap \widetilde{B} \neq \varnothing \\
\Longleftrightarrow&
( B(x;r) \cap A \neq \varnothing
  \text{ or } B(x;r) \cap B \neq \varnothing )
\text{ and }
B(x;r) \cap \widetilde{A} \cap \widetilde{B} \neq \varnothing \\
\Longrightarrow&
( B(x;r) \cap A \neq \varnothing
  \text{ and } B(x;r) \cap \widetilde{A} \neq \varnothing)
\text{ or } \\
&( B(x;r) \cap B \neq \varnothing
  \text{ and } B(x;r) \cap \widetilde{B} \neq \varnothing) \\
\Longleftrightarrow&
x \in \partial A \text{ or } x \in \partial B \\
\Longleftrightarrow&
x \in \partial A \cup \partial B.
\end{align*}
\item[(2)]
$(\supseteq)$
Since $\overline{A} \cap \overline{B} = \varnothing$ and Exercise 3.51,
$\partial A \cap \partial B = \varnothing$
and $\partial A \cap B^{\circ} = \varnothing$
and $A^{\circ} \cap \partial B = \varnothing$.
Given any $x \in \partial A \cup \partial B$.
There are only two possible cases.
  \begin{enumerate}
  \item[(a)]
  \emph{$x \in \partial A$ and $x \not\in \partial B$.}
  \begin{align*}
  x \not\in \partial B
  \Longleftrightarrow&
  x \in M - \partial B \\
  \Longleftrightarrow&
  x \in M - ( \overline{B} \cap \overline{M-B})
    &\text{(Exercise 3.51)} \\
  \Longleftrightarrow&
  x \in (M - \overline{B}) \cup (M - \overline{M-B}) \\
  \Longleftrightarrow&
  x \in (M-B)^{\circ} \cup B^{\circ}
    &\text{(Exercise 3.43, 3.44)} \\
  \Longleftrightarrow&
  x \in (M-B)^{\circ} \text{ or } x \in B^{\circ} \\
  \Longrightarrow&
  x \in (M-B)^{\circ}
    &(\partial A \cap B^{\circ} = \varnothing)
  \end{align*}
  $x$ is an interior point of $\widetilde{B}$.
  Hence there exists $r_0 > 0$ such that $B(x;r_0) \subseteq \widetilde{B}$.
  Given any $r_0 > r > 0$, we have
  \begin{align*}
  x \in \partial A
  \Longleftrightarrow&
  B(x;r) \cap A \neq \varnothing
  \text{ and } B(x;r) \cap \widetilde{A} \neq \varnothing \\
  \Longrightarrow&
  B(x;r) \cap (A \cup B) \neq \varnothing \text{ and }\\
  &B(x;r) \cap \widetilde{A \cup B}
  = B(x;r) \cap \widetilde{A} \cap \widetilde{B}
  \neq \varnothing \\
  \Longleftrightarrow&
  x \in \partial(A \cup B).
  \end{align*}
  \item[(b)]
  \emph{$x \in \partial B$ and $x \not\in \partial A$.}
  Similar to (a).
  \end{enumerate}
\end{enumerate}
$\Box$ \\\\



% No exercises left.

%%%%%%%%%%%%%%%%%%%%%%%%%%%%%%%%%%%%%%%%%%%%%%%%%%%%%%%%%%%%%%%%%%%%%%%%%%%%%%%%
%%%%%%%%%%%%%%%%%%%%%%%%%%%%%%%%%%%%%%%%%%%%%%%%%%%%%%%%%%%%%%%%%%%%%%%%%%%%%%%%



\end{document}