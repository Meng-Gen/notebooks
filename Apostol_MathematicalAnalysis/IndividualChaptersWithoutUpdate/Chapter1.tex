\documentclass{article}
\usepackage{amsfonts}
\usepackage{amsmath}
\usepackage{amssymb}
\usepackage{hyperref}
\usepackage[none]{hyphenat}
\usepackage{mathrsfs}
\usepackage{physics}
\parindent=0pt

\def\upint{\mathchoice%
    {\mkern13mu\overline{\vphantom{\intop}\mkern7mu}\mkern-20mu}%
    {\mkern7mu\overline{\vphantom{\intop}\mkern7mu}\mkern-14mu}%
    {\mkern7mu\overline{\vphantom{\intop}\mkern7mu}\mkern-14mu}%
    {\mkern7mu\overline{\vphantom{\intop}\mkern7mu}\mkern-14mu}%
  \int}
\def\lowint{\mkern3mu\underline{\vphantom{\intop}\mkern7mu}\mkern-10mu\int}

\begin{document}

\textbf{\Large Chapter 1: The Real And Complex Number Systems} \\\\



\emph{Author: Meng-Gen Tsai} \\
\emph{Email: plover@gmail.com} \\\\



%%%%%%%%%%%%%%%%%%%%%%%%%%%%%%%%%%%%%%%%%%%%%%%%%%%%%%%%%%%%%%%%%%%%%%%%%%%%%%%%
%%%%%%%%%%%%%%%%%%%%%%%%%%%%%%%%%%%%%%%%%%%%%%%%%%%%%%%%%%%%%%%%%%%%%%%%%%%%%%%%



\textbf{\large Integers} \\\\



\textbf{Exercise 1.1.}
\emph{Prove that there is no largest prime. (A proof was known to Euclid.)} \\

There are many proofs of this result. We provide some of them. \\

\emph{Proof (Due to Euclid).}
If
$p_1, p_2, \ldots, p_t$ were all primes, then
we consider $$n = p_1 p_2 \cdots p_t + 1.$$
Thus there is a prime number $p$ dividing $n$.
$p$ can not be any of $p_i$ for $1 \leq i \leq t$;
otherwise $p$ would divide the difference $n - p_1 p_2 \cdots p_t = 1$.
That is, $p \neq p_i$ for $1 \leq i \leq t$,
contrary to the assumption.
$\Box$ \\

\textbf{Supplement (Due to Euclid).}
\begin{enumerate}
\item[(1)]
  \emph{Show that $k[x]$, with $k$ a field,
  has infinitely many irreducible polynomials.}
  If
  $f_1, f_2, \ldots, f_t$ were all irreducible polynomials, then
  we consider $$g = f_1 f_2 \cdots f_t + 1 \in k[x].$$
  So there is an irreducible polynomial $f$ dividing $g$
  (since $\deg g = \deg f_1 + \deg f_2 + \cdots + \deg f_t \geq 1$).
  $f$ can not be any of $c_i f_i$ for $1 \leq i \leq t$ and $c_i \in k - \{0\}$;
  otherwise $f$ would divide the difference $g - f_1 f_2 \cdots f_t = 1$.
  That is, $f \neq c_i f_i$ for $1 \leq i \leq t$ and $c_i \in k - \{0\}$,
  contrary to the assumption.

\item[(2)]
  \emph{Show that any algebraically closed field is infinite.}
  Let $k$ be an algebraically closed field.
  If $a_1, \ldots, a_n$ were all elements in $k$, then
  we consider a monic polynomials
  \[
    F(X) = (X - a_1) \cdots (X - a_n) + 1 \in k[X].
  \]
  Since $k$ is algebraically closed,
  there is an element $a \in k$ such that $F(a) = 0$.
  By assumption, $a = a_i$ for some $1 \leq i \leq n$,
  and thus $F(a) = F(a_i) = 1$, contrary to the fact that
  a field is a commutative ring where $0 \neq 1$ and all nonzero elements are invertible.
\end{enumerate}
$\Box$ \\



\emph{Proof (Unique factorization theorem).}
Given $N$.
\begin{enumerate}
\item[(1)]
\emph{Show that $\sum_{n \leq N} \frac{1}{n}
\leq \prod_{p \leq N} \left( 1 - \frac{1}{p} \right)^{-1}$.} \\
By the unique factorization theorem on $n \leq N$,
$$\sum_{n \leq N} \frac{1}{n}
\leq \prod_{p \leq N} \left( 1 + \frac{1}{p} + \frac{1}{p^2} + \cdots \right)
= \prod_{p \leq N} \left( 1 - \frac{1}{p} \right)^{-1}.$$
\item[(2)]
By (1) and the fact that $\sum \frac{1}{n}$ diverges,
there are infinitely many primes.
\end{enumerate}
$\Box$ \\

\emph{Proof (Due to Eckford Cohen).}
\begin{enumerate}
\item[(1)]
\emph{$\text{ord}_p n!
= \left[\frac{n}{p}\right] + \left[\frac{n}{p^2}\right] + \left[\frac{n}{p^3}\right] + \cdots$.}
For any $k = 1, 2, \ldots, n$, we can express $k$ as $k = p^s t$
where $s = \text{ord}_p k$ is a non-negative integer and $(t, p) = 1$.
There are $[\frac{n}{p^a}]$ numbers such that $p^a \mid k$ for $a = 1, 2, \ldots$.
Therefore, there are $$\left[\frac{n}{p^a}\right] - \left[\frac{n}{p^{a+1}}\right]$$
numbers such that $\text{ord}_p k = a$ for $a = 1, 2, \ldots$. Hence,
\begin{align*}
\text{ord}_p n!
&= \left( \left[\frac{n}{p}\right] - \left[\frac{n}{p^2}\right] \right)
 + 2 \left( \left[\frac{n}{p^2}\right] - \left[\frac{n}{p^3}\right] \right)
 + 3 \left( \left[\frac{n}{p^3}\right] - \left[\frac{n}{p^4}\right] \right) + \cdots \\
&= \left[\frac{n}{p}\right] + \left[\frac{n}{p^2}\right] + \left[\frac{n}{p^3}\right] + \cdots.
\end{align*}
\item[(2)]
\emph{$\text{ord}_p n! \leq \frac{n}{p - 1}$ and that
$n!^{\frac{1}{n}} \leq \prod_{p|n!}p^{\frac{1}{p - 1}}$.}
\begin{align*}
\text{ord}_p n!
&= \left[\frac{n}{p}\right] + \left[\frac{n}{p^2}\right] + \left[\frac{n}{p^3}\right] + \cdots \\
&\leq \frac{n}{p} + \frac{n}{p^2} + \frac{n}{p^3} + \cdots \\
&= \frac{\frac{n}{p}}{1 - \frac{1}{p}} \\
&= \frac{n}{p - 1}.
\end{align*}
Thus,
$$n!
= \prod_{p|n!} p^{\text{ord}_p n!}
\leq \prod_{p|n!} p^{\frac{n}{p - 1}}
= \left( \prod_{p|n!} p^{\frac{1}{p - 1}} \right)^n, $$
or
$$n!^{\frac{1}{n}} \leq \prod_{p|n!}p^{\frac{1}{p - 1}}.$$
\item[(3)]
\emph{$(n!)^2 \geq n^n$.}
Write
$(n!)^2 = \prod_{k=1}^n k \prod_{k=1}^n (n + 1 - k) = \prod_{k=1}^n k(n + 1 - k)$,
and $n^n = \prod_{k=1}^n n$.
It suffices to show that $k(n + 1 - k) \geq n$ for each $1 \leq k \leq n$.
Notice that $k(n + 1 - k) - n = (n - k)(k - 1) \geq 0$ for $1 \leq k \leq n$.
The inequality holds.
\item[(4)]
By (3)(4), $\prod_{p|n!}p^{\frac{1}{p - 1}} \geq \sqrt{n}$.
Assume that there are finitely many primes,
the value $\prod_{p|n!} p^{\frac{1}{p - 1}}$ is a finite number
whenever the value of $n$.
However, $\sqrt{n} \rightarrow \infty$ as $n \rightarrow \infty$,
which leads to a contradiction.
Hence there are infinitely many primes.
\end{enumerate}
$\Box$ \\

\emph{Proof (Formula for $\phi(n)$).}
If
$p_1, p_2, \ldots, p_t$ were all primes, then let
$n = p_1 p_2 \cdots p_t$ and all numbers between $2$ and $n$ are
NOT relatively prime to $n$.
Thus, $\phi(n) = 1$ by the definition of $\phi$.
By the formula for $\phi$,
\begin{align*}
\phi(n)
&= n
\left( 1 - \frac{1}{p_1} \right)
\left( 1 - \frac{1}{p_2} \right)
\cdots
\left( 1 - \frac{1}{p_t} \right) \\
1
&= (p_1 p_2 \cdots p_t)
\left( 1 - \frac{1}{p_1} \right)
\left( 1 - \frac{1}{p_2} \right)
\cdots
\left( 1 - \frac{1}{p_t} \right) \\
&= (p_1 - 1)(p_2 - 1) \cdots (p_t - 1) > 1,
\end{align*}
which is a contradiction (since $3$ is a prime).
Hence there are infinitely many primes.
$\Box$ \\\\



%%%%%%%%%%%%%%%%%%%%%%%%%%%%%%%%%%%%%%%%%%%%%%%%%%%%%%%%%%%%%%%%%%%%%%%%%%%%%%%%



\textbf{Exercise 1.2.}
\emph{If $n$ is a positive integer, prove the algebraic identity
$$a^n - b^n = (a - b) \sum_{k=0}^{n-1} a^k b^{n-1-k}.$$} \\

\emph{Proof.}
\begin{enumerate}
\item[(1)]
\begin{align*}
(a - b) \sum_{k=0}^{n-1} a^k b^{n-1-k}
&= a \sum_{k=0}^{n-1} a^k b^{n-1-k} - b \sum_{k=0}^{n-1} a^k b^{n-1-k} \\
&= \sum_{k=0}^{n-1} a^{k+1} b^{n-1-k} - \sum_{k=0}^{n-1} a^k b^{n-k}.
\end{align*}
\item[(2)] Arrange summation index:
\begin{align*}
\sum_{k=0}^{n-1} a^{k+1} b^{n-1-k}
&= \sum_{k=1}^{n} a^{k} b^{n-k}
= a^n + \sum_{k=1}^{n-1} a^{k} b^{n-k}, \\
\sum_{k=0}^{n-1} a^k b^{n-k}
&= b^n + \sum_{k=1}^{n-1} a^{k} b^{n-k}.
\end{align*}
\item[(3)]
By (1)(2),
\begin{align*}
(a - b) \sum_{k=0}^{n-1} a^k b^{n-1-k}
&= \left( a^n + \sum_{k=1}^{n-1} a^{k} b^{n-k} \right)
- \left( b^n + \sum_{k=1}^{n-1} a^{k} b^{n-k} \right) \\
&= a^n - b^n.
\end{align*}
\end{enumerate}
$\Box$ \\

\textbf{Supplement.} Some exercises without proof.
\begin{enumerate}
\item[(1)]
\emph{Let $x$ be a nilpotent element of $A$.
Show that $1+x$ is a unit of $A$.
Deduce that the sum of a nilpotent element and a unit is a unit.}
(Exercise 1.1 in the textbook: \emph{Atiyah and Macdonald,
Introduction to Commutative Algebra})

\item[(2)]
\emph{Prove that $1^k + 2^k + \cdots + (p-1)^k \equiv 0 \: (p)$
if $p - 1 \nmid k$ and $-1 (p)$ if $p - 1 \mid k$.}
(Exercise 4.11 in the textbook: \emph{Kenneth Ireland and Michael Rosen,
A Classical Introduction to Modern Number Theory, 2nd edition})

\item[(3)]
\emph{Use the existence of a primitive root to give another proof
of Wilson's theorem $(p - 1)! \equiv -1 \: (p)$.}
(Exercise 4.12 in the textbook: \emph{Kenneth Ireland and Michael Rosen,
A Classical Introduction to Modern Number Theory, 2nd edition})

\item[(4)]
\emph{Suppose $n$ and $F$ are integers and $n, F > 0$. Show that
$$B_n(Fx) = F^{n-1} \sum_{a=0}^{F-1} B_n \left(x + \frac{a}{F} \right).$$
where $B_n(x)$ are Bernoulli polynomials.}
(Exercise 15.19 in the textbook: \emph{Kenneth Ireland and Michael Rosen,
A Classical Introduction to Modern Number Theory, 2nd edition})

\item[(5)]
Exercise 1.3.

\item[(6)]
Exercise 1.4.
\end{enumerate}
$\Box$ \\\\



%%%%%%%%%%%%%%%%%%%%%%%%%%%%%%%%%%%%%%%%%%%%%%%%%%%%%%%%%%%%%%%%%%%%%%%%%%%%%%%%



\textbf{Exercise 1.3.}
\emph{If $2^n - 1$ is a prime, prove that $n$ is prime.
A prime of the form $2^p - 1$, where $p$ is prime, is called a Mersenne prime.} \\

It suffices to prove that:
\emph{If $a^n - 1$ is a prime, show that $a = 2$ and that $n$ is a prime.}
Primes of the form $2^p - 1$ are called Mersenne primes.
For example, $2^3 - 1 = 7$ and $2^5 - 1 = 31$.
It is not known if there are infinitely many Mersenne primes. \\

\emph{Proof.}
\begin{enumerate}
\item[(1)]
\emph{$n$ is a prime.}
Assume $n$ were not prime, say $n = rs$ for some $r, s > 1$.
By Exercise 1.2,
$a^{rs} - 1 = (a^s - 1)(\sum_{k=0}^{r-1} a^{sk})$.
$a^s - 1 = 1$ since $a^s - 1 < a^{rs} - 1$ and $a^{rs} - 1$ is a prime.
Hence $s=1$ and ($a=2$), which is absurd.
\item[(2)]
\emph{$a = 2$.}
If $a$ is odd, then $a^p - 1 > 2$ is even, which is not a prime.
If $a > 2$ is even,
$a^p - 1 = (a - 1)(\sum_{k=0}^{p-1} a^k)$.
Both $a - 1 > 1$ and $\sum_{k=0}^{p-1} a^k > 1$, which is absurd.
\end{enumerate}
By (1)(2), $a = 2$ and that $n$ is a prime if $a^n - 1$ is a prime.
$\Box$ \\\\



%%%%%%%%%%%%%%%%%%%%%%%%%%%%%%%%%%%%%%%%%%%%%%%%%%%%%%%%%%%%%%%%%%%%%%%%%%%%%%%%



\textbf{Exercise 1.6.}
\emph{Prove that every nonempty set of positive integers contains a smallest member.
This is called the well-ordering principle.} \\

\emph{Proof.}
Use mathematical induction to establish
that the well-ordering principle.

\begin{enumerate}
\item[(1)]
Given a set $S$ of positive integers,
let $P(n)$ be the proposition
`If $m \in S$ for some $m \leq n$, then $S$ has a least element'.
Want to show $P(n)$ is true for all $n \in \mathbb{N}$.
\begin{enumerate}
\item[(a)]
$P(1)$ is true.
For $m \in S$ with $m \leq n = 1$,
or $m = 1$ by the minimality of $1 \in \mathbb{N}$,
$S$ has a least element $1$ ($m$ itself) in $\mathbb{N}$.
\item[(b)]
Suppose $P(n)$ is true.
If $n+1 \in S$, then there are only two possible cases.
  \begin{enumerate}
  \item[(i)]
  There is a positive integer $m \in S$ less than $n+1$.
  So $n \geq m \in S$.
  Since $P(n)$ is true, $S$ has a least element.
  \item[(ii)]
  There is no positive integer $m \in S$ less than $n+1$.
  In this case $n+1$ is the least element in $S$.
  \end{enumerate}
  In any cases (i)(ii), $S$ has a least element, or $P(n+1)$ is true.
\end{enumerate}
By mathematical induction, $P(n)$ is true for all $n \in \mathbb{N}$.
\item[(2)]
\emph{Show that the well-ordering principle holds.}
Let $T$ be a nonempty subset of $\mathbb{N}$,
so there exists a positive integer $k \in T$.
Notice that $P(k)$ is true by (1),
thus $T$ has a least element since $k \leq k$.
\end{enumerate}
$\Box$ \\

\textbf{Supplement.}
\emph{Show that the well-ordering principle implies
the principle of mathematical induction.} \\

\emph{Proof.}
Suppose that
\begin{enumerate}
\item[(1)]
$P(n)$ be a proposition defined for each $n \in \mathbb{N}$,
\item[(2)]
$P(1)$ is true,
\item[(3)]
$[P(n) \Rightarrow P(n+1)]$ is true.
\end{enumerate}

Consider the set
$$S = \{ n \in \mathbb{N} : P(n) \text{ is false} \} \subseteq \mathbb{N}.$$
Want to show
\emph{$S$ is empty, or the principle of mathematical induction holds.}
If $S$ were nonempty,
by the well-ordering principle $S$ has a smallest element $m$.
$m$ cannot be $1$ by (2).
Say $m > 1$.
Therefore, $m - 1 \in \mathbb{N}$
and $P(m-1)$ is true by the minimality of $m$.
By (3), $P((m-1)+1) = P(m)$ is true, which is absurd.
$\Box$ \\\\



%%%%%%%%%%%%%%%%%%%%%%%%%%%%%%%%%%%%%%%%%%%%%%%%%%%%%%%%%%%%%%%%%%%%%%%%%%%%%%%%
%%%%%%%%%%%%%%%%%%%%%%%%%%%%%%%%%%%%%%%%%%%%%%%%%%%%%%%%%%%%%%%%%%%%%%%%%%%%%%%%



\textbf{\large Rational and irrational numbers} \\\\



\textbf{Exercise 1.11.}
\emph{Given any real $x > 0$,
prove that there is an irrational number between $0$ and $x$.} \\

\emph{Proof.}
There are only two possible cases: $x$ is rational, or $x$ is irrational.
\begin{enumerate}
\item[(1)]
\emph{$x$ is rational.}
Pick $y = \frac{x}{\sqrt{89}} \in (0, x) \subseteq \mathbb{R}$. $y$ is irrational.
\item[(2)]
\emph{$x$ is irrational.}
Pick $y = \frac{x}{\sqrt{64}} \in (0, x) \subseteq \mathbb{R}$. $y$ is irrational.
\end{enumerate}
$\Box$ \\

\emph{Proof (Exercise 4.12).}
Pick
$$y
= \lim_{m \rightarrow \infty}[\lim_{n \rightarrow \infty} \cos^{2n}(m!\pi x)]
\cdot \frac{x}{\sqrt{89}}
+
(1 - \lim_{m \rightarrow \infty}[\lim_{n \rightarrow \infty} \cos^{2n}(m!\pi x)])
\cdot \frac{x}{\sqrt{64}}.$$
\begin{enumerate}
\item[(1)]
\emph{$x$ is rational.}
$y = \frac{x}{\sqrt{89}} \in (0, x) \subseteq \mathbb{R}$ is irrational.
\item[(2)]
\emph{$x$ is irrational.}
$y = \frac{x}{\sqrt{64}} \in (0, x) \subseteq \mathbb{R}$ is irrational.
\end{enumerate}
$\Box$ \\\\



%%%%%%%%%%%%%%%%%%%%%%%%%%%%%%%%%%%%%%%%%%%%%%%%%%%%%%%%%%%%%%%%%%%%%%%%%%%%%%%%
%%%%%%%%%%%%%%%%%%%%%%%%%%%%%%%%%%%%%%%%%%%%%%%%%%%%%%%%%%%%%%%%%%%%%%%%%%%%%%%%



\textbf{\large Upper bounds} \\\\



%%%%%%%%%%%%%%%%%%%%%%%%%%%%%%%%%%%%%%%%%%%%%%%%%%%%%%%%%%%%%%%%%%%%%%%%%%%%%%%%
%%%%%%%%%%%%%%%%%%%%%%%%%%%%%%%%%%%%%%%%%%%%%%%%%%%%%%%%%%%%%%%%%%%%%%%%%%%%%%%%



\textbf{\large Inequalities} \\\\



\textbf{Exercise 1.23.}
\emph{Prove Lagrange's identity for real numbers:
$$\left( \sum_{k=1}^{n} a_k b_k \right)^2
= \left( \sum_{k=1}^{n} a_k \right)^2 \left( \sum_{k=1}^{n} b_k \right)^2
- \sum_{1 \leq k < j \leq n}
(a_k b_j - a_j b_k)^2.$$}

Note that this identity implies the Cauchy-Schwarz inequality. \\

\emph{Proof.}
Put $(a_k, b_k, A_k, B_k) \mapsto (a_k, b_k, a_k, b_k)$
in the following generalization (Binet-Cauchy identity).
$\Box$ \\

\textbf{Generalization (Binet-Cauchy identity).}
\begin{align*}
&\sum_{1 \leq k < j \leq n}
(a_k b_j - a_j b_k)(A_k B_j - A_j B_k) \\
= &\left( \sum_{k=1}^{n} a_k A_k \right)\left( \sum_{k=1}^{n} b_k B_k \right)
- \left( \sum_{k=1}^{n} a_k B_k \right)\left( \sum_{k=1}^{n} b_k A_k \right).
\end{align*}

\emph{Proof.}
\begin{align*}
&\sum_{1 \leq k < j \leq n}
(a_k b_j - a_j b_k)(A_k B_j - A_j B_k) \\
= &\sum_{1 \leq k < j \leq n}
(a_k b_j A_k B_j + a_j b_k A_j B_k)
- \sum_{1 \leq k < j \leq n}
(a_k b_j A_j B_k - a_j b_k A_k B_j) \\
= &\sum_{1 \leq k < j \leq n}
(a_k A_k b_j B_j + a_j A_j b_k B_k)
- \sum_{1 \leq k < j \leq n}
(a_k B_k b_j A_j + a_j B_j b_k A_k) \\
= &\sum_{1 \leq k \neq j \leq n} a_k A_k b_j B_j
 - \sum_{1 \leq k \neq j \leq n} a_k B_k b_j A_j \\
= &\sum_{1 \leq k, j \leq n} a_k A_k b_j B_j
 - \sum_{1 \leq k, j \leq n} a_k B_k b_j A_j \\
  & \text{(since $a_k A_k b_j B_j - a_k B_k b_j A_j = 0$ as $k = j$)} \\
= &\left( \sum_{k=1}^{n} a_k A_k \right)\left( \sum_{j=1}^{n} b_j B_j \right)
- \left( \sum_{k=1}^{n} a_k B_k \right)\left( \sum_{j=1}^{n} b_j A_j \right) \\
= &\left( \sum_{k=1}^{n} a_k A_k \right)\left( \sum_{k=1}^{n} b_k B_k \right)
- \left( \sum_{k=1}^{n} a_k B_k \right)\left( \sum_{k=1}^{n} b_k A_k \right).
\end{align*}
$\Box$ \\

\textbf{Supplement ($\mathbb{Z}[i]$).}
As $n = 2$,
$(a_1^2 + a_2^2)(b_1^2 + b_2^2)
= (a_1 b_1 + a_2 b_2)^2 + (a_1 b_2 - a_2 b_1)^2$. \\

Define $N: \mathbb{Z}[i] \rightarrow \mathbb{Z}$ by
$N(a+bi) = a^2 + b^2$.

\begin{enumerate}
\item[(1)]
\emph{Verify that for all $\alpha, \beta \in \mathbb{Z}[i]$,
$N(\alpha\beta) = N(\alpha)N(\beta)$,
either by direct computation or using the fact that
$N(a+bi) = (a+bi)(a-bi)$.
Conclude that if $\alpha \mid \gamma$ in $\mathbb{Z}[i]$,
then $N(\alpha) \mid N(\gamma)$ in $\mathbb{Z}$.}
\item[(2)]
\emph{Let $\alpha \in \mathbb{Z}[i]$.
Show that $\alpha$ is a unit iff $N(\alpha) = 1$.
Conclude that the only unit are $\pm 1$ and $\pm i$.}
\item[(3)]
\emph{Let $\alpha \in \mathbb{Z}[i]$.
Show that if $N(\alpha)$ is a prime in $\mathbb{Z}$ then
$\alpha$ is irreducible in $\mathbb{Z}[i]$.
Show that the same conclusion holds
if $N(\alpha) = p^2$, where $p$ is a prime in $\mathbb{Z}$,
$p \equiv 3 \pmod{4}$.}
\item[(4)]
\emph{Show that $1-i$ is irreducible in $\mathbb{Z}$
and that $2 = u(1-i)^2$ for some unit $u$.}
\item[(5)]
\emph{Show that every nonzero, non-unit Gaussian integer $\alpha$
is a product of irreducible elements, by induction on $N(\alpha)$.}
\item[(6)]
\emph{Use the unique factorization in $\mathbb{Z}[i]$ to prove that
every prime $p \equiv 1 \pmod{4}$ is a sum of two squares.}
\item[(7)]
\emph{Describe all irreducible elements in $\mathbb{Z}[i]$.}
\end{enumerate}



%%%%%%%%%%%%%%%%%%%%%%%%%%%%%%%%%%%%%%%%%%%%%%%%%%%%%%%%%%%%%%%%%%%%%%%%%%%%%%%%
%%%%%%%%%%%%%%%%%%%%%%%%%%%%%%%%%%%%%%%%%%%%%%%%%%%%%%%%%%%%%%%%%%%%%%%%%%%%%%%%



\textbf{\large Complex numbers} \\\\



\textbf{Exercise 1.48.}
\emph{Prove Lagrange's identity for complex numbers:
$$\abs{ \sum_{k=1}^{n} a_k b_k }^2
= \sum_{k=1}^{n} \abs{a_k}^2 \sum_{k=1}^{n} \abs{b_k}^2
- \sum_{1 \leq k < j \leq n}
\abs{ a_k \overline{b_j} - a_j \overline{b_k} }^2.$$}

\emph{Proof.}
Put $(a_k, b_k, A_k, B_k) \mapsto (a_k, \overline{b_k}, \overline{a_k}, b_k)$
in the generalization to Exercise 1.23 (Binet-Cauchy identity) and use
the identity $|z| = z \overline{z}$.



\end{document}